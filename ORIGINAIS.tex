%!TEX root=./LIVRO.tex
\newcommand{\href}[1]{}

\renewcommand*{\chapnumfont}{\normalfont\HUGE\bfseries\sffamily}
\renewcommand*{\chaptitlefont}{\normalfont\fontsize{50}{60}\selectfont\bfseries\sffamily}

\chapter{A \textarn{a} \textarc{a} \textart{a}}


% \textarc{a}a-\textarc{b}b-\textarc{c}c-\textarc{d}d-\textarc{e}e-\textarc{f}f-\textarc{g}g-
% %\textarc{h}h-\textarc{i}i-\textarc{j}j-\textarc{k}k-\textarc{l}l-\textarc{m}m-
% %\textarc{n}n-\textarc{o}o-\textarc{p}p-\textarc{q}q-\textarc{r}r-\textarc{s}s-
% %\textarc{t}t-\textarc{u}u-\textarc{v}v-\textarc{x}x-\textarc{y}y-\textarc{w}w-\textarc{z}z

\section{\versal{AGRICULTURA}}\label{agricultura}

Os estudos arqueológicos realizados em toda a Escandinávia revelaram que
a prática da agricultura nessa região era feita em campos e destinada
principalmente ao cultivo de grãos, como diversas espécies de trigo,
cevada, sorgo, centeio e aveia. Além das descobertas dos campos, algumas
práticas agrícolas da Era Viking também foram reveladas. Na Dinamarca,
por exemplo, foram encontrados vestígios de sulcos rasos, que comprovam
o uso de uma espécie de arado simplesmentes para preparar o solo antes da
semeadura. Esse arado primitivo fazia um sulco no solo, mas não revolvia
mais profundamente a terra, soltando apenas uma fina camada que receberia
semente. Os arados simples foram usados até o final da Era Viking,
quando um arado mais pesado e com uma capacidade maior de sulcar a terra
foi introduzido.

Outros implementos agrícolas devem ter sido comuns nas fazendas, mas
nenhum foi preservado em grandes quantidades. Os grãos mais variados e o
feno eram colhidos com uma foice e a vegetação destinada à alimentação
animal era cortada com uma faca simples, denominada faca de folha.
Foram encontrados fragmentos de ancinhos e pás feitos de madeira, bem
como lâminas e peneiras para debulhar e peneirar os grãos, provavelmente
feitos também de madeira. Barris de madeira e cestos de vime eram
usados para armazenamento de grãos, e o feno para alimentação animal
provavelmente teria sido transportado dos campos em carrinhos de
madeira. As aldeias estavam cercadas por campos cultivados, mas também
havia acessos às áreas de pastagem para gado. A criação de animais era
tão importante quanto o cultivo do solo. A criação de gado provavelmente
ocupava uma boa parte do trabalho no campo, fornecendo leite e carne,
além da força de trabalho desses animais que puxavam o arado e as
carroças. Porcos e ovelhas também foram criados.

Além dos campos cultivados com grãos, havia também o cultivo de hortas e
jardins. O mais comum na Era Viking era o cultivo de um ``jardim da
cozinha'', isto é, um jardim localizado perto da habitação e
caracterizado pelo cultivo em pequena escala de determinadas plantas que
seriam utilizadas como condimentos e remédios. A horta, por sua vez, era
delimitada, cultivada e
os vegetais ali plantados destinavam-se basicamente à alimentação.
O cultivo de uma horta de cozinha, geralmente, se distinguia da
agricultura pelos cuidados diários que exigia, já que várias espécies
diferentes eram plantadas em um mesmo espaço de terra. Em uma horta,
cada espécie estava representada por um número relativamente pequeno de
plantas, em contraste com o cultivo de campo em larga escala de uma única
cultura. Algumas das plantas cultivadas nas hortas e jardins exigiam um
cuidado mais intensivo do que as culturas de campo, uma vez que algumas
plantas são mais exigentes no que diz respeito à adubação, rega e manejo
do solo.

As árvores frutíferas e arbustos também podiam ser considerados um
elemento comum do jardim na Era Viking. O ``jardim de prazer'', aquele
local onde eram cultivadas somente espécies ornamentais de plantas e
flores e que durante a Idade Média Central e Baixa é representado em
iluminuras e descrito na literatura cortês, não existe na Era Viking. No
entanto, muitas plantas úteis também podiam ser ornamentais, e um jardim
com uma composição ornamental seria possível mesmo em um contexto mais
rústico como o da área nórdica, embora isso não possa ser comprovado.
Pesquisadores acreditam que espécies como a \emph{Polemonium caeruleum
L}., conhecida como escada de Jacó, e a margarida comum (\emph{Bellis
perennis L.}) seriam cultivadas somente como ornamento. No entanto, é
difícil encontrar evidências no material arqueológico para o cultivo de
plantas apenas como ornamento.

A detecção física de um espaço onde se localizava o jardim no contexto
arqueológico é frequentemente indicada por elementos como cercas de
madeira, cercas de pedra, bem como terraços, aterros e estradas. A cerca
em torno de um jardim indicava o seu local na propriedade e o protegeria
dos animais e do vento.

Para diferenciarmos uma planta de jardim ou horta de uma planta de campo
é necessário estudar os métodos de colheita e os sistemas de cultivo
rotativo, pois muitas das plantas denominadas como plantas de jardim têm
o seu cultivo incompatível com os sistemas de cultivo rotativo de campo,
já que são plantas perenes ou bienais. Além disso, muitas plantas ricas em
óleo e fibras prosperam no cultivo de campo, como o falso linho
(\emph{Camelina sativa L}.) e linho (\emph{Linum usitatissimum L}.), mas
também foram cultivadas por métodos de horticultura no sul da Suécia
durante o início da Idade do Ferro. As leguminosas, como a ervilha
(\emph{Pisum sativum L}.) e o feijão (\emph{Vicia faba L}.), prosperavam
em cultivo em larga escala e muitas vezes eram consideradas culturas de
campo, mas evidências adicionais apontam que as ervilhas foram provavelmente
cultivadas em jardins. As leguminosas tinham vários efeitos positivos no
solo e eram utilizadas para repor nutrientes, razão pela qual eram
cultivadas tanto nos jardins como nos campos.

Em algumas das primeiras fontes escritas que tratam de jardinagem e
culturas de jardim, o termo \emph{kålhave} (jardim de cauda) é mencionado;
nesse jardim seriam cultivadas várias espécies de vegetais foliares da
família Brassicaceae, das quais a couve, o repolho e o rabanete fazem
parte. Muitas vezes é difícil determinar as espécies exatas de
\emph{Brassica} com base em macrofósseis de plantas que foram encontradas em
escavações arqueológicas. As espécies selvagens de \emph{Brassica} são
frequentes em vários tipos de solos e, portanto, podem ser consideradas
parte da flora local em áreas com atividade humana, como os
assentamentos.

Muitas plantas comestíveis foram encontradas em muitos sítios
investigados arqueobotanicamente e as sementes dessas plantas
foram detectadas também no intestino dos cadáveres datados da Idade do
Ferro. No entanto, é necessário levar em conta que determinadas plantas
podiam ser ervas daninhas comuns em culturas de campo e que o seu
consumo podia estar associado diretamente à carestia.

\SIG{Luciana de Campos}

Ver também Alimentação; Cotidiano; Cultura material; Era Viking.

\begin{itemize}
\item
  \versal{BOYER}, Régis. \emph{Les Vikings}. Paris: Perrin, 2004.
\item
  \versal{EGGEN}, Mette. The plants used in a Viking Age garden \versal{A.D.} 800-1050.
  In: \versal{MOE}, Dagfinn; \versal{DICKSON}, James \versal{H}; \versal{JØRGENSEN}, Per Magnus (eds.).
  \emph{Garden History}. \versal{PACT} Belgium, Rixensart, 1994, pp. 45-46.
\item
  \versal{HAYWOOD}, John. Agriculture. In: \emph{Encyclopaedia of the Viking
  Age}. London: Thames and Hudson, 2000, pp. 20-21.
\item
  \versal{SLOTH}, Pernille; \versal{HANSEN}, Ulla; \versal{Karg}, Sabine.
  \href{http://www.tandfonline.com/doi/abs/10.1080/21662282.2012.750445}{Viking
  Age garden plants from southern Scandinavia -- diversity, taphonomy
  and cultural aspects}.
  \href{http://www.tandfonline.com/toc/rdja20/1/1}\emph{Danish
  Journal of Archaeology~}, 1(1), 2012, pp. 27-38.
\end{itemize}

\section{\versal{ALIMENTAÇÃO}}\label{alimentauxe7uxe3o}

Quando pensamos em alimentação na Era Viking, imediatamente imaginamos a
clássica cena popularizada pelo cinema, quadrinhos, manuais de \versal{RPG} e
literatura de fantasia: carnes assadas em abundância acompanhadas por
grandes canecas transbordantes de espumante cerveja. Essas imagens são
estereótipos já cristalizados sobre os vikings, mas não correspondem ao
que realmente compunha a mesa nórdica. O consumo de carne assada, como
representado na arte, era muito comum, pois acreditava-se que esse
alimento proporcionava poder e força, e o seu consumo era uma predileção
dos guerreiros e dos nobres; não se tratava apenas de uma questão de
se apreciar o alimento preparado de uma determinada maneira, há uma
explicação ``técnico-gastronômica'' para essa predileção. Ela se opõe ao
gosto dos camponeses, mais fracos, que preferiam a carne cozida: quando a
carne é cozida em água, ou em ocasiões festivas em cerveja,
acreditava-se que o processo de cozimento retiraria toda a força da
carne e, portanto, esse seria um alimento mais rico e bem aproveitado.
Os nobres e guerreiros que apreciavam especialmente a carne de caça
preferiam que esse alimento fosse preparado assado, sobre grelhas ou,
então, em espetos colocados diretamente no fogo, conservando assim o
sabor e conseguindo muitas vezes extrair ainda
um pouco de sangue presente em suas fibras.

Maneiras diferentes de se preparar o mesmo alimento têm, é claro, uma
razão baseada no gosto de cada grupo social: camponeses precisavam
trabalhar muitas horas e preparar a própria comida, portanto, optavam
pelo método do cozimento que permitia que trabalhassem enquanto a comida
era preparada praticamente sozinha, pois a carne permanecia no fogo sem
a necessidade de alguém para vigiá-la. Deixavam a carne muitas vezes
em pedaços grandes e duros com legumes e verduras cozinhando na água por
horas a fio em grossos e pesados caldeirões de ferro. Essa forma de
preparo do alimento não requeria grandes cuidados e podia ser aproveitada
ao máximo. O caldo desse ensopado podia ser consumido com pedaços de pão
elaborado com toda a sorte de farinhas e, mesmo velho e duro, amoleceria,
permitindo que a refeição ficasse mais substanciosa. Já a carne assada,
tão apreciada pelos nobres -- seja nos espetos ou sobre grelhas --,
exigia mais cuidados na hora do preparo. A temperatura do fogo
influenciava na textura da carne: muito fogo poderia queimá-la; 
um fogo fraco deixaria a carne dura e com uma textura pouco
agradável ao ser saboreada. Portanto, a carne assada não reflete apenas
um gosto propriamente dito de saborear o alimento: mostra como grupos
sociais mais abastados, além de terem acesso a carnes mais nobres, podiam
contar com serviçais para prepará-la, preocupando-se apenas com a degustação.

A carne de caça, de gado criado nas pastagens, ou dos peixes com
muita gordura como o salmão e bacalhau abundantes nas águas da Europa
Setentrional, era alimento essencial para um homem,
segundo os ensinamentos do médico medieval Antimo, que no século~\versal{VI}
dizia que a carne possuía o mais alto teor nutritivo. Séculos mais
tarde, outro médico, Aldebrandin de Siena, afirmava que a carne era o mais
completo dos alimentos porque ela não somente alimenta o homem, mas,
acima de tudo, o engorda e lhe concede força. Com a força advinda da
carne, homens e deuses ficavam tonificados para enfrentarem seus
inimigos e adversidades impostas tanto pela natureza como por seres
míticos que os obrigavam a partir em jornadas para recuperarem seus
objetos de poder, como Thor que vai buscar seu martelo, cuja aventura está
descrita na \emph{Þrymskviða}\textbf{.} Guerreiros festejando com carne
em abundância, oriunda de um caldeirão que nunca esgota o seu conteúdo, é
tema recorrente em várias mitologias, como por exemplo o caldeirão
da abundância do deus celta Dagda, o bom. No caldeirão de Dagda
nunca se esgota a comida mágica que alimenta os guerreiros em
quantidade e também traz de volta à vida os guerreiros valentes e
poderosos que tombaram no campo de batalha. Mas a carne não constituía o
único alimento consumido pelos nórdicos e, em muitos momentos, era
escassa, obrigando todos a se alimentarem muitas vezes com o que as
florestas ofereciam como brotos, raízes e pequenos frutos. A alimentação
era variada. Podemos constatar essa variedade na alimentação cotidiana,
observando quais produtos eram consumidos ao longo do dia durante as
principais refeições.

A primeira e mais importante refeição do dia para os nórdicos acontecia
por volta das nove da manhã (e era denominada \emph{dagverd}). Constituía-se 
de papas de cereais: centeio, aveia e cevada com pedaços de
peixe, fresco ou seco, majoritariamente arenque, pães feitos com farinha
de centeio ou aveia, leite (fervido ou coalhado), mel, frutas, como
amoras, framboesas e mirtilos e, no caso da Islândia, consumia-se também o
\emph{skyr}, uma espécie de queijo cremoso que até hoje
é~consumido e fabricado praticamente da mesma maneira da época da
colonização viking. Eventualmente bebia-se cerveja, mais espessa e
amarga, lembrando muitas vezes um caldo grosso, de sabor forte e amargo
e nacos de carnes ensopadas ou assadas com pão. A segunda refeição seria
o jantar (\emph{nåttverđr}), logo após o término dos trabalhos do
dia, quando se comia ensopados de carne ou peixe com pão. Os ovos também
eram consumidos nas sopas, pães e bolos e, muitas vezes cozidos com
frutas, mel e alguns legumes e verduras que encorpavam os ensopados.
Por volta das vinte e uma horas, finalmente, havia uma ceia, que
consistia de uma sopa acompanhada de pão e legumes. Peixes assados, pernis de carneiros ou
mesmo assados inteiros eram comidas reservadas às festividades, pois
exigiam um preparo mais cuidadoso. No cotidiano, a alimentação era
variada, mas preparada de maneira mais simples, já que os trabalhos nos
campos, como a fiação, a tecelagem e a moagem de grãos exigia muita dedicação
de todos.

A alimentação cotidiana de camponeses, fazendeiros e aristocratas era de
certa forma de boa qualidade e contava com uma determinada variedade de
alimentos. É importante ressaltar que nessas refeições não só as carnes
eram consumidas -- os vegetais também recebiam destaque. Legumes, como por
exemplo cenouras, vagens, beterrabas, alho-poró, cebolas, nabos e favas
eram muito usados em ensopados de carne e também em sopas; as frutas,
como os mirtilos, morangos silvestres, framboesas, maçãs, peras e amoras
eram consumidas frescas e secas para conservá-las e assim durarem boa
parte do inverno. E, claro, o mel era usado em pequenas quantidades,
pois seu acesso era restrito. Havia também um grande consumo de ervas em
sopas, caldos e ensopados. A urtiga (\emph{Uritca dioica),} rica em
ferro, cálcio, proteínas e fibras, era utilizada principalmente na
primavera, quando essa erva brotava em abundância pelos campos. O seu
consumo ajudava o organismo a se recuperar de meses consumindo uma dieta
rica em sódio devido aos peixes e carnes conservados em sal, aos frutos
secos e cereais. Tanto a sopa como o chá da urtiga eram amplamente
consumidos e essa erva é também uma das Nove Ervas de Odin, devido a sua
grande importância na alimentação, medicina e magia nórdica.

É importante salientar que essa dieta possuía um caráter salutar, devido
às grandes porções de peixes frescos, secos ou defumados, consumidos
diariamente. Camponeses, fazendeiros e guerreiros alimentavam-se bem,
mas os camponeses sempre viviam com o fantasma da fome a rondar suas
portas e colheitas ruins, invernos muito rigorosos, pouca caça e pesca
também eram tormentos constantes em suas vidas e mesas, tanto quanto
eram comuns as papas de aveia com arenque, tão apreciadas pelo deus
Thor.

Os cereais, como a aveia, trigo, trigo sarraceno (\emph{Fagopyrum
esculentum}), cevada e o sorgo eram largamente utilizados para a
manufatura de farinhas. Os grãos secos eram moídos em moinhos manuais de
pedra. A farinha obtida dessa moagem não era muito fina e os grãos
não eram totalmente moídos, de modo que alguns pedaços ficavam inteiros e deixavam
os pães mais duros, dificultando a mastigação. Para se obter uma farinha
mais fina era necessário refazer a moagem várias vezes até que os grãos
fossem reduzidos a pó. Esse processo era difícil e exigia muito tempo de
trabalho, portanto somente os mais ricos podiam dispor de servos que se
dedicavam somente à moagem de grãos para a obtenção de uma farinha mais
fina. Os mais pobres consumiam o pão com a farinha mais rústica. O
arroz, como se difunde atualmente, não era conhecido na Era Viking.
Alguns restaurantes brasileiros que apresentam um menu dedicado à 
``comida viking'' servem um típico prato muito apreciado nas festas natalinas
da Escandinávia contemporânea: o arroz doce, que começou a ser
consumido somente a partir do século~\versal{XIX}, bem como a canela, especiaria
desconhecida pelos nórdicos medievais.

Todas as casas, das mais pobres até as mais abastadas, possuíam uma horta,
onde se cultivava uma grande quantidade de vegetais. As hortaliças e os
legumes dividiam espaço com uma grande quantidade de ervas,
utilizadas tanto como condimento para as comidas que eram preparadas
como também para fins medicinais e mágicos, sendo largamente empregadas
em chás, unguentos e emplastros para curar todo o tipo de mal que os
afligia.

Podemos afirmar que os nórdicos, apesar de viverem em regiões com
grandes adversidades climáticas e dificuldades de cultivo da terra
possuíam uma alimentação rica tanto em variedade como em nutrientes.

\SIG{Luciana de Campos}

Ver também Caça; Cerveja; Cotidiano; Festas e festins; Hidromel;
Sociedade.

\begin{itemize}
\item
  \versal{BOYER}, Régis. Comer y beber. In: \emph{La vida cotidiana de los
  vikingos (800-1050)}. Barcelona: José J. de Olañeta, Editor, 2000, pp.
  96-101.
\item
  \versal{CAMPOS}, Luciana de. Da suposta noiva que comia demais. Uma proposta de
  análise da alimentação na Þrymskviða. \emph{Roda da Fortuna}, 6 (1),
  2017, pp. 159-173.
\item
  \versal{CAMPOS}, Luciana de. Um banquete para Heimdallr: uma análise da
  alimentação viking na Rígsþula. \emph{História, imagem e narrativas},
  12, 2011, pp. 1-14.
\item
  \versal{FLANDRIN}, Jean-Louis; \versal{MONTANARI}, Massimo (orgs.). \emph{História da
  alimentação}. São Paulo: Estação Liberdade, 1998.
\item
  \versal{HAYWOOD}, John. Food and drink/Feasts and feasting. In:
  \emph{Encyclopaedia of the Viking Age}. London: Thames and Hudson,
  2000.
\end{itemize}

\section{\versal{ALTHING}}\label{althing}

A \emph{Althing} pode ser entendida como assembleia geral da Islândia,
instaurada em 930 usando um sistema legal baseado no
\emph{Gulathing} norueguês. A \emph{Althing} era realizada em um espaço
aberto na planície de Thingvellir, a aproximadamente 50
quilômetros leste da Reykjavík contemporânea, no sudoeste da ilha. A
assembleia era iniciada pelo \emph{Allsherjargodi}, que sacralizava o
início da assembleia geral. Todos os homens livres, excluindo aqueles
declarados fora da lei, encontravam-se na Thingvellir por duas semanas
durante o solstício de verão, e lá as disputas legais mais importantes
eram resolvidas por meio do auxílio do \emph{lögsögumadr,} o
falador das leis.

O falador das leis era eleito por um período de três anos pelos
\emph{godar}, líderes, e tinha que recitar um terço das leis todo ano na
\emph{lögberg}, pedra da lei, de forma que todas as leis islandesas eram
declaradas durante os três anos no cargo. O primeiro falador das leis
foi um homem chamado Úlfjót, que era também responsável por rascunhar as
primeiras leis da Islândia, a \emph{Úlfjótslög}. O falador das leis
também presidia o conselho legislativo da \emph{Althing}, a
\emph{lögrétta}, que era composta por 36 \emph{godar} (esse número
aumentou para 39 depois de 965, e 48 depois de 1005). Algum tempo
depois adicionou-se dois bispos Islandeses ao conselho.

Em 960 a \emph{Althing} foi suplementada por quatro novos tribunais. As
\emph{fjórdungsdómr}, os tribunais de quadrante, eram onde os casos dos
novos quadrantes regionais eram ouvidos caso não pudessem ser
resolvidos nas suas respectivas \emph{things} distritais. Em 1005, uma
quinta corte, \emph{fimtardómr}, foi criada para resolver problemas que
os \emph{fjórdungsdómr} não conseguiam resolver. Nessa corte as
decisões eram feitas pelo voto da maioria, diferente do modelo das
\emph{fjórdungsdómr}, que requeria uma decisão unânime. Com a
conversão ao cristianismo uma outra corte foi criada, a
\emph{préstadómr}, o tribunal dos padres, que tinha a função de
administrar a lei cristã. Ao final de cada \emph{Althing} a reunião era
oficializada com o bater das armas, a \emph{vápnatak,} palavra que
também deu nome as divisões administrativas da Danelaw.

A Islândia perdeu sua independência em 1262-64 e com ela a
\emph{Althing} perdeu muito poder. Os \emph{godar} foram substituídos
por oficiais reais e uma nova legislação foi implementada, modelada a
partir da prática norueguesa e introduzida
oficialmente em 1271. A assembleia se encontrava somente alguns dias
durante o ano, desempenhando um papel somente judiciário em vez de
legislativo. Em 1798, quando a Islândia estava sob controle
dinamarquês, a última sessão da \emph{Althing} foi realizada na
Thingvellir. Em 1800, o rei dinamarquês decidiu que ela deveria ser
substituída por uma suprema corte em Reykjiavik. A \emph{Althing} foi
restabelecida em Reykjavik no ano de 1843 como uma assembleia
consultiva. Essa nova \emph{Althing} consistia de 20 representantes
eleitos, um para cada condado, um de Reykjavik, e seis escolhidos pelo
rei da Dinamarca. O Parlamento Islandês atual é chamado de Althing e
clama ser o parlamento mais antigo do mundo.

\SIG{André Araújo de Oliveira}

Ver também: Godi; Islândia na Era Viking; Thing.

\begin{itemize}
\item
  \versal{HOLMAN}, Katherine. \emph{Histocial Dictionaries of the Vikings}.
  Oxford: The Scarecrow Press Inc., 2003.
\item
  \versal{LINDKVIST}, Thomas. Early political organisation, Introductory survey.
  In: \versal{HELLE}, Knut. (org.). \emph{The Cambridge History of Scandinavia},
  vol. 1. Cambridge: University of Cambridge Press, 2003, pp. 160-167.
\item
  \versal{SIGURÐSSON}, Jón Viðar. Iceland. In: \versal{BRINK}, Stefan; \versal{PRICE}, Neil (eds.).
  \emph{The Viking World}. New York. Routledge, 2008, pp. 571-578.
\item
  \versal{VÉISTEINSSON}, Orri. \emph{The Christianization of Iceland}: Priest,
  Power and social change 1000-1300. Oxford: Oxford University Press,
  2000.
\end{itemize}

\section{\versal{ÂMBAR}}\label{uxe2mbar}

O âmbar foi muito utilizado na Europa por povos como os
romanos e os micênicos para a fabricação de joias,
amuletos, contas, anéis e muitos outros artefatos. O âmbar em forma de
matéria-prima era extraído no sul da Escandinávia, onde os achados
arqueológicos da substância datam de 7000 a.C. Contudo, mesmo em regiões
como a Noruega, onde o âmbar em forma bruta não existe, essa resina
adquiriu também papel em produções de artefatos, como os achados de
pendentes e botões em áreas como a de Trondelag, já durante o período
neolítico.

O âmbar utilizado no mundo escandinavo faz parte do denominado âmbar
báltico, resina proveniente de árvores coníferas que já cresciam no
norte da Europa entre 55 e 35 milhões de anos atrás, durante o Período
Eocênico. Nessa época a Fenoscândia, formada pelo sul da Suécia, o sul
da Finlândia e o Báltico, era uma massa de terra contínua, que por cerca
de 15 a 20 milhões de anos foi coberta pela denominada floresta de
âmbar. Sob a ação do degelo e das águas que se moveram
nas eras procedentes, a matéria-prima foi espalhada por regiões como as partes costeiras
centrais e do sul do atual Báltico, a parte oeste da Jutlândia, a parte
norte da atual Alemanha, os Países Baixos e a Ânglia do Leste.

\SIG{Munir Lutfe Ayoub}

Ver também Arqueologia da Era Viking; Arte; Cotidiano; Cultura material.

\begin{itemize}
\item
  \versal{LARSSON}, Lars. The Sun from the sea-amber in the Mesolithic and
  Neolithic of Southern Scandinavia. \emph{Proceedings of the
  International Conference Baltic Amber in Natural Sciences, Archaeology
  and Applied Arts}, vol. 22, 2001, pp. 65-75.
\item
  \versal{RESI}, Heid Gjøstein. Amber and Jet.~In: \versal{SKRE}, Dagfinn (ed.).
  \emph{Things from the Town: artefacts and inhabitants in Viking-Age
  Kaupang}. Aarhus \& Oslo: Aarhus University Press, 2011, pp. 107-128.
\item
  \versal{SHASHOUA}, Yvonne. Raman and \versal{ATR}‐\versal{FTIR} spectroscopies applied to the
  conservation of archaeological Baltic amber.~\emph{Journal of Raman
  Spectroscopy}, vol.~37, 2006, pp. 1221-1227.
\item
  \versal{WEITSCHAT}, Wolfgang; \versal{WILFRIED}, Wichard. Baltic amber. In: \versal{WEITSCHAT},
  Wolfgang; \versal{WILFRIED}, Wichard; \versal{PENNEY}, David (eds.).~\emph{Biodiversity
  of fossils in amber from the major world deposits}. Manchester: Siri
  Scientific Press,~2010, pp. 80-115.
\end{itemize}

\section{\versal{ANGLO}-\versal{SAXÕES E NÓRDICOS}}\label{anglo-saxuxf5es-e-nuxf3rdicos}

As relações entre povos de origem anglo-saxã e escandinavos na
Inglaterra, ao longo dos séculos \versal{IX}--\versal{XI}, foram marcadas por constantes
contatos. A primeira fase, que vai até metade do século \versal{IX}, pode ser
identificada por diversos ataques esporádicos e em territórios diversos
da ilha, de acordo com a \emph{Crônica Anglo-Saxônica}. A partir de 865,
começamos a observar referências à permanência dos exércitos ao norte e
no antigo reino anglo da Mercia na organização de \emph{wintersetl}
(acampamentos de inverno), os quais normalmente duravam a estação ou
cerca de um ano. Foi certamente nesse período que uma maior quantidade
de grupos de origem nórdica se dirigiu para a região.

Os ataques foram múltiplos e em locais diversos, envolvendo líderes,
tais como Ivar, possivelmente associado ao rei dos vikings na Irlanda,
Guthrum e Halfdan, dentre outros. Concomitantemente aos ataques
escandinavos, o rei de Wessex, Alfred (871-899), direciona suas forças e
estratégias para a construção dos \emph{burhs} (fortificações), peças
fundamentais para conter o avanço dos invasores.

O Tratado de Wedmore foi firmado em 878, após a vitória anglo"-saxã sobre
os escandinavos na batalha de Edington, no mesmo ano. O acordo entre
Alfred e Guthrum delimitou a área, que posteriormente ficará conhecida
como \emph{Danelaw}, área das Midlands que estaria sob influência
escandinava, fora do escopo de Wessex. Outra consequência gerada pelo
acordo foi o batismo de Guthrum e sua incorporação ao sistema de
liderança dos anglo-saxões.

A partir do século~\versal{IX}, já há na Inglaterra assentamentos de origem
escandinava em áreas anteriormente anglo"-saxãs. Os ataques são retomados
depois durante os reinados de Edgar (957-975) e posteriormente de
Æthelred~\versal{II} (978-1016), passando ao método de extorquir a população nativa,
concentrando-se bem mais ao sul, no centro do poder de Wessex.

O ápice da ocupação política da ilha são as
batalhas travadas entre Æthelred~\versal{II} e Sueno Barba Bifurcada, que resultaram na
vitória deste e exílio do rei inglês na Normandia. Após a morte de
Sueno, em 1014, seu filho Canuto assumia o controle da Inglaterra, dando
continuidade à influência escandinava. Æthelred retornaria à ilha no
mesmo ano e permaneceria como rei até 1016, quando, após o ataque de
Canuto novamente, os escandinavos estariam estabelecidos como governantes
da ilha uma vez mais.

Na documentação escrita em latim e em inglês antigo nos deparamos
constantemente com rótulos como \emph{Angelcyn} (inglês),
\emph{Angulsaxonum} (anglo-saxão), \emph{paganus} (\emph{pagão)} e
\emph{Dane} (danes), todas essas denominações criadas no ambiente
da corte real de Wessex como uma maneira de diferenciar as rivalidades
políticas, principalmente no contexto de invasão e ocupação escandinava.
Todavia, as identidades impostas por externos não traduzem
necessariamente os habitantes envolvidos, nem definem suas próprias
identidades em si.

Os contatos entre anglo-saxões e escandinavos, fossem amistosos ou não, 
acabaram por impactar e transformar ambas as
sociedades. Estudos arqueológicos apontam que os contatos entre
anglo-saxões e nórdicos foram bem mais complexos do que apontam as
denominações na documentação escrita. Nas regiões em que organizaram
assentamentos, os escandinavos acabaram por se mesclar à população
local, principalmente nas regiões norte e nordeste da Inglaterra, em
Derbyshire, Lincolnshire, Yorkshire, Lancashire e Cumbria.

O conceito de anglo-escandinavo, portanto, é o que melhor traduz as
aspirações e necessidades na compreensão acerca das identidades nos
assentamentos, as quais não eram constituídas unicamente em termos
étnicos, mas a partir de relações sociais específicas e das escolhas dos
sujeitos em privilegiar certos elementos em detrimento de outros. As
principais fontes que permitem ao pesquisador melhor compreender o
período são basicamente vestígios da cultura material, haja vista que as
fontes escritas produzidas na região, sejam elas narrativas ou legais,
são muito escassas.

A análise das relações linguísticas entre anglo-saxões e escandinavos é
um caminho para a compreensão de como era a convivência entre os dois
grupos. A adoção de nomes próprios anglo-saxões por escandinavos (e vice-versa)
nos dão algumas pistas das relações entre as elites locais
em áreas de assentamentos. Adotar um novo nome ou um novo idioma é uma
forma de reconfigurar os laços com a comunidade local.

De acordo com Julian Richards, há quatro categorias principais para se
avaliar os topônimos escandinavos: 1) a partir do sufixo -\emph{by}, que
significa aldeia; 2) a partir do sufixo -\emph{thorp}, que designa normalmente
áreas secundárias subordinadas a outra em termos de exploração; 3)
\emph{Grimston hybrids}, uma combinação de elementos de nomes próprios
escandinavos com o sufixo em inglês antigo -\emph{ton}; 4) mudanças na
pronúncia de palavras anglo-saxãs, a fim de evitar sons
não escandinavos.

Os tipos de escultura nas quais podemos encontrar traços escandinavos
são muitos, tais como cruzes, tábuas, tampos de tumbas etc. e são
encontrados majoritariamente nas regiões norte e nordeste da Inglaterra.
As esculturas de pedra dos séculos~\versal{X}-\versal{XI} diferem da do período
anglo-saxão não só em ornamentação, mas também com relação à sua função.
A utilização destes monumentos enquanto artefatos funerários, pois a
maioria se encontra em cemitérios paroquiais, sugere que estes foram
feitos para uma elite escandinava. A confecção de cruzes já era uma
prática recorrente no período anglo-saxão, mas com a presença
escandinava podem ser nelas observados elementos estilísticos distintos, 
nos quais está presente a referência a um passado pré-cristão.
Exemplos desses artefatos são as cruzes de Gosforth, no noroeste da
Inglaterra, no cemitério da igreja de St. Mary, e a Cruz de Middleton,
datada do século \versal{IX-X}, que se encontra na igreja de St. Andrew, em Yorkshire

\SIG{Isabela Dias de Albuquerque}

Ver também Crônica anglo-saxônica; Danevirke; Danelaw; Inglaterra da Era
Viking;

\begin{itemize}
\item
  \versal{HADLEY}, Dawn M; \versal{RICHARDS}, Julian D. \emph{Cultures in Contact:
  Scandinavian Settlements in England in the Ninth and Tenth Centuries}.
  Turnhout: Brepols, 2009.
\item
  \versal{HADLEY}, Dawn M. \emph{The Vikings in England: Settlement, society and
  culture}. Manchester: Manchester University Press, 2006.
\item
  \versal{HADLEY}, Dawn M. Viking and native: re-thinking identity in the
  Danelaw. \emph{Early Medieval Europe}, vol. 11, Issue 1, 2002, pp.
  45-70.
\item
  \versal{SAWYER}, Peter (ed.). \emph{The Oxford Illustrated History of the
  Vikings.} Oxford: Oxford University Press, 2001.
\item
  \versal{RICHARDS}, Julian D. \emph{Viking Age England}. Stroud: The History
  Press, 2007.
\item
  \versal{TOWNEND}, Matthew. \emph{Language and History in Viking Age England:
  Linguistic Relations between Speakers of Old Norse and Old English}.
  Turnhout: Brepols, 2002.
\end{itemize}

\section{\versal{ANNÁLA ULADH (ANAIS DE
ULSTER)}}\label{annuxe1la-uladh-anais-de-ulster}

Os \emph{Anais de Ulster} (\emph{Annála Uladh}) são uma das mais relevantes
fontes manuscritas irlandesas medievais e fazem um registro que cobre
dos anos 431 d.C. até 1540 d.C. Esse documento sobreviveu em dois
manuscritos atualmente em posse da biblioteca do Trinity College em
Dublin e da biblioteca Bodleian em Oxford. Seus principais escribas são
Ruaidhrí Ó Luinín e Ruaidhrí Ó Caiside com o patrocínio de
Cathal Mac Maghnusa. Outras informações foram adicionadas
posteriormente por outros escribas. Apesar disso, o manuscrito ainda
possui algumas lacunas. Existem também algumas cópias tardias, com
traduções feitas em inglês e latim derivadas da pesquisa histórica de
Sir James Ware e que são dignas de nota por conterem algumas adições e
releituras de acontecimentos pós século~\versal{XII}.

Os \emph{Anais de Ulster} são conhecidos pela fidelidade com a qual antigas
estruturas léxicas em irlandês antigo foram mantidas, mesmo com
arcaísmos, o que faz deles uma fonte que goza de grande autoridade em
comparação a outros anais irlandeses. No entanto, o texto apresenta
alguns problemas cronológicos de continuidade, sobretudo nos períodos
iniciais. O texto em si é baseado nas crônicas de Iona de c. 740, bem
como em anais compilados em Armagh e Clonard por volta de meados do
século~\versal{X}.

Os \emph{Anais de Ulster} são, das fontes irlandesas, aquela que contém o maior
número de informação sobre o período inicial dos assentamentos
escandinavos na ilha da Irlanda, principalmente informações sobre as
primeiras invasões ao solo irlandês após o saque de Lindisfarne em 793.
Bem verdade que as primeiras menções aos vikings nesse período são bem
vagas, usando diferentes palavras para mencionar o ocorrido e com termos
como ``heathens'', indicando que são grupos estrangeiros que o fazem.
Nos diversos registros dos \emph{Anais de Ulster} é possível ver uma variedade
de nomes diferentes para designar os povos escandinavos que os atacam.
Além de estrangeiros propriamente ditos, encontramos termos como
``Homens do Norte'', ``Nórdico-irlandeses'', ``Dinamarqueses'', ``belos
estrangeiros'', ``estrangeiros escuros'' etc. Por conta de os relatos
serem em geral curtos e bem resumidos, fica difícil inferir que estes
nomes possam designar alianças da época ou algum tipo de nacionalidade,
mesmo que desde o primeiro relato existente nos \emph{Anais} todos indiquem
que são grupos desconhecidos ou de fora que os atacam.

Esse primeiro registro é também um tanto quanto breve, mencionando
apenas ``O Incêndio de Rechru pelos estrangeiros [\emph{heathens}], e
Scí foi sobrepujada e deixada apodrecer''. Sendo ``Rechru'' a
ilha de Rathlin, localizada ao norte da costa do condado de Antrim e
``Scí'' a ilha de Sky nas ilhas hébridas escocesas.

Os \emph{Anais} também relatam a formação dos primeiros assentamentos
escandinavos na região irlandesa, como Dublin, por exemplo, em notas
esparsas como é o caso do registro \versal{AU}\,841.4, onde é mencionado que
``Havia um acampamento naval em Linn Duachaill, que saqueou os povos e
igrejas da Tehba. Havia um acampamento naval em
Duiblinn, que saqueou os Laigin e os Uí Néill, tanto as
localidades quanto as igrejas até Sliab Bladma''. Neste trecho fica bem
claro que, para além da invasão de igrejas da região de Tehba (região
onde hoje é o condado de Longford e Westmeath) e do saque de províncias
como Laigin e Uí Néill, bem como as igrejas até a região de Sliab Bladma
(cadeia de montanhas entre os condados de Offaly e Laois), é relatada a
formação de assentamentos escandinavos como o porto de Linn Duachaill e
o assentamento de Dubh Linn. Enquanto o primeiro foi abandonado com o
tempo, o segundo formou o que hoje conhecemos como a cidade de Dublin.

Além de relatos fundacionais como os descritos acima, os \emph{Anais de Ulster}
também são conhecidos por relatar alguns contatos cotidianos e bélicos
entre os vikings e grupos locais. Nestes relatos, algumas das batalhas
mais conhecidas do período são mencionadas como a Batalha de Brunanburth
(\versal{AU} 937.6), a Batalha de Tara (\versal{AU} 980.1) e a Batalha de Clontarf (\versal{AU}
1014.1). Esta última tem grande expressão na história irlandesa por
envolver a figura de Brian Boru, líder que teria expulsado os vikings da
região, decretado um fim à Era Viking e que nos dias atuais goza de
certo prestígio mítico entre os heróis históricos irlandeses. No
entanto, os mesmos relatos mencionados acima comprovam uma versão
diferente da que popularmente descreve os escandinavos apenas como
invasores inimigos, pois em muitos deles pode-se encontrar vikings como
aliados de grupos gaélicos rivais, batalhando por um lado ou outro das
disputas internas da ilha, além de encontrarmos descrições de chefes
gaélicos saqueando de maneira parecida com as desses estrangeiros.

Os \emph{Anais de Ulster}, então, constituem uma das principais fontes para se
compreender alguns aspectos não apenas da presença escandinava na
Irlanda, mas também da sua própria história. Os \emph{Anais} também servem de
base para a concepção de estudos de linguística por conta de suas partes
em irlandês arcaico e para os demais estudiosos de manuscritos
históricos, visto que eles oferecem embasamento para o estudo de outros
manuscritos, como os \emph{Annals of the Four Masters} (\emph{Anais dos Quatro
Mestres}) e o texto \emph{Cogad Gáedhel re Gallaib} (\emph{A Guerra dos
Irlandeses com os Estrangeiros}).

\SIG{Erick Carvalho de Mello}

Ver também Brian Boru; Celtas e nórdicos; Dublin; Irlanda da Era Viking.

\begin{itemize}
\item
  \versal{DOWNHAM}, Clare. Irish chronicles as a source for inter-Viking rivalry,
  \versal{A.D.} 795-1014. \emph{Northern Scotland}, 26, 2006, pp. 51-63.
\item
  \versal{DOWNHAM}, Clare. \emph{Viking Kings of Britain and Ireland}. Edinburgh:
  Dunedin Academic Press, 2007.
\item
  \versal{DUFFY}, Seán. \emph{Brian Boru and the Battle of Clontarf}. Dublin:
  Gill Books, 2014.
\item
  \versal{MAC} \versal{NIOCAILL}, Gearóid. \emph{The medieval Irish annals}. Dublin:
  Dublin Historical Association, 1975.
\item
  \versal{\versal{Ó CUÍV}}, Brian. Ireland in the Eleventh and Twelfth Centuries c.
  1000-1169\emph{.} In: \versal{MOODY}, Theodore \versal{W}. \& \versal{MARTIN}, Francis \versal{X}.
  \emph{The Course of Irish History.} Cork: Mercier Press, 2011, pp.
  107-122.
\item
  \versal{PAOR}, Liam de. The Age of the Viking Wars: 9\textsuperscript{th} and
  10\textsuperscript{th} centuries\emph{.} In: \versal{MOODY}, Theodore \versal{W}. \&
  \versal{MARTIN}, Francis \versal{X}. \emph{The Course of Irish History}. Cork: Mercier
  Press, 2011, pp. 91-106.
\item
  \versal{RICHTER}, Michael. \emph{Medieval Ireland: The Enduring Tradition}.
  Dublin: Gill and Macmillan, 1988.
\end{itemize}

\section{\versal{ANNALES DE FLODOARDO DE REIMS}}

Padre e cônego de Reims, Flodoardo (893/4-966) foi o responsável pela
escrita dos anais carolíngios durante um período de mais de 40 anos.
Alguns autores apontam que ele teria começado a escrevê-los em 919,
outros em 925, mas há um consenso sobre o fato de que ele os escreveu
até sua morte em 966. Flodoardo foi um importante ator político da
porção ocidental do decadente Império Carolíngio e os anais escritos por
ele têm grande relevância historiográfica devido à sua visão
privilegiada sobre o conturbado período de desmantelamento e consequente
desintegração do Império. Os textos foram publicados de forma completa
por Phillipe Lauer em 1905 sob o título de \emph{Les annales de
Flodoard}, e além disso figuram constantemente em estudos sobre o
estabelecimento viking na Normandia e em coletâneas de fontes sobre o
período.

Para os propósitos do presente Dicionário, a importância de Flodoardo se
dá mais especificamente por sua visão e narrativa sobre o
estabelecimento do futuro ducado da Normandia, no noroeste da atual
França, durante o começo do século~\versal{X}. Comentando sobre este fato,
Elizabeth Van Houts afirma que Flodoardo é o autor que melhor oferece
uma visão contemporânea do nordeste da França sobre a chegada e
estabelecimento dos vikings na Normandia. Por conta disso, sua visão é
bastante única e consequentemente muito difícil de ser verificada.

Por viver a uma grande distância do lugar sobre o qual narrava,
Flodoardo de Reims tinha sua fonte de informações em homens que lutavam
contra os vikings e se engajavam em atividades missionárias. Seus
relatos são curtos e parecem comunicar informações a pessoas que já
sabem minimamente sobre o que ele está falando, em textos
difíceis de serem interpretados. No entanto, seus relatos muitas vezes
constituem a única fonte de informação existente sobre a atividade
viking na costa logo após seu estabelecimento na região da atual
Normandia em 911.

Os escritos de Flodoardo são também muito utilizados pelos historiadores
que estudam o período de consolidação da Normandia como forma de
estabelecer uma contraposição contemporânea à visão proveniente dos
próprios normandos, fornecida por Dudo de St-Quentin em sua \emph{Gesta
Normannorum}. Em sua tese sobre o tema, Katherine Cross utiliza os
relatos de Flodoardo, que apontam uma série de concessões de terra aos
normandos, como uma forma de contrapor a versão fornecida por Dudo, que,
motivado por uma série de questões políticas de seu próprio tempo,
afirma que o território da Normandia teria surgido no começo do século~\versal{X}
já com todas as suas fronteiras posteriores firmemente estabelecidas.

Em vários trechos de seus anais, Flodoardo se refere aos vikings
estabelecidos na região da Normandia sob um ponto de vista não muito
favorável. Em uma passagem importante, das páginas 15 a 17 na edição de
Lauer, o autor fala sobre Rognvald, um líder dos normandos na região do
Loire que, segundo os anais, reuniu uma quantidade razoável de homens e
passou a pilhar algumas regiões francas ao longo do rio Oise. Rechaçados
por hostes comandadas por alguns líderes locais francos, os normandos
deixaram as áreas que antes ocupavam ao longo rio, levando consigo
grande valor de saque.

É a partir desta narrativa que Flodoardo aponta que a região que havia
sido concedida aos normandos por Carlos em 911 era violenta e
conturbada. O autor aponta os normandos como culpados, dizendo que eles
não honraram sua conversão na fé de Cristo e a paz que havia sido
acordada em decorrência. Como decorrência deste episódio, aponta o
autor, os normandos chegaram a um acordo de paz com os francos e
acabaram por ocupar uma quantidade maior de terras através do Sena do
que lhes originalmente havia sido concedido.

Em uma grande quantidade de relatos, Flodoardo aos poucos relata a
violência que foi trazida pelos vikings com seu estabelecimento na
região da Normandia. Sendo assim, seus relatos são extremamente
importantes por nos possibilitar uma visão de um importante ator
político e religioso franco sobre o estabelecimento dos povos vikings
liderados por Rollo no noroeste da atual França. Além disso, por ser um
dos únicos relatos sobre o período, os escritos de Flodoardo são de
inestimável valor para as pesquisas sobre o estabelecimento do ducado
normando, fornecendo-nos um contexto político que vai desde a liderança
de Rollo, passando pela governança de seu filho Guilherme e a conturbada
ascensão de seu neto Ricardo~\versal{I}, responsável pela consolidação do ducado
da Normandia e de sua linhagem na segunda metade do século~\versal{X}.

\SIG{Thiago Brotto Natário}

Ver também França na Era Viking; Rollo; Normandia; Vikings na França.

\begin{itemize}
\item
  \versal{CROUCH}, David. \emph{The Normans: the history of a dinasty}. London:
  Hambledon Continuum, 2002.
\item
  \versal{CROSS}, Katherine Clare.~\emph{Enemy and ancestor: viking identities
  and ethnic boundaries in England and Normandy, c. 950-c. 1015}. Tese
  de Doutorado, \versal{UCL} (University College London), 2014.
\item
  \versal{LAUER}, Philippe (ed.).~\emph{Les annales de Flodoard}. Paris: Alphonse
  Picard et fils, 1905.
\item
  \versal{LE PATOUREL}, John. \emph{The Norman Empire}\textbf{.} Oxford: Oxford
  University Press, 1976.
\item
  \versal{VAN} \versal{HOUTS}, Elizabeth. \emph{The Normans in Europe}. Manchester:
  Manchester University Press, 2000.
\end{itemize}

\section{\versal{ANNALES FULDENSES}}\label{annales-fuldenses}

Os \emph{Annales Fuldenses}, ou simplesmente \emph{Annales de Fulda},
receberam grande atenção por parte da historiografia dedicada ao estudo
do período carolíngio, principalmente por serem a principal fonte
narrativa escrita sob a perspectiva da porção do Império Carolíngio que
ficou a leste do rio Reno após as sucessivas divisões. Os registros
abordam o período que vai desde os últimos anos do Império ainda unido
sob o governo de Luís, o Pio, até o fim do domínio efetivo dos
carolíngios sobre a Frância em 900.

Não há um consenso historiográfico acerca da autoria destes
\emph{annales}, mas é possível afirmar, sem grande margem de erro, que
estes documentos foram escritos no monastério beneditino de Fulda,
região da atual Alemanha. Alguns historiadores especulam que o
responsável pela escrita dos textos seria um monge de Fulda chamado
Rudolf, que teria escrito os \emph{annales} de 838 até 863, dois anos
antes de sua morte. O historiador Roger Collins corrobora tal visão,
apontando que foi a presença de uma referência a tal autor em um dos
manuscritos mais antigos que levou a atribuição de toda a obra ao
monastério de Fulda.

Collins comenta que atualmente temos conhecimento de três diferentes
versões do texto. A primeira versão narra os eventos de 838 até 863 e
geralmente é lida como uma espécie de contraparte aos \emph{Annales
Bertiniani}, escritos na parte oeste do Império Carolíngio, uma vez que
ambos descrevem praticamente os mesmos eventos sob perspectivas
diferentes. A segunda das versões demonstra uma grande parcialidade em
relação ao futuramente malfadado imperador Carlos, o Gordo, morto em
888. Já a terceira inclui uma continuação cobrindo os anos que vão de
882 até 901 e provavelmente foi escrita na região da Bavária.

A narração dos eventos começa a partir do fim do domínio de Luís, o Pio
e de sua morte em 840. Os \emph{Annales} falam também sobre a divisão do
Império Carolíngio em três partes no Tratado de Verdun em 843. Após o
ano de 860, o principal foco dos Annales Fuldenses passa a ser os
eventos que tomam lugar na porção oriental do Império Carolíngio,
discorrendo principalmente sobre Luís, o Germânico e seus filhos e
sucessores.

Um dos principais usos dos \emph{Annales Fuldenses} dizem respeito ao
estudo das principais invasões sofridas pelo Império Carolíngio após
suas sucessivas divisões. Um exemplo disso está presente no livro
\emph{Franks, Moravians, and Magyars: The Struggle for the Middle
Danube, 788-907}, de Charles R. Bowlus. Ao falar sobre os \emph{magyars},
denominação da época para os húngaros, Bowlus aponta para a
incongruência entre as duas principais fontes sobre o período. Enquanto
os \emph{Annales Bertiniani} afirmam que os \emph{ungri} devastaram a
parte oriental do Império em 862, os \emph{Annales Fuldenses} mantêm
silêncio sobre o assunto. Utilizando os \emph{Annales Bertiniani} e
outras fontes é possível afirmar que houve de fato ataques vindos do
leste sobre o Império de Luís, o Germânico, mas que os \emph{Annales de
Fulda} optaram por não mencioná-los em seus relatos.

Para os propósitos do presente Dicionário, são de extrema importância os
relatos trazidos pelos \emph{Annales Fuldenses} sobre as invasões
vikings acometidas ao longo de todo Império Carolíngio a partir de 845.
Os textos descrevem desde os esforços empreendidos pelos carolíngios
para conter as invasões até as negociações e pagamentos realizados para
afastá-los das fronteiras do Império.

Falando sobre o ano de 845 em si, os \emph{Annales} apontam para a
existência de uma grande força de homens vikings que teriam saqueado o
reino de Carlos, o Calvo, navegando pelo rio Sena até Paris, e que estes
teriam sido afastados após receberem uma grande quantidade em dinheiro
do rei e dos habitantes da região. Já em 850, os \emph{Annales} falam
sobre um Roric, homem dinamarquês que assolou o território de Lotário e teria
finalmente sido aceito em seu reino sob juramento de fidelidade. No
mesmo ano há a descrição sobre um Godafrid, que teria sido também aceito
sob o território de Carlos.

Os \emph{Annales Fuldenses} são essenciais para o estudo da Era Viking e
de sua integração com o Ocidente medieval porque nos mostram que as
invasões vikings não foram conduzidas em lugares específicos e nem por
um mesmo grupo. Os \emph{Annales} nos dão relatos de vários episódios em
que diferentes grupos e regiões da Escandinávia assolaram a costa do
Império Carolíngio, tanto a oeste quanto a leste. Além disso, mostram-nos
também que o estabelecimento de vikings e sua posterior cristianização e
criação do ducado da Normandia em 911 têm precedentes anteriores, uma
vez que os \emph{Annales} falam sobre vários líderes integrando-se à
política e ao território do Império Carolíngio.

\SIG{Thiago Brotto Natário}

Ver também França na Era Viking; Rollo; Normandia; Vikings na França.

\begin{itemize}
\item
  \versal{BOWLUS}, Charles R.~\emph{Franks, Moravians, and Magyars: the struggle
  for the Middle Danube, 788-907}. Filadélfia: University of
  Pennsylvania Press, 1995.
\item
  \versal{COLLINS}, Roger.~\emph{Early Medieval Europe, 300-1000}. New York:
  Palgrave Macmillan, 2010.
\item
  \versal{COUPLAND}, Simon.~\emph{Carolingian Coinage and the Vikings: Studies on
  Power and Trade in the 9th Century}. Aldershot: Ashgate Publishing,
  Ltd., 2007.
\item
  \versal{REUTER}, Timothy (ed.).~\emph{The Annals of Fulda: Ninth-century
  Histories}, vol. 2. Manchester: Manchester University Press, 1992.
\end{itemize}

\section{\versal{ANNALES REGNI FRANCORUM}}\label{annales-regni-francorum}

Os \emph{Annales Regni Francorum} são textos em latim escritos durante o
Império Carolíngio, cobrindo o período que vai desde a morte de Carlos
Martel até o início da desintegração do Império sob Luís, o Pio, em 829.
A autoria dos \emph{Annales} é desconhecida e a maioria dos
historiadores afirmam que eles teriam sido escritos por diversos autores
diferentes e posteriormente compilados como uma obra única.

Em sua tradução e compilação da obra em 1970, Bernhard Scholz aponta que
o manuscrito mais antigo dos \emph{Annales} foi encontrado no monastério
de Lorsch, próximo a Worms, atual Alemanha. No entanto, historiadores
clássicos do século \versal{XIX} como Leopold Von Ranke apontaram que os textos
haviam sido escritos dentro da corte real, por conta de suas
características de uma escrita breve e direta, que pressupõe que seu
leitor já tenha conhecimento prévio dos temas políticos, diplomáticos e
militares sobre os quais os textos discorrem. Além disso, Scholz comenta
que a iniciativa da escrita dos \emph{Annales} muito provavelmente foi
incentivada por Carlos Magno, como parte de sua política de preservação
de documentos e da história de seu reino.

É de certa forma consenso entre os historiadores do período dividir os
\emph{Annales Regni Francorum} em três partes. A primeira, narrando
eventos de 741 até 795 teria sido escrita por um único autor. Ranke
apontou que essa primeira parte omite muitos problemas internos e parece
demonstrar um grande e próximo conhecimento das relações políticas
internas registradas no texto. Em uma visão que é de maneira geral
aceita até os dias atuais, Ranke afirmou que esta primeira parte dos
\emph{Annales} era de uma compilação oficial da história carolíngia,
encomendada por Carlos Magno a um monge, que teria recebido da corte
todas as informações necessárias para a escrita dos textos. A segunda
(795-807) e terceira (808-829) partes registram eventos aos quais muito
provavelmente foram contemporâneas, com alguns autores apontando para
uma subdivisão da terceira parte, com marco em 819.

De modo geral, os \emph{Annales Regni Francorum} dividem-se em
narrativas sobre cada um dos anos do período que cobrem, comentando
os grandes feitos dos homens de seu tempo, em especial os monarcas,
destacando as vitórias em batalha e expansão da cristandade promovida
pelos carolíngios. Os relatos dão especial destaque as vitórias
militares de Carlos Magno e omitem suas derrotas, principalmente as
batalhas de Roncesvales e Süntel, bem como a conspiração promovida por
um suposto filho bastardo de Carlos Magno, Pepino, o Corcunda.

Representando uma espécie de ``visão oficial'' dos carolíngios, os
\emph{Annales Regni Francorum} não relatam muitos detalhes sobre as
invasões vikings sofridas no período. No entanto, podemos constatar o
surgimento de alguns importantes atores vikings em diversos momentos dos
textos. Emissários de um rei nórdico chamado Sigifrid aparecem em uma
assembleia de Carlos Magno em 782. Já em 808 podemos encontrar um relato
da tentativa de um rei chamado Godofrid de invadir a Saxônia, sendo
prontamente rechaçado pelo filho do rei, Carlos. Em 810 há relatos de um
novo ataque de Godofrid, que conseguiu pilhar toda a costa da Frísia,
cobrando tributos dos nobres da região. Os \emph{Annales} incluem uma
série de justificativas, desde a morte do próprio Godofrid até a ocasião
de uma peste dentre o gado, que teriam impedido o imperador Luís de
cumprir seu desejo de lutar contra os dinamarqueses.

A partir de 814 os \emph{Annales} falam brevemente sobre uma disputa
interna em um suposto reino dos normandos, envolvendo os filhos de
Godofrid e um Heriold, que veio até o imperador pedindo auxílio e
recebeu o território de Riistringen, na Frísia, sob a proteção de Luís,
o Pio. Em 828 os \emph{Annales} nos informam que Luís mantinha-se como
aliado de Heriold contra os filhos de Godofrid, até que Heriold ataca os
normandos após tentativas de paz, gerando nova guerra entre francos e
normandos. Não fica claro como o conflito termina, uma vez que os
relatos se encerram com uma descrição do momento em que os filhos de
Godofrid mandam uma embaixada ao imperador afirmando que haviam sido
propelidos à guerra e solicitam seu arbítrio na questão.

Os \emph{Annales Regni Francorum} foram e continuam sendo estudados por
diversos historiadores, uma vez que são um dos poucos e mais completos
textos sobre o período de ascensão e queda do Império
Carolíngio, destrinchando boa parte de suas questões políticas internas.
Além disso, como predecessores dos \emph{Annales Fuldenses e
Bertiniani}, eles nos fornecem uma grande quantidade de informações
sobre as relações estabelecidas por homens vindos do norte com os
carolíngios, sejam elas bélicas ou negociadas, permitindo-nos uma melhor
compreensão do período.

\SIG{Thiago Brotto Natário}

Ver também França na Era Viking; Rollo; Normandia; Vikings na França.

\begin{itemize}
\item
  \versal{BRINK}, Stefan; \versal{PRICE}, Neil (eds.).~\emph{The Viking World}. Abingdon:
  Routledge, 2008.
\item
  \versal{CROSS}, Katherine Clare.~\emph{Enemy and ancestor: viking identities
  and ethnic boundaries in England and Normandy, c. 950-c. 1015}. Tese
  de Doutorado, \versal{UCL} (University College London), 2014.
\item
  \versal{MCKITTERICK}, Rosamond.~\emph{History and memory in the Carolingian
  world}. Cambridge: Cambridge University Press, 2004.
\item
  \versal{MCKITTERICK}, Rosamond.~\emph{Charlemagne: The formation of a European
  identity}. Cambridge: Cambridge University Press, 2008.
\item
  \versal{SCHOLZ}, Bernhard Walter (ed.).~\emph{Annales regni Francorum}, vol.
  186. Ann Arbor: University of Michigan Press, 1970.
\end{itemize}

\section{\versal{APARÊNCIA E COSTUMES}}\label{aparuxeancia-e-costumes}

No que diz respeito à aparência tanto de homens como de mulheres
durante a Era Viking, podemos dizer que, de um modo geral, eram
elegantes, pois os mais abastados podiam vestir-se com roupas feitas com
a lã de carneiros especiais que produziam uma lã macia e sedosa, ou
então com linho e até com tecidos finos, como a seda vinda de
Bizâncio. Além dos tecidos caros também usavam joias feitas com contas
de vidro, âmbar, conchas, pedras e metais preciosos. Altos, de 
estatura muito próxima a dos escandinavos contemporâneos, esses
homens e mulheres tinham uma altura que variava entre 1,70cm e 1,80cm,
com cabelos e tez clara em sua maioria. A constituição física dos
nórdicos medievais era muito parecida com a nossa. Podemos dizer que,
devido aos constantes trabalhos braçais realizados tanto por homens como
por mulheres, a musculatura das populações da Era Viking devia ser mais
forte do que atualmente. Os rostos de homens e mulheres eram mais
parecidos do que são hoje. Os rostos femininos possuíam os nervos da
testa protuberantes e, por outro lado, os homens tinham o maxilar
saliente e os nervos menos ressaltados. Essas características faciais
ambíguas significam que é mais difícil decidir sobre o sexo de um
esqueleto nórdico baseado apenas no crânio. Portanto, outros traços
precisam ser estudados para identificar o sexo dos esqueletos, como
a largura da pelve.

Os cabelos sempre foram um dos adornos mais importantes usados tanto por
homens como por mulheres e a eles era dada uma atenção maior, daí a
necessidade de pentes especiais para os seus cuidados diários. Os
cabelos podiam ser utilizados simplesmente para se enfeitar ou
seduzir, ou arrumados para agradar aos deuses e, assim, protegerem-se
contra possíveis infortúnios, bem como demonstravam o \emph{status} social.
Uma farta cabeleira bem arrumada era mais do
que um simples acessório de beleza em uma mulher. Esses penteados podiam 
apresentar maiores possibilidades de análise do seu uso e não apenas
restringir-se à habilidade manual para a composição de tranças:
essas tramas capilares são reveladoras de posições sociais, de
estado civil, de serviço religioso e de utilização mágica.

Os cabelos longos sempre estiveram ligados à virilidade, à força e
também à liberdade. A literatura, as artes plásticas, o cinema e mais
recentemente os jogos de \versal{RPG} e eletrônicos sempre apresentaram os
guerreiros mais fortes e as mulheres mais belas com vastas e espessas
cabeleiras -- as madeixas femininas muitas vezes caíam até a altura da
cintura ou ainda mais longas. A arte pré-rafaelita sempre
apresentou as mulheres, que na maioria das vezes eram personagens da
mitologia e do folclore nórdicos, com cabelos muito longos e geralmente
soltos para reforçar o seu caráter de sedução e também mostrar que os
cabelos muito longos constituíam um padrão de beleza da Era Viking.
Essas representações das longas cabeleiras, tanto masculinas como
femininas, que sobreviveram ao longo do tempo nas artes e no
imaginário popular, foram preservadas em pingentes, em múmias e na
iconografia e são fundamentais para entendermos como as tramas capilares
femininas foram importantes meios de demonstração de condição social e
também de práticas mágico-religiosas. Os cabelos femininos bem compridos
eram deixados soltos pelas mulheres solteiras, sem necessidade
de ocultá-los sob lenços ou toucas, acessórios que eram evidência de matrimônio.
As mulheres trançavam seus cabelos e depois faziam um nó triplo, o
\emph{valknut}, ou nó dos mortos e envolviam toda a cabeça com uma espécie de
touca.

As joias eram muito usadas por todos e os homens usavam uma grande
quantidade de pulseiras de prata, que podiam ser úteis nas transações
comerciais. Por serem objetos fáceis de transportar, essas joias serviam
como valiosas moedas de câmbio. As mulheres usavam colares de contas
caras e raras, pendurados nos broches que prendiam o avental
sobre a túnica. Os broches feitos em prata trabalhada serviam de suporte
para o colar, que muitas vezes tinha como maior pingente as chaves dos
baús e arcas, demonstrando o poder dessa mulher naquela família. Também
usavam brincos de prata com contas e pingentes.

\SIG{Luciana de Campos}

Ver também Cultura material; Joias e ourivesaria; Sexo e sexualidade;
Sociedade; Vestuário.

\begin{itemize}
\item
  \versal{BRØNDSTED}, Johannes. Aparência. In: \emph{Os vikings}. São Paulo:
  Hemus, s.d., pp. 228-229.
\item
  \versal{CAMPOS}, Luciana de. Como elaborar tranças nórdicas da Era Viking.
  \emph{Canal do \versal{NEVE}/Youtube}, 2017. Disponível em:
  \textless{}https://www.youtube.com/watch?v=M1IJ0SykjvM\textgreater{}
\item
  \versal{HALL}, Richard. Costume and appearance. In: \emph{Exploring the world
  of the vikings}. London: Thames and Hudson, 2007, pp. 36-39.
\item
  Museu nacional da Dinamarca, 2017. \emph{What did the Vikings look
  like?} \href{http://en.natmus.dk/}{\emph{http://en.natmus.dk}}
\end{itemize}

\section{\versal{ÁRABES E VIKINGS}}\label{uxe1rabes-e-vikings}

Quando tratamos de contato entre os vikings e os árabes, os
historiadores, arqueólogos, estudiosos e especialistas em mundo viking,
tendem a pensar automaticamente nos contatos ao leste, e
nomes de árabes que escreveram sobre os rus logo vêm às nossas mentes.
Entre esses árabes temos Ibn Fadlan e Ibn Rustah que nos apresentam
detalhes físicos dos homens que viram, detalham-nos ritos (como o funerário) e
fornecem-nos características geográficas do Leste Europeu e de seus povos.

Ahmad Ibn Rustah Isfahani, geógrafo persa do século~\versal{X}, escreveu sobre os
búlgaros, os rus e os eslavos em sua obra denominada \emph{Livro das
gemas preciosas}, trabalho no qual o árabe analisa o comércio de
peles que ocorria no Leste Europeu, fornecendo informações dos povos do
Báltico e da Eurásia. O comércio de peles praticado no Leste Europeu
pode ser evidenciado também nas fontes arqueológicas, momento de grande
fluxo de dirhams do leste para o oeste escandinavo, moedas que seriam
utilizadas por todo o Báltico como possível método de pagamento em
momentos de troca e que foram descobertas em locais de comércio e
manufatura como Kaupang, Birka e Gotland.

O que devemos a notar é que os árabes supracitados,
fonte de estudos sobre os rus, eram ambos denominados árabes do leste,
provenientes do Oriente Médio ou da Ásia ocidental, além de serem ambos
contemporâneos do Período Viking. Ibn Fadlan era um funcionário do
califa abássida al-Muqtadir, e Ibn Rustah provinha do distrito
de Rosta, localizado na atual cidade de Isfahan, no atual Irã.
Entretanto, os muçulmanos que nesse período ocupavam a atual região da
península ibérica passaram a conhecer também os ataques e saques dos
vikings, contato estabelecido depois das incursões de saque na costa de
al-Andalus.

Diferente dos árabes do leste, os da Península Ibérica passaram
a escrever sobre esses povos apenas em períodos posteriores, entre os
séculos \versal{XII} a \versal{XIV}, e não os chamavam apenas de rus, mas davam a esses
também a nomenclatura de mäjus, magos, devido aos seus costumes
pré-cristãos de cremação de seus mortos, interpretação que aproximava os
homens do norte com os zoroastras, homens que recebiam também a
nomenclatura de mäjus na Pérsia pré-islâmica.

Entre os árabes ibéricos temos Ibn Idhari, que em seu livro
\emph{Al-Bayan al-Mughrib}, escrito em 1312, contava a história do
Maghreb e da península Ibérica. O árabe relata as embarcações dos
vikings chamando-as de pássaros vermelho escuro e dizendo que esses
saqueadores haviam enchido os corações dos ibéricos com temor. Idhari
relata que esses homens haviam chegado na costa de Lisboa, de Cádis, de
Sidona e de Sevilha, atacando as cidades e aprisionando ou matando seus
habitantes, enchendo os copos dos ibéricos de amargura.

Outro árabe que descreveu os ataques vikings foi al-Zuhri, 
homem que viveu em Granada no século \versal{XII}.
Em seu \emph{Livro da Geografia}, al-Zuhri relata grandes embarcações que chegaram em al-Andalus:
segundo o geógrafo os povos vikings eram fortes, corajosos e os mais
tenazes na arte da navegação. O árabe nos diz que quando os vikings
chegavam em suas embarcações os povos da península ibérica fugiam para
regiões mais ao interior, em um estado de total amedrontamento. Segundo
al-Zuhri os mäjus se alçavam ao mar a cada seis ou sete anos, com uma
frota composta por ao menos oito mas podia chegar a até cem embarcações, 
com o objetivo de aprisionar todos os homens
que encontravam em seu caminho.

Pelos relatos de Ibn Fadlan, Ibn Rustah, Ibn Idhari e al-Zuhri podemos
dizer que tanto os rus quanto os mäjus não podem ser determinados
precisamente, pois a relação étnica desses homens se perdeu
nesses relatos, mas podem ser relacionados a atividades de comércio,
ataques e saques, conectando esses povos com o termo ``viking'', que pode ser
compreendido como ``pirata''.

Para além do contato desses árabes com esses piratas, podemos dizer que
os relatos demonstram que a preocupação não uma construção
étnica antropológica, mas um discurso voltado à 
compreensão desses homens em suas relações geográficas e
sociais. Dessa maneira, quando tratamos de fontes árabes com relação aos
vikings, temos de compreender que esses árabes não possuíam todos a
mesmas vivências, mas estavam espalhados por regiões geográficas
diferentes, refletindo a preocupação comercial que os árabes do leste
possuíam em relação aos vikings. Relatam também o temor dos cristãos da península
ibérica e de outras tantas partes da Europa em relação a esses
piratas, que influenciou também os árabes que escreveram sobre esses
ataques e pilhagens.

As fontes aqui apresentadas podem muito mais do que evidenciar a vida dos
próprios nórdicos. Devem ser compreendidas como um
exemplo da multiplicidade apresentada no mundo árabe, que por baixo de
nomenclaturas étnicas ou religiosas, como as de muçulmanos, possuíam
vivências e experiências diversas, a depender de suas relações
espaço-temporais.

\SIG{Munir Lutfe Ayoub}

Ver também Escandinávia; Ibn Fadlan; Rus; Viking.

\begin{itemize}
\item
  \versal{CHRISTYS}, Ann. The Vikings in the South through Arab eyes. In:
  \versal{GANTNER}, Clemens (ed.).~\emph{Visions of Community in the Post-Roman
  World: The West, Byzantium and the Islamic World, 300-1100}. London:
  Routledge, 2013, pp. 447-458.
\item
  \versal{MARTIN}, Janet.~\emph{Treasure of the Land of Darkness: the fur trade
  and its significance for medieval Russia}. Cambridge: Cambridge
  University Press, 2004.
\item
  \versal{MONTGOMERY}, James E. Ibn Fadlan and the Rusiyyah.~\emph{Journal of
  Arabic and Islamic Studies}, vol. 3, 2000, pp. 01-25.
\item
  \versal{PRICE}, Neil. Mythic Acts: Material Narratives of the Dead in Viking
  Age Scandinavia. In: \versal{RAUDVERE}, Catharina; \versal{SCHJODT}, Jens Peter (eds.).
  \emph{More Than} \emph{Mythology. Narratives, Ritual Practices and
  Regional Distribution in Pre-Christian Scandinavian Religions}. Lund:
  Nordic Academic Press, 2012, pp. 13-46.
\end{itemize}

\section{\versal{ARMAMENTO}}\label{armamento}

Na Escandinávia da Era Viking, todos os homens livres possuíam o direito
de portar e utilizar armas. Tinham por obrigação se colocar à disposição dos
senhores ou reis para as fileiras de combate da guerra. O modo de
combate viking geralmente era a pé, porém há indícios
arqueológicos de guerreiros escandinavos enterrados junto de seus cavalos,
o que indica que havia também combatentes montados, ainda que não fosse o habitual. Os
navios de guerra eram utilizados principalmente para o transporte dos
exércitos nas batalhas em terra e nas batalhas
marítimas, embora estas últimas ocorressem com menor frequência. Como itens de
proteção, conforme achados da arqueologia, os vikings costumavam utilizar elmos nos modelos cônicos,
escudos redondos, geralmente de madeira com ornamentos,
e suas armaduras eram comumente de cotas de malha.

Esses guerreiros impunham terror a seus adversários, tanto por 
seu vigor e disposição para a batalha quanto pela imponência do
armamento que utilizavam. Seu modo de guerra era baseado em coragem e
força física, seguido de boas armas que fossem duráveis, resistentes e
eficientes em combate. As principais armas utilizadas pelos escandinavos
eram a espada, a lança e os machados de guerra, além de 
facas de combate, arcos e flechas.

A espada era um item prestigioso para o combatente nórdico e largamente
encontrado em achados arqueológicos em sepulturas ou depositado em lagos
e rios. Eram forjadas com sofisticadas técnicas de fabricação para
garantir sua resistência e durabilidade. Costumavam ser objetos
decorados com metais preciosos de acordo com a riqueza de seu portador,
além de receberem nomes próprios como sinal de estima do guerreiro
ou por motivos místicos e/ou devocionais. Eram
consideradas as melhores espadas as produzidas pelos francos, que os
vikings costumavam importar ou pilhar dos mesmos ou de outros povos
contra os quais entravam em batalha. Sendo assim, era muito comum que tomassem para
si as armas de outros guerreiros mortos. Importavam a lâmina básica e
depois os próprios artesãos escandinavos acabavam de fabricar o restante
da arma, o que não os impedia também de produzirem a arma inteira e até
a exportarem. As técnicas de ferraria nórdicas foram se aperfeiçoando
com o passar do tempo.

A espada padrão era reta, empunhada com uma só mão, possuía duas lâminas
e media aproximadamente 90 cm. Houve também variações de tipos de
espadas ao longo do tempo, como a utilização de
modelos de espadas menores, que funcionavam como espadas curtas e facas
de combate. As espadas, além de serem utilizadas em campo de batalha,
também eram empregadas em duelos de honra, conforme o costume dos
nórdicos.

As lanças eram armas mais comuns portadas por guerreiros de menor
estamento social por serem mais fáceis de serem manufaturadas e
disponibilizadas, embora pudessem também ser elevadas a um padrão
aristocrático, podendo ser utilizadas tanto para ataques à distância
quanto para ataques corpo a corpo, variando de função de acordo com o
tamanho do armamento. A lança era considerada 
a arma preferida do deus Odin dentro da mitologia nórdica. Segundo a crença, as paredes de Valhalla,
o salão dos guerreiros mortos, eram feitas de lanças e acreditava-se que
o guerreiro que morresse portando sua lança teria a mesma adicionada ao
salão desse deus.

As lanças mais leves eram utilizadas para arremesso, enquanto que as
mais pesadas para combate corpo a corpo. As lanças leves de arremesso
geralmente eram feitas de forma que dificultasse a sua retirada do corpo
do inimigo atingido; já as pesadas, para combate corpo a corpo, ao
contrário, eram geralmente feitas de forma que fossem rapidamente
retiradas de volta do corpo do adversário, para que continuasse a ser
utilizada em batalha. Costumava-se iniciar uma batalha arremessando
lanças por sobre o exército inimigo, que poderiam ser defendidas pelos
escudos, mas ainda assim causar desordem e prejudicar a formação de
batalha dos guerreiros adversários. Como o número de
lanças arremessadas era grande, poderia causar um grande número de
baixas.

O machado de guerra foi uma das armas que tornaram os vikings famosos e
distintos de guerreiros de outros povos, pois era mais comum que os mesmos
utilizassem espadas, lanças e facas. Os machados eram considerados a
marca registrada dos nórdicos no campo de batalha, as pesadas armas de
corte que causavam grandes baixas aos inimigos. A popularidade dos
machados se devia também ao impacto psicológico que causavam no oponente,
podendo medir até um metro e meio. Possuíam uma lâmina de
até 56~cm e eram empunhados com as duas mãos. A
maior parte do peso se concentrava na cabeça do machado e a eficácia do
seu golpe, mais que a força física, dava-se pela própria gravidade, sendo
capaz de cortar o escudo, o elmo e a cota de malha do inimigo.
Assim, o machado de guerra se tornava uma arma brutal e eficaz.

O machado de guerra, por conta de seu tamanho, conferia também a
vantagem de um maior alcance em campo de batalha. Era usado
estrategicamente para cortar uma parede de escudos para além do alcance
de uma espada e empunhado por guerreiros
treinados em seu manuseio. Devido a seu tamanho e eficácia em
violentos ataques, aliado ao impacto psicológico, o uso do machado resultava
geralmente no recuo das formações das paredes de escudo inimigas. Assim
como as espadas, os machados de guerra também costumavam ser
``batizados'' por seus portadores com nomes próprios e possuíam também
versões ornamentadas dependendo da posição social do guerreiro que os
possuíssem. Além dos grandes machados empunhados no campo de batalha com
as duas mãos, há evidências da existência também de versões menores e
mais leves da arma, empunhadas com apenas uma das mãos e para arremesso.

As facas de combate e as espadas curtas eram utilizadas pelos nórdicos
tanto como armas quanto como ferramentas e possuíam uma utilidade próxima
às facas dos soldados de infantaria atuais, servindo
como ótimas armas para combate corpo a corpo. As facas da Era Viking
geralmente lembravam em tamanho as atuais facas de açougueiro. Além de
serem usadas para o combate corpo a corpo, as facas também possuíam a
possibilidade em combate de serem arremessadas. A arquearia também era
inseparável do combate viking: os arcos utilizados em combate poderiam
ser de vários tipos, tanto curtos quanto longos, e as pontas das flechas
também variavam em estilo, podendo ser carregadas por cada arqueiro por
volta de quarenta flechas. A arquearia era muito utilizada pelos
nórdicos em situações de cerco, para se disparar além de paliçadas de
fortificações. Em batalhas navais se disparava sobre inimigos nas proas
dos navios, e em linhas de batalhas campais também era familiar o uso
de arquearia montada entre os nórdicos tanto em combate como para caça.

\SIG{Fábio Baldez Silva}

Ver também Espada; Estratégia de combate; História da guerra; Viking.

\begin{itemize}
\item
  \versal{BRØNDSTED}, Johannes. \emph{Os Vikings: história de uma fascinante
  civilização}. São Paulo: Hemus, s.d.
\item
  \versal{GRAHAM}-\versal{CAMPBELL}, James. \emph{Os Viquingues: Origens da Cultura
  Escandinava}, vol. 1. Madrid: Del Prado, 1997.
\item
  \versal{GRIFFITH}, Paddy. \emph{The Viking art of war.} Newbury: Casemate,
  1995.
\item
  \versal{SPRAGUE}, Martina. \emph{Norse warfare: unconventional battle
  strategies of the ancient Vikings.} New York: Hippocrene Books, 2007.
\end{itemize}

\section{\versal{ARQUEARIA}}\label{arquearia}

Embora não seja a arma principal da cultura militar escandinava, marcada
pela predominância de lanças e machados, frequentemente em conjunto com
escudos, o arco e flecha é parte intrínseca do fazer militar nórdico.
Podemos observar a presença da arquearia entre esses povos através 
dos achados arqueológicos e da literatura produzida ao longo dos
séculos.

Estudos arqueológicos indicam que o tipo mais comum de arco utilizado
por esses povos é o arco simples, dos quais o \emph{longbow} (arco
longo) seja talvez a expressão mais conhecida. O arco simples
caracteriza-se por sua confecção a partir de uma peça única de madeira,
ao contrário do arco composto, que se vale de diversos materiais em sua
construção. O arco longo possui características específicas de
proporção e tipo de madeira utilizada. Era feito
comumente a partir de madeiras como o freixo, o olmo e o teixo, 
esta última a mais indicada devido às suas propriedades naturais de
elasticidade de seu alburno -- a parte mais externa da madeira -- e
compressão de seu cerne -- a parte mais central. Para ser considerado um
\emph{longbow} é necessária uma proporção de aproximadamente 1.1:1
(largura e profundidade em comparação com a altura), o que confere a esse
arco um formato particular: quando retesado, adquire a forma de um
``\versal{D}''. O arco composto, por sua vez, é classificado por frequentemente
utilizar chifres de animais, tendões, e outros
possíveis materiais colados a uma peça central de madeira, o que garantia
mais resistência e elasticidade ao arco.

Registros arqueológicos sobre o uso da arquearia podem ser
encontrados em diversas regiões da Escandinávia e no norte da Germânia,
sobretudo em charcos, datando de períodos tão antigos quanto o
Neolítico. Além de arcos, pode-se encontrar também pontas de
flecha, como as da Jutlândia, cujas datações variam entre o
Neolítico e a Idade do Cobre, o que, de acordo com a historiografia, 
aponta para um uso predominantemente militar, em
contraponto a um uso meramente para fins de caça. Destacamos aqui também
o charco de Nydam, na atual Dinamarca, onde mais de trinta arcos longos
foram encontrados, além de centenas de partes de flechas, todos com
datação entre os séculos \versal{II} e \versal{IV} a.C. Nos séculos da
expansão viking é possível encontrar também diversos túmulos
contendo de pontas de flecha até arcos em túmulos de guerreiros
escandinavos na Islândia, na ilha de Orkney e outras regiões visitadas
pelos povos do norte.

Para os arcos compostos, muito comuns entre os povos nomádicos das
estepes, olhemos para o entreposto comercial de Birka (região da atual
Suécia), repleta de itens relacionados à prática da arquearia comumente
vistos entre os povos orientais -- datando do século \versal{X}, aproximadamente.
Devido à natureza dos materiais utilizados (madeira, couro e tecido), a
arqueologia depende dos elementos metálicos que faziam parte dos
objetos. Dessa forma, alguns dos achados de Birka apontam para a
presença de aljavas fechadas, ``coldres'', nos quais os
arcos ficavam encordoados e prontos para combate, além de diversas pontas de
flechas e uma estrutura de osso que pertencia a um arco composto.

A literatura escandinava é repleta de menções à arquearia em suas sagas
e poemas, auxiliando-nos a compreender um pouco mais sobre a função militar
do arqueiro. Como exemplo pode-se citar a saga de Njall, em que
o protagonista, ao tentar defender o seu lar, exalta que enquanto
possuir um arco jamais será sobrepujado por seus inimigos. 
O principal emprego tático da arquearia dava-se não no combate
terrestre, mas no combate naval. O poema \emph{Sexstefja},
escrito por Þjóðólfr Arnórsson sobre os feitos do rei Haroldo Hardrada
e a Heimskringla de Snorri descrevem combates navais nos quais
ambos os lados se fustigavam com flechas. Além desses exemplos,
temos ainda a \emph{Áns Saga Bogsveigis}, uma saga na qual o
protagonista é um arqueiro, cuja arma fora confeccionada magicamente
quando ainda em sua infância. A saga de Án encontra-se na
\emph{Hrafnistumannasögur} (saga dos povos de Hrafnista), datada do
século \versal{XIV}, ainda que os acontecimentos retratados sejam atribuídos ao
século \versal{VIII}.

Destarte, os arcos possuíam seu lugar no fazer militar dos povos
escandinavos, presentes na cultura nórdica desde muito antes da Idade do Cobre, 
conforme demonstram a arqueologia e a literatura. 
Ademais, embora nos enfrentamentos diretos entre guerreiros a lança e o
machado predominassem -- sobretudo nas paredes de escudos --, era nos
combates marítimos que os arqueiros obtinham seu destaque, além de seu
papel defensivo contra ataques às fortificações, conforme vimos no caso
de Birka.

\SIG{Hiram Alem}

Ver também Armamento; Espada; Estratégia de combate; História da guerra;
Viking.

\begin{itemize}
\item
  \versal{ALEM}, Hiram. Onde estão os arcos? A arquearia na série Vikings.
  \emph{Notícias Asgardianas}, n. 10, 2016, pp. 128-136.
\item
  \versal{CEOLIN}, Martina. \emph{Saga di Án l'Arciere - Proposta di traduzione
  con commento filológico}. Veneza: Università Ca' Foscari Venezia,
  2013.
\item
  \versal{CLARK}, John G. D. Neolithic bows from Somerset, England and the
  prehistory of archery in north-western Europe. In: \emph{Proceedings
  of the Prehistoric Society}. Cambridge, vol. 29, 1963.
\item
  \versal{HUGHES}, Shaun F. Saga of Án Bow-Bender. In: \versal{OHLGREN}, Thomas H. (ed.).
  \emph{Medieval outlaws: twelve tales in modern english translation}.
  Indiana: Parlor Press, 2005, pp. 290-337.
\item
  \versal{LUNDSTRÖM}, Fredrik; \versal{HEDENSTIERNA}-\versal{JONSON}, Charlotte; \versal{OLAUSSON}, Lena
  Holmquist; Eastern archery in Birka's Garrison. In: \versal{OLAUSSON}, Lena
  Holmquist; \versal{OLAUSSON}, Michael (eds.). \emph{The Martial Society:
  Aspects of Warriors, Fortifications and Social Change}. Stockholm
  University: Estocolmo, 2009, pp. 105-116.
\item
  \versal{MOOSBURGER}, Théo de Borba. \emph{Brennu-Njáls Saga: Projeto Tradutório
  e Tradução para o Português}, 442 f. Florianópolis: Tese de Doutorado,
  \versal{UFSC}, 2014.
\item
  \versal{POOLE}, Russell G. \emph{Viking poems on war and peace: A study in
  skaldic narrative}. Toronto: University of Toronto Press, 1991.
\item
  \versal{SARAW}, Torben. Male symbols or warrior identities? The ``archery
      burials'' of the Danish Bell Beaker Culture. \emph{Journal of
  Anthropological Archaeology}, n. 26, 2007, pp. 65-87.
\item
  \versal{STRICKLAND}, Matthew; \versal{HARDY}, Robert. \emph{The great warbow: from
  Hastings to the Mary Rose}. Gloucestershire: Sutton Publishing, 2005.
\end{itemize}

\section{\versal{ARQUEOLOGIA DA ERA VIKING}}\label{arqueologia-da-era-viking}

As fontes literárias escritas pelos
escandinavos são posteriores ao Período Viking -- a maioria delas data dos séculos \versal{XII}
e \versal{XIII}. São fruto de trabalhos e retrabalhos de conhecimentos passados pela
oralidade, o que provavelmente incorreu em inúmeras
modificações e as mais variadas supressões. A arqueologia,
é, portanto, fonte imprescindível para o estudo dessa era,
única fonte direta das atividades exercidas sob o
controle dos escandinavos durante tal momento histórico.

Contudo, a arqueologia também é
problemática quanto à produção e organização de suas fontes. Podemos, por exemplo, 
apontar escavações do século \versal{XIX} e início do \versal{XX} que decidiram
pelo descarte de artefatos e pela não produção de diários de escavação, o que levou
à perda permanente desses estudos e a um grande dano aos monumentos históricos e aos fatos
arqueológicos. Como exemplo dos artefatos descartados podemos apontar as
ossadas, fonte que hoje poderia nos fornecer uma série de informações,
como a determinação do sexo do indivíduo em um
determinado depósito funerário, além da análise do
isótopo de estrôncio desses ossos, que dá informações como o seu
local de nascimento e os seus locais de vivência.

É essencial para compreender partes dessas problemáticas, próprias da
arqueologia do Período Viking, levar em consideração que a maior parte
dos achados, nas regiões da atual Suécia e Noruega, são originários de
depósitos funerários, países onde o número de sepulturas é alto e onde
os artefatos depositados em cada uma também são abundantes. Os depósitos
funerários provenientes das atuais Suécia e Noruega são normalmente de
monte funerários, mas demarcações de padrões feitos em pedra também são
encontrados. Esses monumentos são pontos-chave para compreender a
abundância de achados de tais depósitos, uma vez que facilitam a delimitação da área de
ação, necessitando-se apenas de uma análise a
olho nu das paisagens das regiões.

A arqueologia dinamarquesa, por sua vez, não conta com tantos depósitos
funerários escavados, além de os depósitos não possuírem tantos artefatos
quanto nos dois países supramencionados. Isso acontece porque em
tal localidade grande parte dos depósitos funerários se encontra no
nível do solo, o que torna mais difícil a detecção e aquisição provenientes de práticas sem a
presença de métodos, como em momentos de cultivo das terras
da região.

A fácil detecção de tais depósitos funerários e sua presença física na
geografia das atuais Suécia e Noruega fez com que nos séculos~\versal{XIX} e
início do \versal{XX} abundassem campanhas para a escavação dos mesmos, além de
muitos desses terem sido destruídos durante o intensivo processo de
cultivo das terras. Antes de 1905 a Noruega não possuía nenhuma lei para
proteger os antigos monumentos e os montes foram destruídos em grandes
proporções. Os museus, no entanto, possuíam uma política de coleta e,
dessa forma, obtiveram grande número dos achados de suas coleções. Os
modos de produção dessas fontes variaram grandiosamente entre um achado
e outro, mas são normalmente responsáveis por uma pobre determinação de
contexto arqueológico que possibilitasse a compreensão dos artefatos, a
única informação conseguida por muitas das vezes era a de que o artefato
havia sido achado durante processos de trabalho das terras de uma
determinada localidade.

Atualmente não é possível apresentar os números exatos dos depósitos
funerários já escavados nem delimitar o número de
depósitos funerários ainda não escavados que pertençam ao período
viking, uma vez que os padrões de pedra e os montículos que demarcam
esses depósitos não são próprios apenas desse período, mas ocorrem desde
a Idade do Bronze escandinava.

Contudo, apesar de todos os problemas enfrentados pela arqueologia
funerária, muitos métodos atuais têm permitido uma melhor reconstrução
dos artefatos já escavados. Entre esses artefatos, podemos citar
embarcações como as de Oseberg, Gokstad e Tune. As embarcações que hoje
possuem papel chave no Museu das Embarcações de Oslo foram datadas por
estudos dendrocronológicos, método científico de estabelecer a idade de
uma árvore baseado nos padrões dos anéis internos de composição do
tronco, que levou em consideração estudos botânicos feitos nas árvores
do sul da Escandinávia. Os padrões de formações dos anéis, conhecidos
pelos estudos botânicos, foram comparados à formação dos anéis das
madeiras que fazem parte dos achados dessas embarcações. Os arqueólogos
conseguiram dessa maneira aproximar as datações de Oseberg para 834, de
Gokstad para 887 e de Tune para 900.

Outra problemática própria da arqueologia do Período Viking enfrenta as
dificuldades de delimitação dos assentamentos, uma vez que muitas
das edificações do período são de difícil detecção, pois não deixaram
marcas muito claras no solo, tornando as marcas de assentamentos
anteriores facilmente confundidas com as do Período Viking. Algumas
edificações, como por exemplo a de Uppakra, haviam sido construídas
durante a Idade do Ferro Germânica, conservadas e reconstruídas
sob o mesmo padrão arquitetônico até o século~\versal{X}, fato que dificulta
ainda mais a separação de períodos durante uma escavação.

Equipamentos como o \emph{scanner} de
\emph{laser} aéreo, o radar de penetração no solo e o detector de metais vêm
sendo utilizados nas leituras e mapeamentos das regiões e de seus
artefatos, com o objetivo de solucionar os problemas já supracitados. O
primeiro equipamento permite uma leitura óptica de detecção remota que
mede as propriedades da luz refletida para mapear distâncias geográficas
e extensões de determinadas composições, podendo medir até mesmo
construções derivadas da interação do ser humano com o espaço, em forma
de monumentos, por exemplo. O segundo, por sua vez, é uma técnica de
aquisição de informações espaciais de distribuição e localização de
artefatos ou estruturas sob o solo, levando a cabo até mesmo leituras
geofísicas que permitem a detecção de elementos de composição que
alteraram as camadas do solo durante os diferentes períodos históricos.
O radar de penetração no solo funciona por um método de alta frequência
que gera imagens do subsolo se utilizando de uma antena eletromagnética,
que por sua vez emite um sinal a uma frequência fixa, considerada ideal para
penetração de solos, rochas, sedimentos, gelo ou outras tantas camadas
naturais ou artificiais. Os métodos supracitados levaram a um aumento no
número de escavações de assentamentos nas atuais Suécia e Dinamarca, bem como 
nos estudos que fornecem informações sobre os padrões das edificações, das
construções e organizações dos assentamentos. Contudo, a metodologia contava ainda com um
problema advindo da reduzida quantidade de artefatos descobertos,
o que foi resolvido com a utilização de sistemas detectores de metais que
funcionam por campos eletromagnéticos e podem ser ajustados a
frequências diversas, a depender da quantidade de metal de detecção
desejada ou até mesmo ao tipo de metal que se pretende detectar.

Os novos assentamentos escavados desde os anos 1980 fazem uso
desses novos sistemas de detecção e método de escavação, e estão atualmente
gerando grandes descobertas e debates, sobretudo em relação ao que
seriam denominadas regiões centrais da Escandinávia, locais que
integravam as residências dos chefes locais, a produção manufatureira, o
comércio e os ritos. Artefatos como moedas, ferramentas de manufatura e
depósitos de fundação das edificações, ocorridos nos postes de
sustentação das mesmas ou nas valas para a colocação de paredes, 
passaram a serem detectados e escavados, revelando dois dos processos
característicos dessa era: a criação das cidades e a criação de centros
aristocráticos. Muitos estudos ainda estão sendo desenvolvidos e a
revolução no sistema de produção presenciado nessas localidades
estudadas passa agora a revelar novos mecanismos de produção em série,
como por exemplo fôrmas para moldar metais derretidos e produzir
objetos como broches e pingentes.

\SIG{Munir Lutfe Ayoub}

Ver também Cultura material; Era Viking; Escandinávia; Viking.

\begin{itemize}
\item
  \versal{BONDE}, Niels; \versal{CHRISTENSEN}, Arne Emil. Dendrochronological dating of
  the Viking Age ship burials at Oseberg, Gokstad and Tune,
  Norway.~\emph{Antiquity}, vol. 67, n. 256, 1993, pp. 575-583.
\item
  \versal{LARSSON}, Lars. The iron age ritual building at Uppåkra, southern
  Sweden.~\emph{Antiquity}, vol. 81, n. 311, 2007, pp. 11-25.
\item
  \versal{PEDERSEN}, Unn.~\emph{I smeltedigelen: finsmedene i vikingtidsbyen
  Kaupang}. Tese de Doutorado. Institutt for arkeologi, konservering og
  historie, Det humanistiske fakultet, Universitetet i Oslo, 2010.
\item
  \versal{PRICE}, Thomas; \versal{FREI}, Karen Margarita; \versal{DOBAT}, Andres Siegfried;
  \versal{LYNNERUP}, Niels; \versal{BENNIKE}, Pia. Who was in Harold Bluetooth's army?
  Strontium isotope investigation of the cemetery at the Viking Age
  fortress at Trelleborg, Denmark.~\emph{Antiquity}, vol. 85, n. 328,
  2011, pp. 476-489.
\end{itemize}

\section{\versal{ARTE}}\label{arte}

Os nórdicos da Era Viking tiveram a arte como um traço bastante marcante
em sua sociedade, que abrangeu um amplo leque de materiais e esteve
presente em todos os espaços da vida cotidiana. Seus diversos estilos,
motivos e materiais estavam diretamente ligados com a tendência e o
contexto histórico de sua produção. Através dos vestígios arqueológicos
de utensílios, ferramentas, joias, armas, roupas, construções, ou seja,
pela cultura material, podemos perceber que esses povos apreciavam
bastante a arte estética do adorno e abusavam dela.

Ainda que uma grande quantidade de objetos artísticos tenha sobrevivido
até nossos dias, eles são apenas uma pequena amostra da diversidade 
daquela época, tendo em vista que a grande maioria
das joias que sobraram foram encontradas junto a tesouros escondidos ou
a sepulturas, podemos imaginar as diversas produções que foram levadas
para outros donos ou derretidas e destruídas.

Mesmo recebendo influências externas, é notória a expressão de um gosto
propriamente escandinavo nas execuções artísticas. Gosto este que
remonta ao século~\versal{V}, período muitos anos antes da Era Viking, de quando
são datadas as primeiras expressões artísticas propriamente
escandinavas. Os estilos aos quais essas produções foram posteriormente
divididas ocupam uma posição temporal de difícil precisão, pois quando
uma nova moda surgia, a anterior não era esquecida. Portanto, ainda
que cada estilo possua características sobressalentes, ele não está
totalmente desvinculado ao estilo anterior ou posterior, logo a
cronologia é relativamente flexível. Entretanto, costuma-se dividir
cronologicamente as produções em estilos distintos.

Para as expressões de arte pré-viking e que serviram de base para as
produções realizadas na Era Viking, utilizamos a divisão fornecida por
Bernhard Salim, que cataloga em três estilos distintos as
produções artísticas escandinavas dos séculos anteriores à Era Viking
propriamente dita. O Estilo~\versal{I}, desenvolvido entre os séculos~\versal{V}~e~\versal{VI},
corresponde as obras que têm sua confecção inspirada nos modelos
romanos, apresentando uma arte naturalista de surpreendente qualidade na
execução dos motivos, utilizando uma técnica derivada do
\emph{chip-carving}. Um bom exemplo desse tipo são os medalhões chamados
bracteates. Por terem sido inspiradas em modelos de moedas romanas,
pode-se perceber a semelhança nos elementos presentes, entretanto, são
animais, símbolos e deuses do norte que estão ali representados.

No Estilo~\versal{II}, surgido nos fins do século~\versal{VI}, encontramos formas
abstratas de grande originalidade e extrema tortuosidade, componentes
inovadores da arte escandinava que serão mantidos pelos artistas durante
os próximos séculos. Essa manutenção do gosto atravessou o Período de
Migrações, o Período Vendel e toda a Era Viking, tornando a
Escandinávia, em certa medida, um expoente cultural, pois, ainda durante
o Estilo~\versal{II}, províncias latinas ao sul da Europa reproduziram elementos
artísticos tipicamente escandinavos.

Já o Estilo~\versal{III}, produzido no princípio da Era Viking, diferente dos
seus antecessores, não pode ser considerado pré-viking e foi nomeado
Estilo Broa, sendo contemporâneo ao Estilo Oseberg, e que introduziu a
\emph{gripping beast}, um elemento decorativo constituído da reprodução
de um animal em que utiliza suas patas para segurar alguma coisa
próxima, podendo ser as bordas do objeto, o próprio corpo, outras feras
ou qualquer outro elemento decorativo. Uma característica que o Estilo
de Oseberg vai difundir no mundo escandinavo

O Estilo Oseberg é o primeiro a surgir no Período Viking. Iniciado no
fim do século~\versal{VIII}, ele recebe esse nome devido ao local onde foi
encontrado um navio-sepulcro decorado com o conjunto de características
que lhe são pertinentes, na atual Noruega. Nesta sepultura, junto dos
corpos humanos de duas mulheres e das oferendas animais, o navio fora
carregado de bens ricamente ornamentados de diversos materiais, como
tecidos, metais e madeira, incluindo cama, baú, trenó, carroça, entre
outros.

A peça que define o Estilo Oseberg é a quilha da embarcação. Nela pode
ser identificado o elemento \emph{gripping beast} com bastante clareza,
onde os animais entalhados aparecem com seus corpos em padrões
retorcidos e entrelaçados e suas patas segurando as bordas, tanto na
popa como na proa. Esse elemento é o marco definidor da separação entre
este estilo e o Vendel, seu antecessor, sendo a diferença não apenas a
mudança do ornamento da fita em forma de serpentina da Era Vendel para a
musculosidade nodosa de Oseberg, como também a transformação da sua
superfície em um entrelaçado de alto-relevo e grande efeito.

Além do trabalho da quilha, outras peças chamas a atenção. A carroça
entalhada, dentre os seus painéis, ostenta uma cena em que, seguindo o
motivo do estilo, uma mulher segura a mão armada de um homem que, por
sua vez, agarra as rédeas de um cavalo montado, impedindo que o homem
mate o cavaleiro. Os postes com cabeça de fera apresentam uma decoração
que recobre toda a cabeça do animal. Composta de diversas feras
retorcidas, essa decoração demonstra o desenvolvimento do Estilo
Oseberg, abrindo as portas para o próximo estilo da arte viking.

Surgido em meados do século~\versal{IX} e preso dentro de figuras geométricas, o
Estilo Borre mantém o elemento do \emph{gripping beast} de corpo
retorcido dos estilos anteriores, porém o adapta para preencher os
espaços de maneira mais eficaz. Nomeado a partir de um grupo de objetos
de bronze encontrados em um navio-sepulcro próximo ao vilarejo Borre, na
Noruega, apresenta uma produção mais voltada para o trabalho em metais,
como pingentes e arreios, onde desenvolve a fera retorcida para um
modelo de laços ainda mais apertados e composições ainda mais próximas,
o que resultou na ausência de plano de fundo.

O pingente encontrado em Hedeby, um centro mercantil localizado ao sul
da Dinamarca, representa bem os elementos do Estilo Borre. O animal
representado, aparentemente um gato, aparece abstratamente retorcido,
com suas patas segurando a borda circular do pingente e a ele mesmo. A
borda deste pingente também nos mostra o anel encadeado, outro aspecto
desenvolvido nesse estilo. Este traço do Estilo Borre se caracteriza
pelo encadeamento de adornos que, comumente seccionados por cabeças de
animais, formam figuras geométricas, no caso do pingente citado, o
círculo no qual o gato está inserido.

As técnicas metalúrgicas desenvolvidas durante o período foram anexadas
as produções durante o desenvolvimento do Estilo Borre. Filigranas,
verdadeiras ou imitações, começam a aparecer como detalhes nos corpos
dos animais e nas bordas de suas molduras. Sob a forma de pequenos
pontos em alto-relevo, essas adições valorizavam ainda mais o trabalho
do artesão e aumentavam sua fama e prestígio, sendo essencial para que
se tornasse uma moda entre a elite.

O estilo artístico Jelling, originário do fim do século~\versal{XI} e que durou
por todo o século seguinte, apresenta um grande contraste com seu
antecessor. Os retorcidos adornos de Borre foram substituídos por um
arranjo mais aberto e fluido no corpo do animal. Embora os animais
apareçam retorcidos e alongados, não estão mais com as patas segurando
as bordas e seu corpo passa a ser representado com divisões
longitudinais em formato de faixa, tornando a sua postura mais fácil de
ser identificada.

O cálice que inaugura o estilo é conhecido como Cálice de Thyra. Foi
encontrado na região de Jelling, na Dinamarca, e ostenta duas feras
quadrúpedes enroscadas em sentidos opostos, além de uma outra decoração de
difícil identificação. É uma peça de pequenas proporções, mas a taça
feita em prata permite perceber as mudanças e inovações do novo
estilo. No cálice, os corpos dos animais, já divididos em faixas, carregam
minúsculas linhas perpendiculares, sendo um desenvolvimento, a
partir da filigrana, para este novo modelo.

As mudanças implementadas pelo Estilo Jelling representam uma grande
inovação nas representações artísticas. Contudo, alguns aspectos foram
adaptados e mantidos: a fera não mais utiliza patas para se
fixar aos elementos adjacentes, porém é comum encontrá-la mordendo ou
se enroscando em algo. Essas mudanças se popularizaram
principalmente durante o reinado de Haroldo~\versal{I}, o Dente Azul, na Dinamarca,
cujas mudanças políticas levaram a uma
transformação também na feitura dos monumentos pétreos, que,
impulsionados pelo monumento erguido pelo próprio Haroldo, passaram a
incorporar desenhos, elemento anteriormente comum apenas 
aos monumentos produzidos na ilha sueca de Gotland.
Dessa forma, os estilos artísticos se tornaram cada vez mais comuns em
pedras.

O Estilo Mammen, surgido em meados do século~\versal{X}, mostra uma ascensão dos
motivos vegetais e valoriza uma maior quantidade
de ramos. Esse estilo recebeu seu nome devido a uma lâmina de machado
ornamentada com fios e pontos de prata encontrada na vila de Mammen,
na Dinamarca. O principal tema era a dualidade presente entre
cristianismo e paganismo de um povo recém-cristianizado que não
abandonou os antigos costumes.

O machado de Mammen ostenta em sua lâmina uma ave com grande cauda que
se divide em ramos, assim como suas asas, que se alongam e se entrelaçam
por todo o corpo do animal. A ave, possivelmente um galo, foi
confeccionada com um grande olho redondo, uma das características
marcantes desse estilo. Também percebemos outra adição marcante do
Estilo Mammen, que são as espirais presentes nos locais onde os membros
se conectam ao corpo da fera. Na obra em questão, as formas espiraladas
podem ser encontradas claramente na base das asas da ave.

O corpo desse animal contém pequenas secções nas áreas onde as asas se
conectam ao pescoço, cauda e patas e, assim como no estilo anterior,
possui linhas longitudinais, que por sua vez são mais discretas. A parte
interna do corpo apresenta uma diferença estética do 
estilo antecessor, sendo preenchida por diversos pontos que substituem
as linhas transversais e as imitações de filigrana.

A grande valorização dos motivos vegetais desenvolvida durante esse
estilo prepara a arte viking para a mudança estética após a transição
do século~\versal{X} para o século~\versal{XI}. Porém, ainda no século~\versal{X}, um terceiro
estilo surge em suas últimas décadas. Nomeado devido à Pedra de Vang
localizada no distrito de Ringerike, na Noruega, o Estilo Ringerike coloca
os motivos vegetais no mesmo patamar de importância dos motivos animais,
sendo possível encontrá-los como a figura central da obra em alguns
casos. Esse monumento de Vang possui uma ornamentação com grande
quantidade de ramos que brotam de espirais na base, dirigindo-se ao
topo, e entrelaçam-se simetricamente até trespassarem uma figura em
formato de cruz floreada.

Acima de tudo isso está gravada uma fera corpulenta, possivelmente um
lobo, cujo corpo foi desenhado com duplo contorno, grandes olhos redondos
e espirais nas articulações, aspectos similares ao Mammen. Entretanto, o
novo detalhe dos pequenos cachos começa a surgir nessa fase artística e
se torna um dos elementos definidores do estilo, percebido
na cauda, dorso e focinho da fera de Vang.

As inovações dos ramos presentes na Pedra de Vang são a preocupação maior com a
simetria e a ordem, além do retorno da ``grande fera'', cujo corpo passa
por um aprimoramento ao longo do Estilo Ringerike: o
olho do animal começa a apresentar o formato de gota apontando
para o focinho em vez da forma arredondada que o precedeu. A fera
torna a constituir figura de bastante importância nas representações e
alcança grandes proporções, mas não ofusca as temáticas vegetais, que
passam a ser utilizadas mais livremente no preenchimento dos
espaços.

Essa última inovação anuncia uma mudança no gosto escandinavo e
equilibra os elementos decorativos, abrindo as portas para o Estilo
Urnes, o último estilo da Arte Viking. Surgido nas primeiras décadas do
século~\versal{XI}, seu nome advém dos detalhes esculpidos na igreja de Urnes,
Noruega, que possui a simplicidade, estreitamento e estilização dos
corpos animais, juntamente com a popularização de serpentes e dragões,
como suas principais características e a luta entre a grande fera e
a serpente-dragão o seu principal tema.

O portal de madeira esculpido da igreja de Urnes exibe um grande
confronto, no qual há várias serpentes, dragões e uma fera quadrúpede.
Os animais apresentam corpos extremamente esguios e modelos lisos,
possuindo apenas as espirais nas articulações como adorno no seu corpo.
Além dos animais, o painel também possui motivos vegetais marcados pela constituição delgada e
simples de seus ramos, adornados apenas com pequenas folhas.

Ainda que o portal da igreja de Urnes tenha cedido seu nome ao estilo ele
é, na realidade, uma das suas últimas expressões, haja vista que é
datado do século \versal{XII}, havendo um intervalo de mais de meio século entre
ela e as primeiras produções, categoria na qual se inserem as Pedras
Rúnicas da Inglaterra. Durante o século \versal{XI}, período do Danegeld e o
reinado de Canuto, o Grande, ocorreu uma aproximação entre a cultura das
Ilhas Britânicas e da Escandinávia através da migração e interação entre
os povos, fomentando uma partilha cultural e de vivências. Na
Escandinávia, por exemplo, diversas pedras rúnicas foram erguidas com
textos que se referem aos acontecimentos ocorridos durante as viagens em
terras britânicas, sendo, portanto, nomeadas de Pedras Rúnicas da
Inglaterra.

A Pedra Rúnica de Ölsta, integrante desse grupo, exibe em maior
destaque uma grande serpente-dragão que morde a si mesma, enquanto
outra fera, localizada ao lado, reproduz o mesmo ato. Junto dos grandes
animais foram gravadas pequenas serpentes que compõem
emaranhados, unindo todos os seres. Suas formas complexas nos enlaces,
mas desprovidas de muitos detalhes, anunciavam, já no início, a marcante
valorização destes elementos pelo Estilo Urnes.

\SIG{Ricardo Wagner Menezes de Oliveira}

Ver também Cotidiano; Cultura material; Religião; Simbolismo animal;
Sociedade.

\begin{itemize}
\item
  \versal{LANGER}, Johnni. As Estelas de Gotland e as Fontes Iconográficas da
  Mitologia Viking: os Sistemas de Reinterpretações Oral-Imagéticos.
  \emph{Brathair}, vol. 6, n. 1, 2006, pp. 10-41.
\item
  \versal{OLIVEIRA}, Ricardo Wagner Menezes de. Esculpindo símbolos e seres: A
  arte viking em pedras rúnicas. \emph{Notícias Asgardianas,} n. 7,
  2014, pp. 43-49.
\item
  \versal{OLIVEIRA}, Ricardo Wagner Menezes de. As religiosidades vikings em
  monumentos de pedra. \emph{Notícias Asgardianas,} n. 8, 2014, pp.
  43-52.
\item
  \versal{OLIVEIRA}, Ricardo Wagner Menezes de. Arte. In: \versal{LANGER}, Johnni \&
  \versal{AYOUB}, Munir Lutfe (orgs.). \emph{Desvendando os vikings}: estudos de
  cultura nórdica medieval. João Pessoa: Idéia, 2016, pp. 84-97.
\item
  \versal{SAWYER}, Birgit. \emph{The Viking-age rune-stones}: custom and
  commemoration in early medieval Scandinavia. New York: Oxford
  University Press Inc., 2000.
\item
  \versal{WILSON}, David M. \& \versal{KLINDT}-\versal{JENSEN}, Ole. \emph{Viking art}. New York:
  Cornell University Press, 1966.
\end{itemize}

\section{\versal{ASTRONOMIA}}\label{astronomia}

O conhecimento astronômico dos nórdicos durante a Era Viking é difícil
de ser reconstituído, mas recentemente várias abordagens e pesquisas vêm
obtendo um panorama mais preciso da área, especialmente no que diz respeito aos usos
para orientação náutica e suas relações com a mitologia e crenças
folclóricas.

As evidências materiais apontam um sofisticado conhecimento náutico dos
nórdicos durante a Era Viking, seja por suas descobertas geográficas
no Atlântico Norte durante o medievo, seja pelos vestígios arqueológicos
de instrumentos de navegação como bússolas solares (para indicar a
latitude e certos horários). Nesse sentido, os pesquisadores concluem
que esses navegantes não poderiam confundir um planeta de brilho
estável e que percorre a trajetória da eclíptica com movimentos
retrógrados (no caso, Vênus, o objeto mais brilhante do céu depois do
Sol e Lua) com uma estrela, um objeto celeste cintilante e fixo. Como a
maioria dos antigos povos navegadores do hemisfério norte, os nórdicos
devem ter utilizado a estrela Polaris (alfa da Ursa Maior) como
referencial indicador do Norte e principal meio de navegação astronômica
durante a noite dos tempos pré-cristãos. Localizam-se as constelações da Ursa
Maior e Menor para encontrar a Polaris, distante
1 grau do ponto norte celeste. Traça-se uma linha imaginária entre este
ponto até o horizonte logo abaixo e obtém-se aí o norte geográfico.

Não existe nenhuma fonte escandinava que aponte o conhecimento direto da
Polaris, mas ela é citada no poema rúnico anglo-saxão: ``\emph{Tir} é
uma estrela guia que mantém a promessa com os príncipes; está sempre em
seu curso sobre as brumas da noite e nunca cai''. Aqui evidentemente
temos a noção de Polaris como estrela do norte (indicadora de orientação
geográfica). No final da Alta Idade Média era recorrente denominar
Polaris como a estrela dos navegadores que transitavam pela Europa Setentrional -- como
descreve um estudo sobre calendário e astronomia da Inglaterra do
século~\versal{X} (\emph{De temporibus anni}, Alfirico de Eynsham) --, devido
a sua utilização para localizar o norte geográfico.

Em um manuscrito islandês (\versal{GKS} 1812 4to, \emph{De} \emph{ordine ac
positone stellarum in signis}), na sua seção datada de 1192 d.C., existe
a menção a cinco constelações que seriam conhecidas no mundo escandinavo
antes da cristianização, utilizando nomes nativos: \emph{Kvennavagn}, A
carroça da Mulher ou Senhora (identificada com a moderna constelação da Ursa
Menor); \emph{Karlvagn,} A carroça do Homem ou Senhor (Ursa Maior);
\emph{Fiskikarlar}, Os pescadores (o cinturão de Órion); \emph{Ulf's
Keptr}, a Boca do lobo (o aglomerado das Híades na constelação de
Touro); \emph{Asar Bardagi}, Campo de batalha dos deuses (constelação de
Cocheiro). A região abrangida por estas constelações consiste de um céu
particularmente vislumbrado na Escandinávia de outubro a fevereiro --
época importante para a religiosidade, especialmente no momento
culminante do Jól. Ou seja, nem todo o firmamento celeste foi
alvo de apropriações míticas pelos nórdicos.

Na Alemanha, a Ursa Maior foi relacionada ao deus Odin e seu veículo:
\emph{Wotanswagen} e \emph{Irmineswagen}. Na Estônia, essa constelação
era conhecida como \emph{Otava}, e segundo alguns pesquisadores foi
influenciada pela região escandinava (teria sido originada de
\emph{Óðins vagn}, carroça de Odin), do mesmo modo que o finlandês
\emph{Otawa}. O termo mais genérico usado folcloricamente na
Escandinávia não se atrela individualmente e objetivamente a uma
deidade, mas somente a designação da carroça de um homem, como em outras
regiões da Europa (Dinamarca, \emph{Karlsvogn}; Suécia, \emph{Karlwagn}
e \emph{Herrenwagen}).

Odin é conhecido na poesia escáldica como \emph{runni vagna} (condutor
de carroças); \emph{vinr vagna} (amigo das carroças); \emph{vári vagna/vagna ver}
(protetor/senhor das carroças); \emph{valdr vagnbrautar} 
(protetor da estrada das carroças); \emph{runni vagna}
(movedor da carroça/constelação); \emph{vagna Grimnir} (carroça de
Grimnir) \emph{reiðartýr} (deus da carroça). Para o mitólogo Thomas
DuBois, o termo \emph{karl} (constante na forma
\emph{karlavagnen/Karlvagn}) se refere a um homem de alta
condição social nas sociedades germânicas antigas, o fazendeiro livre,
membro do \emph{comittatus} dos líderes e reis e, portanto, tendo alto
significado militar. Desse modo, a associação dessa palavra à mais
reconhecível constelação do hemisfério norte confere a ela um estatuto
de marcialidade, explicando sua associação ao deus Odin. Concordamos com
esse referencial, ainda mais se observarmos que as duas narrativas da
criação de estrelas (``O dedo de Aurvandil'' e ``Os olhos de Tiazi'') estão
conectadas com o desmembramento ou morte de gigantes por parte de algum
deus (ou Odin ou Thor, dependendo da versão do mito).

Existem mais dúvidas do que certezas com relação ao conhecimento
astronômico nórdico na Era Viking. Muitas fontes precisam ser
exploradas, assim como algumas narrativas míticas precisam receber
melhores análises em relação a outros referenciais como a cosmologia, a
cosmogonia, a cultura material e religiosa, entre outros aspectos.
Também não conhecemos em detalhes as relações entre fenômenos puramente
atmosféricos (como parélios e auroras) com os mitos celestes
escandinavos. Os mitos relacionados aos planetas foram detalhados no
\emph{Dicionário de Mitologia Nórdica}. Também existem indícios de
crenças relacionadas a cometas, meteoros, lua e sol na Escandinávia
antes da cristianização.

\SIG{Johnni Langer}

Ver também Calendário e contagem do tempo; Bússola solar; Navegação
marítima; Pedra solar.

\begin{itemize}
\item
  \versal{DUBOIS}, Thomas. Underneath the self-same sky: comparative perspectives
  on sámi, finnish, and medieval Scandinavia astral lore. In:
  \versal{TANGHERLINI}, Timothy (ed.). \emph{Nordic Mythologies:
  interpretations, intersections, and institutions}. Berkeley: North
  Pinehurst Press, 2014, pp. 184-260.
\item
  \versal{ETHERIDGE}, Christian. A systematic re-evaluation of the sources of Old
  Norse Astronomy. \emph{Culture and Cosmos}, vol. 16, 2013, pp. 01-12.
\item
  \versal{KNIGHT}, Dorian. A Reinvestigation Into Astronomical Motifs in Eddic
  Poetry, with Particular Reference to Óðinn's Encounters with Two
  Giantesses: Billings Mær and Gunnlöð. \emph{Culture and cosmos}, vol. 17, n. 1,
  2013, pp. 31-62.
\item
  \versal{LANGER}, Johnni. Thor, estrelas e mitos: uma interpretação astronômica
  da narrativa de Aurvandil. \emph{Revista Brasileira de História das
  Religiões} (no prelo).
\item
  \versal{LANGER}, Johnni. Planetas e mitos nórdicos. In: \versal{LANGER}, Johnni (org.).
  \emph{Dicionário de mitologia nórdica}. São Paulo: Hedra, 2015, pp.
  174-177; 371-372.
\item
  \versal{LANGER}, Johnni. Constelações e mitos celestes na Era Viking: reflexões
  historiográficas e etnoastronômicas. \emph{Roda da Fortuna}, vol. 1, n. 4,
  2015, pp. 107-130.
\item
  \versal{LANGER}, Johnni. O zodíaco viking: reflexões sobre etnoastronomia e
  mitologia escandinava. \emph{História, imagem e narrativas}, vol. 16, 2013,
  pp. 01-32.
\item
  \versal{LANGER}, Johnni. O céu dos vikings: uma interpretação etnoastronômica
  da pedra rúnica de Eckelbo (Gs 19). \emph{Domínios da imagem}, vol. 6, n. 12,
  2013, pp. 97-112.
\item
  \versal{LANGER}, Johnni. Cometas, eclipses e Ragnarök: uma interpretação
  astronômica da escatologia nórdica pré-cristã. \emph{Revista Mundo
  Antigo}, vol. 4, 2013, pp. 67-91.
\end{itemize}

\section{\versal{AUD, A DE MENTE PROFUNDA}}\label{aud-a-de-mente-profunda}

A \emph{Laxdaela saga} apresenta em suas linhas uma das personagens
femininas mais interessantes e marcantes da literatura nórdica. Trata-se
de Aud, a de mente profunda (Auðr djúpúðga Ketilsdóttir, também
conhecida como Unn). Aud era filha de Ketil Flatnose, um nobre norueguês
que se refugiou com sua família na Escócia para não se submeter ao rei
Haroldo Cabelos Belos. Após a morte de seu marido e de seu pai na
Escócia, Aud audaciosamente reuniu sua família e os seus agregados e
partiu para a Islândia. A narrativa da viagem de Aud apresenta uma
alternativa às narrativas de fundação e colonização da Islândia, sempre
protagonizadas por personagens masculinos.

Aud vivia com a sua família em uma região do extremo norte da Escócia,
chamada Caithness, onde seu filho foi morto em um
ataque promovido por escoceses contrários à presença dinamarquesa
em suas terras. Como seu marido e
seu pai também haviam morrido, ela reconheceu não haver mais nenhuma
esperança de futuro e prosperidade nessas terras. Nesse momento, Aud
reúne todas as suas riquezas e, junto com o que sobrou de sua família,
parte para uma nova vida nas terras islandesas. O caso de Aud é, por
assim dizer, \emph{sui generis}; é difícil encontrar outro relato de
uma mulher escapando de tais hostilidades com toda a sua família, riquezas e muitos
seguidores em direção a uma terra
desconhecida. Em sua viagem, Aud foi acompanhada por muitos homens
notáveis e bem-nascidos, que mais uma vez atestam o prestígio e poder
dessa mulher. Um homem chamado Koll destaca-se entre os
agregados de Aud devido ao seu nascimento aristocrático. Antes de ir
para a Islândia, Aud dirigiu-se para as ilhas Orkneys e logo após para
as Ilhas Faroé, onde permaneceu por um curto espaço de tempo.

Nas Ilhas Feroé ela organizou um casamento para uma de suas netas. 
Logo após o casamento, Aud ordenou que os preparativos para a partida
para a Islândia fossem finalizados. Após uma boa viagem, chegaram a
Vikrarskeid, no sul da Islândia, onde passaram por um naufrágio, mas
felizmente não houve baixas e nenhum bem foi perdido.

Depois de desembarcar, Aud formou uma comitiva com vinte dos seus
homens e foi ao encontro de seu irmão Bjorn em sua casa em
Breidafjord. Quando Bjorn soube de sua chegada ele a recebeu
calorosamente e a convidou para ficar, pois sabia que sua irmã
pretendia estabelecer-se naquelas terras. Ali, Aud e os seus passaram o
inverno. Na primavera, ela atravessou todos os vales de Breidafjord e
apossou"-se de várias porções de terras. Ela estabeleceu, então, sua
fazenda, conhecida posteriormente como Hvamm. Nessa mesma
primavera, Aud arranjou o casamento para a filha de um agregado que se
chamava Thorgerd, filha de Thorstein, o Vermelho. Não poupou custos para
a realização do casamento e entregou a Thorgerd o todo o \emph{Laxdaela}, como
seu dote.

A velhice impactou a vida dessa mulher ativa e decidida: passou a deitar-se
cedo e não se levantava antes do meio-dia, mas ricamente vestida e
sempre disposta, atendia a todos aqueles que vinham dos mais longínquos
territórios pedir os seus sábios conselhos. Aud, a de mente profunda,
era respeitada por sua sabedoria e também pela sua intrepidez ao
comandar os seus em uma empreitada de colonização a uma terra distante e
desconhecida.

Durante uma festa de mais um casamento arranjado por Aud, ela recebeu os
convidados e a família com a honra que lhe era costumeira. Ao se
despedir dos comensais alegando cansaço, Aud relembrou sua
trajetória, suas aventuras, desventuras e tudo o que havia
conquistado. Depois disso, Aud levantou-se e disse que estava se
retirando para os seus aposentos, mas antes incentivou a todos a se
divertirem e anunciou que havia cerveja em quantidade o suficiente para
uma longa festa. Enquanto caminhava, as pessoas comentavam que, apesar da
idade, Aud era alta e forte e ainda era uma mulher esplêndida.

No dia seguinte, Olavo, neto de Aud, foi até os aposentos da avó, e a
encontrou sentada, apoiada nos travesseiros. Aud havia adentrado o
palácio de Freya.

Olavo anunciou aos convidados o ocorrido e todos admiravam como Aud,
mesmo diante da morte, havia mantido sua dignidade. O corpo de Aud foi
preparado e levado para um local preparado para esse fim. 
Junto de seu corpo foram depositados objetos simbolizando não só o poder
político e social dessa mulher, mas principalmente a sua sabedoria.

\SIG{Luciana de Campos}

Ver também Islândia da Era Viking; Mulheres; Sociedade.

\begin{itemize}
\item
  \versal{HAYWOOD}, John. Aud the Deep-Minded. In: \emph{Encyclopaedia of the
  Viking Age}. London: Thames and Hudson, 2000, p. 29.
\item
  \versal{HOLCOMB}, Kendall M. \emph{Pulling the Strings: The Influential Power
  of Women in Viking Age Iceland}. Western Oregon University, Studen
  Theses, 2015.
\item
  \versal{JESCH}, Judith. \emph{Women in the Viking Age}. London: Boydell \&
  Brewer Ltd, 1999.
\item \versal{JOCHENS}, Jenny. \emph{Women in Old Norse
  Society.} Ithaca\emph{:}
  \href{http://www.cornellpress.cornell.edu/publishers/?fa=publisher\&NameP=Cornell\%20University\%20Press}\textbf{Cornell
  University Press},
  1995.
\end{itemize}


\chapter{B \textarn{b} \textart{b}}

\section{\versal{BATALHA DE BRAVALLA}}

A batalha de Bravalla ou Bravellir foi um combate lendário que teria
ocorrido provavelmente na Suécia no século~\versal{VIII}, entre as forças dos
reis Haroldo Dente Azul da Dinamarca e Sigurd Hring da Suécia. De acordo
com o historiador dinamarquês do século~\versal{XIII} Saxo Grammaticus, a batalha
teria sido parte da chamada guerra sueca. Foi uma batalha de
grandes proporções, muitos combatentes e muitas baixas. Relatos 
apontam o possível local da batalha como a região sueca de Braviken.

Segundo os relatos de Saxo Grammaticus, a descrição de Haroldo Dente Azul é
a de um grande rei guerreiro, que teria sido escolhido desde jovem como
favorito por Odin. Após conquistar um grande reino na Dinamarca, Haroldo
entra em conflito com três reis suecos, pedindo aos deuses por 
conselhos para entrar em batalha, e o próprio Odin o teria revelado uma
estratégia de guerra, segundo a lenda. Haroldo foi, então, vitorioso na
batalha: dois dos reis suecos morreram em combate, e o terceiro,
Ingild, aliou-se a Haroldo. Ingild tinha um filho com a
irmã de Haroldo, chamado Hring.

Quando Ingild morre, Haroldo indica uma regência para governar a região
enquanto Hring não atinge a maioridade, estreitando ainda mais os laços
entre os dois reinos, em paz duradoura. Um
amigo de Haroldo, chamado Bruni, atua como mensageiro entre os dois reis,
mas morre acidentalmente; Odin assume, então, a sua forma e
começa a semear a desconfiança entre Haroldo e Hring. Assim quebra-se a
paz entre os dois reis e reiniciam-se as hostilidades, tendo os dois
reis durante anos se preparado para a grande batalha.

Haroldo estava já idoso, cego e com dificuldades de locomoção. Teve de ser 
conduzido para o campo de batalha na Suécia em uma carruagem conduzida
pelo próprio Odin personificado em Bruni. Em Bravalla, Haroldo enfrenta bravamente um
exército inimigo muito maior do que o seu, 
até acabarem suas forças e suas tropas serem dizimadas. Haroldo
pede a Bruni, ainda sem saber que este é Odin personificado, que o informe sobre a
formação do exército inimigo, e é por ele enganado. Quando
percebe a traição, Haroldo é morto pelo próprio Odin. Com o fim da batalha,
Hring realizou um honroso funeral a Haroldo por ser ele considerado um
grande rei guerreiro. Relata-se que muitos líderes noruegueses e
suecos participaram da batalha, a maioria ao lado de Hring. 
Saxo relata também a participação de mulheres como escudeiras na
batalha de Bravalla.

Segundo a historiografia, tanto Haroldo quanto Hring (se realmente tiverem
existido, pois os relatos sobre a batalha de Bravalla são permeados por
muitas lendas) não podem ser considerados como soberanos de reinos
unificados, e sim do que poderia se chamar de confederações de pequenos
reinos, das quais os mesmos seriam os reis mais proeminentes nas regiões
da Dinamarca e Suécia, além de possuírem aliados de outras regiões, como
a Noruega por exemplo. Não há consenso sobre o momento correto em que a
batalha teria ocorrido e se ela realmente
aconteceu, mas estima-se que teria sido no século~\versal{VIII}, 
\versal{VII}, ou até mesmo \versal{VI}; entre os pesquisadores 
parece haver um maior consenso de que teria ocorrido no século~\versal{VIII}.

\SIG{Fábio Baldez Silva}

Ver também Armamento; Arquearia; Espada; Estratégia de combate; Guerra e
técnicas de combate; Viking.

\begin{itemize}
\item \versal{GRAMMATICUS}, Saxo. Book Eight. \emph{In: The Danish History.} Tradução
ao inglês por Oliver Elton. Disponível em: \textless{}
\href{http://omacl.org/DanishHistory/}{\emph{http://omacl.org/DanishHistory/}}\textgreater{}
Acesso em: 11 de junho de 2017.

\item \versal{HAYWOOD}, John. Battle of Bravalla. In: \emph{Encyclopaedia of the
Viking} \emph{Age}. London: Thames and Hudson, 2000, pp. 35-36.

\item \versal{JONES}, Gwyn. \emph{A history of the vikings.} Oxford: Oxford University
Press, 1984.
\end{itemize}

\section{\versal{BATALHA DE BRUNANBURH}}

A batalha de Brunanburh ocorreu na Inglaterra no século~\versal{X}, mais
possivelmente no ano de 937 segundo a \emph{Crônica Anglo-Saxônica}. Nela
se opuseram as forças do rei Athelstan de Wessex e de seu irmão
Edmundo contra as forças coligadas de Olavo Guthfrithson de Dublin e
Constantino~\versal{II} da Escócia, cuja vitória foi de Athelstan, e os relatos
da batalha, contidos principalmente na \emph{Saga de Egil
Skallagrimson}, mostram a participação de mercenários vikings.

A batalha foi consequência das disputas que ocorriam na Inglaterra do
século~\versal{X} entre o reino de Wessex e os líderes que compunham o Danelaw no
norte do país, os escoceses e os governantes nórdicos ou de descendência
nórdica de Dublin. Durante os anos precedentes, os saxões passaram a
construir fortalezas chamadas de \emph{burhs} com o objetivo de se
protegerem de ataques inimigos. Chegando ao poder em 924, Athelstan
reforçou as fronteiras do ocidente e lançou um ataque contra o reino de
York em 927. Após construir alianças com governantes celtas, escoceses
e galeses, conseguiu governar um reino unido para além da atual fronteira
inglesa para o norte.

Apesar do sucesso de Athelstan em 927, no ano de 937 seu reino foi
ameaçado por uma grande aliança entre os vikings de Dublin e os celtas,
principalmente com intuito de levar a efeito uma campanha militar
para restaurar as conquistas dos escandinavos e derrotar os ingleses.
Assim, os exércitos oponentes se enfrentaram em Brunanburh, um local até
hoje não identificado, em uma violenta batalha que
escritores celtas posteriores passariam a chamar de ``grande guerra''.
Foi relatado que os ingleses venceram a batalha após horas de combate,
porém os domínios ingleses nos territórios do norte permaneceram em
instabilidade. Após a morte de Athelstan, em 939, os governantes de
Dublin retomaram sua posse, situação que perdurou até a
desagregação do domínio nórdico na região devido a disputas internas,
que culminaram com a morte de Érico Machado Snagrento, o último rei viking a governar York, 
na batalha de Stainmore em 954.

A \emph{Saga de Egil Skallagrimson} relata que o próprio Egil e seu
irmão Thorolf se encontravam entre mercenários vikings a serviço do rei
Athelstan após ficarem sabendo que o mesmo estava organizando um
exército para enfrentar a coligação dos governantes da Irlanda e Escócia, 
comandados por Olavo Guthfrithson, que ameaçava seus domínios. Relata-se 
que quando os dois exércitos adversários estavam frente a
frente para a batalha, viu-se que ambos eram tão grandes que era
impossível se dizer qual era o maior, e o exército de Olavo (composto em sua
maioria por escoceses, segundo a saga) aproximou-se da coluna
liderada por Thorolf para enfrentá-la, o que resultou na morte de
Thorolf em combate. Egil, então, junto com o exército do rei Athelstan, 
que incluía seus mercenários vikings, lutou bravamente e, segundo os
relatos da saga, impôs muito baixas ao exército inimigo, 
resultando numa retirada de suas tropas, que foram perseguidas
pelas forças de Athelstan, vitorioso no confronto.

O local da batalha ainda continua incerto, pois a historiografia não tem
indícios unânimes de onde teria realmente ocorrido. Tudo o que se têm
são os relatos de fontes primárias um tanto quanto imprecisos; sabe-se, porém,
que na batalha morreram 
reis e Earls a serviço de Olavo, além do filho de Constantino da Escócia
e ainda um grande número de vikings e escoceses. Na \emph{Saga de Egil} é
relatada a morte do rei Olavo, porém pesquisadores afirmam que o mesmo
teria sobrevivido à batalha e retornado à Dublin.

\SIG{Fábio Baldez Silva}

Ver também Armamento; Arquearia; Espada; Estratégia de combate; Guerra e
técnicas de combate; Viking.

\begin{itemize}
\item \versal{BRØNDSTED}, Johannes. \emph{Os Vikings: história de uma fascinante
civilização}. São Paulo: Hemus, s.d.

\emph{Egil's Saga}. Tradução de Bernard Scudder. In: \emph{The Sagas of
Icelanders}. London: Penguin Books, 2001.

\item \versal{GRAHAM}-\versal{CAMPBELL}, James. \emph{Os Viquingues: Origens da Cultura
Escandinava}, vol. 1. Madrid: Del Prado, 1997.

\item \versal{GRIFFITH}, Paddy. \emph{The Viking art of war}. Newbury: Casemate, 1995.

\item \versal{JONES}, Gwyn. \emph{A history of the vikings}. Oxford: Oxford University
Press, 1984.

\item \versal{ROESDAHL}, Else. \emph{The Vikings}. London: Penguin Books, 1998.
\end{itemize}

\section{\versal{BATALHA DE CLONTARF}}

Ocorrida em 1014, na Irlanda, a norte de Dublin, foi a batalha que pôs fim à
hegemonia viking sobre a ilha. O enfrentamento de celtas irlandeses e
vikings em Clontarf foi decisivo para a destruição do poder viking sobre
a Irlanda, embora a um alto custo, com a morte do alto rei dos
irlandeses, Brian Boru.

Nos anos anteriores a 1014, a Irlanda viu uma série de embates entre
seus reis e chefes guerreiros locais pelo poder no norte e no sul da
ilha, em disputas que remontavam ao estabelecimento de um entreposto
viking que viria a se tornar a cidade de Dublin. Outras
cidades, como Arklow, Cork, Limerick, Waterford e Wexford se tornariam
pontos comerciais que gerariam grande fluxo de riquezas, criando
condições para que quem as dominassem tivesse poder e homens a serviço.

A chegada dos vikings na Irlanda se dá em 795, quando registros apontam
os primeiros ataques esporádicos por via marítima, até
820, quando passam a se tornar frequentes, com incursões
pelos rios interiores locais. Os saques a monastérios logo
se expandiram e visavam também cidades, enfraquecendo os reinos locais.

Entre 830 e 840, os vikings começam a estabelecer pontos permanentes na
Irlanda, além de explorarem o tráfico de escravos na ilha, entrando
assim na complexa e fracionada luta pelo poder, com o estabelecimento de
alianças com os reinos locais, casamentos, e derrotas frente aos reis
irlandeses. Desses entrepostos, antes bases para manter as pilhagens e
ataques, floresceram cidades totalmente integradas no jogo do poder pela
Irlanda.

Entre 850 e 900, cresceu o poder de Dublin como reino viking na Irlanda
e as frotas vikings começaram a servir como mercenárias para os reis
irlandeses em suas guerras. Paradoxalmente, enquanto Dublin e as outras
cidades vikings na Irlanda cresciam de importância por se tornarem
pontos de apoio para ataques destes à Escócia e ao norte da Inglaterra,
o poder delas se enfraqueceu com disputas dinásticas.

A partir de 900, as cidades vikings são alvos de ataques, e uma trégua
entre os reis irlandeses leva à expulsão dos invasores em 902, com a
tomada de Dublin. Mas eles acabam por retornar e se estabelecer na cidade, 
bem como em outros pontos de domínio na ilha. O retorno dos vikings acaba
por colocá-los de vez dentro do processo político irlandês.

Entre 940 até 980 o que se observa é que, embora possuidora de
riquezas oriundas do comércio e do tráfico de escravos irlandeses, bem
como de pilhagens vindas da Escócia e Inglaterra, a presença viking na
Irlanda é cada vez mais contestada e enfrentada abertamente pelos reis
irlandeses, que estavam em uma grande guerra pela hegemonia local. Em
980, o reino viking de Dublin é derrotado pelas forças irlandesas na
Batalha de Tara e, embora não conquistado, é forçado a aceitar a
soberania de um rei irlandês.

O rei de Dublin, Sigtrygg Barba de Seda, decide se revoltar contra o
alto rei dos irlandeses, Brian Boru, e monta uma coalizão com alguns
outros reis vikings e irlandeses locais. Em Clontarf, ao norte de
Dublin, as duas forças se encontram e combatem pelo dia todo, segundo as
fontes, com pesadas baixas para ambos os lados. 

Brian Boru, a época com 88 anos, não tomou parte nos combates contra os vikings, que foram
liderados por familiares seus, como seu filho, tios e sobrinhos.
Embora vencedor, não testemunhou o fim da batalha, já que
foi morto em sua tenda enquanto rezava, segundo as fontes, por vikings
que fugiam da batalha.

A batalha ajudou a submeter de vez os vikings da Irlanda a autoridade
dos altos reis, mas não os expulsou -- muitos já se encontravam
perfeitamente adaptados à vida na ilha, e foram assimilados pela
população local em algumas gerações.

A Batalha de Clontarf acabou por ganhar um forte peso propagandístico
séculos depois, sendo utilizada como peça motivacional para lembrar aos
irlandeses sobre a resistência a um invasor estrangeiro, no caso, os
ingleses.

\SIG{Sandro Teixeira Moita}

Ver também Espada; Estratégia de combate; Guerra e técnicas de combate;
Guerra e simbolismos; História da Guerra.

\begin{itemize}
\item \versal{BRINK}, Stefan; \versal{PRICE}, Neil (eds.). \emph{The Viking World}. Abingdon:
Routledge, 2008.

\item \versal{CRAUGWELL}, Thomas J. \emph{How the Barbarian Invasions Shaped the Modern
World -- The Vikings, Vandals, Huns, Mongols, Goths, and Tartars who
Razed the Old World and Formed the New}. Beverly: Fair Winds Press,
2008.

\item \versal{HARRISON}, Mark; \versal{EMBLETON}, Gerry. \emph{Anglo-Saxon Thegn 449-1066 \versal{AD}}.
London: Osprey Publishing, 1993.

\item \versal{HARRISON}, Mark; \versal{EMBLETON}, Gerry. \emph{Viking Hersir 793-1066 \versal{AD}.}
London: Osprey Publishing, 1993.

\item \versal{HEATH}, Ian; \versal{MCBRIDE}, Angus. \emph{The Vikings}. London: Osprey
Publishing, 1985.

\item \versal{HOLMAN}, Katherine. \emph{The \versal{A} to \versal{Z} of the Vikings}. Plymouth: The
Scarecrow Press, 2009.

\item \versal{SAWYER}, Peter (ed.). \emph{The Oxford Illustrated History of the
Vikings}. Oxford: Oxford University Press, 1997.

\item \versal{WINROTH}, Anders. \emph{The Age of the Vikings}. Princeton: Princeton
University Press, 2014.
\end{itemize}

\section{\versal{BATALHA DE EDINGTON}}

Batalha ocorrida em 878 na qual o rei anglo-saxão Alfredo, o Grande tornou-se
um dos mais proeminentes reis anglo-saxões após
vencer os vikings de Guthrum e garantir a sobrevivência do Reino de
Wessex. A presença viking na Inglaterra data de 793, com o ataque ao mosteiro de
Lindisfarne, e daí se seguiram uma série de invasões bem-sucedidas que
logo deram lugar a tentativas de estabelecimento permanente por parte
dos vikings.

Em 865 desembarcou na Inglaterra o chamado Grande Exército Pagão,
composto por vikings e liderado por uma série de reis e chefes
guerreiros, empreendendo campanhas contra os reinos anglo-saxônicos, com
a maior força escandinava que já havia posto o pé na Inglaterra até
aquele momento.

Entre 865 e 875, o Grande Exército realizou uma série de ações que
acabaram por conquistar diversas áreas, destruindo reinos como o da
Ânglia do Leste e Mércia, mas apesar da dura resistência, não conseguiu
impedir a conquista viking. Com a divisão do Grande Exército, a
parte comandada por Halfdan seguiu para Nortúmbria, para assegurar a
conquista feita anteriormente, e a outra parte se manteve no sul, liderada por
Guthrum.

Após conquistar Londres, Guthrum decidiu atacar Wessex e impôs uma série
de derrotas a Alfredo. Atacou de surpresa seu palácio e
forçou o rei a se retirar para uma área pantanosa em torno de
Somerset. Ali o rei efetuou uma série de ataques de guerrilha contra os
invasores, que serviram mais como reforço de sua autoridade do que
efetivamente causaram danos à campanha de Guthrum de saques por Wessex.

Mas Alfredo soube usar o tempo a seu favor: concentrou forças
e as organizou enquanto Guthrum enfrentava problemas de natureza
diversa (em especial tentando conciliar seus interesses com outros chefes guerreiros de suas forças
que tinham desejos diferentes dos seus). Alfredo reuniu seus aliados antes
de Páscoa de 878 e lançou um ataque, pressionando as forças de Guthrum.
Após um dia de combate, os vikings romperam a formação e fugiram para
uma fortificação próxima, que já tinha tido sua comida saqueada por uma
incursão a mando de Alfredo.

Sitiados por duas semanas e sem comida, Guthrum e seus homens se
renderam nos termos de Alfredo, assinando o Tratado de Wedmore, que
basicamente assegurou a Alfredo o controle de Wessex e parte da Mércia,
enquanto reconhecia a área de soberania viking, que viria a se chamar
Danelaw. Guthrum foi convertido ao cristianismo tendo Alfredo
como padrinho, mas isso não pacificou as coisas.

De fato, a vitória garantiu tempo a Alfredo, que consolidou as defesas
de Wessex e criou uma marinha de guerra que ajudou a impedir futuras
invasões do reino, como uma outra tentativa de Guthrum em 885. Alfredo não só o
derrotou como ainda retomou Londres com ataques de sua frota. Guthrum
não teve alternativa senão aceitar novamente os termos de Alfredo,
renovando o Tratado de Wedmore em 886. Alfredo ainda travaria combates
com os vikings em algumas invasões, repelindo todas elas, mas foi a vitória em
Edington que consolidou seu nome e reputação, tornando-se 
referência para os ingleses que resistiam aos vikings.

\SIG{Sandro Teixeira Moita}

Ver também Espada; Estratégia de combate; Guerra e técnicas de combate;
Guerra e simbolismos; História da Guerra.

\begin{itemize}
\item \versal{BRINK}, Stefan; \versal{PRICE}, Neil (eds.). \emph{The Viking World}. Abingdon:
Routledge, 2008.

\item \versal{CRAUGWELL}, Thomas J. \emph{How the Barbarian Invasions Shaped the Modern
World -- The Vikings, Vandals, Huns, Mongols, Goths, and Tartars who
Razed the Old World and Formed the New}. Beverly: Fair Winds Press,
2008.

\item \versal{HARRISON}, Mark; \versal{EMBLETON}, Gerry. \emph{Anglo-Saxon Thegn 449-1066 \versal{AD}}.
London: Osprey Publishing, 1993.

\item \versal{HARRISON}, Mark; \versal{EMBLETON}, Gerry. \emph{Viking Hersir 793-1066 \versal{AD}.}
London: Osprey Publishing, 1993.

\item \versal{HEATH}, Ian; \versal{MCBRIDE}, Angus. \emph{The Vikings}. London: Osprey
Publishing, 1985.

\item \versal{HOLMAN}, Katherine. \emph{The \versal{A} to \versal{Z} of the Vikings}. Plymouth: The
Scarecrow Press, 2009.

\item \versal{SAWYER}, Peter (ed.). \emph{The Oxford Illustrated History of the
Vikings}. Oxford: Oxford University Press, 1997.

\item \versal{WINROTH}, Anders. \emph{The Age of the Vikings}. Princeton: Princeton
University Press, 2014.
\end{itemize}
\section{\versal{BATALHA DE HAFISFJORD}}

Batalha naval ocorrida em 872 ou entre 885 e 890 (há debates sobre a datação
correta), na qual Haroldo Cabelos Belos venceu uma aliança de chefes
guerreiros e se tornou rei da Noruega.

Haroldo estava em um processo de consolidação de seu poder, 
empreendendo a conquista da Noruega fracionadamente,
derrotando reis e chefes guerreiros locais em busca de aumentar os seus
domínios. Esse movimento de Haroldo é tido como causa da saída de
muitos vikings noruegueses para o mar em busca de novos lugares para
viver.

O processo de conquista de várias regiões e aumento de poder de Haroldo
não era visto com bons olhos pelos chefes locais e, após uma
série de movimentos, era óbvio que a ambição do rei seria posta à prova,
com forças do norte e do oeste da Noruega formando alianças para enfrentá-lo.

A batalha ocorreu em Hafrsjord, próximo a moderna Stavanger, na Noruega. 
Embora se tratasse de um combate entre navios, a luta acabou reproduzindo uma
lógica terrestre, com abordagens entre belonaves e uso de dardos e
projéteis entre inimigos. Haroldo atacou diretamente os líderes que se
opunham a ele, visando quebrar a organização da aliança contrária e
afetar o moral dos homens. Não se sabe o efetivo preciso, mas as fontes
registram que foi a maior batalha travada por Haroldo.

Um relato da batalha encontra-se no \emph{Heimskringla} e conta que
Haroldo era conhecido por suas tropas de \emph{berserkir} e
\emph{ulfhednar}, combatentes distintos de outros, seja por usarem peles
de urso ou lobo em suas vestes para a batalha, seja pela atuação como
forças de choque, eliminando diversos líderes inimigos.

Com a dura luta, eliminando diversos oponentes, Haroldo venceu, ao final
do dia, e foi declarado rei da Noruega, sinalizando uma unificação em
torno de sua figura. De fato, estima-se que, com a vitória em Hafrsjord,
o poder do rei não deve ter ultrapassado Trondheim, devido ao grande
poder que gozavam os chefes guerreiros do norte da Noruega, e que só
vieram a ser submetidos anos depois de muita luta, com os descendentes
de Haroldo.

Independente da extensão da soberania de Haroldo, a vitória em Hafrsjord
lançou a ideia de uma Noruega unida, objetivo a ser mantido por vários 
de seus descendentes, como Olavo Tryggvason. A coroa norueguesa seria
objeto de muita disputa nos anos vindouros, mas sua validade não se
encontrava mais em questão, sendo este o grande legado da vitória de
Haroldo.

\SIG{Sandro Teixeira Moita}

Ver também Espada; Estratégia de combate; Guerra e técnicas de combate;
Guerra e simbolismos; História da Guerra.

\begin{itemize}
\item \versal{BRINK}, Stefan; \versal{PRICE}, Neil (eds.). \emph{The Viking World}. Abingdon:
Routledge, 2008.

\item \versal{CRAUGWELL}, Thomas J. \emph{How the Barbarian Invasions Shaped the Modern
World -- The Vikings, Vandals, Huns, Mongols, Goths, and Tartars who
Razed the Old World and Formed the New}. Beverly: Fair Winds Press,
2008.

\item \versal{HARRISON}, Mark; \versal{EMBLETON}, Gerry. \emph{Anglo-Saxon Thegn 449-1066 \versal{AD}}.
London: Osprey Publishing, 1993.

\item \versal{HARRISON}, Mark; \versal{EMBLETON}, Gerry. \emph{Viking Hersir 793-1066 \versal{AD}.}
London: Osprey Publishing, 1993.

\item \versal{HEATH}, Ian; \versal{MCBRIDE}, Angus. \emph{The Vikings}. London: Osprey
Publishing, 1985.

\item \versal{HOLMAN}, Katherine. \emph{The \versal{A} to \versal{Z} of the Vikings}. Plymouth: The
Scarecrow Press, 2009.

\item \versal{SAWYER}, Peter (ed.). \emph{The Oxford Illustrated History of the
Vikings}. Oxford: Oxford University Press, 1997.

\item \versal{WINROTH}, Anders. \emph{The Age of the Vikings}. Princeton: Princeton
University Press, 2014.
\end{itemize}

\section{\versal{BATALHA DE MALDON}}

Batalha ocorrida entre vikings e anglo-saxões na Inglaterra em 991, na
qual os últimos falharam em impedir uma ressurgência de dinamarqueses e
noruegueses na ilha, fazendo com que novos ataques vikings se seguissem
nos anos seguintes.

O rei anglo-saxão, à época Etelredo~\versal{II}, mostrou-se incapaz de oferecer, 
à maneira de seus antecessores, uma resposta rápida e enérgica frente ao
renovado perigo das invasões vikings a partir de 980. De caráter
vacilante, o rei é conhecido pelo apelido 
\emph{unraed}, que pode ser traduzido do inglês antigo como ``despreparado'' ou ``mal
aconselhado'', que parece melhor para descrever seu comportamento ao
longo da vida, sempre dependente de seus conselheiros.

Ethelred descendia de reis anglo-saxões que, com dura luta, tinham
conquistado o Danelaw, e tinha uma posição invejável, mas era
incapaz de aproveitar tal potencial. A partir de 980, uma série de ações
vikings de pilhagem, em especial feitas por dinamarqueses e que se
desenvolviam sem forte oposição do rei, foram se transformando em
invasões e em uma tentativa de restabelecer o Danelaw, embora o
combate em Maldon ocorresse entre forças anglo-saxãs e um grupo de
vikings que estava em expedição de pilhagem.

Havia uma divisão na corte sobre como responder aos ataques vikings: 
um grupo defendia o pagamento de tributos para que voltassem de onde
tinham vindo sem atacar a Inglaterra, enquanto outro grupo defendia uma
resposta militar, com mobilização de forças para deter, impedir e
expulsar os vikings.

Esse último grupo tinha como um de seus líderes o \emph{ealdorman} Bryhtnoth de
Essex, cujo título de alta importância era equivalente ao de \emph{dux} ou
duque e indicava um nobre com grande responsabilidade militar. Ele tomaria a frente das
forças anglo-saxãs contra Olavo Tryggvason, que liderava uma poderosa
frota de 93 navios, tendo entre dois e três mil homens sob
seu comando.

Não há informação exata sobre o efetivo comandando por Bryhtnoth, mas
não deveria divergir muito do que os vikings tinham disposto no campo de
batalha. A luta foi renhida e dura, mas as forças comandadas por
Bryhtnoth eram de qualidade inferior em treinamento e equipamento frente
aos vikings, o que lhe custou demasiadamente caro em combate, com a
perda da própria vida.

Embora com forças inferiores, o nobre anglo-saxão estava em posição
vantajosa no início da batalha, forçando os vikings a cruzar
uma pequena ponte terrestre para atacar a linha anglo-saxã, que 
foi defendida com grande tenacidade segundo as fontes 
inglesas, que buscam ressaltar a bravura dos combatentes contra o
invasor.

Os vikings solicitaram uma trégua e requisitaram a passagem pela ponte
para terra firme, para continuar a luta, ao que Bryhtnoth aquiesceu, em
um episódio que as fontes procuram demonstrar como de elevada coragem e
lealdade. A despeito disto, a batalha foi retomada e a qualidade dos
vikings tanto em treinamento quanto em armamento começaram a pesar, levando 
alguns anglo-saxões a fugir. 

Bryhtnoth tinha colocado seu
cavalo aos cuidados de um homem chamado Godric, e isto se provou um erro
fatal. Vendo a pressão dos ataques vikings, Godric se desesperou e fugiu no
cavalo de Bryhtnoth, fato que boa parte das forças anglo-saxãs interpretou como
seu líder fugindo e agiu de acordo, fugindo de maneira desordenada,
deixando Bryhtnoth com seus guardas pessoais, que ficaram para lutar e
morrer junto a seu senhor. Ele foi morto antes do fim do combate e seus
guardas tombaram defendendo seu corpo.

A derrota em Maldon foi mal recebida por Etelredo, que imediatamente fez
um pagamento de dez mil libras em ouro e prata a Olavo e aos vikings
vencedores da batalha, iniciando uma nova política no reino da
Inglaterra quanto aos invasores: o pagamento de tributos para que
simplesmente fossem embora, que ficaria conhecido pelo nome de
\emph{danegeld}.

Entretanto, a política, em vez de promover segurança, estimulou um
grande número de invasões, com tributos sendo pagos por Etelredo a
invasores vikings em diversos anos, com somas cada vez maiores, dada a
incapacidade real em responder com força bélica aos ataques. Isso estimulou um
movimento na nobreza anglo-saxã, que acabaria na criação dos
\emph{huscarls}, infantaria pesada eficiente, bem equipada e treinada,
sendo decisiva em diversas batalhas no futuro contra os vikings.
temida por estes.

\SIG{Sandro Teixeira Moita}

Ver também Espada; Estratégia de combate; Guerra e técnicas de combate;
Guerra e simbolismos; História da Guerra.

\begin{itemize}
\item \versal{BRINK}, Stefan; \versal{PRICE}, Neil (eds.). \emph{The Viking World}. Abingdon:
Routledge, 2008.

\item \versal{CRAUGWELL}, Thomas J. \emph{How the Barbarian Invasions Shaped the Modern
World -- The Vikings, Vandals, Huns, Mongols, Goths, and Tartars who
Razed the Old World and Formed the New}. Beverly: Fair Winds Press,
2008.

\item \versal{HARRISON}, Mark; \versal{EMBLETON}, Gerry. \emph{Anglo-Saxon Thegn 449-1066 \versal{AD}}.
London: Osprey Publishing, 1993.

\item \versal{HARRISON}, Mark; \versal{EMBLETON}, Gerry. \emph{Viking Hersir 793-1066 \versal{AD}.}
London: Osprey Publishing, 1993.

\item \versal{HEATH}, Ian; \versal{MCBRIDE}, Angus. \emph{The Vikings}. London: Osprey
Publishing, 1985.

\item \versal{HOLMAN}, Katherine. \emph{The \versal{A} to \versal{Z} of the Vikings}. Plymouth: The
Scarecrow Press, 2009.

\item \versal{SAWYER}, Peter (ed.). \emph{The Oxford Illustrated History of the
Vikings}. Oxford: Oxford University Press, 1997.

\item \versal{WINROTH}, Anders. \emph{The Age of the Vikings}. Princeton: Princeton
University Press, 2014.
\end{itemize}

\section{\versal{BATALHA DE STANFORD BRIDGE}}

Famosa batalha, ocorrida em 25 de setembro de 1066 na qual o rei inglês
Haroldo Godwinson derrotou Haroldo Hardrada, rei
da Noruega, e derrotou a última grande invasão viking da Inglaterra,
fato que representa para a historiografia clássica o ponto final da Era Viking.

Em 1066, a Inglaterra vivia dias bem diferentes dos tempos das batalhas
de Edington e Maldon. O reino anglo-saxão era funcional, suas
forças eram capazes de rivalizar com as vikings no campo de batalha, e
os \emph{huscarls} eram temidos por sua qualidade militar.
Tal organização seria colocada à prova quando da invasão de
Hardrada, na qual as capacidades do reino seriam testadas.

O rei inglês vivia momentos difíceis. Seu antecessor, Eduardo, o
Confessor, não tinha deixado herdeiros e ainda por cima existiam
reivindicações ao trono da Inglaterra por parte do duque da Normandia,
Guilherme, o Bastardo, e do rei norueguês, Hardrada. O duque
era poderoso, tinha um grande exército e uma boa frota, além de
experiência vasta em combate. O rei norueguês não ficava atrás -- tendo participado da primeira
batalha de sua vida aos quinze anos, viajado pela Rússia, onde serviu o
Grande Príncipe de Kiev, seguiu até o Império Bizantino onde se tornou
um dos comandantes da Guarda Varangiana, força de elite imperial e
guarda pessoal dos imperadores.

No serviço da Guarda, Hardrada viu ação extensiva, tendo
participado de campanhas na Itália, Sicília, Bulgária e no Oriente,
tendo ainda viajado a Jerusalém, possivelmente em escolta de 
peregrinos de alta importância no Império Bizantino. Sua vida no leste
lhe garantiu fama, reputação e uma boa fortuna, que usou para financiar
sua reivindicação ao trono da Noruega.

Hardrada se tornou rei em 1046 como corregente de seu sobrinho,
Magno, o Bom, mas esse sistema durou pouco devido à morte de
Magno em 1047. Foi um rei guerreiro, consolidando a autoridade da coroa
norueguesa muitas vezes de maneira brutal, além de ter guerreado contra a 
Dinamarca diversas vezes sem, no entanto, conquistá-la.

A reivindicação do rei norueguês ganharia outro impulso quando Tostig,
irmão do futuro rei inglês, foi deposto de seus títulos e propriedades e
banido do reino em 1064 por decreto de Eduardo, o Confessor,
segundo conselho do próprio Haroldo, devido ao brutal tratamento dado
por Tostig aos chefes locais e a população geral. Tostig assassinou
nobres sob sua proteção, violando tradições e leis anglo-saxônicas.
Banido, viajou para oferecer seus serviços ao duque da Normandia, que
recusou, e acabou se aliando a Hardrada, que por sua vez dizia ter direito ao
trono inglês por um tratado assinado pelos reis da Inglaterra e da
Noruega, no qual garantiam-se os direitos de herança caso morressem sem
herdeiros.

Portanto, como Magno tinha morrido sem herdeiros, Hardrada
considerava-se na posição de reivindicar o trono inglês e lançou uma
invasão da Inglaterra com 300 navios, trazendo entre sete e nove mil
homens consigo. Mas Haroldo não considerava sua ameaça, pois estava
ocupado planejando a defesa da Inglaterra desde que tinha recebido a
notícia de que Guilherme, o Bastardo, duque da Normandia, estava reunindo uma frota de 600
navios para a invasão.

Hardrada fez algumas razias no litoral e, percebendo a pouca
oposição, lançou um ataque na direção de York, sendo detido pelos
\emph{earls} Edwin e Morcar, que conseguiram mobilizar três mil homens em poucos dias, 
valendo-se do sistema de defesa anglo-saxão. Parte
dessas forças eram \emph{huscarls} ligados aos dois chefes.

O combate se deu em Fulford, em 20 de setembro de 1066. O número das tropas e o
destemor de Hardrada abriram caminho nas linhas anglo-saxãs. 
As forças de Morcar, apesar de uma resistência brava no primeiro
momento da batalha, sofreram pesadas baixas e foram destruídas após um
ataque liderado pessoalmente pelo rei norueguês.

Hardrada e Tostig seguiram para York, que foi rapidamente rendida,
com promessas de tributos e reféns. Retiraram-se para as proximidades do
rio Derwent, onde descansavam. Mas não sabiam que Haroldo Godwinson já
se encontrava ali. O rei inglês, tendo reunido seu exército para
enfrentar a invasão que viria do duque da Normandia, valeu-se de uma
tempestade que danificou muitos navios no Canal da Mancha para usar o
tempo a favor de destruir a expedição de Hardrada e Tostig.

Assim, em 25 de setembro, em uma marcha que cobriu mais de 220
quilômetros em apenas quatro dias, Haroldo Godwinson estava pronto para
atacar os vikings, que por sua vez estavam dispersos: vários homens estavam sem equipamento e
armamento, além de dois a três mil estarem no litoral, junto aos
navios. Hardrada estava em completa desvantagem, não só pela
posição, mas também em efetivo.

Uma história é dita sobre o início da batalha, na qual Haroldo
Godwinson buscou chegar a termos com Tostig e Hardrada. Ele
ofereceu ao primeiro um terço de seu reino e quando foi perguntado
sobre o que concederia ao rei norueguês, Haroldo respondeu que
concederia ``sete palmos de terra, já que ele é mais alto que a
maioria dos homens''.

Os vikings entraram em posição circular, formando uma parede de escudos,
embora muitos estivessem sem armadura ou sequer vestidos, tamanha a
surpresa causada pela chegada dos anglo-saxões. A retaguarda foi
ordenada por Hardrada para deter os inimigos o máximo de tempo
possível enquanto mensageiros eram enviados aos navios para que os
homens que lá estavam viessem em reforço.

A mais famosa história da Batalha de Stamford Bridge é justamente sobre
a retaguarda viking. Os primeiros ataques anglo-saxônicos a destruíram
quase completamente, com exceção de um \emph{berserker}, alto e que
sozinho na ponte os detém. Esse guerreiro, girando seu machado de
batalha, acaba por matar quarenta anglo-saxões antes de ser morto por
uma lança usada por um saxão que estava em um bote, atacando seu flanco
em meio às tábuas da ponte. Isso deu tempo aos vikings de aprimorar sua
formação defensiva e lutar contra o inimigo superior em número.

O combate foi duro, mas a qualidade dos \emph{Huscarls} pesou, e logo a
linha viking foi quebrada, bem como sua parede de escudos. Totalmente
cercado, Hardrada foi atingido por uma flecha na garganta e caiu
morto. Tostig assumiu o comando, mas logo tombou em seguida, e quando o
contra-ataque viking chegou, a ``tempestade de Orri', liderado por
Eystein Orri, deteve Haroldo por pouco tempo, devido à exaustão dos
guerreiros que percorreram mais de vinte quilômetros totalmente
equipados.

Haroldo Godwinson perseguiu os sobreviventes que fugiram para os navios após a falha
do contra-ataque de Orri, que por sua vez também foi morto em combate. 
Era vencedor, mas não haveria descanso para ele, visto que tinha
sofrido pesadas baixas, tendo perdido um terço de seu exército. Mas
tinha destruído a última grande incursão viking na Inglaterra. Dos 300
navios que tinham vindo da Noruega, apenas 24 ou 25 retornariam para lá 
com os sobreviventes.

Ao receber a notícia de que Guilherme da Normandia havia posto pé na
Inglaterra no dia 28 de setembro, Haroldo tratou de marchar rapidamente
de volta ao sul, e em 14 de outubro de 1066, os dois contendores do
trono inglês lutaram na Batalha de Hastings, na qual Haroldo Godwinson
veio a morrer e Guilherme da Normandia foi intitulado ``o
Conquistador'', subjugando resistências em campanhas brutais entre 1067
e 1070.

\SIG{Sandro Teixeira Moita}

Ver também Espada; Estratégia de combate; Guerra e técnicas de combate;
Guerra e simbolismos; História da Guerra.

\begin{itemize}
\item \versal{BRINK}, Stefan; \versal{PRICE}, Neil (eds.). \emph{The Viking World}. Abingdon:
Routledge, 2008.

\item \versal{CRAUGWELL}, Thomas J. \emph{How the Barbarian Invasions Shaped the Modern
World -- The Vikings, Vandals, Huns, Mongols, Goths, and Tartars who
Razed the Old World and Formed the New}. Beverly: Fair Winds Press,
2008.

\item \versal{HARRISON}, Mark; \versal{EMBLETON}, Gerry. \emph{Anglo-Saxon Thegn 449-1066 \versal{AD}}.
London: Osprey Publishing, 1993.

\item \versal{HARRISON}, Mark; \versal{EMBLETON}, Gerry. \emph{Viking Hersir 793-1066 \versal{AD}.}
London: Osprey Publishing, 1993.

\item \versal{HEATH}, Ian; \versal{MCBRIDE}, Angus. \emph{The Vikings}. London: Osprey
Publishing, 1985.

\item \versal{HOLMAN}, Katherine. \emph{The \versal{A} to \versal{Z} of the Vikings}. Plymouth: The
Scarecrow Press, 2009.

\item \versal{SAWYER}, Peter (ed.). \emph{The Oxford Illustrated History of the
Vikings}. Oxford: Oxford University Press, 1997.

\item \versal{WINROTH}, Anders. \emph{The Age of the Vikings}. Princeton: Princeton
University Press, 2014.
\end{itemize}

\section{\versal{BATALHA DE STIKLESTAD}}

Batalha na qual Olavo Haraldsson (filho de Haroldo~\versal{I}), enfrentou chefes
guerreiros locais, apoiados por Canuto~\versal{II}, o Grande, em 29 de julho
de 1030, pela primazia do trono da Noruega.

Olavo é um dos mais famosos vikings e tinha como seu meio-irmão Haroldo
Hardrada, que lutou em Stiklestad ao seu lado com quinze anos.
Incursões militares ainda na juventude pareciam ser uma tendência na família, já que Olavo participou
de sua primeira expedição com apenas doze anos. Em uma dessas
expedições, foi convertido ao cristianismo em 1014, em Rouen, na
Normandia, em uma campanha a serviço do rei inglês Etelredo~\versal{II}, exilado
após uma devastadora invasão da Inglaterra por vikings dinamarqueses.

Retornando com renome e fortuna a Noruega, e sendo filho de um rei
local, Olavo foi condecorado rei da Noruega em 1015, em Trondheim. Seu
reino foi marcado por sua dureza enquanto governante, promovendo batalhas diversas
como a de Nesjar, em 1016, e pelo seu fervor em favor do cristianismo, ao
qual tentou forçar a conversão do país, erigindo igrejas e
executando os que se recusassem a abraçar a nova fé.

Tais medidas acabaram por gerar forte opositores e alianças entre chefes
guerreiros locais, que formaram um exército superior ao seu.
Esses chefes tinham o apoio de Canuto \versal{II}, rei da Inglaterra e Dinamarca,
que desejava anexar a Noruega a seu império no Mar do Norte. Olavo,
diante da concentração de forças contra si, partiu para o exílio em 1028
no Principado de Kiev, sendo abrigado por Iaroslav, o Sábio.

A Noruega passou a ser governada pelo \emph{earl} Hákon, um dos chefes
locais que tinha deposto Olavo e era o representante de Canuto. Em 1029
ele morreu e Olavo viu uma chance de recuperar o trono perdido.
Em seu retorno de Kiev pela Suécia, o rei exilado alistou homens no caminho e montou um
exército que julgava ser capaz de sustentar seu retorno ao trono.

Porém, pesava contra Olavo um fato do passado, de pouco antes de sua
deposição: a morte do chefe guerreiro Erling Skjalgsson, com quem tinha feito aliança
após a Batalha de Nesjar, e que o ajudara a se tornar rei. 
Embora tenho lutado contra Olavo, Erling e o rei fizeram uma
aliança instável, gerando desconfianças e uma tensão que veio a
explodir em 1027, quando Olavo prendeu o filho de Erling, acusando-o de
assassinato. Erling mobilizou forças, mas o rei perdoou seu filho e o
libertou.

Daí em diante, Erling passou a se preparar para derrubar Olavo,
liderando os chefes guerreiros. No mesmo ano ele foi à Inglaterra e conseguiu 
o apoio de Canuto, voltando em 1028 para enfrentar Olavo na
Batalha de Boknafjorden, onde foi isolado de suas forças e capturado. A
\emph{Heimskringla} conta que o rei estava disposto a perdoar o antigo
aliado, mas foi impedido quando um guerreiro decapitou Erling. O
incidente ficou famoso, e o muitos dos aliados de Erling juraram
vingança a Olavo.

A oportunidade se mostrou quando de seu retorno, com os chefes
guerreiros preparando um grande exército contra as forças de Olavo.
Marchando pela Suécia, Olavo tinha como objetivo atacar e tomar
Trondheim, mas foi detido pelos chefes em Stiklestad. Deu-se uma batalha 
na qual Olavo, com três mil homens, foi cercado por quatorze mil
liderados pelos chefes guerreiros.

Olavo foi ferido e morto em combate, mas os relatos de quem o matou diferem 
conforme as fontes e a passagem do tempo. O certo é que sua morte
desintegrou o exército, que fugiu, ou já tinha sido retirado antes da
batalha por ferimentos, como foi o caso de seu meio-irmão,
Hardrada, que seguiria para o leste, abrigando-se em Kiev
mediante a proteção de Iaroslav, tal como seu irmão tinha feito
anos antes.

O rei morto foi enterrado às margens de um rio em Trondheim, e quase que
imediatamente começou o culto a santidade de Olavo. A Igreja católica
foi rápida em canonizá-lo, considerando histórias de milagres ocorridos
até mesmo na própria batalha, quando o sangue de Olavo teria curado os
ferimentos de um dos guerreiros que lhe matou. Outros milagres ocorridos
em Trondheim e adjacências também foram levados em conta.

Canuto nomeou Sueno, seu filho, para ser seu regente na Noruega, mas o
reinado deste não foi diferente no que tange à brutalidade no trato com a
população. O fato de Sueno ser estrangeiro aumentou ainda mais a popularidade de
Olavo, que teve seu culto impulsionado ao longo dos anos. Durante a
Idade Média, sua figura cresceu em importância ao ponto de ser 
declarado rei eterno da Noruega em 1163, através da Lei de Sucessão
Real. Foi também por meio de seu culto e imagem que a conversão da
Noruega ao cristianismo, um de seus objetivos, consolidou-se.

\SIG{Sandro Teixeira Moita}

Ver também Espada; Estratégia de combate; Guerra e técnicas de combate;
Guerra e simbolismos; História da Guerra.

\begin{itemize}
\item \versal{BRINK}, Stefan; \versal{PRICE}, Neil (eds.). \emph{The Viking World}. Abingdon:
Routledge, 2008.

\item \versal{CRAUGWELL}, Thomas J. \emph{How the Barbarian Invasions Shaped the Modern
World -- The Vikings, Vandals, Huns, Mongols, Goths, and Tartars who
Razed the Old World and Formed the New}. Beverly: Fair Winds Press,
2008.

\item \versal{HARRISON}, Mark; \versal{EMBLETON}, Gerry. \emph{Anglo-Saxon Thegn 449-1066 \versal{AD}}.
London: Osprey Publishing, 1993.

\item \versal{HARRISON}, Mark; \versal{EMBLETON}, Gerry. \emph{Viking Hersir 793-1066AD.}
London: Osprey Publishing, 1993.

\item \versal{HEATH}, Ian; \versal{MCBRIDE}, Angus. \emph{The Vikings}. London: Osprey
Publishing, 1985.

\item \versal{HOLMAN}, Katherine. \emph{The \versal{A} to \versal{Z} of the Vikings}. Plymouth: The
Scarecrow Press, 2009.

\item \versal{SAWYER}, Peter (ed.). \emph{The Oxford Illustrated History of the
Vikings}. Oxford: Oxford University Press, 1997.

\item \versal{WINROTH}, Anders. \emph{The Age of the Vikings}. Princeton: Princeton
University Press, 2014.
\end{itemize}


\section{\versal{BERGEN}}

Bergen (Bjǫrgvin em nórdico antigo) atualmente é a segunda maior cidade
da Noruega, e por algumas décadas foi a capital do país e seu principal
centro político e econômico. Situada na costa
oeste da parte sul do país, no Condado de Hordaland, Bergen hoje em dia é conhecida por ser
uma cidade turística, principalmente em razão de sua bela paisagem natural.
A cidade é cercada por sete montanhas, fiordes, rios, ilhas e
penínsulas.

Entre as cidades escandinavas fundadas durante a época viking, Bergen
encontra-se no grupo das últimas cidades criadas no período. Sua
origem é normalmente associada ao governo do rei Olavo, o Pacífico (Olav
Kyrre), monarca da Noruega entre os anos de 1067 e 1093, que por
volta do ano de 1070 teria fundado Bergen entre sete montanhas. Tal
característica foi tão significante que o nome da cidade é traduzido por
alguns como ``pastagens à beira da montanha''.

Bergen começou como um pequeno
porto de uma comunidade pesqueira e agrícola anos antes da chegada do
rei Olavo. A cidade possuía acesso ao mar através da península de
Bergenshalvøyen, e
acesso ao interior através dos fiordes, tornando a localização da
cidade bem central naquela região, que décadas depois se tornaria o polo
político do país.

Com a morte do rei Haraldo Hardrada (c. 1015-1066), seu filho Olavo
assumiu o trono com a idade de 16 anos. Ao retornar para a
Noruega após a frustrada tentativa de conquistar a Inglaterra, o que
resultou na morte de seu pai, Olavo Haraldsson tratou de consolidar seu
poder no trono, restaurar a ordem no país e investir na paz e
na prosperidade. Tais medidas lhe renderam o epíteto de ``o Pacífico''. E
entre suas reformas esteve a criação de Bergen como sede de sua
residência.

Além de servir de localidade para a residência real, apesar de não ser a
capital do país, Bergen também se tornou, por volta da década de 1080, a
capital de uma diocese. O fato de parte da Noruega já ser cristã (além do
próprio rei Olavo ser convertido) contribuiu para Bergen também ganhar
notoriedade como diocese. Durante o governo do rei Olavo algumas
igrejas foram construídas na cidade. As bases da catedral seriam
erguidas décadas depois.

Ao longo do século~\versal{XII} a cidade de Bergen continuou a crescer em vários
sentidos, recebendo cada vez mais riquezas, produtos e habitantes. Sua
catedral foi inaugurada, assim como um seminário para a formação de
monges e padres. Bispos e reis passaram a ser empossados na cidade,
embora Bergen tenha sido oficialmente declarada capital do reino da Noruega 
apenas em 1247, pelo rei Hakon Hakonarson (c. 1204-1263), como
parte da celebração pela coroação do novo monarca.

A cidade continuou a prosperar e quase um século depois ela caiu nas
mãos da poderosa Liga Hanseática, coalizão formada por dezenas de
cidades alemãs, que passaram a controlar os principais centros
políticos e econômicos do norte da Europa ao longo de séculos. Mesmo
estando sob jurisdição estrangeira, Bergen ainda continuou a ser
oficialmente a capital do país até ser substituída por Oslo, atual
capital norueguesa.

Entre algumas documentações mais antigas conhecidas sobre a cidade de
Bergen estão o \emph{Historia Ecclesiastica} (c. 1135) do monge inglês
Orderic Vitalis, e os livros \emph{Heimskringla}, o \emph{Morkinskinna}
e o \emph{Flateyjarbók} datados dos séculos \versal{XIII} e \versal{XIV}, que narram
acontecimentos do reinado do rei Olavo, o Pacífico, mencionando a
condição de ele ter fundado uma cidade.

\SIG{Leandro Vilar Oliveira}

Ver também Comércio; Era Viking; Moedas e cunhagem; Noruega da Era
Viking.

\begin{itemize}
\item \versal{HOLMAN}, Katherine. \emph{Historical dictionary of the vikings}. Lanham:
Scarecrow Press Inc, 2003.

\item \versal{KRAG}, Claus. The Creation of Norway. In: \versal{BRINK}, Stefan; \versal{PRICE}, Neil
(eds.). \emph{The Viking World}. London/New York: Routledge, 2008, pp.
645-651.

\item \versal{KRAG}, Claus. The early unification of Norway. In: \versal{HELLE}, Knut (org.).
\emph{The Cambridge History of Scandinavia}, vol. 1. Cambridge:
Cambridge University Press, 2008, pp. 184-201.
\end{itemize}

\section{\versal{BIRKA}}

Situada no leste da Suécia, na ilha de Björko, no centro do lago
Mälaren, a cidade de Birka entre os séculos~\versal{IX}~e~\versal{X} foi um dos mais
importantes centros urbanos da Suécia e da Escandinávia. 
Situada a oeste de Estocolmo, atual capital do país, Birka está
desabitada há pelo menos mil anos, consistindo num dos mais notáveis
sítios arqueológicos da Suécia e Patrimônio Mundial da \versal{UNESCO}.

A cidade voltou à tona na história a partir da segunda metade do século~\versal{XIX}, 
quando expedições arqueológicas começaram a escavar a região
denominada de \emph{hemlanden}, a zona de cemitérios da
cidade. Quase três mil túmulos foram escavados desde então, fornecendo
uma grande diversidade de objetos, moedas e materiais que atestam a
posição de Birka como uma cidade mercantil. Não obstante, tais objetos
também revelam a extensão dos contatos culturais e econômicos que os
habitantes de Birka possuíam.

Foram achadas moedas árabes, contas de vidro da Rússia, seda bizantina,
espadas germânicas e, pelos detalhes de determinados trajes, alguns
dos mortos ali sepultados eram estrangeiros. A cidade
estiva conectada ao comércio intercontinental, mas também foi
exportadora. A grande quantidade de pentes, acessórios, utensílios e
peles de animais indica que essa cidade era um mercado exportador de
tais produtos. Boa parte desses itens era comercializada nos portos do Báltico,
como Hedeby, importante cidade comercial no sul da Dinamarca, e talvez
em localidades ainda mais distantes.

Todavia, somente a partir da década de 1990 novas expedições
arqueológicas voltaram a ocorrer em Birka, dessa vez focadas em
escavações do território da cidade propriamente dita. A partir dessas escavações
ocorridas nas últimas três décadas, os arqueólogos assinalaram que a
cidade de Birka deveria cobrir em sua máxima extensão um território de
11 hectares, tendo sido cercada por um muro de madeira. Um pequeno forte
foi erguido no promontório ao lado da cidade. Seu porto foi guarnecido
com um molhe, que reduzia a dimensão da entrada, forçando as embarcações
a terem que entrar em menor quantidade. O molhe em questão possuía uma
função defensiva contra possíveis ataques marítimos e não uma defesa
contra a maré.

As casas da cidade eram de madeira como o costume local, e seguiam mais 
ou menos um padrão, que variava entre 5 a 8 metros. 
Constatou-se que algumas casas possuíam dependências como forjas,
armazéns e oficinas. A relativa existência de oficinas e forjas e a
quantidade significativa de matéria-prima como peles, ferro, prata,
madeira, osso etc. são indicativos de que Birka, além de ter sido uma
cidade mercante, também tivesse sido um polo manufatureiro.

Quanto à sua história, apesar de ter sido uma cidade econômica ativa,
pouco se conhece. Dagfinn Skre aponta que Birka é considerada por alguns
historiadores e arqueólogos como a cidade mais antiga da Suécia, tendo
surgido em meados do século~\versal{VIII}. A cidade teria começado talvez como um
posto de parada para os viajantes que cruzavam o lago Mälaren, maior
lago daquela região, com saída para o mar Báltico.

Com o tempo aquele local de parada tornou-se uma vila e posteriormente
uma cidade. No século~\versal{IX}, Birka despontava como um núcleo urbano promissor,
alcançando o auge no século seguinte. Nesse ponto, Hans Andersson
comenta que, ainda no século~\versal{IX}, Birka teve contato com o cristianismo.
Com base nos relatos de São Rimberto de Hamburgo (c. 830-888), seu
amigo, o missionário São Oscar (801-865) visitou a cidade por volta dos anos de
829 a 830. De acordo com Rimberto, Oscar viajou para lá a
convite do rei Bjorn. Oscar foi acolhido pelo rei e por Herigar, um
representante seu, a quem Rimberto chamava de prefeito.

São Oscar somente retornou a Birka vinte anos depois. Nesse tempo houve
alguns conflitos, já que alguns não aceitavam o cristianismo em
sua cidade. Ainda assim, ao retornar por volta de 852, o novo monarca, o
rei Olavo, concedeu um terreno para que Oscar pudesse construir sua
igreja e um mosteiro. Tais características revelam como ainda no século \versal{IX}
o cristianismo havia conseguido se firmar rapidamente em certas
localidades da Escandinávia.

Embora Birka seja lembrada por seu comércio e tenha possuído quase 900
habitantes, por volta da década de 970 a cidade começou a ser
abandonada. Helen Clark assinala que os motivos para que Birka tenha
perdido sua importância política e econômica não são claros. Existem
duas teorias comumente citadas, mas ainda não comprovadas.

Uma das teorias, assinalada por Helen Clarke e Dagfinn Skre, indica que a foz do Mälaren
(que no século~\versal{X} era uma enseada e não um lago) teve seu nível de
profundidade reduzido, o que teria dificultado a passagem de navios de grande
calado, tornando inviável que naus mercantes para ali
se dirigissem, pois seria necessário recorrer à distribuição da carga em
embarcações menores para poderem ser conduzidas até o porto de Birka.

A segunda teoria sugere que o emergente crescimento de Sigtuna, cidade
localizada a poucos quilômetros ao norte de Birka, possa ter levado a
uma competição entre as duas pela disputa das rotas comerciais e
controle político. Sigtuna, que prosperou no século~\versal{XI}, aparentava
possuir mais recursos e poder do que Birka, que se encontrava em crise no final
do~\versal{X}.

\SIG{Leandro Vilar Oliveira}

Ver também Comércio; Moedas e cunhagem; Suécia da Era Viking.

\begin{itemize}
\item \versal{ANDERSSON}, Hans. Urbanisation. In: \versal{HELLE}, Knut (ed.). \emph{The
Cambridge History of Scandinavia}. New York:
Cambridge University Press, 2003, vol. 1: Prehistory to 1520, pp. 312-342.

\item \versal{CLARKE}, Helen. Cidades, comércio e ofícios. In: \versal{GRAHAM}-\versal{CAMPBELL}, James
(org.). \emph{Os vikings}. Barcelona: Editora Folio \versal{S.A.} 2006, pp.
78-88.

\item \versal{HOLMAN}, Katherine\emph{. Historical dictionary of the vikings}. Lanham:
Scarecrow Press Inc, 2003.

\item \versal{SKRE}, Dagfinn. The development of urbanism in Scandinavia. In: \versal{BRINK},
Stefan; \versal{PRICE}, Neil (eds.). \emph{The Viking World}. London/New York:
Routledge, 2008, pp. 83-93.
\end{itemize}

\section{\versal{BÓNDI}}

O termo em nórdico antigo \emph{bóndi} (\emph{búandi},
\emph{boándi}, pl. \emph{boendr}) significava originalmente o senhor da casa
(com a variação \emph{húsbóndi}, que originou o inglês \emph{husband}).
Numa sociedade predominantemente agrícola e rural como na Escandinávia
da Era Viking, a palavra era utilizada no sentido de fazendeiro, 
pessoa que possui uma fazenda. Também é encontrada frequentemente em
inscrições em pedras rúnicas com o significado de marido.
Posteriormente, com o desenvolvimento das cidades e do comércio após o
Período Viking, o termo passou a ser utilizado como a pessoa que tinha
propriedades. Nos códigos de leis islandeses, cada \emph{bóndi} era requisitado
para acompanhar o líder durante as assembleias. No poema éddico
\emph{Rígsthula}, \emph{bóndi} era o nome de um dos filhos de Karl (homem
livre).

Segundo Régis Boyer, o termo \emph{bóndi} pode ser empregado para toda a base
social da Era Viking. Seria a forma contraída de um particípio presente
substantivado, \emph{bóandi}, do verbo \emph{búa}, cujo sentido próprio
era preparar a terra para fazê-la produzir frutos. O sentido de morar/habitar 
é uma acepção secundária. Assim, \emph{bóndi} seria tanto o camponês
quanto o pescador e o proprietário livre, sendo uma categoria social que
não se expressava em termos de riqueza, mas em critérios de
antiguidade. O \emph{bóndi} deveria ser capaz de recapitular sua linhagem em
várias gerações. Tinha direito a efetuar uma ação na justiça e em caso
de sofrer uma ofensa, de exigir compensação plena (\emph{bót}). Existiam
também variações da terminologia, como o \emph{ódalsbóndi} (em norueguês
\emph{hauldr}), aquele que habita uma propriedade depois de várias
gerações.

Os \emph{stórboendr} (os grandes \emph{boendr}) eram os grandes proprietários,
aqueles que administravam e governavam a comunidade, criando 
certos laços de fidelidade e clientelismo com os outros estratos
sociais. A maior parte dos fazendeiros eram os \emph{smáboendr}, os
pequenos proprietários. Uma outra categoria de fazendeiro era o
\emph{bryti}, geralmente encarregado da administração das propriedades e
provavelmente originado da escravidão. Mas no final da Era Viking, entre
os aristocratas e a realeza, essa categoria poderia desfrutar de grande
\emph{status} e prestígio.

\SIG{Johnni Langer}

Ver também Era Viking; Islândia da Era Viking; Sociedade; Viking.

\begin{itemize}
\item \versal{BOYER}, Régis. Les Stórboendr/Les smáboendr. In: \emph{L´Islande
Médiévale}. Paris: Les Belles Lettres, 2002, pp. 46-50.

\item \versal{BOYER}, Régis. Le boendr. In: \emph{Les vikings}. Paris: Perrin, 2002,
pp. 260-273.

\item \versal{BOYER}, Régis. La sociedad viking. In: \emph{La vida cotidiana de los
vikingos}. Barcelona: José J. de Olañeta, Editor, 2000, pp. 51-70.

\item \versal{HOLMAN}, Katherine. Bóndi. In: \emph{Historical Dictionary of the
Vikings}. Oxofrd: The Scarecrow Press, 2003, pp. 49-50.

\item \versal{SHORT}, William R. Social structure and gender. In: \emph{Icelanders in
the Viking Age}. London: McFarland \& Company, 2010, pp. 32-39.
\end{itemize}

\section{\versal{BRACTEATAS}}

A bracteata é um tipo de medalhão produzido na Idade do Ferro germânica
entre os séculos~\versal{V}~e~\versal{VI}. A maioria deles possuía 
ilhós que os indicavam como uma indumentária para ser usada no pescoço, já que esses ganchos eram
feitos para passar cordões. Provavelmente eram peças que adornavam os
pescoços de aristocratas, uma vez que a grande maioria era feita de
ouro.

As bracteatas foram divididas por estudos arqueológicos em 7 tipos,
divisão que leva em consideração os motivos gravados nesses artefatos e
que os classificam em \versal{A, B, C, D, E, F, M}. O tipo~\versal{A} é gravado com um
rosto antropomórfico, tendo como modelo as moedas do Império Romano; o
tipo~\versal{B} mostra de uma a três figuras antropomórficas de pé, sentadas ou
de joelhos, acompanhadas geralmente por figuras zoomórficas; o tipo~\versal{C} é
representado por uma figura antropomórfica sobre a figura zoomórfica de
um quadrúpede, normalmente interpretado como Odin e Sleipnir; o tipo~\versal{D}
apresenta apenas figuras zoomórficas; o tipo~\versal{E} apresenta um \emph{trisquel}
formado por figuras zoomórficas em cima de uma figura geométrica
circular; o tipo~\versal{F} é um subgrupo dos bracteates~\versal{D}, apresentando também
figuras zoomórficas; e o tipo~\versal{M} consiste de imitações bifaciais dos medalhões
imperiais romanos.

Foram descobertos mais de 970 bracteatas dos tipos \versal{A, B, C, D e F}; 
destes, cerca de 135 levam inscrições rúnicas com caracteres do
futhark antigo. Dentre as inscrições que mais reverberaram na
historiografia encontra-se a da bracteata Seeland-\versal{II-C}, que oferece
proteção para viagem a quem o leva. As gravações rúnicas das bracteatas
demonstram a importância da arte de gravar runas como instrumento de
ação mágica que favorece o contato entre os homens e o sagrado por meio
da gravação de artefatos. Outras 270 bracteatas do tipo~\versal{E}, pertencentes
ao Período Vendel (e portanto posteriores à maior parte das bracteatas
da era das migrações), foram encontradas apenas em Gotland,
na atual Suécia, e foram feitas em prata e bronze, diferente das demais, 
feitos em ouro.

Contudo, nem todas as bracteatas pertencem ao esquema já
supramencionados. Dentre as que fogem ao padrão temos a representação do
deus Týr, datada do século \versal{VI}, artefato encontrado na região de
Trollhättan, na atual Suécia. A bracteata mostra o deus Týr tendo uma de
suas mãos mordida pelo lobo Fenrir, representação que levou arqueólogos
a concluírem que alguns dos mitos que nos chegam pelas compilações do
século \versal{XIII} teriam surgido e ganhado certa importância perante os povos
nórdicos já no século~\versal{V}.

O mito da perda da mão de Týr encontra-se na \emph{Edda em prosa} e na
\emph{Edda poética} e nos diz que os deuses, ao tentarem prender o lobo
Fenrir, fazem com ele uma aposta: ele não conseguiria quebrar as
correntes que nele fossem colocadas. Mas as primeiras correntes foram
quebradas facilmente. Depois disso, porém, os deuses pedem a alguns
anões que construam uma corrente muito forte, denominada Gleipnir, feita
com o som da pisada de um gato, a barba de uma mulher, as raízes de uma
montanha, tendões de urso, respiração dos peixes e saliva de pássaros.
O lobo Fenrir logo fica sabendo da artimanha dos deuses e pede para que
algum deles cumpra o trato de colocar a mão em sua boca, para garantir
que ele não seria preso porque, se acaso o fosse, iria comer a mão do
deus. Týr foi o único deus a se propor a enfrentar a fera e, por esse
motivo, perdeu sua mão. Por fazer com que o acordo entre os deuses e o
lobo fosse cumprido, mesmo com a perda de sua mão, Týr é considerado
pelos historiadores como o deus das leis e da ordem (\emph{Gylfagining}
33).

A imagem de deuses e as inscrições de proteção presentes nas bracteatas
indicam sua conexão a atividades ritualísticas, mas
para compreendermos essa conexão devemos buscar a padronização de seus
depósitos. Os depósitos padronizados excluem a possibilidade de que
foram feitos em épocas de crise e guerra com a finalidade de serem
recuperados, uma vez que os depósitos de crises e de guerras são
compostos de resquícios de valor em associações aleatórias. O rito é
considerado, assim, uma atividade padronizada, ou seja, pobre em potencial
semântico e em sua característica argumentativa lógica. Podemos dizer,
desse modo, que os estudos do rito devem compreendê-lo como forma de
expressão que se difere de uma linguagem natural, na qual podemos dizer
novas coisas e criarmos argumentos, e assim, a comunicação ritual se
encontra protegida de rápidas modificações.

Os achados de bracteatas em depósitos singulares ou até mesmo em
depósitos múltiplos de no máximo vinte bracteatas nos indicam uma
padronização para o início do período da Idade do Ferro germânica. Os
depósitos estavam em pântanos, terra e até mesmo em praias e lagos.
Aqueles depósitos que contavam com uma ou duas bracteatas são
encontrados quase em sua totalidade sem conexão alguma com qualquer
outro tipo de objeto; quando se encontram conectados, estão sempre
junto de colares, anéis e braceletes de ouro, mas nunca com contas de
âmbar e de vidro. Os que se encontram associados com broches são em sua
maioria depósitos com mais de duas bracteatas e podem contar com outras
associações, como contas de vidro, contas de âmbar e com anéis em formato
de espiral, mas raramente com colares, braceletes, tiras e barras de
ouro que aparecem em outros depósitos. Assim, podemos concluir que os
depósitos de bracteatas que ocorriam por toda a Escandinávia durante a
Idade do Ferro germânica contam com certas padronizações, que nos
indicam uma utilização ritualística.

\SIG{Munir Lutfe Ayoub}

Ver também Arqueologia da Era Viking; Cultura material; Religião.

\begin{itemize}
\item \versal{AXBOE}, Morten. The Scandinavian Gold Bracteates. Studies on their
manufacture and regional variations. With a supplement to the catalogue
of Mogens B. Mackeprang.~\emph{Acta Archaeologica}, 1981, vol. 52, pp.
1-100.

\item \versal{HEDEAGER}, Lotte. \emph{Iron Age Myth and Materiality: An Archaeology of
Scandinavia ad 400-1000}. New York: Routledge, 2011.

\item \versal{HEDEAGER}, Lotte. \emph{Iron-Age Societies}. Tradução de John Hines.
Cambridge: Three Cambridge Center, 1992.

\item \versal{MEES}, Bernard. On the typology of the texts that appear on migration‐era
bracteates.~\emph{Early Medieval Europe}, vol. 22, n. 3, 2014, pp.
280-303.

\item \versal{STURLUSON}, Snorri. Edda Snorra Sturlusonar. In: \versal{JÓNSSON}, Finnur (ed.).
\emph{Edda Snorra Sturlusonar}. Reykjavík: Kostnadarmadur: Sigurdur
Kristjánsson, 1907.
\end{itemize}


\section{\versal{BRATTAHLID}}

Nome da localidade na Groenlândia (c. \emph{Brattahlíð}) em que o viking
norueguês Érico, o Vermelho (\emph{Eiríkr Þorvaldsson} ou \emph{Eiríkr
hinn rauði}) constrói seu assentamento em meados do século~\versal{X}. O lugar
fica na região sudoeste da Groenlândia, bem próxima ao litoral do fiorde
de Tunulliarfik ou fiorde de Érico, composta por terras férteis e de declives e
aclives, tanto que seu nome de batismo significa ``encostas íngremes'',
em uma tradução livre. Atualmente a localidade deste assentamento se
encontra a poucos metros de Qassiarsuk, que é pertencente ao
município de Kujalleq, que
contém um pouco mais de cem habitantes.

A presença deste assentamento é confirmada pela presença de vestígios
arqueológicos e por uma série de fontes, principalmente as sagas do
descobrimento da América (\emph{Vínland sagas}), composta pela
\emph{Saga de Eiríkr, o Vermelho} (\emph{Eiríks Saga Rauða)} e pela \emph{Saga dos
Groenlandeses} (\emph{Grænlendinga Saga}). Uma das primeiras revelações
desta última é que Érico se tornou um proscrito devido
a contendas familiares, o que possibilitou sua expedição em busca de
novas terras, com base nos boatos de Gunnbjörn. Ao encontrar essa nova
terra -- que seria a Groenlândia -- chamando a localidade encontrada de
Miðjökull (região de Ammassalik, leste da Groenlândia), ele
seguiu costeando em direção ao sul, onde permaneceu durante o primeiro
inverno na ilha de seu nome.

Após quatro verões de explorações ele retorna para a Islândia, como
demonstra a \emph{Saga dos Groenlandeses}, avisando que encontrará uma nova
terra, chamando-a de \emph{Grænland}, ou seja, Terra Verde. Após um
tempo na Islândia a saga nos diz: ``Ele morou em Brattahlíð, no Fiorde
de Eiríkr'', e depois, ``Eiríkr Vermelho morava em Brattahlíð. Ele vivia
lá com a máxima honra e todos buscavam o seu conselho. Eram estes seus
filhos: Leifr, Thorvaldr e Thorsteinn, já a sua filha se chamava
Freydís'' (Anônimo, 2007a, p. 58-60).

Já na \emph{Saga de Eiríkr, o Vermelho}, a aparição da localidade pela primeira
vez no texto tem alguns problemas documentais, já que alguns não a
consideram como parte original da saga, apesar de fazer parte do
\emph{Hauksbók} editado por Matthías Thórðasson, em que se lê: ``Depois
Eiríkr tomou posse do fiorde de Eiríkr e passou a morar em Brattahílð''
(Anônimo 2007b, p. 91).

Arqueologicamente, desde o século~\versal{XIX} ocorreram exploração na área, mas
foi em 1961, com um trabalho iniciado em Qassiarsuk, que as pesquisas
e achados se ampliaram, revelando, através de cinco
escavações feitas no ano seguinte, um cemitério com cerca de cento e cinquenta sujeitos
enterrados ao redor de uma pequena igreja. Nove desses esqueletos foram
datados via radiocarbono como pertencentes ao intervalo entre os séculos~\versal{X}~e~\versal{XI--XII}.
Além dessa comprovação, acredita-se que a ermida encontrada tenha sido 
construída pela mulher de Leifr, filho de Érico, o responsável por
trazer o cristianismo para aquelas terras, que é demonstrada na \emph{Saga de
Eiríkr, o Vermelho}: ``Leifr aportou no Fiorde de Eiríkr e foi em seguida
para casa em Brattahlíð. Todos lá o receberam bem. Ele logo propôs o
cristianismo e a fé católica na terra [...] Eiríkr não gostou muito
da ideia de abandonar a sua religião, já Thjóðhildr converteu-se
rapidamente e mandou construir uma igreja, num local não muito próximo
da fazenda. Aquela construção foi chamada de Igreja de Thjóðhildr
(\emph{Þjóðhildarkirkja})'' (Anônimo, 2007b, p. 100-101).

Essa localidade tornou-se um marco tanto pelo assentamento do primeiro
colonizador da Groenlândia e de seus descendentes como também pela fundação
da primeira igreja na região (há uma reconstrução moderna desta ermida
no local), e também da realização do primeiro \emph{Þing}
groenlandês. Sabe-se que até o século \versal{XV} a região permaneceu habitada
por descendentes de Érico, mas uma série de fatores, como uma pequena
era do gelo, a erosão do solo, as disputas comerciais da liga Hanseática,
a competição com os inuítes, somados a uma redução do interesse
norueguês na área, acabaram por causar um isolamento da região como um
todo, findando o assentamento. Os relatos mostram que o
último barco da Groenlândia para a Europa teria saído em meados de 1410,
revelando a conjunção de fatores que culminou no abandono da localidade.

\SIG{José Lucas Cordeiro Fernandes}

Ver também Esquimós (inuítes) e nórdicos; Freydis; Leif Eriksson; Sagas
do Atlântico Norte; Vínland.

\begin{itemize}
\item \versal{ANÔNIMO}. A Saga do Groenlandeses. In: \emph{As três sagas Islandesas}.
Tradução de Théo Moosburger. Curitiba: Editora \versal{UFPR}, 2007a.

\item \versal{ANÔNIMO}. A Saga de Eiríkr Vermelho. In: \emph{As três sagas Islandesas}.
Tradução de Théo Moosburger. Curitiba: Editora \versal{UFPR}, 2007b.

\item \versal{ARNEBORG}, Jette. The Norse Settlements in Greenland. In: \versal{BRINK}, Stefan;
\versal{PRICE}, Neil (eds.). \emph{The Viking world}. London: Routledge, 2012,
pp. 588-597.

\item \versal{DIAMOND}, Jared.~\emph{Collapse: how societies choose to fail or
succeed}. New York: Viking Penguin, 2005.

\item \versal{GWYN}, Jones. \emph{La saga del
Atlántico Norte: establecimiento de los vikingos en Islandia,
Groenlandia y América}. Barcelona: Oikos-Tau, \versal{S.A.} Ediciones, 1992.

\item \versal{SHAFER}, John Douglas. \emph{Saga accounts of norse far-travellers}.
Durham: Durham University, 2010.

\item \versal{UMBRICH}, Andrew. \emph{Early Religious Practice in Norse Greenland:
From the Period of Settlement to the 12 th Century}. Reykjavík:
Universidade da Islândia, 2012.
\end{itemize}

\section{\versal{BREVIS HISTORIA REGUM DACIE}}

A \emph{Brevis historia regum dacie}, também conhecida como
\emph{Compendiosa regum daniae historia}, foi escrita em latim por volta
de 1188 por Sueno Aggesen. O conteúdo da obra apresenta brevemente a
sucessão genealógica dos reis da Dinamarca, uma forma de escrita da
história muito comum entre os séculos~\versal{XII}~e~\versal{XIII}. O período abordado por
Sueno é aproximadamente entre 300 e 1185, abrangendo desde o período do
rei mítico Skjöld, primeiro rei da Dinamarca, até o início do reinado de
Canuto~\versal{VI}.

A forma de escrita da \emph{Brevis historia}, de acordo com o próprio
Sueno Aggesen, foi baseada em antigas tradições orais e em fontes
escritas consultadas pelo autor. Considerando a tradição histórica
danesa existente antes da composição da \emph{Brevis historia}, deve-se
salientar que a obra de Sueno não pode ser destacada como a pioneira
história nórdica referente ao território da Dinamarca. De acordo com
Eric Christiansen, o autor da \emph{Brevis historia} provavelmente
conhecia essa tradição e fez uso da mesma para compor sua obra. Mesmo
assim, Aggesen rompeu de duas maneiras com a tradição de escrita da história. 
Em primeiro lugar, preocupou-se em estabelecer um considerável
vínculo com o passado, já que destacou em sua obra a existência da
monarquia danesa como uma herança de tempos antigos, reivindicados pela
coragem e pela sabedoria do seu contexto; em segundo lugar, apresentou
em sua obra uma perspectiva de continuidade entre o contexto anterior e
posterior à presença do cristianismo, considerando os dois momentos como
uma história contínua e não como um rompimento.

Pouco se sabe sobre a vida de Sueno Aggesen. Provavelmente foi um nobre
danês, cuja família tinha influência sobre os arcebispados de Lund e de
Viborg. Sua família pertencia a um destacado estamento e participava da
política danesa da época, já que seu avô, Kristiarn Svensen, e seu pai,
cujo nome provavelmente era Aggi, foram guerreiros de elite com uma
reconhecida reputação, e seu tio, Eskil, foi considerado o segundo
arcebispo mais importante da Dinamarca em sua época. Provavelmente
passou por uma preparação intelectual na França e suas obras estão
relacionadas a um contexto no qual a independência danesa era ameaçada
pela influência germânica. De acordo com Eric Christiansen, Sueno Aggesen
provavelmente teve um necessário conhecimento da lei, manteve um certo
contato com Saxo Gramático e fez parte de um destacado estamento
político ao qual sua família pertencia.

Existe somente um manuscrito que contém o texto da \emph{Brevis historia
regum dacie}. Identificado como ``\versal{A}'', encontra-se na Biblioteca de
Copenhagen (\versal{AM} 33 4º), e é uma cópia de um manuscrito medieval composto
aproximadamente em 1570 e patrocinado por Claus Lyschander (1558-1624).
A \emph{editio princeps}, editada por Stephan J. Stephanius no século
\versal{XVII}, é identificada como ``\versal{S}'', e é uma versão melhorada e corrigida de
um manuscrito medieval perdido, composto no começo século~\versal{XIII}.

O conteúdo da \emph{Brevis historia} apresenta uma perspectiva
geográfica e cronológica bem rudimentar. A obra é composta de um
prefácio e vinte capítulos. Logo no prefácio, Sueno explica sua decisão
em narrar os feitos dos antigos reis da Dinamarca. O primeiro capítulo
refere-se ao reinado de Skjöld, cuja principal característica foi a
proteção feita às fronteiras do reino. Além disso, destaca também as
diversas sucessões após Skjöld: Halfdan, Helghi, Rolf Kraki, Rokil
Slagenback e Frothi, o Ousado. Nos dois capítulos seguintes é comentado
o ataque dos Teutônicos durante o reinado de
Wermund, o sábio, assim como a luta de seu filho, Uffi, contra os
teutônicos. O quarto capítulo apresenta uma longa sucessão de reis,
desde Dan o orgulhoso até Ingiald, com um intervalo no qual o autor explica
a sucessão sendo feita não por parte dos filhos dos reis, mas somente
por seus sobrinhos e netos, e retoma a narrativa sucessória a partir de
Canuto (filho de Sighwarth, que por sua vez era filho de Regner
Lothbrogh) até Gorm, o velho. Destaca-se na narrativa a atenção dada às
ações da mulher de Gorm, o Velho, chamada Thyrwi, no quinto capítulo,
assim como as negociações entre a mesma e o imperador Oto~\versal{I}, que se
estende também pelos dois capítulos seguintes com o fim do tributo pago
pelos daneses. O capítulo oitavo apresenta o reinado de Haroldo Dente Azul,
filho de Gorm, o Velho e Thyrwi, primeiro rei cristão da Dinamarca, e a
subida do seu filho Sueno Haraldsson ao trono durante o seu exílio, o
qual foi capturado pelos eslavos e posteriormente resgatado. O reinado
de Canuto, filho de Sueno Haraldsson, quando foram expandidas as
fronteiras do reino, e o casamento entre sua filha, Gunilda, e o
imperador Henrique~\versal{III} são os temas do nono capítulo. Também apresenta
seus dois filhos, Canuto~\versal{III} (rei da Inglaterra e da
Dinamarca) e Sueno Alfivason (que sucedeu o pai como rei da Noruega). O
décimo capítulo apresenta a sucessão na Dinamarca pelo sobrinho de
Canuto, Sueno Estrithson e, posteriormente à morte deste, a sucessão por
seu filho, Haroldo. O décimo primeiro capítulo comenta sobre a sucessão
no trono da Dinamarca pelo irmão de Haroldo, Canuto~\versal{II}, passada no capítulo
seguinte para seu irmão, Olavo, seguido pelos seus outros irmãos, Érico,
o Bom e Nicolau, o Velho, assim como o nascimento de Magno, filho de 
Nicolau. Os dois capítulos seguintes comentam sobre a morte de Canuto de
Ringsted por Magno e o início de uma guerra
civil (que se estende pelos capítulos seguintes) entre Érico, o
Memorável e Nicolau, o Velho, citando as batalhas de Rønbjerg e da ponte
de Onsild, assim como a derrota de Nicolau. O capítulo que se sucede 
apresenta o reinado de Érico, o Memorável, após a sua vitória sobre
Nicolau, o Velho, assim como a sua morte, e a sucessão por Érico, o
Cordeiro. No décimo sétimo capítulo encontra-se a luta entre Canuto,
filho de Magno, e Sueno, filho de Érico, o Memorável, a tentativa de paz
entre estes dois e Valdemar, filho de Canuto de Ringsted, e a morte de
Canuto, filho de Magno. A luta entre Sueno e Valdemar, a vitória deste
e suas ações para assegurar as fronteiras do reino continuam a ser
explicadas no capítulo seguinte, e suas características pessoais são
relembradas no capítulo décimo oitavo. O capítulo décimo nono e o
vigésimo apresentam o casamento entre Valdemar e a rainha Sofia e a
sucessão de Valdemar pelo seu filho Canuto, o último rei presente na
narrativa da \emph{Brevis historia}.

\SIG{Luciano José Vianna}

Ver também Dinamarca da Era Viking; Escandinávia; Fontes primárias.

\begin{itemize}
\item \versal{CHRISTIANSEN}, Eric. Sueno Aggonis. In: \emph{Medieval Nordic Literature
in Latin. A Website of Authors and Anonymous Works (c. 1100-1530)}.
Disponível em: \emph{https://wikihost.uib.no/medieval/index.php/Sueno\_Aggonis}.
Acesso em 18/06/2017.

\item \versal{HOLMAN}, Katherine. \emph{Historical Dictionary of the Vikings}. Lanham,
Maryland, and Oxford: The Scarecrow Press, Inc. 2003, pp. 54-55, 264.

\item \versal{MORTENSEN}, Lars Boje. Comparing and Connecting: The Rise of Fast
Historiography in Latin and Vernacular
(12\textsuperscript{th}-13\textsuperscript{th} Century). \emph{Medieval
Worlds}, 1, 2015, pp. 25-39.

\item \versal{MORTENSEN}, Lars Boje. Historia Norwegie and Sven Aggesen: Two Pioneers
in Comparison. In: \versal{GARIPZANOV}, Ildar (ed.). \emph{Historical Narratives
and Christian Identity on a European Periphery: Early History Writing in
Northern, East-Central, and Eastern Europe (c. 1070-1200)}. Brepols:
Turnhout, 2011, pp. 57-70.

\item \emph{The Works of Sven Aggesen. Twelfth-Century Danish Historian}.
Translated with Introduction and Notes by Eric Christiansen. London:
Viking Society for Northern Research, 1992.
\end{itemize}
\section{\versal{BRIAN BORU (BÓRUMA)}}

Brian Bóruma, conhecido como Brian Boru, é possivelmente o rei medieval
irlandês mais lembrado de todos ao longo da história, principalmente por
ser considerado o último grande rei da Irlanda e também porque seu nome
está associado ao fim da Era Viking na ilha após sua vitória na Batalha
de Clontarf, em 1014.

A carreira de Brian tem início com a morte de seu irmão
Mathgamain, rei dos Dál Cais em 976. Os Dál Cais formavam uma
pequena dinastia familiar na região de Munster, próximo a Limerick, que
teria se assentado por volta do século~\versal{VIII} em terras onde hoje são o
condado de Clare. Com Mathgamain, os Dál Cais assumem o poder na
região de Munster ao destronar os Eóganacht de Cashel e, após 976, 
com a liderança de Brian Boru, seu poder apenas aumentou, 
combinando habilidade militar e ações políticas coligadas à Igreja ou por
meio de casamentos estratégicos.

Ao assumir o poder, Brian tratou de consolidar seu domínio em Munster
atacando os vikings de Limerick (que teriam sido responsáveis pela morte
de seu irmão) e os demais reinos menores que compunham a região. Brian
não se limita apenas à sua província e marcha sobre as demais regiões
irlandesas, impondo, assim, sua autoridade, principalmente à região de
Leinster. Lá encontra certa oposição do rei Máel Sechnaill
(Malachy), que tinha pretensões similares às de Brian, o que ocasionou uma
disputa entre os dois pelo domínio de outras regiões, como Connacht, por
exemplo.

Os dois reis possuíam estratégias diferentes. Brian possuía não apenas o
domínio militar por terra, mas manejava bem algumas embarcações e o uso
de diversos portos, muitos deles sob o domínio de populações vikings com as
quais Brian fez alianças. Dessa maneira, o avanço de Brian se tornou de
difícil contenção e em 996 ele já havia dominado a região de Leinster,
motivo pelo qual Máel Sechnaill fora obrigado a reconhecer a
autoridade de Brian sobre a região de Leth Moga (ou seja, o sul
da ilha composto por Leinster e Munster).

Em 997, os dois reis entram em comum acordo e em 999 marcham juntos
contra os escandinavos de Dublin, na grande Batalha de Glenn Máma. Nessa
batalha, as forças de Brian massacram os vikings da região e o rei de
Dublin, Sitric, torna-se vassalo de Brian logo após selar o
acordo de submissão ao se casar com Sláine, uma das filhas de
Brian Boru.

Brian continuou seu avanço militar sobre a Irlanda e, em 1002, sua
supremacia foi reconhecida sobre a região de Leth Cuinn (o norte
da ilha, composto por Connacht, Ulster e Meath). Brian continua suas
incursões pelo norte até o ano de 1005, quando demonstra seu poder
militar e torna-se inquestionavelmente o rei de toda a Irlanda. É em
1005 também que Brian estreitará seu poder eclesiástico ao fazer boas
doações para a igreja em Armagh (grande centro monástico irlandês da
época) que o reconhecerá como imperador dos irlandeses
(\emph{Imperatoris Scotorum}).

Neste momento, a região de Leinster possui um novo rei chamado
\emph{Máel Morda macMurchada}. Esse rei tentará se opor ao poder de
Brian Boru na região e fará alianças com o rei de Dublin, Sitric, 
e outros chefes escandinavos. A disputa será resolvida na famosa Batalha
de Clontarf (1014), narrada em diversos registros textuais, sendo o mais
conhecido entre eles o \emph{Cogadh Gáedhel re Gallaibh} (\emph{Guerra dos
irlandeses com os estrangeiros}).

Por razões de propaganda política, o \emph{Cogadh} apresenta os vikings
e Brian como forças opostas visando a dominação da Irlanda. No entanto,
o próprio Brian Boru possuía aliados entre os escandinavos, notoriamente
entre os que habitavam regiões como Wexford, Cork e Waterford. A vitória
das forças de Brian na Batalha de Clontarf muda o panorama político não
apenas dos irlandeses na ilha, mas também dos nórdicos em questão.

Apesar da vitória em 1014, Brian morre em decorrência da batalha.
Coincidentemente o poder político das comunidades escandinavas na
Irlanda também vai diminuindo aos poucos nessa época, feito que será
atribuído a Brian Boru, mesmo que não seja essa a sua intenção. Com sua
morte, segundo narra o \emph{Cogadh}, ele será enterrado com toda pompa
em Armagh e seu nome preservado como o grande imperador dos irlandeses.
Seus descendentes buscarão em seu nome legitimidade e muitos outros reis
procurarão pleitear seu antigo \emph{status} (surge o termo ``rei com
oposição'', por exemplo), mas nunca conseguirão igualar seu poder sobre
a ilha, que se manterá em constante disputa por um novo grande rei até a
invasão normanda.

\SIG{Erick Carvalho de Mello}

Ver também Celtas e nórdicos; Dublin; Irlanda da Era Viking.

\begin{itemize}
\item \versal{DOWNHAM}, Clare. Irish chronicles as a source for inter-Viking rivalry,
\versal{A.D.} 795-1014. \emph{Northern Scotland}, 26, 2006, pp. 51-63.

\item \versal{DUFFY}, Seán. \emph{Brian Boru and the Battle of Clontarf}. Dublin: Gill
Books, 2014.

\item \versal{\versal{Ó CUÍV}}, Brian. Ireland in the Eleventh and Twelfth Centuries c.
1000-1169. In: \versal{MOODY}, Theodore \versal{W}. \& \versal{MARTIN}, Francis \versal{X}. \emph{The
Course of Irish History}. Cork: Mercier Press, 2011, pp. 107-122.

\item \versal{RICHTER}, Michael. \emph{Medieval Ireland: The Enduring Tradition}.
Dublin: Gill and Macmillan, 1988.
\end{itemize}

\section{\versal{BÚSSOLA SOLAR}}

A bússola solar é um objeto feito de madeira ou pedra utilizado para
orientação náutica no medievo, popularizada na mídia com a série
\emph{Vikings} (2013). A primeira evidência de um objeto deste tipo
ocorreu em 1948, quando o arqueólogo Christen Leif Vebæk descobriu,
em um convento beneditino de Uunartoq, na Groenlândia, um disco de madeira quebrado,
datado de 1200 d.C. Medindo 70 mm de diâmetro, ele é
atualmente conservado no Museu Nacional de Copenhagen. Tempos depois, em
1953, o capitão e historiador marítimo Carl~\versal{V}. Sølver reconheceu que
duas linhas gravadas no disco poderiam corresponder a curvas das sombras
de um gnômon (ponteiro ou haste vertical). As curvas variam segundo a
altura e a estação do ano e as duas linhas gravadas no disco de madeira
correspondem ao trajeto do Sol durante os equinócios e solstícios. Além
disso, o artefato marcaria 32 posições diferentes. Durante os anos 1990,
o arqueólogo Christen Vebæk e Soren Thirslund publicaram alguns livros e
artigos sobre o artefato. Alguns pesquisadores questionaram esse
artefato como sendo uma bússola, acreditando que ele teria sido um
``disco confessional'', um objeto utilizado pelos sacerdotes nórdicos
medievais para contar o número de confissões.

Mas outros vestígios semelhantes foram descobertos. Na Groenlândia
(Vatnahverfi) foi recuperada uma peça feita de esteatite, contendo uma
curva gnômica e um buraco central (talvez para portar uma haste de
gnômon). Também um disco de madeira se encontra no Museu de Wolin,
Polônia, datado do século~\versal{XI} d.C. Ele foi encontrado em uma escavação no
ano de 2000, junto a uma embarcação nórdica e objetos eslavos de Wolin.
Possui 81 mm de diâmetro e um buraco central, possivelmente
utilizado para inserir a haste central do gnômon. Segundo análises de
pesquisadores poloneses, o artefato teria sido utilizado para determinar
a sombra do Sol durante os equinócios e o verão. Em 2004, um experimento
náutico realizado na ilha de Møn (Dinamarca) testou uma réplica desse
disco. Quatro discos de madeira foram utilizados, três em branco e uma
cópia de Wolin. O experimento confirmou o artefato como sendo um
marcador solar e que teria sido utilizado na latitude do Báltico e do
norte europeu; suas 12 linhas demarcariam pontos específicos do
horizonte. Outro objeto considerado um marcador solar foi encontrado em
uma tumba escandinava da ilha de Grix (França), datado do século~\versal{X} e
feito de metal, com quatro círculos concêntricos demarcando o Sol
durante suas diversas posições diurnas.

Apesar de não existir nenhuma referência literária ou alguma menção em
documentos medievais sobre bússolas solares, a maioria dos pesquisadores
e navegadores acreditam que ele foi utilizado como gnômon para
determinar a latitude e, portanto, determinar a localização da embarcação.
Algumas réplicas foram construídas e navegadores como Robin Knox-Johnson
e Mike Cowham a testaram em experimentos náuticos, com pequenas margens
de erro. Um objeto muito semelhante foi o \emph{solskuggerfjol}: um
gnômon inserido dentro de um recipiente com água, utilizado para
determinar a latitude do local e descrito por marinheiros modernos das
ilhas Feroé.

As mais completas e atualizadas discussões sobre as bússolas solares
nórdicas tiveram início a partir de 2013, com uma equipe
multidisciplinar composta por pesquisadores húngaros, liderados por
Bálazs Bernárth. Em uma primeira publicação, os investigadores
concluíram que os artefatos foram utilizados para determinar o meio-dia
solar de uma localidade e a sombra do meio-dia no mar aberto. Ele
combinaria duas funções como instrumento náutico: relógio
solar e marcador de borda das sombras do Sol. Pode ter funcionado
como calendário portátil, mas não como relógio de Sol, pois suas
marcações são incompletas para um dia inteiro. Os pesquisadores também
realizaram algumas hipóteses sobre a origem do instrumento -- apesar de
os nórdicos não terem um equipamento semelhante na área escandinava,
muitos gnômons eram conhecidos por cristãos na Europa Setentrional,
incluindo áreas em que os vikings realizaram incursões. Também a
experiência comercial e militar entre o mundo nórdico e outras áreas da
Europa e Ásia podem ter levado ao conhecimento empírico que originou os
artefatos, especialmente o mundo islâmico.

A mesma equipe húngara realizou outros estudos e testes em 2014,
afirmando que a bússola solar pode ter sido utilizada conjuntamente com
pedras solares feitas de calcita, especialmente no momento do crepúsculo
-- para determinar a posição do Sol e também associadas a bastões para
demarcações de ângulos e horários solares -- como por exemplo um
pingente nórdico feito de osso da Estônia, datado do século~\versal{XI}, com
forma alongada e repletos de sinalizações circulares.

\SIG{Johnni Langer}

Ver também Astronomia; Embarcações; Mar Báltico; Navegação marítima;
Oseberg; Pedra solar; Sagas do Atlântico Norte.

\begin{itemize}
\item \versal{BERNÁRTH}, Bálazs \emph{et al}. An alternative interpretation of the
Viking sundial artefact: an instrument to determine latitude and local
noon. \emph{Proceedings of the Royal Society} 469, 2013, pp. 01-16.

\item \versal{BERNÁRTH}, Bálazs \emph{et al.} How could the Viking Sun compass be used
with sunstones before and after sunset? \emph{Proceedings of the Royal
Society} 470, 2014, pp. 01-18.

\item \versal{COWHAM}, Mike. The viking sun compass. \emph{Bulletin of the Scientific
Instrument Society}, 2007, pp. 01-05.

\item \versal{INDRUSZEWSKI}, George \& \versal{GODAL}, Jon. Maritime skills and astronomical
knowledge in the Viking Age Baltic Sea. \emph{Studia Mythologica
Slavica} 9, 2006, pp. 15-39.

\item \versal{STANISLAWSKI}, Blazej. Dysk drewniany z Wolina jako kompas sloneczny.
\emph{Materialy Zachodniopomorskie} 46, 2000, pp. 157-176.

\item \versal{VEBAEK}, Christen Leif; \versal{THIRSLUND}, Søren. \emph{The Viking compass:
guided Norsemen first to America}. Denmark: Humlebæk, 2002.
\end{itemize}


\chapter{C}
\section{\versal{CAÇA}}

Desde a mudança de hábitos caçadores-coletores para a agricultura e o
pastoreio ostensivamente desde o final da Idade do Bronze, os grupos
humanos instalados na Escandinávia, de maneira geral, praticavam pouco
as atividades de caça. Essa regra foi mantida na Era Viking, mas havia
exceções a depender da região e do contexto ambiental. O consumo de
carne de animais domésticos era muito maior do que a selvagem,
desaparecendo na dieta do sul da Escandinávia após o período das
migrações germânicas. Mas nas regiões ao norte da Noruega, Suécia e
grande parte da Finlândia, ao contrário, a carne provinda da 
caça foi bem mais recorrente do que a doméstica. Por ali, renas, alces,
cervos e lebres eram muito caçados. Na área dinamarquesa da Inglaterra 
existem indícios de veados vermelhos (\emph{Cervus elaphus}) perseguidos
pelos nórdicos e também aves silvestres, como a tarambola dourada
(\emph{Pluvialis apricaria}). Em Dublin se consumia o ganso selvagem
(\emph{Anser anser}).

Além da carne, os animais forneciam possibilidade para a extração ou
fabricação de diversos produtos. Cervídeos eram utilizados para obtenção
de pele e chifres, com os quais eram fabricados pentes, colares, objetos
de uso doméstico, acessórios para roupas e equipamentos. Esquilos e
raposas eram os animais mais procurados para o comércio internacional de
pele no norte europeu e possivelmente eram também consumidos.

Alguns animais possuíam um valor simbólico, religioso e marcial em
caçadas, como os ursos (\emph{Ursus arctos arctos}). Em Frösö (Suécia),
foram encontrados indícios de banquetes rituais (\emph{blóts}) com
vários tipos de ossos de animais, como ursos marrons, alces e
javalis. Não existem evidências de como esses animais foram mortos, mas
provavelmente morreram em seu ambiente natural e foram transportados
para a área de culto. Os pesquisadores acreditam que houve
influência dos cultos sámi, mas ao contrário destes, os ossos de ursos
foram encontrados mesclados e não ordenados em uma pilha e em posição
ereta. A pele de urso era importante para o \emph{status} social da elite
governante e era especialmente valiosa para o comércio no norte europeu.
Por sua vez, o consumo da carne e sangue de urso envolvia simbolismos
relacionados diretamente com o mundo do caçador e a guerra -- no mundo
nórdico (citado na \emph{Gesta Danorum} de Saxo Grammaticus) e em várias
culturas euroasiáticas existem exemplos disso: os heróis matam o animal
e bebem seu sangue ou comem seu coração, sendo impregnados de valentia,
poder e força.

As fontes apontam dois tipos de caçadores de urso no mundo nórdico:  
havia os que perseguiam os animais no intuito de obter a sua pele
para a venda e distribuição no comércio, e havia também uma elite marcial e aristocrática, 
que capturava os ursos com finalidades
simbólicas específicas ou para treinamento militar. A caça aos ursos era também 
conectada à iniciação de jovens guerreiros: para ser admitido no grupo,
o novato deveria matar o animal sozinho. Na literatura e iconografia, a morte do
urso por um herói ganhava dimensão miraculosa e sobrenatural -- o
mais apropriado inimigo para testar as habilidades e a competência do
guerreiro. O método tradicional de matar ursídeos era uma lança,
arma conectada a Odin durante a Era Viking. Alguns monumentos
pré-cristãos da área dinamarquesa na Inglaterra (a exemplo de hogback de
Branpton, século~\versal{X} d.C.), estão repletos de símbolos odínicos como a
triquetra e cercados por ursos. Em uma ponta de lança da sepultura~\versal{XII}
de Vendel Kyrka (Suécia), foram esculpidas as figuras de dois ursos 
de cada lado da lâmina. Apesar de não refletir um padrão real durante as
caçadas, a morte desse animal também está ligada ao uso de espadas,
totalmente idealizado, como na imagem sueca da plaqueta de Torslunda e
na \emph{Grettis saga} 21, onde o herói Grettir mata um urso com sua
espada. O encontro de restos ursídeos em sepultamentos demonstra
também o seu uso como troféus de caça e insígnias do herói. E para o
pesquisador Sigmund Oerhrl, o encontro de ossos em sepulturas femininas
sugere que eram utilizadas como símbolos da dignidade da família -- o
marido teria depositado seu maior troféu no túmulo de sua esposa.

Outros animais também eram caçados pela elite nórdica com fins marciais
e aristocráticos, como a prática da falcoaria e a perseguição de cervídeos
e javalis utilizando cães. Em recente pesquisa, Karyn Bellamy-Dagneau
concluiu que a prática da falcoaria nórdica, desde os tempos da idade do
Bronze, esteve conectada fortemente a questões de espiritualidade e
narrativas míticas de certas aves, como o xamanismo. Em sepulturas da
elite nórdica, remanescentes de ossos de certas aves indicariam a
conexão entre ideologia social e religiosidade.

\SIG{Johnni Langer}

Ver também Comércio; Finlândia da Era Viking; Noruega da Era Viking;
Religião; Sociedade.

\begin{itemize}
\item \versal{BELLAMY}-\versal{DAGNEAU}, Karyn. \emph{A falconer´s ritual: A study of the
cognitive and spiritual dimensions of pre-Christian Scandinavian
Falconry}. Dissertação de Mestrado na Universidade da Islândia, 2015.

\item \versal{KELLER}, Christian. Furs, Fish and Ivory - Medieval Norsemen at the
Arctic Fringe. \emph{Journal of the North Atlantic}, vol. 1, n. 23, 2010, pp.
01-23.

\item \versal{MAGNELL}, Olga \& \versal{IREGREN}, Elisabeth. Veitstu hvé blóta skal? The Old
Norse blót in the light of osteological remains from Froso church,
Jamtland, Sweden. \emph{Current Swedish Archaeology}, vol. 18, 2010, pp.
223-250.

\item \versal{OEHRL}, Sigmund. Bear hunting and its ideological context. In: \versal{GRIMM},
Oliver \& \versal{SCHMOLCKE}, Ulrich (orgs.). \emph{Hunting in northern Europe
until 1500 \versal{AD}}. Schleswig: Wachholtz, 2013, pp. 297-332.
\end{itemize}

\section{\versal{CALENDÁRIO E CONTAGEM DO TEMPO}}

\emph{Aspectos gerais}: Para determinações de tempo, sazonalidade,
contagem de tempo, os germanos antigos e os nórdicos medievais
utilizavam principalmente referenciais astronômicos para a elaboração 
de calendários (a exemplo de vários povos) que pudessem auxiliá-los a
determinar as épocas para a realização de cultos, plantações e
colheitas, guerra, atividades políticas e institucionais etc. Os
principais astros utilizados para a demarcação de calendários eram o
Sol, a Lua e as estrelas. O ano era determinado pelo curso do Sol, o mês
pela Lua e o início do ano pela posição de certas estrelas.

\emph{Astronomia e sazonalidade no mundo germânico antigo}: Os fenômenos
celestes eram parte importante da vida nas comunidades europeias da
Antiguidade. Para os povos neolíticos, germanos, celtas,
eslavos e habitantes do Mediterrâneo pré-clássico, o céu
propiciava não só a regulamentação do calendário (com os movimentos do Sol e
Lua) e da sazonalidade agrícola (determinação da época exata de plantar
e colher pelo avistar de certas constelações), mas também a projeção de
mitos produzidos pelo referencial cultural (as mitologias celestes e as
cosmogonias). Também os medos escatológicos eram associados com
fenômenos desconhecidos ou não previsíveis (como passagens de cometas, a
visão de eclipses ou fenômenos atmosféricos) e transformados em mitos. 
Alguns rituais eram executados de acordo com o calendário astronômico,
relacionados com os movimentos do Sol e da Lua e também
investidos de significados simbólicos. Assim como outros povos, os
germanos antigos tiveram grande interesse pela astronomia -- não no
referencial moderno, obviamente, mas por meio da visualização a olho nu
de fenômenos celestes considerados importantes para a vida
cotidiana e revestidos de sentidos mítico-religiosos. Apesar dos registros não serem
detalhados ou tão elaborados como os realizados após a
cristianização (fundindo-se com a tradição astronômica clássica da
Europa continental e a originada no Oriente), existem algumas fontes que
apontam para isso. Tácito mencionou que as atividades políticas e o
calendário germânico foram baseados no ciclo lunar (\emph{Germânia} 11).
Júlio César afirmou que os germanos não realizavam batalhas antes da Lua
nova (\emph{Comentários da guerra gálica} 50). Jordanes enunciou que os
antigos Godos tinham conhecimentos de constelações e do movimento de
planetas e estrelas (\emph{Sobre a origem e feito dos Godos} 10).

\emph{Arqueoastronomia e registros solares e lunares}: Mas os mais
surpreendentes registros são provenientes da Arqueologia. Em 1999 foi
descoberto na Alemanha o disco de Nebra, datado de 1700 a.C. Trata-se de
um disco de bronze contendo as figurações do Sol, da Lua e de dois arcos
laterais, além de várias estrelas. Um dos arcos é interpretado como
sendo uma barca solar, um mito comum a várias culturas do Ocidente. 
Um achado análogo é o carro solar de Trundholm, Dinamarca, datado de 1400 a.C.
Essas relíquias arqueológicas representam um
dos momentos fundamentais da cosmologia antiga: a jornada simbólica dos
astros pelos vários mundos, especialmente o dos mortos. Além disso,
o disco de Nebra também registra as Plêiades -- um dos mais importantes asterismos
do céu, demarcadora das épocas de colheita na Europa. Os temas da
barca solar e das Plêiades vêm sendo identificadas também em
diversos sítios arqueológicos de arte rupestre na Suécia da Idade do Bronze, como
apontadas pelo astrônomo Göran Henriksson.

Dois sítios nórdicos estão apresentando antigas orientações solares:
Ales e Tysnes. Os megálitos suecos de Ales, com formato de navio e
datação incerta (Idade do Bronze Tardia ou do Ferro), foram estudados
por Mörner e Lind e considerados como um sofisticado calendário solar
dos solstícios de verão e inverno, as duas datas mais importantes do
calendário religioso da Europa pré-cristã. O pilar cerimonial de
Tysnes, Noruega (Idade do Ferro Tardia) foi encontrado entre 
vestígios religiosos e associado toponimicamente aos deuses
germânicos desde o início do século \versal{XX}. Durante o período do solstício
de inverno, a luz solar incide diretamente sobre seu topo, iluminando o monólito. O
fenômeno foi constatado visualmente pela primeira vez pelo pesquisador Eldar Heide e
possivelmente o efeito tinha caráter intencional, mas
ainda faltam medições geoastronômicas pormenorizadas nesse local.
Infelizmente, a quantidade de investigações de campo e pesquisadores em
Arqueoastronomia na Escandinávia ainda é muito reduzida.

Segundo Rudolf Simek e Régis Boyer, existem muitas evidências de culto
ao Sol na Idade do Bronze, evidenciados pelo grande número de
grafismos rupestres e do disco da carroça de Trundholm. No
\emph{Encantamento de Merseburg}, a deusa Sunna é citada como irmã de
Sinthgun, mas Simek acredita que a combinação dos antigos símbolos
solares com o navio nos contextos ritualísticos (que ocorrem
frequentemente da Idade do Bronze aos tempos medievais), parecem estar
conectados a cultos de deuses da fertilidade (como Njórd e Freyr, mas
que não possuem conexões diretas com personificações solares). Em 1936
Vilhelm Kiil argumentou que o nome \emph{Solberg} significava montanha
do sol, evidenciando algum tipo de culto solar na Escandinávia. Em 1981
o francês Régis Boyer realizou um extenso estudo sobre o simbolismo dos
mitos solares na Idade do Bronze da Escandinávia, inseridos em sua obra
\emph{Yggdrasill: La religion des anciens scandinaves}. Algumas das
principais pinturas de Bohuslän analisadas por Boyer, embarcações
transportando discos (relacionadas a procissões e rituais solares),
foram analisadas pelo astrônomo Göran Henriksson em 1996, sendo
associadas a eclipses totais do Sol na região.

A Lua também aparece nos registros arqueoastronômicos,
confirmando os relatos de Tácito e Júlio César. Göran Henriksson
identificou marcações em sepulturas da ilha de Gotland que pressupõem 
registros lunares (um possível calendário), indicando fases da Lua nova
ou cheia durante o solstício de inverno. E o arqueólogo Mike
Parker-Pearson comparou diversos sítios arqueológicos da Idade do Ferro em áreas
germânicas e nórdicas que possuem alinhamentos voltados para eclipses
totais da Lua durante o solstício de inverno, demonstrando observações e
registros desses fenômenos. Em recente publicação, o historiador Dorian
Knight analisou o episódio de Odin e Gunnlod no \emph{Hávamál} como
sendo uma descrição do ciclo lunar, com resultados surpreendentes. Em
síntese, a pesquisa de Knight conclui que a descrição do relacionamento
fracassado de Odin com a filha do gigante Billing (\emph{Hávamál}
96-102) corresponde à fase da Lua cheia para nova: o astro possui
ligações simbólicas com o feminino; o cachorro, no final do relato, é uma
simbolização da morte, do outro mundo e da escuridão do disco (Lua
nova), transfigurados no medo do desaparecimento da Lua, devorada por canídeos.
A narrativa triunfante de Odin acasalando com
Gunnlod (\emph{Hávamál} 103-110), por sua vez, corresponde à fase da
Lua nova à Lua cheia. Nesse caso, a interpretação de Knight leva em
conta também o simbolismo do hidromel associado à Lua cheia,
conhecido no folclore por lua de mel (conexão entre casamento e
fertilidade).

\emph{Ano estelar}: Segundo o pesquisador Otto Sigfrid Reuter, os
nórdicos pré-cristãos demarcavam o início do ano quando o aglomerado das
Plêiades (constelação do Touro) era visível. Também denominadas de Sete
Estrelas no medievo tardio, na área nórdica o folclore as associava a 
galinhas e pintos. Na reconstituição do programa \emph{Stellarium}
0.14.3 (dados para Estocolmo) as Plêiades foram visíveis na direção
leste no dia 22 de agosto de 901~d.C., a partir das 18:15 horas. 
Outras estrelas que podem ter sido utilizadas como demarcadoras
do ano são as que compõem o cinturão de Órion (conhecidas como Três Pescadores ou a
Roca de Frigg).

\emph{O ano lunar entre os germanos antigos}: O cronista Beda relatou
que os anglo-saxões utilizavam um calendário baseado na Lua, com 13
meses, dividido entre duas metades, verão e inverno. O início do inverno
se dava na Lua cheia (correspondente ao mês de outubro do calendário
juliano) denominada como Winterfyllith. Nas ilhas Faroé também se
utilizava um calendário lunar, influenciado pelos noruegueses.

\emph{Calendário germânico antigo}: Os povos germânicos desenvolveram
quatro tipos básicos de calendários, com vários formatos. O ano solar
geralmente possuía 360 dias, mais cinco dias extras, baseado na
observação e não intercalação (Noruega do século~\versal{VI}~d.C.); 360 dias eram
divididos geralmente em 12 meses de trinta noites e provavelmente
originaram os 4 dias extras da Noruega (Islândia, século~\versal{X}); 364(5) dias
divididos em 13 meses de 28 noites (Islândia, século~\versal{X}); 354 dias com
semanas intercaladas (Islândia, século~\versal{X}). O ano lunar também foi usado
na Noruega, Suécia, Dinamarca e ilhas Faroé e era intercalado com alguns
meses, mas já era obsoleto na Islândia do século~\versal{X}, onde somente
utilizava-se o ano solar.

Evidências de uso de calendário lunisolar existem nas \emph{Eddas}, como
no poema \emph{Vafþrúðnismál} 23, que menciona a contagem da idade
de um homem pelo movimento do Sol e da Lua pelo céu, bem como aponta 
que as fases da Lua também serviam para contar a idade de uma
pessoa. No \emph{Alvísmál} 14 a Lua também é mencionada como
determinadora de idade.

Em 2006, o pesquisador Andreas Nordberg reconstituiu o antigo sistema
lunisolar dos nórdicos, estabelecendo que os meses começassem na Lua nova
e que a Lua cheia tivesse início na metade do mês; o próximo mês
se iniciava na próxima Lua nova. Devido ao movimento anual do Sol e da Lua,
esse calendário deveria ser regulado frequentemente. Assim, durante o
período do \emph{Jól} (Yule) existiriam dois meses lunares. A nova Lua
do segundo mês do Yule ocorreria onze dias após o solstício de inverno.
O décimo terceiro mês lunar era acrescentado treze dias após o solstício
de verão. O mês bissexto era acrescentado a cada três anos. Para
Nordberg, a \emph{vetrnætr} (noites de inverno, o período dos primeiros
três dias de inverno) teria início em 20 de outubro; o solstício de
inverno em 21 de dezembro; o verão em 20 de abril e o \emph{midsommar}
em 21 de julho. A polêmica envolve saber exatamente a época em que era
realizado o Yule, se coincidia ou não com a data do solstício de inverno
(como menciona Snorri Sturluson na \emph{Hákonar saga Góða}). No período
das \emph{vetrnætr,} as colheitas haviam terminado, o dia se tornava
menos luminoso e o frio tinha início. As interpretações sobre o ritual
do Yule variam entre as hipóteses de uma festa solar, um ritual para os mortos ou para fertilidade.
Segundo Andreas Nordberg, os rituais sazonais refletiriam um drama
essencialmente cosmológico, apelando para que a natureza tivesse seu
curso de repetição e criação garantidos.

\emph{Calendário islandês antigo}: O conceito antigo de ano era
denominado de \emph{ár} e envolvia noções de fertilidade e abundância,
estreitamente relacionadas à vida rural. A produção do ano não requeria
apenas a colheita, mas também leis, que eram determinadas pela
assembleia (\emph{þing}), quando o
calendário do próximo ano era fixado. A divisão do ano era feita em dois
semestres (\emph{misseri}): inverno (\emph{vetur}) e verão
(\emph{sumar}). No verão se produzia e o inverno era o momento
de consumo. Segundo Terry Gunnell essa divisão sazonal tinha implicações
de gênero: enquanto o inverno era dominado pela mulher, pela magia e a
morte, o verão era dominado pelo homem, pelo comércio e a guerra.

A transição dessas épocas era marcada socialmente. Até hoje, os
islandeses celebram o primeiro dia de verão (frequentemente em abril) e
na Noruega o dia 14 de abril é chamado de \emph{sommermál} (medida do
verão). A natureza dicotômica do ano é refletida nos nomes dos meses. No
verão, os meses recebem alcunhas relacionadas a atividades econômicas
(feno, semeadura, ovelha). No inverno, recebem nomes de atividades não
econômicas (\emph{Þorri}, \emph{gói,} termos com origens obscuras no
calendário pré-histórico). No verão, muitas atividades têm relação com a
produção e no inverno os rituais têm mais importância que a economia.

O dia era marcado pelo movimento do Sol
(\emph{sólarhringr}). Além da divisão em dia e noite, o
\emph{sólarhringr} era dividido em oito partes, relacionadas com os
pontos cardeais, cada um tendo um nome, como \emph{miðdegi} (meio-dia),
\emph{miðrmorgun} (hora de acordar), \emph{dagmál} (hora da refeição).
De um ponto de vista moderno, os meses não são indicadores precisos de
tempo, mas o são para a perspectiva do medievo. Genericamente, o tempo
medieval escandinavo era medido pela pessoa no centro de seu próprio
mundo. O tempo do dia era medido pela posição do Sol no horizonte ou
pela refeição que deveria ser feita. O tempo do ano era medido de acordo
com as atividades sociais, definidas pelas pessoas. A escala era tanto
qualitativa quanto quantitativa. Mas também o tempo era algo sagrado
para o mundo pré-cristão e relacionado diretamente com as atividades
religiosas.

As semanas eram um dos métodos tradicionais de contar o tempo. Eram
usadas para medir intervalos de tempos e para especificar
datas. Não havia um único dia fixo iniciando as semanas. Os meses tinham
pouca importância. Eles foram claramente definidos em \emph{Grágás} 18,
mas são pouco citados na literatura nórdica medieval e alguns nem eram
utilizados na prática. A exceção fica para \emph{Þorri} e \emph{gói.} Em
930 os islandeses resolveram estabilizar o \emph{þing.} Levaram cinco
semanas para determinar um registro preciso do tempo, levando à reforma
do calendário em 955, registrado no \emph{Íslendingabók}, com um
calendário anual com 52 semanas.

\SIG{Johnni Langer}

Ver também Astronomia; Bússola Solar; Navegação; Pedras solares;
Religião.

\begin{itemize}
\item \versal{HASTRUP}, Kirsten. Calendar and time reckoning. In: \versal{PULSIANO}, Philip
(ed.). \emph{Medieval Scandinavia: An Encyclopedia}. New York and
London: Garland, 1993, pp. 65-66.

\item \versal{JANSSON}, Svante. The icelandic calendar. \emph{Scripta Islandica} 62.
Uppsala: Isländska Sällskapets årsbok, 2011, pp. 51-104.

\item \versal{NORDBERG}, Andreas. \emph{Jul, disting och förkyrklig tideräkning:
Kalendrar och kalendariska riteri det förkristna Norden}. Uppsala: Kungl.
Gustav Adolfs Akademien, 2006.

\item \versal{POWELL}, Avery. \emph{Primstav and Apocalypse: Time and its Reckoning in
Medieval Scandinavia}. Universidade de Oslo, Dissertação de Mestrado em
Cultura Nórdica Viking e Medieval, 2011.

\item \versal{REUTER}, Otto Sigfrid. Skylore of the North. \emph{Stonehenge Viewpoint}
47-50, 1982.

\item \versal{VILHJÁLMSSON}, Þorsteinn. Time-reckoning in Iceland before literacy. In:
\versal{RUGGLES}, Clive (ed.). \emph{Archaeoastronomy in the 1990s}. London:
Loughborough, 1991, pp. 69-76.
\end{itemize}

\section{\versal{CELTAS E NÓRDICOS}}

A relação entre grupos convenientemente chamados pela alcunha de celtas
e as comunidades nórdicas é no mínimo complexa. Afinal, os dois grupos
possuem uma diversidade cultural sem precedentes e sua interação alterna
entre momentos de conflito e de total integração nos campos político,
cultural e econômico.

Para compreender efetivamente a complexidade dessas interações é
necessário antes se entender quem são os celtas, muitas vezes confundidos aos nórdicos
por grande parte do senso comum, quando na verdade os dois grupos
compõem culturas distintas com grande variedade e rivalidades entre si.
Entre os grupos chamados de celtas essa variação é ainda maior, formando
tradições culturais diversas entre as populações da Idade do Ferro e
aqueles que ainda hoje advogam algum pertencimento étnico céltico
contemporâneo.

O termo celta deve ser compreendido, portanto, como uma tentativa
generalizante de classificar diferentes realidades culturais definidas
dentro de um grupo étnico. O termo tem certo respaldo histórico, mas
suas diversas apropriações ao longo da História o transformaram em um
termo genérico de difícil compreensão.

Segundo a tradição histórica da Antiguidade, as populações europeias ao
norte dos Alpes passaram a ser denominadas genericamente como
\emph{keltoi}, a primeira forma encontrada historicamente para
definir os celtas. Os antigos helenos passaram a designar toda a
população ao norte dos Alpes como \emph{keltoi}, termo que aparece
em autores como Heródoto em suas \emph{Histórias} já no século~\versal{V}~a.C., mas
também tem relação com os textos presentes em Estrabão, Posidônio,
Atheneu e Diodoro Siculus na Antiguidade entre os séculos~\versal{III}~e~\versal{I}~a.C. Essa
visão promovida pelo mundo clássico grego e romano é quase contemporânea
na maioria dos casos e sua influência sobre o que se entende por celtas
dois mil anos depois de sua escrita ainda é digna de nota.

A grande questão gira em torno de apurar a ligação entre o termo ``celta'',
tão abertamente utilizado hoje, com as realidades dessas populações da Antiguidade,
já que os próprios celtas não deixaram nenhum vestígio textual que
efetivamente elucide esse tópico. O que resta à análise histórica é 
justamente a cultura material e os textos de autores gregos e romanos.
As informações sobre os celtas oferecidas pelos autores antigos não seguem o
mesmo padrão, já que esses autores escreveram em
diferentes épocas e com diferentes interesses. O único ponto de
articulação entre os termos é que todos se referiam aos povos bárbaros,
fato preponderante na associação incorreta que o senso
comum fará entre as distintas culturas célticas e nórdicas já na Idade
Média.

Mesmo assim, autores que redigiram documentos versando diretamente sobre os celtas, como
Políbio, Posidônio e Júlio Cesar (ao escrever seu \emph{Comentário sobre a
Guerra das Gálias}) possuem definições diferentes sobre o que seriam os
celtas, os gauleses, os gálatas etc. Esses termos que evocam certa
identificação geográfica e encontram assimilações diferentes entre
os autores clássicos gregos e romanos, o que gera um controverso debate
histórico.

Segundo Barry Cunliffe, essa diferenciação de nomes se explicaria pela 
uma análise diferenciada de cada autor. Em termos
gerais, \emph{keltoi/celtæ} denominariam grupos de regiões mais afastadas,
e \emph{galli/galatae} seria um
termo mais específico, utilizado para designar grupos com os
quais os gregos e romanos mantinham mais contato, mais especificamente as tribos que
interagiam diretamente com o mundo mediterrâneo na região da Gália e da
Ásia Menor, respectivamente.

Por meio destas fontes percebe-se que não existia uma unidade em torno do termo 
``celta'' que permitisse o uso modernamente a ele atribuído. A 
proposição de que a língua seria o fator de articulação entre esses grupos
na antiguidade é também questionável, por falta de registros que a
comprovem.

Dessa maneira, por mais que exista uma correlação cultural e mais
especificamente linguística entre os grupos que mais tarde serão
denominados como irlandeses, escoceses, galeses e bretões (franceses), é
impossível afirmar que existisse uma unidade cultural entre esses grupos
na Antiguidade. No entanto, pode-se inferir, por meio das fontes
medievais, que nas regiões onde a presença de grupos nórdicos se fez
presente, uma certa diferenciação de caráter étnico começa a ocorrer,
sobretudo a partir do século \versal{VIII}.

As populações de irlandeses e escoceses, por exemplo, passam a se
identificar nessa época como ``gaélicos'', nome derivado da palavra
\emph{goídel} e que na verdade é emprestada da palavra galesa
\emph{gwyddel}, que significa, de forma pejorativa, algo como
``selvagem''. O termo ganha força para esses grupos justamente a partir
do século \versal{VIII}, data de seu primeiro aparecimento registrado nos anais.

É a partir de então que o termo ganha contornos étnicos bem claros, sobretudo para
contrastar com os nórdicos invasores que aos poucos ganhavam posições e
se assentavam em seus territórios. Os vikings ocupavam nessa época
regiões da Escócia, da Ilha de Man e Irlanda, introduzindo muito de seus
costumes; ao longo dos séculos de ocupação, foram mesclando-se à população
local e formando grupos designados como gáelicos-nórdicos
(\emph{Norse-Gaels}).

Este processo pode se constatar de maneira clara entre os séculos \versal{IX} e
\versal{XII}, quando na chamada Irlanda da Era Viking e na Escócia Escandinava,
populações nórdicas, identificadas popularmente como vikings, passaram a
atacar e aos poucos se assentar nos territórios gaélicos.

A língua dos invasores e, posteriormente, colonizadores escandinavos foi
aos poucos se adaptando à região, e a interação entre os povos, que nos primeiros anos
das invasões era praticamente inexistente, foi se tornando aos poucos possível e
depois extremamente efetiva, ganhando uma realidade
sólida por meio de alianças políticas, culturais e econômicas.

Os invasores que originalmente eram de origem norueguesa e conhecidos
pela alcunha de \emph{fingaill} (belos estrangeiros), em oposição aos
dinamarqueses que eram minoria e conhecidos como \emph{dubhgaill}
(estrangeiros negros), vão aos poucos se gaelicizando de tal maneira que
gradativamente não mais identificam-se como grupos estrangeiros
simplesmente.

Em meados do século~\versal{IX} pode-se claramente dizer que existe
um grupo identificado como gaélico-nórdico. A partir desse momento 
seus assentamentos ficam cada vez melhor estruturados e muitos deles se
tornam referenciais políticos nos territórios dominados por prosperarem
sobretudo no campo econômico, como será o caso de regiões como Dublin,
Limerick, Wexford, Man ou mesmo Galloway.

Estes grupos se mesclaram não apenas por meio do intercâmbio econômico,
mas também por meio de casamentos, conversão ao cristianismo e demais
vivências cotidianas que ao longo do tempo deixaram nessas regiões vestígios
linguísticos e culturais que comprovam esse processo.

Palavras ligadas a questões e afazeres nórdicos são comumente uma
herança dessa época, principalmente as relacionadas ao comércio, pesca e
demais assuntos marítimos. Também destacam-se os diversos nomes de
clãs escoceses de origem nórdica, como MacLeod, MacDonald ou MacÍomhair. Para além da
questão linguística, a própria geopolítica destas regiões foi modificada
por esses grupos, que após três séculos de ocupação nórdica muda seu
eixo e suas possíveis trocas e conexões para o mar irlandês.

Nesse sentido, a relação entre grupos celtas e nórdicos se transforma de
um primeiro momento belicoso e de disputa para uma total integração
entre os grupos, que mesclaram-se de tal maneira que tornou-se 
impossível distingui-los entre si e ressaltar suas diferenças, visto que os nórdicos são
totalmente assimilados pelas culturas a que se assentaram, não apenas
trazendo práticas novas, mas adotando os estilos de vida e participando
ativamente das interações políticas e sociais destas regiões.

\SIG{Erick Carvalho de Mello}

Ver também Brian Boru; Dublin; Irlanda da Era Viking.

\begin{itemize}
\item \versal{BRUNAUX}, Jean-Louis. \emph{Les Celtes: Histoire d'un mythe}. Paris:
Éditions Belin, 2015.

\item \versal{CUNLIFFE}, Barry. \emph{The Ancient Celts}. London: Penguin Books, 1997.

\item \versal{DOWNHAM}, Clare. \emph{Viking Kings of Britain and Ireland}. Edinburgh:
Dunedin Academic Press, 2007.

\item \versal{DOWNHAM}, Clare. \emph{``Hiberno-Norwegians'' and ``Anglo-Danes'':
Anachronistic ethnicities in Viking-Age England}. Mediaeval Scandinavia,
vol. 19, 2009, pp. 139-169.

\item \versal{PAOR}, Liam de. The Age of the Viking Wars: 9\textsuperscript{th} and
10\textsuperscript{th} centuries\emph{.} In: \versal{MOODY}, Theodore \versal{W}. \&
\versal{MARTIN}, Francis \versal{X}. \emph{The Course of Irish History}. Cork: Mercier
Press, 2011, pp. 91-106.
\end{itemize}

\section{\versal{CEMITÉRIO DE BORRE}}

Borre se localiza atualmente na parte norte do condado de Vestfold, 30
km ao norte de Sandefjord. Foram detectados, no cemitério, 40
montes funerários ainda preservados; sete deles são visíveis do alto do fiórde da região devido ao seu
tamanho, que varia entre 30 e 45 m de diâmetro e 5 a 7 m. de altura. Sabe-se que
pelo menos três montes similares aos encontrados na atualidade foram
demolidos durante o século \versal{XIX}, fator que leva Borre a ser apontado como localidade
que abriga o maior grupo de montes funerários monumentais achados na Noruega.

O cemitério da região foi por muito tempo apontado como local de
depósito de alguns reis citados na \emph{Ynglinga Saga}, parte inicial
da obra \emph{Heimskringla}, que narra a origem da linhagem
real apontada como responsável pela unificação do reino da
Noruega, fato dito como acontecido durante o reinado de Haroldo Cabelos Belos.
É dito, portanto, que os reis Oystein Fret e Halvdan Kvitbein teriam 
seus depósitos mortais abrigados no cemitério aqui relatado.

Devido a sua relação com a linhagem dos Ynglingos construída na saga 
supramencionada, o cemitério de Borre tornou-se uma ferramenta de
territorialização da história que seria construída no século~\versal{XIX} e \versal{XX}
sobre a construção da nação norueguesa. A Noruega buscava, no século~\versal{XIX}, 
sua independência em relação ao reino da Suécia. No século seguinte, durante a Segunda
Guerra Mundial, a nação sofreria pressões de domínio nazista, sendo 
também terreno utilizado pelo partido de ultradireita
denominado Nasjonal Samlig, que durante os anos de 1940 a 1945 alçou uma 
construção de superioridade racial que demarcaria para sempre a história
dos noruegueses em relação a suas interpretações historiográficas e
construções ideológicas provenientes da utilização das fontes do período
viking. Isso levou a apontamentos críticos que, a partir dos anos 1990,
passaram a demandar maior historicidade das fontes, tanto a respeito do cemitério de Borre quanto a
respeito da própria \emph{Ynglinga Saga}, indicando também a impossibilidade de
confirmação de Borre como local de deposito dos Ynglingos.

\SIG{Munir Lutfe Ayoub}

Ver também Arqueologia da Era Viking; Cultura material; Funerais e
enterros; Sepultamentos.

\begin{itemize}
\item \versal{MYHRE}, Bjorn. Agrarian development, settlement history, and social
organization in southwest Norway in the Iron Age. \emph{New directions
in Scandinavian archaeology}, vol. 1, 1978, pp. 224-272.

\item \versal{MYHRE}, Bjorn. The Royal Cemetery at Borre, Vestfold: a Norway centre in
a European periphery. In: \versal{CARVER}, Martin (ed.). \emph{The Age of Sutton
Hoo}. Woodbridge: Boydell Press, 1992, pp. 301-314.

\item \versal{MYHRE}, Bjorn. The early viking age in Norway.~\emph{Acta Archaeologica},
vol. 71, n. 1, 2000, pp. 35-47.

\item \versal{MYHRE}, Bjorn. The Significance of Borre. In: \versal{FLADMARK}, Jan Magnus (ed.).
\emph{Heritage \& Identity; Shaping the Nations of the North}. New York:
Routledge, 2002, pp. 19-34.
\end{itemize}

\section{\versal{CERCOS DE PARIS (845, 885)}}

Das várias cidades atacadas, saqueadas e destruídas pelos vikings, Paris
tem uma importância especial. A icônica capital francesa que se tornou
conhecida por seus monumentos, museus, palácios, cafeterias e as luzes
do Iluminismo era, no século~\versal{IX}, uma cidade bem mais modesta, limitada a
Île de la Cité, em meio ao Sena. Uma cidade murada e insular, que 
alguns consideravam inexpugnável e acabou sendo alvo de vários ataques
vikings, pois se acreditava que a capital da Francia guardasse, atrás de
suas altas muralhas, grandes tesouros.

Antes de Paris ser o alvo das expedições, desde 799 ataques vikings
ocorriam esporadicamente a Francia. No entanto, na década de 830, os
ataques se intensificaram e cidades como Dorestad, Ruão, Saint Denis e
Quentowic foram saqueadas. Em 843 foi a vez de Nantes. 
No ano de 845, dois importantes ataques ocorreram no continente: o rei
Horik~\versal{I} da Dinamarca ordenou a invasão de Hamburgo (Alemanha) e, no mesmo
ano, um chefe chamado Ragnar atacou Paris. Algumas crônicas
posteriores a essa época sugerem que se tratava do lendário Ragnar
Lothbrok, embora não se possua certeza.

Acerca do primeiro ataque a Paris, Janet Nelson cita o relato de um
monge do monastério de Saint-Germain-des-Prés, onde consta que pelo
menos 120 navios cercaram Paris. Eram embarcações dos normandos (termo
usado pelos francos para ser referir aos vikings). Tais guerreiros
brutos forçaram a invasão da cidade. Para esse monge, aquele terrível
acontecimento era uma punição de Deus, pois o rei Luís, o Pio (778-840),
havia decidido em vida dividir o reino entre seus três filhos: Lotário,
Carlos e Luís, que entraram em guerra entre si, pois cada um queria se tornar senhor
e soberano de toda a Francia. Para o monge, essa contenda familiar teria irado Deus,
que teria permitido, punitivamente, o ataque daqueles bárbaros do norte.

Para evitar que a cidade sofresse maiores danos, o rei Carlos, o Calvo
(823-877), que governava desde 840 tendo Paris como capital, decidiu
oferecer uma proposta de rendição. Segundo os relatos da época, o rei
teria oferecido 7 mil libras de prata (cerca de 3 toneladas) para que os
vikings desistissem do ataque. Para alguns historiadores, tal pagamento
poderia ser considerado um \emph{danegeld}. Esse tributo de
extorsão começou a ser cobrado por essa época na França e Inglaterra. O
pagamento das 7 mil libras de prata assegurou que novos ataques à Paris
não ocorressem nos anos seguintes, embora expedições continuaram a
atacar outras localidades do reino.

O segundo ataque a Paris é controverso. Alguns historiadores apontam
falta de relatos históricos que o comprovem, pois os \emph{Anais
Francos} não são específicos se de fato a cidade foi ou não atacada. O
segundo ataque teria ocorrido entre 856-857, comandado supostamente por
Björn, Costas de Ferro. Inicialmente os nórdicos optaram em montar base
na ilha de Oscellus, dando preferência a atacar as povoações, fazendas e
mosteiros nos arredores de Paris antes de se aventurar no ataque à 
capital. Novas investidas lideradas por Weland à região do Sena se
sucederam até 859, quando em 860, o rei Carlos, o Calvo, ofereceu novo
\emph{danegeld}, ofertando 3 mil libras de prata. Seis anos depois, novos
chefes vikings retornaram a região do Sena e, para não atacar Paris,
cobraram outro \emph{danegeld}.

Todavia, um novo ataque a Paris somente ocorreria muitos anos depois, em
885. Após vários ataques às terras nos arredores de Paris,
facilitados pelo fato de o rio Sena ser um território sem controle à época, 
permitindo que os navios nórdicos navegassem livremente para saquear várias
localidades, o rei Carlos, o Gordo (839-888), que governava desde 884,
ordenou que fortificações fossem erguidas ao longo do Sena para barrar o
avanço dos invasores. No ano de 885, dois chefes de nome Siegfried e Gorm
solicitaram passagem segura pelo bloqueio do Sena, mas o rei Carlos
negou, acreditando se tratar de um engodo. De fato, ele estava certo.

Siegfried e Gorm ordenaram que o bloqueio formado por duas torres e uma
corrente defensiva que cruzava o rio fossem destruídos. Segundo relatos
da época, eles comandavam 700 navios. Desse modo, a frota invasora deu
início ao cerco de Paris. Todavia, além dessas defesas erguidas no rio,
o rei Carlos havia incumbido o conde Odo (c. 852-898) da liderança do
exército de Paris. Odo conseguiu com bravura e estratégia impedir que
Paris fosse invadida e saqueada novamente, embora isso não tenha como
resultado uma vitória plena, mas um novo acordo de trégua. Em
886, um novo \emph{danegeld} foi oferecido aos invasores, que por fim se
retiraram.

Odo se opôs à necessidade de pagar para que o inimigo partisse, e tampouco o pagamento 
cessou novas investidas vikings através do Sena. Em 888, o rei
Carlos, o Calvo, faleceu, e Odo assumiu o trono, dando continuidade ao
combate aos nórdicos e ordenando a construção de acampamentos e
fortificações ao longo do Sena. As medidas de Odo foram continuadas pelo
seu sucessor Carlos, o Simples (879-929). Paris manter-se-ia segura por
vários anos.

Na série \emph{Vikings}, escrita por Michael Hirst, e produzida pelo
History Channel, o cerco a Paris é retratado na terceira e quarta
temporadas, todavia os acontecimentos ali encenados são 
anacrônicos, pois mesclam os relatos das três expedições à Paris.
Na série, o conde Odo é assassinado, o que é outro desvio historiográfico,
já que ele se tornou rei depois da morte de Carlos, o
Gordo. Outra imprecisão apontada diz respeito ao fato de que o rei
apresentado na série trata-se de Carlos, o Simples, e não Carlos, o
Gordo. Além disso, a série indica que Rollo participou
dessas expedições e, depois da primeira, aceitou se aliar aos francos,
tornando-se duque da Normandia, quando historicamente isso só ocorreu em 911.


\SIG{Leandro Vilar Oliveira}

Ver também França da Era Viking; Normandia; Rollo; Viking; Vikings na
França.

\begin{itemize}
\item \versal{ARBMAN}, Holger. \emph{Os Vikings}. Lisboa: Editorial Verbo, 1967.

\item \versal{CALLMER}, Johan. Scandinavia and the continent in the Viking Age. In:
\versal{BRINK}, Stefan; \versal{PRICE}, Neil (eds.). \emph{The Viking World}. London/New
York: Routledge, 2008, pp. 439-452.

\item \versal{HAYWOOD}, John. \emph{Historical Atlas of Vikings}. London: The Penguin
Books, 1995.

\item \versal{LOGAN}, F. Donald. \emph{The Vikings in History}. London/New York:
Routledge, 1991.

\item \versal{NELSON}, Janet L. The Frankish Empire. In: \versal{SAWYER}, Peter (ed.). \emph{The
Oxford Illustrated History of the Vikings}. New York: Oxford University
Press, 1997, pp. 19-47.

\item \versal{STREISSGUTH}, Thomas. \emph{Life among the Vikings}. San Diego, \versal{CA}:
Lucent Books, 1999.
\end{itemize}


\section{\versal{CERVEJA}}

Uma das bebidas fermentadas mais antigas elaborada pelo ser humano, a cerveja é feita a 
partir de uma mistura simples de cereais dos mais diversos,
grosseiramente moídos, fervidos com a adição de algumas ervas para
conferir sabor e colocados para fermentar em algum local escuro e seco.
Essa é a receita mais antiga e simples para se fazer cerveja.

A bebida, muito apreciada pelos germânicos desde a Antiguidade, era
consumida por todos -- dos mais nobres aos mais pobres, das crianças,
até os idosos --, e todos os dias. A necessidade de se purificar a água,
vetor de várias doenças, estava intimamente ligada ao ato de se fazer e
consumir cerveja. Esse fermentado, que recebeu o nome genérico de cerveja,
era uma bebida essencial para uma sociedade que, diferentemente das
mediterrâneas, cultivavam a vinha e tinham no vinho a sua melhor e mais
consumida bebida. A cerveja garantia a potabilidade da água e 
a fermentação conferia uma dose extra de nutrientes à dieta.

Os primeiros relatos sobre esse fermentado, que inicialmente levava em
seu preparo somente ervas, água e leveduras naturais, remontam à
Pré-história. Tácito, na \emph{Germânia}, explica que os germanos
apreciavam consumir em grandes quantidades um fermentado à base de
cereais e ervas de sabor amargo e nada agradável se comparado ao 
vinho consumido pelos romanos. O fermentado de água com ervas 
não possuía grande durabilidade e era consumido em grande
quantidade, o que obrigava as mulheres a abastecerem suas casas com
cerveja diariamente.

A cerveja não era a única bebida fermentada consumida pelos germanos alto-medievais,
incluindo os escandinavos da Era Viking, que também produziam vinhos, hidromel e cidras.

\emph{Öl} é o nome genérico usado para toda bebida alcoólica, mas em
alguns casos refere-se especificamente à cerveja tipo ale. \emph{Bjórr} se
refere a cervejas mais fortes (por isso sua associação com os deuses
Aesir). \emph{Veig} e \emph{hreinalög} são termos para bebidas claras e 
frescas, e o hidromel (\emph{mjöð}) 
era o licor preferido no palácio do Valhala, a morada
de Odin. \emph{Sumbl} é o nome dado aos banquetes e está relacionado ao mito do
gigante Súttungr, que esconde o hidromel em uma montanha. A cerveja (\emph{bjórr}, em nórdico antigo) era
consumida em todas as refeições e também ao longo do dia, substituindo
em alguns momentos a própria água, que em determinadas regiões
apresentava altos índices de contaminação, pois era uma grande
disseminadora de doenças. Por ter um teor alcoólico baixo (algo em torno
de 3 a 5 graus), também oferecia calorias e certa dose de nutrientes. A
cerveja consumida pelos vikings e pelos anglo-saxões possuía
praticamente a mesma composição: cereais, água, levedura e ervas
aromatizadas -- que além de conferirem um sabor especial à bebida 
também tinham a função de conservantes. É preciso ressaltar que o lúpulo
(\emph{Humulus lupulus}), ingrediente indispensável na fabricação
da cerveja contemporânea, só começou a ser incorporado em larga escala
no século~\versal{XI}. A erva mais utilizada como aromatizante na fabricação das
cervejas alto-medievais era a \emph{Glechoma hederacea}, popularmente conhecida
como erva de São João ou hera-terrestre, de sabor amargo, 
rica em ácidos fenólicos e tanino, dois anti-oxidantes e
conservantes naturais, que em certa medida também conferem amargor à
bebida. Diferentemente de hoje a produção de
cerveja, vinho e outros fermentados, não se dava de forma industrial
e nem havia a excessiva preocupação com a excelência na qualidade e
seleção dos ingredientes como vemos atualmente. A produção de bebidas
era tarefa feminina por excelência. As mulheres deviam cuidar para que
as despensas estivessem sempre bem abastecidas de ingredientes tanto
para a elaboração da comida de todos os dias como também para as festas. A
cerveja produzida pelos nórdicos possuía um sabor e também coloração
diferentes das equivalentes atuais, já que não possuía conservantes e
clarificantes.

\SIG{Luciana de Campos}

Ver também Alimentação; Cotidiano; Festas e festins; Hidromel;
Sociedade.

\begin{itemize}
\item \versal{CAMPOS}, Luciana de. A sacralidade que vem das taças: o uso de bebidas no
Mito e na Literatura Nórdica Medieval. \emph{Revista Brasileira de
História das Religiões}, vol. 23, 2015, pp. 97-107.

\item \versal{CAMPOS}, Luciana de \& \versal{LANGER}, Johnni. Brindando aos deuses:
representações de bebidas na Era Viking, no cinema e nos quadrinhos.
\emph{Revista de História Comparada} (\versal{UFRJ}), vol. 6, 2012, pp. 141-164.

\item \versal{HAGEN}, Ann. \emph{Anglo-Saxon food and drink}. London: Anglo Saxon Book,
2010.

\item \versal{WARD}, Christie. Alcoholic beverages and drinking customs of the Viking
Age. \emph{The Viking Answer Lady}, 2005. Disponível em:
{http://www.vikinganswerlady.com/drink.shtml}.
Acesso em 14/04/2017.
\end{itemize}

\section{\versal{CIDADES, POVOAÇÕES E LOCALIDADES}}

Ver Arhus; Bergen; Birka; Danevirke; Dinamarca da Era Viking; Dorestad;
Dublin; Eketorp; Finlândia da Era Viking; França da Era Viking; Gamla
Uppsala; Gotland; Groenlândia Nórdica; Hedeby; Helgo; Ilhas Faroé;
Inglaterra da Era Viking; Irlanda da Era Viking; Islândia da Era Viking;
Jorvik; Kaupang; Kiev; Lejre; Lindisfarne; Mikligardr (Bizâncio);
Noruega da Era Viking; Ribe; Roskilde; Rússia da Era Viking; Sigtuna;
Suécia da Era Viking; Vínland; Wolin.

\section{\versal{CANUTO II, O GRANDE}}

Knútr inn ríki foi rei da Dinamarca entre o período de 1018 até
1035, tendo também reinado na Noruega de 1028 até 1035 e estado à frente do trono da
Inglaterra de 1016 até os anos de 1035. Canuto era filho de Sueno Barba
Bifurcada (Svend Tveskæg), rei da Dinamarca de 986 até a sua morte,
rei da Noruega entre 986 e 995 e depois 1000 e 1014, assim como
rei da Inglaterra a partir de 1013. Sueno era filho do
rei Haroldo~\versal{I} da Dinamarca. Canuto nasceu em meados do ano 995 d.C na Dinamarca e
faleceu com cerca de 40 anos de idade em 1035, sendo sepultado
na~Catedral de Winchester, uma das maiores catedrais da Inglaterra.

Sueno, durante seu período enquanto monarca, formou as bases do
largo domínio de governo que será consolidado por Canuto. Etelredo~\versal{II}, o
Despreparado, foi rei
da Inglaterra entre 978 e 1013 e depois de 1014 até 1016, enfrentou
ataques de Olavo~\versal{I} da Noruega (Óláfr Tryggvason) durante uma
parcela de seu reinado. Após várias querelas, o rei inglês conseguiu sair
vitorioso, ordenando o massacre de comunidades nórdicas que se
estabeleceram na costa inglesa durante os ataques vikings. Tal
atitude provocou uma reabertura de conflitos de Etelredo com os
nórdicos, pois a partir de 1002, Sueno iniciará ataques e expedições
contra ele.

Depois de anos de conflitos e ataques, nos idos de 1013 o rei Etelredo
busca refúgio na Normandia, sendo expulso de seu próprio domínio, dando
início a um novo regime régio na Inglaterra. Apesar de Sueno ser o
conquistador da região, sua morte em 1014 acaba por jogar suas
conquistas nas mãos de Haroldo~\versal{II} (Harald Svendsen), 
governante da Dinamarca até seu falecimento em 1018. A partir de então, Canuto assume a coroa e
consequentemente as diversas conquistas e direitos que toda uma hereditariedade lhe
permitiu.

Depois de auxiliar seu pai nas lutas com Etelredo, Canuto precisou
dedicar inicialmente todos os seus esforços para conter o retorno do rei inglês,
liderando em 1016 uma nova invasão
contra Edmundo~\versal{II} da Inglaterra, na conhecida batalha de Assadun, visto que
Canuto havia retornado para Dinamarca desde 1014. Após a invasão, se
forma a tessitura de um tratado de paz, em que as partes em conflito
logram por via diplomática um acordo que partilha do reino inglês. Essa
partilha perdura até novembro de 1016, quando Edmundo~\versal{II} acaba falecendo
ou sendo assassinado por mando de Canuto, logo, o filho de Sueno passou
a ser reconhecido como único monarca inglês.

Após tais querelas, com o intento de firmar seu novo reinado, Canuto
acaba por se casar com Ema da Normandia, esta que era viúva de Etelredo~\versal{II}, o Despreparado. 
Portanto, Canuto foi diretamente responsável pela
restauração inglesa, assim como pela organização política e por uma
constituição mais firme do ``sistema'' de propriedades fundiárias no
regime régio inglês. Além disso, foi diretamente responsável pela
criação de importantes condados, como os de Wessex, Mércia, Anglia
Ocidental e Nortúmbria.

Como supracitado, Haroldo~\versal{II} morre em 1018, deixando Canuto também rei
da Dinamarca, além da Inglaterra que já estava sobre seu controle. Em
1028 entrará em conflito com Olavo~\versal{II}, o
Santo (Ólafr Haraldsson), realizando a grande batalha de Helgeå (\emph{Slaget ved Helgeå}).
Sua vitória nesse embate naval lhe permitiu subir também ao
trono norueguês, consolidando-se como rei da Dinamarca, Noruega e
Inglaterra. Após essas vastas conquistas, suas terras ficam ou sobre seu
controle direto ou na regência de seus filhos: Hardacanuto (Hardeknud)
e Haroldo Pé de Lebre. Imediatamente após sua morte em
1035, o primeiro herda o torno da Dinamarca, o segundo passa a ser rei da Inglaterra e
 Magno~\versal{I}, o Bom (\emph{Magnús goði}) passa a ser o rei da Noruega, cargo
constantemente almejado por disputas e conflitos nos anos que
se seguem.

Canuto tem um governo marcado pela aproximação com o cristianismo e uma
paz e crescimento interno marcado pela formação de aliança e eliminações
pontuais de inimigos ao longo do seu governo. Sua ligação com o
cristianismo não se resume aos marcos de suas viagens, como a ida até
Roma, mas em relatos como a história \emph{Rei Canuto e as
ondas}, apócrifo escrito por Henrique de Huntingdon, datado do século~\versal{XII}. 
Além disso a \emph{Knýtlinga saga}, a saga dos descendentes de
Canuto, uma saga real da segunda metade do século~\versal{XIII}, fornece
muitas informações sobre a grandiosidade do seu reinado e de suas
medidas, sendo considerada obra de Óláfr Þórðarson por alguns
acadêmicos. Ainda podemos destacar os escaldos que lhe acompanhavam, que
produziram os \emph{Knútsdrápa}. O \emph{Skáldatal} (\emph{Catálogo dos
escaldos}), preservado no \emph{Codex Uppsaliensis}, nos revela que havia
ao menos oitos escaldos que acompanhavam Canuto, produzindo obras como o
\emph{Liðsmannaflokk}, \emph{Höfuðlausn}, \emph{Tøgdrápa}
e \emph{Eiríksdrápa}. Em meio a tantas fontes e suas múltiplas
representações, torna-se inviável neste espaço refletir acerca da 
construção da imagem de um rei considerado tão grandioso. 
Seu regime régio e sua larga associação ao cristianismo
possibilitaram que ele se tornasse um dos maiores reis nórdicos, não
apenas pelos atos de sua vida, mas pela forma como os sujeitos
manipularam suas narrativas com novas narrativas. Portanto, esperamos
abrir uma porta para as vastas possibilidades de estudo e pesquisa sobre
esse sujeito e suas reverberações na trama histórica.

\SIG{José Lucas Cordeiro Fernandes}

Ver também Danelaw; Dinamarca da Era Viking; Inglaterra da Era Viking.

\begin{itemize}
\item \versal{BOLTON}, Timothy. \emph{The Empire of Cnut the Great: Conquest and the
Consolidation of Power in Northern Europe in the Early Eleventh Century}.
Leiden: Brill, 2009.

\item \versal{EKREM}, Inger; Mortensen, Lars Boje (eds.).~\emph{Historia Norwegie}.
Museum Tusculanum Press, 2003.

\item \versal{FRANK}, Roberta.~King Cnut in the verse of his skalds.~In:~\versal{RUMBLE},
Alexander R. (ed.).~\emph{The Reign of Cnut: King of England, Denmark
and Norway}. London: Leicester, 1994, pp. 106-124.

\item \versal{HUNTINGDON}, Henry of.~\emph{The Chronicle of Henry of Huntingdon,
comprising The History of England, From the Invasion of Julius Caesar to
the accession of Henry~\versal{II}}. Forester, London: Henry, G. Bohn, 1853.

\item \versal{MALMESBURY}, William of.~\emph{Gesta Regnum Anglorum}. Mynors, Oxford:
Clarendon Press, 1998.

\item \versal{NORDEIDE}, Sæbørg Walaker. \emph{Christianization of Norway}. Paris 1
University: Conference paper, 2007.

\item \versal{RUMBLE}, Alexander R. (ed.).~\emph{The Reign of Cnut: King of England,
Denmark and Norway}. London: Leicester, 1994.

\item \versal{SAWYER}, Peter (ed.). \emph{The Oxford Illustrated History of the
Vikings}. Oxford: Oxford University Press, 2001.

\item \versal{STURLUSON}, Snorri. \emph{Heimskringla: History of the Kings of Norway}.
Trad. Lee M. Hollander. Austin: University of Texas Press, 1991.

\item \versal{TSCHAN}, Francis Joseph. \emph{Adam of Bremen: History of the
Archbishops of Hamburg-Bremen (Gesta Hammaburgensis ecclesiae
pontificum)}. New York: Columbia University Press, 1959.
\end{itemize}
\section{\versal{COGADH GÁEDHEL RE GALLAIBH}}

O texto conhecido como \emph{Cogadh Gáedhel re Gallaibh} (\emph{A guerra
entre os irlandeses com os estrangeiros}) é um texto medieval irlandês
que narra uma série de ataques vikings aos irlandeses entre o século~\versal{X} e
início do século~\versal{XI}, principalmente sobre a resistência feita pela
dinastia conhecida como Dál Cais contra os grupos escandinavos.

No entanto, a narrativa foi feita no início do século~\versal{XII}, baseada em
outras fontes textuais disponíveis na época, sob a tutela de
Muirchertach Ua Brian, descendente de Brian Boru e que possuía
claro interesse propagandístico e político para com a elaboração do
texto em si.

Essa visão dos acontecimentos relacionados com o conflito entre
irlandeses e vikings influenciou muitos historiadores séculos depois de
sua elaboração, especialmente Geoffrey Keating em sua \emph{História da
Irlanda} escrita c. 1634 (com tradução para o inglês em 1723). No
entanto, foi no século~\versal{XIX} que as narrativas presentes no \emph{Cogadh
Gáedhel re Gallaibh}, bem como a própria batalha de Clontarf narrada
dentro do corpo do texto, ganharam notoriedade.

Em 1867 é publicada, afinal, uma grande edição com tradução em larga
escala da obra, o que ajudou não apenas na divulgação da mesma no
imaginário histórico local, mas também na consolidação de um sentimento
nacionalista na Irlanda da época, elevando não só a narrativa, mas a
figura de Brian Boru ao status de ícone nacional.

A narrativa se inicia com a batalha de Sulchóit
(Solloghod), em 967, quando, segundo o \emph{Cogadh}, Mathgamain
macCennétig da dinastia dos Dál Cais, teria enfrentado e vencido
Ivar, chefe estrangeiro de Limerick. Nessa primeira parte da narrativa
são totalmente omitidos os conflitos internos que a dinastia dos
Dál Cais passava para se legitimar, sobretudo entre os
Eóganachta. Para o narrador do \emph{Cogadh}, Ivar era o grande
chefe por trás de todos os problemas da região de Munster, ao sul da
Irlanda.

A primeira parte do Cogadh se encerra com a vitória
inconteste de Mathgamain sobre os estrangeiros, aqui
representados por Ivar. Ao longo da narrativa são descritos os grandes
valores militares dos irlandeses em contraposição aos vikings, narrados
como terríveis, sanguinários e assassinos.

A narrativa segue com os acontecimentos da batalha de Clontarf, que
ocorre em 1014 e contrapõe as forças de Brian Boru, aclamado como o
grande rei da Irlanda da dinastia de Dál Cais e de seus
opositores, notadamente os vikings representados pelas forças do rei
Sitric de Dublin e os demais aliados do rei de Leinster, 
Máel Morda. O \emph{Cogadh} 
apresenta uma série de outros nomes e aliados de ambos os lados, sem, entretanto,
delimitar de fato os que realmente poderiam ter participado do conflito. Embora
alguns casos específicos possam ser comprovados, como é o caso dos nomes que
apareceram também nos demais anais irlandeses, não é possível usá-los
como base para verificação, já que não há como comprovar se o autor do
\emph{Cogadh} os incorporou porque usou registros presentes nos anais, ou
se os registros presentes nos anais usaram os registros antigos do
\emph{Cogadh}.

A narrativa descreve, então, combates singulares e atos de bravura das
forças aliadas de Brian Boru, sobretudo de seu filho Murchadmac
Brian, morto em combate, mas demonstrando inúmeros atos memoráveis que
o fazem ser equiparado às figuras lendárias como Heitor, Sansão,
Hércules ou mesmo Lug Lámfhada. Vale lembrar que Brian também será
equiparado às figuras lendárias como César Augusto ou mesmo Alexandre, o
Grande.

A batalha em si é descrita ocorrendo entre as marés do alvorecer e
anoitecer do dia 23 de abril de 1014, durando o dia todo. 
É justamente a descrição das marés no corpo do
texto que atribui certa veracidade à narrativa, passível de comprovação por
cálculos metereológicos. Para o historiador Seán Duffy, esse fato
é uma evidência de que se o autor da descrição não era
uma testemunha ocular da batalha, teve suas informações retiradas de quem de fato
o foi.

Na sangrenta batalha, as forças de Brian Boru saem vitoriosas sobre os
estrangeiros. Os líderes opositores como Máel Morda e
Bródir são mortos, bem como o filho de Brian, Murchad e o
próprio Brian. Sitric, o rei de Dublin, é descrito como tendo
aproveitado a maré para escapar do seu destino em batalha. A descrição,
no entanto, deixa claro o cristianismo de Brian Boru, bem como seu
valor em batalha ao descrever sua morte em luta contra Bródir e a
ideia de que com a vitória em Clontarf seus guerreiros teriam expulsado
as forças estrangeiras dos vikings da Irlanda. Em verdade, o grande
desfecho do \emph{Cogadh Gáedhel re Gallaibh} marca o triunfo orgulhoso
da dinastia de Dál Cais sobre seus opositores e usa da História para
hiperbolizar sua propaganda política.

\SIG{Erick Carvalho de Mello}

Ver também Brian Boru; Dublin; Irlanda da Era Viking.

\begin{itemize}
\item \versal{DOWNHAM}, Clare. Irish chronicles as a source for inter-Viking rivalry,
\versal{A.D.} 795-1014. \emph{Northern Scotland}, vol. 26, 2006, pp. 51-63.

\item \versal{DUFFY}, Seán. \emph{Brian Boru and the Battle of Clontarf}. Dublin: Gill
Books, 2014.

\versal{Ó CUÍV}, Brian. Ireland in the Eleventh and Twelfth Centuries c.
1000-1169\emph{.} In: \versal{MOODY}, Theodore \versal{W}. \& \versal{MARTIN}, Francis \versal{X}. \emph{The
Course of Irish History}. Cork: Mercier Press, 2011, pp. 107-122.

\item \versal{RICHTER}, Michael. \emph{Medieval Ireland: The Enduring Tradition}.
Dublin: Gill and Macmillan, 1988.
\end{itemize}

\section{COMÉRCIO}\label{comércio}

Apesar dos vikings serem lembrados mais como bravos guerreiros e
intrépidos navegantes, eles também foram comerciantes bem resolvidos. De
fato, os vikings da Suécia se destacaram dos seus vizinhos escandinavos,
por terem aderido mais ao comércio do que a campanhas de invasão,
pilhagem e conquista. O contato da costa sueca com o Mar
Báltico, aonde desaguam vários rios suecos, favoreceu a predisposição ao comércio desde meados do século~\versal{VI}.

Mas isso não significa que a Noruega e a Dinamarca não tenham
participado do comércio. A Noruega, por estar voltada ao Mar do Norte, 
teve facilidade para empreender viagens ao Arquipélago Britânico, e
décadas depois já havia rotas comerciais norueguesas ligando o reino a
Inglaterra, Escócia, Irlanda, Frísia e Islândia. A posição geográfica privilegiada
da Dinamarca permitiu aos dinamarqueses acessar tanto o
Mar Báltico quanto o Mar do Norte, assim como usufruir das rotas
comerciais terrestres da França, Alemanha, Polônia e Rússia.

Os escandinavos estabeleceram relações comercialiais com distintos povos e regiões da Europa e do oeste
asiático entre os séculos~\versal{IX} e \versal{XI}, fato que por si só
já desmente a noção retrógrada de que na Idade Média não havia
comércio de longa distância. Entretanto, não foi tão instantaneamente
outra que os nórdicos se lançaram ao comércio, o Período Vendel 
(séculos~{\versal{V}-\versal{VIII}}), época que antecede a expansão viking, já era conhecido pelo
comércio. Cidades costeiras como Menzlim, Rostock e Oldenburg, na
Alemanha; Wolin, Truso e Kolobrzeg na Polônia; Grobin na Letônia e
Tallin na Estônia, já mantinham contato com cidades suecas e talvez
dinamarquesas.

Não obstante, Guy Fourquin comenta que em várias regiões da Europa entre
os séculos~\versal{VIII} e \versal{X} observou-se um avivamento no comércio 
regional e internacional. Ele menciona a formação de entrepostos
comerciais nos Estados italianos, no califado de Córdoba no sul da atual
Espanha e Portugal, na França, Inglaterra, Irlanda, Germânia e na
Frísia. E grande parte desse comércio se desenvolveu inicialmente por
comerciantes não profissionais, que em geral eram camponeses,
pescadores e artesãos que aproveitavam os períodos de feira para
vender suas mercadorias ou se dedicavam a atividades mercantis no
período entressafras.

No caso da Escandinávia, isso não foi diferente. 
Devido ao solo pouco fértil da região, além de verões e
primaveras breves, sobrava tempo para outras atividades além da agricultura,
propiciando a existência de homens que se dedicavam as pilhagens, invasões e ao
comércio de longa distância. Esses mercadores viajavam
longas distâncias para vender seus produtos sobretudo
para adquirir artigos de luxo ou que inexistiam em sua terra, para
revendê-los a altos preços, o que lhe proporcionava um bom lucro.

O comércio na Era Viking era pautado principalmente nas
seguintes mercadorias: peles, penas, madeira, alcatrão, sal, minério de
ferro, xisto para pedras de amolar, âmbar, marfim de morsa, mel, peles
de foca e peixe salgado, como salmão e arenque. Em descobertas
arqueológicas nessas localidades foram achadas moedas romanas, francas,
saxãs, árabes, espadas e joias, e em Helgö, na Suécia, acharam uma
pequena estátua de Buda, o que atesta que os contatos com o leste e o
sul fossem bem mais longínquos do que se imagina.

Além de mercadorias comercializadas no
comércio interno e que depois passaram a ser exportadas, a aristocracia
e a nobreza da Era Viking também importavam muitos produtos, como ouro,
prata, ferro, vidro, seda, tecidos, vinho, azeite de oliva, pedras
preciosas, especiarias etc. Os túmulos reais da Suécia,
Noruega e da Dinamarca atestam a presença de tais produtos.

No século~\versal{VIII}, o comércio escandinavo já era significativo a tal ponto
que data de pelo menos 750 um assentamento comercial em Staraya Ladoga,
ao sul do lago Ladoga (atualmente na Rússia). Staraya Ladoga era um entreposto
comercial avançado, que servia de depósito de mercadorias e também
como polo manufatureiro. No século seguinte, acentuou-se o avanço
comercial escandinavo no Leste Europeu. Os eslavos começaram a se
referir aos vikings chamando-os de rus.

A partir de Staraya Ladoga os vikings possuíam acesso às rotas fluviais
dos rios Volkhov, Lovat, Oka, Volga e Dnepr, que permitiam ir para
o interior da Rússia, assim como seguindo para o sul, atravessando a
Polônia, Romênia, Hungria, Ucrânia e Bulgária. Outros rios como o Don, o
Dniepre e o Danúbio também foram usados para se chegar ao Mar Negro. Em
839, encontra-se o relato dos primeiros vikings que chegaram a
Constantinopla (atual Istambul), então rica capital do Império
Bizantino.

Constantinopla, chamada pelos nórdicos de Mikligardr (``a
grande cidade''), tornou-se uma importante rota comercial para
exportação e importação devido a sua localidade. Em Constantinopla
chegavam mercadorias advindas do Mediterrâneo, norte da África e Ásia.
Produtos como especiarias, seda, prata, ouro, joias etc.

A rota viking para Constantinopla atravessava todo o Leste Europeu.
Mercadores partindo do Mar Báltico seguiam para o sul, e no caminho
começaram a se mudar para cidades eslavas como Novgorod, Gnezdovo, Turov
e Kiev. No final do século~\versal{IX}, Novgorod (atual Rússia) e Kiev (atual Ucrânia)
estavam sob controle de chefes nórdicos. Nessas cidades eles
comercializavam principalmente peles, âmbar, marfim, minério de ferro,
vidro, cera e escravos. Os vikings desenvolveram um lucrativo comércio
escravocrata no Leste Europeu, pois os eslavos, cazares e árabes
adotavam o sistema escravocrata, que já havia sido abolido na Europa
ocidental.

Além do comércio na Europa oriental e com Constantinopla,
os comerciantes também desenvolveram a ``rota da prata'' para a Bulgária do
Volga (sul da Rússia), que dava acesso ao Canato de Cazar e ao império
árabe da Dinastia Abássida, onde comercializavam peles, madeira,
âmbar, escravos etc., obtendo em troca muita prata, metal bastante valioso para
sua metalurgia, pois os árabes naquele tempo controlavam as maiores
minas de prata da Ásia. Essas trocas comerciais duraram poucos anos, tendo
diminuído significante ainda no século~\versal{IX}, devido a crises no Império
Abássida.

A importância da prata no comércio escandinavo devia-se não apenas pelo
fator de riqueza, mas pelo motivo que por muito tempo a prata foi usada
como moeda de troca. Na Escandinávia antes do século~\versal{X}, o uso de moedas
não era homogêneo. As moedas mais antigas cunhadas na Escandinávia datam
do século~\versal{IX}, pertencentes às cidades de Hedeby e Ribe, ambas na
Dinamarca. A moeda de Hedeby foi datada por volta de 825, tendo sido
baseada em moedas frísias, mas se desconhece se seu emprego foi regular.
Na maior parte do tempo o comércio foi feito com base no peso da prata.
Lingotes, anéis e outros objetos de prata eram usados como moeda
corrente: media-se o valor das mercadorias com base em seu peso.

Na segunda metade do século~\versal{X}, alguns monarcas passaram a cunhar suas próprias
moedas de prata; é o caso de Érico Machado Sangrento (947-948/952-954), Sueno
Barba Bifurcada (c. 985-1014), Olavo Haraldsson (1015-1028), e Canuto, o
Grande (1016-1035). A cunhagem de moedas vikings foi, segundo Svi Gullbekk,
bastante influenciada por modelos de moedas anglo-saxãs e germânicas,
assim como adotaram fatores de ordem política, exibindo o nome dos
monarcas, no intuito de reforçar sua autoridade sobre o reino.

Mas o comércio não proliferou apenas no Leste Europeu e no oeste
asiático, a Saxônia e Frísia (ambos na Alemanha) e as Ilhas Britânicas a
partir do século~\versal{IX} tornaram-se rotas comerciais contínuas para
mercadorias escandinavas devido às ocupações nórdicas ou os acordos
comerciais regulares, como no caso dos territórios germânicos, já que os
territórios insulares da Inglaterra, Irlanda e Islândia foram ocupados e
colonizados pelos escandinavos.

A partir dessa ocupação iniciada ainda no século~\versal{IX}, cidades como Dublin
(Irlanda) e York (Inglaterra) tornaram-se importantes centros comerciais
no século~\versal{X}. Os ingleses e irlandeses já mantinham comércio
com os francos, frísios e germanos pelo menos desde o século~\versal{VII}, como
sugere a importação de vinhos francos para mosteiros ingleses e
irlandeses, mas com a ocupação escandinava de parte dessas ilhas, o comércio
foi substancialmente desenvolvido.

A cidade de York, chamada de Jorvik pelos vikings, tornou-se, no século~\versal{X}, 
um centro econômico que recebia mercadorias como seda bizantina, vinhos
germânicos e âmbar do Báltico. A cidade tornou-se também um
polo manufatureiro de utensílios domésticos, joias em prata, ouro e
âmbar, vestuário, calçados, ferramentas etc. Tudo isso foi reflexo do
mercado escandinavo introduzido na Inglaterra, que alavancou o comércio
inglês com sua ampla rede de contatos.

\SIG{Leandro Vilar Oliveira}

Ver também Bergen; Birka; Caça; Hedeby; Navegação.

\begin{itemize}
\item \versal{FOURQUIN}, Guy. \emph{História econômica do ocidente medieval}. Lisboa:
Edições 70, 2000.

\item \versal{GRAHAM-CAMPBELL}, James (org.). \emph{Os vikings}. Barcelona: Editora
Folio \versal{S.A.} 2006.

\item \versal{GULLBEKK}, Svein H. Coinage and monetary economies. In: \versal{BRINK}, Stefan;
\versal{PRICE}, Neil (eds.). \emph{The Viking World}. London/New York: Routledge,
2008, pp. 159-169.

\item \versal{SINDBÆK}, Søren Michael. Local and long-distance exchange. In: \versal{BRINK}, Stefan; \versal{PRICE}, Neil (eds.). \emph{The Viking World}. London/New York:
Routledge, 2008, pp. 150-158.

\item \versal{SKYRE}, Dagfinn (ed.). \emph{Means and Exchange: dealing with Silver and
the Viking Age}. Oslo: Aarhus University Press, 2007. (Kaupang Excavation
Project Publications Series, vol. 2).
\end{itemize}

\section{\versal{COMPORTAMENTO}}\label{comportamento}

Ver Alimentação; Aparência e costumes; Casamento e divórcio; Cosméticos;
Crianças e infância; Duelos; Estupro; Festas e festins; Guerra e
religião; Higiene; Sexo e sexualidade; Sociedade; Suicídio.

\section{\versal{CONVERSÃO AO CRISTIANISMO}}

Ao se falar em conversão ao cristianismo, a primeira coisa que se deve
entender é o seu significado para aqueles que estavam sendo
cristianizados. As crenças nórdicas, assim como o próprio cristianismo, não possuíam
o mesmo significado em outros lugares da Europa já cristianizada. O
processo de conversão era entendido como uma \emph{siðaskipti}, mudança
de costumes: os antigos costumes eram abandonados em prol de uma nova
tradição, que aumentava cada vez mais sua presença entre os grupos com os quais os escandinavos se
relacionavam, seja por razões de comércio, viagens ou saques. O ``costume'' (\emph{siðr}, em nórdico antigo) 
englobava os modos de viver, moralidades, entre outros
significados. Desse modo, a conversão não representava simplesmente uma mudança
externa dos antigos costumes (\emph{forn siðr}) para
os novos costumes (\emph{nýr siðr}).

É importante ressaltar que existe uma diferenciação entre conversão e
cristianização. Os conceitos de conversão e cristianização causam 
muita confusão em suas definições. James C. Russell (1994) compilou uma análise das definições,
na qual inicialmente ``conversão'' é a modificação comportamental e
ideológica do indivíduo, resultando em uma nova visão de mundo. Assim,
a conversão implica numa grande mudança de
consciência, entendendo o antigo como parâmetro errado e o novo como parâmetro certo.

É praticamente um consenso para a historiografia contemporânea que a
conversão da Escandinávia não foi um processo abrupto e repentino, mas
algo que levou anos e dificilmente pode-se apontar uma
data de início ou fim. Os estudos sobre a cristianização dos vikings
acabam dividindo as pesquisas normalmente pelas regiões que hoje
representam os países nórdicos, como Noruega, Islândia e Dinamarca. Cada
uma dessas regiões possuiu suas características particulares e por isso
devem ser entendidas dentro do seu contexto.

As relações comerciais e intercâmbios com as regiões
vizinhas e distantes trouxeram aos vikings o contato com o cristianismo,
que inicialmente era somente mais uma religião dentro da miríade
conhecida por eles. A religiosidade escandinava não era algo monolítico:
regiões que ficavam alguns quilômetros de distância podiam possuir
formas de cultos e adorações diferentes, ou uma preferência por deuses
diferentes, mas sempre mantendo uma semelhança básica. Devido ao
posicionamento geográfico a primeira região a possuir maior contato com
o cristianismo foi a Dinamarca.

A Dinamarca era geograficamente mais próxima e possuía maior contato com o 
Sacro Império Romano Germânico, o que levava a uma presença contínua de missionários 
em seu território. No início do século~\versal{IX} chegaram lá os primeiros missionários, 
cujo trabalho de cristianização e a pressão vinda do sul pelo Sacro Império levaram à
resolução da disputa política que existia na região: Haroldo Klak foi
batizado e, apoiado pelo Luís, o Piedoso, tomou o trono dinamarquês. As
principais fontes envolvendo a conversão de Haroldo são a \emph{Vita
Anskarri} e a \emph{Carmen in honorem Hludowici}, escritas por dois
clérigos que acompanharam o rei no seu retorno a Dinamarca após a sua
conversão, em 826. Apesar dos esforços iniciais, a cristianização da
Dinamarca só foi estabelecida firmemente no reino de Haroldo
Dente Azul, que tornou o cristianismo a religião nacional depois de 
sua própria cristianização, em 965.

A cristianização da Noruega tem história similar ao caso dinamarquês. 
Apesar da presença de missionários casuais, a efetividade das suas
tentativas de conversão não é clara nas fontes arqueológicas. Esse
cenário começa a mudar com o reinado de Hakon, o Bom. Hakon, filho de
Haroldo Cabelos Belos, passou sua juventude na Inglaterra já
cristianizada sob a tutela do rei Athelstan. Ao retornar para a Noruega
para confrontar seu tio Érico Machado Sangrento, Hakon associou-se com o
\emph{Jarl} de Lade e foi vitorioso. Hakon decide iniciar a
cristianização da Noruega, mas sua tentativa acabou lidando com
desavenças políticas que o levaram à derrota narrada pelo poema
\emph{Hákonarmál}, que relata sua batalha final e posterior ascensão a
\emph{Valhalla}.

A derrota de Hakon pelo seu sobrinho, Haroldo Capa Cinzenta, apoiado
pelo rei Dinamarquês Haroldo Dente Azul, levou a Noruega a um período
de descentralização do poder, funcionando como uma forma de reino
subordinado da Dinamarca. Haroldo Capa Cinzenta, diferente de seu tio,
foi descrito com um mau rei -- possivelmente essa representação 
também se deva ao fato de ele ter tornado ilegal o culto aos deuses antigos.
Esse mau governo acabou levando-o a ser morto em um complô entre Haroldo
Dente Azul e o \emph{Jarl} de Lade da época, Hakon, que se tornaria o
último governante pagão da Noruega.

Com a morte do último líder pagão, a Noruega seria governada por Olavo
Tryggvason, neto de Haroldo Cabelos Belos, que viveu seus anos
anteriores como saqueador na Inglaterra e posteriormente se converteu
ao cristianismo. Sua conversão é um ponto de debate, pois os
documentos discordam, apontando duas possíveis versões: em uma, ele
teria sido batizado na Inglaterra; na outra, seu batizado teria sido feito por
missionários dinamarqueses na Noruega. A versão mais aceita é o do
batismo na Inglaterra em 991, como demonstra na \emph{Anglo-Saxon
Chronicle}. Sua conversão, acredita-se, faz parte de um acordo
político com a Inglaterra, garantindo uma aliança entre os dois reinados.
Em 995, Olavo assumiu o reinado e segundo a \emph{Heimskringla} e a
\emph{Gesta Danorum}, realizou a conversão da Noruega de maneira
forçosa, além de exercer uma pressão sobre outras regiões, como a
Islândia.

A cristianização da Islândia ocorreu de uma maneira particular. A
Islândia foi ocupada, segundo o \emph{Landnámabók}, no ano de 870, com
noruegueses buscando fugir da tirania do rei Haroldo Cabelos Belos. A
ilha já tinha outros habitantes, mas em um número inferior aos
colonizadores noruegueses. Esses noruegueses trouxeram a religião de sua
terra natal, o culto aos deuses nórdicos. A Islândia vivia em um sistema
político horizontal, no qual as decisões eram resolvidas em
\emph{Things}, ou na \emph{Althing}. Na \emph{Althing} ocorrida no ano
999, decidiu-se, após os devidos debates, pela conversão formal ao
o cristianismo. Entende-se que essa conversão para o cristianismo se
deve em parte a não ortodoxia do paganismo. O cristianismo era visto
como mais uma forma de crença diferente, assim como várias outras por eles já
conhecidas. Essa mudança foi uma conversão política. Os estudos
mais recentes sobre a cristianização da Islândia argumentam acerca da 
violência dessa conversão, que embora não congregue formas de
violência física, foi simbólica e
representativa.

O resultado desses processos é um cristianismo diverso e
variado. Mesmo que apresentado sob a mesma nomenclatura, 
as peculiaridades regionais, como a cultura local e a política de cada região
influenciaram na visão de mundo e nas práticas cristãs. As práticas pré-cristãs
não desapareceram com a conversão ao cristianismo, mas encontraram sua resistência 
na cultura local por meio de adaptações na religiosidade, procedimento que também ocorreu com as 
práticas sociais. Apesar de ter-se abandonado o culto aos deuses
antigos, as práticas pré-cristãs se sedimentaram naquela sociedade de
tal forma que os efeitos de longa duração são observáveis até os dias
atuais.

\SIG{André Araújo de Oliveira}

Ver também Althing; Godi; Islândia da Era Viking; Religião.

\begin{itemize}
\item \versal{BAGGE}, Sverre. A Hero between Paganism and Christianity. \emph{Poetik
und Gedächtnis, Festschrift für Heiko Uecker zum 65} Bonn, 2004, pp.
185-210.

\item \versal{BAGGE}, Sverre. Christianization and State Formation in early Medieval
Norway\emph{.} \emph{Scandinavian Journal of History}, vol. 30, n. 2,
2005, pp. 107-134.

\item \versal{BEREND}, Nora. \emph{Christianization and the Rise od the Christian
Monarchy}. Cambridge: Cambridge University Press, 2007.

\item \versal{NORDEIDE}, Saebjorg Walaker. The Christianization of Norway.
\emph{Speculum}, vol. 88, n. 4, 2013, pp. 1139-1140.

\item \versal{RUSSELL}, James C. \emph{The Germanization of Early Medieval
Christianity}, a socio-historical approach to religious transformation.
Oxford, 1994.

\item \versal{SELF}, Kathleen M. Remembering our violent convertsion: Conflict in the
Icelandic conversion narrative. \emph{Religion}, vol. 40, n. 3, 2010,
pp. 182-192.

\item \versal{WILLIAMS}, Gareth; \versal{BIBIRE}, Paul. (orgs.). \emph{Sagas, Saints and
Settlements}. Boston: Brill, 2004.
\end{itemize}

\section{\versal{COSMÉTICOS}}\label{cosméticos}

O uso de cosméticos na Era Viking era muito difundido tanto entre os
homens como entre as mulheres. A maquiagem estava entre os itens de
beleza usados cotidianamente e homens e mulheres da cidade usavam
maquiagem para parecerem mais jovens e atraentes. Uma mistura de carvão
vegetal reduzido a pó finíssimo misturado com gordura era passado ao
redor dos olhos, formando círculos pretos que atualmente pareceriam
olheiras, mas que na Era Viking realçavam a beleza.

Alguns óleos extraídos de oleaginosas, como as nozes e avelãs, e gorduras
animais mais finas, como banha do porco e de aves, eram misturadas com
ervas, como a artemísia e a camomila, e essa pasta era aplicada no rosto e
no corpo, conferindo uma boa hidratação e umectação para a pele que
estava exposta ao frio no exterior das casas e ao calor do fogo e
fuligem no interior delas. As gorduras misturadas às ervas formavam
uma camada protetora para a cútis. As mulheres também faziam uso de
pedras, que tinham o tamanho de uma pequena bola e eram bem lisas, para
apertá-las contra a pele do rosto de modo a aliviar as rugas e linhas de
expressão. Com os contatos e as trocas comerciais com o Mediterrâneo e
Bizâncio, outros produtos usados como cosméticos também passaram a ser
utilizados pelos mais abastados, como a mistura de açafrão e camomila
para clarear os cabelos e o óxido de chumbo -- este, apesar de altamente tóxico,
também passou a ser usado como mais um item de cuidados com a beleza.

Os cuidados com os cabelos também eram constantes:
eram penteados, depois trançados com nós e tranças complexas, no caso das
mulheres. Neles também eram aplicados óleos perfumados para deixá-los
mais brilhantes e atraentes. Homens e mulheres utilizavam essas
técnicas.

Homens e mulheres também se banhavam pelos
menos uma vez por semana e usavam óleos e ervas perfumadas para deixarem
seus corpos limpos. Era hábito comum, depois do banho, esfregar
vigorosamente o corpo com galhos de plantas aromáticas para
retirada de todas as impurezas e para dar à pele um bom aspecto. O
hábito de bater um galho de folhas aromáticas no corpo pode ser
observado até hoje na Finlândia durante a sauna.

\SIG{Luciana de Campos}

Ver Cotidiano; Mulheres; Cultura material; Medicina e botânica mágica.

\begin{itemize}
\item \versal{CAMPOS}, Luciana de. Cosmética, plantas e saúde na Era Viking.
\emph{Youtube/Canal do \versal{NEVE}}, 2017. Disponível em:
{\emph{https://www.youtube.com/watch?v=4TDvmqRKjWc}}. Acesso em 24 nov. 2017.

\item \versal{HARVIG}, Lise \emph{et al}. Death in Flames:
Human Remains from a Domestic House Fire from Early Iron Age, Denmark.
\emph{International Journal of Osteoarchaeology}, vol. 25, n. 5, 2015, pp.
701-710.
\end{itemize}

\section{\versal{COTIDIANO}}\label{cotidiano}

Ver Agricultura; Alimentação; Aparência e costumes; Arte; Bóndi; Caça;
Comércio; Comportamento; Cosméticos; Embarcações; Ferreiros e ferraria;
Festas e festins; Habitação; Higiene; Hnefatafl; Jogos e esportes;
Música; Poesia escáldica; Religião; Sexo e sexualidade; Sociedade.

\section{\versal{CRIANÇAS E INFÂNCIA}}

A documentação sobre a vida das crianças e sobre a infância na Era
Viking é escassa, mas os vestígios de cultura material nos oferecem algum
subsídio para que possamos entender como era o mundo infantil nórdico.
Nas sagas quase não há menção às crianças e à infância, limitando-se
muitas vezes a mencionar apenas o número de filhos que determinada
personagem possuía, sem descrições mais pormenorizadas ou detalhes 
acerca da educação, de quais eram os brinquedos e brincadeiras e qual
era o \emph{status} da criança dentro da comunidade. É importante salientar que em
praticamente todo o mundo nórdico a prática do infanticídio era comum. O
número de crianças expostas era grande, pois muitas vezes para o menos
abastados criar muitos filhos era uma tarefa difícil e trazia despesas
com as quais a família não podia arcar, principalmente durante os rigorosos
invernos, quando a comida era pouca. Os filhos concebidos fora do casamento
ou filhos de mães solteiras que não podiam sustentar essas crianças
eram também vítimas do infanticídio. A mortalidade infantil também
atingia altos índices: a desnutrição, as doenças típicas da primeira
infância e muitas vezes a falta de cuidados de higiene formavam um
ambiente de risco para os recém-nascidos e as crianças pequenas, tanto
para as mais ricas, como para as mais pobres.

As sepulturas de crianças, embora poucas se comparadas às de
adultos, também são fundamentais para se compreender como era essa
primeira fase da vida dos homens e mulheres nórdicos. A análise
osteológica de alguns esqueletos infantis apresentavam hipoplasia
dentária (que é a formação incompleta do esmalte dentário) e hiperostose
porótica, doença que está diretamente relacionada a deficiência de
vitaminas do complexo \versal{B}, como a \versal{B2} e \versal{B12}. Geralmente esse mal tem início
no momento em que a criança está mudando da dieta exclusiva de leite materno para o
alimento sólido, e acontece especialmente por insuficiência de vitaminas, ou em casos em que a criança
ainda depende do leite da mãe, mas ela não está ingerindo nutrientes
suficientes para suprir tanto as suas necessidades diárias como as do lactente.
Apesar de os nórdicos possuírem uma dieta
mais rica em nutrientes devido ao grande consumo de peixes de águas
frias, ricos em ômega 3 e 6, tais como o arenque e o salmão, por exemplo, em
algumas épocas do ano, as colheitas eram pobres e o rigores do inverno
ceifavam muitas vidas. Além disso, muitas vezes o estoque de alimentos
não era suficiente para todos por um grande período -- sendo
assim, as crianças, tanto as que ainda dependiam do leite materno
como as que já conseguiam se alimentar sozinhas eram as primeiras a
perecerem diante da carestia e de toda e qualquer adversidade climática.

As crianças que sobreviviam aos primeiros e mais difíceis anos da
infância e não sucumbiam a nenhuma doença ou intempérie iniciavam a sua
educação junto à família e ao restante da comunidade. Os meninos
aprendiam as tarefas diretamente ligadas ao trabalho no campo, como arar
e semear. Também eram iniciados na arte da guerra e aprendiam brincando
de lutar com pequenos escudos e espadas de madeira. Pequenos barcos de
madeira e cavalos entalhados também eram comumente utilizados pelos
meninos em suas brincadeiras. As meninas aprendiam a fiar e a
tecer com pequenos fusos e teares apropriados ao seu tamanho. Ajudavam
suas mães nas tarefas domésticas de cozinhar e ordenhar as vacas e
aprendiam a conhecer
as ervas medicinais e comestíveis e a iniciarem-se nas artes da cura. No
tempo livre elas brincavam com suas bonecas e meninos e meninas jogavam
Hneftall. Durante o inverno, os rios e os lagos congelados
proporcionavam bons momentos de diversão ao ar livre, permitindo que as
crianças patinassem no gelo.

Durante a infância as crianças aprendiam sobre o que seria a sua vida
futura: as meninas precisam seguir os conselhos e orientações de suas
mães e das demais mulheres da comunidade para viverem em um mundo que
seria regido pelos homens; os meninos aprendiam a defender a si e a sua
família e ambos deviam estar preparados para se proteger contra todo e
qualquer invasor.


\SIG{Luciana de Campos}

Ver também Cotidiano; Família; Mulheres; Sociedade.

\begin{itemize}
\item \versal{GIBSON}, Michael. A educação das crianças. In: \emph{Os Vikings}. São
Paulo: Melhoramentos, 1990, pp. 22-23.

\item \versal{HAYWOOD}, John. Children. In: \emph{Encyclopaedia of the Viking Age}.
London: Thamas and Hudson, 2000, p. 43.

\item \versal{JAKOBSSON}, Ármann. Troublesome Children in the Sagas of the Icelanders.
\emph{Saga-Book}, vol. 27, 2003, pp. 05-24.

\item \versal{MCALISTER}, Deirdre. Childhood in Viking and Hiberno-Scandinavian Dublin,
800--1100. In: \versal{HADLEY}, Dawn; \versal{TEN-HARKEL}, Letty (eds.). \emph{Everyday
Life in Viking `Towns': Social Approaches to Towns in England and
Ireland c. 800-1100}. Oxford: Oxbow, 2013, pp. 86-102.

\item \versal{MORGAN}, Rachel. \emph{Children in Viking Studies: a case for material
culture studies}. University of York, s.d.
\end{itemize}

\section{\versal{CRÔNICA ANGLO-SAXÔNICA}}

Nome derivado do inglês \emph{Anglo-Saxon Chronicles}, convencionalmente
aplicado por pesquisadores modernos a uma série de anais e crônicas
produzidos, compilados e organizados a partir do final do século~\versal{IX}, por
volta do ano 890. A \emph{Crônica} ao todo é marcada por 8 manuscritos
escritos em inglês antigo e apenas um com tradução latina. Os
manuscritos são: \versal{MS. A} -- \emph{The Parker Chronicle} ou \emph{The
Winchester Chronicle} (c. 891-1093); \versal{MS. B} -- \emph{The Abingdon
Chronicle~\versal{I}}, compilado a partir do ano 1000; \versal{MS. C} -- \emph{The
Abingdon Chronicle~\versal{II}}, compilado na 2ª metade do século~\versal{XI}-1066; \versal{MS. D}
-- \emph{The Worcester Chronicle}, compilado a partir do século~\versal{XI} e que
inclui algum material de Beda; \versal{MS. E} -- \emph{The Peterborough
Chronicle}; \versal{MS. F} -- \emph{The Cantebury Bilingual}, duas compilações:
em latim e inglês antigo; \versal{MS G}, que é uma cópia do \versal{MS. A}; e \versal{MS. H} --
apenas um fragmento que contém os anos de 1113-1114.

Para os que se dedicam aos estudos da Inglaterra durante a Era Viking,
trata-se de uma documentação bastante rica em informações, sobretudo
acerca dos ataques escandinavos à ilha. A natureza desses ataques pode
ser dividia em 4 etapas a partir da \emph{Crônica}: 1) ataques esporádicos e
pilhagem (789-864); 2) ocupação permanente (865-896); 3) extorsão de
tributos (980-1012); 4) conquista política (1013-1016).

A primeira fase, que vai até metade do século~\versal{IX}, é marcada por diversos
ataques, de acordo com a \emph{Crônica}. Entre finais do século~\versal{VIII} até a
década de 30 do século~\versal{IX}, os ataques foram espaçados. No entanto, a
partir de 836, eles passaram a ser cada vez mais regulares, com
intervalos de mais ou menos 2 anos, concentrados mais na parte sul da
ilha e direcionados para os reinos de Mercia, Kent e Wessex.

A narrativa inicia-se com as Ilhas Britânicas e a história do Império
Romano a partir da chegada de Júlio César. Eventos do continente europeu
e de outras partes do mundo também são por vezes destacados, sobretudo
os que estão relacionados à história da Igreja. Há um prefácio topográfico
que versa sobre a extensão territorial da ilha, seus
primeiros habitantes e quais povos a habitavam no momento em que
possivelmente a obra começou a ser escrita, no final do século~\versal{IX}.

Essa documentação foi provavelmente produzida a partir de anais de Kent,
Sussex, Mercia e, principalmente, Wessex. Além dessa tradição diversa da
ilha, as \emph{Crônicas} foram supostamente escritas por vários compiladores,
numa espécie de processo colaborativo, de forma que não é possível
identificar traços individuais no texto a quem possamos atribuir
autoria.

O \versal{MS. A} é também conhecido como \emph{The Parker Chronicle} por ter
pertencido a Matthew Parker, bispo de Canterbury (1559-1575) e é o
manuscrito mais antigo das \emph{Crônicas.} Além das informações
presentes acerca das lutas de Alfred e de seu filho Edward contra os
nórdicos, há ainda referências a vitórias épicas dos anglo-saxões, como,
por exemplo, a batalha de Brunanbuhr, em 937. A partir de 975 parece
haver um hiato com relação aos eventos narrados pelo \versal{MS. A} e os anos
subsequentes não vêm descritos com a mesma riqueza de detalhes que os
séculos \versal{IX-X}. Em finais do século~\versal{X}, no entanto, os manuscritos \versal{D} e \versal{E}
tendem a apresentar os eventos de maneira mais completa e dão enfoque
particular à relação entre anglo-saxões e escandinavos, como, por
exemplo, a referência ao massacre do Dia de São Brício (1002).

Nas \emph{Crônicas} encontramos ainda uma série de referências ao passado
pré-cristão, muito embora no momento em que a compilação do texto foi
iniciada, os reinos anglo-saxões já fossem cristianizados. Todavia, a
presença de elementos culturais anteriores ao cristianismo não
relativiza a crença desses povos, mas nos mostra que eles estavam
inseridos numa tradição cultural que remetia ao período anterior às
migrações para a ilha e que esta tradição fazia parte de sua construção
identitária enquanto grupo.

Logo ao início do texto foi incluída a linhagem de Wessex, desde Cerdic
e seu filho Cynric -- com sua chegada à ilha no ano de 494 -- até
Alfred. Apesar da nítida preponderância de Wessex, a genealogia que
observamos ao longo do texto não é exclusiva do reino em questão, pois
contém ainda outras linhagens reais como as da Nortúmbria, da Mércia e
de Kent. A inclusão destas diferentes tradições e mitos de origem
presentes ilustra quão distintos os povos anglo-saxões eram uns dos
outros, suas particularidades étnicas e o percurso de cada um até sua
chegada à ilha. Mesmo assim, embora cada reino tenha um passado
específico, marcado por eventos particulares, todos eles compartilham
elementos e se inserem no contexto de uma história coletiva.

Com relação ao gênero narrativo, anais e crônicas normalmente são
identificados como uma forma de contar eventos, organizados a partir de
uma linha cronológica. Dessa forma, os fatos estão relacionados ao tempo
em que eles teriam ocorrido. Lida de maneira superficial, a crônica
medieval parece mais uma série de exemplos citados, divididos
normalmente por ano, mas aparentemente sem um fio condutor entre si. De
certa forma, são narrativas ``abertas'', pois há sempre o potencial de
continuação do ``enredo'' da história.

As \emph{Crônicas Anglo-Saxônicas}, tais como outras do mesmo gênero escritas
no período, sofreram influência direta das tabelas de Páscoa, de autoria
atribuída a Dionysius Exiguus (532-626), com o objetivo de determinar a
data da Páscoa para os anos futuros a partir do calendário juliano e
indicando a numeração dos anos como \emph{ab incarnatione Domini}. As
\emph{Crônicas} representam, portanto, uma sequência linear da narrativa típica
dos primeiros tempos medievais sobre como o tempo era datado, entendido
e interpretado: uma representação linear de todos os anos que se
passaram desde a encarnação de Cristo, do passado, através do presente e para um
futuro infinito.

Uma particularidade é que as \emph{Crônicas Anglo-Saxônicas} podem ser
identificadas a partir da parataxe, sequência de frases
justapostas, sem conjunção coordenativa, fato que não é comum em outras
crônicas produzidas na Alta Idade Média. Conjunções como ``e'' e
``então'' não eram utilizadas ao longo do texto, talvez apenas por
uma questão de estilística, talvez para dar espaço para que o leitor
pudesse refletir e tirar suas próprias conclusões a respeito do texto.
Como exemplo de parataxe podemos citar o trecho 900, a respeito da
morte de Alfred ``\emph{Her gefor Ælfred Aþulfing, syx nihtum ær ealra
haligra mæssan; Se wæs cyning ofer eall Ongelcyn butan ðæm dæle þe under
Dena onwalde wæs}'' (``Neste dia morreu Alfred filho de Ӕthelwulf seis
noites antes da missa de Todos os Santos. Ele foi rei de todos os
ingleses, com exceção daqueles que estavam sob domínio dos daneses'').

\SIG{Isabela Dias de Albuquerque}

Ver também Anglo-saxões e nórdicos; Danelaw; Danevirke; Inglaterra da
Era Viking.

\begin{itemize}
\item \versal{ANLEZARK}, Daniel. Sceaf, Japhteth and the origins of the Anglo-Saxons.
\emph{Anglo Saxon England}, vol. 31, pp. 13-46.

\item \versal{FOOT}, Sarah. Finding the meaning of form: narrative in annals and
chronicles. In: \versal{PARTNER}, Nancy. \emph{Writing Medieval History}. New
York: Oxford University Press, 2005, pp. 88-105.

\item \versal{SWANTON}, Michael. Introduction. In: \emph{The Anglo-Saxon Chronicles}.
London: Phoenix Press, 2000.
\end{itemize}
\emph{\versal{The Anglo-Saxon Chronicle}}. Versão original de todos os
manuscritos. Disponível em:
\href{http://asc.jebbo.co.uk/intro.html}{\emph{http://asc.jebbo.co.uk/intro.html}}. Acesso em: 24 
nov. 2017.

\section{\versal{CRÔNICA DOS ANOS PASSADOS}}

A fonte conhecida como \emph{Crônica dos Anos Passados} (\emph{Povest
Vremennykh Let}) é o principal meio escrito disponível para o estudo dos
primeiros séculos da Rus de Kiev. Trata-se de uma compilação analística
que narra a história da Rus, com ênfase na cidade de Kiev, desde o
dilúvio bíblico até o século~\versal{XII}, dependendo da versão. Acredita-se que
ela tenha sido escrita e compilada pela primeira vez entre os séculos~\versal{XI}
e \versal{XII}. A cronologia presente na \emph{Crônica} está em formato
\emph{Anno Mundi}, ou seja, a contagem dos anos tem como base a criação
do mundo conforme a Bíblia. A fonte ainda não possui tradução para o
português, mas os brasileiros -- pesquisadores ou leigos -- interessados na
\emph{Crônica} e que não compreendem os idiomas eslavos 
podem recorrer às
traduções para o inglês, espanhol, francês e alemão.

A fonte por muito tempo foi conhecida como \emph{Crônica de Nestor},
pois acreditava-se que o monge São Nestor, o Cronista (século~\versal{XI}-\versal{XII}),
seria o seu único autor/compilador, ideia proposta pela primeira vez no
século~\versal{XVIII} pelo distinto historiador russo Vassílii Tatíschev. A
partir do século~\versal{XIX}, porém, mesmo sendo acusados de antipatriotismo,
alguns historiadores e filólogos passaram a questionar tal autoria.
Muitos pesquisadores passaram a atribuí-la ao abade Silvestre, que
assina a entrada de 6624 (1116) em algumas versões da fonte.
Recentemente, há um consenso entre os pesquisadores sobre a
impossibilidade de um único autor ou compilador da \emph{Crônica} devido
ao estilo inconsistente da escrita ao longo das passagens e o tempo que
deve ter levado para o trabalho ser feito. Atualmente, alguns autores
ainda se referem à \emph{Crônica dos Anos Passados} como \emph{Crônica
de Nestor}. Outros nomes bastante comuns dados à fonte na historiografia
são \emph{\versal{PVL}}, abreviação do nome em russo e ucraniano, e \emph{Crônica
Primária} (\emph{Primary Chronicle}), termo popularizado pelo eslavista
estadunidense Samuel Cross.

O filólogo russo Aleksey Shakhmatov afirma que antes da primeira
compilação já havia uma versão anterior datada do século~\versal{XI} que teria
sido base também para a \emph{Crônica de Novgorod}, a qual Shakhmatov
acredita ser mais próxima do manuscrito original que a própria
\emph{Crônica dos Anos Passados}. Essa hipótese é amplamente aceita pela
filologia e historiografia, bem como nas diversas tentativas de reconstrução da
versão original (em russo, \emph{Nachalny Svod}). Há diversas versões da
\emph{Crônica}. A mais antiga é a laurentiniana, compilada no
século~\versal{XIV} e que recebeu esse nome pela assinatura do monge Lavrenty. A
versão ipatiana -- que, datada do século~\versal{XV}, tem esse nome por ter sido 
descoberta no Monastério
de Ipatiev --, é uma reedição da versão laurentiniana,
com entradas posteriores pertencendo às fontes conhecidas como
\emph{Crônica de Kiev} (1118 até 1200) e \emph{Crônica da
Galícia-Volínia} (1201 até 1292). Ambas as versões são as mais
utilizadas pelos especialistas nos estudos sobre Rus. Existem várias
outras versões posteriores, como a de Radzwiłł, datada do final do
século~\versal{XV}; a de Khlebnikov, de meados do século~\versal{XVI}; e a 
de Pogodin, do século~\versal{XVII}. Todas elas, quando comparadas entre si, apresentam 
alguma omissão de detalhes ou
uso de frases diferentes.

A \emph{Crônica} foi feita com base em crônicas bizantinas, sendo muito
semelhante à \emph{Cronogaphia}, de João Malalas (século~\versal{VI}), e ao
\emph{Chronicon}, de George Hamartolos (século~\versal{IX}). Ainda que esses
textos serviram como base para o formato da \emph{Crônica} e para
algumas de suas passagens -- como as datas de eventos ocorridos em Bizâncio e as
diversas citações de cunho religioso --, o conteúdo da compilação teve a
memória do povo rus como matéria-prima. O fato de haver um número
consideravelmente pequeno de fontes escritas que tratem do período da
Rus de Kiev faz com que a \emph{Crônica} seja a principal fonte
utilizada por especialistas, mesmo sendo realisticamente muito tardia em
comparação aos eventos lá descritos. É consenso entre os pesquisadores
que as entradas a partir da segunda metade do século~\versal{XI} foram
testemunhados diretamente pelos cronistas, sendo mais seguro trabalhar
com a \emph{Crônica} como fonte base para o estudo desse período.

Após descrever o dilúvio bíblico e nomear as etnias provenientes dos
filhos de Noé conhecidas pelo cronista, os eventos descritos na
\emph{Crônica} antes do batismo do príncipe Vladimir~\versal{I} Sviatoslavich
(980-1015) glorificam os varegues que se estabeleceram na Rússia
europeia no século~\versal{IX}. Conforme a fonte, os aldeões eslavos pediram a
Riúrik, fundador da dinastia Ruríquida que permaneceu no poder na Rússia
até 1598, bem como a seus irmãos Sineus e Truvor, para que governassem a terra
que seria Rus. A \emph{Crônica}, então, narra os acontecimentos que
envolviam Constantinopla e os povos eslavos, assim como os feitos dos
líderes varegues pagãos, como Oleg, o Profeta (882-912); Igor Riurikovich
(912-945); Olga (945-964), eventualmente batizada em meados do
século~\versal{X}; Sviatoslav~\versal{I} Igorevich (964-972); Iaropolk Sviatoslavich
(972-980); e Vladimir~\versal{I}, que se converteu ao cristianismo de rito grego
no final do século~\versal{X}. O foco começa a ser realmente os varegues e Kiev a
partir da entrada de 6452 (944). Igualmente a partir desta data, as menções a
Bizâncio e aos povos das estepes passam a aparecer somente quando têm relação
direta com os varegues ou com os rus.

A partir do batismo de Vladimir~\versal{I} e da cristianização da Rus, o teor da
narrativa passa a ser mais moralista e os cronistas intervêm com mais
frequência ao narrarem os acontecimentos. A partir do conflito
fratricida pelo trono de Kiev ocorrido após a morte de Vladimir, a
compilação foca nos governos dos príncipes de Kiev: Sviatopolk
Vladimirovich, o Amaldiçoado (1015-1016, 1018-1019); Iaroslav
Vladimirovich, o Sábio (1016-1018, 1019-1054); Iziaslav Iaroslavich
(1054-1068, 1069-1073, 1077-1078); Sviatoslav~\versal{II} Iaroslavich
(1073-1076); Vsevolod Iaroslavich (1076-1077, 1078-1093); Sviatopolk~\versal{II}
Iziaslavich (1093-1113); e Vladimir Vsevolodovich Monômaco (1113-1125).
Os príncipes de outras partes da Rus, como Novgorod e Chernigov, também
aparecem esporadicamente. A presença de varegues e escandinavos em geral
é escassa, sendo mencionados esporadicamente como membros reunidos para
um determinado exército. A versão laurentiniana vai até o ano de 6624
(1116), mas as outras versões vão mais adiante devido à adição de
eventos posteriores.

O tratamento dado pelos cronistas aos escandinavos é inconsistente. Por
um lado, em sua maioria, os varegues são celebrados e os cronistas
assumem a identidade dos rus como sendo descendentes de escandinavos. Os
varegues martirizados por serem cristãos enquanto Vladimir~\versal{I} ainda era
pagão são celebrados na \emph{Crônica}. Por outro lado, os cronistas
geralmente criticavam alguns varegues que eram pagãos por serem
``ignorantes'', como Oleg, o Profeta, Sviatoslav~\versal{I} e os primeiros anos de
Vladimir~\versal{I}. Um varegue em particular, o comandante e possível príncipe
Sveneld, é responsabilizado indiretamente pela morte de Igor e diretamente
pelo conflito entre os irmãos Iaropolk Sviatoslavich e Oleg
Sviatoslavich. A \emph{Crônica} também narra sobre a crueldade dos
varegues comandados por Iaroslav, em Novgorod, e a subsequente revolta do
povo do principado. É interessante perceber que mesmo os nórdicos pagãos
são elogiados nas passagens sobre batalhas, tendo seu aspecto guerreiro exaltando sempre que
possível, mesmo nas derrotas.

\SIG{Leandro César Santana Neves}

Ver também: Kiev; Rus de Kiev; Rússia da Era Viking; Novgorod; Olga de
Kiev; Staraia Ladoga; Varegues; Vladimir~\versal{I} de Kiev.

\begin{itemize}
\item \versal{FRANKLIN}, Simon. \emph{Writing, Society and Culture in Early Rus, c.
950-1300}. Cambridge: Cambridge University Press, 2004.

\item \versal{LIKHACHIOV}, Dmitry S. (ed.). \emph{Slovar Knizhnikov i Knizhnosti
Drevnei Rusi. Vyp.~\versal{I} (\versal{XI} ‒ pervaia polovina \versal{XIV} v.)} [\emph{Dicionário de escribas e de livros da Rus Antiga. vol. 1 (séculos~\versal{XI} ‒ primeira
metade do~\versal{XIV})}]. Leningrado: Náuka, 1987.

\item \versal{OSTROWSKI}, Donald (org.). \emph{The Povest' Vremennykh Let: An
Interlinear Collation and Paradosis}. 3 Vols. Cambridge: Harvard
University Press, 2004.

\item  \emph{\versal{Povest Vremennykh Let}} [\emph{Crônica dos Anos
Passados}]. Traduzido e comentado por Dmitry S. Likhachiov e
revisão de Varvára P. Adrianova-Peretts. São Petersburgo: Nauka, 1996.

\item  \emph{\versal{The Russian Primary Chronicle: Laurentian Text}}. Editado e
traduzido por Samuel Hazzard Cross e Olgerd P. Sherbowitz-Wetzor.
Cambridge: The Medieval Academy of America, 1953.

\item \versal{TOLOCHKO}, Oleksiy. On `Nestor the Chronicler'. \emph{Harvard Ukrainian
Studies}, vol. 29, nº 1/4, 2007, pp. 31-57.
\end{itemize}

\section{\versal{CULTURA MATERIAL}}

Ver Âmbar; Armamento; Arquearia; Arqueologia da Era Viking; Arte;
Bracteatas; Bússola solar; Cemitério de Borg; Embarcações; Espada;
Ferreiros e ferraria; Fortificações; Funerais e enterros; Gokstad;
Habitação; Joias e ourivesaria; Metalurgia; Mobiliário; Moedas e
cunhagem; Pedra solar; Pentes; Sepultamentos; Tapeçaria de Bayeux;
Tapeçaria de Oseberg; Tapeçarias de Överhogdal; Tapeçaria de Skog;
Tecelagem; Vestuário.


\chapter{D \textarn{d} \textarc{d} \textart{d}}

\section{\versal{DANEGELD}}

Com a formação do Grande Exército Pagão viking nas terras inglesas, os ataques
constantes no interior da ilha foram cada vez mais frequentes. Contudo,
os vikings perceberam que havia uma outra forma mais fácil de acumular
tesouros sem arriscar a vida de seus homens. Eles
aprenderam a extorquir dinheiro e levaram essa prática para todos os
territórios que atacavam, não só às Ilhas Britânicas, mas também a
toda a Europa.

Quando um bando viking sitiava uma cidade ou desembarcava em suas
proximidades, muitas vezes apenas demonstrava seu poder de combate
e isso, normalmente, era o suficiente para deixar seus inimigos
propensos a evitar o enfrentamento. Os oponentes imaginavam que, diante do poderio viking, seriam massacrados,
razão pela qual buscavam outros meios para se livrar da presença dos saqueadores. Dessa
forma, os reinos ingleses do século~\versal{IX} começaram a pagar os vikings para
que fossem embora. O primeiro registro disso foi em 865 d.C, conforme
consta nas \emph{Crônicas Anglo-saxônicas}.

Essa prática ficou conhecida nos anais ingleses como \emph{danegeld},
que, em tradução literal, quer dizer ``ouro dinamarquês'', ou seja, uma quantia
de dinheiro para que os invasores -- que, em sua esmagadora maioria, eram
compostos de colonos e guerreiros dinamarqueses -- deixassem aquele povo
em paz. Ainda que os nórdicos da região partissem logo após receber as
quantias, não havia muito tempo de paz, pois logo em seguida outro
local se tornava alvo das ameaças. Cada vez mais dinheiro era concedido,
gerando um acúmulo de riquezas e uma zona de conforto que fomentou o
estabelecimento desses povos na região, bem como a constituição de uma
verdadeira província dinamarquesa na Inglaterra anglo-saxã, o
Danelaw.

\SIG{Ricardo Wagner Menezes de Oliveira}

Ver também Danelaw; Dinamarca da Era Viking; Inglaterra da Era Viking;
Moedas e cunhagem.

\begin{itemize}
\item \versal{BRØNDSTED}, Johannes\emph{. Os Vikings}. São Paulo: Hemus, s.d.

\item \versal{GRAHAM-CAMPBELL}, James. \emph{Os vikings}. São Paulo: Folio, 2006.

\item \versal{HAYWOOD}, John. Danegeld. In: \emph{Encyclopaedia of the Viking Age}.
London: Thames and Hudson, 2000, pp. 51.

\item \versal{HOLMAN}, Katherine. \emph{Historical dictionary of the vikings}. Oxford:
The Scarecrow Press, 2004, pp. 73-74.
\end{itemize}

\section{\versal{DANELAW}}

As sucessivas incursões vikings, que ocorreram na Inglaterra anglo-saxã
durante os fins do século~\versal{VIII} e o início do século~\versal{IX}, trouxeram para
as Ilhas Britânicas uma grande quantidade de indivíduos escandinavos,
que, em sua esmagadora maioria, possuíam origem dinamarquesa. Essas populações
não chegaram somente para saquear, mas também buscaram se estabelecer e
formar uma comunidade agrícola.

O domínio exercido pelas forças dinamarquesas em território inglês --
obtido já no fim do século~\versal{VIII} através das batalhas contra os reinos da Nortúmbria, Mércia e
Wessex -- possibilitou o surgimento de diversos
assentamentos dinamarqueses, bem como a integração dessas populações em
centros urbanos já estabelecidos. As marcas dessa ``colonização'' podem
ser percebidas até os dias de hoje através da análise da toponímia, como
em nomes de localidades com terminações \emph{-by e} \emph{--thorp}, tipicamente
escandinavas.

Essas regiões onde os dinamarqueses se estabeleceram foram conquistadas
dos anglo-saxões. Os seus novos líderes começaram a impor suas
normas, cobrar impostos e tomar demais medidas administrativas, definindo uma
verdadeira província dinamarquesa, onde a cultura, a língua e a
administração passaram a ser moldadas por estrangeiros. Assim,
instaurou-se o Danelaw, forma como ficou conhecida a região
controlada pelos dinamarqueses, que se estendia ao norte do rio Tâmisa até
Chester, bem como para o leste até o Mar do Norte.

O Danelaw não possuía capital ou unidade política definida. O
poder da região estava distribuído entre os chefes-guerreiros que se
assentaram nas chamadas Cinco Aldeias, sendo elas Lincoln, Stamford,
Leicester, Nottingham e Derby. Outro centro de grande poder estrangeiro
na região foi a cidade de York, porém esta ficou a maior parte do tempo sob
o domínio norueguês. Vale ressaltar que os governantes nessas cidades e
vilas nem sempre estavam de acordo entre si, tampouco permaneciam no
poder por longos períodos.

Essa forma de assentamento praticada pelos dinamarqueses, pelo que indicam
as pesquisas, não incluía o extermínio ou escravização da população
trabalhadora anglo-saxã. É bem provável que os invasores tenham tomado
para si as melhoras porções de terra, mas isso não implica que os
habitantes locais tenham sido expulsos. A permanência desses indivíduos
fomentou uma rica troca cultural entre os dois povos, pois, ainda que
ambos os povos tenham tido uma origem germânica, viveram
desenvolvimentos históricos muito diferentes.

Um dos pontos mais marcantes dessa coexistência, além da já referida
toponímia, são os aspectos culturais, como os estilos artísticos e a
mitologia. Durante os séculos seguintes à instauração
dinamarquesa, muitos dos mitos nórdicos, além do gosto nórdico pela arte,
foram disseminados. A cultura local passou a ter um caráter
híbrido, no qual aspectos nórdicos e anglo-saxões, pagãos e cristãos,
locais e estrangeiros, se tornaram comuns a ambos os povos.

O Danelaw durou por muitas décadas, até o século~\versal{XI}. Após
o falecimento, em 1035, do rei dinamarquês Canuto, o Grande (que havia sido coroado rei da
Inglaterra, Dinamarca, Noruega e parte da Suécia), seu
império tornou-se fragmentado e a influência dinamarquesa na Inglaterra foi
gradualmente diminuindo. A coroação de Eduardo, o Confessor,
como rei da Inglaterra, em 1042, praticamente encerrou o expressivo
controle estrangeiro na região, sendo um dos sinais de que já se
adiantava o fim da Era Viking.

\SIG{Ricardo Wagner Menezes de Oliveira}

Ver também Danegeld; Dinamarca da Era Viking; Inglaterra da Era Viking.

\begin{itemize}
\item \versal{BRØNDSTED}, Johannes. \emph{Os Vikings}. São Paulo: Hemus, s.d.

\item \versal{GRAHAM-CAMPBELL}, James. \emph{Os vikings}. São Paulo: Folio,
2006.

\item \versal{HAYWOOD}, John. Danelaw. In: \emph{Encyclopaedia of the Viking Age}.
London: Thames and Hudson, 2000, pp. 51-52.

\item \versal{HOLMAN}, Katherine. Danelaw. In: \emph{Historical dictionary of the
vikings}. Oxford: The Scarecrow Press, 2004, pp. 74-75.
\end{itemize}

\section{\versal{DANEVIRKE}}

Consta nos anais francos que, no ano de 808 d.C., o rei dinamarquês
Godofredo mandou que a fronteira de suas terras com as terras dos saxões
ao sul fosse fortificada por uma longa muralha que possuísse apenas um
portão para passagem de comerciantes e viajantes. Ele teria definido que
ela deveria começar na baia Ostersalt, ao leste, e seguisse para o oeste
até encontrar o oceano. Esta longa série de fortificações ficou
conhecida como Danevirke.

Pesquisadores como James Graham-Campbell chamam a atenção para o fato de
que nos séculos anteriores ao século~\versal{IX}, os dinamarqueses não tinham o
costume de fortificar cidades, o que pode indicar certo contexto pacífico. 
Contudo, as técnicas de erguer muralhas foram aprimoradas e o
Danevirke é um bom exemplo desse caso, pois já em seu surgimento, uma
parte de uma antiga elevação de terra construída 75 anos antes fora
incorporada a esta muralha.

Quando concluída, a obra de Godofredo era uma impressionante
construção que possuía cerca de 30 quilômetros de extensão. Iniciando-se
junto as muralhas da cidade de Hedeby, o muro de terra e paliçada seguia
para o oeste até o lago Danevirke, e de lá, seguia a sudoeste até
adentrar uma região pantanosa próxima ao litoral do Mar do Norte,
dividindo a península da Jutlândia em seu estreito ao sul.

O Danevirke pode ser entendido como um complexo de fortificações pelo
fato de que, ao longo dos anos, foi diversas vezes restaurado,
reforçado, aprimorado, alongado e anexado a outros muros menores. 
Durante o governo de Haroldo Dente Azul, 160 anos depois de
Godofredo, uma última parte da muralha foi implementada e reforçada com
pedras, adquirindo, em algumas partes, 13 metros de largura. Porém,
em períodos ainda posteriores à Era Viking, ela continuou a ser
utilizada: foi reforçada na guerra contra a Prússia e
mesmo quando recebeu defesas antitanques durante a Segunda Guerra
Mundial.

Atualmente, a região do Danevirke pertence ao território alemão e possui
diversas aberturas em sua extensão, para passagem de trens e outros
veículos. Um museu dedicado ao muro foi construído na cidade de
Schleswig, localizada próxima à antiga Hedeby. Sua função de
defesa e demarcação de fronteiras não é mais utilizada, tornando-o hoje,
em sua maior parte, apenas um relevo coberto de relva e árvores no
horizonte de uma área rural, mas que dá, ainda, uma ideia da imponente
expressão de poder que representou em seu passado.

\SIG{Ricardo Wagner Menezes de Oliveira}

Ver também Dinamarca da Era Viking; Fortificações; Guerra e técnicas de
combate.

\begin{itemize}
\item \versal{BRØNDSTED}, Johannes\emph{. Os vikings}. São Paulo: Hemus, s.d.

\item \versal{GRAHAM-CAMPBELL}, James. \emph{Os vikings}. São Paulo: Folio, 2006.

\item \versal{HAYWOOD}, John. Danevirke. In: \emph{Encyclopaedia of the Viking Age}.
London: Thames and Hudson, 2000, pp. 51-53.

\item \versal{HOLMAN}, Katherine. Danevirke. In: \emph{Historical dictionary of the
vikings}. Oxford: The Scarecrow Press, 2004, pp. 75-76.
\end{itemize}

\section{\versal{DINAMARCA DA ERA VIKING}}

O reino dinamarquês da Era Viking abrangia a península de Jylland, as
ilhas de Fyn, Sjælland, várias outras ilhas menores e as partes mais
ao sul do atual território sueco, como as províncias de Skåne, Halland e
Blekinge. Duas pedras rúnicas encontradas em Jelling são
tradicionalmente consideradas monumentos que inauguram a história
dinamarquesa propriamente dita. Na menor das pedras há uma inscrição
dizendo ``O Rei Gorm fez esse monumento em honra à sua esposa, Tyre, o
orgulho da Dinamarca'', enquanto que na pedra maior encontram-se os
dizeres ``O Rei Haroldo ordenou que esses monumentos fossem feitos em
honra ao seu pai, Gorm, e sua mãe, Tyre -- sendo Haroldo que conquistou
para si toda a Dinamarca, a Noruega, e fez dos dinamarqueses cristãos''.
Até o presente momento não foi possível apontar com
precisão a data da criação desses monumentos, mas acredita-se, com base nos
nomes reais neles mencionados, que as pedras rúnicas de Jelling 
datem da segunda metade do século~\versal{X}.

As inscrições nos apresentam às duas primeiras gerações da dinastia
real que governou a Dinamarca enquanto reino. Conforme aponta Inge
Skovgaard-Petersen, nos séculos~{\versal{IX} e \versal{X}} o título de \emph{konungr} (rei)
pode ter sido usado por vários pequenos governantes simultaneamente, mas
nas pedras de Jelling o termo fora reservado para o único governante de
todo o reino dinamarquês -- e de parte da Noruega. Foi também nessas
pedras em que, com a escrita rúnica, o nome ``Dinamarca'' foi registrado
pela primeira vez.

As principais fontes para estudo da Dinamarca da Era Viking são escritas
e arqueológicas. Há, nos \emph{Anais Reais Franceses}, escritos por volta dos
anos 800, um tópico acerca das relações entre o reino franco e os reis
dinamarqueses. Entre outros escritos estrangeiros sobre a Dinamarca
encontram-se as obras \emph{Vita Anskarii}, de Rimbert, escrita por
volta de 875, e a \emph{Gesta Hammaburgensis ecclesiae pontificum}, de
Adão de Bremen, já mais posterior, datando de 1075. Esta última é
considerada a fonte estrangeira mais importante no que diz respeito aos
primórdios não só da história dinamarquesa, mas da Escandinávia de
maneira geral.

Contudo, grande parte desses materiais escritos, por serem estrangeiros,
lidam muitas vezes com a questão dinamarquesa de maneira periférica.
Além disso, essas são obras com fins apologéticos e missionários que não
podem, portanto, ser consideradas fontes imparciais. Em suma, ressalta
Skovgaard que os escritos geralmente encontrados sobre a formação do
reino dinamarquês e a Dinamarca viking são fontes posteriores, já do
período medieval, que oferecem pareceres tendenciosos e muitas vezes
escassos sobre o assunto em questão. As fontes arqueológicas representam 
uma outra alternativa de pesquisa.

A principal hipótese acerca da unificação dinamarquesa em um só reino
seria a de que vários dos pequenos reinos que existiam inicialmente
foram sendo englobados em reinos cada vez maiores, algumas vezes por
meio de conquistas militares, outras por amalgamação pacífica. Esse
processo teria continuado até que tivesse sido formado um único reino
dinamarquês sobre controle inteiramente do mesmo governante. É comum
atribuir este feito ao rei Haroldo, quem, no ano de 970 d.C, ordenou que
fossem feitas as supracitadas pedras de Jelling para celebrar seu feito.
Depois de Haroldo, a Dinamarca foi comandada sempre por um único rei --
com apenas algumas pequenas e pontuais exceções no século~\versal{XII}.

Contudo, Olaf Olsen apresenta a hipótese de que o rei Haroldo não teria
unificado a Dinamarca, mas a reunificado. Sua teoria se baseia em
relatos advindos dos \emph{Anais Imperiais do Reino Franco}, que narram quando,
em 808 d.C, o exército franco liderado por Carlos Magno invadiu a
fronteira sul da Dinamarca. Na ocasião, é mencionado um rei chamado
Godofredo, responsável por ter fortificado a fronteira de Jylland com
uma grande muralha que ia desde o Mar Norte, a oeste, até o Mar Báltico,
no leste. Uma estrutura desse porte não poderia ter sido levada a cabo
por um rei sem grande significância, influência, e que não possuísse
grande poder de centralização. Acredita-se que sua construção tenha
demandado a participação de milhares de homens, além de uma organização
efetiva, dirigida por um rei que tivesse grandes recursos -- provavelmente
um rei de todo o território dinamarquês, ou ao menos de toda a península
de Jylland.

Essa fortificação, chamada Danevirke, era então creditada ao rei
Godofredo, tido até recentemente como o primeiro rei da Dinamarca. No
entanto, certas descobertas arqueológicas apontadas por Olsen apresentam
fortes argumentos em favor de um rei anterior, mais antigo, que teria
governado todo o território: um exame dendrocronológico da madeira
presente em 7 quilômetros da muralha de Danevirke indica a mesma data de
construção: 737 d.C. O rei Godofredo não poderia, então, ter sido o
primeiro a construir a fortificação. Aparentemente, ele teria
fortificado uma construção que, na verdade, já existia há pelo menos
80 anos. Assim, há fortes indícios de que tenha existido um poderoso
rei, por volta dos anos 737 d.C, tão poderoso quanto Godofredo. Outras
fontes, como as escritas, não oferecem indícios claros de quem teria
sido esse rei, seu nome ou seus feitos, tornando esse período da
história dinamarquesa de certa forma obscuro. É possível que o rei
Ongendus -- Angantyr em nórdico antigo -- tenha sido o responsável pela
construção da fortificação, já que provavelmente era o governante da
região de Jylland onde ela foi construída, no período de 737 d.C. De
qualquer forma, a existência de um poderoso rei governando na Dinamarca
do século~\versal{VIII} teria sido uma condição \emph{sine qua non} para que o
Danevirke pudesse ter sido construído.

Outro ponto central na história do reino dinamarquês foi o núcleo de
estratégias utilizadas pelas elites centrais -- reis e \emph{Jarls} --
para integrar as províncias da Dinamarca e unificá-la aos poucos. Alguns
fizeram o uso direto e coercitivo da força, outros utilizaram de seu
poder e influência: houve a promoção e consolidação do militarismo, que
facilitou o controle real sobre cada vez mais pessoas e lugares; a
manipulação de antigos códigos de lei para minar certos direitos e
obrigações dos homens livres; a construção de grandes centros urbanos e,
por fim, a adoção de uma nova religião, o cristianismo, utilizado como
ferramenta para aumentar e sustentar novas relações de poder, mais
abrangentes e influentes.

Os fatores decisivos para unificação da Dinamarca enquanto reino foram,
conforme aponta Tina Thurston, o militarismo, a transformação do aparato
jurídico, a urbanização dos grandes centros e a ascensão do comércio.
Além da construção de Danevirke, outro projeto monumental foi a criação
do canal de Kanhave, criado entre os anos de 728 a 737 d.C. Este canal
dividia toda a ilha de Samsø, visando proporcionar a rápida movimentação
marítima de tropas por todo o arquipélago dinamarquês. Posteriormente,
por volta do século~\versal{X}, surgiram várias fortificações militares
espalhadas por todo o reino, construídas em pontos
estratégicos que permitiam o controle e a vigilância militar de toda
Dinamarca.

As leis, passadas oralmente durante a Era Viking, também foram sendo
modificadas aos poucos. As assembleias -- \emph{Thing} -- provinham a todos
os dinamarqueses o direito de registrar legalmente suas queixas e
resolver suas disputas legais. Elas existiam não somente como
um lugar onde as pessoas se reuniam para participar no governo, mas
eram também onde os reis mantinham sua corte oficial. Havia as pequenas
\emph{Thing}, onde aqueles que moravam no mesmo distrito se encontravam,
e também as \emph{Landstings}, maiores, que eram uma espécie de
assembleia regional. Dentre as principais \emph{Thing}, havia Viborg na
Jutlândia, Odense em Fyn, Lund na Scania e Ringsted em Sjælland.
Curiosamente, todos esses lugares tinham conexões com o sagrado do
paganismo e, posteriormente, vínculos com a Igreja: Viborg significa
``lugar de oferenda na colina''; Odense, ``lugar de oferenda a Odin'';
Lund, por fim, significa ``arvoredo sagrado''. Além de servir como
assembleia regional, Viborg era a maior \emph{Thing} de toda nação
dinamarquesa da Era Viking: era lá onde o rei da Dinamarca era eleito
pelo povo.

Segundo Thurston, no quesito urbanização, a ideia de cidade era um
conceito completamente estranho durante a Era Viking. Ainda assim, aos
poucos foi se consolidando uma hierarquia urbana relacionada ao poder
real, começando a intervir em questões como o comércio ou questões
políticas importantes. Esses núcleos administrativos foram
construídos em lugares onde fosse possível aproveitar os recursos
naturais de maneira favorável, como em canais, portos e em terras
defensáveis que se conectavam a locais economicamente importantes --
estradas, pontes, fiordes, rotas de comércio, mercados e locais de
pesca. Os primeiros centros urbanos da Dinamarca foram as cidades de
comércio próximas à costa, sendo o mais antigo deles a cidade de Ribe,
na parte sul da Jutlândia.

A religião adotada pela Dinamarca da Era Viking, o paganismo nórdico,
não possuía um corpo dogmático centralizado. A própria ideia de
religião não circulava entre os dinamarqueses, que sequer tinham uma
palavra que representasse esse conceito. Ao invés disso, circulava a
palavra \emph{sidr}, que significava algo como ``costume''. Nenhum dogma
específico, chefes religiosos ou templos são mencionados até um período
mais tardio e provavelmente pertencem já ao fim da Era Viking, fruto de
um movimento de resposta ao crescente cristianismo e à construção de
igrejas -- e, portanto, não representavam o jeito costumeiro dos
dinamarqueses conceberem sua crença.

Como não possuía um corpo unificado, a religiosidade viking apresentava
diferenças sociais, temporais e geográficas. Outra razão para que
isso acontecesse era o fato de que os rituais pré-cristãos eram
conduzidos de maneira pessoal, numa relação diretamente situada entre os
homens e os deuses, sendo costumeiro que oferendas e a comunicação com o
divino fossem conduzidas particularmente. Não existiam regras a respeito
de como a pessoa deveria se dirigir ao sobrenatural, aos deuses, ou como
entrar em contato com eles, apesar de alguns encantamentos e entoações
desse período terem sido preservados. Era comum que casamentos, rituais
de passagem, juramentos e punições fossem testemunhados de maneira mais
pública e comunitária, acompanhados por rituais religiosos específicos.

Estudos recentes conduzidos pelo arqueólogo Søren Sindbæk alegam que a
Dinamarca foi a responsável por proporcionar as condições sociais e
materiais que inaugurariam o que chamamos de Era Viking. Os estudos em
questão apontam que viagens marítimas eram realizadas entre a Noruega e
Ribe, o centro comercial dinamarquês mais antigo, muito antes da Era
Viking ter começado oficialmente. É possível, portanto, que os vikings
tenham começado a desenvolver sua cultura marítima e a estabelecer
relações comerciais ainda no ano de 725 d.C; muito antes da invasão à
Lindisfarne, em 793 d.C.

Os achados arqueológicos em questão eram chifres de rena, utilizados
sobretudo na produção de pentes. No entanto, os chifres encontrados em
Ribe eram de uma espécie de rena típica do território norueguês. Tal
descoberta aponta para alguns importantes fatos: primeiramente, que os
vikings realizavam grandes viagens marítimas já no começo do século~\versal{VIII}; que essas famosas viagens marítimas eram feitas para que se
estabelecessem relações de comércio, e não somente para invasões, como se
pensava; por fim, que Ribe, a cidade mais antiga na Dinamarca, já era
desenvolvida o bastante, também no século~\versal{VIII}, para ser palco de um
intenso comércio que abarcava visitantes estrangeiros de regiões
distantes.

Foi descoberta, no ano de 2017, uma sepultura em território dinamarquês
que continha os restos de um viking, provavelmente de alta posição
social e, junto dele, uma rédea de cavalo com detalhes de
ouro. Segundo arqueólogos, esse tipo de rédea só estaria acessível para
pessoas altamente poderosas durante a Era Viking, podendo ter
sido um presente do rei em agradecimento a algum tipo de aliança. Os
acessórios datam do ano de 950 d.C., o que significa que os restos
encontrados podem ser de um grande aliado do rei Gorm, o Velho.

\SIG{Victor Hugo Sampaio Alves}

Ver também Canuto; Danevirke; Danelaw; Era Viking; Fortificações;
Viking.

\begin{itemize}
\item \versal{ARCHAEOFEED}. High status early Viking Age grave found in Denmark.
\emph{ArchaeoFeed}, 2017. Disponível em:
\href{https://archaeofeed.com/2017/03/high-status-early-viking-age-grave-found-in-denmark/}{\emph{https://archaeofeed.com/2017/03/high-status-early-viking-age-grave-found-in-denmark/}}
Acesso em: 25 jul. 2017.

\item \versal{OLSEN}, Olaf. Royal Power in Viking Age Denmark. In: \emph{Les mondes
normands (\versal{VIII}e-\versal{XII}e s.). Actes du deuxième congrès international
d'archéologie médiévale} (Caen, 2-4 octobre 1987) Caen: Société
d'Archéologie Médiévale, 1989, pp. 27-32.

\item \versal{PERSSON}, Charlotte Price. The Viking Age began in Denmark. \emph{Science
Nordic}, 2015. Disponível em:
\href{http://sciencenordic.com/viking-age-began-denmark}{\emph{http://sciencenordic.com/viking-age-began-denmark}}.
Acesso em: 25 jul. 2017.

\item \versal{SKOVGAARD-PETERSEN}, Inge. The making of the Danish Kingdom. In: \versal{HELLE}, Knut (org.). \emph{The Cambridge History of Scandinavia}. Cambridge:
Cambridge University Press, 2003, pp. 168-183.

\item \versal{THURSTON}, Tina. \emph{Landscapes of power, landscapes of conflict: State
formation in the South Scandinavian Iron Age}. New York: Kluwer Academic
Publishers, 2002.
\end{itemize}

\section{\versal{DIRHEM (MOEDAS ÁRABES)}}

A quantidade enorme de dirhem (moedas árabes) encontradas em toda a
Escandinávia e espalhada por várias regiões da Europa deve-se
principalmente às expedições feitas pelos vikings realizadas entre os
séculos~{\versal{IX} e \versal{X}}, ao Leste Europeu e Oriente Médio. Grande parte das
dirhens encontradas no norte da Europa estavam fragmentadas, a
explicação para tal fato segundo Brøndsted é que durante a maior parte
do Período Viking, a moeda, como tal, não tinha valor simbólico, era
somente um pedaço de metal precioso. Assim, ao serem divididas o seu
valor era pesado em balanças para que os nórdicos conseguissem a medida
exata para comercializar seus produtos.

Segundo Brown, mais de 85 mil moedas árabes foram desenterradas na
Escandinávia na época moderna -- mais da metade delas foi
descoberta em Gotland, devido a sua localização no centro do Mar Báltico,
a ilha havia se tornado uma das escalas preferidas dos mercadores
nórdicos que navegavam pela rota oriental de comércio. Em um estudo mais
recente, Kovalev acredita que em média 125 milhões dirhens inteiros
tenham sido exportados para o norte da Europa, 
não somente obtidos por meio do comércio com o oriente, mas também
fruto de incursões viking realizadas nas regiões dominadas pelos árabes
da Espanha até o leste do Afeganistão.

Juntamente com as incursões e o comércio viking pela rota do Oriente, a
difusão dos dirhens pela Europa teve a contribuição de mercadores
muçulmanos, que passaram a utilizar os entrepostos comerciais entre os
rios Don e Volga. Uma das principais fontes que relatam esse intenso
comércio é a descrita pelo cronista árabe Ibn Fadlan, enviado pelo
califa de Bagdá como embaixador ao reino dos búlgaros. Fadlan
testemunhou as relações comerciais entre os vikings e os mercadores nas
margens do Volga. A crônica de Fadlan, denominada \emph{Risala},
descreve, entre outras coisas, um dos rituais para que o comércio fosse
bem-sucedido: nesse relato é descrito que um nórdico pede a um ídolo de
madeira que traga um comerciante com muitos dinares e dirhens para
comprar seus produtos.

A descrição feita por Fadlan demonstra como os dirhens eram muito bem
recebidos pelos nórdicos, segundo Campbell a ponto de se tornarem o
metal mais apreciado no norte da Europa. Estima-se que a quantidade de
moedas encontradas tenha sido pequena em relação as que foram
exportadas, o que se deve ao fato delas terem sido em sua grande maioria
fundidas e transformadas em joias ou barras de prata.

No século~\versal{XI} o comércio escandinavo no Oriente entra em declínio,
juntamente com a exportação dos dirhens para os países nórdicos. Dentre
os principais motivos para o fim das relações comerciais estão a
cristianização da Escandinávia e a proibição pela Igreja do comércio de
escravos, sobretudo escravos cristãos que eram vendidos aos muçulmanos
em troca dos dirhens. Kovalev acredita também que o fim das relações
comerciais ocorreu devido ao esgotamento das minas de prata nos países
árabes, o que diminuiu o teor desse metal nas moedas, levando as atividades comercias a
entrarem em colapso.

\SIG{Marlon Ângelo Maltauro}

Ver também Árabes e vikings; Comércio; Ibn Fadlan; Moedas e cunhagem.

\begin{itemize}
\item \versal{BRØNDSTED}, Johannes. \emph{Os Vikings}. São Paulo: Hemus, s.d.

\item \versal{BROWN}, Dale W. (ed.). \emph{Os Vikings: intrépidos navegantes do Norte}.
São Paulo: Abril/Time Life, 1999, pp. 64-65.

\item \versal{GRAHAM-CAMPBELL}, James. \emph{Os viquingues: origens da cultura
escandinava}. Madri: Edições Del Padro, 1997, pp. 197-198.

\item \versal{KOVALEV}, Roman K. Dirham Mint Output of Samanid Samarqand and its
Connection to the Beginnings of Trade with Northern Europe (10th
century).~\emph{Histoire \& mesure}, vol. 17, ns. 3-4, 2006, pp. 197-216.
\end{itemize}

\section{\versal{DORESTAD}}

Dorestad, Dorostate ou Dorestadum, também conhecida nos registros
medievais como \emph{emporium} (cidade mercado), foi um próspero ponto
de comércio inter-regional e o principal centro comercial da Frísia
até meados do século~\versal{IX}. Segundo Brown, Dorestad foi um dos mais ricos
mercados de toda Europa ocidental pela intensidade do comércio e
por ser o local onde eram cunhados os \emph{deniers}, as moedas de prata
de Carlos Magno.

Embora haja um intenso debate entre arqueólogos sobre o local exato de
Dorestad, Brøndsted relata que arqueólogos holandeses encontraram no
centro da Holanda vestígios de uma fortaleza carolíngia, construída
provavelmente no governo de Carlos Magno. Situava-se entre o rio
Reno e vários outros rios menores, sendo ponto de
ligação comercial entre o Império Carolíngio, a Inglaterra e o norte da
Escandinávia, o que explicaria o sucesso da região como zona de
mercancia.

Fontes cristãs relatam que as primeiras incursões vikings a Dorestad
ocorreram por volta de 834 d.C. Cooijmans descreve em sua pesquisa o
relato do missionário cristão Luidger, que teria visto, em um sonho premonitório, 
o desaparecimento do sol, afugentado por terríveis nuvens que
escureceram a terra, seguido de hordas do norte que traziam guerra
e destruição e arrasavam a terra. Após o ataque inicial, os
nórdicos saquearam a cidade por três verões subsequentes até que
Dorestad lhes fosse cedida.

Luit acredita que a concessão feita por Lotário~\versal{I} aos vikings tivesse
dois objetivos: mostrar a insatisfação referente à partilha dos
territórios do Império Carolíngio feita por seu pai Luís, o Piedoso, a
seus descendentes, e forçar os próprios
escandinavos a defender a região de novos ataques nórdicos.

Em 850 d.C. o comando de Dorestad foi cedido ao líder dinamarquês Horik.
Nesse período, a cunhagem das moedas na cidade já havia chegado ao fim e
começavam a aparecer os primeiros sinais de sua decadência. Horik teve como
incumbência cobrar impostos dos moradores da região, bem como defender a
costa da Frísia. Luit descreve que o governo da cidade e a alta
cobrança de impostos exercida pelos pagãos passou a desagradar os
comerciantes locais, o que fez com que eles começassem a se mudar para
outros locais como Tiel e Deventer, cidades próximas que não eram
parte dos territórios governados pelos dinamarqueses.

Horik aceitou o batismo e se converteu ao cristianismo em 860 d.C.
A conversão provavelmente ocorreu graças à pressão do Império
cristão juntamente com a tentativa de fazer com que a cidade voltasse a
ter o esplendor comercial de outras épocas, já que desde 830 d.C. 
vinha perdendo sua importância. Em 870 d.C., Dorestad foi
invadida mais uma vez por uma horda de vikings que aproveitaram enquanto
Horik estava defendendo o norte da Saxônia. Saquearam a cidade até que Horik retornasse, 
agindo como mediador para que os piratas concordassem em recuar.

Após a morte de Horik, o comando de Dorestad passou para Gurdrod e o local não
foi mais mencionado nos registros históricos. Várias hipóteses sobre o
eventual desaparecimento de Dorestad foram apresentadas, dentre elas a
mais popular relata como motivo principal do colapso as incursões e sua
concessão aos vikings, já que as fontes cristãs relatam que
dinamarqueses pagãos e, portanto, tido como incivilizados, seriam incapazes de
fazer a região prosperar.

Cooijamans relata em sua pesquisa que a hipótese acima mencionada sobre
a visão cristã referente à decadência da cidade tinha o óbvio intuito de
depreciar os povos pagãos. Para ele, as razões do declínio foram outras, como os novos
tratados de limites territoriais feitos pelo Império Carolíngio, que
transferiram para Utrecht a função comercial e administrativa da cidade. 
Outros pesquisadores são unânimes em concluir
que, dentre as hipóteses para o declínio de Dorestad, foram decisivas uma série de
enchentes ocorridas entre 864 e 870 d.C, juntamente com inúmeras cheias
das marés que atingiram a cidade. Esses dois motivos fizeram com que a
região não se tornasse mais atraente, levando Dorestad ao colapso.

\SIG{Marlon Ângelo Maltauro}

Ver também Comércio; Era Viking; Viking.

\begin{itemize}
\item \versal{BRØNDSTED}, Johannes. \emph{Os Vikings}. São Paulo: Hermus, s.d, pp.
39-43.

\item \versal{BROWN}, Dale W. (ed.). \emph{Os Vikings: intrépidos navegantes do Norte}.
São Paulo: Abril/Time Life, 1999, pp. 64-65.

\item \versal{COOIJMANS}, Christian. The Controlled Decline of Viking-Ruled Dorestad.
\emph{Northern Studies,} vol. 47, 2014, pp. 32-42.

\item \versal{TUUK}, Luit van der. Denen in Dorestad. De Deense rol in de ondergang van
Dorestad. \emph{Jaarboek Oud-Utrecht}, 2005, pp. 05-40.
\end{itemize}

\section{\versal{DUBLIN}}

A cidade que hoje é capital e maior cidade da República da Irlanda pode
ser identificada como a mais significante cidade irlandesa do último
milênio. No entanto, suas origens se encontram em um passado remoto e
muitas vezes controverso.

De acordo com relatos de analistas medievais e da arqueologia, os
primórdios de Dublin são duas ``proto-cidades'' que
futuramente seriam chamadas de Dublin. Uma delas teria origem no
assentamento gaélico chamado Áth Cliath (``Vau dos obstáculos''),
localizado ao norte do rio Liffey; a outra, localizada ao sul, de base eclesiástica e
chamada Duiblinn (``Lagoa negra''), que seria a base do
assentamento escandinavo que tomaria forma futuramente. O nome em
gaélico, entretanto, seria hoje utilizado para designar a cidade moderna
de Dublin, que em seu registro oficial chama-se Baile Áth Cliath.

O assentamento viking tomou o eclesiástico na primeira
metade do século~\versal{IX}, mais especificamente em 841, conforme relatam os
\emph{Anais de Ulster} (\emph{Annála Uladh}): ``Havia um
acampamento naval em Linn Duachaill que saqueou os povos e igrejas da Tehba.
Havia um acampamento naval em Duiblinn que saqueou os
Laigin e os UíNéill, tanto as localidades quanto as
igrejas até Sliab Bladma''.

Com essa descrição podemos compreender que, em meio à primeira fase dos
ataques da Era Viking à Irlanda, com início a partir de 795 e vindos do
norte, as primeiras levas de escandinavos promovem uma série de invasões
sistêmicas ao território irlandês. Para além da invasão de igrejas da região de 
Tehba (região próxima a Longford e Westmeath), do saque de províncias, como
Laigin e UíNéill, e de igrejas até a região de Sliab
Bladma (montanhas entre os condados de Offaly e Laois), relatam-se a
formação de assentamentos escandinavos, como o porto de Linn
Duachaill, e o assentamento de Dubh Linn. O primeiro foi
abandonado ao longo dos anos, mas o segundo prosperou e tornou-se a
cidade de Dublin.

A partir de 841, então, é possível delimitar a fundação da cidade de
Dublin e sua ocupação pelos vikings. No entanto, isso não quer dizer que
a ocupação do assentamento foi inconteste. Afinal, no ano de 851 os
dinamarqueses, liderados por Ivar, o Desossado, tomam a região, 
que também terá a influência de outro líder, Olavo, o Branco. Os
dois aparentemente chegaram a um acordo entre si sobre o território.

Olavo desposou a filha de Áed Findliath, o grande rei da Irlanda, 
descendente da dinastia dos Uí Néill e homem de 
grande influência política na Irlanda da Era Viking. Esse fato demonstra
como assentamentos como Dublin servem de exemplo para compreensão de 
como os vikings tornaram-se gradativamente mais aceitos na sociedade
irlandesa, misturando-se e participando da vida e das disputas locais.

Nessa época, Dublin e York mantinham certo grau de interação e, com o
domínio de Olavo e posteriormente Ivar, Dublin
transforma-se em um grande centro de troca e negociação de escravos.
Esse relativo domínio da região por parte dos vikings rompe-se
apenas no início do século~\versal{X}, quando o poder dos vikings
enfraquece e, em decorrência disso, Dublin será saqueada e destruída no
ano de 902.

Os vikings retomam Dublin apenas em 917 e consolidam sua supremacia na
região, sobretudo em 919, ao derrotar em batalha o rei de Tara, Nial Glúndub. 
Nesse momento Dublin volta a ter reis nórdicos e retoma seu
desenvolvimento. Escavações arqueológicas nas docas de madeira
demonstram que, ao longo do século~\versal{X}, a cidade torna-se um grande centro
comercial, com ruas e casas de madeira abrigando oficinas, centros de
artesanato, manufatura e um amplo mercado.

Os reis de Dublin, então, ocupam lugar de destaque na vida política
irlandesa nesse período. Seus feitos se concentraram na administração da
cidade e também na manutenção estreitos laços com a região da Nortúmbria
e na imposição de sua autoridade sobre os demais centros e assentamentos
vikings na Irlanda.

Aos poucos, os nórdicos já integrados na sociedade irlandesa da época
tornam-se parte das disputas políticas da região. Seus líderes fariam 
alianças em meio à guerra das grandes dinastias, que tem início já no
final do século~\versal{X}. A cidade se expande, fortificações são reforçadas e
outras tantas criadas. Nesse momento, no início do século~\versal{XI}, a guerra
das grandes dinastias tem seu clímax na batalha de Clontarf, região que
atualmente faz parte do subúrbio dublinense, mas que na época não se
encontrava nos limites da cidade.

Na Batalha de Clontarf (1014), narrada em textos como o \emph{Cogadh
Gáedhel re Gallaibh} e a \emph{Brennu-Njáls saga}, o rei
Sitric de Dublin alinha-se contrariamente à pretensão do chefe
irlandês Brian Boru de dominar e tornar-se o grande rei da Irlanda.
Apesar disso, mesmo após o desfecho da batalha e das significativas
mudanças na estrutura política irlandesa da época, Dublin se mantém com
relativa autonomia.

No entanto, essa independência da qual Dublin gozará a partir do início
do século~\versal{XI}, duraria apenas até o ano 1052, quando o então rei de
Leinster, Diarmait mac Máel na mBó, forçou os dublinenses a
aceitarem seu filho, Murchad, como seu governante. Entretanto,
mesmo com o crescente revés em seu poder político, economicamente a
cidade prosperava e, de certa maneira, qualquer um que um dia viesse a
pleitear o direito de se tornar grande rei da Irlanda deveria controlar
Dublin para alcançar tal feito.

Em verdade, ao longo do século~\versal{XI} e sobretudo após a batalha de
Clontarf, o papel político dos vikings gradativamente se esvai até 
tornar-se apenas uma força comercial. Dublin segue o mesmo 
destino. Em 1171, a cidade foi capturada e totalmente dominada pelo rei de
Leinster, Dermot MacMurrogh, com a ajuda de mercenários
anglo-normandos. Esse fato decreta o fim da dominação viking em Dublin e
marca o início de outra era, não apenas para os rumos da cidade, mas
para a Irlanda como um todo.

\SIG{Erick Carvalho de Mello}

Ver também Brian Boru; Celtas e nórdicos; Irlanda da Era Viking.

\begin{itemize}
\item \versal{CONNOLLY}, Sean J. \emph{The Oxford companion to Irish History}. Oxford:
Oxford University Press, 1998.

\item \versal{DOWNHAM}, Clare. \emph{Viking Kings of Britain and Ireland}. Edinburgh:
Dunedin Academic Press, 2007.

\item \versal{DUFFY}, Seán. \emph{Brian Boru and the Battle of Clontarf}. Dublin: Gill
Books, 2014.

\item \versal{MARTIN}, Francis \versal{X}. \& \versal{MOODY}, Theodore \versal{W}. (orgs.). \emph{The Course of Irish History}. Cork: Mercier
Press, 2011, pp. 91-122.

\item \versal{RICHTER}, Michael. \emph{Medieval Ireland: The Enduring Tradition}.
Dublin: Gill and Macmillan, 1988
\end{itemize}

\section{\versal{DUELOS}}

Entre os escandinavos da Era Viking a noção de honra (familiar ou individual) era de extrema
importância familiar. Era uma instância
de equilíbrio que homem algum podia deixar que se abalasse e, portanto, ao
se deparar com o risco de vê-la manchada, a passividade não era uma opção
para lidar com o assunto. Muitas vezes tinha início um
longo e tortuoso processo de disputas, duelos e rixas, às vezes
individuais, às vezes entre famílias inteiras que, não raramente,
resultava em mortes.

Segundo Gunnvör Silfrahárr, era comum nas sociedades germânicas que esse
processo de retaliação e vingança tomasse proporções muito
maiores do que as do insulto primeiro e originário. Muitas vezes esse
padrão de comportamento vingativo tornava-se um sangrento ciclo entre
famílias: quando uma delas acreditava ter obtido sua vingança, a outra
sentia-se no direito de vingar-se, e assim continuamente. A tendência
era que esse processo se perpetuasse até que, com o passar de gerações,
ou a ofensa inicial fosse esquecida, ou então toda uma linhagem
terminasse morta. Esse tipo de disputa violenta foi glorificada em
várias das sagas islandesas e estava presente também no poema épico
anglo-saxão \emph{Beowulf}, imortalizando o tema por meio da literatura.

Mesmo um conhecimento superficial sobre as sagas islandesas suscita a
percepção de que elas tratam, centralmente, sobre as disputas, rixas e
duelos. William Miller afirma que esse tipo de disputa nos informa sobre
basicamente todos os aspectos legais e políticos da Islândia medieval.
Para o autor, os duelos surgem nas sagas como uma estrutura social que
permite a expressão de toda a conjuntura das atividades política, moral
e jurídica, atuando também como meios pelos quais essas mesmas
atividades são buscadas, como uma espécie de sanção extrema.

As disputas e rixas de sangue eram, portanto, de cunho moral em seu aspecto
vingativo, um meio de se punir violações de normas sociais,
como a ofensa da honra, o roubo e a morte. Eram de cunho jurídico quando, na
Islândia, ofereciam sanções que atuavam embasadas em acordos e
julgamentos legais. Nesse aspecto, as disputas ocupavam o lugar de uma
espécie de poder executivo frente a um sistema jurídico que não detinha
qualquer outro aparato de execução instituído formalmente pelo Estado.
Por último, elas consistiam também em atos políticos porque funcionavam
como uma das estruturas-chave dentro da qual a competição por poder e a
luta pela dominância ocorriam livremente.

Com o tempo, ao menos no contexto islandês, as disputas e rixas passaram
a ser mediadas e arbitradas por terceiros -- que seriam, em tese,
imparciais -- e aconteciam, quase que exclusivamente, por questões
relacionadas à vingança e manutenção da honra. A mediação e arbitragem
das rixas surgiram, segundo Gunnvör, como um mecanismo social de
limitação e contenção das disputas e matanças excessivas. Visto que elas
tendiam a aumentar e se proliferar de maneira contínua, passando a
envolver em seus laços de sangue e ódio cada vez mais e mais membros da
sociedade, era benéfico e prudente que essa sociedade desenvolvesse
métodos de proteger não somente os membros envolvidos diretamente e suas
famílias, mas a sociedade como um todo. Do contrário, seria perigoso que
se atingisse um número de mortes que colocasse em risco a estrutura
social, ocasionando diversas lacunas -- como a falta de diversificação
de papéis sociais e de execução de trabalhos, baixo índice de natalidade
e alto índice de mortalidade, dificuldade em proteger as terras contra
outros etc. 

Contudo, esses não foram os únicos motivos para que
as disputas e rixas começassem a ser reguladas. Conforme elucida Carol Clover, na Islândia
medieval, assim como em outros lugares onde ocorriam essas supracitadas
práticas, as disputas eram um modo de legislação e regulação baseadas e
centradas nos clãs. Era do interesse do Estado -- e posteriormente, da
Igreja -- reduzir o poder dos clãs, das suas práticas e
legislações. A isso soma-se, segundo Jesse Byock, uma das principais
preocupações da sociedade nórdica durante a Era Viking, muito refletida
nas sagas: canalizar a violência desmedida em padrões de disputa
aceitáveis, regulando conflitos. Como proposta a essa resolução surge a
prática dos duelos, que consiste basicamente no desenvolvimento de um
duelo ``de honra'' entre dois homens como um modo socialmente aceito de
se retificar uma ofensa, injúria, prejuízo ou dano. Esse primeiro passo
consistiu numa tentativa de estreitar o círculo daqueles diretamente
envolvidos em cada disputa, reduzindo o número a apenas um representante
de cada um dos lados.

A partir desse ponto, o duelo passou a ser uma forma legalizada de
disputa. Ele se embasava na crença que circulava por todo o norte da
Europa segundo a qual o guerreiro era glorificado e o covarde,
desprezado. Enfatiza Marlene Ciklamini que as sagas, por lidarem
predominantemente com as vidas de famílias notáveis e distintas, sugerem
que era principalmente a classe da aristocracia que se engajava em duelos,
por mais que, aparentemente, o homem comum, mesmo que pobre (desde que
não fosse escravo) também tivesse direito a duelar. Mas a prática, assim como as leis de maneira geral na Islândia medieval,
era feita para favorecer os poderosos e bem abastados.

Contudo, era possível que a lógica se invertesse. Na Noruega e Islândia,
duelar passou a ser um modo de ganhar a vida e conquistar bens. Homens
com poucos recursos podiam reivindicar uma propriedade e desafiar o dono
para um duelo. Caso vencesse ou o dono legítimo se recusasse lutar, a
propriedade estaria garantida ao desafiador. Os famosos \emph{berserkir}
e outros guerreiros com reputação igualmente sinistra por sua força e
intrepidez oprimiam frequentemente os ricos nesse sentido, desafiando-os
para duelos, sobre risco de tomarem sua propriedade ou bens.

A primeira forma de duelo legalizado na Escandinávia viking foi o
\emph{einvigi}, palavra que literalmente significa ``combate
individual''. É possível que o conceito envolvendo este formato de duelo
seja anterior à Era Viking, tendo se originado entre seus ancestrais
germânicos. Cognatos aparecem em sueco antigo (\emph{einvighe}), antigo
alto-alemão (\emph{einwic}) e inglês antigo (\emph{artwig}).
Basicamente, o \emph{einvigi} consistia numa forma de duelo sem regras e
restrições, lutado com armas à escolha dos participantes, em
qualquer localização e valendo-se de qualquer método de disputa. Segundo
Gwyn Jones, tratava-se, em suma, de uma briga em que se valia tudo para
fazer o adversário desistir.

Os combatentes envolvidos no \emph{einvigi} não possuíam qualquer jurado
ou árbitro que apontasse o resultado. Ao contrário dos duelos
anglo-saxões, não havia nenhuma espécie de invocação ou \emph{judicium
dei} antes da luta, ou seja, os participantes não enxergavam o resultado
como uma preferência de algum deus ou revelação da vontade divina. Dessa
forma, ao invés de invocar alguma divindade, os participantes confiavam
em suas habilidades, força e sorte. No \emph{Gylfaginning}, Snorri
Sturluson cita Ullr como o deus do \emph{einvigi}, o que poderia ser um
indício de que os participantes o invocassem pontualmente pedindo sua
ajuda para vencer o duelo, porém, não há qualquer indício dessa
interferência divina nas sagas ou qualquer outro material literário.
Essa escassez de referências reforça a ideia de que os participantes não
contavam com proteção ou assistência divina, mas com a própria
competência e agilidade. Por isso, Ciklamini afirma que o \emph{einvigi}
era uma forma de duelo secular, não sendo visto como um julgamento
divino, mas como um atalho direto para resolução formal de conflitos e
para a proteção dos indivíduos contra calúnias, quebra de juramentos e
ofensas à sua honra vindas de seus oponentes.

Apesar de ter atuado minimizando os efeitos das grandes disputas, ao
longo do tempo o \emph{einvigi} mostrou possuir defeitos. Se um dos
combatentes fosse assassinado durante o duelo, a seus familiares era
concedido o \emph{eptirmál}, direito legal de processar o responsável
pelo assassinato. As opções para os familiares eram, em seguida, receber
uma indenização pela morte do ente querido, ou requerer formalmente a
vingança. O padrão nesses casos, ressalta Gunnvör, era o caminho da
vingança, que gerava outro processo de vingança pela outra parte, e
assim sucessivamente. Tantas vinganças resultavam, muitas vezes, em
grandes rixas generalizadas entre duas famílias, desembocando no
processo sangrento que o duelo visava evitar a princípio.

Eis que, posteriormente, outro tipo de duelo surge na parte oeste da
Escandinávia. O \emph{hólmgang} (ou \emph{hólmganga}) era um compromisso
firmado que surgiu como fruto da união entre as forças antagônicas do
direito privado e do direito comum. Explica Gwyn Jones que, por um lado,
o \emph{hólmgang} era um duelo direto e simples, lutado até que um dos
lados obtivesse a vitória; por outro, tratava-se de um tipo de disputa
cujos passos eram todos regulamentados por medidas estipuladas em um
rígido código de etiqueta dos duelos, o \emph{hólmgõngulög}. O
\emph{hólmgang} era, portanto, um julgamento do valor pessoal de um
homem e o teste de suas aptidões físicas. Sua existência se justificava
graças à valorização dada aos atributos da força e coragem, algo
fortemente presente na sociedade islandesa, e também como um testemunho
do quanto as leis islandesas validavam e apreciavam essas
características ao resolver problemas práticos.

Diferentemente do \emph{einvigi}, os duelos no \emph{hólmgang} deveriam
acontecer em lugares específicos, demarcados e delimitados, os
\emph{hólmgangustadr}, que de maneira geral ficavam em áreas
geograficamente mais isoladas. Era comum que cada distrito tivesse seu
lugar específico onde os duelos eram tipicamente realizados. A área onde
a luta ocorria era delimitada com pedras e grandes estacas de madeira,
constituindo uma espécie de ringue, e os participantes não podiam evadir
esse espaço. A arma utilizada costumava ser a espada; cada participante
podia contar também com um escudeiro que substituiria seu escudo durante
a luta caso o mesmo quebrasse. Uma importante característica do
\emph{hólmgang} é o fato de que as feridas causadas pela disputa eram
raramente mortais, pois estipulava-se que a luta deveria ser encerrada
assim que a primeira gota de sangue fosse derramada. Explica Gwyn Jones
que, mesmo em caso de morte durante o duelo, os familiares do falecido
eram proibidos de buscar vingança ou retribuição.

O \emph{hólmgang} também era desprovido de qualquer caráter relacionado
a um juízo ou escolha divina de um lutador em detrimento do outro. Não
há nenhuma evidência nas sagas que aponte para esse tipo de percepção
acerca do duelo. Ao que tudo indica, os islandeses, com sua clareza de
visão e ceticismo, enxergavam a futilidade de tal pretensão. Para eles,
Thor, Odin, Njord e Freyr compartilhavam com os seres humanos muitas de
suas fraquezas e jamais poderiam ser vistos como árbitros imparciais
daquilo que é certo, errado ou justo. Por isso, o \emph{hólmgang} podia
até ser, por vezes, um teste moral, mas sempre resolvido com base
exclusivamente nas aptidões físicas dos competidores que, por sua vez,
confiavam em nada mais que sua própria força.

Apesar disso, o duelo podia ser acompanhado de alguns ritos, como nos
mostram, por exemplo, as sagas de Egil e de Kormáks. Ainda assim, eram
cerimônias religiosas pontuais que não ofereciam quaisquer indícios de
pretender invocar os deuses ou suplicar por sua interferência em prol do
competidor cuja reivindicação fosse a mais justa ou correta. Segundo
Jones, há indícios de casos em que um touro era sacrificado aos deuses.
Aparentemente, consta no \emph{hólmgõngulög} que este sacrifício,
chamado \emph{blótnaut}, deveria ser feito em todas as ocasiões em que
houvesse um \emph{hólmgang}, preferencialmente antes da realização do
duelo propriamente dito. Havia também a performance do \emph{tjösnublót}
que, ressalta Ciklamini, era um ritual mágico realizado com o intuito de
afastar possíveis influências malignas ou invocações feitas previamente
para trazer má sorte a um dos duelistas. Em suma, era um modo de tornar
o duelo mais justo e assegurar que seu resultado fosse mera consequência
da força e habilidade dos participantes.

\SIG{Victor Hugo Sampaio Alves}

Ver também Armamento; Guerra e técnicas de combate; Guerra e
simbolismos; Sociedade; Viking.

\begin{itemize}
\item \versal{BYOCK}, Jesse. \emph{Feud in the Icelandic Saga}. Berkeley: University of
California Press, 1993.

\item \versal{CIKLAMINI}, Marlene. The Old Icelandic Duel. \emph{Scandinavian Studies},
vol. 35, n. 3, 1963, pp. 175-194.

\item \versal{CLOVER}, Carol. Hildigunr's Lament. In: \versal{LINDOW}, John; \versal{LÖNNROTH}, Lars;
\versal{WEBER}, Gerd (eds.). \emph{Structure and meaning in Old Norse
Literature}. Odense: Odense University Press, 1986, pp. 141-183.

\item \versal{JONES}, Gwyn. Some Characteristics of the Icelandic `Hólmganga'.
\emph{The Journal of English and Germanic Philology}, vol. 32, n. 2,
1933, pp. 203-224.

\item \versal{JONES}, Gwyn. The Religious Elements of the Icelandic `Hólmganga'.
\emph{The Modern Language Review}, vol. 27, n. 3, 1932, pp. 307-313.

\item \versal{MILLER}, Willian Ian. \emph{Bloodtaking and peacemaking: feud, law and
society in Saga Iceland}. Chicago: University of Chicago Press, 1996.
\end{itemize}


\chapter{E \textarn{e} \textarc{e}}

\section{\versal{EGILS SAGA}}

A \emph{Egils saga Skalla-grímssonar} é o exemplo mais refinado de um
subgrupo das \emph{Íslendingasögur} conhecido como ``sagas de poetas''
ou ``sagas de escaldos''. É uma das sagas mais extensas que se conhecem
e narra a vida do guerreiro e poeta Egill Skallagrímsson, assim como
de seus ancestrais entre os anos de 850 a 1000. A saga foi escrita na
primeira metade do século~\versal{XIII} (\emph{circa} 1220) e se conserva em três
diferentes redações. A mais importante delas, habitualmente
denominada~\versal{M}, está contida no códice denominado \emph{Möðruvallabók} \versal{AM}
132 fol. de meados do século~\versal{XV}, que até o momento tem sido a base para
a maioria das edições do texto, apesar de não conter dois dos
grandes poemas da saga, \emph{Höfuðlausn} e \emph{Sonatorrek}, embora
seja a única que contém o terceiro poema, \emph{Arinbjarnarkviða}. A
segunda redação da saga, denominada~\versal{W}, está contida em um códice
islandês do século~\versal{XIV} que se conserva na biblioteca alemã de
Wolfenbüttel junto com outros fragmentos entre os que destacam o~\versal{AM} 463
4to e o~\versal{AM} 560 d 4to, que conservam algumas passagens não incluídas nos
manuscritos mais antigos. A terceira redação, chamada~\versal{K}, se conserva
em duas cópias em papel de meados do século~\versal{XVII} escritas por Ketill
Jörundsson, daí o nome do códice, \emph{Ketilsbók}, que contém a única
cópia conhecida do \emph{Sonatorrek}.

Embora a Egils saga seja em princípio uma obra anônima, não são
poucos os investigadores que, por sua profusão nos poemas escáldicos,
bem como por suas similitudes estilísticas e narratológicas com a
\emph{Heimskringla} de Snorri Sturluson, não tem duvidado em admitir
essa autoria ao político e literato islandês. Isso sem contar que
Egill e Snorri estiveram aparentados por linha materna e que o último
adquiriu fama e riqueza na zona em que haviam residido os descendentes de
Egill desde os tempos da colonização. Do mesmo modo que Snorri, Egill
era um poeta notável e muito valorizado pela sociedade
islandesa, assim como pelos monarcas estrangeiros, por sua habilidade
em incrementar a honra das personagens a que dirigia seus esforços
poéticos, mas também para destruí-las com insultos e difamações. A
relação de Egill com a nobreza norueguesa é ambígua, já que era na
Escandinávia onde os islandeses podiam fazer fortuna e melhorar seu
\emph{status}, porém também onde eram suscetíveis de incorrer na ira
real.

Na \emph{Egils saga} estão incluídos um total de 56 poemas
escáldicos independentes (\emph{lausavísur}), 48 dos quais
são atribuídos ao próprio Egill, além dos três poemas de maior
envergadura acima mencionados e outros três denominados \emph{drápur,}
dos que somente se mencionam os versos introdutórios. As circunstâncias
da transmissão da poesia e sua inclusão na prosa narrativa não estão
isentas de polêmica, embora seja possível que boa parte das estrofes
foram originadas como respostas espontâneas do poeta a determinadas
situações. Os ditos poemas haviam sido memorizados e transmitidos em
forma oral até que se transformaram em poemas de maior tamanho ou
estrofes soltas que serviram como acompanhamento a um texto em prosa.

A poesia da \emph{Egils saga} não somente responde à função laudatória
própria do ofício dos escaldos dos reis, tal como o caso de
\emph{Höfuðlausn}, senão que também servia ao objeto de registrar de uma
forma significativa e verdadeira, situações e personagens do passado,
assim como veículo de sentimentos que raramente aparecem recolhidos na
prosa, como as estrofes em que Egill expressa seu amor pela viúva de seu
irmão, e claro, além disso, do \emph{Sonatorrek}.

A \emph{Egils saga}, pela sua incomum quantidade de estrofes, parece dirigida a
uma audiência interessada em poesia e contém também um bom número de
referências à arte de compor poesia e a famosos poetas islandeses da
época. De um ponto de vista meramente estrutural, a \emph{Egils saga} parece
girar em torno de dois grandes temas: a descendência de
Kveld-Úlfr e a relação entre os protagonistas e os monarcas
europeus, em cujas cortes recorrem em busca de honra, riquezas e estima
social. A saga divide-se em duas grandes unidades, que às vezes
têm recebido o nome de seus protagonistas principais, a \emph{Þórólfs
saga} e a \emph{Egils saga}. A primeira é uma parte introdutória que tem
lugar na Noruega e que começa com a tragédia do desterro forçado da
família pela censura que o rei toma por Þórólfr Kveldúlfsson e que o
obriga a fugir da Islândia junto com sua família. A segunda é muito
maior e heterogênea e, a concordar com muitos investigadores, carente de
um foco claro, o que possivelmente contribuiu para a dispersão
geográfica e temporal das aventuras que narra. O começo da segunda parte
contém o primeiro desacordo entre Egill, então um menino de
três anos, e seu pai, seguido de sua triunfal entrada na festa de seu avô materno que o
pai havia proibido de assistir. Essa primeira declaração de interesses
foi o começo de uma viagem do personagem já
crescido em suas aventuras pela Noruega, Suécia, Inglaterra, Frísia,
Dinamarca, incluindo Kúrland no Báltico, até seu regresso à Islândia,
resignado e mal-humorado até a conformação com a morte em seu leito.

O princípio do ocaso de Egill, introduzido pela elegia da morte dos seus
filhos, \emph{Sonatorrek}, constitui um dos pontos altos da poesia
escáldica. Pode ser interpretado por meio do referencial mitológico, associando-o ao
deus Odin e à perda de Baldr, ou por meio das tradições europeias protocristãs
em torno da função ritual da dor pela perda de um ser querido. Além da
abundância de frases do narrador para chamar a atenção da audiência
sobre um feito ou sobre a relação com algo já dito, a estrutura geral de
sua obra está caracterizada por um uso profuso de repetições, de
contraposições entre personagens com elementos positivos -- loiros,
corajosos e bondosos --, como os de Þórólfr, contrapostos a outros com
elementos negativos -- morenos, feios e malvados --, como o próprio Egill
e seu pai Skallagrim. O texto também está cheio de paralelismos entre as
duas partes da saga e entre episódios de cada uma das partes que servem
de objeto para apresentar os conflitos em que estão imersas as diferentes
gerações de famílias e a maneira que cada um tem de restaurar o
equilíbrio em sua volta. Dos temas principais destacamos a amizade de
Arinbjörn e Egill, o enfrentamento com um anfitrião pouco dado a tratar
bem seus convidados, o conflito entre irmãos, não somente pelas
desavenças existentes entre Egill e seu irmão Þórólfr, senão também, por
exemplo, entre os filhos do rei Haroldo Cabelos Belos. Junto a uma clara
coerência interna entre as partes da obra, não são poucas as ocorrências
de várias tramas ou linhas de ação paralelas, nas quais a ação
discorre lentamente, interrompida por contínuas digressões do
autor sobre o tempo meteorológico, as armas, ou a
indumentária de um personagem. Esses recursos servem para elevar o
valor estético da paisagem, mas também têm o objetivo de criar
suspense e elevar o tom até o clímax, habitualmente uma ação de caráter
violento.

Diferente de outras sagas de escaldos, como a \emph{Kormáks saga}, a
\emph{Gunnlaugs saga ormstungu,} ou a \emph{Hallfreðar saga
vandræðaskálds}, a \emph{Egils saga} mostra uma relação ambígua, ou
diretamente negativa, com os reis cujas cortes Egill visitou
assiduamente, como é o caso de Érico Machado Sangrento (Haraldsson). Isto
não quer dizer, entretanto, que na \emph{Egils saga} não apareçam ideias
refletidas que tanto sancionam como discutem a ideologia por
trás da autoridade real. Um dos temas principais que a saga aborda é o
da luta pela independência dos camponeses islandeses contra os monarcas
injustos e tirânicos. O fato de a saga ter sido escrita no século~\versal{XIII},
momento de agitação sociopolítica, faz com que, como em todos os
momentos de crise, qualquer tempo passado comece a ser considerado sob
uma perspectiva diferente, que conduz a sua idealização, o que sem
dúvida contribui para a apresentação de Egill como um poeta genial e com
força extraordinária. Mas Egill era mais do que isso: era um homem preocupado
com a política de seu tempo, versado na magia das runas e um
reconhecido adorador de Odin, embora também tenha recebido a
\emph{primasignatio} em uma de suas viagens.

Por tudo isso, Egill é considerado uma figura de transição entre dois
mundos, entre a Noruega e a Islândia, entre reis e camponeses, entre
poetas e pessoas comuns. Seu lado obscuro, que o acompanha em suas
viagens e de onde se mostra com uma feroz oponente, veio por herança
genética recebida de seus antepassados, dentre os quais se mencionam
os homens-lobo (\emph{Kveld-úlfr}), as mulheres que são meio \emph{troll}
(\emph{Hallbera Hálftroll}) e sua avó materna, que afirmava ser
filha de um \emph{berserkr}. Uma vez de volta à Islândia, sem dúvida,
Egill viveu uma vida tranquila longe de problemas, já que a sociedade
islandesa, povoada por homens livres, era uma sociedade
imersa em um processo de mudança, não somente política e social,
mas também religiosa. Quando, ao final de sua vida, somos testemunhas de
sua entrada triunfal na Assembleia, bem acompanhado de oitenta homens
armados para defender o direito de seu filho em uma ação judicial, e
escutamos seu discurso em que recorda as origens de sua família e sua
chegada à Islândia, assistimos ao fim de suas aventuras e ao
começo de sua lenda.

\SIG{Teodoro Manrique Antón}

Ver também Egil Sakallagrimsson; Islândia da Era Viking; Literatura;
Poesia escáldica; Sagas islandesas; Sonatorrek.

\begin{itemize}
\item \versal{FINLAY}, Alison. \emph{Egils saga} and Other Poets sagas. In: \versal{HINES}, John; \versal{SLAY}, Desmond (eds.). \emph{Introductory Essays on Egils saga and
Njáls saga}. London: Viking Society for Northern Research, 1992, pp.
33-48.

\item \versal{HASTAÐ}, Baldur. \emph{Die Egils saga und ihr Verhältnis zu anderen
Werken des nordischen Mittelalters.} Reykjavík: Rannsóknarstofnun
Kennaraháskóla Íslands, 1995.

\item \versal{HARRIS}, Joseph. Sacrifice and Guilt in \emph{Sonatorrek.} In: \versal{UECKER}, Heiko (ed.). \emph{Studien zum Altgermanischen. Festschrift für Heinrich
Beck}. (Ergänzungsbände zum Reallexikon der Germanischen Altertumskunde,
11), Berlim: De Gruyter, 1994, pp. 173-196.

\item \versal{LOOZE}, Laurence de; \versal{HELGASON}, Jon Karl; \versal{POOLE}, Russell, \versal{TULINIUS}, Torfi
H. (eds.). \emph{Egil, the Viking Poet: New Approaches to 'Egil's Saga}.
(Toronto Old Norse-Icelandic Series). Toronto: Toronto University Press,
2015.

\item \versal{NORTH}, Richard. The Pagan Inheritance of Egill´s \emph{Sonatorrek}. In:
\emph{Atti del 12 Congresso Internazionale di Studi sull´Alto Medioevo,
Spoleto 4-10 settembre 1988}. Spoleto: Presso la Sede del Centro Studi,
1990, pp. 147-167.

\versal{O'DONOGHUE}, Heather. \emph{Skaldic Verse and the Poetics of Saga
Narrative}. Oxford/New York: Oxford University Press, 2005.

\item \versal{ROSS}, Margaret Clunies. The Art of Poetry and the Figure of the Poet in
\emph{Egils saga}. In: \versal{TUCKER}, John (ed.). \emph{Sagas of the
Icelanders: A Book of Essays}. (Garland Reference Library of the
Humanities, 758). New York: Garland Publishing, 1989, pp. 126-145.

\item \versal{TULINIUS}, Torfi H. \emph{The Enigma of Egill: The Saga, the Viking Poet,
and Snorri Sturluson}. Islandica 57. Ithaca, New York: Cornell
University Library, 2014.
\end{itemize}

\section{\versal{EGIL SKALLAGRÍMSSON}}

Poeta, espadachim, orador, mercador, fazendeiro e ferreiro, Egil é o
protagonista de uma das mais famosas sagas islandesas, a \emph{Saga de Egil
Skallagrímsson}, ou \emph{Egils saga Skallagrímssonar}, escrita na
Islândia provavelmente no início do século~\versal{XIII}. Descrito como grande,
deformado e feio como um \emph{troll}, Egil é sempre representado em um
espectro emocional controverso -- enérgico na brutalidade com a qual
aplica a violência e melancólico na sensibilidade com que entoa seus
versos poéticos. Sua saga representada-o entre aventuras pela
Europa ocidental nos séculos~{\versal{IX} e \versal{X}} e também pontua as
políticas de convivência diplomática entre a Islândia e a Noruega.

Uma das versões mais antigas de sua saga é encontrada no
\emph{Mǫðruvallabók}, manuscrito compilado entre 1320 e 1350. A versão de
conteúdo mais extenso encontra-se no manuscrito \versal{AM}453 4to, mas a saga, tal como
temos acesso hoje, em edições modernas, recebe adições de manuscritos
muito posteriores. De certa maneira, a narrativa da sua saga é também um
desenvolvimento sobre as relações entre os islandeses e os governantes
noruegueses. Os escritores dessa saga, entre os quais
Snorri Sturluson, descreveram nas
entrelinhas os anseios políticos de tais relações nos séculos seguintes,
quando a Islândia já havia se tornado súdita da coroa norueguesa e as
famílias poderosas que comandavam a ilha lutavam entre si pela memória
dos homens notáveis. Egil Skallagrímsson foi apontado como o ancestral
de uma dessas parentelas, os \emph{Mýramenn}.

O passado da família de Egil está ligada ao processo de formação e
unificação do reino norueguês por Haroldo Cabelos Belos (Haraldr
Hárfagri), cujo o percurso teria sido ditado pela apreensão dos direitos
às terras pelos homens livres e consequente fuga desses indivíduos e
suas famílias para a Islândia, tema muito comum nas \emph{Sagas dos
Islandeses}, \emph{Íslendingasǫgur}. De seu avô, Egil herdou o humor
taciturno, muitas vezes melancólico, e o temperamento colérico,
características que também foram atribuídas a seu pai. Tanto o seu tio
quanto o seu irmão, ambos de nome Thorolf (Þorolfr), foram agentes reais:
o primeiro foi morto pelas intrigas originadas das disputas de poderes e
das terras que a família havia abandonado na Noruega, e o segundo morreu
em combate na batalha de Brunanburh.

A saga opõe sempre os homens da família de Egil Skallagrímsson: ele e 
seu pai herdaram as características do avô -- eram homens rudes, feios,
simples, habilidosos com a poesia e guerreiros tempestuosos; seu tio e
irmão herdaram as características da avó -- eram homens belíssimos,
habilidosos em tudo, bem-apessoados e que trabalharam em prol do rei
Haroldo Cabelos Belos (Haraldr Hárfagri), ou do rei Ӕthelstan da
Inglaterra. Um exemplo do comportamento violento de Egil está na
passagem sobre um jogo de bola que ele e outras crianças brincavam,
até que ele se sente injustiçado e ataca um outro menino com um
machado, matando-o no local, levando seu pai a precisar matar toda a
família da criança assassinada, a fim de que eventuais vinganças
familiares não surgissem.

Um grande amigo e companheiro de aventuras de Egil, Arinbjǫrn, encontrou
uma posição de agente, \emph{hersir}, ao lado de Érico Machado Sangrento
(Eiríkr Blóðøx) e, mais tarde, de seus filhos. Érico e a sua esposa,
Gunnhildr, são os grandes antagonistas da saga, em diversos momentos
representando um perigo físico e mágico, ao mesmo tempo em que saem da
Noruega e se instalam em York. A origem desse longo conflito está na
morte de Bard, um homem leal ao casal, que durante uma festa em honra às
\emph{dísir}, serviu aos convidados iogurte ao invés de cerveja, sendo
ridicularizados pelos versos de escárnio de nosso herói. Ao descobrir,
por meio de magia rúnica, que seria envenenado pelo chiste, Egil não
pensou duas vezes antes de matar o anfitrião.

Apesar de ser descrito na saga como exímio guerreiro e de grandes feitos
marciais, incluindo um duelo contra um certo Átila (Atli), cuja
proteção por feitiços o fez resistir aos golpes de espada de Egil, sendo
vencido apenas quando o protagonista abandonou a espada e atacou o
oponente na garganta com os próprios dentes, seus êxitos mais notáveis
podem ser atestados com alguma segurança através de sua produção
poética. Entre os seus poemas mais famosos podemos listar: Resgate Pela
Cabeça, \emph{Hǫfuðlausn}, Lamento Pelos Meus Filhos, \emph{Sonatorrek},
e Elogio a Arinbjǫrn, \emph{Arinbjarnarkviða}.

Na narrativa da saga, o poema Resgate Pela Cabeça é apresentado por Egil
ao rei Érico em troca de sua vida, quando, após um longo histórico de
inimizade com o rei Érico Machado Sangrento, Egil naufraga na Inglaterra
e é levado para a presença do rei, passando a noite preso. O personagem
enfrenta o dilema de se acovardar e fugir ou de encarar Érico, mas ser
morto, já que o rei havia pronunciado sua pena de antemão. Resta a
Arinbjǫrn, seu amigo de longa data, mediar o encontro entre os dois e
oferecer uma solução engenhosa, sugerindo a Egil que componha algo em honra ao
rei. O escaldo compôs da noite para o dia e recitou os versos de memória
um poema elegíaco que foi aprovado por Érico e sua corte, garantindo a
liberdade, mas não a amizade entre as partes. Esse poema é encontrado em
manuscritos tardios, levando alguns pesquisadores a acreditar que os
versos foram adições posteriores.

O Lamento Pelos Meus Filhos é um poema sobre a dor da perda de seus dois
filhos, Gunnar e Bǫðvar. Deprimido, Egil prometeu a sua filha Thorgerd
(Þorgerð Egilsdóttir) que iria morrer de fome. Ela, por sua vez,
deu-lhe duas saídas para o problema: compor os versos que seriam conhecidos
como \emph{Sonatorrek}, ou ser responsável pelo fim da filha, pois
Thorgerd prometeu que, sendo esse o destino do pai, lhe acompanharia na
morte. No Lamento Pelos Meus Filhos, o escaldo contrasta bastante com o
personagem da narrativa, talvez pelo tom melancólico de seu conteúdo,
em que a agressividade de seus atos dá lugar à perda dos parentes, amigos,
irmão e filhos, retirados pelo deus Odin, que o compensa com o dom da
poesia.

O Elogio a Arinbjǫrn é um poema feito em homenagem ao amigo de longa
data na ocasião em que o cenário político norueguês estava em crise
pelas disputas de poderes promovidas pelos filhos da rainha Gunnhild
após a morte do rei Érico. Arinbjǫrn esteve em batalha ao lado de Haroldo
Capa Cinzenta (Haraldr Gráfeld) e passou a viver dos ganhos
administrativos recolhidos na região de Fjordane. Recuperando da
melancolia que o abatera com a morte de seus filhos, Egil compôs um
poema ao qual não temos acesso integral, já que algumas linhas foram
perdidas nos manuscritos, mas que celebra o valor da amizade, a coragem
do amigo e a presteza de Arinbjǫrn ao servir os seus senhores.

Todos esses elementos da narrativa sobre a \emph{Saga de Egil Skallagrímsson},
nos oferecem uma visão de como a escrita dessa fonte acaba reorganizando
o jogo político no Atlântico Norte, trocando o foco da narrativa da
Noruega para a Islândia, e se concentrando enfaticamente nas ações de um
islandês. Na saga, os islandeses não antagonizam os reis noruegueses, mas
os reis noruegueses antagonizam um islandês em especial, anti-heroico e
em nenhum momento um modelo ideal, mas ainda assim conhecido e celebrado
pela audiência islandesa. Essa troca, muito sutil, requer uma política
diplomática que ironiza a ambiguidade do caráter de Egil.

A velhice não tornou Egil um homem sossegado: seus
últimos anos, que atravessaram episódios ao mesmo tempo cômicos e
trágicos, demonstram uma pessoa ativa em sua comunidade. Mesmo quando
perdeu os seus filhos e se recolheu, deprimido, o plano de sua filha para
salvá-lo contou, em parte, com uma possível cumplicidade do velho
aventureiro.

A sua capacidade para pilhérias não diminuiu mesmo quando ele ficou cego
ao fim da vida: em um determinado episódio, propôs jogar dois cofres de
prata, que outrora havia recebido do rei Athelstan na Inglaterra, no
meio da Assembleia para provocar confusão e brigas entre os indivíduos.
Contrariado, ele faz com que dois escravos joguem o tesouro em uma
cachoeira e Egil carregue o segredo de sua
localidade para o túmulo. Esse episódio final da sua vida não pode
deixar de ser comparado com o tesouro heroico dos \emph{Niflungar},
obtido originalmente por Sigurð, o Matador do Dragão.

Os islandeses lidaram com um tema que já devia ser previamente de seu
conhecimento, Egil pode ter representado um grande poeta, exaltando sua
figura heroica contra a coroa norueguesa. Já os noruegueses poderiam
perceber a produção dentro dos termos da personalidade do personagem, ao
invés de termos historiográficos. Os islandeses observariam a
resistência heroica e o orgulho ancestral, enquanto os noruegueses
ridicularizariam seu ufanismo e as suas reivindicações vazias. O Egil
das sagas deve ter agradado ambas as audiências.

\SIG{Pablo Gomes de Miranda}

Ver também Egils saga; Sagas islandesas; Islândia da Era Viking; Viking.

\begin{itemize}
\item \versal{ANDERSSON}, Theodore M. \emph{The Growth of the Medieval Icelandic Sagas
(1180-1280)}. Ithaca: Cornell University Press, 2006.

\item \versal{ÓSKARSDÓTTIR}, Svanhildur. Introduction. In: \emph{Egil's Saga}. London:
Penguin Books, 2004, pp. vii--xxix.

\item \versal{ROSS}, Margaret Clunies. A Tale of Two Poets: Egill Skallagrimsson and
Einarr sklaglamm. \emph{Arkiv for Nordisk Filologi}, n. 120, vol. 1,
2005, pp. 69-82.

\item \versal{ROSS}, Margaret Clunies. Conjectural Emendation in Skaldic Editing
Practice, with Reference to Egils saga. \emph{Journal of English and
Germanic Philology}, n. 104, vol. 1, 2005, pp. 12-30.

\item \versal{TULINIUS}, Torfi \versal{H}. The Prosimetrum Form 2: verses as an influence in
saga composition and interpretation. In: \versal{POOLE}, Russell (org.).
\emph{Skaldsagas: text, vocation, and desire in the Icelandic sagas of
poets}. New York: Walter de Gruyter, 2001, pp. 191-217.
\end{itemize}

\section{\versal{EKETORP}}

Eketorp Borg é um forte circular presente em Öland, uma longa e estreita
ilha do Báltico, na costa sudeste da Suécia, único forte completamente
escavado na região. A ilha ainda conta com outras 19 fortificações que
foram apenas parcialmente estudadas, todas construídas de pedra
calcária, que têm seus formatos ovais ou circulares. Dentre as
fortificações de Öland a que se localiza na região mais ao sul da ilha é
conhecida como Eketorp, estando próxima da vila homônima, na paróquia de
Gräsgard. As escavações da fortificação geraram um total de 24 mil
artefatos e pela sua significação, em quantidade de artefatos e em
caráter arquitetônico, foi declarado patrimônio mundial pela \versal{UNESCO}. O
forte de Ektorp possuía paredes com 5~m de altura e 6~m de largura. Sua
construção estava parcialmente desabada e foi reconstruída no século~\versal{XX} para abrigar um centro turístico com museus e atividades de \emph{living
history}.

As escavações que ocorreram na região entre os anos de 1964 e 1973
revelaram três momentos diferentes de utilização dessa fortificação: o
primeiro momento ocorreu na Idade do Ferro Romana, no século~\versal{IV}; o
segundo momento ocorreu na Idade do Ferro Germânica, entre os séculos~{\versal{V} e
\versal{VII}} e por fim o terceiro momento, que ocorreu já na Idade Média, entre os
séculos~{\versal{XII} e \versal{XIII}}. As paredes do forte inicialmente cercavam uma área de 57~m, mas durante a Idade do Ferro Germânica a mesma foi ampliada
para uma área de 80 m e durante a Idade Média o forte ainda receberia
uma segunda muralha externa que ampliaria a capacidade de defesa da
região. O modelo circular do forte foi escolhido, de acordo com os arqueólogos,
devido às características geográficas da região: um
terreno extremamente plano que facilitaria um ataque de qualquer uma das
direções.

Dentro do forte foi encontrado, em 1741, um poço de água que ainda
funciona atualmente, datado pelos arqueólogos como tendo sido criado na
Idade do Ferro Romana. Nesse poço achou-se, durante as escavações,
grande número de ossos de animais,
como cavalos, ovelhas, cabras, porcos e bois.
Contudo, o estudo da região apontou para uma presença majoritária de
ossos de cavalos, mais de 50\%, o que contrastou com os ossos apontados
como descarte de alimentação, que eram sua maioria de gado. Os ossos
encontrados no poço passavam a apontar, dessa forma, uma clara diferença
entre os animais depositados e os animais associados com atividades do
cotidiano, fato que evidencia uma prática ritualística.

O estudo dos ossos de cavalo definiu a presença de crânios, falanges e
vértebras caudais intactos, fato que levou os arqueólogos a
apontarem para uma prática realizada com os animais ainda com suas
peles. Os estudos sugeriram, dessa forma, uma atividade realizada com as
partes animais colocadas em postes ao redor do poço de água, membros que
sofreriam um processo de decomposição, que terminaria com os ossos
caindo dentro do poço. A prática ritual de Eketorp foi assim
comparada com o antigo rito escandinavo denominado \emph{nidstang}, rito
no qual se empalava uma cabeça de cavalo e deixava a mesma exposta para
o processo de decomposição.

Em Eketorp encontra-se ainda a fundação de 53 edificações
apontadas como residências, das quais 3 continham
locais de abrigo para animais durante o inverno, além da delimitação
das fundações de outros 13 locais de abrigos de animais
não conectados com nenhuma residência, todas edificações
datadas da Idade do Ferro Germânica. O recurso arqueológico de animais
encontrado em Eketorp é um dos maiores da Escandinávia e conta com 0.5
toneladas de ossos para a Idade do Ferro Germânica e 1.3 toneladas de
ossos datados para a Idade Média. Aproximadamente 75\% do gado morto em
idade adulta eram do sexo feminino, fato que aponta para a utilização da
produção de leite em Eketorp.

Eketorp, por fim, pode ser definido como um forte que concentrava
criação de animais, residências e ritos, não garantindo apenas uma
proteção para a região, mas sendo um local de múltiplas funções 
(do cotidiano, da religião e da guerra), o que
evidencia o controle da localidade por um determinado grupo.

\SIG{Munir Lutfe Ayoub}

Ver também Arqueologia da Era Viking; Dinamarca da Era Viking;
Fortificações.

\begin{itemize}
\item \versal{BACKE}, Margareta; \versal{EDGREN}, Bengt; \versal{HERSCHEND}, Frands. Bones thrown into a
water hole. In: \versal{ARWIDSSON}, Greta (ed.). \emph{Sources and Resources.
Studies in honour of Birgit Arrhenius}. Stockholm: Pact, 1993, pp.
327-342.

\item \versal{HELBAEK}, Hans. Vendeltime farming products at Eketorp on Oland,
Sweden.~\emph{Acta archaeologica}, vol. 37, 1966, pp. 216-221.

\item \versal{STENBERGER}, Mårten. Eketorps borg, a fortified village on Öland, Sweden.
Some results from the present investigations.~\emph{Acta Archaeologica},
vol. 37, 1966, pp. 203-214.

\item \versal{TELLDAHL}, Ylva; \versal{SVENSSON}, Emma; \versal{GOTHERSTROM}, Anders; \versal{STORA}, Jan. Typing
late prehistoric cows and bulls---osteology and genetics of cattle at
the Eketorp ringfort on the Öland island in Sweden.~\emph{PloS one},
vol. 6, n. 6, 2011, p. e20748.
\end{itemize}

\section{\versal{EMBARCAÇÕES}}

A presença de embarcações entre os escandinavos pode ser constatada
desde os tempos mais remotos: há representações em petróglifos que atestam
sua existência referentes à Idade do Bronze (período que dura na
Escandinávia entre 1800 a 800 a.C.). Adicionalmente, podemos ter alguma
ideia das primeiras embarcações escandinavas muito anteriores a Era
Viking através dos vestígios das embarcações mais rudimentares, a
exemplo da canoa de Hjortspring, escavada no sudoeste dinamarquês,
datada aproximadamente de 350 d.C., no ato de seu afundamento como parte de um
sacrifício humano. Em termos de material construtivo, dimensões, e
capacidade de carga, a canoa possui 19~m de comprimento por 2~m 
de largura, e é feita em visgo. O seu esqueleto se apresenta
amarrado a grampos, possui bancadas que estão dispostas entre os vãos do
esqueleto da embarcação (que o mantem estável na finalização) com espaço
para dois homens, reservando-se o espaço de 1~m
entre cada banco. Tal canoa era impulsionada por 24 pás,
havendo também espaço para mais quatro homens que a manobravam
utilizando as pás como leme, tanto na proa quanto na popa. É importante
salientar que temos aqui a primeira amostra de uma construção em casco
trincado, muito popular no norte da Europa e característico das
embarcações escandinavas.

Separados por quase 600 anos estão os três barcos achados em Nydam,
também no sudoeste da Dinamarca e encontrados em um contexto sacrificial
similar. Do conjunto todo, o barco maior possui 23,5~m de
comprimento por 3,5~m de largura e 1,2~m de profundidade,
sendo dois deles feitos em carvalho e um em pinheiro.~Em sua fabricação,
o esqueleto está unido de maneira similar ao do modelo de Hjortspring,
porém o casco foi armado com cravos de ferro em vez de fibras, como no
exemplo anterior. A maior mudança está no método de propulsão dessas
embarcações, que usavam remos ao invés de pás, junto a um leme fixo para
realizar suas manobras.

As velas foram introduzidas nos séculos imediatamente anteriores à Era
Viking. Os navios à vela já navegavam na Europa há centenas de anos e
possibilitaram lançar as embarcações em jornadas aos recantos mais
longínquos, algo outrora~impossível para as embarcações anteriores.
Imagens em Runestones de Gotlândia e bracteatas (pequenas medalhas) de
Hedeby dos séculos~{\versal{VII} e \versal{VIII}}, mostram marinheiros da região do Báltico utilizando velas em suas embarcações e não temos razões para achar que
elas fossem diferentes daquelas usadas em ações vikings.

Os restos do navio de Oseberg é o achado direto mais antigo que atesta o
uso de velas entre as embarcações da Escandinávia. Datado de 820, o
navio de Oseberg foi reutilizado, depois de várias empreitadas, como
nave funerária e provavelmente representa o que havia de mais refinado
em tecnologia náutica na época. Medindo 21,5~m de comprimento,
5,1~m de largura e apenas 1,6~m da quilha, tal embarcação era
impulsionada por 15 pares de remos e uma vela quadrada;
infelizmente, nenhum vestígio da vela sobreviveu. O mastro foi montado na
sobrequilha, no meio do navio, e mantido ali por uma peça chamada
forquilha (\emph{kløften}), o que nesse navio parece ter
dimensões incompatíveis, já que ela teve de ser reparada com
um reforço metálico. A forquilha tem a função de guiar o mastro (quando
erguido ou removido) e de lhe conferir suporte quando a vela estiver
montada, funcionando em conjunto com o cordame.

Um defeito no navio pode mostrar a continuidade desse barco com os
anteriores: a largura exagerada de sua sobrequilha. Nos navios
escandinavos, a distância das peças do esqueleto do navio se manteve em
mais ou menos um metro e as bancadas dos remadores foram suprimidas,
de maneira que os remadores deveriam se sentar em bancos ou baús. A
sobrequilha ajudava a espalhar a tensão do mastro sobre uma parte
significativa do casco, o que não aconteceu nesse caso por causa da
cobertura sobre o esqueleto do navio, cobrindo duas cavernas quando
deveria cobrir quatro, sobrecarregando o cordame, o mastro e,
consequentemente, a forquilha. Provavelmente tal erro foi cometido
porque funcionaria sem problemas em uma embarcação menor, como se fazia
até então. Em compensação, a estabilidade do navio foi melhorada,
mudando o seu esqueleto que, ao invés de ser feito em peça única, passou
a ser composto por armações de diversas peças, além de oferecer uma continuidade
do fundo do navio para suas laterais, mantendo o formato de casco
trincado. Os remos passavam por buracos nas laterais que poderiam ser
fechados em navegação com vela, mudança que conferiu aos remos o apoio e
o ângulo necessários para o impulso de maior força ao navio.

A continuação das novas tecnologias náuticas apresentadas na embarcação
funerária de Oseberg pode ser encontrada nas embarcações achadas em
túmulos funerários em Gokstad, na Noruega, escavados em 1880 e com uma
datação que varia entre 900 a 905. São embarcações bem mais
robustas: 23,2~m de comprimento e 5,2~m de largura de boca,
mantendo distância de 2~m entre a quilha e a borda enquanto na
água, e equipados com 32 remos. Podemos dizer que essa embarcação
não só é 8\% mais comprida que a de Oseberg, mas também 25\% mais alta.
Sua quilha mais resistente e o casco com maior envergadura nos sinaliza
uma melhoria de navegação.

O navio de Tune, também da Noruega, foi construído na mesma época e é
ligeiramente menor que os de Oseberg e Gokstad (19,2~m de
comprimento e 4,2~m de largura). Uma análise recente dos seus
vestígios demonstrou uma construção parecida, nos possibilitando pensar
em um quadro homogêneo das características dos navios escandinavos
anteriores ao século~\versal{X}. Em contrapartida, o navio escavado em Ladby, na
Dinamarca, nos mostra uma construção distinta. Assim, dos restos
analisados e reconstruídos (originalmente sobraram apenas as marcas das
peças de ferro no solo) observamos sua caracterização compacta, pesada e
com diferenças significativas no casco: acreditamos que a construção desse
navio foi planejada para uma navegação no mar Báltico e em Kattegat, não
o mar do Norte como é o caso dos navios noruegueses.

Os navios citados até aqui são exemplos de um momento tecnológico e
cultural que sinalizaram o início das especializações navais próprias da
Era Viking. Sabemos que apesar dos restos dessas embarcações terem sido
encontradas em péssimo estado, os vestígios materiais do fim do século~\versal{IX} já nos apresentam navios delgados, rápidos (exemplos noruegueses) e
embarcações pesadas (exemplo dinamarquês). É comum encontrarmos estudos
que estabelecem uma diferenciação das embarcações desse momento em duas
categorias principais: 1) os navios de guerra ou voltados para as
comitivas reais, geralmente longos e leves, com pouca capacidade de
carga e desenvolvidos para a navegação de cabotagem; 2) as embarcações de
carga, bojudas e pesadas, utilizadas para o transporte de produtos e
mercadorias em geral, sendo empregadas nas viagens ao Atlântico Norte
pela sua capacidade de viajar em mar aberto.

Façamos uma breve análise dessa primeira categoria: alguns navios que
foram afundados deliberadamente em Skuldelev, na Dinamarca, para servirem
como barreira protetora ao fiorde de Roskilde, são os modelos desses
navios guerreiros. O conjunto é composto por cinco embarcações com
tamanhos e estruturas diferentes, uma das maiores fontes
arqueológicas náuticas vikings.

Os vestígios classificados como Skuldelev 5 estão no limiar do que
classificamos como navios de guerra, com seus 26 remos e 18,3~m de
comprimento, dividindo muitas características com o navio Skuldelev 3,
um cargueiro de 14~m, 6 remos e com capacidade de carga de 4,6
toneladas, provavelmente utilizado em negócios locais, já que seu sistema
de propulsão e o volume que pode transportar é limitado se comparado a
outros exemplos. Segundo o pesquisador Jan Bill,
o Skuldelev 5 foi feito em 1040 na Zelândia, de
maneira econômica, e provavelmente a sua construção envolveu
algum tipo de coerção por parte da realeza contratante ou para o
fortalecimento da defesa local, justificando o seu tamanho. Os
carpinteiros responsáveis pela sua construção reutilizaram diversas
porções de outras embarcações: as bordas, por exemplo, pertenciam a um
bote cujo esqueleto era menor, tornando necessário fechar as antigas
aberturas dos remos para que as novas aberturas ficassem simétricas ao
tamanho do navio.

Podemos citar o Skuldelev 2 como contraste. Foi um dos maiores navios de sua
época, provavelmente construído entre 1042 e 1066 em Dublin. Sabemos,
pelo pouco que foi preservado, que ele possuiu 30~m de comprimento
e surpreendentes 30 pares de remos, possibilitados pelo esqueleto
compacto que tinha apenas 70~cm de distância entre suas
armações. É provável que ele carregasse cerca de 100 guerreiros e que
tenha sido reparado diversas vezes antes de ser afundado no fiorde em
1133. Os diversos vestígios arqueológicos em Roskilde nos mostram que os
barcos podiam ser construídos em tamanhos ainda maiores.

O Roskilde 6 é datado de 1025, próximo ao final da Era Viking, mesma
época do reinado do rei dinamarquês Canuto, o Grande. Dos vestígios
podemos observar que somente a quilha é medida em 32~m de
comprimento (sendo que a reconstrução parcial do navio acusa um tamanho
total de 36~m), uma boca de 3,5~m e uma altura por volta de
1,7~m, contando provavelmente com 74 remos para a sua propulsão. A
solução para uma quilha tão longa é um conjunto de duas escarfagens
medindo 2~m que une três seções de madeira, solução única até agora
nos achados envolvendo a arqueologia náutica do período estudado.

Por fim, um exemplo de como, mesmo no período das
especializações navais, alguns navios continuaram apresentando
modificações distintas em sua construção. Os vestígios do barco Hedeby
1, construído em torno de 985, revelam um cuidado requintado na seleção
de seu material e de seu desenho, mesmo que no final ele tenha sido
utilizado como Brulote (navio em chamas lançado contra as estruturas
inimigas) em Hedeby, no início do século~\versal{XI}. Com um comprimento total de
30,9~m, 60 remos, com 2,6~m de borda e uma altura de apenas
1,5~m na meia-nau, era uma embarcação bem estreita. A madeira
utilizada na sua construção provém do Báltico ocidental, sugerindo que,
com as medidas específicas dessa embarcação, Hedeby 1 foi utilizado
especialmente na mesma região.

\SIG{Pablo Gomes de Miranda}

Ver também Bússola solar; Gokstad; Navegação marítima; Oseberg; Pedra
solar.

\begin{itemize}
\item \versal{BILL},~Jan.~Ships~and Seamanship.~In: \versal{SAWYER}, Peter (ed.). \emph{The Oxford Illustrated History of the Vikings}. New York: Oxford University
Press, 2001, pp. 182-201.

\item \versal{BILL},~Jan. Viking Ships and The Sea. In: \versal{BRINK}, Stefan; \versal{PRICE}, Neil (orgs.). \emph{The Viking World}. New York: Routledge, 2008, pp.
170-180.

\item \versal{GRAHAM-CAMPBELL}, James. \emph{Os Viquingues: origens da cultura
escandinava}. Barcelona: Folio, 2006.

\item \versal{HEIDE}, Eldar. \emph{The Early Viking Ship Types}. Bergen: Stiftelsen
Bergens Sjøfartsmuseum, 2014.

\item \versal{ROESDAHL}, Else. \emph{The Vikings}. London: Penguin Books, 1998.
\end{itemize}

\section{\versal{ENCOMIUM EMMAE REGINAE}}

Também conhecido como \emph{Gesta cnutonis regis}, o \emph{Encomium
emmae reginae} foi patrocinado pela rainha Emma, viúva de Canuto, rei da
Inglaterra e Dinamarca, provavelmente na metade ou no terceiro quarto do
século~\versal{XI}, e escrito na região de Flandres, provavelmente no mosteiro de
Saint Bertin. Emma era filha de Ricardo~\versal{I} da Normandia. Durante o
reinado de Canuto, Emma desfrutou de um considerável \emph{status},
sendo representada em documentos escritos e visuais, o que contrasta com
o período de exílio pelo qual passou após a morte de Canuto, um contexto
no qual esteve afastada das relações de poder. Durante o reinado de seu
filho, Harthacnut, que assim como seu pai foi rei da Inglaterra e da
Dinamarca, Emma voltou a ter uma proeminência em termos de reinado, e
foi durante este contexto que o \emph{Encomium emmae reginae} foi
encomendado.

Existe somente um manuscrito do \emph{Encomium}, datado da metade do
século~\versal{XI} e escrito em latim (British Library, \versal{BL} Addtional 33241). De
acordo com a descrição de Simon Keynes, o manuscrito contém, logo no início, uma imagem
na qual a rainha Emma está entronizada e aparece
recebendo um livro de um personagem que está ajoelhado, enquanto dois
homens, provavelmente seus filhos Harthacnut e Eduardo, olham com
admiração em sua direção.

O autor do documento, chamado pela historiografia de \emph{Encomiast},
tem sua origem desconhecida, a qual é objeto de discussão. É necessário
considerar o contexto no qual o \emph{Encomium emmae reginae} foi
escrito, ou seja, um contexto de crise política e de necessidade de
legitimação por parte da rainha Emma através de seu filho Harthacnut, já
que a obra serviu para legitimar seus interesses na corte, o
que fez do \emph{Encomium} um texto latino produzido com o objetivo de
ter um impacto particular em uma determinada audiência. De acordo com
Elizabeth M. Tayler, a construção da narrativa encontrada no
\emph{Encomium emmae reginae} foi uma resposta a um problema social e
político do contexto de composição do documento, no qual houve a
necessidade de apresentar uma versão dos eventos ocorridos para um
público que conhecia a rainha Emma.

Deve-se destacar a presença de uma literatura latina na narrativa do
\emph{Encomium}, já que há referências a autores como Salústio,
Lucano, Ovídio, Horácio, Juvenal, entre outros. Em seu conteúdo
encontramos uma proposta de glorificação de Canuto e uma difamação da
imagem do filho ilegítimo de Canuto, Haroldo Pé de Lebre, que conduziu
Emma e Harthacnut, filho de Canuto, para o exílio. Provavelmente o
objetivo da composição do \emph{Encomium} era auxiliar na ascensão de
Harthacnut ao trono inglês contra Eduardo, o Confessor, filho de Emma
com o rei Etelredo. O \emph{Encomium} apresenta um prólogo, um
\emph{argumentum} e três livros. Mesmo que o foco do documento seja
a conquista danesa da Inglaterra entre os anos de 1013 e 1016,
realizada por Sueno Haraldsson, pai de Canuto, apresentando o
reinado de Canuto e os acontecimentos após a sua morte, bem como destacando a
questão da sucessão do trono inglês, é necessário considerar a
associação feita por alguns autores, que estabelecem o \emph{Encomium emmae
reginae} como parte de uma tradição literária conhecida como relatos
religiosos de personagens femininas reais, já que o \emph{Encomiast}
iniciou a narrativa do documento exaltando a rainha Emma como a mais
admirável das mulheres em seu modo de vida.

\SIG{Luciano José Vianna}

Ver também Canuto, o Grande; Dinamarca da Era viking; Fontes primárias.

\begin{itemize}
\item \emph{\versal{Encomium emmae reginae}}. Edited by Alistair Campbell with a
Supplementary Introduction by Simon Keynes. Cambridge: Cambridge
University Press, 1998.

\item \emph{\versal{Encomium emmae reginae}}. Edited for the Royal Historical Society
by Alistair Campbell. Camden Third Series, vol. \versal{LXXII}. London: Offices
of the Royal Historical Society, 1949.

\item \versal{HAYWOOD}, John. \emph{Encyclopaedia of the Viking Age}. London: Thames \&
Hudson, 2000, pp. 63-64.

\item \versal{HOLMAN}, Katherine. \emph{Historical Dictionary of the Vikings}. Lanham,
Maryland, and Oxford: The Scarecrow Press Inc., 2003, pp. 87-89.

\item \versal{JOHN}, Eric. \emph{The Encomium Emmae Reginae: a Riddle and a Solution}.
\emph{Bulletin of the John Rylands Library}, n. 63, vol. 1, 1980, pp. 58-94.

\item \versal{TYLER}, Elizabeth M. Talking about History in eleventh-century England:
the \emph{Encomium Emmae Reginae} and the Court of Harthacnut.
\emph{Early Medieval Europe}, n. 13, vol. 4, 2005, pp. 359-383.
\end{itemize}

\section{\versal{ERA VIKING}}

\emph{Conceito geral}: A Era Viking é considerada um período de grande
irrupção e atividade do Norte nas terras povoadas do sudoeste e sudeste
europeu. Comumente, o período é balizado entre as datas de 800 a 1100
depois de Cristo, com diversas variações e diferenças cronológicas ou
conceituais, dependendo do autor. Também de forma tradicional é dividida
em dois períodos: Primeira Era Viking, que se inicia com as incursões
hostis, os ataques de surpresa (razias) no final do século~\versal{VIII} e as
povoações criadas na região escocesa, britânica e francesa. A Segunda
Era Viking foi caracterizada pela criação de dinastias permanentes e do
processo intensificado de cristianização. Os mercadores escandinavos
continuaram afetando o processo de urbanização da Europa. Segundo Henry
Loyn, durante o final desse período, um escandinavo deixava de ser um
viking quando se tornava cristão.

\emph{Historiografia}: Apesar do termo viking ter sido utilizado após o
renascimento em diversas línguas escandinavas, foi com o romantismo
moderno que ele adquiriu um novo contexto, atrelado às ideias
nacionalistas. No século~\versal{XVIII}, a Suécia e Noruega estavam ocupadas com
questões de fronteira. A Suécia perde parte de sua região oriental para
a Rússia e no início do século~\versal{XIX} entra em conflito com a Noruega.
Nesse contexto da formação de uma nova identidade espacial e imaginária,
os vikings tornam-se um tema privilegiado tanto na arte em geral
(literatura, teatro, festivais) quanto na academia e nos trabalhos de
reconstituição do passado, fornecendo valores idealizados para a nova
sociedade almejada pelas nações escandinavas. Nas leituras públicas de
História do sueco Erik Gustava Geijer, em 1815, enfatizava-se a vida
nórdica nos tempos vikings como sendo de total liberdade, em contraste
com o feudalismo vigente na Europa continental. Assim a suposta
``natureza'' destes antigos povos será utilizadas como modelo político,
artístico e cultural.

A primeira utilização do termo Era Viking (em sueco \emph{Vikingatiden})
foi no artigo \emph{Om do gamle Nordboers Bekjendtskab med don
pyrenæiske Halvöe} (``Sobre o conhecimento da presença nórdica na
Península Ibérica''), escrito por E. C. Wérlaoff e publicado na revista
\emph{Annaler for Nordisk oldkyndighed og historie} (\emph{Anais da
Antiguidade e História Nórdica}) em 1836. Nesse artigo não há maiores
explicações ou detalhamentos sobre esse conceito, empregado apenas
para determinar o período em que os vikings realizavam suas atividades. Em algumas
sistematizações europeias sobre a história escandinava (como
\emph{Histoire des Étates Scandinaves}, de A. Geffroy, 1851), a
experiência nórdica é inserida dentro do expansionismo germânico e
tratada sem maiores diferenciações em um período que se estende do
século~\versal{V} ao \versal{XI} d.C.

Entre os anos 1830 a 1850, a literatura acadêmica escandinava foi
fortemente embasada pelas pesquisas arqueológicas provindas da
Dinamarca, inicialmente pela classificação das três Eras (Pedra, Bronze,
Ferro), criadas por Christian Jurgensen Thomsen, pesquisador do Museu
Nacional de Copenhagem. Logo em seguida, em 1843, o livro \emph{Danmarks
Oldtid oplyst ved Oldsager} (\emph{Antiguidades da Pré-História dinamarquesa}),
do arqueólogo Jens Worsaae é publicado. Nele, os vikings são inseridos
na Idade do Ferro, num período que vai das migrações germânicas à
cristianização da Escandinávia, sem distinções muito rígidas entre os
momentos anteriores e os seguintes às expedições predatórias na virada
dos séculos~{\versal{VII} ao \versal{VIII}}, por exemplo. Outra designação genérica para a divisão da Era do Ferro muito comum neste livro é ``tempos pagãos'', não
estabelecendo diferenças entre os primeiros grupos humanos da tecnologia
do ferro (na Antiguidade) até o momento em que são estabelecidas as
primeiras sedes de bispados na Escandinávia, já no final da Alta Idade
Média. No subcapítulo sobre pedras rúnicas, Worsaae comenta sobre os
``tempos antigos das runas'', o ``período pagão'' e o ``período cristão
inicial'', a respeito da temporalidade das inscrições. Em um dos trechos
mais importantes da obra, o arqueólogo analisa os vestígios dinamarqueses como
verdadeiros monumentos do passado, e ao comentar sobre o século~\versal{VIII}, o
caracteriza como um ``período guerreiro''. Como grande parte
dos acadêmicos de sua época, Worsaae foi influenciado pelos poetas e
pintores que adotaram a imagem de ``nobres selvagens'' do Norte no
espírito de Montesquieu.

Logo em seguida, em 1952, Worsaae publica em Londres o livro \emph{Danes
and norwegians in england, Scotland and Ireland}, baseado em suas
investigações extensivas sobre os monumentos nórdicos preservados nas
ilhas britânicas. A diferença em relação ao livro anterior é a
concentração exclusiva em estudos sobre a Alta Idade Média, no momento
em que os escandinavos iniciam suas ações predatórias e o posterior
desenvolvimento de colônias e áreas de influência cultural. Logo no
prefácio o autor desmistifica o erro de uma interpretação corrente no
período (a palavra viking ser traduzida ao inglês como \emph{vi-king},
rei do mar) e a relaciona diretamente ao islandês \emph{vik} e ao dinamarquês
\emph{vig}, baía. Apesar de em diversos momentos da obra empregar o
termo viking para pirata (concepção ocupacional), também o emprega como
sinônimo para nórdico (concepção étnica). Apesar de ainda não empregar
objetivamente o termo Era Viking, Worsaae já caminha muito neste
sentido, ao utilizar a expressão ``tempos
dos vikings'', não estipulando precisamente o início deste período
(manifestado tradicionalmente com os ataques predatórios do século
\versal{VIII}), mas com um fim bem delimitado: o início da dinastia dos Waldemar
na Dinamarca, em 1146 d.C. As publicações de Jens Worsaae auxiliaram a
consolidar o estudo do passado nórdico, especialmente sua materialidade,
mas ainda não possuíam nenhuma noção mais rigorosa de cronologia ou ao
menos um diálogo maior entre os registros históricos e o método
arqueológico, que ainda estava em desenvolvimento neste período.

Segundo Jørgen Haavardsholm, alguns historiadores passaram a
utilizar o conceito dentro de padrões nacionalistas, a exemplo de Peter
Andreas Munch (1810-1863), diretor de um museu etnográfico que utilizou
a Era Viking como argumento para a unificação da Noruega. Ele apresentou
um mito de origem no qual os noruegueses seriam mais loiros e puros do que
os suecos e dinamarqueses. Utilizando figuras históricas e literárias
nórdicas, ele apresentou uma visão romântica do Período Viking:
guerreiros brutais, mas organizados, criativos e preparando a vinda da
cristandade.

O termo Era Viking foi empregado genericamente após a década de 1860,
especialmente com o escritor Frederik Svanberg, que utilizou o conceito
identificando-o com o final da periodização anterior de Idade do Ferro.
A partir dele, os intelectuais escandinavos passam a utilizar a Era
Viking como um conceito e um momento fixo na história escandinava.
Devido ao escasso conhecimento arqueológico da cultura material nórdica
da Alta Idade Média, a periodização utilizada para a Era Viking proveu de
documentos escritos, especialmente as anglo-saxônicas descrevendo os
ataques às ilhas britânicas.

O arqueólogo sueco Oscar Montelius é um exemplo do período de
transição da utilização do termo Idade do Ferro até a consolidação 
do termo Era Viking. Inicialmente,
seus primeiros livros enfatizavam as terminologias para a temporalidade
que eram convencionais até Jens Worsaae, como no livro \emph{Från
jernåldern} (\emph{Sobre a Idade do Ferro}, 1869). Mas na década de 1870 passa
não somente a adotar a nova convenção, mas também a ser um grande
entuasiasta dela. No artigo \emph{Lifvet i Sverige under vikingatiden}
(``A vida na Suécia durante a Era Viking''), da revista \emph{Förr och Nu}
(1872), ele compreende este período como um "tempo heroico", no momento
em que os ``filhos do Norte'' estavam com as suas vidas muito limitadas e
``vagavam pelos mares'' em busca de ``honra e ouro'', para estabelecer
``novos domínios em países distantes'' e ``com o seu sangue rejuvenescer
os povos do sul da Europa''. Também no livro \emph{Om lifvet i Sverige
under hednatiden} (\emph{Da vida na Suécia durante o paganismo}) de 1873,
dedica o capítulo final da obra para tratar da \emph{Vikingatiden}.

Aqui a delimitação espaço-temporal acaba reunindo diversas idealizações
românticas sobre a figura do viking -- é o momento em que se firmam as
raízes históricas de uma Europa em formação, servindo de modelo de
identidade formativa para os países escandinavos ao final do século \versal{XIX}.
Esse conceito de temporalidade passa neste momento a ser divulgado em
manuais escolares escandinavos, obras acadêmicas e de divulgação, mesmo
nos Estados Unidos. Mais do que uma área de investigação de
historiadores e arqueólogos, a Era Viking transformou-se num conceito
artístico e político que auxiliou diversos países oitocentistas a
moldar sua identidade nacional.

Para as pesquisas em língua inglesa, certamente a publicação mais
importante na virada do século \versal{XIX} para o \versal{XX} foram o livro \emph{The
Viking Age} (1890), de Paul du Chaillu e o periódico \emph{The saga Book
of the viking club}, 1895. A obra do
franco-americano Paul du Chaillu foi dividida em dois grandes volumes, o
primeiro cobrindo aspectos de civilização, cultura e mitologia, enquanto
o segundo abrangia elementos do cotidiano e da cultura material. Logo no
prefácio, Chaillu caracteriza o norte europeu como local de passado
glorioso, berço de uma ``nova época'' na história da
humanidade, ancestrais dos ingleses (cujo país ele denomina de ``a mãe das
nações''), caracterizando os nórdicos no primeiro capítulo não como
bárbaros, mas pelo contrário, como criadores de civilizações. De modo diferente
de muitos britânicos da época, o autor não utiliza o ataque de
Lindisfarne como marco inicial do período, e sim o segundo século depois
de Cristo, terminando no século~\versal{XII}. Justamente por incluir capítulos
sobre a Era do Ferro da Escandinávia, Chaillu não concebe muitas
diferenças culturais entre a Antiguidade nórdica e a medieval, preocupando-se
muito mais com aspectos civilizatórios, como a mitologia, as
runas e a sociedade.

Quanto ao período referido em \emph{The saga Book}, bem ao contrário, os estudos
focam no período após as razias da Alta Idade Média até o reinado do
rei Canuto ou a batalha da ponte de Stamford. São utilizados
concomitantemente em diversos artigos de pesquisadores
norte-americanos, escoceses e britânicos os termos ``Período
Viking'' e ``Era Viking'' -- especialmente em contextos relacionados à
cultura material. Um tipo de estudo particularmente enfatizador deste
emprego encontra-se em análises de monumentos dinamarqueses do século~\versal{X} da
região da Cúmbria, na Inglaterra.

No início do século~\versal{XX}, diversos arqueólogos popularizaram o conceito de
Era Viking em suas publicações pela Europa: \emph{Nordisk og fremmed
Ornamentik i Vikingetiden} (1921), do dinamarquês Johannes Brøndsted;
\emph{Vikingetidens smykker} (1928), do norueguês Jan Peterson;
\emph{Die normannen der Wikingerzeit} (1930), do russo W. J. Raudonikas;
\emph{Die Schatzfunde Gotlands der Wikingerzeit} (1947), do sueco Mårten
Stenberger. Mas certamente o estudo arqueológico mais influente sobre o
conceito de Era Viking -- especialmente de um ponto de vista da
cronologia e das divisões temporais mais precisas para período -- ecnontra-se
no livro \emph{De Norske Vikingsverd} (1919), do norueguês Jan
Petersen. Nele, o arqueólogo analisa os tipos de espadas escandinavas
por meio do estilo de seus diferentes punhos, criando uma tipologia
utilizada até nossos dias: tipo transicional (\versal{A} e \versal{B}), utilizados entre
os séculos~{\versal{VII} e \versal{VIII}}; a Era Viking inicial, com os tipos \versal{C} e \versal{D} (entre 750 e 800 d.C.); os tipos \versal{H} e \versal{I} (datados entre 800 e 950 d.C.); tipo \versal{K} e
\versal{M} (século \versal{IX}); tipos \versal{O} e \versal{X} (século~\versal{X} d.C.), denominados de estilos
tardios. Em 1927 o renomado arqueólogo britânico Mortimer Wheeler
publica \emph{London and the Vikings} e amplia a tipologia e a
cronologia das espadas nórdicas, conservando o ano 800 como marco
inicial (tipo \versal{I}), mas prolongando o final da Era Viking entre os séculos~\versal{X} e \versal{XI} d.C. (tipo \versal{VIII} e \versal{IX}). Essas últimas seriam uma forma transitória entre a espada nórdica tradicional e a espada da cavalaria medieval.
Assim, a Era Viking passou de um período construído historicamente para
um período definido pela arqueologia.

Outros estudos que reforçaram a classificação temporal da Era Viking
pela cultura material foram os que envolviam arte. Thomas Downing Kendrick
publica em 1949 a obra \emph{Late saxon and viking art}, explorando os
diferentes estilos artísticos nórdicos e suas periodizações internas e
externas -- especialmente as desenvolvidas nas Ilhas Britânicas. Essa
mesma tendência seria seguida depois por outros especialistas
britânicos, como David Wilson e Graham Campbell, além do dinamarquês Ole
Klindt-Jensen. Todos esses pesquisadores convencionam a arte nórdica da
Alta Idade Média como sendo delimitada pelo estilo inicial de Oseberg
(800 d.C.) e finalizada pelo estilo tardio de Urnes (1.100 d.C.),
popularizando essas duas datas como limítrofes para a própria Era Viking
em geral. Esse mesmo referencial cronológico seria seguido estritamente
por outros autores escandinavos, como o arqueólogo dinamarquês Johannes
Brøndsted com o livro \emph{Vikingerne} (1960). Vinte anos mais
tarde, outra arqueóloga de mesma ascendência, Else Roesdhal, em seu
livro \emph{Vikingernes verden} (1987), repete essa tendência,
abandonando o marco categórico do ataque a Lindisfarne (793), muito
popularizado pela historiografia inglesa, e reitera o século~\versal{VIII} ou o
ano 800 como data inicial da Era Viking. Para a sua finalização,
novamente deixa de lado o padrão britânico (em especial a morte do rei
Canuto~\versal{III} em 1042) e elegendo o século~\versal{XI} como marco final.

Um momento especial de consagração do conceito no século~\versal{XX} foi a
publicação do livro \emph{The Age of the Viking} (1962), do historiador inglês
Peter Sawyer. Nele, o autor concebe a Era Viking como um período
de expansão de uma população amadurecida com experiências anteriores dos
grupos germânicos (como os saxões e os francos), mas acima de tudo como
uma noção fabricada pelos acadêmicos modernos e sujeita a adaptações e
transformações críticas.

\emph{Novas concepções sobre a Era Viking}: A partir dos anos 1980,
alguns acadêmicos iniciaram algumas contestações ao conceito de Era
Viking, além de outros proporem novas datações para o seu início e
término, bem como implicações para o seu uso. O livro \emph{Decolonizing
Viking Age} (vol 1., 2003) do arqueólogo sueco Frederik Svanberg
argumenta que o conceito de Era Viking foi um sistema de conhecimento
construído essencialmente durante o século \versal{XIX} e suas ideias básicas se
mantêm até hoje. Esse sistema foi fortemente influenciado pelos
pensamentos evolucionistas e nacionalistas de sua época e pode ser
caracterizado como um colonialismo sobre o passado. Em uma perspectiva
pós-colonial, Svanberg criticou o referencial de uma única ``cultura
viking'', sugerindo que o conceito de nórdico antigo substitua a velha e
romântica noção de Era Viking.

Para Richard Hodges (\emph{Goodbye to the vikings?,} 2006) a afirmação
de que os vikings foram agentes de mudança na história europeia foram
exagerados, percebendo a diáspora nórdica mais como consequência de
fatores econômicos do que culturais. Sua visão dos ataques a Inglaterra
e França como fruto da economia carolíngia foram criticados por Clare
Downham. Charlotta Hillerdal em uma resenha no periódico \emph{Journal
of Archaeology and Ancient History} (n. 17, 2016) aponta os diversos
questionamentos sobre esse conceito: um período histórico que compreende
diversas perspectivas cronológicas, geográficas e culturais, dependendo
da região. Assim, a Era Viking é caracterizada enquanto um período de
migrações e movimentos dos povos, uma contínua renegociação de áreas
culturais. Na Finlândia, por exemplo, a fundação de Staraia Ladoga em
753 d.C. e a falta de centralização política e a cristianização tardia
na região tardam o desfecho da Era Viking na região para 1250, com o
período tardio situado entre 1050-1250.

Estudando o sítio de Borre (Noruega), o arqueólogo Bjørn Myhre
considerou que o a arqueologia da Era Viking é superdependente do
referencial do historiador, baseado em fontes escritas, especialmente em
questões relativas a religião, política e sociedade. Já os arqueólogos
têm se dedicado muito mais a problemas de cronologia, arte, tecnologia,
sepulturas e assentamentos. Mas nas últimas três décadas o conhecimento
arqueológico progrediu muito, além de contar com diversas outras
disciplinas correlatas em apoio às análises dos vestígios. Dessa
maneira, estudos demonstram que fortes centros políticos já existiam no
século \versal{VIII}, e os estilos artísticos tradicionais
do Período Viking já ocorriam antes do século~\versal{IX} (como no estilo de
Broa). Também em sepultamentos existem evidências de primeiros contatos
entre britânicos e escandinavos desde a Era das Migrações Nórdicas, estendendo"-se
até o Período Vendel. Análises vêm demonstrando que embarcações a vela
já eram usadas no início do século \versal{VII}. Evidências de incursões vikings
foram encontradas nas ilhas Hébridas em 750 d.C. e nas ilhas Faroé ainda
no século \versal{VII}. Assim, Bjørn Myhre conclui que o início da Era Viking
pode ser fixado em qualquer um dos vários pontos ao longo de uma escala
de tempo que vai entre os anos de 700 a 800 d.C., dependendo do critério
escolhido.

Os estudos das conexões entre Escandinávia e Leste Europeu também
modificaram o conceito de Era Viking, ampliando o impacto antes visto
centralmente apenas nas Ilhas Britânicas e Bretanha francesa. Estudos
recentes sobre o Estado dos rus, suas redes de conexões comerciais com
o Oriente e a Europa, postos militares e a política de dominação com as
populações regionais justificam a criação de uma temporalidade para o
período de 700 a 1100 d.C. Em 2008, foi encontrado um sepultamento em um
barco na região de Saaremaa (Estônia), com 28 esqueletos de suecos,
dentre os quais vários nobres. Eles possivelmente foram enterrados após
um saque malsucedido, visto que a maioria morreu perfurada, decapitada
ou teve o topo do crânio arrancado. Os corpos foram juntados e colocados
no fundo do casco, cobertos com panos e escudos. O evento data de
750 d.C. segundo o arqueólogo Jüri Peets.

Em seu artigo ``Viking ethnicities'' (\emph{History Compass}, 2012), a
historiadora britânica Clare Downham defendeu a continuidade do termo
Era Viking, alegando que esse conceito possui grande importância
cultural, econômica e política para diversas regiões na história
europeia. As mais recentes teses de doutorado em Arqueologia e História
tanto de países escandinavos quanto de línguas germânicas em geral, além
de livros e artigos especializados, continuam a utilizar
majoritariamente o conceito, com alterações quanto a cronologias ou
amplitude da diáspora nórdica.

\SIG{Johnni Langer}

Ver também Dinamarca da Era Viking; Escandinávia; Finlândia da Era
Viking; Islândia da Era Viking; Noruega da Era Viking; Suécia da Era
Viking; Viking.

\begin{itemize}
\item \versal{AYOUB}, Munir Lutfe. Repensando o conceito de período Viking. \emph{Anais
do \versal{XXI} Encontro Estadual de História}, \versal{ANPUH}, 2012, pp. 01-14.

\item \versal{BLANCK}, Dag. The transnational viking. \emph{Journal of transnational
American Studies}, vol. 7, n. 10, 2016, pp. 01-19.

\item \versal{BJORN}, Myhre. The beggining of the Viking Age: some current
archaeological problems. In: \versal{FAULKES}, Anthony \& \versal{PERKINS}, Richard.
\emph{Viking revaluations}. London: University College London, 1993, pp.
182-203.

\item \versal{HAAVARDSHOLM}, Jørgen. \emph{Vikingtiden som 1800-tallskonstruksjon}.
Oslo: Det historisk-filosofiske fakultet, Universitet i Oslo, Unipub,
2005.

\item \versal{HAGERMAN}, Maja. \emph{Det rena landet}: Om konsten att uppfinna sina
förfäder. Stockholm: Prisma, 2006.

\item \versal{LIND}, John. ``Vikinger'', vikingetid og vikingeromantik. \emph{Kulm}
(Årbog for Jysk Arkæologisk Selskab), vol. 61, 2012, pp. 151-168.

\item \versal{PPETS}, Jüri \emph{et al}. Archaeological investigations of pre-viking
age burial boat in Salme Village at Saaremaa. \emph{Archaeological
Fieldwork in Estonia}, 2010, pp. 29-48.

\item \versal{SORENSEN}, Preben Meulengracht \& \versal{ROESDAHL}, Else. \emph{The Waking of
Agantyr}: The Scandinavian Past in European Culture. Aarhus: Aarhus
University Press, 1996.

\item \versal{SVANBERG}, Frederik. \emph{Decolonizing Viking Age,} vol. 1. Acta
Archaeologica Lundensia. Series in 8° 43. Lund: Lund University
Puvblications, 2003.
\end{itemize}

\section{\versal{ÉRICO MACHADO SANGRENTO (ERIK HARALDSSON)}}

Érico Machado Sangrento (Erik Haraldsson) nasceu por volta de 885, como um dos 
vários filhos do rei Haroldo
Cabelos Belos (c. 850-943) e da rainha Ranghild, uma das últimas esposas
de Haroldo. Seu pai foi o primeiro rei de uma Noruega unificada, embora
a Noruega daquela época não correspondesse territorialmente ao atual
país. Apesar de não ser o primogênito, Haroldo decidiu abdicar do trono
em favor de seu filho Érico. Os motivos não são claros, mas isso teria
revoltado os irmãos mais velhos. A história relata que, temendo que fosse
destronado por seus vários irmãos, Érico matou pessoalmente
alguns deles. Esses atos de fratricídio lhe renderam o título póstumo de ``Machado Sangrento'' (\emph{Blóðøx}, em nórdico antigo).

O fato de ter assassinado a própria família levou a nobreza norueguesa a
se posicionar contra seu rei, já que Érico assumiu o trono por volta de
930 ou 933. Com isso, alguns de seus opositores decidiram apoiar um dos
irmãos de Érico, o príncipe Haakon, o Bom (c. 920-961), um dos últimos
filhos que Haroldo teve. Nessa época Haakon vivia na Inglaterra sob
proteção do rei Athelstan (924-939). Tal condição era uma trégua entre o
rei Haroldo Cabelos Belos e o rei inglês e dessa forma Haakon foi educado à
moda inglesa e convertido ao cristianismo.

Quando a notícia de que Érico havia se tornado rei e estava matando a
família chegou aos ouvidos do monarca inglês, Athelstan temeu que o
impulsivo e sanguinário novo rei da Noruega rompesse a trégua firmada
por seu pai. Com isso, Athelstan enviou Haakon de volta para casa,
fornecendo-lhe navios e homens, no intuito de que destronasse seu irmão mais
velho. Haakon conseguiu apoio na Noruega e declarou guerra ao irmão,
forçando-o a renunciar ao trono em 934 ou 935. Érico ainda tentou reaver o
trono, mas como não obteve sucesso exilou"-se no arquipélago das Órcades,
ao norte da Escócia. Posteriormente viajou para o reino de York.

No reino de York, território de influência dinamarquesa sobre o antigo
reino anglo-saxão da Nortúmbria, Érico passaria o restante da vida e
disputaria o trono. Érico chegou a York na década de 940, passando alguns
anos participando de incursões locais de pilhagem na Escócia, Irlanda e
na Inglaterra. Nesse tempo ele firmou alianças e com isso disputou o
trono, assumindo-o brevemente entre 947-948. Seus atos na Nortúmbria
levaram o rei inglês Edredo (946-955) a combatê-lo. Érico Machado
Sangrento não dispunha de um exército forte para assegurá"-lo no trono, e
com isso renunciou.

Aproveitando a renúncia de Érico, um chefe chamado Olavo Cuárán de
Dublin viajou para a Nortúmbria e assumiu o trono de York. Nos anos
seguintes, Érico reuniu novos aliados e homens para tentar reaver o trono,
obtendo vitória sobre Olavo, e reassumindo o poder em 952. No entanto,
o novo reinado de Érico também seria breve. Após cerca de dois anos de
governo, Érico sofreu uma traição e com isso perdeu muito de seu apoio.
Enfraquecido, foi cercado por tropas inglesas e morto em combate em
Stainmore. Sua morte marcou o fim do domínio viking na Nortúmbria, assim
como pôs fim ao Reino de York.

A história de Érico Machado Sangrento foi relatada em diferentes crônicas
de origem saxã, norueguesa, irlandesa etc., embora nem sempre tratassem
de narrativas coesas. Na \emph{Heimskringla} (\versal{XIII}), de autoria atribuída a Snorri
Sturluson, conta-se como Érico matou alguns homens e participou de
expedições pelo mar Báltico, chegando a territórios russos. Por sua vez
a \emph{Saga de Egil} (\emph{Egils saga}), traz o poema \emph{Höfuðlausn}, escrito por Egil
Skallagrimsson, o qual atua como um panegírico para Érico. Outra menção
honrosa ao último rei de York advém do \emph{Eiríksmál}, canção encomendada
pela viúva de Érico, Gunnhild, cuja obra enaltecia os feitos guerreiros
do falecido marido, e até mesmo menciona sua viagem ao Valhala.

Apesar de haver controvérsias e dúvidas sobre todos os feitos narrados em
poemas e sagas referentes a Érico Machado Sangrento, ainda assim, sua
história como monarca foi mais significativa no reino de York do que
como rei da Noruega. Em York, apesar de ter governado de forma breve por
dois mandatos a ponto de mandar cunhar moedas com seu nome em latim,
Érico conseguiu prolongar até o último suspiro aquele reino viking no
norte da Inglaterra.

\SIG{Leandro Vilar Oliveira}

Ver também Era viking; Jorvik; Noruega da Era Viking; Viking.

\begin{itemize}
\item \versal{GRAHAM-CAMPBELL}, James (org.). \emph{Os vikings}. Barcelona: Folio \versal{S.A},
2006.

\item \versal{HOLMAN}, Katherine. \emph{Historical dictionary of the vikings}. Lanham:
Scarecrow Press Inc, 2003.

\item \versal{STURLUSON}, Snorri. \emph{Heimskringla}. Translated Alison Finlay and
Anthony Faulkes. London: University College London/Viking Society for
Northern Research, 2011.

\item \versal{WINROTH}, Anders. \emph{The Age of the Vikings}. Princenton: Princenton
University Press, 2014.
\end{itemize}

\section{\versal{ÉRICO, O VERMELHO}}

\emph{Eiríkr Þorvaldsson} ou Érico, o Vermelho (\emph{Eiríkr hinn
rauði}), cujo apelido é associado à coloração de seu cabelo e barba,
foi um nórdico responsável pela descoberta da Groenlândia e influenciador
direto da descoberta do continente americano, sendo um dos personagens
mais famosos da literatura nórdica. Nascido em meados de 950, na Noruega,
filho de Thorvaldr, pai e filho foram obrigados a deixar a região por
conta de assassinatos e tomaram terras em Hornstrandir, como nos revela
a \emph{Saga de Eiríkr, o Vermelho}, enquanto ele tinha ainda tenros dez anos
de idade.

A história de vida de Érico é marcada por conflitos e assassinatos, que servem de dínamo para o avanço das narrativas de suas fontes literárias. Pouco tempo após a
morte de seu pai, Érico se casa, mudando-se para o ``Lugar de Érico''.
Nesse local, envolve-se em conflitos que culminam no assassinato, por parte de Érico, de duas pessoas.  Como resultado disso, tornou-se alvo de um processo
legal e foi expulso da região -- a região de Haukadalr. De lá, ele
toma as terras de Brokey e Øxney, passando a morar em Traðir, ao sul da Islândia. Mas, novamente, devido a uma questão de empréstimo de
tábuas, envolve-se em novos confrontos e mortícinios. Estes
últimos são solucionados na assembleia de Thórnes, em que
Érico e sua gente se tornam proscritos.

Sabendo da sua situação delicada na ilha, Érico resolve buscar as terras
relatadas por Gunnbjörn, filho de Úlfr Corvo, uma vez que foram descritas como uma terra nova e
com abertura para um sujeito com um histórico
familiar de expulsões legais de suas localidades. Ele zarpa
da ilha, após se esconder de seus inimigos, e busca a localidade
relatada, encontrando as geleiras de Hvítserkr, já na Groenlândia. A
partir desse ponto, ele passa a explorar toda a região: mora em alguns
lugares, conhece e nomeia outros (c.
982-983). Após cerca de dois anos, ele retorna para a Islândia, trazendo
notícias da terra que denominará ``Groenlândia'' (que significa Terra
Verde), ``[...] pois disse que as pessoas desejariam muito ir até lá
se a terra fosse bem denominada'' (Anônimo, 2007b, p. 90).

Érico, com sua mulher Thjóðhildr, tiveram três filhos homens:
Thorsteinn, Thorvaldr (as \emph{Sagas do Atlântico Norte} divergem 
quanto à existência de Thorvaldr como um outro filho de Eíríkr) e Leifr, 
sendo este último o
grande descobridor da América. Além destes, Érico teve também uma filha ilegítima chamada
Freydís. Na Groenlândia,
Érico tomou para si as regiões do fiorde de Eiríkr, onde ficava
Brattahlíð, localidade em que morava. Durante toda sua vida, foi um
sujeito considerado por muitos e apreciado pelas suas conquistas e como
uma liderança forte em toda região. As \emph{Sagas do Descobrimento da América}
nos revelam muito de Érico, embora, por outro lado, a composição
estilística das sagas acabem por nos revelar menos do que gostaríamos.
Afinal, o estilo mais objetivo e seco das sagas frequentemente apresenta apenas
uma superficialidade psicológica dos participantes das narrativas, o que
dificulta criar hipóteses sobre algumas características de Érico.

Essas mesmas narrativas literárias exaltam a importância de Érico, mas
são seus filhos que se tornam os protagonistas das narrativas. Leifr, que
traz o cristianismo para a Groenlândia, mesmo que seu pai se mantenha pagão,
acaba por ser o grande personagem do texto. Soma-se a isso a noção de
destino como grande recurso narrativo para a retirada
de Érico do descobrimento do continente americano, como se vê na \emph{Saga
dos Groenlandeses}: ``Não me é destinado encontrar mais terras além desta
que agora habitamos [Groenlândia]; daqui não seguiremos mais
todos juntos'' (Anônimo, 2007a, p. 63). Essa fala ocorre
após sua queda do cavalo, o que impede acompanhar seu filho,
provavelmente como um recurso narrativo cristão, que cria
uma dimensão de nobre pagão ao redor de Érico, mas que ainda mantém seu
lugar de pagão dentro da narrativa cristã.

O fim da vida de Érico é revelado pela mesma saga,
de uma maneira típica de sua estilística: ``Naquele inverno apareceu uma
forte doença no bando de Thórir, e morreu o próprio Thórir e também
grande parte do seu bando. Naquele inverno morreu também Eiríkr
Vermelho'' (Anônimo, 2007a, p. 68). A narrativa, de forma bem seca,
revela a morte de Érico, sem de fato nos revelar uma causa, o que
faz presumir que seja através de causas naturais. Todavia, a \emph{Saga de Eiríkr,
o Vermelho} não apresenta uma morte tão prematura assim, algo que se vê, por exemplo, pelo fato de que
os que retornam da exploração na América do Norte se abrigam junto dele. Sua morte não é um consenso dentro das fontes literárias, mas se pode traçar para aproximadamente 1003-1004 d.C., embora se afirme na \emph{Saga dos Groenlandeses} que Érico morre antes
cristianismo chegar à região, inferindo a um período um pouco anterior ao ano 1000.

Considerado um dos grandes símbolos do mundo nórdico e da
cultura viking, Érico está presente em monumentos por todo o mundo, em símbolos
comemorativos e como tema de músicas e álbuns de bandas de
várias nacionalidades. Seu papel como descobridor da América é
ímpar no curso do estudo da Escandinávia, assim como a trajetória
histórica das fontes em que aparece seu nome, colocando-o como um dos mais
famosos vikings de todos os tempos. Sua liderança e força marcaram a
região da Groenlândia, fazendo a área do seu fiorde e dos vizinhos
atingirem cerca de 5 mil habitantes em seu ápice, sem falar nos
400 a 500 habitantes que foram de forma mais imediata
para a Terra Verde graças a sua descoberta e habilidade, algo que fica
evidente no \emph{Íslendingabók} (\emph{Livro dos Islandeses}, que narra sobre
os assentamentos nórdicos, falando da colonização da Islândia e
comentando sobre a descoberta e assentamentos da Groenlândia) e no
\emph{Landnámabók} (\emph{Livro da Colonização}, que descreve de forma mais
detalhada o povoamento da Islândia nos séculos~{\versal{IX}-\versal{X}} e que terce
comentários sobre a Terra Verde).

\SIG{José Lucas Cordeiro Fernandes}

Ver também Brattahlid; Islândia da Era Viking; Sagas do Atlântico Norte.

\begin{itemize}
\item \versal{ANÔNIMO}. A Saga do Groenlandeses. In: \emph{As três sagas Islandesas.}
Tradução de Théo Moosburger. Curitiba: Editora \versal{UFPR}, 2007a.

\item \versal{ANÔNIMO}. A Saga de Eiríkr Vermelho. In: \emph{As três sagas Islandesas.}
Tradução de Théo Moosburger. Curitiba: Editora \versal{UFPR}, 2007b.

\item \versal{ARNEBORG}, Jette. The Norse Settlements in Greenland. In: \versal{BRINK}, Stefan \&
\versal{PRICE}, Neil (eds.). \emph{The Viking world}. London: Routledge, 2012,
pp. 588-597.

\item \versal{FERNANDES}, José Lucas Cordeiro;~\versal{CARDOSO}, Gleudson Passos;~\versal{SANTOS}, André
Luiz Campelo dos. A descoberta do horizonte: a cristianização dos
Vikings na América. \emph{Revista Brasileira de História das Religiões,}
vol. 8, 2015, pp. 109-124.

\item \versal{GWYN}, Jones.~\emph{La saga del Atlántico Norte: establecimiento de los
vikingos en Islandia, Groenlandia y América}. Barcelona: Oikos-Tau, \versal{S.A.}
Ediciones, 1992.

\item \versal{RAFNSSON}, Sveinbjörn. The Atlantic Islands. In: \versal{SAWYER}, Peter (ed.).
\emph{The Oxford Illustrated History of the Vikings}. Oxford: Oxford
University Press, 2001, pp. 110-133.

\item \versal{SHAFER}, John Douglas. \emph{Saga accounts of norse far-travellers}.
Durham: Durham University, 2010.

\item \versal{THORGILSSON}, Ari; \versal{ANÔNIMO}. \emph{Íslendingabók, Kristni Saga: The book
of the icelanders, the story of the conversion}. Tradução de Sion Gronlie.
Viking Society for Northern Research: University College of London,
2006.

\item \versal{UMBRICH}, Andrew. \emph{Early Religious Practice in Norse Greenland:
From the Period of Settlement to the 12th Century}. Reykjavík:
Universidade da Islândia, 2012.
\end{itemize}

\section{\versal{ESCANDINÁVIA}}

\emph{Conceito geral}: Escandinávia é um termo que designa uma região do norte europeu definida pela geografia, cujos contornos foram também elaborados por
referenciais históricos e linguísticos. Alguns geógrafos a definem como
a península montanhosa situada entre a Noruega e a Suécia, enquanto outros
a conceituam baseando-se nos antigos reinos da Suécia, Noruega e Dinamarca.
Devido ao alcance da colonização destes três países pela Europa
Setentrional, o conceito de Escandinávia por vezes se confunde com o de
povos nórdicos ou Norte, fazendo com que a Islândia, Ilhas Faroé e
Finlândia sejam integradas à região.

\emph{Origem do termo}: A citação mais antiga ocorreu em \emph{História
Natural} (79 d.C.) de Plínio, o velho. Nessa obra, o autor credita o termo ao que seria
uma grande ilha do Báltico, \emph{Scatinavia}, ideia seguida por
vários autores da época. É consenso entre a grande maioria dos pesquisadores
que o termo que surge nas fontes latinas proveio do germânico
∗\emph{skadan} (perigo), significando originalmente perigo; e
*\emph{awjō}, terra das águas. Isso seria reflexo do perigo enfrentado
pelos navegadores nas águas turbulentas do local. Plínio também se
refere a um grupo de ilha chamadas de \emph{Scandiae.} Posteriormente,
os escritores latinos passam a empregar o termo tanto para identificar a
região de Skåne (sul da Suécia) quanto a Escandinávia como um todo. Para
Ptolomeu (90-168 d.C.), \emph{Scandiae} se referia a um conjunto de
ilhas.

\emph{Scandia e a deusa Skadi}: Alguns pesquisadores, como Régis Boyer,
consideram que o termo Escandinávia proveio do nome da deusa Skadi (ilha
de Skadi: ∗\emph{Skathin-auja}, \emph{Scandzia insula})\emph{.} Para o
mitólogo John Mckinnel, Skadi teria sido a personificação da terra
nórdica e também representaria uma figura do submundo, como no sentido
apresentado em outros termos germânicos semelhantes: o gótico
\emph{Skadus} e o anglo saxão \emph{Scadu}, significando sombra. Na
\emph{Crônica de Frédégaire} (c. 642 d.C.), escrita em burgúndio, a
região é chamada de Scathanavia, e no \emph{Vita Sancti Sigismundi} (c.
720 d.C.) é denominada de Scarthoari.

Segundo Peter e Birgit Sawyer, o termo Escandinávia não foi muito citado
durante o período medieval, sendo preterida a expressão
\emph{Septentrionale}, em referência à extrema visibilidade da
constelação da Ursa Maior em altas latitudes (chamada pelos romanos
antigos de \emph{Septem triones}). Em seu prefácio da \emph{Gesta
Danorum} (século~\versal{XIII} d.C.), Saxo Grammaticus não utiliza o termo
Escandinávia, caracterizando o clima e a geografia da Noruega, Suécia e
Dinamarca como sendo o mesmo. Segundo ele, o céu nórdico (especialmente
a visibilidade das constelações do Boieiro e da Ursa maior) das altas
latitudes e quase tocando a zona ártica seria uma de suas maiores
características físicas.

\emph{Escandinávia nos tempos modernos}: Para Zenon Ciesielski, o
significado do termo Escandinávia depende do critério adotado, seja
geográfico, histórico, linguístico, étnico, político ou administrativo.
A diversidade escandinava é baseada na diferença entre grupos de cada
região, classificados como cultura nova ou antiga, estritamente
relacionadas à formação das identidades nacionais durante o século \versal{XIX}.
Grupos antes não classificados como escandinavos, como os lapões (sámi),
groelandeses e finlandeses agora já são reconhecidos como tal.

Segundo Bo Strath, o termo Escandinávia foi um dos muitos
utilizados pelas ideologias políticas de unificação na Europa durante o Oitocentos.
Em oposição aos movimentos de unificação alemão e italiano, o caso
nórdico foi uma \emph{nazione mancata}, uma imagem de nação que nunca se
concretizou. Após a década de 1830, os intelectuais nórdicos
frequentemente utilizaram imagens de um passado glorioso, como símbolos
góticos, os vikings e a idealização do fazendeiro medieval
(\emph{odalbonde}). Os movimentos escandinavistas emergiram na
literatura, na arte, na cultura e na academia. Alguns educadores desse
momento sugeriram que as universidades deveriam se pautar nos valores
nórdicos antigos, em vez do dominante referencial latino.

Para a pesquisadora Marja Jalava, o termo Escandinávia foi usado
frequentemente como sinônimo para Norte na língua inglesa. É uma região
historicamente definida, produzindo e reproduzindo durante séculos
diversos aspectos cotidianos da vida social, cultural, econômica e
política da Alta Idade Média, passando pela União de Kalmar até a
unificação da Dinamarca e Suécia. Em tempos mais recentes, o luteranismo
se apresentou como uma estrutura histórica e institucional comum aos
países nórdicos, além do Estado centralizado e o ruralismo. A partir dos
anos 1830 foi desenvolvido o mito romântico do panescandinavismo, em
torno da ideia de que o Norte (\emph{Norden}) constitui uma nação
(\emph{Volk}) baseada em um patrimônio linguístico-cultural e histórico
em comum. Desse modo, a Era Viking e a União de Kalmar tornaram-se
elementos centrais para a ideia de uma identidade nórdica antiga. Após o
fracasso do panescandinavismo em 1864, ele foi substituído por um
escandinavismo prático ou ``nordismo''. A imagem de um passado em comum
foi reconstruída, sendo agora a ideia de Norte manifestada como um
elemento natural das nações-estado, anacronicamente projetada sobre
períodos históricos antigos.

Ainda segundo Marja Jalava, existiu uma tensão entre os conceitos de
Escandinávia e Norte. Enquanto o primeiro significava uma unificação
nórdica sem a presença da Finlândia, o segundo a incluía. Em islandês,
Escandinávia é utilizado somente para a península escandinava (ou seja,
Noruega e Suécia), enquanto muitos acadêmicos distinguem Norte Atlântico
(Dinamarca, Islândia e Noruega) do Norte Báltico (Finlândia e Suécia).
No final do século \versal{XIX} surgiu um novo tipo de escandinavismo, atrelado a
ideias evolucionistas, raciais, filológicas e antropológicas, que
misturava o pangermanismo com darwinismo social, gerando a ideia da raça
nórdica como superior às demais. Em 1929 foi fundado na Suécia o
periódico histórico \emph{Scandia}, fundindo história nacionalista e
ciência moderna, mas defendendo a perspectiva de um patrimônio
escandinavo-nórdico em comum. Em uma perspectiva mais moderna,
foi fundado, em 1976, o periódico \emph{Scandinavian Journal of History}.

\SIG{Johnni Langer}

Ver também Dinamarca da Era Viking; Era Viking; Finlândia da Era Viking;
Islândia da Era Viking; Noruega da Era Viking; Suécia da Era Viking;
Viking.

\begin{itemize}
\item \versal{BOYER}, Régis. Skadi. \emph{La grande déesse du Nord}. Paris: Berg
International, 1995, pp. 184-204.

\item \versal{CIESIELSKI}, Zenon. The culture of Scandinavia. \emph{Folia Scandinavica}
n. 4, 1997, pp. 167-175.

\item \versal{HELLE}, Knut. (org.). \emph{The Cambridge History of Scandinavia}.
Cambridge: Cambridge University Press, 2008, vol. 1.

\item \versal{JALAVA}, Marja. The Nordic Countries as a Historical and
Historiographical Region: Towards a Critical Writing of Translocal
History. \emph{História da Historiografia}, n. 11, 2013, pp. 244-264.

\item \versal{STRATH}, Bo. The idea of a Scandinavian Nation. \emph{Opiskelijakirjaston
verkkojulkaisu}, University of Helsinki, 2007, pp. 208-223.
\end{itemize}

\section{\versal{ESPADA}}

\emph{Aspectos gerais}: Os nórdicos medievais estão estreitamente
vinculados ao uso do machado e da espada, mas sem sombra de dúvida a
espada era a marca do guerreiro na Era Viking. As espadas eram feitas de
aço, sendo simples e funcionais, mas difíceis de serem feitas. Na
\emph{Fóstbrœðra saga} 3, menciona-se que poucos homens eram armados com
espadas na Islândia no período das sagas. Nos sepultamentos islandeses
escavados, somente 16 espadas foram recuperadas. A espada era o item
mais valioso que um homem possuía. A única referência de preço na
literatura é mencionada na \emph{Laxdæla saga} 13, onde o rei Hákon
Haraldsson presenteia Hoskuldr Kollson com uma espada no valor de
dezesseis vacas leiteiras, um preço extremamente alto para os padrões
medievais. Apesar de serem mais pesadas do que as lanças, eram muito
mais duráveis.

Seu porte ia além de suas vantagens técnicas e militares: era símbolo
de prestígio e poder. Quanto mais elevado o estatuto social do
indivíduo, melhor e mais cara era a sua espada. Por vezes o cabo era
ricamente adornado e a lâmina da espada deveria ser bem
resistente. As lâminas eram importadas da região dos francos, mas os
cabos geralmente tinham origem local, sendo adornados de acordo com os
estilos artísticos vigentes em sua época e região. O modelo
franco-carolíngio adotado nas lâminas nórdicas foi derivado da espada
merovíngia. Algumas espadas produzidas pelos ferreiros nórdicos também
acabaram sendo exportadas. Espadas eram roubadas e adquiridas muitas
vezes através de inimigos mortos, juntamente com escudos, elmos, cotas
de malha e outros equipamentos. Algumas espadas eram passadas de geração
a geração, herdadas como bens de família, a exemplo de Grettir. Perder
uma espada era uma catástrofe.

\emph{Morfologia e detalhes técnicos}: Em 1919, foi criada a tipologia
básica para o estudo das espadas nórdicas, realizada em Oslo por Jan
Peterson e posteriormente aperfeiçoada por Mortimer Wheeler em 1927. A
tipologia das espadas segue o padrão dos diferentes tipos de punhos,
sendo nove no total (desde o padrão \versal{I}, do século~\versal{IX}, até o 
padrão \versal{IX}, do 
século~\versal{XIII}). Uma espada nórdica era composta das seguintes partes:
empunhadura (composto pelo pomo, cabo ou punho e guarda-mão) e lâmina
(sulco central, os fios ou gumes laterais e a ponta). O cabo era curto
para encaixar somente um único punho. Eram feitos desde madeira até
materiais mais elaborados e outros cobertos com metais preciosos.
Existiam muitas variedades, principalmente no formato do pomo, guarda
mão e cabo do punho. O peso das espadas girava em torno de dois quilos e
sua lâmina possuía cerca de 90~cm de comprimento.

Era normalmente transportada em uma bainha de couro forrada com pele de
carneiro ou em cintos especiais, dos quais praticamente não sobreviveu
nenhum exemplar. Segundo William Short, não existem evidências
históricas de espadas vikings que portaram gemas. Não existem detalhes
de como os nórdicos mantinham suas armas. A rotina de amolar armas e
equipamentos era diária, mas geralmente o próprio portador da espada
a afiava, embora em alguns casos o trabalho ia para outra pessoa. Existem
indícios de afiadores profissionais de espadas (\emph{Brennu-Njáls saga}
44). Evidências arqueológicas reforçam a ideia de que algumas espadas
tiveram uso contínuo -- um exemplar do século \versal{XI} possuía uma lâmina que
havia sido confeccionada durante o período das migrações germânicas
(século~\versal{III} ao \versal{VII} d.C.), muito antes da Era Viking. O equipamento era
utilizado por séculos e continuava a receber manutenções e cuidados.

\emph{Técnicas de combate}: Matar um inimigo com espada era considerado
tão nobre quanto matar à distância com lança. As espadas eram usadas
para cortar e empurrar armas, mesmo com lâminas cegas. Elas eram mantidas
com as bordas afiadas, muito usadas para cortar, bater e se chocar
contra escudos de madeira, armaduras, armamentos e pernas. Nestas
circunstâncias, a literatura nórdica medieval contém diversos casos de
espadas que foram impulsionadas com tanta força que ficaram presas em
madeiras, ossos, ferro ou outro material, momentaneamente deixando o seu
possuidor desarmado e morrendo nas mãos dos inimigos. Elas eram
devastadoras em poder de corte e penetração. Thorstein Midlang lutou
contra Bue do grupo de elite Jomsvikings, atravessando o seu nariz até a
metade do elmo. Escudos e armaduras eram frequentemente descritos como
sendo cortadas como gelo pelo poder da espada.

As espadas possuíam duas faces de corte, podendo ser usadas por
ambas as mãos, sem necessidade de verificar qual lado tinha fio, mas
normalmente eram empunhadas com a mão direita, enquanto a esquerda portava um
escudo. Os manuais de combate germânicos denominam os dois lados da
espada de gume longo (para a aresta que fica situada do lado frontal a
partir do momento que está empunhada pela mão do combatente) e gume
curto (para a aresta que fica atrás da espada empunhada). Cortes com o
gume longo são mais poderosos, mas existiam técnicas para utilizar
também o gume curto. A literatura preservou alguns casos em que eram
utilizados métodos de combate com espadas em ambas as mãos, como na
\emph{Þorskfird̄inga saga} 10. Existem referências de espadas quebradas
durante as batalhas (\emph{Gísla saga} 1), fato também confirmado pela
arqueologia.

\emph{Metalurgia}: O ferreiro muitas vezes padronizava as lâminas para
que tivessem uma resistência maior. Isso era realizado mesclando-se várias tiras de
metal conjuntamente, torcendo o metal, martelando-o por fora e
deixando-o liso. Ao adicionar carbono, quando ele ainda se encontra
quente e vermelho, ele produz bordas de aço afiadas. As espadas eram
elaboradas seguindo o método metalúrgico conhecido como padrão de
soldagem, no qual diferentes concentrações de ferro, aço e derivados formam
um desenho retorcido para as espadas, deixando-as com uma lâmina
flexível, leve e resistente. Os dois nomes mais comuns marcados em
espadas nórdicas foram Ingelri e Ulfberth, mas também sobreviveram
outros nomes de ferreiros ou fabricantes: Nisomefecit, Banto, Atalbald,
Leutfrit, Benno, Erolt, Inno, Gecelin.

\emph{Espadas Ulfberth}: Trata-se de um padrão de espadas produzidas na Europa entre
os séculos~\versal{IX} e \versal{XII} d.C., do qual se conhecem 170 exemplares,
encontrados principalmente na Noruega e Suécia. As suas lâminas
receberam a inscrição +Vlfberth+, um nome tipicamente franco. Não se
sabe a origem desta denominação, que pode ter sido uma marca de ferreiro,
centro de produção ou oficina, o nome de um monge ou mosteiro francês.
Elas caracterizam uma transição entre o padrão geral das espadas nórdicas da Era
Viking e o início das espadas da cavalaria medieval, sendo muitas
lâminas do padrão Oakeshott tipo 10 (tipologia de Ewart Oakeshott).
Segundo Frederik Jungqvist, o aço produzido para as espadas
Ulfberth possivelmente era proveniente da Ásia central, com a qual os
nórdicos tinham um intenso comércio redistribuído via região eslava e
Báltico. O aço era trocado especialmente por peles do norte europeu.

Pesquisas e exames recentes de laboratório concluíram que a qualidade do
aço destas espadas era muito superior à do aço medieval, sendo próximos do
moderno e possuindo poucas escórias e impurezas. O conteúdo do carbono é três
vezes superior ao da média das espadas medievais. Testes também
demonstraram que o método metalúrgico empregado para obter a qualidade
do aço foi o cadinho ou crisol. Apesar da inscrição franca, existe a
possibilidade de que os próprios nórdicos tenham fabricado a espada. Mas,
apesar disso, William Short destaca que não existem referências na
literatura nórdica acerca da fabricação de espadas. O uso do nome inscrito na
lâmina é explicado pela finalidade mágica: além e confirmar a qualidade,
daria mais poder ao seu possuidor. Em muitos enterros, suas lâminas eram
entortadas com o intuito de quebrar seu caráter sobrenatural e o guerreiro morto
não retornar do além. Também existem indícios, segundo Alan Williams, de
cópias de qualidade inferior do padrão Ulfberth feitas por pessoas analfabetas
que tentaram imitar a espada.

\emph{Simbolismos}: A espada estava associada intimamente com os laços
familiares de um homem, a lealdade ao seu senhor, o excitamento da
batalha, a realização da masculinidade e aos últimos ritos funerários.
Recebiam nomes próprios, como \emph{Fótbírt} (Mordedora de Pernas,
\emph{Laxdæla saga} 30). Na poesia escáldica as espadas foram alcunhadas
com inúmeros evocativos poéticos (\emph{kenningar}): Cobra da Batalha;
Víbora; Serpente de Sangue; Gelo da Batalha; Cachorro do Elmo; Fogo do
Rei do Mar etc.

A literatura nórdica preservou algumas indicações de runas
gravadas em espadas. A \emph{Völsunga saga} 21 descreve que runas de
vitória seriam talhadas nas lâminas, assim como a runa do deus Tyr (\textarc{t}).
Não se conhece nenhum exemplar de espada germânica ou nórdica que tenha
recebido esse tipo de inscrição rúnica. O único exemplar de espada da
Era Viking que possui algum tipo de runa é Sæbø, Noruega, descoberta em
1821, com o texto: 
oh卍muþ. 
A suástica central foi reconstituída por
George Stephens em ilustração de 1867, mas a lâmina foi deteriorada e é
difícil de ser confirmada atualmente. A interpretação clássica dos
pesquisadores é de que essa suástica tinha uma função de proteção para o
possuidor da espada e no mundo nórdico era associada a Thor e a
Odin. A espada do rei Horik, da série de televisão \emph{Vikings}, é fantasiosa,
apresentando runas do antigo período germânico (Elder futhark), sem
relação com o padrão de escrita da Era Viking.

Algumas espadas apresentam imagens de animais, como a \versal{JPO} 2242, datada
do século \versal{XI}. As bestas possuem formas de cachorros e monstros com
cabeças e detalhes que lembram peixes. Outras espadas, por sua vez,
possuíam representações de animais no pomo e guarda-mão, como na espada
finlandesa de Suontaka (tipo \versal{AE}, 1100 d.C.), apresentando serpentes
entrelaçadas entre si. Em 1998 foram descobertos em Birka, Suécia, uma
série de depósitos militares, entre os quais foram encontradas ponteiras de
bainhas feitas de bronze. Elas apresentavam ornamentações no estilo
Borre, também contendo símbolos entrelaçados de triquetras, em meio a
figurações antropomórficas e animais. Elas foram datadas na metade do
século~\versal{X} e estariam conectadas à então crescente burocracia e aparato
militar em torno das lideranças políticas da região. Simbolizariam a
lealdade e aliança, tendo a arte um uso de identidade e de
propaganda.

Em 2011, durante as escavações em uma sepultura norueguesa em Langeid,
foi descoberta uma espada incomum, datada do ano 1030 d.C. A lâmina
estava muito enferrujada, mas o punho foi bem conservado, apresentando
fios de prata e detalhes em ouro, com o pomo feito com ligas de cobre.
Ela ainda apresenta decorações em espirais, letras em latim e ornamentos
em forma de cruz. No topo do pomo, há a representação de uma mão
segurando uma cruz, algo totalmente incomum em espadas da Era Viking. A
arqueóloga responsável pelas escavações, Camilla Cecilie Wenn, acredita
que o encontro de uma espada cristã em uma sepultura pagã indica um
tesouro trazido para a Noruega por um homem muito poderoso. Pesquisas
mais recentes indicam que essa espada pode ter sido usada por algum
membro da guarda pessoal ou exército do rei Canuto em batalhas contra o
rei Etelredo da Inglaterra.

\emph{Iconografia}: A espada como motivo icônico e idealizado como
principal arma de combate surge na Escandinávia durante a Era Viking.
Anteriormente, no período das migrações, percebemos em várias
representações de bracteatas uma vasta iconográfica referente ao uso da
lança, arma tradicionalmente associada ao deus Wotan/Odin, e nesses
objetos quase sempre estava relacionada junto a suásticas, cavalos e
pássaros. No Período Vendel a lança ainda é majoritária, especialmente
em relevos de elmos, junto a guerreiros e evocações ao culto de Odin. A
partir do século \versal{VIII} a espada torna-se o elemento principal de diversas
cenas de caráter mítico, social e histórico na Escandinávia, aparecendo
em representações de tapeçaria, esculturas, monumentos cruzes e portas
de igrejas. Nas Ilhas Britânicas e na Suécia ela se transfigura na espada
Gram do herói Sigurd, surgindo especialmente no momento da morte do
dragão Fafnir -- cena que é o elemento central de diversas
pedras rúnicas, como Ramsund (Sö 101), Gök (Sö 327), \versal{U} 1163 e \versal{U} 1175.

Na ilha de Gotland, Suécia, percebemos claramente a alta idealização
desse instrumento marcial como elemento de identidade da classe
aristocrática e guerreira, sempre junto às representações de cultos
fúnebres ou à ideia de renascimento junto ao Valhalla. Na pedra pintada
de Stora Hammars~\versal{I} (datada de 700 a 800 d.C.), cinco das seis cenas
possuem conexão direta com espadas. Das 26 pessoas
representadas, 19 estão armadas com esta arma. A segunda cena
(contando de baixo para cima) do monumento alude a uma batalha, onde
talvez o morto homenageado pela pedra tenha sofrido seu ferimento
mortal, situado abaixo de um cavalo e de uma águia. Logo a seguir temos
a representação de Hildr e os exércitos inimigos e logo acima uma cena
de sacrifício odínico, onde quatro guerreiros brandem suas espadas acima
de suas cabeças. No alto, duas espadas estão cravadas no chão, entre um
cavalo e dois homens. Mais acima, um ser masculino sentado (talvez um rei,
talvez Odin) está no meio de dois guerreiros com suas espadas na mão e em
posição ofensiva. Todo o conjunto denota ligação com o culto a Odin,
com exceção da lança, que aparece uma única vez em todo o conjunto (na
cena de sacrifício), a espada é a arma dominante, simbolizando o poder
da elite guerreira.

Em outra pedra pintada gotlandesa (Lärbro Tängelgårda~\versal{I}, 700 a 800
d.C.), a identidade dessa elite com os cultos religiosos e a ideologia
política é bem mais acirrada. Na cena superior, um homem paira sobre o
ar, segurando um machado, e, logo acima, uma valquíria porta um corno de
hidromel. Abaixo de imensas águias, um homem foi representado sob um
cavalo de quatro patas; logo ao lado, três guerreiros em posição de
ataque. Possivelmente trata-se do homenageado pelo monumento no momento
em que este falece em uma batalha. Ao lado dessa cena, duas pessoas
seguram espadas com as lâminas para cima, uma defronte da outra, o que
parece denotar algum juramento ou comunhão. Na segunda cena (ou central
do monumento), três homens caminham segurando as lâminas das espadas
para baixo, num sentido de derrota ou morte do falecido, enquanto um
cavalo com oito patas está ao lado direito (Sleipnir, o corcel de Odin).
Na terceira cena, quatro guerreiros seguram braceletes para o alto,
enquanto suas espadas estão nas bainhas. Um cavaleiro segue mais adiante
(o falecido ou o deus caolho?), portando um escudo com espirais,
enquanto seu cavalo possui três \emph{valknuts} representados entre suas
pernas. Nesse contexto, a espada está relacionada com a marcialidade,
com os cultos fúnebres, com os juramentos entre irmãos de guerra e os
cultos ao deus Odin. Ela reforça o alto status social do falecido e de
sua integração com a aristocracia guerreira.

\SIG{Johnni Langer}

Ver também Armamento; Arquearia; Duelos; Guerra e técnicas de combate.

\begin{itemize}
\item \versal{ANDROSHCHUK}, Fedir. \emph{Viking swords: swords and social aspects of
weaponry in Viking Age society}. Stockholm: Historiska Museet, 2014.

\item \versal{GRANCSEY}, Stephen. A Viking chieftain´s sword. \emph{The Metropolitan
Museum of Art Bulletin}, march 1959, pp. 173-182.

\item \versal{GRIFFITH}, Paddy. \emph{The viking art of war}. London: Greenhill Books,
1995.

\item \versal{HEDENSTIERNA-JONSON}, Charlotte. A group of Viking Age sword chapes
reflecting the political geography of the time. \emph{Journal of Nordic
Archaeological Science}, vol. 13, 2002, pp. 103-112.

\item \versal{LANGER}, Johnni. Espadas míticas. In: \versal{LANGER}, Johnni (org.).
\emph{Dicionário de Mitologia Nórdica}. São Paulo: Hedra, 2015, pp.
169-172.

\item \versal{LINDHOLM}, David \& \versal{NICOLLE}, David. \emph{Medieval Scandinavian armies}.
Oxford: Osprey, 2003.

\item \versal{MACN'AH}, Chris. \emph{Swords: a visual history}. London: Dorling
Kindersley, 2010.

\item \versal{PEIRCE}, Ian. \emph{Swords of the Viking Age}. London: Boydell Press,
2002.

\item \versal{SHORT}, William. \emph{Viking weapons and combat techniques}.
Pennsylvania: Westholme, 2009.

\item \versal{SPRAGUE}, Martina. \emph{Norse warfare}. New York: Hippocrene, 2007.

\item \versal{YOST}, Peter (dir.). \emph{Secrets of the Viking sword}. New York:
Nova/National Geographic Television, 2012, 54 m.
\end{itemize}

\section{\versal{ESQUIMÓS (INUÍTES) E NÓRDICOS}}

Um dos grandes acontecimentos das Sagas do Atlântico Norte é caracterizado pelo
encontros de exploradores nórdicos na América do Norte com as populações
autóctones -- que aqui se generalizará como esquimós. As fontes literárias e escritas apresentam dois momentos-chave sobre o tema: um primeiro momento de troca comercial e paz e um segundo, de conflito físico e violência,
que não necessariamente ocorrem nesta ordem.

Antes de considerar esses momentos, é preciso tratar do termo
\emph{skrælingi} (p. \emph{skrælingjar}), usado para se
referir aos esquimós dentro do âmbito cultural nórdico e das fontes.
A tradução para esse termo é difícil, pois
representam-se nele sentidos mais amplos e de identificação étnica. Há traduções para ``feio'', ``nativo'' ou ``autóctone'', as quais não são bem aceitas academicamente. A última, por exemplo, negligencia que o sentido dessa palavra não almeja essa representação de povo local. Na \emph{Saga de Eiríkr, o Vermelho}, tem-se
também o uso da expressão \emph{smáir} e \emph{svartir} (dependendo da
edição e preservação documental), que também foram usadas para esses
esquimós, significando ``pequeno'' e ``de cabelo negros'',
respectivamente. Sobre a etimologia dessa palavra, pode-se traçar do
norueguês \emph{skræla}, que significa gritar, gritar muito alto, ou do
islandês \emph{skrælna}, que pode ser lido como murchar ou secar. De
fato, é um termo derrogatório, que no norueguês moderno --
\emph{skræling} -- quer dizer fraco ou pessoa miserável, enquanto
\emph{skrælingi}, do islandês moderno, quer dizer bárbaro.

O primeiro encontro narrado, pensando em uma sequência de
fatos na dinâmica das duas narrativas que compõem as \emph{Sagas do Atlântico
Norte}, é a viagem de Thorvaldr, realizada logo após o retorno de
Leifr Eiríksson da Terra das Vinhas. Este viaja com cerca de trinta
homens para a América do Norte, passando um bom tempo viajando pela
terra e admirando sua beleza, até que encontram três canoas de couro com
nove homens, capturando oito deles e posteriormente matando"-os. Após
isso, eles têm um encontro agressivo: ``Então surgiram de dentro do
fiorde incontáveis canoas de couro, e elas se lançaram na direção deles
[...] os \emph{skrælingjar} atiraram neles por um tempo e depois
foram embora [...]'' (Anônimo, 2007a, p. 70-71). Esse
encontro acaba por resultar na morte de Thorvaldr, que é sepultado na
nova terra, Krossanes (Cabo das Cruzes).

O segundo encontro, por sua vez narrado de forma bem mais ampla e densa pelas
sagas, é o da expedição de Thorfínnr Karlsefni. Ele já tinha noção do
encontro ocorrido com os esquimós por Thorvaldr, e o seu contato inicial
fora bem amistoso e comercial: ``E, quando começou a primavera, eles
puderam ver cedo em uma manhã um grande número de canoas de couro remando
ao sul do cabo, tantas que pareciam ter"-se a enseada salpicada de carvão
[...] fizeram comércio entre si, e o que aquela gente mais queria
obter era tecido vermelho'' (Anônimo, 2007b, p. 115). Essa
sistematização de escambo e contato ocorre nos dois primeiros encontros,
e a mudança para o ponto de violência é destoante entre as \emph{Sagas do
Descobrimento da América}: uma atribui a uma movimentação de um touro
que assusta os esquimós, outra atribui a um esquimó que é morto ao tentar
apanhar armas de um dos homens de Karlsefni (cena repleta de uma tensão
pela composição narrativa).

As duas sagas, apesar de possuírem diferenças claras, trazem um relato
similar sobre o grande ataque dos esquimós que se segue, falando de uma
quantidade massiva deles, de um grande número de lanças atiradas e de
combates com mortes em ambos os lados, principalmente do lado dos
esquimós. Vale nota que uma das mais belas cenas das sagas islandesas
ocorre na narrativa da \emph{Saga de Eiríkr, o Vermelho} sobre esse conflito:
vendo o pavor que se abate sobre Karlsefni, devido ao poderoso ataque
dos esquimós, Freydís Eiríksdóttir, mesmo grávida, pega uma espada de um
nórdico morto -- Thorbrandr Snorrasson --, põe um seio para fora do
casaco e bate com a espada nos esquimós que lhe alcançaram,
assustando-os e lhe rechaçando para longe.

Há ainda, na saga supracitada, um relato do
nascimento do filho de Karlsefni -- Snorri -- e do encontro com meninos
esquimós que lhes revelam, após serem ensinados a falar e terem sidos
levados com o bando e batizados, que seus pais se chamavam
Vethildi e Óvægi e que a terra dos \emph{skrælingjar} era
governada por reis, um chamado Avaldamon e outro
Valdidida. Após esses encontros, em que a imensidão dos
\emph{skrælingjar} demonstra sua força, Karlsefni resolve abandonar
aquela terra, sendo expulso pela força desses habitantes locais que
tornavam a possibilidade de assentamento naquela terra um grande risco e
de grande dispêndio. Há ainda outros relatos de combates e mortes,
assim como tentativas de expedição, mas pode-se definir a
relação dos nórdicos medievais com os esquimós em dois períodos: uma fase comercial e
uma fase de confronto que acaba por inviabilizar seu assentamento na
América do Norte.

\SIG{José Lucas Cordeiro Fernandes}

Ver também Brattahlid; Groenlândia nórdica; Sagas do Atlântico Norte.

\begin{itemize}
\item \versal{ANÔNIMO}. A Saga do Groenlandeses. In: \emph{As três sagas Islandesas.}
Tradução de Théo Moosburger. Curitiba: Editora \versal{UFPR}, 2007a.

\item \versal{ANÔNIMO}. A Saga de Eiríkr Vermelho. In: \emph{As três sagas Islandesas.}
Tradução de Théo Moosburger. Curitiba: Editora \versal{UFPR}, 2007b.

\item \versal{ARNEBORG}, Jette. The Norse Settlements in Greenland. In: \versal{BRINK}, Stefan;
\versal{PRICE}, Neil (eds.). \emph{The Viking world.} London: Routledge, 2012,
pp. 588-597.

\item \versal{Barnes}, Geraldine. \emph{Viking America: The First Millennium}.
Cambridge: D.S. Brewer, 2001.

\item \versal{Bergersen}, Robert. \emph{Vinland Bibliography: Writings Relating to the
Norse in Greenland and America}. Tromsø: University of Tromsø, 1997.

\item \versal{GWYN}, Jones.~\emph{La saga del Atlántico Norte: establecimiento de los
vikingos en Islandia, Groenlandia y América}. Barcelona: Oikos-Tau, \versal{S.A.}
Ediciones, 1992.

\item \versal{INGSTAD}, Helge; Ingstad, Anne Stine. \emph{The Discovery of a Norse
Settlement in America: Excavations of Norse Settlement in L'Anse aux
Meadows, Newfoundland}. New York: Checkmark Books, 2001.

\item \versal{RAFNSSON}, Sveinbjörn. The Atlantic Islands. In: \versal{SAWYER}, Peter (ed.).
\emph{The Oxford Illustrated History of the Vikings.} Oxford: Oxford
University Press, 2001, pp. 110-133.

\item \versal{SHAFER}, John Douglas. \emph{Saga accounts of norse far-travellers}.
Durham: Durham University, 2010.
\end{itemize}

\section{\versal{ESTUPRO}}

O estupro é uma das mais populares imagens associadas aos vikings, seja
na arte, mídia ou ficção contemporânea. Mesmo na academia ainda vigora
essa representação equivocada, como em determinadas alegações de que os vikings
possuíam uma ``cultura de estupro''. Na realidade, não existem evidências
históricas dessa interpretação. Segundo John Haywood, as documentações
evidenciam pilhagens, ataques, assassinatos, queimas, extorsões e
capturas de prisioneiros, mas não contêm qualquer tipo de informação
sobre estupros. É significativo nesse contexto a referência dos
\emph{Anais de São Bertin} (século \versal{IX} d.C.), que possui duas referências
sobre cristãos cometendo estupro contra freiras, mas não há uma simples
referência ao mesmo ato em expedições vikings. Na Escandinávia, os
estupros eram severamente punidos. O cronista Adão de Bremen (século~\versal{XI})
mencionou que o estupro de virgens era punido com a morte na
Escandinávia.

\emph{O estupro nas fontes não-escandinavas}: Uma das mais conhecidas
fontes sobre os ataques e expedições predatórias nórdicas, as
\emph{Crônicas Anglo-saxãs}, não mencionam nenhum tipo de ato sexual
violento por parte dos incursionistas. Pesquisas nos anais francos
sugerem que na região francesa os nórdicos não foram conhecidos como
``estupradores notórios'', nem mesmo em ataques a mosteiros e conventos.

No relato de Ibn Fadlan sobre os nórdicos da região do Volga
(\emph{Risala}, século~\versal{X} d.C.), não foram mencionadas práticas de
estupro. A cena em que uma escrava prestes a ser sacrificada com o líder morto
faz sexo com vários homens em torno do navio sacrificial é consensual,
assim como o próprio ato de ser sacrificada foi voluntário.

O estupro existe na literatura nórdica medieval, mas ele nunca é
glorificado. Descrições de pilhagens e lutas em locais fora da
Escandinávia não incluem cenas de estupro ou rapto de mulheres
estrangeiras. Quando ocorrem situações de violência sexual masculina nas
narrativas, existem diversas consequências sociais violentas para os
protagonistas (eles são mortos ou perseguidos pela família da vítima,
por exemplo). Segundo Frederik Ljungqvist, nas sagas islandesas a
violação feminina era percebida como uma violação da integridade física
da mulher e desonra para a vítima. Por outro lado, o tema central nas
representações de estupro nas sagas é que a agressão sexual era
considerada altamente difamatória para os parentes do sexo masculino da
mulher, exigindo vingança de sangue em troca. Uma vez que as agressões
sexuais contra as mulheres podiam ser usadas para desonrar os homens,
segue-se que há exemplos nas sagas de como a violação era usada como uma
``arma'' durante embates. É óbvio, no entanto, que isso era considerado um
crime de vendeta, e a violação, na maioria dos casos, era socialmente
inaceitável e denunciada.

\emph{O viking estuprador como uma construção artística}: A historiadora
Erika Sigurdson realizou um detalhado estudo demonstrando que a visão
contemporânea dos vikings como saqueadores e estupradores foi uma
construção literária iniciada no Oitocentos, posteriormente popularizada
pelo cinema e \versal{TV} no século \versal{XX}. Especialmente a literatura vitoriana vai
elaborar imagens de vikings atrelada ao imaginário do pirata sedutor.
Também a expressão dupla ``estupro e pilhagem'' passou a ser uma imagem
canônica em língua inglesa a partir de 1817, sempre associada a
invasores de outros países e culturas em um determinado período
histórico, especialmente vindos do mar. E durante o início do século
\versal{XIX}, os escritores fundiam as imagens do pirata com as de rei do mar e
os vikings, criando assim as narrativas ficcionais de um homem perigoso, violento e imprevisível, mas também atraente, inteligente e refinado.

O romance \emph{The viking: an epic} (1849), de Zagar, contém cenas de um
viking de nome Vali, que parte para excursões na Inglaterra para ``se
vangloriar do sangue dos saxões e do estupro de virgens''. Nesse e em
outros livros e poemas, o estupro foi tratado como uma parte essencial
da masculinidade viking, atraindo não somente a atenção de leitores, mas
também de leitoras. Em 1811 o escritor sueco Erik Gustav Geijer criou o
poema \emph{Vikingen}, apresentando explicitamente o tema da abdução e
estupro de uma jovem mulher em uma incursão marítima. Apesar disso, o
herói da narrativa acaba se relacionando amorosamente com a vítima,
criando um desfecho romântico e idealizado.

Também as artes plásticas auxiliaram na popularização desse estereótipo.
Em 1841, o pintor norueguês Freferik Nicolai Jensen realizou a pintura
\emph{Viking abducting a Southern woman.} Nela, um guerreiro nórdico
portando elmo, escudo e machado captura uma jovem aristocrata,
possivelmente de alguma região do Mediterrâneo. A expressão de desespero
da mulher contrasta com o olhar frio e decidido do abdutor. Mas
certamente o mais famoso artista relacionado a essa temática foi o
francês Evariste Vital Luminais. Em 1887 ele executa a tela \emph{The
abduction}, na qual um guerreiro germânico transporta em seu cavalo uma
mulher de longos cabelos negros, mantida cativa à força. O que torna a
tela incomum é o fato de ambos estarem totalmente nus. Em outra pintura,
\emph{Pirates normands au \versal{IX} siècle} (1893), Luminais apresenta um
normando levando cativa uma mulher por uma praia, prestes a embarcar com
seu companheiros em um navio nórdico. O abdutor é loiro e está com o busto descoberto, enquanto a
cativa possui a mesma posição e cabelos da pintura \emph{The abduction}.
Já em outra tela, bem mais famosa e com maior repercussão (mas de mesmo
título), o artista francês sofisticou o tema em detalhes mais
aprimorados (\emph{Pirates normands au \versal{IX} siècle}, 1897).
Dois homens capturam uma jovem e a carregam pela praia,
portando machados e escudos. A jovem está na mesma posição das telas
anteriores, mas desta vez seu cabelo é de um loiro muito intenso, com o
corpo totalmente claro, contrastando fortemente com os captores, de
cabelos e roupas escuras. Somando ao fato dos seios da moça estarem
expostos, a tela apresenta uma impressionante confirmação do referencial
romântico sobre a imagem dos vikings como saqueadores e do poder sexual
masculino sobre o universo feminino.

O cinema tratou de popularizar ainda mais os estereótipos românticos do
Oitocentos. Uma das mais importantes e famosas produções cinematográficas
norte-americanas sobre o tema, \emph{Vikings, os conquistadores} (\emph{The
vikings}, 1958), logo em seu início, possui uma cena muito icônica: o
personagem Ragnar, após matar e saquear uma comunidade da Nortúmbria
(Inglaterra), ataca e estupra a rainha, engravidando-a de um filho bastardo. Essa cena foi imitada e parodiada em outra produção, a comédia
\emph{Erik o viking} (1989), na qual o protagonista é incentivado pelos
companheiros de saque a estuprar uma jovem anglo-saxã, que ele acaba
matando não intencionalmente.

A mídia também incentivou a propagação dessa imagem. Em 1975, em uma
propaganda britânica do xampu Super Soft com a então famosa atriz
Madeleine Smith, esta interpreta uma camponesa em uma aldeia atacada
pelos vikings. Ao narrar as vantagens de utilizar o produto para os
cabelos, ela adverte que sempre deve-se estar preparada para qualquer
ocasião, momento no qual um nórdico arromba a porta da casa e a leva sobre
seus ombros. A última cena, com imagens do xampu, contém os suspiros
da atriz. Também romances contemporâneos, tanto produzidos por
escritoras para o público feminino (como as séries da escritora Sandra
Hill) quanto de séries ficcionais com caráter erótico em geral (a
exemplo da série \emph{Valhalla Hot}) utilizam temas de estupros
vikings, bondages e humilhações sexuais.

A série \emph{Vikings} (2013, primeira temporada), do History Channel,
possui várias cenas com o tema do estupro. No primeiro capítulo, ``Ritos
de passagem'', a personagem Lagertha é ameaçada de estupro enquanto
estava sozinha em sua casa. No episódio 2, ``A ira dos homens do norte'', o
irmão de Ragnar, Rollo, violenta uma escrava enquanto se prepara para
uma incursão. No quarto episódio, ``Tentativas'', o personagem Canuto tenta
violar uma mulher anglo-saxã durante um ataque. Impedido por Lagertha,
ele tenta violar esta mas acaba sendo morto. Segundo a historiadora
Erika Ruth Sigurdson, em todos estes incidentes o estupro é utilizado
como um dispositivo historicizador -- ele sinaliza que estamos diante de
tempos brutais e misteriosos. Apesar de alguns momentos o ato ser punido
de alguma forma na série \emph{Vikings}, ele retoma o estereótipo dos
filmes anteriormente mencionados.

Violentar, matar e cometer violência contra as mulheres. Essa trilogia
tornou-se uma imagem icônica sobre a identidade viking desde o século
\versal{XIX}, fortemente implantada na imaginação popular. Para Erika Sigurdson,
não bastava o nórdico matar, saquear e profanar locais sagrados ao
aterrorizar as nações. Em vez disso, o estupro tornou-se o crime
definidor do viking. A literatura e as artes plásticas criaram o
referencial do estupro e abdução como elementos da identidade viking,
mas também introduziram o tema do pirata sedutor e ameaçador. O corpo
das mulheres foi apresentado como um objeto do botim e a sexualidade
viking é interpretada como poderosa, dominante e fundamentada na
violência e sendo interpretada tanto como atrativa ou como aterrorizante
para o público consumidor.

\SIG{Johnni Langer}

Ver também Mulheres; Sociedade; Sexo e sexualidade.

\begin{itemize}
\item \versal{FAULKES}, Anthony. \emph{The viking mind (víkingahugr) or in pursuit of
the viking}. Viking Society Web Publications, 1998.

\item \versal{HAYWOOD}, John. Rape. In: \emph{Encyclopaedia of the Viking Age}. London:
Thames and Hudson, 2000, pp. 154-155.

\item \versal{LJUNGQVIST}, Frederik Charpentier. Rape in the Icelandic Sagas: An
Insight in the Perceptions about Sexual Assaults on Women in the Old
Norse World. \emph{Journal of Family History}, vol. 40, n. 4, 2015, pp. 431-447.

\item \versal{MCKENNA}, Alexandra. Norsemen and vikings: the culture that inspired
decades of fear. \emph{\versal{WEI} International Academic Conference}, 2014, pp.
18-27.

\item \versal{PISTONO}, Stephen. Rape in the medieval Europe. \emph{Atlantis}, vol. 14, n. 2,
1989, pp. 36-43.

\item \versal{SIGURDSON}, Erika Ruth. Violence and Historical Authenticity: Rape (and
Pillage) in Popular Viking Fiction. \emph{Scandinavian Studies}, vol. 86, n. 3,
2014, pp. 249-267.
\end{itemize}

\section{\versal{EXPANSÃO NÓRDICA}}

Entre os povos da Europa medieval, provavelmente os povos da Escandinávia
foram os que mais empreenderam longas viagens no continente e para
além deste. Viajando por terra, mar e rios, os nórdicos atravessaram
grandes extensões da Europa, desbravaram o mar do Norte, o arquipélago
britânico, chegaram à América do Norte, passaram pelo norte da África, o
Mediterrâneo e se aventuraram na Ásia. Tudo isso caracterizou um processo longo
que se estendeu por quase três séculos.

As fontes para se estudar a expansão nórdica são diversas e redigidas em
diferentes épocas e línguas, o que revela não apenas a extensão temporal
dessas viagens, mas também a variedade de lugares e povos que os
nórdicos visitaram e conheceram. Além dos relatos escritos, a
arqueologia consiste em outro meio para se estudar a história dessas
expansões, pois os vikings construíram assentamentos, cidades,
entrepostos, túmulos etc. Todavia, o que se conhece sobre a expansão
nórdica é limitado a acontecimentos que foram preservados na escrita ou
a vestígios arqueológicos, de modo que tais fontes não compreendem todas as
expedições realizadas, podendo haver muito mais do que se supõe.

Embora o início da Era Viking comumente seja situado no século \versal{VIII},
período que marca o início das viagens para outros territórios,
especialmente a Inglaterra, é provável que antes disso já houvessem
viagens regulares para o norte da Alemanha e outros territórios que
margeiam o mar Báltico. Devido à proximidade destas terras com a
Dinamarca, Noruega e Suécia, além dos achados de moedas
romanas, árabes, francas e outros objetos, tudo indicaria um comércio
produtivo na região báltica. Sendo esse comércio um dos possíveis
fatores para que grupos de nórdicos se aventurassem para o leste europeu.

O historiador James Graham-Campbell também cogita que poderia já existir
comércio entre a Escandinávia e a Inglaterra pelo menos desde o século
\versal{VII}. Ele defende tal teoria com base na cultura material referente a
objetos e túmulos, que em ambos os lugares eram bem similares. Ele aponta, inclusive, que o estilo artístico do Período Vendel (séculos~\versal{VI}-\versal{VIII}) possuía algumas semelhanças com a arte anglo-saxã da parte oriental da ilha. Tais fatos seriam indicativos do porquê os vikings decidirem
atacar a Inglaterra, pois já existiria determinado conhecimento a
respeito da ilha, e eles não teriam chegado lá por acaso.

No entanto, os motivos que levaram às expedições nórdicas partiram de
diferentes locais e se constituíram por distintos fatores. James H.
Barret comenta que as justificativas clássicas tendiam a apontar que
fatores de ordem climática, como temperaturas mais baixas, teriam levado
grupos a se deslocar da Escandinávia para outras terras a fim de escapar
do frio. Incluem-se também motivos relacionados a perseguições
políticas, guerras e insegurança. O aumento populacional teria agravado
surtos de fome, obrigando populações de certas regiões a migrarem. E por
fim, mencionam-se fatores econômicos relacionados a uma suposta
``corrida da prata'' e por metais em geral, devido a sua escassez.

Quando se passa para a história das expedições, porém, nota-se que as
justificativas clássicas nem sempre eram respostas definitivas. O início
das primeiras expedições aconteceu ainda no século \versal{VIII}. Em 750 havia um
assentamento viking em Staraia Ladoga, ao sul do lago Ladoga, na atual
Rússia. O assentamento teria servido de entreposto comercial para
negócios na região. Pois Staraia Ladoga ficava situada numa região onde
passavam rotas comerciais que ligavam o mar Báltico até a Bulgária do
Volga (Rússia), além de ser ponto de caminho para viagens ao sul. No
século \versal{IX}, especificamente em 839, é registrada a primeira menção de nórdicos em
Constantinopla (atual Istambul), capital do Império Bizantino.

O caminho para Constantinopla foi uma rota comercial bastante importante
no Leste Europeu, a ponto de os vikings estabeleceram vários contatos
com os povos eslavos, chegando a formar assentamentos, entrepostos
comerciais e até mesmo a controlar algumas cidades, como no caso de
Novgorod (Rússia) no século \versal{IX} e Kiev (Ucrânia) no século~\versal{X}. Não
obstante, a partir do rio Volga e do Mar Negro, incursões se aventuraram
cada vez mais adentro da Ásia, chegando ao território do Canato de
Cazar. Entre 911 e 912 encontram-se relatos de pirataria viking no mar
Cáspio. Data também do ano de 921 o relato do embaixador árabe Ahmad
ibn Fadlan e seu encontro com um grupo de nórdicos na Bulgária do Volga.

Entretanto, se as expedições no leste parecem ter seguido para um lado
mais comercial, as expedições no oeste, realizadas pelos noruegueses e
dinamarqueses, por muito tempo tiveram um caráter bélico. Na data de 8 de
junho de 793, como consta na \emph{Crônica anglo-saxã}, o mosteiro de São
Cuteberto, na ilha de Lindisfarne, no Reino da Nortúmbria, foi atacado
por pagãos do norte.

Tal acontecimento foi considerado um marco para a história. Entre os
anos de 794 e 799 foram relatadas novas incursões à Inglaterra, Escócia
e Irlanda, todas basicamente restritas ao intuito da pilhagem. O fato é
interessante, pois a Dinamarca, devido a sua proximidade com a Alemanha,
possuía vários importantes centros comerciais onde se negociava com os povos germânicos, além de acesso a rotas comerciais que desciam até o
Império Franco.

Tentar justificar as incursões dinamarquesas como oriundas de falta de
recursos é problemático. O mais provável é que se tratasse de
iniciativas organizadas por chefes de determinadas comunidades, os quais
buscavam riqueza e fama pessoal, lembrando que a Dinamarca, Noruega e
Suécia não eram Estados unificados, mas conjuntos de reinos e Estados
vassalos que se digladiavam pelo poder. A presença de reis envolvidos
nessas expedições data de vários anos depois.

Porém no século~\versal{IX} ocorreram grandes mudanças a respeito da forma como a
expansão nórdica se processou. O século~\versal{IX} foi o auge dessas expansões.
Nesse período encontramos reis da Noruega e Dinamarca envolvidos em
algumas expedições, como, por exemplo, o rei Godofredo da Dinamarca declarando guerra ao imperador Carlos Magno da Francia, para disputar o
controle da Frísia, importante região comercial. Anos depois,
em 845, o rei Horik~\versal{I} da Dinamarca ordenou ataque à cidade de Hamburgo
(na atual Alemanha), como data também desse período o início da cobrança
do \emph{danegeld}, tributo em prata cobrado dos povos atacados, para
evitar novas ondas de invasão.

Não obstante, a década 840-880 foi bastante intensa na Europa Ocidental.
Paris, capital do Império Franco, foi saqueada pelo menos três vezes.
Londres, Kent, Rochester e outras cidades inglesas foram atacadas
regularmente. Datam também da década de 840 as primeiras incursões à
Península Ibérica, com ataques a Lisboa, Sevilha, Cádiz e várias outras
cidades, sobretudo no ano de 844. A partir de tais expedições, os vikings entraram em contato com os muçulmanos do Ocidente, como também visitaram
brevemente o norte da África e se aventuraram pelo mar Mediterrâneo,
passando pelo sul da França e Itália.

Mas, para além dessa intensa onda de ataques, o século~\versal{IX} também foi
marcado pela colonização norueguesa e dinamarquesa de distintos
territórios, principalmente situados no Atlântico Norte. No caso inglês,
o ano de 865-866 culminou com a chegada do Grande Exército Pagão, que
conquistou os reinos saxões da Nortúmbria, Mércia e Ânglia Oriental,
constituindo o Danelaw. A ilha da Irlanda, a ilha de Man, e os
arquipélagos escoceses das Órcades, Faroe, Hébridas e posteriormente a
Islândia foram colonizados no mesmo século.

Nota-se uma mudança no comportamento das expedições ocidentais, as quais eram
inicialmente esporádicas e motivadas por atos de pirataria e
pilhagem. A partir de meados do século \versal{IX}, no entanto, tornaram-se expedições
voltadas para assegurar territórios nas terras atacadas, o que culminou
no estabelecimento sedentário de comunidades nórdicas, principalmente
nas ilhas mencionadas. Mas esse processo somente ocorreu décadas depois
das primeiras expedições ao arquipélago britânico e à Francia, o que põe
em dúvida a natureza dos fatores como excesso populacional, fome,
guerras, economia e o frio como motivos que levaram as
expedições do século \versal{VIII} a ter início.

Quando se adentra o século~\versal{X}, as expedições haviam sofrido uma longa
pausa, pois, devido à colonização e permanência na Inglaterra, Irlanda,
ilhas escocesas, Normandia e no Leste Europeu, reis e chefes optaram em
não investir em expedições militares com maior regularidade. Em termos
de novas expansões, destaca-se a descoberta da Groenlândia em 985 por
Érico, o Vermelho.

A partir da colonização do sul da Groenlândia, navegantes noruegueses
começaram a explorar os arredores, vindo avistar terras no Ocidente no
que corresponde à atual região do Canadá. Os territórios costeiros foram
nomeados pelos nomes de Helluland, Markland e Vínland. Por volta do ano
1000, um dos filhos de Érico, Leif Ericsson, fundou um povoado em Vínland,
mais exatamente na atual ilha de Newfoundland, Canadá. Consistindo na
única povoação viking conhecida na América. A fundação de uma povoação
em Vínland é considerada por alguns historiadores o último grande feito
da expansão nórdica.

\SIG{Leandro Vilar Oliveira}

Ver também Escandinávia; Era Viking; Expansão nórdica; Viking.

\begin{itemize}
\item \versal{BARRET}, James H. What caused the Viking Age? \emph{Antiquity}, n. 82,
2008, pp. 671-685.

\item \versal{GRAHAM-CAMPBELL}, James (org.). \emph{Os viking}s. Barcelona: Folio \versal{S.A.},
2006.

\item \versal{HAYWOOD}, John. \emph{Historical Atlas of Vikings}. London: The Penguin
Books, 1995.

\item \versal{LOGAN}, F. Donald. \emph{The Vikings in History}. London/New York:
Routledge, 1991.

\item \versal{SAWYER}, Peter (ed.). \emph{The Oxford Illustrated History of the
Vikings}. New York: Oxford University Press, 1997.

\item \versal{STREISSGUTH}, Thomas. \emph{Life among the Vikings}. San Diego:
Lucent Books, 1999.
\end{itemize}

\section{\versal{EYRBYGGJA SAGA}}

A \emph{Eyrbyggja saga} é uma obra anônima considerada um dos
máximos expoentes de um subgrupo das \emph{Íslendingasögur}, denominado
``Sagas de Distrito''. A saga se conserva em três redações, a primeira delas sendo \emph{Vatnshyrna}, do século~\versal{XIV}, que se perdeu em um incêndio de Copenhague em 1728. A redação~\versal{B} está contida em um manuscrito do século~\versal{XIV}, 
conservado em Wolfenbüttel (Alemanha). E ainda a este grupo~\versal{B} pertence o mais antigo manuscrito da saga, o fragmento~\versal{AM} 162 \versal{E} fol. de final do século~\versal{XIII}. A redação~\versal{C} se conservou em um manuscrito incompleto de finais do século \versal{XIV}, conhecido como \emph{Melabók}.

Como se pode deduzir do nome do subgrupo em que se geralmente enquadra a
saga, a \emph{Eyrbyggja} não versa sobre as aventuras de somente um
personagem, mas, como é dito em um de seus manuscritos, sobre os
habitantes da península de Þórsness, os de Eyr e os de Álpafjörð; sobre
a colonização das terras desta península ao oeste da Islândia e suas
lutas de poder no primeiro século do período da colonização. A saga muda
um pouco sua estrutura narrativa quando chega à terceira geração e se
introduz as figuras do \emph{goði} Snorri e a de Arnkell de Bólstað como
os centros de poder em torno dos quais se reúnem os habitantes da zona.

A apresentação das personagens da saga se caracteriza por um
realismo que não é compartilhado com os expoentes mais importantes das
\emph{Íslendingasögur,} estas mais interessadas na idealização de uma época e
seus protagonistas. Na \emph{Eyrbyggja saga} predominam as
caracterizações negativas e um interesse pouco usual pelo lado mais
negativo da sociedade ou por acontecimentos moralmente repudiados. Os
personagens que aparecem sob uma luz positiva têm finais desastrosos,
como o próprio Arnkell, que é assassinado, ou Þórarinn, o Negro e Björn, o
Campeão de Breiðavík, que são obrigados a exilar-se da Islândia. Os menos exemplares, por outro lado, alcançam melhores destinos, como o próprio \emph{goði} Snorri, que termina seus dias convertido em uma das pessoas mais ricas e consideráveis do país.

O alcance territorial da obra, conjuntamente ao interesse de seu autor
por incluir nela detalhes que não eram habituais na época, fazem com
que a narração discorra de maneira mais livre entre protagonistas e
sucessores, mudando o ponto de vista da personagem individual a um grupo
já conhecido, assim como introduzindo novos atores à trama ou incluindo novas
localizações. Graças ao interesse por oferecer uma imagem real dos
acontecimentos do período pré-cristão da Islândia, a saga geralmente é
considerada um verdadeiro tesouro para os estudiosos das crenças pagãs
dos primeiros colonizadores, assim como suas tradições no momento da
escolha dos assentamentos, como é o caso de Þórólfr, o Barbudo de Mostur,
que segue pela costa do mar com um mastro com a efígie do deus Thor, retirada do templo dedicado a este na Noruega. Ou mesmo as ideias sobre as últimas
moradas dentro da montanha sagrada de Helgafell. Também, o fato de que nem todos
os colonizadores procediam de famílias norueguesas possivelmente
permitiu que nesse relato tivesse lugar um número considerável de
superstições religiosas pertencentes à religião popular, assim como
algumas situações de caráter cômico, como quando, no capítulo nove, os
personagens principais da saga disputam pelo direito de fazerem suas necessidades onde considerem oportuno.

Essa riqueza de origens e circunstâncias vitais contribuiu para que,
depois de algumas gerações, o lugar e a posição de cada um na
nova sociedade dependesse unicamente de seu esforço e dos méritos
pessoais. O que não foi obstáculo, todavia, para que seu autor, possivelmente um homem de boa educação, mas não um clérigo, mostre um certo desdém pelos
escravos, assim como por certos aspectos da educação e modos dos primeiros
religiosos que se formaram na ilha. Esse mesmo autor incluiu na saga um
número nada desprezível de estrofes escáldicas, 37 no total,
boa parte das quais lhe servem para conceder uma aparência de
autenticidade à obra, buscando não só uma fusão entre o ficcional e o histórico, mas também a iluminação do caráter de alguns
personagens ou marcações estruturais entre os diferentes episódios
da saga.

\SIG{Teodoro Manrique Antón}

Ver também Linguagem; Literatura; Norreno; Poesia escálidca; Sagas
islandesas.

\begin{itemize}
\item \versal{HOLLANDER}, Lee \versal{M}. The Structure of Eyrbyggja saga. \emph{The Journal of
English and Germanic Philology}, vol. 58, n. 2, 1959, pp. 222-227.

\item \versal{PHELPSTEAD}, Carl. Ecocriticism and Eyrbyggja saga. \emph{Leeds Studies
in English}, New Series 45, 2014, pp. 01-18.

\item \versal{TULINIUS}, Torfi \versal{H}. Political Echoes: Reading Eyrbyggja Saga in Light of
Contemporary Conflicts. In: \versal{QUINN}, Judy; \versal{HESLOP}, Kate; \versal{WILLS}, Tarrin
(eds.). \emph{Learning and Understanding in the Old Norse World: Essays
in Honour of Margaret Clunies Ross}. Turnhout: Brepols Publishers, 2007,
pp. 49-63.
\end{itemize}



\chapter{F \textarn{f} \textarc{f}}
\section{\versal{FAGRSKINNA}}

Trata-se de uma narrativa anônima vernacular sobre os reis noruegueses
entre os séculos~{\versal{IX} e \versal{XII}}, composta aproximadamente durante os primeiros
decênios do século \versal{XIII}, provavelmente na localidade de Trondheim, no
Reino da Noruega. Pertence à
tradição das \emph{konungasögur}, presente no âmbito historiográfico
desde o começo do século \versal{XII}. Provavelmente foi patrocinada
pelo rei Hákon Hákonarsson, embora sua composição seja de autoria
anônima.

Aparentemente para a sua composição foram utilizadas outras fontes da
época, como, por exemplo, a \emph{Historia de antiquitate regum
norwagensium}, assim como a \emph{Historia norwegiae}. De acordo com
Katherine Holman, há uma série de similaridades com a
\emph{Heimskringla}, provavelmente pelo fato de terem utilizado as
mesmas fontes que serviram para a composição das narrativas, embora seja
muito mais breve em termos de conteúdo e tenha sido escrita
posteriormente.

O texto apresenta uma história vernacular dos reis da Noruega, desde
Halfdan, o Negro, um proto-histórico rei norueguês e pai de Haroldo
Cabelos Belos, até a batalha de Ré, em 1177, no contexto do reinado de Magnus
Erlingsson. Na narrativa, observa-se temas voltados para uma integração
com contexto aristocrático norueguês, não apresentando tentativas de
vincular os eventos narrados com outros territórios. A cristianização da
Noruega, apresentada de forma gradual, também é um tema presente na
narrativa. De forma geral, trata-se de um texto importante no que diz
respeito à abordagem da situação da escrita da história norueguesa
durante os primeiros decênios do século~\versal{XIII}.

\SIG{Luciano José Vianna}

Ver também Fontes primárias; Historia de Antiquitate Regum Norwagensium;
Historia Norwegiae; Íslendingabók; Laxdaela saga; Morkinskinna; Noruega
da Era Viking.

\begin{itemize}
\item \versal{ALLPORT}, Benjamin. \emph{A Long Time in Politics. The Relevance of
Icelandic Techniques of Time Reckoning for our Understanding of the
Medieval Icelandic World View}. Master of Philosophy Thesis in Viking
and Medieval Norse Studies. Universitetet i Oslo, 2014.

\item \versal{HOLMAN}, Katherine. \emph{Historical Dictionary of the Vikings}. Lanham,
Maryland, and Oxford: The Scarecrow Press, Inc. 2003, p. 196.

\item \versal{JAKOBSON}, Ármann. Royal Biography. In: \versal{McTURK}, Rory (ed.). \emph{A
Companion to Old Norse-Icelandic Literature and Culture}. Oxford:
Blackwell Publishing, 2005, pp. 388-402.

\item \versal{LINCOLN}, Bruce. \emph{Between History and Myth. Stories of Harald
Fairhair and the Founding of the State}. Chicago and London: The
University of Chicago Press, 2014.

\item \versal{SYRETT}, Martin. \emph{Scandinavian History in the Viking Age. A Select
Bibliography}. 3rd Edition Revised by Haki Antonsson and Jonathan Grove.
Department of Anglo-Saxon, Norse, and Celtic. Univeristy of Cambridge,
2004.
\end{itemize}
\section{\versal{FAMÍLIA}}

Durante a Era Viking, podemos afirmar que a família constituía a unidade
social e central da vida. Essas famílias moravam e trabalhavam na mesma
propriedade rural e, muitas vezes, dividiam a mesma casa. Essa ``grande
família'' desempenhou um papel importante na formação da sociedade
nórdica, bem como de suas leis e costumes. Uma família podia ser constituída por vários casais ligados
por laços de sangue ou alianças políticas, pelos seus filhos,
sobrinhos e netos e também pelas famílias de servos. Pode-se dizer que,
durante a Era Viking, o tamanho da família era provavelmente de dez a
vinte pessoas. Os dados sobre famílias nórdicas nessa época ainda são
escassos e muito do que se conhece sobre elas vem das sagas de família
islandesas e da
Arqueologia.

A expectativa de vida nessas comunidades não era muito alta, a mortalidade
infantil era elevada e muitos daqueles que sobreviviam ao nascimento
viviam apenas até o final da primeira infância. As crianças menores de
quinze anos, assim, constituíam boa parte da população. Dos que atingiram
a idade de vinte anos, cerca de metade conseguiu chegar saudável
aos cinquenta anos de idade, e apenas um a cada três indivíduos
alcançou mais de sessenta, embora tenha existido casos
de pessoas que atingiram idades mais avançadas.

Para reforçar o vínculo existente entre as famílias, muitas vezes os
filhos de uma determinada família eram adotados por outra de mais posses
e prestígio, e vice-versa. Quando uma família de menos posses adotava uma
criança de origem mais abastada, ela recebia uma espécie de
pagamento, e era celebrada entre as duas famílias uma espécie de
aliança que fortalecia os laços entre elas e que, na maioria das vezes, 
era mais forte e duradoura do que as relações de sangue.
Além disso, esses pactos de adoção eram uma forma de redistribuir as
crianças entre as famílias. Como a taxa de mortalidade infantil era tão
elevada, alguns casais não possuíam filhos nascidos vivos e essa adoção
era uma maneira de levar uma criança a uma família que não possuía
prole. Aos dezesseis anos, era esperado que o menino assumisse
todos os papéis de um homem adulto, e até mesmo as crianças mais novas já
assumiam algumas responsabilidades adultas. O mesmo acontecia com as
meninas que, com essa idade, já tendo passado pela menarca, deviam estar
prontas para o casamento, que era uma etapa importante para a vida não só
dos jovens, mas de toda a comunidade.

O casamento era um acordo entre a família da noiva e a família do noivo,
que poderia ser proposto pelo pretendente masculino e aprovado pelo pai
da mulher. Em muitos casos, os casamentos eram celebrados para construir
uma aliança entre as famílias. O casamento era o meio pelo qual a
riqueza das famílias era legada para as próximas gerações. No entanto, era
importante também considerar os sentimentos dos noivos nessa aliança. O
ato de cortejar uma mulher podia ser considerado normal, mas
em certos casos também poderia ser visto como um ultraje pela família da mulher. O cortejo
poderia ocorrer de várias maneiras, como visitas do homem à casa da mulher,
conversas com a mulher ou até poemas de louvor à mulher, isso no caso de
famílias abastadas e com certo nível de letramento.

Quando um homem desejava casar-se ele primeiramente procurava os membros
da família para se aconselhar e, depois, buscava uma noiva, pois qualquer
casamento mal planejado ou que fosse desastroso logo no início podia
trazer prejuízos materiais à família do noivo. Se após a proposta de
casamento ser feita o casamento não ocorresse logo, a família da noiva
se sentiria insultada. Se uma proposta de casamento fosse rejeitada, a
família do homem ficaria como a da noiva e, em ambos os casos, as
famílias podiam exigir a vingança de sangue. O ritual do casamento era
dividido em duas partes: o noivado e o casamento. O noivado era uma
espécie de contrato entre o guardião da mulher, geralmente seu pai, o
pretendente ou o representante do pretendente, geralmente também o pai. Não
é claro se o consentimento da mulher era solicitado ou não. No entanto,
a coerção muitas vezes era utilizada para forçar um vínculo político ou
econômico especialmente atraente entre duas famílias.

A família do noivo prometia pagar uma quantia chamada \emph{mundr}
(preço da noiva) para obter o consentimento para o casamento, e, assim, o
pai da noiva declarava a sua filha como prometida e comprometia-se a
pagar um \emph{heimangerð} (dote) no casamento. As duas partes apertavam
as mãos na frente de testemunhas para selarem o contrato e já marcavam
a data para o enlace, geralmente dentro de um ano. Assim, o noivado se
diferenciava de qualquer outra transação comercial: havia um preço
acordado, um aperto de mão e testemunhas. O casamento era celebrado com
uma festa elaborada, com banquete que durava vários dias e geralmente
ocorria na casa dos pais da noiva. O casamento era considerado
obrigatório quando pelo menos seis testemunhas vissem o casal ir para a
cama juntos.

Se o casamento não fosse consumado, o divórcio poderia ser facilmente
obtido por qualquer uma das partes por várias razões. Por exemplo, se
nenhuma criança nascesse do casamento, a união poderia simplesmente ser
dissolvida. Não era incomum que uma mulher se casasse várias vezes. Na
Era Viking, o divórcio era realizado por qualquer uma das partes,
simplesmente declarando o divórcio na frente de testemunhas. Uma vez
que o casamento resultava na junção da riqueza de duas famílias, não é
surpreendente que o divórcio muitas vezes resultasse em disputas sobre
como a riqueza do casal deveria ser dividida e retornar às famílias
originais. O acerto das finanças resultantes de um divórcio muitas vezes
resultava em disputas de sangue entre as famílias que poderiam durar diversas
gerações. Após o divórcio, a mulher tinha direito a metade da
propriedade. Além disso, se o homem tivesse dado o motivo para a
separação, tanto o preço que havia sido dado pela noiva quanto o dote
deveriam ser pagos integralmente à mulher. Assim, depois do divórcio,
uma mulher poderia manter uma substancial independência econômica e
poderia casar-se novamente.

Um homem era considerado adulto depois de ter sobrevivido a quinze
invernos e as mulheres casavam-se bem cedo, geralmente aos doze, treze
anos, logo depois da menarca. Até a idade de vinte anos, praticamente todas as mulheres estavam
casadas. Quando uma criança nascia, ela era
aceita na família por meio de um conjunto de rituais: a mãe demonstrava
essa aceitação amamentando-a no peito e o pai,
levando a criança ao seu joelho, dando um nome a ela e gotejando água na
sua fronte. Uma vez que a criança fosse nomeada, aspergida e amamentada,
então as leis de herança nórdica entravam em vigor, e a criança teria
garantido a sua herança e outros direitos dentro da família. Uma criança
que não fosse aceita por qualquer razão seria condenada à morte por
exposição: ela era colocada para fora da casa e exposta a toda sorte de
perigos, até a morte. Isso geralmente acontecia no caso de
deformidades no nascimento ou por dificuldades econômicas. Além do
casamento, os indivíduos e as famílias podiam unir-se de outra forma, um
vínculo poderoso, o da fraternidade de sangue.

\SIG{Luciana de Campos}

Ver também Cotidiano; Crianças e infância; Mulheres; Sociedade.

\begin{itemize}
\item \versal{CHRISTIANSEN}, Eric. \emph{The Norsemen in the Viking Age}. Oxford:
Blackwell Publishing, 2006.

\item \versal{GIBSON}, Michael. A vida familiar. In: \emph{Os Vikings}. São Paulo:
Melhoramentos, 1990, pp. 18-19.

\item \versal{HAYWOOD}, John. Family. In: \emph{Encyclopaedia of the Viking Age}.
London: Thames and Hudson, 2000, p. 69.

\item \versal{NOUGIER}, Louis-René. Blutsbande. \emph{Wikinger}. Hamburg: Tesslorf
Verlag, 1983, pp. 30-31.
\end{itemize}
\section{\versal{FÆREYINGA SAGA}}

A história textual da \emph{Færeyinga saga} (\emph{Saga dos Feroeses}) não é
fácil de estabelecer. Na verdade, essa saga anônima não sobreviveu como
manuscrito isolado ou texto individual dentro de uma grande coleção de
manuscritos, mas através de seções desarticuladas e interpoladas em
\emph{Olafs Helga saga} (\emph{Saga de Óláfr, o Santo}), na versão do escritor
islandês Snorri Sturluson em seu \emph{Heimskringla} e na chamada 
\emph{Óláfs saga Tryggvasonar en mesta} (\emph{saga
de Óláfr, o filho de Tryggvi}). Na verdade, a versão que conhecemos hoje
da \emph{Færeyinga saga} foi reconstruída no século \versal{XIX} pelo dinamarquês
Carl Christian Rafn. No entanto, a fonte principal dessa saga está no
\emph{Flateyjarbók}, um pergaminho islandês de 225 páginas redigido
pelos clérigos Jón Þórðarson e Magnús Þórhallsson entre 1387 e 1390. Presentes nesse pergaminho estão a \emph{Jómsvíkinga saga} (\emph{Saga Viking
Jom}); a própria \emph{Færeyinga saga} (que parece corresponder
geralmente ao original e não foi preservada em qualquer outro lugar); a
\emph{Orkneyinga saga} (\emph{Saga dos Orcadianos}) e a \emph{Grænlendinga
saga} (\emph{Saga dos Groenlandeses}), entre outras. De acordo com Halldórsson
(1987, pp. \versal{xcviii-cxx}), o texto da \emph{Færeyinga saga} que aparece no
\emph{Flateyjarbók} parece derivar de três cópias: os capítulos 1-27,
34-42 e 49-59 vêm de um manuscrito perdido; os capítulos 28-33 são cópia
do exemplar da \emph{Óláfs saga Tryggvasonar en mesta} recolhido em
\emph{Flateyjarbók}; os capítulos 43 e 45-48 foram tomados, por sua vez,
da cópia da \emph{Óláfs saga helga} também coletadas no
\emph{Flateyjarbók}.

Existem algumas evidências de que o autor anônimo da \emph{Saga dos
Feroeses} utilizou uma série de fontes, tanto escritas e orais, para a
composição do texto. Novamente citando Halldórsson (1987, pp.
\versal{clii-clxvi}), entre as prováveis fontes escritas estão uma versão inicial
do islandês \emph{Landnámabók} (\emph{Livro da colonização}), uma obra
esquemática sobre a vida dos reis da Noruega (possivelmente também
contendo material relacionado com os reis da Dinamarca e Suécia), a
\emph{Jómsvíkinga saga} em alguma versão primordial e a \emph{Hlaðajarla
saga} (\emph{Saga do jarlar de Hlaðir}). Além dessas fontes potenciais, também
se pode estabelecer certas relações literárias entre a \emph{Saga dos
Feroeses} e outros textos islandeses medievais, como a \emph{Eyrbyggja
saga} (\emph{Saga dos habitantes de Eyrr}) ou a \emph{Laxdæla saga} (\emph{Saga dos
habitantes do Vale do Salmão}). É interessante o desconhecimento do autor da
\emph{Saga dos Feroeses} sobre a geografia das Ilhas Faroé, indicando que ele
nunca visitou esse arquipélago do Atlântico, embora pareça provável que
tenha obtido informações sobre determinados locais das ilhas Faroé e os
nomes dos personagens principais da saga a partir de fontes indígenas
orais. Baseado em uma série de dados filológicos, culturais, literários
e comparativos, parece seguro assumir que a composição da versão original
da \emph{Saga das Ilhas Faroé} ocorreu entre 1210 e 1215 e que seu autor era
provavelmente um nativo de Eyjafjörður, Islândia.

Sobre o gênero literário da \emph{Saga dos Feroeses}, é uma prática
comum entre os estudiosos da literatura islandesa medieval incluírem
essa obra no chamado \emph{Konungasögur} (sagas de reis) e
\emph{Íslendingasögur} (sagas dos islandeses). No entanto, parece mais
correto classificar essa saga dentro do subgênero chamado sagas
políticas. De acordo com Berman (1985), esse gênero seria composto
principalmente de três sagas: a \emph{Jómsvíkinga saga}, a
\emph{Orkneyinga saga} e a \emph{Færeyinga saga}, cujo tema
comum seria uma série de lutas pela independência e liderança política
entre as autoridades norueguesas e suas colônias.

Pelo que a temática se refere, a \emph{Saga dos Feroeses} conta a história
desse pequeno arquipélago atlântico desde aproximadamente 825 d.C., com
a chegada de Grímr Kamban, o primeiro colono conhecido, até 1035, com a
hegemonia de Leifr, filho de Özurr, nas ilhas. É uma história contada em
vários níveis: por um lado descreve como os reis da Noruega tentam
conquistar as ilhas através de esforços políticos, em vez da
guerra. Por outro lado, é uma história dramática sobre uma série de
brigas de família que vão até três gerações e que retrata um
momento de condições religiosas e ideológicas em que ocorre uma colisão
entre valores conflitantes das antigas crenças pagãs, que são
substituídas pelo cristianismo. Em suma, a luta pela hegemonia política
das ilhas Faroé e o conflito subsequente irrompeu entre os partidários
da manutenção de autonomia cultural e política das ilhas Faroé
(representados pelo Pagan Þrándr) e os partidários da sujeição à
monarquia norueguesa (representada pelo cristão Sigmundr). São esses os eixos
principais da \emph{Saga dos Feroeses}.

De acordo com Skyum-Nielsen (1973, p. 14), enquanto documento político, o
autor da saga tentou enviar uma mensagem para a classe dominante
islandesa do século \versal{XIII}, para que a partir da análise da situação dos
feroeses, percebem-se as consequências das disputas políticas, podendo, assim,
salvaguardar a independência da Islândia, reduzindo o poder
da Igreja e redistribuindo de forma mais equitativa a terra e o trabalho.

A \emph{Saga dos Feroeses} era tão popular em seu tempo que chegou a ter
interessantes versões rimadas, como a islandesa \emph{Sigmundar rímur}
e \emph{Þrænlur} ou a feroese \emph{Sigmundarkvæði}.

\SIG{Mariano González Campo}

Ver também Egils saga; Era Viking; Ilhas Faroé; Viking.

\begin{itemize}
\item \versal{ALMQVIST}, Bo. Some folklore motifs in Færeyinga saga. In:
 \versal{DHUIBHNE-ALMQVIST}, Éilís; \versal{Ó CATHÁIN}, Séamas (eds.). \emph{Viking Ale.
Studies in Folklore Contacts between the Northern and the Western
Worlds}. Aberystwyth: Boethius Press, 1991, pp. 114-126.

\item \versal{BERMAN}, Melissa. The political sagas. \emph{Scandinavian Studies}, vol. 57,
1985, pp. 113-129.

\item \versal{FOOTE}, Peter G. \emph{On the Saga of the Faroe Islanders}. London:
University College London, 1965.

\item \versal{HALLDÓRSSON}, Ólafur (ed.). \emph{Færeyinga saga}. Reykjavík: Stofnun
Árna Magnússonar á Íslandi, 1987.

\item \versal{MUNDAL}, Else. Færeyinga saga- a Fine Piece of Literature in Pieces. In:
 \versal{MORTENSEN}, Andras; \versal{ARGE}, Símun \versal{V}. (eds.). \emph{Viking and Norse in the
North Atlantic}. Tórshavn: Føroya fróðskaparfelag, 2005, pp. 43-51.

\item \versal{SKYUM-NIELSEN}, Erik. Færeyinga saga: Ideology transformed into Epic. In:
\emph{Alþjóðlegt fornsagnaþing, Reykjavík 2.-8. ágúst 1973.
Fyrirlestrar: 2. hefti}. Reykjavík, 1973.
\end{itemize}

\section{\versal{FÉLAG}}

O termo em si é de difícil conceituação e ainda há um debate acirrado em
torno de seu significado. O seu sentido imediato pode ser entendido como
``companheirismo'', ``camaradagem'', ``parceria'' ou mesmo
``sociedade''. O \emph{félag} pode ser encontrado em cerca de 22
pedras rúnicas da Era Viking, sendo a Pedra de Berezan um exemplo:
\emph{nessa pequena pedra, um certo Grani escreve runas em memória de
seu amigo Karl, seu parceiro} (\textbf{fi:laka : si}).

É discutível aqui a possibilidade do companheirismo entre esses Grani e
Karl estar ligado ao empreendedorismo de longas distâncias, quando os
homens dividiriam os custos e os lucros da atividade comercial, tendo em
vista que a Pedra de Berezan foi achada no ponto de encontro entre o rio
Dnieper e o mar Negro, rota de passagem para as oportunidades oferecidas
pelos mercados orientais. Normalmente o conceito é entendido como os
esforços de empreendedorismo entre dois homens que investem seus
recursos para preparar e tripular um navio com o objetivo de fazer
comércio, mas em algum momento essa ligação também pode significar o
emprego da violência durante a jornada, como nos atestam várias
inscrições rúnicas.

Provavelmente o \emph{félag} era um dos pilares fundamentais para o
desenvolvimento comercial escandinavo. De modo geral, o comércio no
medievo ocorria em uma dimensão mais local, pois trocas feitas por
longas distâncias precisavam, dentre outras coisas, de mercados, e muitos
desses eram influenciados pelas ambições políticas de reis e governantes,
constituindo uma realidade que não foi diferente na Era Viking, a
exemplo de Ribe, Birka e Hedeby. Esses mercados não possuíram
fortificações sólidas até o século~\versal{X} e poderiam estar vulneráveis aos
ataques de saqueadores (a exemplo de Londres, Paris e Dorestad).
Provavelmente a segurança dos mercados era feita, em grande parte, com o
consenso e ajuda dos próprios frequentadores. As aparições de navios com
maiores tonelagens de carga seriam mais comuns a partir do século~\versal{X};
antes disso é possível que os navios carregassem seus próprios bandos
armados para a proteção dos homens e das cargas.

Isso nos leva a um aspecto curioso do termo, relativo às oportunidades
guerreiras em meio à jornada mercantil. A Runestone \versal{DR} 66 contém uma
mensagem em memória de um amigo morto quando os reis lutaram \emph{(þo
kunukaʀ barþusk)}, talvez indicando também um contexto de conflito
náutico, o que está alinhado a esse sentido de empreendedorismo marítimo
cercado pelo termo. Na Runestone Arhus 6, encontramos uma mensagem feita
por Tosti, Hofi e Freybjǫrn em memória de Asur Saksa, seu parceiro e um
\emph{drengr} muito bom (\textbf{filaka : sin : harþa : kuþan : trik}).
A inscrição ainda revela que o morto possuía um navio em parceria com
outrem, um que não encomendou a Runestone. Talvez Tosti, Hofi, Freybjorn
e o falecido homenageado operassem em parceria, com vários navios em
expedições guerreiras e mercantis conjuntas.

É curioso o fato de que as maiores menções ao termo estão circunscritas
no campo das inscrições rúnicas, sendo elas quase ausentes entre os
versos da poesia escáldica. Esse tipo de poesia nos diz pouco sobre o
comércio e os aspectos econômicos da guerra, que estão intimamente
ligados às recompensas entregues pelos reis e líderes guerreiros aos
seus homens, muito diferente do comum acordo entre parceiros de
negócios. Os ganhos em guerra na poesia são presentes, concessões e
recompensas, não um acordo de divisões dos espólios. Essa característica
de acordo pode ser examinada no termo \emph{Feolaga} do inglês antigo.
Segundo o relato do ano de 1016 das \emph{Crônicas Anglo-Saxônicas},
especificamente no acordo de Olney, firmado por Knútr e Edmund, os reis
concordam na partilha inglesa, tornando-se assim \emph{feolagan wed
broðra}, \emph{feolagan} e irmãos jurados\emph{.}

Resta-nos lembrar que as mulheres se encontravam excluídas dessa relação
de \emph{félag}. Fora das granjas e do seio familiar, a sociedade
nórdica restringia bastante as oportunidades das mulheres. Temos notícia
de escaldas e de escultoras de runas, mas nenhuma certeza sobre a
existência de mercadoras, de artífices ou de guerreiras, e, se de fato existiram,
possivelmente foram escassas a ponto de não termos qualquer acesso a sua
inserção e convívio em uma parceria nos moldes do \emph{félag} (ainda
que o termo \emph{drengskapr} pudesse ser aplicado para definir algumas
mulheres no contexto medieval, mas fora da Era Viking).

\SIG{Pablo Gomes de Miranda}

Ver também Comércio; Era Viking; Expansão nórdica; Sociedade.

\begin{itemize}
\item \versal{JESCH}, Judith. \emph{Ships and Men in the Late Viking Age}: the
vocabulary of runic inscriptions and skaldic verse. Woodbridge: The
Boydell Press, 2001.

\item \versal{PAGE}, Raymond Ian. Scandinavian Society, 800-1100: the contribution of
runic studies. In: \versal{FAULKES}, Anthony; \versal{PERKINS}, Richard (orgs.).
\emph{Viking Revaluations}. Birmighan: Viking Society for Northern
Research, 1993, pp. 145-159.

\item \versal{PAGE}, Raymond Ian. \emph{Runes and Runic Inscriptions}. Woodbridge:
Boydell Press, 1999.

\item \versal{ROESDAHL}, Else. \emph{The Vikings}. London: Penguin Books, 1998.
\end{itemize}

\section{\versal{FERREIROS E FERRARIA}}

Ferreiros são os trabalhadores que tradicionalmente fabricam armas,
ferramentas e outros utensílios utilizando como matéria-prima principal
o ferro, bem como os fabricantes de joias que trabalham recorrentemente
com o ouro e a prata. Denominam-se por ferrarias os utensílios que fazem
parte da oficina do ferreiro que o auxiliam no seu trabalho, tais como
martelos de diferentes pesos, formatos e tamanhos; bigornas grandes e
pequenas para que sejam trabalhados diferentes tipos de produtos;
tenazes de diversos tamanhos para que se possa colocar e retirar os
objetos de dentro da forja; formões, que são utilizados para perfurações
e cortes das peças que estão sendo moldadas, conforme podemos ver nos
textos de Colins sobre o tema. No período medieval, para Eliade, os
ferreiros podem ser subdivididos em três grupos, muito embora seja
possível que houvesse alguns deles que dominassem todo o processo: a) o
ferreiro de mina ou de alto forno, que extrai os minérios da terra e
funde os metais; b) o ferreiro do ferro negro, que trabalha na forja,
mas não extrai minérios; c) o ferreiro dos metais preciosos, que
trabalha como joalheiro.

Os trabalhos comuns de um ferreiro tais como a fabricação e colocação de
ferraduras, de cravos/pregos, a forja de tenazes, martelos, panelas,
pás, picaretas, serras e demais ferramentas de uso próprio do camponês,
poderiam ser classificados dentro da categoria de \emph{labor} ou
trabalho, pois tinham como objetivo confeccionar objetos meramente
funcionais, sem beleza, sem valor estético, destinados apenas para o
trabalho árduo dos camponeses. Grosso modo, o ferreiro não possuía as
técnicas apuradas de combate, mas ele dominava a forja de espadas,
machados, lanças, flechas, escudos, armaduras e ferraduras para os
cavalos utilizados em combate pelos guerreiros. As beligerâncias eram
tornadas possíveis mediante o seu trabalho.

Todavia, quando um ferreiro produzia espadas, armaduras, machados para os
guerreiros, sobretudo para os mais abastados, os objetos além de um
caráter funcional, tinham também símbolos heráldicos, referências a figuras
sagradas para os nórdicos medievais, pedras preciosas, detalhes
em prata, ouro e bronze. Sem contar, obviamente, que o metal utilizado
para esses objetos era de qualidade muito superior aos empregados na
forja de uma ferramenta simples; sob esse aspecto, a criação do ferreiro
não seria \emph{labor}, seria \emph{opus} por conta das técnicas e
materiais utilizados. Entretanto, nunca é demais lembrar que é anacrônico
classificar esses detalhamentos empregados nesses objetos como arte, tal
qual costumamos fazer com esse tipo de trabalho nos dias atuais.

Na Escandinávia da Era Viking os ferreiros eram figuras predominantemente
masculinas e, até o momento, desconhece-se a presença de mulheres
trabalhando nos ambientes das forjas. Essas oficinas, para Embleton e
Harrison, eram ocupadas por homens de diversas idades, que eram iniciados
na profissão ainda na infância, executando tarefas mais simples, até que
um dia pudessem se tornar mestres forjadores. Dentre as tarefas
executadas podemos citar: alimentar o fogo com carvão, manter os
recipientes de água e de óleo abastecidos para a atividade de têmpera,
movimentar os foles que injetavam ar na forja para que o fogo atingisse
temperaturas mais altas, utilizar tenazes para segurar pedaços de ferro
incandescentes que eram combinados em ligas, martelar as peças de
metal que eram forjadas, cuidar do desbaste e afiação das lâminas
etc. Era também atribuição das forjas desse período saber trabalhar
com madeira para fazer os escudos, cabos das armas e
ferramentas, passando pelas etapas de corte, entalhamento,
endurecimento/impermeabilização feito com óleos vegetais e chamas;
também era atribuição de pelo menos um dos trabalhadores da forja saber
operacionalizar os processos de curtimento, corte, costura e desenhos
com couro para a confecção das bainhas das lâminas e capas dos escudos.

No cenário mitológico, segundo Hedeager, a figura dos ferreiros nórdicos
está diretamente atrelada aos anões (\emph{dvergar}). Dentre os mais
célebres podemos destacar os filhos de Ivaldi e os irmãos Brokkr e
Eitri. Esses anões aparecem na \emph{Skáldskaparmál} ao serem enganados
por Loki em uma espécie de aposta que definiria quem seriam os ferreiros
mais habilidosos e ao mesmo tempo traria certos benefícios para Loki
diante dos outros deuses. Aos filhos de Ivaldi é atribuída a produção do
Skidbladnir, o navio de Freyr, de Gungnir, a lança de Odin e do cabelo
dourado de Sif que foi colocado em sua cabeça para substituir o que Loki
havia cortado. Os irmãos Brokkr e Eitri foram os responsáveis pela
fabricação do javali de ouro~Gullinbursti, do anel de ouro de~Draupnir
e do martelo~Mjöllnir. Desse relato, depreendemos a grande capacidade de
criação dos ferreiros, indo desde joias e adereços, passando por navios,
armas e outros objetos com propriedades mágicas.

\SIG{Michel Roger Boaes Ferreira}

Ver também Comércio; Espada; Metalurgia.

\begin{itemize}
\item \versal{ANÔNIMO}. \emph{Prose Edda}. Oxford: Clarendon Press, 1982.

\item \versal{COLINS}, John. \emph{The European Iron Age}. London: Routledge, 1984.

\item \versal{ELIADE}, Mircea. \emph{Ferreiros e Alquimistas}. Rio de Janeiro: Zahar
Editores, 1979.

\item \versal{EMBLETON}, Gerry; \versal{HARRISON}, Mark. \emph{Viking Hersir - 793-1066 \versal{AD}}.
Oxford: Osprey Publishing, 1993.

\item \versal{HEDEAGER}, Lotte. \emph{Iron Age Myth and Materiality}: An Archaeology of
Scandinavia ad 400-1000. New York: Routledge, 2011.
\end{itemize}

\section{\versal{FESTAS E FESTINS}}

As festas, festins e banquetes, 
além de expressarem poder e riqueza, possuíam a finalidade de celebrar os laços de amizade, as alianças
políticas e guerreiras. Portanto, essas festividades serviam como uma
maneira de mostrar que esses compromissos seriam honrados. O ato de
comer e beber juntos, embriagar-se e fartar-se de todos os tipos de
carnes era uma maneira de mostrar que esses laços não seriam rompidos.
No momento do festim celebrava-se, portanto, a durabilidade dos laços e
os comensais mostrariam uns para os outros que eles não se desatariam.
Os festins tinham o intuito de celebrar as relações de paz.

Durante a Era Viking, os festins e as festas eram sempre organizados em
torno de ocasiões importantes para a comunidade, tais como o nascimento
de uma criança, filha de um chefe ou guerreiro importante, casamentos que
selariam alianças entre as comunidades, o início ou o final de guerras,
celebração de amizades e alianças, além, é claro, das festividades religiosas
que envolviam alimentos e comidas consideradas sagradas. Em todas essas
ocasiões havia o comprometimento de quem oferecia o festim para com os
convidados especiais e com a comunidade em que vivia: a comida e a
bebida deviam ser fartas e ninguém devia ficar insatisfeito ou ser mal
servido. A hospitalidade devia estender-se a todos. Essas refeições, que
tinham um caráter solene, aconteciam de tempos em tempos e com
regularidade, não só como uma maneira de celebração, mas de constante
demonstração de poder.

A duração de um banquete ou festim podia alcançar dias. Os anfitriões se
empenhavam na tarefa de entreter os convidados, que possuíam uma grande
capacidade de comer e beber, pois a gula ia além da demonstração de
força: mostrava como o glutão era poderoso e também respeitado
entre os seus. A música, a dança, jogos de tabuleiros e ao ar livre
faziam parte do entretenimento para essas celebrações, que envolviam
discussões sobre assuntos importantes para a comunidade e, nesse clima
de alegria, com comida e bebida em abundância, os líderes selavam sua
amizade e se comprometiam a mantê-la. Essa mescla de comida, bebida,
jogos, risos e diversão com a tomada de decisões e alianças
político-militares tem uma raiz antiga que, já havia sido descrita por
Tácito. Entre os germanos, o autor latino havia observado que essas
tribos discutiam questões importantes durante os banquetes e festas,
conferindo a essas comemorações um caráter político e não
somente festivo. Ainda segundo Tácito, o excesso de bebidas alcoólicas
permitia que as opiniões mais sinceras e muitas vezes nada agradáveis
fossem ditas, levando a discussões acaloradas que não permitiam que
decisões importantes fossem tomadas, adiando para o dia
seguinte, quando não estivessem mais sob o efeito do álcool, as
importantes deliberações que precisavam ter.

Essas reuniões serviam também para demonstrar o poder econômico de quem
as oferecia: nenhuma modéstia ou avareza era tolerada. O anfitrião não
economizava na comida servida, embora pouco se saiba quais eram
os pratos consumidos nessas ocasiões. Acredita-se que carnes
cozidas e assadas, papas, sopas e pães eram os mais comuns, pois o
importante era a quantidade de bebidas que havia nesses festins. A
cerveja era servida em abundância e o hidromel não faltava nas taças e,
para reforçar o seu poder e mostrar o quão ricos eram, alguns chefes não
hesitavam em servir vinho, vindo das regiões mediterrâneas. O vinho,
assim como o hidromel, por se tratarem de bebidas caras, estavam
diretamente ligadas ao deus Odin e, portanto, tanto quem as consumisse
como quem as servisse em abundância mostraria também uma ligação estreita com
o deus e daria ao festim um caráter sagrado.

É importante ressaltar que aquele que oferecia o festim devia também,
além de garantir comida e bebida aos participantes, protegê-los
pelas leis da hospitalidade, assegurando a integridade física e moral de
cada um que estivesse ali presente, não permitindo a ocorrência de brigas que levassem
alguém à morte, traições, injúrias e vendetas enquanto
durasse a celebração. Mas, mesmo com essas leis, muitas vezes mortes e
traições aconteciam e pode-se dizer que alguns festins eram organizados
justamente com esse fim.

\SIG{Luciana de Campos}

Ver também Alimentação; Cerveja; Hidromel; Sociedade.

\begin{itemize}
\item \versal{ALTHOFF}, Gerd. Comer compromete: refeições, banquetes e festas. In:
 \versal{FLANDRIN}, Jean-Louis; \versal{MONTANARI}, Massimo. \emph{História da
Alimentação}. São Paulo: Estação Liberdade, 1998, pp. 300-309.

\item \versal{HAGEN}, Ann. \emph{Anglo-Saxon food and drink}. London: Anglo Saxon Book,
2010.

\item \versal{WARD}, Christie. Alcoholic beverages and drinking customs of the Viking
Age. \emph{The Viking Answer Lady}, 2005. Disponível em:
http://www.vikinganswerlady.com/drink.shtml. Acesso em 14/04/2017.
\end{itemize}

\section{\versal{FINLÂNDIA DA ERA VIKING}}

É provável que o que fez Katherine Holman, em seu útil \emph{Historical dictionary of the Vikings}, afirmar que pouco se sabia sobre a Finlândia durante a Era
Viking tenha sido a falta de documentos escritos sobre o local antes do século
\versal{XII} d.C., quando o reino da Suécia começou a incorporar a região. No entanto, novas pesquisas propõem uma interpretação diferente
sobre a relação da Finlândia com o mundo nórdico. A própria cronologia
``clássica'' da Era Viking passou a ser questionada por estudiosos que
advogam em prol de um alargamento temporal (750"-1250 d.C.) já que a
cronologia estabelecida -- vinculada aos eventos ocorridos
principalmente nas Ilhas Britânicas -- pouco (ou nada) interferiram na
relação dos nórdicos com a Finlândia.

A região dos principais assentamentos do período é específica e
argumenta-se que durante a Era Viking essas áreas não foram
expandidas. De forma geral, a região sudoeste da Finlândia, próxima à
costa báltica e à Suécia, sendo as províncias da Finlândia Própria,
Satakunta, Taváscia Própria e o arquipélago de Alanda os principais
cenários do desenrolar das atividades humanas na região. Além dessas
ocupações balto-fínicas, o norte -- principalmente a Lapônia e a
Ostrobótnia -- é, de forma esparsa, o local do assentamento das
populações sami. Estima-se que a população da Finlândia no período
beirava 50 mil pessoas.

É durante a Era Viking que a essa região concentrada no sudoeste da
Finlândia experimenta um crescimento populacional e econômico que, antes de ser derivado de um sistema regional de contatos
comerciais, é fruto de mudanças domésticas como o estabelecimento de
vilas fixas e o fim da agricultura de queimadas. A partir disso, o
contato com outras regiões, principalmente Birka, pode ser levado em
consideração. As ilhas de Alanda são o melhor exemplo disso. Repleto de
descontinuidades históricas por ter sido uma região de repetidas guerras,
esse arquipélago foi importante ponto de contato com a Uppland
sueca (onde se localiza Birka) e isso é atestado graças aos produtos
fínicos encontrados em enterramentos em Birka. Além disso, essas ilhas
são importantes como referenciais para a navegação costeira, uma vez que
o arquipélago está localizado numa região de passagem para o Báltico
oriental via Suécia.

Dentro dessas dinâmicas comerciais estabelecidas na região Báltica, as
terras fínicas contribuíam, principalmente, com peles de animais como
castor, lobo, alce e urso marrom. Esses pequenos grupos sociais que
viviam da agricultura, pesca e caça tinham no inverno o seu período
mais delicado, já que é justamente nesse solstício que ocorriam, de
acordo com as sagas, as principais atividades de ganho: a caça e o
comércio com os povos regionais.

Nas três províncias continentais citadas, há ainda a presença de pelo
menos uma igreja, erguida durante a Alta Idade Média, geralmente
localizada próxima a cemitérios pré-cristãos. Esses enterramentos são
vistos pela nova literatura como um ``fenômeno finlandês", já que são
cremações no nível do chão. Estabeleceu-se, ainda, um paralelo com os
enterramentos com barcos da Escandinávia, com a diferença que estes não
eram cremados. As exceções desse fenômeno são Eura e Köyliö, na
província de Satakunta, onde a tradição de conduzir a inumação dos
corpos se inicia e se espalha pela região no século \versal{XI} d.C. Esses
enterramentos contêm vestimentas e joias que nos ajudam a estabelecer um
panorama mais complexo das dinâmicas estabelecidas. No caso das joias, a
principal matéria-prima para sua fabricação é o bronze e, com o aumento
populacional da região, enxerga-se um aumento da produção desses
artifícios -- assim como o declínio de seu refinamento.

O fluxo de peças de prata que se estabelece na região provém dos
territórios vizinhos, tanto no Ocidente quanto no Oriente, e eram usadas
principalmente como moeda ou joias. Achados arqueológicos do Período
Viking na região contêm balanças e pesos e corrobora com o uso da
prata como meio de troca, que tinha o valor estabelecido a partir de sua
pesagem. Já as moedas árabes e bizantinas datadas do século~\versal{X} d.C., por
exemplo, trazidas por caravanas, eram perfuradas e utilizadas em colares
como joias.

Diversas moedas foram encontradas em enterramentos, sendo quatro mil
delas de origem germânica, mil e seiscentas islâmicas, mil anglo-saxãs e
outras de regiões diversas. As mais antigas, encontradas em Alanda, são
datadas do século~\versal{IX} e \versal{X} d.C. e são do Oriente, o que ajuda a especular
sobre o fluxo de caravanas que iam ao leste e retornavam. Já as moedas
encontradas na Finlândia e Taváscia Própria são provenientes do
Ocidente e datadas do final do século~\versal{XI} d.C. O principal problema
referente aos enterramentos são justamente as tentativas de estabelecer
seu contexto, ou seja, os motivos daquele entesouramento
(\emph{hoarding}). No fim, acabam sendo exercícios especulativos, pois
existem situações que extrapolam as considerações práticas (enterrar
para resgatar no futuro), como, por exemplo, as possíveis relações com a esfera
sagrada.

\SIG{Vítor Bianconi Menini}

Ver também Birka; Gotland; Sámi, fínicos e nórdicos.

\begin{itemize}
\item \versal{EDGREN}, Torsten. The Viking age in Finland. In: \versal{BRINK}, Stefan; \versal{PRICE},
Neil (eds.). \emph{The Viking World}. New York: Routledge, 2008, pp.
470-484.

\item \versal{JOONAS AHOLA}, Frog; \versal{LUCENIUS}, Jenni. The Viking Age in Åland:
Insights into Identity and Remnants of Culture. \emph{Annales Academiæ Scientiarum
Fennicæ}, n. 372. Helsinki: Finnish
Academy of Science and Letters, 2014.

\item \versal{JOONAS AHOLA}, Frog; \versal{TOLLEY}, Clive. Fibula, Fabula, Fact:
The Viking Age in Finland. \emph{Studia Fennica Historica}, n. 18. Helsinki: Finnish Literature Society, 2014.
(Studia Fennica Historica, 18).

\item \versal{MÄGI}, Marika. Viking Age Finland: the Land of Samis and Finns.
\emph{Estonian Journal of Archaeology}, Tallinn, vol. 2, n. 19, 2015,
pp. 168-172.

\item \versal{ODNER}, Knut. Saamis (Lapps), Finns and Scandinavians in history and
prehistory: Ethnic origins and ethnic processes in
Fenno-Scandinavia.~\emph{Norwegian Archaeological
Review}\textbf{,~}Abingdon, vol. 1-2, n. 18, 1985, pp. 01-12.

\item \versal{ZACHRISSON}, Inger. Comments on Saamis, Finns and Scandinavians in
history and prehistory. \emph{Norwegian Archaeological Review},
Abingdon, vol. 1-2, n. 18, 1985, pp. 19-22.
\end{itemize}
\section{\versal{FLATEYJARBÓK}}

\emph{Flateyjarbók} é como se nomeia o códice~\versal{GKS} 1005 fol., que
em tamanho é o mais extenso dos manuscritos islandeses medievais conservados. Está formado por um total de 225 folhas, das quais 202 foram
escritas no final do século~\versal{XIV} e outras 23 adicionadas ao manuscrito na segunda metade do~\versal{XV}. A obra foi encarregada pelo latifundiário Jón
Hákonarson, que vivia em Víðidalstunga, no distrito de Húvanatn ao norte
da Islândia, tal como afirma no prólogo do mesmo. Os escritores, ou
melhor, os compiladores que se encarregaram de sua redação foram os
monges Jón Þórðarson (até a página 134v) e Magnús Þórhallsson, que
após terminar a obra ficou responsável por suas requintadas
iluminuras. Sabemos que o primeiro nasceu na Noruega em 1388, de modo
que a primeira parte do manuscrito foi com toda certeza escrita em 1387
e também que Magnús Þórhallsson fez alguns acréscimos nos seis anos
posteriores, o que indicaria que no ano de 1394 o manuscrito já estaria
concluído. Para sua elaboração se utilizou a pele de mais de cem
bezerros, o que nos dá a verdadeira medida dos recursos de seu promotor,
assim como da importância que se concedia na Islândia da época para a
fixação das tradições próprias, mas também das que as uniam com a
Noruega, lugar de onde procediam a maioria de seus antepassados.

Por razões paleográficas se indica que o local mais provável de sua
confecção foi o monastério beneditino de Þingeyrar, embora alguns
pesquisadores também mencionem a região oriental de Húnavatn, no
denominado Skagafjörður. O nome Flateyjarbók foi dado em decorrência de sua origem na
ilha de Flatey, uma pequena ilha situada em Breidafjörð, onde residiu
Jón Finnsson, o último de seus donos islandeses, que em
1647 entregou o manuscrito ao bispo Brynjólfur Sveinsson, que por sua vez cedeu 
o documento, em 1656, ao rei dinamarquês Federico~\versal{III} para sua biblioteca real. O manuscrito
permaneceu em Copenhage até o inverno de 1971 quando foi devolvido para
as autoridades islandesas.

O Flateyjarbók cobre a história dos reis da Noruega desde o ano de 850
até \emph{ca}. 1260, representada pela denominada \emph{Grande saga de Ólafr
Tryggvason}, a \emph{Saga de Ólafr, o Santo}, junto com a \emph{Saga do rei Sverrir} e a \emph{Saga de Hákon Hákonarson}, em que se incluem materiais não conhecidos por outras fontes e que em sua maior parte aparecem na forma de
\emph{þættir} ou relatos curtos. Estes desempenham uma função estrutural
com respeito as sagas que os contêm, na medida que apresentam
informações relevantes de caráter genealógico ou histórico, ou
simplesmente ajudam a caracterização dos personagens principais das
ditas sagas.

Porém além das grandes sagas sobre os reis noruegueses e outros relatos
breves encerrados nelas, o manuscrito contém uma versão quase completa
da \emph{Orkneyinga saga}, a \emph{Fóstbræðra saga}, com capítulos não
contidos em outras versões da mesma, assim como a \emph{Færeyinga saga}
e a \emph{Grænlendinga saga}; esta última não se conserva em nenhum outro
manuscrito e é uma das fontes mais antigas do descobrimento de
Vínland. Digna de menção é a predileção do segundo escriba, Magnus,
pelos documentos de tipo histórico que com toda certeza se encontram
detrás da inclusão no manuscrito de crônicas (\emph{Hversu}
\emph{Noregr} \emph{byggðist}), genealogias (\emph{Ættartölur}) e anais
(\emph{Flateyjarbókarannáll}) com o fim de glorificar as monarquias
norueguesas e com ela os próprios islandeses em que os documentos são
representados como iguais; como coparticipantes do sangue real
norueguês. Também neste manuscrito encontramos a única cópia que se
conservou do poema \emph{Hyndluljoð}, no qual algumas estrofes
versam sobre a importância do conhecimento da genealogia, neste caso a
de Óttarr, o protegido da deusa Freyja.

\SIG{Teodoro Manrique Antón}

Ver também Islândia da Era Viking; Literatura; Poesia éddica; Poesia
escáldica; Sagas islandesas.

\begin{itemize}
\item \versal{ASHMAN ROWE}, Elisabeth. \emph{The Development of Flateyjarbók. Iceland
and the Norwegian Dynastic Crisis of 1389}. Odense: University Press of
Southern Denmark, 2005. (The Viking Collection:
Studies in Northern Civilization, vol. 15)

\item \versal{DUBOIS} Thomas A. A History Seen: The Uses of Illumination in
\emph{Flateyjarbók}. \emph{The Journal of English and Germanic Philology},
vol. 103, n. 1, 2004, pp. 01-52

\item \versal{HALLDÓRSSON}, Ólafur. Af uppruna Flateyjarbókar. In: \versal{STEINGRÍMSSON},
Sigurgeir; \versal{KARLSSON}, Stefán; \versal{TÓMASSON}, Sverrir (eds.).
\emph{Grettisfærla: Safn ritgerda eftir Ólaf Halldórsson.} Rit 38.
Reykjavík: Stofnun Árna Magnússonar á Íslandi, 1990, pp. 427-431.

\item \versal{HARALDSDÓTTIR}, Kolbrún. Die Flateyjarbók als Quelle zur Geschichte des
Isländischen-annähernd aud halbem Wege zwischen erster Besiedlung und
Gegenwart. In: \versal{WESTERMANN}, Rainer (eds.). Bruno-Kress-Vorlesung: Ernst
Moritz Arndt Universität, 2004.

\item \versal{WÜRTH}, Stefanie. \emph{Elemente des Erzählens: Die Þættir der
Flateyjarbók}. Basel: Helbing
\& Lichtenhahn, 1991. (Beiträge zur nordischen Philologie, vol. 20)
\end{itemize}

\section{\versal{FOLCLORE}}

``Folclore'' é uma transliteração do termo em inglês \emph{folklore},
criado no século \versal{XIX}. Sua etimologia remete à ``sabedoria popular'' (do
inglês, \emph{folk}, ``povo'', e \emph{lore}, ``sabedoria'' ou
``conhecimento''). Foi empregado pela primeira vez em agosto de 1846,
quando o antiquarista, editor e escritor William John Thoms (1803-1885),
sob o pseudônimo de Ambrose Merton, publicou um breve artigo em forma de
carta em \emph{The Athenaeum}, um influente periódico inglês de
literatura, ciência e belas artes que circulou de 1828 a 1921.

O neologismo de Thoms logo foi adotado por correspondentes da revista e
se popularizou, difundindo-se na Inglaterra e, posteriormente, no
exterior. Antes disso, temas relacionados a folclore eram comumente
referenciados como \emph{popular antiquities} (antiguidades populares),
\emph{popular literature} (literatura popular) ou \emph{popular
mythology} (mitologia popular). A internacionalização do termo
``folclore'', portanto, adveio da língua inglesa -- em alemão, o termo
\emph{Volkskunde} já existia, mas seu uso era escasso.

Academicamente, o folclore é contemplado pela folclorística, disciplina
voltada para estudos folclóricos que possui metodologia própria,
conforme defendido por Alan Dundes. De acordo com o
autor, a metodologia empregada pela folclorística pode ser sintetizada
em dois passos fundamentais: identificação (busca por similaridades) e
interpretação (demarcação de diferenças). Teorias e métodos de outras
áreas também podem contribuir nos estudos de folclore, como história,
antropologia, sociologia, literatura.

A folclorística oferece muitas possibilidades para se investigar e
compreender a cultura, crenças, hábitos, valores e normas sociais de
determinadas comunidades. Como objeto de estudo, pode englobar diversas
produções da cultura popular, tais como gêneros discursivos (mito,
lenda, conto, provérbio, trava-língua etc.), festividades, danças,
vestimentas, arte decorativa, ritos, música, entre outros. Apesar das
várias vertentes da folclorística e respectivos enfoques, pode-se dizer
que a interdisciplinaridade e até mesmo a multidisciplinaridade são
bem-aceitas nessa área de pesquisa.

A coleção sistemática de folclore começou junto ao desenvolvimento de
paradigmas comparativos na linguística e no estudo da mitologia.
No âmbito nórdico, publicações de cunho
folclórico já despontam a partir da segunda metade do século \versal{XVI} com
obras dinamarquesas, como por exemplo: \emph{Hundredvisebogen} (\emph{Livro das
cem canções}), publicada em 1591 por Anders Sørensen Vedel, e
\emph{Aldmindelige Danske Ord-Sproge og korte Lærdomme} (\emph{Provérbios
dinamarqueses comuns e breves ensinamentos}), dois volumes publicados
entre 1682-88 por Peder Pedersen Syv.

Ainda na Dinamarca deu-se início a uma grande coleção de baladas,
conhecida como \emph{Danmarks gamle Folkeviser} (\emph{Antigas baladas da
Dinamarca}), cabeceada por Svend Hersleb Grundtvig em meados do século
\versal{XIX} e continuada por vários outros autores (entre eles, Axel Olrik) no
decorrer dos séculos~{\versal{XIX} e \versal{XX}}. Em relação a contos (popularmente
conhecidos como ``contos de fadas''), há publicações como \emph{Eventyr,
fortalte for Børn} (\emph{Aventuras, contadas para crianças}) -- publicada
entre 1835-37 por Hans Christian Andersen -- e \emph{Danske
Folkeæventyr} (\emph{Contos dinamarqueses}) -- três volumes publicados
entre 1876-84, também por Grundtvig.

Na Finlândia, Elias Lönnrot e Johan Oskar Immanuel Rancken foram
expoentes oitocentistas, sendo que Rancken se destacou por sua atuação
na coleta de folclore, enquanto Lönnrot prevaleceu na poesia e canções
folclóricas. Zacharias Topelius, influenciado por
Hans Christian Andersen, publicou contos ao longo do século \versal{XIX}, como
\emph{Sagor} (\emph{Contos}), em 1847, e a série \emph{Läsning för barn}
(\emph{Histórias para crianças}), entre 1865-96.

Sobre o folclore das ilhas Faroé, John Frederick West publicou
\emph{Faroese Folk-Tales and Legends} (\emph{Contos e lendas feroesas}) em
1980. Na Islândia, Árni Magnússon foi pioneiro no esforço de coletar
contos islandeses. Jón Árnason e Magnus Grimson,
em 1852, publicaram \emph{Íslenzk æfintýri} (\emph{Contos islandeses}) e,
entre 1862-64, dois volumes de \emph{Íslenzkar þjóðsögur og æfintýri}
(\emph{Lendas e contos islandeses}) foram publicados somente por Árnason
(idem).

Cento e dez anos depois da primeira obra de Árnason como único autor, a
pesquisadora inglesa Jacqueline Simpson publica, em 1972,
\emph{Icelandic Folktales and Legends} (\emph{Contos e lendas islandesas}).
Com base em uma seleção dos três primeiros capítulos de \emph{Íslenzkar
þjóðsögur og æfintýri}, Simpson aborda o sobrenatural no folclore
islandês em sete categorias: \emph{huldufólk} (``povo oculto'',
comumente associados aos elfos); \emph{trolls}; habitantes da água;
fantasmas; magia negra; tesouro escondido; Deus e o diabo.

Na Noruega, Peter Christen Asbjørnsen e Jørgen Engebretsen Moe coletaram
contos e, em 1841, publicaram \emph{Norske Folkeeventyr} (\emph{Contos
noruegueses}). Magnus Brostrup Landstad também contribuiu no cenário
folclórico, ao publicar \emph{Norske Folkeviser} (\emph{Baladas
norueguesas}) entre 1852-53. Em 1957, Reidar Thoralf Christiansen,
falecido professor de folclorística na Universidade de Oslo, publica
\emph{Norske folkeviser} (\emph{Contos da Noruega}), na qual subdivide
contos noruegueses em oito seções: lendas históricas; lendas sobre magia
e bruxaria; lendas sobre fantasmas, alma humana e metamorfose; lendas
sobre espíritos do mar, lagos e rios; lendas sobre espíritos do ar;
lendas sobre espíritos da floresta e da montanha; espíritos domésticos;
e contos fictícios. A exemplo de alguns espíritos, há o \emph{nøkk}
(espírito da água metamórfico) e a \emph{huldra} (espírito feminino da
floresta), ambos descritos como seres que causam atração.

Na Suécia, Arvid August Afzelius, Erik Gustaf Geijer, Gunnar Olof
Hyltén-Cavallius e George Stephens se empenharam na coleta de folclore
entre as décadas de 1830-40, resultando na obra \emph{Svenska folksagor
och äfventyr} (\emph{Lendas e contos suecos}), mas esta não logrou tanto
sucesso quanto os contos noruegueses. Em 1978,
John Lindow publica \emph{Swedish Legends and Folktales} (\emph{Lendas e
contos suecos}), no qual discorre sobre lendas, contos e memórias de
tradições orais do meio rural na Suécia dos séculos~{\versal{XIX} e \versal{XX}}.

Eldar Heide, Etunimetön Frog e Terry Gunnell são alguns defensores
atuais do estudo de folclore na Escandinavística Medieval. Segundo Heide, 
o folclore tardio pode servir tanto como material
adicional quanto ferramenta que mostra como compreender a religião
nórdica antiga e o plano cultural da literatura nórdica antiga. O autor
também ressalta que pesquisadores céticos, no que tange à utilidade do
folclore pós-medievo, são os que mais desconhecem esse tipo de abordagem.
Outros pesquisadores dedicados a estudos do
folclore nórdico, além daqueles previamente citados, são: Stephen
Mitchell, Timothy Tangherlini, Reimund Kvideland, Henning K Sehmsdorf,
Lauri Honko, Aili Nenola, Ulrika Wolf-Knuts, Arja Anna-Leena Siikala,
Satu Apo, entre outros.

Demais produções que tratam de temas folclóricos nórdicos em língua
inglesa e se encontram em domínio público: \emph{Northern mythology:
comprising the principal popular traditions and superstitions of
Scandinavia, North Germany, and the Netherlands} (\emph{Mitologia
setentrional: abrangendo as principais tradições e superstições
populares da Escandinávia, Norte da Alemanha e Holanda}), três volumes
publicados entre 1851-52 por Benjamin Thorpe; \emph{Folk-lore and
legends -- Scandinavian} (\emph{Folclore e lendas -- Escandinavos}), 1890,
por Charles John Tibbits; \emph{Scandinavian Folk-Lore -- Illustrations
of the traditional beliefs of the Northern peoples} (\emph{Folclore
escandinavo -- Exemplos das crenças tradicionais dos povos
setentrionais}), 1896, por William Alexander Craigie; \emph{Popular
Tales from the Norse} (\emph{Contos populares dos nórdicos}), 1903, por
George Webbe Dasent.

\SIG{Andressa Furlan Ferreira}

Ver também Guerra e simbolismos; Religião.

\begin{itemize}
\item \versal{DUNDES}, Alan. The Study of
Folklore in Literature and Culture: Identification and Interpretation.
\emph{The Journal of American Folklore}, vol. 78, n. 308, 1965, pp.
136-142.

\item \versal{EMRICH}, Duncan. ``Folk-Lore'': William John Thoms. \emph{California
Folklore Quarterly}, vol. 5, n. 4, 1946, pp. 355-374.

\item \versal{HEIDE}, Eldar. Why Care about Later Folklore in Old Norse Studies?
\emph{Fifteenth International Saga Conference}, 2012, pp. 87-89.

\item \versal{HULT}, Marte. Iceland. In: \versal{HAASE}, Donald (ed.). \emph{The Greenwood
Encyclopedia of Folktales and Fairy Tales}: \versal{Q-Z}. vol. 3. Greenwood
Press, 2008a, p. 835.

\item \versal{HULT}, Marte. Sweden. In: \versal{HAASE}, Donald (ed.). \emph{The Greenwood
Encyclopedia of Folktales and Fairy Tales}: \versal{Q-Z}. vol. 3. Greenwood
Press, 2008b, pp. 838-840.

\item \versal{HULT}, Marte. Swedish Finland. In: \versal{HAASE}, Donald (ed.). \emph{The
Greenwood Encyclopedia of Folktales and Fairy Tales}: \versal{Q-Z}. vol. 3.
Greenwood Press, 2008c, pp. 840-841.

\item \versal{KATAJAMÄKI}, Sakari; \versal{LUKIN}, Karina. Textual Trails from Oral to Written
Sources: An Introduction. \emph{Limited Sources, Boundless
Possibilities} -- \emph{Textual Scholarship and the Challenges of Oral and
Written Texts, A special issue of Retrospective Methods Network
Newsletter}, n. 7, 2013, pp. 08-17.
\end{itemize}


\section{\versal{FONTES PRIMÁRIAS}}

Ver Annála Uladh; Annales Fuldenses; Annales de Flodoardo de Reims;
Annales Fuldenses; Annales regni Francorum; Brevis Historia Regum Dacie;
Cogadh Gáedhel re Gallaibh; Crônica Anglo-saxônica; Crônica dos Anos
Passados; Egils saga; Encomium emmar reginae; Eyrbyggja saga;
Fagrskinna; Flateyjarbók; Gesta Hammaburgensis ecclesiae pontificum;
Gesta Normannorum Ducum; Grágás; Guta saga (História dos gotlandeses);
Haustlong; Historia de Antiquitate Regum Norwagensium; Historia
Norwegiae; Inscrições rúnicas; Íslendingabók; Landnámabók; Laxdaela
saga; Morkinskinna; Njáls saga; Poesia éddica; Poesia escáldica;
Sonatorrek; Tapeçaria de Bayeux; Tapeçaria de Oseberg; Tapeçarias de
Överhogdal; Tapeçaria de Skog.

\section{\versal{FORTIFICAÇÕES}}

As principais fortificações escandinavas foram construídas no final da
chamada Era Viking, também conhecida como Era Viking Tardia.
Esse período foi marcado pela vagarosa cristianização dos territórios, a
começar pela Dinamarca que mantinha um contato mais estreito com o Sacro
Império Romano Germânico, suscitando uma presença precoce e contínua de
missionários, ainda no início do século \versal{IX}. Entrementes, o
estabelecimento da nova crença cristã somente ocorreu a partir do
reinado de Haroldo~\versal{I} da Dinamarca, também designado Haroldo Dente Azul
(935-986), que estabeleceu o cristianismo como religião nacional,
acabando por se converter em 965. Quase todas as evidências
arqueológicas sugerem que foi durante o controverso governo do rei
cristão Haroldo que se construíram as chamadas Fortalezas Circulares
ou Fortificações Vikings em Anel.

Essas monumentais estruturas receberam o citado título por possuírem uma
planta básica perfeitamente circular que define todo sistema defensivo.
Cinco exemplares desse tipo de fortificação foram encontradas na Dinamarca:
Trelleborg, próximo a comuna de Slagelse, Fyrkat, próximo de Hobro,
Aggersborg em Løgstør, Nonnebakken em Odense e Vallø Borgring ou
Borrering, na comuna de Køge, a leste de Lellinge. No entanto, apenas as
três primeiras estão preservadas e continuam visíveis destacando-se na
paisagem como grandiosos marcos históricos. Para além dessas, em fins do
século~\versal{XX}, outras duas foram encontradas no extremo sul da atual Suécia,
ambas na região de Skåne -- província histórica da antiga Götaland, que
foi parte da Dinamarca desde o século \versal{IX} até ao século \versal{XVII} -- um em
Borgeby, a norte de Lund e outro na cidade sueca igualmente chamada de
Trelleborg. Esse último achado não foi uma grande surpresa, pois a
coincidência do seu nome sempre sugeriu aos especialistas que um
complexo fortificado podia ter sido construído ali. Recentemente, novos
estudos arqueológicos apontam a possibilidade de existirem outros
sítios, inclusive fora da Escandinávia.

Todas as fortificações descobertas possuem as mesmas características e o
mesmo formato, a semelhança dos seus planos sugere que foram construídas
por uma única autoridade organizadora, provavelmente o próprio rei
Haroldo Dente Azul. Dada a regularidade da disposição, não é
surpreendente que, no princípio, fossem concebidas para uma função
estritamente militar. Trelleborg, próximo a Slagelse, na ilha Sjælland,
foi o primeiro a ser escavado e sua pesquisa arqueológica
desenvolveu-se entre 1936 e 1941. A tradição tentou explicar a origem do
nome Trelleborg como uma fortificação construída por escravos, devido ao
significado do termo \emph{borg}, referente a castelo ou fortaleza e a
semelhança da palavra dinamarquesa \emph{træl} para designar escravo.
Porém, o vocábulo \emph{trelle} oferece uma explicação mais plausível,
uma vez que pode fazer referência aos pilares de madeira que recobriam
ambos os lados da muralha circular. Como os demais sítios arqueológicos
encontrados posteriormente seguem o mesmo padrão, tais construções
também ficaram conhecidas como Fortes Trelleborg ou Fortificações
tipo Trelleborg.

Essas não foram as primeiras formas defensivas que surgiram no
território escandinavo. Antes delas, diversos fortes menores
coroavam os cumes dos outeiros rochosos na paisagem acidentada,
principalmente da Suécia. No entanto, as fortificações do século~\versal{X}, por
suas características tipológicas e a extrema precisão dos traçados, são
inegavelmente as mais emblemáticas. A estrutura básica era composta por
uma espessa muralha circular de terra, revestida/reforçada por troncos
de madeira na face externa -- em alguns casos internamente também -- de
modo a conferir maior resistência e evitar que os invasores escalassem.
Acompanhava o perímetro circular um fosso, provavelmente repleto de
estacas afiadas e acima da barreira circular de terra havia uma paliçada
de largos troncos de carvalho com aproximadamente 7~m de altura.
Um caminho de ronda margeava a paliçada e passava por cima dos portões
de entrada na forma de pontes, permitindo fácil acesso dos guardas de um
ponto a outro da extensa circunferência terrosa. O plano
meticulosamente regular possuía quatro entradas equidistantes de onde
surgiam ruas axiais que cortavam o interior do perímetro em forma de
cruz. As ruas eram pavimentadas com madeira e dividiam o pátio interno
em quadrantes iguais; dentro de cada uma destas quartas partes,
situavam-se longos edifícios implantados no terreno de forma geométrica.

O axiomático desenho definido pelo anel de terra circular e as ruas
internas em eixos ortogonais -- ainda visíveis em alguns sítios
arqueológicos -- dividem as opiniões dos teóricos: alguns o associam
diretamente à forma da cruz sagrada, símbolo da religião cristã
recém"-instaurada; outros notam semelhanças com os fortes militares
romanos, os famosos \emph{Castra} -- plural de \emph{Castrum} --
construídos mil anos antes e que, estrategicamente, também eram
divididos em quatro seções através da imposição de dois eixos dispostos
em ângulo reto, o \emph{Cardo} -- disposição norte-sul -- e o
\emph{Decumanos} -- disposição leste-oeste. Esses eixos foram
amplamente utilizados no planejamento de algumas cidades coloniais
romanas. Quanto à possibilidade desta aproximação com a antiga
referência latina, cabe ressaltar que importantes cidades dinamarquesas
da Jylland, como Aarhus e Hedeby, ambas do século~\versal{X}, também tiveram seus
planos estruturados a partir de vias ortogonais.

O fato é que essas fortalezas em anel são os monumentos mais ambiciosos
e notáveis do final da Era Viking. A uniformidade geométrica dessas
construções reflete um alto grau de desenvolvimento técnico e
organização. No entanto, apesar do formato regular, as medidas, as
quantidades e as proporções dos elementos constituintes podiam variar, a exemplo do
anel interno da Trellerborg dinamarquesa, que possuía 136~m
de diâmetro, enquanto Fyrkat e Nonnebakken possuem, ambas, 120~m. A maior fortificação encontrada -- e provavelmente uma
das primeiras a ter sido construída -- é a de Aggersborg, no extremo
norte da Dinamarca, com 240~m de diâmetro interno e
um total de 48 edificações no interior, enquanto
Trellerborg e Fyrkat continham apenas dezesseis edificações, agrupadas
de quatro em quatro dentro de cada quadrante.

Essas edificações eram casas longas, com o madeiramento das cumeeiras
arqueados e paredes laterais curvas, por isso, vulgarmente chamadas de
``forma de barcos''. Possuíam ainda uma estrutura de
postes/colunas de madeira externas ao redor da edificação que poderiam
apoiar um pórtico ou uma galeria avarandada contornando toda construção,
como pode ser identificado nas escavações de Trelleborg. De outro modo,
tais postes/colunas externos poderiam estar inclinados sustentando a
parte superior da parede, onde se apoia o madeiramento do telhado,
gerando uma volumetria de fachada peculiar, com 30~m de
comprimento e 8~m de largura, sustentadas por mais de cinquenta
estacas inclinadas, conforme sugerem as interpretações arqueológicas de
Fyrkat. O interior das casas era dividido em três compartimentos: uma
grande sala no centro e cômodos menores nas extremidades.

Os objetos arqueológicos encontrados nas escavações demonstram que,
apesar de as edificações possuírem praticamente a mesma tipologia
arquitetônica, elas cumpriam funções diferentes, afinal os seus
habitantes não se limitavam as atividades guerreiras -- atrás das
muralhas viviam ourives, carpinteiros e ferreiros e alguns edifícios
eram utilizados como celeiros e estábulos. As pesquisas também
descobriram cemitérios fora do recinto circular. O cemitério de
Trelleborg, localizado no entorno imediato da muralha, possui
135 sepulturas, contendo os restos de pelo menos
157 indivíduos. Em Fyrkat existe uma estrada ligando a
fortificação ao cemitério; embora careça de uma delimitação
precisa, 30 sepulturas já foram encontradas e escavadas no local.

Implantadas em locais considerados estratégicos naquela época, a
execução dessas fortificações deve ter requerido técnicas de
planificação e engenharia consideráveis, além de ter consumido muitos
recursos. Para construir Fyrkat, por exemplo, o local precisou ser
nivelado, alargado e também foi necessário transportar mais de dez mil
metros cúbicos de terra para erigir a muralha. Trelleborg exigiu o
trabalho de centenas de homens durante seis anos e utilizou mais de oito
mil árvores de carvalho. Trelleborg possui mais uma distinção:
externamente à barreira circular, existiam outras quinze edificações que
acompanhavam a curvatura da muralha e onde, provavelmente, viviam as
famílias dos guerreiros.

As motivações que levaram à construção desses extraordinários monumentos
também foi matéria de polêmica desde a escavação dos seus primeiros
vestígios na Dinamarca. Atualmente, a explicação mais aceita é que esses
fortes serviram como verdadeiros centros do poder real, a partir dos
quais as forças armadas de guerreiros podiam ser rapidamente enviadas
para controlar regiões circundantes, reprimindo possíveis
revoltas/insurreições e fazendo respeitar a autoridade do rei. Ademais,
serviram como locais de treinamento das estratégias bélicas e centros
regionais de coleta de impostos em produtos à população rural, neste
caso desempenhando funções de tesourarias onde a riqueza real acumulada
podia manter-se a salvo e ser transformada em ornamentos preciosos para
o rei e sua corte nas oficinas ali existentes. Apesar de claramente
projetadas para a guerra, essas fortificações acabaram servindo como
importantes núcleos comunitários locais durante as épocas de paz e, enfim,
majestosos símbolos do poder de Haroldo Dente Azul que exerceram um
importante papel na unificação do reino. Não obstante, as mudanças
políticas em finais do século, iniciada pela revolta do próprio filho do
rei Haroldo, o rebelde Sueno, conhecido pelo epíteto de ``Barba
Fendida'' ou ``Barba Bifurcada'' (960-1014), tornaram as caríssimas
estruturas fortificadas desnecessárias, deixando-as cair em triste e
célere abandono.

\SIG{João Batista da Silva Porto Junior}

Ver também Dinamarca da Era Viking; Guerra e técnicas de combate.

\begin{itemize}
\item \versal{ANDRÉN}, Anders. Places, Monuments, and Objects: The Past in Ancient
Scandinavia. \emph{Scandinavian Studies}, vol. 85, n. 3, 2013, pp.
267-281.

\item \versal{CAMPBELL}, James Graham. \emph{Grandes Civilizações do Passado:} \emph{Os
Vikings}. São Paulo: Editora Folio, 2006.

\item \versal{LANGER}, Johnni (org.). \emph{Dicionário de Mitologia Nórdica: Símbolos,
Mitos e Ritos}. São Paulo: Editora Hedra, 2015.

\item \versal{LARSEN}, Anne-Christine. \emph{The Trelleborg-Type Fortresses: A
Comparative Analysis of the Danish Viking Age Ring Fortresses}. The
Heritage Agency of Denmark, 2012.

\item \versal{RAFFIELD}, Ben. Antiquarians, Archeologists and Viking Fortifications.
\emph{Journal od The North Atlantic,} n. 20, 2013, pp. 01-29.

\item \versal{ROESDAHL}, Else \emph{et al}. \emph{Aggersborg: The Viking-Age Settlement
and Fortress}. Aarhus: Aarhus University Press, 2014.
\end{itemize}

\section{\versal{FRANÇA NA ERA VIKING}}

Durante a Era Viking (séculos~{\versal{VIII}-\versal{XI}}), o território que hoje
corresponde à França viu suas fronteiras serem constantemente
redefinidas. Ao fortalecimento do Reino dos Francos pela dinastia dos
carolíngios, seguiu-se uma ``restauração'' do Império Romano do Ocidente,
conduzida por Carlos Magno em 800. Logo depois, a unidade imperial foi
dissolvida no Tratado de Verdun (843), de onde surgiu a Frância
Ocidental (\emph{Francia occidentalis}). Esse embrião da atual França
também não ficou incólume à fragmentação territorial: a partir dos
séculos~{\versal{IX}-\versal{X}}, o poder central enfraqueceu, o que contribuiu para o
aparecimento de diversas unidades políticas (principados), que passaram
a agir de forma independente.

Nesse processo político secular, resumido ao extremo nas linhas acima, a
presença escandinava influenciou a dinâmica da região, esta que,
igualmente, os influenciaria. Do ponto de vista econômico, a clássica
tese de Henri Pirenne -- para quem o Império Carolíngio teria (após a
expansão árabe no Mediterrâneo) uma economia carente de recursos e
limitadas ligações mercantis, com uma \emph{villa} ``fechada'' e
autárcica -- não é aceita atualmente. A conquista da Saxônia
possibilitou aos francos novas e lucrativas rotas comerciais no fim do
século~\versal{VIII}. Ainda nesse período, a reforma monetária de Carlos Magno
estabeleceu o monopólio real de cunhagem e introduziu os \emph{novi
denarii}, moeda de prata com boa aceitação, inclusive no mundo nórdico.

Existe um debate sobre o impacto das invasões escandinavas na economia
carolíngia do século \versal{IX}. Algumas interpretações apontam uma época de
crise causada pela escassez sobretudo de moedas, que tinham sido
roubadas pelos invasores ou oferecidas a eles como tributo. Essa visão
de ``decadência'' consta nas fontes daquela época, para as quais seria
preferível pegar em armas em vez de recorrer a esse ``vergonhoso''
artifício, que teria tido efeitos desastrosos. O historiador francês
Simon Coupland afirma que o tributo funcionava relativamente bem para
repelir novos ataques escandinavos, ao contrário do que se costuma
imaginar. Além disso, as informações de que dispomos demonstram que os
francos poderiam pagar tais quantias, sem ocasionar qualquer crise
monetária duradoura entre a população. Outros acadêmicos ainda
argumentam que os saques vikings podem ter sido até benéficos à
economia, já que recolocaram em circulação os tesouros dos edifícios
eclesiásticos (``desentesouramento'').

Uma característica da Frância Ocidental na Era Viking era a difusão dos
laços de vassalagem, largamente propagados desde os primeiros
carolíngios. Há uma discussão sobre o período no qual as relações
feudo-vassálicas foram difundidas na Normandia, com destaque
para uma história (real ou fictícia) ocorrida quando da concessão -- do
monarca Carlos, o Simples, ao chefe viking Rollo (911) -- das terras ao
redor de Ruão (praticamente a Alta Normandia de hoje). Segundo o clérigo
Dudo de Saint-Quentin (c. 965-1026), o viking prestou ``homenagem'' ao
rei franco e seguiu o ritual de colocar as mãos entre as do futuro
senhor. Com um tom lendário e anedótico, ele ainda conta que Rollo,
quando soube que o próximo passo seria ajoelhar e beijar o pé do
soberano carolíngio, recusou, alegando que jamais ficaria de joelhos e
beijaria o pé de um homem. Em seu lugar, o líder enviou um confrade que,
para evitar a genuflexão, ergueu os pés do rei à altura de sua boca.

Outro aspecto da Frância dessa época era o desenvolvimento da cavalaria,
que tinha uma importância crescente desde o período carolíngio. Ainda
que o estribo (elemento asiático que chegou ao Ocidente no século \versal{VII})
tenha possibilitado ao cavaleiro maior firmeza em seu cavalo, não
podemos acreditar numa ``revolução'' militar causada pelo objeto. Com o
objetivo de penetrar o interior da Frância, os próprios vikings se
transformaram em cavaleiros, montando em animais que tinham trazido ou
roubado na região. De modo geral, o equipamento de um cavaleiro
carolíngio dos séculos~{\versal{IX}-\versal{X}} consistia numa lança (acompanhada, por
vezes, de uma espada longa), escudo redondo de madeira, capacete e
armadura com escamas de ferro. Aliás, a arqueologia comprovou que as
armas foram os produtos francos mais exportados para o mundo escandinavo
alto-medieval.

De fato, o principal poderio militar carolíngio -- responsável direto
pela expansão territorial -- era o exército, e não a marinha, cuja
preocupação era basicamente defensiva. Embora a força naval tenha sido
empregada contra os muçulmanos no Mediterrâneo, no mar do Norte as
coisas eram diferentes. Ali, os francos construíram diversas defesas no
litoral, como na Frísia, território que, na definição de Janet L.
Nelson, era ``o calcanhar de Aquiles do Império''. Ao longo do século
\versal{IX}, vários reides vikings atingiram essa região, mas nenhuma fonte cita
um verdadeiro contra-ataque naval lançado por um governante carolíngio.
De toda forma, embarcações foram construídas no tempo de Luís, o Piedoso
(814-840), muitas das quais direcionadas à defesa de Dorestad,
importante entreposto comercial.

No plano teórico, a sociedade francesa da Era Viking estava organizada
em três ordens. Com efeito, o mais antigo escrito sobre esse esquema
trifuncional aparece com o carolíngio Aimon (†865), mestre da escola de
Auxerre, que dividiu o corpo social em três ordens: sacerdotes
(clérigos), guerreiros e produtores. Essa perspectiva tripartite
propagava um modelo que a sociedade deveria ter de si própria, porém não
era uma simples representação da realidade. Tratava-se,
efetivamente, de um projeto e uma construção ideológica elaborada
pela Igreja para justificar a supremacia dos eclesiásticos, os
detentores das superiores armas espirituais. Em 891, por exemplo, os
religiosos receberam uma parte do espólio adquirido dos vikings em
Saint-Omer, visto que tinham ajudado ao rezar pelo triunfo do exército
franco.

A Frância Ocidental caracterizava-se por uma rica diversidade
étnico-cultural. O cronista Flodoardo de Reims (c. 894-966) era um dos
que distinguiam, naquele território, ``os francos, os borgonheses, os
aquitânios, os bretões, os normandos, os homens da Flandres, os da
região gótica, da fronteira com a Espanha''. Considerando as fronteiras
hodiernas da França, o número de habitantes girava em torno de 5 milhões
no ano 800, subindo para 6,5 no ano 1000 e 7,7 em 1100. O impacto da
chegada dos escandinavos nesse crescimento populacional, especialmente
na Normandia, já foi muito debatido pelos pesquisadores. Em um artigo
clássico, Lucien Musset afirma que essa região assistiu a um \emph{boom}
demográfico a partir do final do século~\versal{X}, com destaque no processo para
a contribuição dos escandinavos, sobretudo dos dinamarqueses. Os dados
arqueológicos recentes indicam, contudo, que a presença nórdica não pode
ser apontada como a razão principal para o aumento demográfico da
Normandia. Houve, sem dúvida, um crescimento populacional, com início
ainda no século~\versal{VIII}, mas ele é resultado de um fenômeno interno,
associado ao fim das grandes epidemias, à segurança interna e ao
crescimento agrícola.

A presença dos vikings em solo francês também reaqueceu a prática da
escravidão, ao menos na Normandia. Em Ruão, os normandos organizaram um
importante mercado de escravos, cuja ``mercadoria'' principal era
composta por irlandeses e frísios. Até o século~\versal{XI}, conforme Anne
Nissen-Jaubert, a escravidão era uma característica que distinguia a
Normandia do restante da Frância, onde predominava o regime de servidão.
Alguns eclesiásticos daquela época sentiam-se incomodados com isso: o
itálico Lanfranco (c. 1010-1089), abade de Saint-Étienne de Caen e
depois arcebispo da Cantuária, chegou até a aconselhar o duque Guilherme~\versal{I} 
a abandonar o tráfico de escravos.

Ao longo da Era Viking, a Frância viu o surgimento de muitas
fortificações. Ainda que Carlos, o Calvo, tenha proibido (864) a
construção de qualquer fortaleza sem a sua autorização, a população logo
desrespeitou essa interdição. Já no fim do século~\versal{IX}, muitas defesas
foram erguidas por bispos e abades para conter os vikings; eram simples
obras de madeira, edificadas entre o Sena e o Reno. Com a seguida
descentralização política a partir do século~\versal{X}, a região teve suas
defesas cada vez mais ``privatizadas'', o que levou à construção de
castelos, base do poderio militar feudal. Na primeira metade do século~\versal{XI}, 
surgiram 10 castelos em Anjou e 36 em Charente; seu objetivo não era
mais proteger das ameaças escandinavas, mas dos próprios senhores
vizinhos.


\SIG{Guilherme Queiroz de Souza}

Ver também Era Viking; Normandia; Rollo; Viking; Vikings na França;

\begin{itemize}
\item \versal{COUPLAND}, Simon. The Carolingian Army and the Struggle against the
Vikings. \emph{Viator}, vol. 35, 2004, pp. 49-70

\item \versal{COUPLAND}, Simon. The Frankish tribute payments to the Vikings and their
consequences. \emph{Francia}, vol. 26, 1999, pp. 57-75.

\versal{D'HAENENS}, Albert. \emph{As Invasões Normandas: Uma Catástrofe?} São
Paulo: Perspectiva, 1997.

\item \versal{DUBY}, Georges. \emph{A Idade Média na França} \emph{(987-1460). De Hugo
Capeto a Joana D'Arc}. Rio de Janeiro: Jorge Zahar Ed., 1992.

\item \versal{MCKITTERICK}, Rosamond. \emph{The Frankish Kingdoms Under the
Carolingians, 751-987.} London: Longman, 1983\emph{.}

\item \versal{MUSSET}, Lucien. Essai sur le peuplement de la Normandie
(\versal{VI}\textsuperscript{e}-\versal{XII}\textsuperscript{e} siècles). In: \emph{Actes
des congrès de la Société d'archéologie médiévale}, vol. 2, n. 1, 1989,
pp. 97-102.

\item \versal{NELSON}, Janet L. The Frankish Empire. In: \versal{SAWYER}, Peter (ed.). \emph{The
Oxford Illustrated History of the Vikings.} Oxford-New York: Oxford
University Press, 1997, pp. 19-47.

\item \versal{NISSEN JAUBERT}, Anne. Some aspects of Viking research in France.
\emph{Acta Archaeologica}, vol. 71, 2000, pp. 159-169.

\item \versal{PRICE}, Neil. The Historical Background: France in the Viking Age. In:
\emph{The Vikings in Brittany}. London: Viking Society for Northern
Research, University College London, 1989, pp. 21-54.
\end{itemize}

\section{\versal{FREYDIS EIRÍKSDÓTTIR}}

A \emph{Saga dos Groenlandeses} e a \emph{Saga de Eiríckr} apresentam, ao
longo de suas narrativas, algumas personagens emblemáticas; entre elas, a 
figura de Freydís Eiríksdóttir merece destaque.

Na \emph{Saga dos Groenlandeses} ela é apresentada como uma mulher
ardilosa e traiçoeira, enquanto que na \emph{Saga de Eiríckr} ela é
descrita como excepcionalmente corajosa e forte, entrando em luta
corporal com nativos de Vínland. É importante salientar que ambas as
sagas foram compostas cerca de duzentos anos após os eventos que estão
descritos no seu enredo e podem ser consideradas como evidências
históricas. Muitas dessas narrativas que remetem
a fatos ocorridos nos séculos~{\versal{IX}, \versal{X} e \versal{XI}} contêm vários elementos
históricos, mas também muitos ficcionais, pois não havia documentos
escritos que comprovassem, por exemplo, as viagens e a descoberta de
novas terras. No caso específico de elementos apresentados na \emph{Saga de
Eiríckr}, a arqueologia comprova que no século \versal{XI} existiu um assentamento
nórdico. No sítio arqueológico de L'Anse aux Meadows, na Terra Nova, Canadá, pode-se confirmar
que os nórdicos foram os primeiros europeus a chegar na América do
Norte.

No desenrolar da \emph{Saga de Eiríckr}, apresenta-se ao leitor como
Freydis acompanhou o marido a Vínland em uma expedição liderada por
Þorfinnur Karlsefni e sua esposa Guðríður Þorbjarnardóttir. Quando
chegaram a Vínland, logo tiveram os primeiros contatos com os nativos e
iniciaram uma troca de produtos entre si. Durante uma dessas trocas de
produtos, um touro de propriedade de Þorfinnur escapou e correu furioso
em direção aos nativos. Como estes nunca tinham visto tal animal,
entenderam isso como uma agressão e nesse momento ocorreu uma luta na
qual os nativos tinham a vantagem e Þorfinnur e seus comandados agiram com certa covardia.
Indignada com essa atitude dos homens, Freydis os repreendeu por sua
covardia e tentou encorajá-los e incitá-los a tomar uma posição, para
enfrentarem os nativos. Mas o encorajamento de Freydis não surtiu efeito
e todos correram para a floresta. Ela seguiu os homens, mas como ela
estava grávida, logo começou a diminuir a velocidade e não conseguiu
acompanhá-los. Enquanto perseguiram os homens, os nativos encontraram e
a cercaram. Sem medo, ela agarrou a espada que estava ao lado de um
homem, abriu sua túnica para expor um dos seios e parte de seu ventre
que estava proeminente e empunhou a espada. Este gesto ousado e
inesperado parece ter assustado os nativos que, aterrorizados, fugiram,
pois ao se depararem com uma mulher de cabelos compridos e ruivos com o
seio e parte da barriga à mostra acreditaram estar diante de algo mau,
que aparentemente poderia ser mau presságio.

As descrições dos atos de coragem e ousadia de Freydis na saga vão além
desse confronto com os nativos de Vínland. Freydis, assim como outras
mulheres da Era Viking, encontrou na fofoca e nas intrigas transmitidas
boca a boca uma poderosa arma de poder para provocar conflitos
dentro da comunidade e, assim, conseguir atingir seus objetivos, fossem
eles quais fossem, e também para incitar os homens a lutar pelas causas
que ela julgava importantes.

Freydis é uma mulher descrita como valente e destemida, mas não pode ser
vista como um modelo feminino da Era Viking, pois as mulheres, apesar
de possuírem visibilidade e mobilidade dentro dessa sociedade e poderem
contar com leis de proteção, ainda viviam sob a tutela e o controle
masculino. Podemos analisar Freydis como
uma personagem singular que catalisa dois sentimentos antagônicos: o
temor e a fascinação, pois as características viris que possuia, e que faziam-se fascinantes ao serem vistas, pelo homens, em uma mulher, também eram o que a tornava uma mulher temível e que devia ser controlada. 

Ao enfrentar os nativos com parte do tronco desnudo, empunhando a espada
arrancada das mãos de um morto, ela deu provas da sua coragem frente a
uma situação desesperadora, mostrando que a luta e o enfrentamento do
inimigo deviam ser maiores que o medo do desconhecido.

\SIG{Luciana de Campos}

Ver também Aud; Família; Gudrid; Mulheres; Sociedade; Sexo e
sexualidade.

\begin{itemize}
\item \versal{HOLCOMB}, Kendall M. \emph{Pulling the Strings: The Influential Power of
Women in Viking Age Iceland}. Western Oregon University, Studen Theses,
2015.

\item \versal{JESCH}, Judith. \emph{Women in the Viking Age}. London: Boydell \& Brewer
Ltd, 1999.

\item \versal{JOCHENS}, Jenny. \emph{Women in Old Norse
Society}. Ithaca: Cornell University Press,
1995.
\end{itemize}



\section{\versal{FUNERAIS E ENTERROS}}

Sabe-se pouco sobre as cerimônias fúnebres dos nórdicos antigos,
tendo-se como único relato extenso a descrição feita pelo árabe Ibn
Fadlan, que assistiu ao funeral de um líder viking nas margens do Volga,
em 922. De resto, dispomos apenas de vestígios sob a forma de oferendas,
sepulturas e restos mortais, humanos e animais, uma janela
direta para os costumes fúnebres, mas equivalente ao produto final de
cerimônias e gestos rituais que desconhecemos. Há também descrições
romanceadas distantes do período em que as práticas estavam em voga,
como nas \emph{Eddas} ou nas sagas islandesas, e que tanto podem conter dados
verídicos como enfabulamentos tardios.

É exemplo deste último tipo de fontes a descrição do funeral de Balder
na \emph{Edda} de Snorri Sturluson, em Gylfaginning 49, quando o corpo do filho
de Odin é depositado numa embarcação posta a navegar no mar.
Acompanham-no a sua mulher Nanna, que falece de desgosto no momento, o
anel mágico Draupnir e o cavalo de Balder, em uma simplificação
romanceada de ritos fúnebres antigos, ora por pudor da parte de um autor
cristão que tinha interesse na cultura antiga do seu
país, ora porque o tempo apagou a memória dos detalhes rituais e deixou
apenas uma visão muito geral e já algo distorcida. Podemos fazer a mesma
análise para as descrições breves e por vezes padronizadas dos funerais
pré-cristãos nas \emph{Sagas dos Islandeses}, também elas distantes do período
pagão, religiosa e cronologicamente.

Se compararmos o que Snorri diz a respeito da cerimónia fúnebre de
Balder com o relato de Ibn Fadlan, deparamo-nos com semelhanças. Por
exemplo, a cremação de Nanna ao lado do seu marido tem paralelo com o
sacrifício de uma escrava testemunhado pelo viajante árabe: pelo que
quando o autor islandês diz que a deusa morreu de desgosto, isso pode
ser um embelezamento literário que esconde uma prática histórica mais
violenta. O uso do fogo é outro elemento comum aos dois textos,
embora na versão de Ibn Fadlan ele ocorra em terra, para onde a embarcação foi
previamente arrastada, e há ainda a referência a oferendas fúnebres. No
caso de Balder, um anel dourado que produz outros idênticos, e ainda o
cavalo do deus e o respectivo arnês. No relato árabe a lista é
bastante mais extensa e inclui também um cão, duas vacas, um galo e uma
galinha, para além de comida, armas, uma cama com almofadas de seda e
roupas requintadas.

A descrição feita por Ibn Fadlan contém depois detalhes festivos e
rituais que estão ausentes das fontes nórdicas e que não podem ser
descortinados por via do registro arqueológico. Assim, por exemplo, o
texto árabe fala dos dias de celebração que antecederam o funeral
propriamente dito e que incluíam atos sexuais, o enterro temporário do
morto enquanto eram preparadas as roupas que ele iria usar na cremação e
ainda gestos rituais, como a elevação da escrava acima de uma estrutura
de madeira, momento em que ela diz ver o seu senhor e o ``paraíso". Este
detalhe é curioso, porque remete-nos para as referências a práticas
mágicas nórdicas, onde é comum o uso de plataformas elevadas para ver o
futuro ou contactar o mundo espiritual. Pode ser um caso de cruzamento
ritual, com a inclusão de elementos de Seidr, mas não sabemos até que
ponto isso seria comum na Escandinávia antiga. E o mesmo é verdade para
o recurso a uma bebida por ventura drogada que é oferecida à escrava
momentos antes de ser morta. Seria assim com todos os sacrifícios
humanos?

Embora não seja expectável encontrar vestígios arqueológicos que
coincidam em pleno com as descrições textuais, há alguns indícios que
dão substância física à palavra escrita. O uso de uma embarcação está
amplamente registrado, incluindo a sua deslocação para terra e a
posterior construção de uma colina artificial sobre o barco, mesmo sem
cremação, assim como a inclusão de um poste no topo, por vezes o mastro
do próprio navio, e ainda a prática de sacrifícios animais e humanos.
Não sabemos se envolviam gestos rituais idênticos aos descritos por Ibn
Fadlan, mas a variedade no tipo de oferendas e vítimas sacrificiais
sugere que havia diferenças mesmo quando se usavam modelos fúnebres
próximos. Por exemplo, se o relato árabe refere o sacrifício de dois
cavalos que correm em torno do barco antes de serem mortos, a sepultura
de Oseberg continha os restos de vinte animais idênticos; se o
procedimento ritual foi o mesmo, terá obrigado a uma logística algo
diferente, nem que fosse pela maior monumentalidade. Ou então teremos
que admitir que eles foram sacrificados de outra forma, não se sabe
qual.

Ausente do texto de Ibn Fadlan está o recurso a uma procissão fúnebre,
talvez por não fazer parte dos costumes do grupo de vikings que ele
encontrou no Volga ou porque as condições locais não o permitiam ou não
justificavam. Mas isso não quer dizer que os funerais nórdicos antigos
não incluíssem cortejos, principalmente no caso de cerimônias em honra
de mortos mais ilustres, como uma ou ambas as mulheres de Oseberg. É
dessa sepultura que provém uma tapeçaria que parece retratar uma
procissão com vários veículos, cavaleiros e figurantes. Embora não se
possa dizer que se trata de uma representação do cortejo fúnebre, não é
impossível que parte das cerimônias incluíssem eventos do gênero, com os
corpos e as oferendas a serem exibidos e transportados para o túmulo de
forma memorável, num espetáculo que servia para honrar o defunto
e também para veicular o seu estatuto social.

A sepultura de Oseberg fornece ainda outra pista para o cerimonial
fúnebre dos nórdicos antigos, uma vez que a colina que foi originalmente
erguida sobre o barco parece ter coberto apenas parte da embarcação,
deixando-a parcialmente visível e acessível, não se sabe se para meras visitas ao
túmulo para deixar oferendas, ou se para a realização de cerimônias,
talvez até gestos rituais teatralizados. A câmara fúnebre Bj. 834, em Birka, por exemplo, continha uma
lança que foi cravada numa das paredes do túmulo, algo cujo significado
é desconhecido, mas que não teria sido feito sem mais, sendo plausível
pensar que o ato de fixar a arma no interior do túmulo foi parte de uma
cerimônia cujos contornos e sentido desconhecemos. Encontramos o mesmo
indício de gestos rituais incertos em Bogla, na Suécia, onde ossadas e
objetos teriam sido queimados, quebrados e espalhados, não
necessariamente em resultado de pilhagens. Até porque há casos
arqueologicamente conhecidos de túmulos que teriam sido abertos, mas onde
o conteúdo, longe de ser destruído ou vandalizado, parece ter sido
apenas remexido com algum cuidado. Também nisso há indícios de possíveis
cerimônias aos ou com os mortos.

Outro aspecto relevante ao cerimonial tumular é a
reutilização dos espaços fúnebres, tanto pela inclusão de novos corpos
como pela construção de novos túmulos sobre outros mais antigos. Um caso
emblemático é o de uma sepultura em Kaupang, um posto de comércio no sul
da Noruega, onde uma campa do século~\versal{IX} foi sobreposta por um navio cuja
quilha se alinhava exatamente sobre o defunto original, sugerindo
intencionalidade. Dentro da embarcação foram colocados os corpos de um
homem, uma criança e duas mulheres, uma delas sentada na popa com uma
cabeça de cão no colo e talvez o leme nas mãos, como se fosse conduzir
o barco, o que não terá sido por acaso nem talvez desprovido de
encenação ritual. O mesmo pode dizer-se a respeito dos corpos de
mulheres que, no cemitério de Birka, foram todos orientados de forma a
olharem para a povoação, como que a guardá-la. Uma vez mais, é razoável
assumir que tudo isso envolvesse cerimônias e gestos rituais que hoje
desconhecemos.

A ausência de fontes e de conhecimento afeta também o
significado fúnebre dos sacrifícios animais, que não sabemos se se
justificavam apenas pelo desejo de levá-los para o além, se por um
sentido simbólico maior, uma preferência pessoal ou uma combinação de
vários motivos. E essa incerteza tem dado azo a diferentes teorias, como
a que propõe para o cavalo um significado de movimento ou transporte,
até mesmo uma natureza liminal e, portanto, de transição de um mundo
para outro, dando-lhe no contexto fúnebre um uso próximo ao da
vida cotidiana. Para o cão, surgem como hipóteses maiores a associação
ao mundo dos mortos e à função de guias ou guardiões, algo que talvez
ajude a explicar o já referido caso da mulher do leme na sepultura de
Kaupang. E para as aves, nomeadamente galos e galinhas, há sugestões
relacionadas com as ideias de anúncio, começo ou espaços mitológicos,
algo que tem alguma base nas \emph{Eddas}, de que a estrofe 42 do Völuspá é
exemplo. Mas isso é a nossa leitura simbólica moderna, não havendo
garantias de que os nórdicos partilhavam das nossas concepções e
mundividências.

A mesma incerteza afeta o papel exato das vítimas humanas, supondo-se
que tinham por objetivo acompanhar o defunto, fosse num papel de
auxiliar, cônjuge ou parceiro, mas também não é impossível que se
pensasse na função de guardas do túmulo. Por exemplo, numa sepultura na
ilha de Man, à colina fúnebre que encerrava o corpo de um jovem e
oferendas sobrepôs-se uma mulher decapitada e uma nova camada de terra,
o que pode talvez sugerir que o propósito não seria simplesmente o de
acompanhar o morto, caso em que seria de se esperar que ela tivesse sido
enterrada com ele, mas o de guardar a campa, motivo pelo qual a mulher
foi sacrificada já fora do espaço tumular propriamente dito.

Por fim, se esses dados apontam para cerimônias, gestos rituais,
sentidos e cargas simbólicas que conhecemos parcialmente,
há outro elemento que está ausente do registro arqueológico e que figura
apenas em fonte escrita, em particular no relato de Ibn Fadlan: a palavra
falada. Provavelmente o cerimonial fúnebre incluía cantigas, orações,
poemas ou discursos, tradicionais ou talvez \emph{ad hoc}, mas, a menos
que surjam novas fontes, esse é um elemento que está infeliz e
irrevogavelmente perdido.

\SIG{Hélio Pires}

Ver também Era Viking, Ibn Fadlan, Religião; Sepultamentos; Simbolismo
animal.

\begin{itemize}
\item \versal{BENNETT}, Lisa. Burial practices as sites of cultural memory in the
\emph{Íslendingasögur}. \emph{Viking and Medieval Scandinavia}, vol. 10, 2014,
pp. 27-52.

\item \versal{GRÄSLUND}, Anne-Sofie. Wolves, serpentes, and birds: their symbolic
meaning in Old Norse belief. In: \versal{ANDRÉN}, Anders; \versal{JENNBERT}, Kristina;
\versal{RAUDVERE}, Catharina (eds.). \emph{Old Norse Religion in long-term
perspectives}. Lund: Nordic Academic Press, 2006, pp. 124-129.

\item \versal{LOUMAND}, Ulla. The horse and its role in Icelandic burial practices,
mythology, and society. In: \versal{ANDRÉN}, Anders; \versal{JENNBERT}, Kristina;
\versal{RAUDVERE}, Catharina (eds.). \emph{Old Norse Religion in long-term
perspectives}. Lund: Nordic Academic Press, 2006, pp. 130-134.

\item \versal{LUNDE}, Paul; \versal{STONE}, Caroline. \emph{Ibn Fadlan. Ibn Fadlan and the Land
of Darkness: Arab Travelers in the Far North}. London: Penguin, 2012.

\item \versal{PRICE}, Neil. Dying and the Dead. In: \versal{BRINK}, Stefan; \versal{PRICE}, Neil (eds.).
\emph{The Viking World}. London/New York: Routledge, 2010, pp. 257-273.
\end{itemize}

\chapter{G \textarn{g} \textarc{g} \textart{g}}
\section{GAMLA UPPSALA}

O sítio de Gamla Uppsala (velha Uppsala) é um cemitério real, um
antigo centro cúltico pagão e a residência dos primeiros reis suecos,
situado a poucos quilômetros ao norte da moderna Uppsala, no centro da
Suécia. Três grandes montes de sepultamento de Gamla Uppsala são
tradicionalmente associados aos reis \emph{svear} Aun (século~\versal{VI}), Egil e
Adils da semilendária dinastia dos ynglingos. Dois dos três montes
foram escavados e datados dos séculos~{\versal{V} e \versal{VI}} d.C., e
continham restos cremados de homens de alta posição social. Dezenas de
pequenos montículos funerários rodeiam os grandes montes. Um monte
achatado ao leste das grandes elevações, conhecido como monte das
assembleias, foi utilizado para finalidades cerimoniais.

Antigos cronistas escreveram que Gamla Uppsala tinha sido a residência
de deuses, como Freyr ou Odin. O historiador dinamarquês Saxo
Grammaticus escreveu afirmando que Freyr residiria pelas cercanias e
costumeiramente lhe seriam feitos sacrifícios humanos. O cronista Adão
de Bremen conferiu alguns detalhes destas cerimônias, realizadas a cada
nove anos e envolvendo noves vítimas masculinas de diversas espécies.

A área de Gamla Uppsala foi conhecida desde a Idade do Bronze, mas sua
ocupação foi mais substancial durante o primeiro milênio depois de
Cristo. Apesar de ser escavada arqueologicamente há dois séculos, ainda
se conhece muito pouco do local, visto que os estudos se concentraram
apenas nos grandes monumentos, sendo as pequenas fazendas do seu entorno
muito pouco conhecidas até hoje.

Uppsala situa-se ao extreme norte de Mälardalen, uma área agrícola muito
rica da Suécia. Pesquisas efetuadas por Per Frölund nos assentamentos
agrários da região permitiram detectar mudanças no uso da terra e nos
diferentes assentamentos. No início da Idade do Ferro os principais
cultivos eram os de cereais, especialmente cevada. No período das
migrações houve intensa atividade pecuária, e posteriormente um aumento
das estruturas de poder, recolhendo e distribuindo bens.

Por quase 70 anos a datação dos grandes túmulos de Uppsala foi dominada
pelo referencial de Sune Lindquist, que afirmava que eles datavam do
período das migrações. Novos estudos de John Ljungkvist demonstraram,
porém, que eles datam do início do Período Vendel, sendo bem mais
recentes do que se supunha. O maior túmulo era o centro de todo o sítio,
situado ao norte da igreja de Uppsala. Foi possível distinguir duas
zonas em Gamla Uppsala: um setor elitizado, que produziu as largas
edificações, salões e terraços, além dos centros de produções artesanais
conectados ao salão real. Segundo T. Douglas Price é possível que nesta
área estivesse situado um templo contendo estátuas dos deuses nórdicos. O
montículo real é representado por um salão com 50~m de comprimento,
situado sobre o terraço artificial. Esta construção é uma das mais
largas de toda a Escandinávia anterior à cristianização. As outras
edificações situadas no terraço sul continham produtos artesanais como
contas e joias, além de uma sequência de casas construídas entre os
séculos~{\versal{VI} e \versal{VIII} d.C.}

Nos montes artificiais de Gamla Uppsala foram encontrados vestígios
humanos, mas não se conhece a identificação dos mesmos. Os corpos foram
cremados antes do enterro e poucos fragmentos foram encontrados. No
monte do leste foram recuperados um elmo e diversos fragmentos de
objetos de ouro. Os ossos de um falcão, provavelmente utilizado em
falcoaria, foram encontrados no monte oeste. Esses fragmentos sugerem que
as pessoas encontradas nestes locais eram membros da aristocracia local.
Os arqueólogos vêm considerando Gamla Uppsala uma das mais importantes
``localidades centrais'' da Escandinávia da Era Viking. Essas
localidades centrais seriam espaços que combinariam ao mesmo tempo
funções políticas, religiosas e econômicas. Como Uppåkra, ela emergiu
durante a Idade do Ferro germânica como um local de culto, uma herdade de
conexões reais e uma grande vila. Gamla Uppsala foi lar de atividades
políticas, militares, econômicas e religiosas que foram direcionadas
para uma poderosa elite da sociedade nórdica. Os túmulos e construções
monumentais em forma de salões, terraços e talvez um grande templo,
testemunham a autoridade dessa elite e a imensa força de trabalho que
ela controlava, caracterizando o centro de um enorme poder. Em 1150 d.C. a área foi
concedida ao bispo de Uppsala e a primeira igreja da arquidiocese foi
construída depois de 1160. Um incêndio nessa igreja fez com que sua
influência perdesse prestígio e outra foi construída em uma nova cidade a poucos
quilômetros da atual Uppsala.

\SIG{Johnni Langer}

Ver também Suécia da Era Viking.

\begin{itemize}
\item \versal{FROLUND}, Per. Gamla Uppsala under äldre järnålde. \emph{Arkeologi E4
Uppland} -- studier 4, 2007.

\item \versal{HAYWOOD}, John. Gamla Uppsala. In: \emph{Encyclopaedia of the Viking
Age}. London: Thames and Hudson, 2000, pp. 195-196.

\item \versal{LJUNGKVIST}, John. Uppsala högars datering. \emph{Fornvännen}, vol. 100, n. 4,
2005, pp. 245-259.

\item \versal{PRICE}, T. Douglas. Gamla Uppsala. In: \emph{Ancient Scandinavia: an
archaeological History from the first humans to the Vikings}. Oxford:
Oxford Univesity Press, 2015, pp. 276-279.

\item \versal{SUNDQVIST}, Olof \& \versal{VIKSTRAND}, Per (orgs.). \emph{Gamla Uppsala i ny
belysning}. Uppsala: Swedish Science Press, 2013.
\end{itemize}

\section{\versal{GENEALOGIA}}

Por definição, genealogia é o estudo que procura estabelecer a origem de
um indivíduo ou família. Muito semelhante ao conceito de linhagem, a
genealogia foi objeto de diversos usos políticos e sociais no passado.
Para os vikings, ela era um conceito-chave da vida política e foi muito
utilizada por governantes importantes e grandes famílias locais.

Estudando os elementos de exaltação régia em vários mitos nórdicos da
era pré-cristã, Stephan Brink fala, em seu \emph{The Viking World},
sobre a criação do líder prototípico da era pré-cristã na Escandinávia,
que parece ter sido não um descendente de deuses, mas fruto de uma união
sagrada. Analisando o mito de \emph{Skírnismál}, Brink comenta que nele
o deus da vegetação Freyr senta-se no alto assento de Odin
e, capaz de ver o mundo todo, olha para a terra dos gigantes e é
imediatamente preenchido de desejo por uma donzela gigante,
Gerðr. Freyr manda como emissário seu servo
Skírnir, que deveria oferecer três presentes para convencer a
giganta a casar-se com Freyr: maçãs douradas, um anel e um
cajado. A giganta recusa e Skírnir lhe oferece o anel de
Odin, Draupnir, recebendo outra recusa. Então,
Skírnir usa o cajado para enlouquecer Gerðr e esta enfim
aceita encontrar-se com Freyr.

Brink comenta que em um primeiro momento o mito foi entendido por
historiadores como um mito de vegetação: Freyr em sagrado
intercurso com a deusa da terra, Gerðr, regenerava a vegetação
durante a primavera. No entanto, Brink defende que o mito de casamento
entre o deus e a giganta possui conotações políticas e ideológicas muito
mais profundas. O aparecimento de uma giganta na mitologia nórdica quase
sempre significa o surgimento de algo novo e, nesse caso, o mito aponta
para o surgimento de um novo governo, uma nova linhagem. Outras fontes
literárias nos mostram que o mito contém um aspecto de entronização, um
mito genealógico que legitima as famílias governantes da linhagem de
ynglingar e os \emph{jarls} de Lade.

A análise de Brink parte da iconografia presente no mito, afirmando que
os objetos principais representam os primordiais objetos da
\emph{regalia} régia: o alto assento representando o trono; a maçã como
o símbolo do cosmos, do globo; o anel e o cajado são conhecidos símbolos
de dignidade e poder. Analisando o mito de \emph{Skírnismál} com outras
fontes sobre a ideologia régia (principalmente \emph{Ynglingatal,} a
saga de \emph{Ynglinga, Háleygjatal} e \emph{Hyndluljóð}) é possível
observar o desenho de um padrão mítico sobre tal pensamento. Outras
fontes indicam que o filho, o protótipo de governante, é o resultado da
união erótica entre pais mitológicos. Há inclusive o relato de Snorri
Sturluson na \emph{Saga dos Ynglingos} (baseada em um poema de
aproximadamente 870) que fala que o primeiro dos reis de
\emph{Ynglingar} é Fjolnir, justamente o filho de Freyr e
Gerðr. O mito de casamento sagrado entre um deus e uma giganta
foi também utilizado como base genealógica pela maior família de
governantes na Noruega, a família de \emph{jarls} de Lade. Tal mito está
presente no poema \emph{Háleygjatal}, que relatos indicam ser 100 anos
anterior ao \emph{Ynglingatal}. Nessa tradição escrita, os pais míticos
da linhagem seriam Odin e a giganta Skaði, que deram
origem ao primeiro \emph{jarl} da linhagem, Sæmingr.

O significado imbrincado em construir a imagem do primeiro governante de
uma linhagem como filho de um deus e uma giganta vai ainda além da ideia
de poder mítico associada ao fruto de tal união. Sendo deuses e gigantes
majoritariamente antagonistas nas histórias mitológicas clássicas, a
união sexual entre os dois gera um filho que sintetiza em si mesmo todo
o espectro de poderes cósmicos. O líder representa tanto as qualidades
dos deuses, sua vontade e habilidade de comandar, e a enorme
criatividade e força primitiva das gigantas. Além disso, a giganta como
representante da terra pode ser vista como um símbolo do território que
será governado pelo rei ou \emph{jarl}. Ela é em si uma personificação
da terra que deverá ser conquistada e governada pelo líder.

No período posterior à cristianização da Escandinávia, os principais
líderes islandeses exerciam grande poder sobre a Igreja e,
principalmente, sobre os escritos produzidos na Era Medieval. Sendo
assim, muitas grandes famílias de líderes islandeses utilizaram
amplamente conexões genealógicas dos heróis de grandes sagas com suas
próprias linhagens. É muito provável que essas ligações tenham sido
amplamente utilizadas pelos líderes da Era Medieval como uma importante
conexão entre o mundo islandês da época e a Escandinávia da era heroica.

É provavelmente desta conexão entre nobreza e o registro e retomada de
um passado heroico que derivou a importância política do poeta Snorri
Sturlunson. Herdeiro de grandes propriedades na Islândia, Snorri foi
notável por estabelecer relações com grandes homens não só na vida
política islandesa, mas também norueguesa e sueca. Ocupando lugares de
prestígio na política escandinava por conta de sua origem e de sua fama
como poeta, Snorri escreveu uma série de textos sobre o passado
longínquo da Escandinávia heroica. Seus textos foram extensivamente
utilizados por grandes famílias norueguesas como forma de legitimação de
seu domínio, não só durante o período de vida de Snorri, mas também por
muito tempo depois.

Já na Normandia, habitada por vikings cristianizados desde 911, podemos
enxergar um grande esforço de cristalização e legitimação de linhagem de
governantes na \emph{Gesta Normannorum} de Dudo de St"-Quentin, escrita
no início do século~\versal{XI}. Em um contexto cristão, a \emph{Gesta} omite o
passado pagão de Rollo e por isso não faz uso de elementos mitológicos
para a exaltação da linhagem, substituindo-os por preceitos cristãos. Os
gigantes e deuses são trocados por sonhos enviados ao primeiro líder
normando, Rollo, com anjos contando sobre o seu destino de ocupar e
governar a Normandia.

Além disso, há um esforço por parte de St-Quentin de estabelecer uma
\emph{gens}, uma origem comum para os povos normandos. O autor elabora
uma origem alemã para os vikings e os chama de dácios, em uma
tentativa de traçar uma clara linhagem pagã, como a dos francos por
exemplo, e o marco da cristianização promovida por um grande herói,
nesse caso a figura de Rollo.

De forma semelhante às sagas heroicas escandinavas, Dudo de St"-Quentin
utiliza os grandes personagens da linhagem normanda como uma forma de
exaltá-la como um todo. Iniciando sua descrição com o prototípico
``anti-herói'' Hastings, Dudo confronta a figura do sanguinário e
desonesto viking com Rollo, um líder bondoso e sensato, que é pintado
como um homem que só não era cristão antes porque ainda não conhecia a
``verdadeira'' fé. Parte dessa mesma construção é Guilherme, um homem
tão pio e cristão que só não seguiu o sacerdócio porque Deus havia lhe
dado a missão maior de ser o líder político de seus homens. Ricardo~\versal{I},
por sua vez, reúne em si as qualidades de ambos seus antepassados, sendo
praticamente um governante perfeito na visão que o texto nos apresenta.

Essa somatória de grandes líderes parece culminar na figura de Ricardo~\versal{II}, a partir de então legítimo em seu cargo de duque dos normandos não
só pela sua descendência, mas pelas qualidades de seus antepassados.
Podemos verificar também que os esforços de exaltação de uma linhagem
normanda continuam posteriormente na \emph{Gesta Normannorum Ducum} e
outros textos que a seguiram.

Dessa forma, podemos constatar que, seja na Escandinávia Pré"-cristã, na
Era Medieval ou nos nórdicos cristianizados que habitavam a região da
Normandia, o conceito de genealogia e linhagem assumiu diferentes
formas e foi crucial como forma de legitimação de lideranças e de
distinção social.

\SIG{Thiago Brotto Natário}

Ver também Família; Realeza; Religião; Rollo; Sociedade.

\begin{itemize}
\item \versal{BRINK}, Stefan; \versal{PRICE}, Neil (eds.). \emph{The Viking World}. Abingdon:
Routledge, 2008.

\item \versal{CHRISTIANSEN}, Eric.~\emph{Norsemen in the Viking Age}. Malden: John
Wiley \& Sons, 2008.

\item \versal{CROUCH}, David. \emph{The Normans: the history of a dinasty}. London:
Hambledon Continuum, 2002.

\item \versal{FERGUSON}, Robert.~\emph{The Vikings: a history}. New York: Penguin,
2009.

\item \versal{KENDRICK}, Thomas Downing.~\emph{A History of the Vikings}. New York:
Courier Corporation, 2004.
\end{itemize}
\section{\versal{GESTA HAMMABURGENSIS ECCLESIAE PONTIFICUM}}

Escrita aproximadamente entre 1072 e 1075, as \emph{Gesta hammaburgensis
ecclesiae pontificum} apresentam em seu conteúdo a história da
arquidiocese de Hamburgo-Bremen através de uma série de vidas dos
arcebispos locais, desde a fundação do bispado até a morte do arcebispo
Adalberto de Hamburgo, abordando, portanto, um período aproximadamente
entre o final do século~\versal{VIII} e a segunda metade do século~\versal{XI}. Trata-se
de uma história das origens da arquidiocese de Hamburgo"-Bremen até o
contexto no qual Adão de Bremen compôs o documento, além de ser uma das
principais fontes sobre a atividade da expansão cristã na região da
Escandinávia e do Báltico. A obra está classificada dentro do que é
denominado \emph{gesta episcoporum}, ou seja, feitos dos bispos de um
determinado bispado. É o trabalho mais conhecido de Adão de Bremen, um
dos cronistas mais conhecidos e importantes do medievo, no qual descreve
a expansão do cristianismo na Europa Setentrional.

Sabe-se muito pouco sobre Adão de Bremen. Provavelmente originário da
Saxônia, Turíngia ou Baviera, chegou à cidade de Bremen, entre os anos
1066 e 1067, a convite do arcebispo Adalberto de Hamburgo. Pouco tempo
depois, tornou-se \emph{magister scolarum} na escola da catedral de
Bremen, assim como responsável pelos assuntos de matéria expansionista
cristã. Sua educação esteve voltada não somente para o estudo dos
clássicos, mas também para um \emph{corpus} literário tardo antigo e
medieval, composto por documentos hagiográficos carolíngios, documentos
diplomáticos e cartas, o que se reflete no conteúdo das \emph{Gesta},
além da utilização de fontes orais. Morreu aproximadamente entre os anos
1081 e 1085.

As \emph{Gesta hammaburgensis ecclesiae pontificum} apresentam em seu
conteúdo narrativas voltadas para uma história propagandística, cujo
foco principal é abordar o desenvolvimento da arquidiocese de
Hamburgo-Bremen, mas também na qual se encontram informações sobre a
expansão do cristianismo pelas regiões centro-leste e do norte europeu.
Está dividida em quatro livros. Nos três primeiros encontra-se uma série
de narrativas sobre as vidas de dezesseis arcebispos, começando pelo
arcebispo Willehad de Bremen até o arcebispo Adalberto de Hamburgo,
preocupando-se, então, em apresentar uma ordem cronológica na
organização da narrativa. O livro primeiro é formado por 63
capítulos, nos quais se encontra uma descrição da geografia da Saxônia,
sua evangelização por São Bonifácio e nove narrativas de vidas dos
arcebispos de Bremen e Hamburgo, de Willhead até Unni. Também são
apresentadas as guerras de conquistas dos saxões, a fundação das sés de
Bremen e Hamburgo (787 e 831, respectivamente), as primeiras ações de
conversão das terras do norte europeu e dos ataques vikings no período.
O livro segundo apresenta 82 capítulos nos quais encontra-se
a narrativa da vida de seis arcebispos, de Adaldag até Bezelin, além de
comentar sobre as tentativas de conversão dos dinamarqueses, eslavos,
suecos e noruegueses. O terceiro livro contém 78 capítulos e
apresenta unicamente a vida do arcebispo Adalberto de Hamburgo.
Finalmente, o livro quarto contém a descrição da geografia da região da
Escandinávia e da região do Báltico durante o século \versal{XI}, sendo, assim,
uma fonte importante para o estudo da história destas regiões,
principalmente no que diz respeito à situação da realeza, a geografia e
a topografia. Além disso, a obra também aborda os costumes dos povos das
terras escandinavas, assim como os resultados do esforço expansionista
cristão naqueles territórios.

\SIG{Luciano José Vianna}

Ver também Fontes primárias; Religião.

\begin{itemize}
\item \versal{GRZYBOWSKI}, Lukas Gabriel. O início da missão cristianizadora da
Escandinávia e sua interpretação nas Gesta Hammaburgensis de Adam de
Bremen. \emph{Revista Signum}, vol. 17, n. 1, 2016, pp. 136-160.

\item \versal{HALLENCREUTZ}, Carl F. \emph{Adam Bremensis and Sueonia. A Fresh Look at
Gesta Hammaburgensis Ecclesiae Pontificum}. Uppsala: Upsaliensis
Academiae, 1984, pp. 05-34.

\item \versal{HAYWOOD}, John. \emph{Encyclopaedia of the Viking Age}. London: Thames \&
Hudson, 2000, p. 18.

\item \versal{HOLMAN}, Katherine. \emph{Historical Dictionary of the Vikings}. Lanham,
Maryland, and Oxford: The Scarecrow Press, Inc. 2003, p. 18.

\item \versal{LOYN}, Henry R. Adão de Bremen. In: \emph{Dicionário da Idade Média}. Rio
de Janeiro: Jorge Zahar Editor, 1997.

\item \versal{MARTTIE}, Rodrigo Mourão. Adão de Bremen. In: \versal{LANGER}, Johnni (org.).
\emph{Dicionário de mitologia nórdica. Símbolos, mitos e ritos}. São
Paulo: Hedra, 2015, pp. 15-17.

\item \versal{NORTH}, William. Adam of Bremen. In: \versal{EMMERSON}, Richard K (ed.). \emph{Key
Figures in Medieval Europe. An Encyclopedia}. New York/London:
Routledge, 2006, p. 5.

\item \versal{NYBERG}, Tore. Adam of Bremen. In:
 \versal{PULSIANO}, Phillip (ed.). \emph{Medieval Scandinavia: An Encyclopaedia}.
New York and London: Garland Publishing Inc., 1993, pp. 1-2.

\item \versal{SCIOR}, Volker. Adam Bremensis. In: \emph{Medieval Nordic Literature in
Latin. A Website of Authors and Anonymous Works} (c. 1100-1530).
Disponível em:
\emph{wiki.uib.no/medieval/index.php/Adam\_Bremensis\#Biography}.
Acesso em 24/06/2017.
\end{itemize}
\section{\versal{GESTA NORMANNORUM}}

A \emph{Gesta Normannorum}, ou \emph{De Moribus et Actis Primorum Normanniae
Ducum}, foi escrita pelo cônego Dudo de St-Quentin entre os anos de 995
e 1015, sob encomenda do duque da Normandia Ricardo~\versal{I}. Falando sobre o
período do estabelecimento (911) do viking Rollo no território da futura
Normandia, Dudo traz uma visão proveniente da família ducal normanda
sobre sua própria história, conciliando o passado viking da linhagem com
sua atuação dentro do território cristão.

Não há muita dúvida em torno da autoria da \emph{Gesta}. Acredita-se que o
responsável pela sua confecção, Dudo de St-Quentin, tenha nascido em 960
em Vermandois, Picardia. Servindo como cônego na abadia de St-Quentin,
Dudo foi enviado ao ducado da Normandia como um embaixador no final da
década de 980. Aproximando-se da família ducal, Dudo passou a residir em
Rouen. Próximo a sua morte, o duque Ricardo~\versal{I} (942-996) teria
requisitado de Dudo a escrita de uma história sobre sua linhagem e o
estabelecimento de seu antepassado Rollo no território normando. Todas
estas informações são fornecidas pelo próprio Dudo em sua dedicatória ao
bispo Adalberto de Laon, escrita junto com a \emph{Gesta}.

Assim como inúmeras outras fontes medievais, a \emph{Gesta Normannorum}
foi recuperada e republicada por historiadores e arquivistas do século
\versal{XIX}. O primeiro foi André Duchesne, que a incluiu em uma coletânea sobre
historiografia normanda sob o título de \emph{De Moribus et Actis
Primorum Normanniae Ducum}, parafraseado da carta escrita por Dudo a
Adalberto de Laon. Em uma edição de 1865, Jules Lair manteve tal título.
No entanto, os manuscritos que fornecem um título chamam o texto de
\emph{Gesta} ou \emph{Historia Normannorum} e, portanto, convencionou-se
entre os historiadores manter o título original.

Ao longo do século~\versal{XIX} e começo do~\versal{XX}, a \emph{Gesta Normannorum} foi
sendo relegada ao esquecimento por ser tratada por estudiosos como uma
fonte fantasiosa e nada confiável para estudar o período de
estabelecimento da Normandia (na verdade, alguns dos próprios
historiadores normandos já no século~\versal{XII} apontavam para tal fato). No
entanto, a partir principalmente do início dos anos 2000 a \emph{Gesta}
passou a ser estudada sob um viés diferente, não mais a partir da busca
do que ``de fato havia acontecido'' no começo do século~\versal{X}, mas qual
visão a família ducal buscava passar sobre sua própria história.

Encomendada por Ricardo~\versal{I}, que teve uma difícil ascensão, mas um governo
longo e frutífero, a \emph{Gesta} seguramente pode ser vista como uma
espécie de visão oficial dos duques sobre sua própria história, uma vez
que se refere a tempos há muito idos e foi escrita na própria corte de
Rouen, com informações fornecidas pela família ducal. A \emph{Gesta} é
um claro exercício de legitimação de uma família governante sobre seu
território, buscando também a consolidação de uma identidade Normanda e
a associação direta da família de Rollo e Ricardo~\versal{I} com o território
sobre o qual governavam.

A \emph{Gesta} subdivide-se em quatro partes, cada uma cobrindo a vida e
os feitos dos membros da linhagem ducal. A primeira fala sobre Hastings,
um viking pagão e sanguinário, que Dudo usa como uma contraposição a
Rollo, líder viking que ocupa a segunda parte do texto e tem uma
personalidade extremamente cristã. Rollo é um inimigo temível, mas um
líder amável, tendo ao longo da narrativa uma série de sonhos e
mensagens enviadas por Deus sobre a terra Normanda. A terceira parte
fala sobre o filho de Rollo, Guilherme, e a quarta sobre seu neto
Ricardo~\versal{I}, todos pintados como grandes homens e exaltados por Dudo como
uma forma de enaltecer não só sua linhagem, mas toda uma \emph{gens}
normanda.

Um claro exercício de legitimação política de um governante sobre um
território e de constituição de uma identidade cristã para um grupo
nobre que possuía origens vikings, a \emph{Gesta Normannorum} nos serve
muito mais para o estudo do período que corresponde a sua escrita do que
ao período ao qual seu autor se refere. É através dos grandes feitos dos
líderes normandos presentes na \emph{Gesta} de Dudo de St-Quentin que
podemos abordar a autopercepção da família normanda durante o firmamento
de sua posição política no final do século~\versal{X} e começo do \versal{XI}.

\SIG{Thiago Brotto Natário}

Ver também Fontes primárias; França na Era Viking; Normandia; Rollo;
Vikings na França.

\begin{itemize}
\item \versal{CROSS}, Katherine Clare.~\emph{Enemy and ancestor: viking identities and
ethnic boundaries in England and Normandy, c. 950-c. 1015}. Tese de
Doutorado, \versal{UCL} (University College London), 2014.

\item \versal{DUDO DE SAINT QUENTIN}. \emph{Gesta Normannorum seu de moribus et actis
primorum Normanniae ducum}\textbf{.} ed. Felice Lifshitz, 1996.
Disponível em:
hs-augsburg.de/\textasciitilde{}harsch/Chronologia/Lspost11/Dudo/dud\_no00.html

\item \versal{HUISMAN}, Gerda C.~\emph{Notes on the Manuscript Tradition of Dudo of St
Quentin's Gesta Normannorum}. Anglo-Norman text society\emph{,} 1984.

\item \versal{SEARLE}, Eleanor. \emph{Fact and pattern in heroic history: Dudo of
St."-Quentin}. Tese de Doutorado, California Institute of Technology,
1983.

\item \versal{VAN HOUTS}, Elizabeth. \emph{The Normans in Europe}. Manchester:
Manchester University Press, 2000.
\end{itemize}

\section{\versal{GESTA NORMANNORUM DUCUM}}

Considerada como uma espécie de continuação da \emph{Gesta Normannorum}
escrita por Dudo de St-Quentin no começo do século \versal{XI}, a \emph{Gesta
Normannorum Ducum} começou a ser elaborada pelo monge Guilherme de
Jumièges em algum momento em torno de 1060. Dando continuidade à
proposta de Dudo de escrever uma história dos duques da Normandia, a
\emph{Gesta Normannorum Ducum} narra extensivamente os grandes feitos
dos herdeiros e governantes da linhagem normanda, como uma forma de
exaltar não somente estes grandes homens, mas também toda a sua linhagem
e a família ducal. Podemos considerar que esse texto faz parte de
um esforço, iniciado por Dudo, de consolidação de uma identidade
normanda, principalmente por conta dos relatos de Guilherme sobre a
grande vitória normanda contra a Inglaterra na Batalha de Hastings em
1066.

É um consenso entre os historiadores especialistas no tema que a
\emph{Gesta Normannorum Ducum} sofreu uma série de revisões e adições ao
longo dos séculos \versal{XI} e \versal{XII}. A primeira delas foi concretizada pelo
próprio Guilherme de Jumièges, que por volta do ano 1060 adicionou ao
seu trabalho um relato sobre a conquista da Inglaterra conduzida pelo
duque Guilherme da Normandia em 1066. Já em 1113, a \emph{Gesta
Normannorum Ducum} recebeu uma nova revisão e adições por parte de outro
importante autor normando, Orderic Vitalis, monge de Saint-Evroult, que
copiou e adicionou uma série de notas ao texto original. Outro
importante autor a fazer anotações no texto de Guilherme de Jumièges foi
Roberto de Torigny, que acrescentou um capítulo inteiro contendo a vida
e os feitos de Henrique~\versal{I} da Inglaterra.


Emily Abu estima que existam ainda hoje 47 manuscritos preservados da
\emph{Gesta Normannorum Ducum}. Durante a segunda metade do século~\versal{XIX},
a obra foi parcialmente traduzida para o francês por Jules Lair, como
parte de seu esforço por adaptar para uma língua moderna e publicar em
coletânea os principais textos sobre o passado normando. O autor
concretizou sua tradução e publicação da \emph{Gesta Normannorum} de
Dudo de St-Quentin, mas morreu antes que pudesse finalizar a tradução da
obra de Guilherme de Jumièges. Quem o fez foi Jean Marx, em 1914.
Excertos da obra são constantemente publicados em coletâneas sobre
historiografia normanda, como por exemplo \emph{The Normans in Europe},
organizado por Elizabeth Van Houts. A autora foi também responsável por
uma nova edição e tradução do texto, publicado em dois volumes em 1992 e
1995.

O trecho mais importante da obra é provavelmente o sétimo livro, que traz
uma descrição sobre a batalha de Hastings e a sucessão do trono da
Inglaterra. A narrativa elaborada por Guilherme de Jumièges pretende
legitimar o domínio de Guilherme da Normandia sobre a Inglaterra. Em seu
relato, o autor aponta que, percebendo-se sem herdeiros, o rei Eduardo da
Inglaterra teria mandado o arcebispo de Canterbury, Roberto, até o duque
Guilherme da Normandia para confiar a ele o reino da Inglaterra. Eduardo
teria enviado também o mesmo emissário para fazer com que Haroldo, o
maior de seus duques, prometesse fidelidade a Guilherme. Porém, com a
morte de Eduardo em 1065, Haroldo ignora seus desejos prévios e ascende
ao trono, colocando o povo inglês contra Guilherme.

Com este breve relato, Guilherme de Jumièges coloca Haroldo como um homem
desonrado que traiu seu rei e o julgamento divino que havia prestado.
Essa construção maléfica e traiçoeira da figura de Haroldo tem o
propósito de exaltar a figura de Guilherme e de colocar sua conquista em
1066 como legítima, pelo simples fato de que este já seria o verdadeiro
herdeiro da Inglaterra, conforme o desejo de seu legítimo rei Eduardo. A
batalha de Hastings se torna, assim, mero cumprimento de um juramento
divino que havia sido prestado de Haroldo para Guilherme.

Esse é o tom de todo o relato sobre a conquista, aponta Stephen Morillo,
afirmando ainda que Guilherme de Jumièges foi um contemporâneo dos
acontecimentos e que, apesar de ter sido um monge sem treinamento
militar e não ter tido muito acesso a detalhes sobre a campanha militar,
demonstra um óbvio orgulho pelas conquistas militares dos normandos. Já
a própria Elizabeth Van Houts considera que a adição do livro sobre a
Batalha de Hastings trata-se de uma propaganda que visa justificar as
ações de Guilherme, o Conquistador.

Pelo grande número de revisões e cópias do texto de Guilherme, podemos
constatar sem muita margem de erro que a \emph{Gesta Normannorum Ducum}
foi extremamente importante para a consolidação de uma tradição
histórica normanda, sendo copiada e revisada por boa parte de seus
principais autores. Até hoje o trabalho de Guilherme é essencial para o
estudo da escrita normanda e da visão de sua nobreza e clero sobre si
mesmos.

\SIG{Thiago Brotto Natário}

Ver também Fontes primárias; França na Era Viking; Normandia; Rollo;
Vikings na França.

\begin{itemize}
\item \versal{ALBU}, Emily. \emph{The Normans in their histories: propaganda, myth and
subversion}. Woodbridge: Boydell \& Brewer, 2001.

\item \versal{CROUCH}, David. \emph{The Normans: the history of a dinasty}. London:
Hambledon Continuum, 2002.

\item \versal{MORILLO}, Stephen. \emph{The Battle of Hastings: sources and
interpretacions.} Woodbridge: Boydell \& Brewer, 1996.

\item \versal{VAN HOUTS}, Elizabeth. \emph{The Normans in Europe}. Manchester:
Manchester University Press, 2000
\end{itemize}

\section{\versal{GODI}}

O \emph{godi} (plural \emph{godar)} era o homem que possuía o
\emph{godord}. O \emph{godord} era a liderança local na Islândia. O
\emph{godord} poderia ser herdado, comprado, trocado, compartilhado;
contudo, qualquer mulher que herdasse um \emph{gordord} era obrigada pela
lei a ceder a posição para um homem. Inicialmente havia 26 \emph{godar}, 
aumentando para 39 em 965, e 48 em 1005. A
distribuição era de nove \emph{godi} em cada quadrante, e o
quadrante norte teria três \emph{godar} adicionais.

Os \emph{godar}, além do papel político, tinham uma função no culto:
eram responsáveis pelo culto de deuses e pela construção do
\emph{hof}, local para culto\emph{.} O aspecto religioso do \emph{godi}
é enfatizado em alguns documentos. O exemplo mais conhecido ocorre na
\emph{Eyrbyggja saga,} na qual o \emph{godi} Hrólfr Mostrarskegg trocou
seu nome próprio para Thórólfr em homenagem ao deus Thor. Hrólfr
Mostrarskegg emigrou da Noruega para Islândia, e a sua nova fazenda foi
fundada na península Thórsnes, onde ergueu um \emph{hof} em homenagem a
Thor. O caráter sagrado do \emph{godi} foi perdendo sua importância com
a cristianização da Escandinávia. Uma outra teoria levantada pela
historiografia foi que a sacralidade dos costumes pré-cristãos foi
substituída pelo papel de defensor da fé cristã.

Essas lideranças políticas dos \emph{godar} originalmente não eram
ligadas a territórios específicos, mas sim a uma relação com
seus \emph{Thingmenn}, homens da \emph{Thing}, de modo que esses homens livres
tinham a liberdade de escolher qual \emph{godi} seguir. Os seguidores de
cada \emph{godi} tinham que acompanhá-lo nas assembleias nacionais e
locais, ou pagar impostos para ajudar a cobrir as despesas para aqueles
que iam na \emph{Thing}. Dentro de cada quadrante, os \emph{godar} locais
eram responsáveis por convocar as assembleias de primavera e outono;
entretanto, posteriormente, essas assembleias poderiam ser realizadas por
um \emph{godi} e seus seguidores em vez de todos os homens livres e os
\emph{godar} do quadrante.

Na \emph{Althing,} o \emph{godar} elegia o falador-das-leis e constituía
um conselho legislativo, revisitando e criando leis. Esse conselho
exercia também um poder judiciário que poderia determinar punições
para aqueles que não obedeciam a legislação. Os \emph{godar} e os homens
livres proviam suporte mútuo nas suas contendas e protegiam seus
interesses coletivos nas assembleias locais e nacionais.

Nos séculos~{\versal{XII} e \versal{XIII}} essas lideranças se tornaram associadas com
distritos particulares que eram controlados por famílias e indivíduos
poderosos e menos numerosos, conhecidos como \emph{stórgodar}, o
``grande \emph{godar''.} Esse processo resultou em uma guerra civil
entre as famílias rivais e suas facções. É entendido que essa guerra
civil levou posteriormente a subjugação da Islândia ao reinado norueguês
em 1264, onde o \emph{godord} foi abolido e substituído pelos condados e
\emph{sýsla.}

\SIG{André Araújo de Oliveira}

Ver também Althing; Conversão ao cristianismo; Thing; Islândia na Era
Viking; Religião.

\begin{itemize}
\item \versal{BRINK}, Stefan; \versal{PRICE}, Neil (eds.). \emph{The Viking World}. New York:
Routledge, 2008.

\item \versal{HOLMAN}, Katherine. \emph{Historical Dictionary of the Vikings}. Oxford:
The Scarecrow Press Inc., 2003.

\item \versal{LINDKVIST}, Thomas. Early political organisation, introductory survey.
In: \versal{HELLE}, Knut (org.). \emph{The Cambridge History of Scandinavia}.
Cambridge: University of Cambridge Press, 2003, pp. 160-167, vol. 1.

\item \versal{VÉISTEINSSON}, Orri. \emph{The Christianization of Iceland}: Priest,
Power and social change 1000-1300. Oxford: Oxford University Press,
2000.
\end{itemize}

\section{\versal{GODOS}}

Povo germânico que, segundo as teorias mais aceitas, surgiu no litoral
báltico defronte a Escandinávia no século \versal{II} a.C., empreendendo, no mesmo século,
uma migração que o levou à região entre os rios Danúbio e Don (modernas
Ucrânia, Romênia, Rússia e Moldávia), para depois estabelecer contato com
o Império Romano. Em relações caracterizadas ora pela cooperação, ora
pelo conflito, seus ramos visigodos e ostrogodos acabaram por criar
poderosos reinos em meio à queda do Império, entre os séculos~{\versal{V} e \versal{VIII}},
tendo papel importante na formação da Europa Medieval.

A origem dos godos é objeto de controvérsia até os dias
atuais. Até alguns anos, a hipótese mais aceita era de que eles teriam
surgido no sul da Escandinávia, tal como apontado por Jordanes,
cronista gótico do século \versal{VI}, porém descobertas arqueológicas recentes originaram debates 
que levaram a uma nova teoria, na qual o ponto de origem seria o
litoral báltico, defronte à Escandinávia, ponto de contato
cultural e comercial intenso.

Dali os godos empreenderam uma migração pela Europa Central que os levou
até o domínio que tinham no século \versal{IV}, entre o Danúbio e o Don, onde
acabaram divididos em dois ramos: os \emph{tervingi} e os
\emph{greuthungi}. Embora existam interpretações que indiquem que
visigodos e ostrogodos se originaram a partir desses ramos, tal ideia
não encontra ligação com os estudos histórico-arqueológicos sobre os
primeiros ramos dos godos.

Ainda há uma outra teoria, considerada por alguns historiadores, de que
os godos só teriam surgido no século~\versal{III}, como um povo que criou sua
identidade a partir das dinâmicas de contato com o mundo romano, através
das fronteiras do Império no Danúbio. Tal entendimento é alvo de forte
polêmica, uma vez que despreza o grande material descoberto pela
arqueologia e confrontado com as fontes do período, em especial em
meados do século~\versal{II}, onde surge um povo intitulado \emph{gutones} ou
\emph{gothones}.

É possível que, dada a posição dos godos e a pressão aplicada por seu
deslocamento pelo rio Vístula, tenham eles sido uma das causas das Guerras
Marcomanas (166-180), um devastador conflito onde tribos germânicas
invadiram o Império e causaram destruição no norte da Itália. Os
primeiros contatos registrados entre Godos e romanos se dariam por meio
de choques. No século \versal{III}, entre 220 e 250, há uma série de registros
sobre ações góticas, expedições marítimas que produziram estragos com
pilhagens e saques pelo Império afora. Tais ações desencadearam uma
resposta militar romana.

Em 250, a maior força gótica até então, liderada pelo rei Cniva, cruzou
o Danúbio, invadiu a província da Mésia Inferior e com saques e
pilhagens, sitiou e tomou Nicópolis. O Imperador Décio viu a
oportunidade de obter um triunfo, atacou sem preparação prévia e
ele, seu herdeiro e boa parte das forças imperiais acabaram mortos em combate na
batalha de Abrittus. O sucesso gótico impulsionou a Primeira Guerra Gótica
(250-271), uma série de batalhas e conflitos nos quais os godos prevaleceram
até 268, quando os imperadores Galieno, Cláudio e Aureliano,
sucessivamente, os derrotaram de maneira devastadora, quebrando o poder
militar gótico por um tempo relativo, mas não definitivamente, com
invasões e expedições góticas no futuro. As duras perdas da guerra ainda
aceleraram o processo de divisão dos godos em \emph{tervingi} e
\emph{greuthungi}, com reis e limites entre seus domínios.

O Império ainda testemunharia ações dos dois grupos embora os
\emph{greuthungi} estivessem envolvidos na disputa de poder sobre as
estepes, o qual, em um último momento forçaria a migração deles e dos
\emph{tervingi}, quando das guerras contra os hunos, que acabaram por
pressioná-los contra a fronteira romana. Em 376, autorizados ou não,
grupos de godos, majoritariamente \emph{tervingi}, com alguns
\emph{grethungi}, além de alanos e outros povos da estepe que não
queriam se curvar diante dos hunos, cruzaram o \emph{limes} imperial
(linha de fronteira) e começaram a assentar em províncias romanas. Uma
crise surgiu e estourou a Segunda Guerra Gótica (376-382).

Em 9 de agosto de 378, o imperador Valente, regente da metade oriental
do Império Romano, com a fina flor do Exército Imperial, enfrentou os
godos na Batalha de Adrianópolis e foi derrotado e morto. O choque foi
terrível para o Império, e após anos sem conseguir vitórias sobre os godos no
campo de batalha, foi assinado um tratado de paz que os colocou a
serviço do Império, defendendo parte do \emph{limes} do Danúbio contra
tentativas de invasão principalmente por parte dos hunos, reforçados
pelos \emph{greuthungi} que se colocaram a serviço deles. A relação com
Roma foi determinante neste momento. Entre 395 e 405, formalizam-se novas
divisões dos godos, aparecendo os visigodos e ostrogodos. Os primeiros
lutaram até 418, entre momentos de cooperação e conflito para se
estabelecerem dentro da máquina imperial, empreendendo uma migração que
os fez passar pela Grécia, Bálcãs e Itália, onde saquearam Roma em 410, para
choque de muitos, sob o comando de seu primeiro rei, Alarico.

Em 418, os visigodos foram o primeiro povo germânico a ser assentado
oficialmente em terras imperiais, recebendo como feudo a província da
Aquitânia (sudoeste da moderna França), onde nasceu o Reino visigodo de
Toulouse. Já os ostrogodos se mantiveram como aliados dos hunos desta
época até 454, quando ganharam a independência após a batalha de Nedao.
Visigodos e ostrogodos se enfrentaram em lados opostos em 451, quando os
hunos a serviço de Átila lutaram contra uma coalizão de romanos e
germanos em Châlons, na Gália (moderna França).

O ponto alto de visigodos e ostrogodos se dá depois da queda do Império
Romano do Ocidente, em 476, quando nasce o Reino Ostrogodo da Itália,
com reis como Teodorico, o Grande, que reinaram como patrícios romanos,
reconhecidos pela corte imperial de Constantinopla. Os visigodos serão
empurrados para a Espanha, ao serem derrotados pelos Francos, liderados
por Clóvis, na batalha de Voillé, em 507, quando o rei Alarico~\versal{II} foi
morto e a presença visigótica na Gália foi absorvida pela conquista
franca. Os francos não ousaram explorar o sucesso e invadir a parte
espanhola do reino devido a atuação de Teodorico, rei dos ostrogodos,
que reinou os visigodos como protetor do filho de Alarico~\versal{II}, Amalarico,
até 526.

O Reino Visigodo de Toledo se estabeleceu com o domínio sobre a Espanha,
obtido à custa de campanha contra os suevos e alguns vândalos que tinham
ficado na região durante as migrações. Com uma nobreza dividida e o fato
de que nenhuma casa conseguia se sustentar no trono real por mais que
duas gerações, quando muito, o reino era marcado pela intriga e por
guerras civis cruentas, que foram a razão de seu enfraquecimento. Em
meio a tal ambiente, o poder dos bispos e da Igreja cresceu a tal ponto
que, em meados do século~\versal{VII}, quem elegia o novo rei não eram mais os
nobres, mas os líderes eclesiásticos.

Em 711, uma invasão muçulmana vinda do Norte da África deu fim ao
reino, com a derrota e morte do rei Rodrigo na batalha de Guadalette.
Mitos diversos foram contados pela derrota: o primeiro de que o rei foi
abandonado no campo de batalha por nobres traiçoeiros, que desejavam seu
trono, mas foram mortos pelos muçulmanos. Já no caso dos ostrogodos,
Teodorico foi capaz de criar um reino poderoso e temido pelos bizantinos
na Itália, valendo-se das instituições romanas e de leis próprias, criando um
local de convívio entre romanos e germanos, usando as capacidades dos
dois povos para aumentar seu poder. Sua morte em 526 provocou um baque
no reino pois não havia ninguém capaz de sucedê-lo que possuísse igual qualidade de liderança.

Os bizantinos estavam buscando reconquistar áreas do Império do Ocidente
e lançaram uma invasão da Itália em 535. Varrendo os ostrogodos, que
sofriam com a má liderança, tomaram Ravena em 540, provocando mudanças
na coroa ostrogótica. Porém, o general bizantino, Belisário, começou a
ser temido por seu imperador, Justiniano, e foi chamado a voltar à
capital. No vácuo de poder, surgiu Totila, grande rei ostrogodo,
habilidoso guerreiro e que em dez anos foi capaz de retomar muitas áreas
conquistadas pelos bizantinos. Capturando até Roma, o rei ostrogodo
forçou uma reação imperial, com Justiniano enviando Narses no comando de
um grande exército bizantino. Ainda assim, Totila conseguiu manobrar e
resistir por algum tempo, mas com a perda de sua força naval, não pôde
impedir os desembarques dos bizantinos.

Totila enfrentou os bizantinos na Batalha de Tagina, em 552, e a
despeito da sua habilidade, foi derrotado por Narses, que tinha estudado
seus métodos. Um de seus generais, Teia, o sucedeu, mas também foi
derrotado e morto em 553, na batalha de Mons Lactarius. O
reino ostrogodo na Itália tinha sido destruído, e a população restante
foi absorvida localmente ou seguiu norte para ser absorvida pelos
lombardos, que invadiram e conquistaram a Itália pouco tempo depois da
invasão bizantina, em fins do século \versal{VI}.

O legado dos godos persistiu por causas variadas, apesar de seu desaparecimento na Alta Idade Média. 
Um de seus ramos, o
mais raro e menos conhecido, o dos godos da Crimeia, parece ter existido
até o século \versal{XVIII} segundo fontes. Existe uma disputa sobre quem
representa os godos nos títulos reais entre Suécia e Dinamarca até hoje.
Ambas as casas reais reivindicam e usam o título de rei dos godos.
Ainda existem movimentos reconstrucionistas em diversas partes da
Europa, que buscam reavivar antigos dialetos góticos. Ainda há a
tradição na Espanha de que a Reconquista foi lançada por um godo,
Pelágio, que se tornou rei das Astúrias.

\SIG{Sandro Teixeira Moita}

Ver também Escandinávia; Povos e etnias; Viking.

\begin{itemize}

\item \versal{HALSALL}, Guy. \emph{Barbarian Migrations and the Roman West, 376--568.}
Cambridge and New York: Cambridge University Pres, 2007.

\item \versal{HEATHER}, Peter. \emph{The Goths}. Oxford: Clarendon Press, 1996.

\item \versal{HEATHER}, Peter. \emph{Goths and Romans 332-489}. Oxford: Clarendon
Press, 1991.

\item \versal{HEATHER}, Peter \& \versal{MATTHEWS}, John. \emph{Goths in the Fourth Century}.
Liverpool: Liverpool University Press. 1991.

\item \versal{KALIFF}, Anders: \emph{Gothic Connections. Contacts between eastern
Scandinavia and the southern Baltic coast 1000 \versal{BC} -- 500 \versal{AD}}. Occasional
Papers in Archaeology (\versal{OPIA}), vol. 26. Uppsala, 2001.

\item \versal{KULIKOWSKI}, Michael. \emph{Guerras Góticas de Roma}. São Paulo: Madras,
2008.

\item \versal{TODD}, Michael. \emph{The Early Germans}. Oxford: Blackwell Publishing,
2004.

\item \versal{WOLFRAM}, Herwig. \emph{History of The Goths.} Berkeley: University of
California Press, 1988.
\end{itemize}

\section{\versal{GOKSTAD}}  

Somente através do domínio dos mares os nórdicos puderam se deslocar da
Escandinávia para comercializar, descobrir novas terras e logicamente
invadir e conquistar lugares distantes. Tais empreitadas tiveram êxito
unicamente devido a exímia habilidade de seus construtores de navios,
fazendo com que se tornassem os senhores dos mares.

Dentre as descobertas arqueológicas que evidenciaram a perícia na
construção naval escandinava, uma das que mais chama a
atenção é o navio de Gokstad. Encontrado em abril de 1880, em uma
fazenda da Noruega, num local conhecido popularmente como Vale dos Reis.
A descoberta ocorreu por mera especulação dos filhos de um fazendeiro que
resolveram explorar uma colina. A notícia das escavações iniciais chegou
aos ouvidos de Nikolas Nikolaysen, pesquisador e presidente das
Antiguidades de Oslo, que rapidamente assumiu as investigações.

As escavações arqueológicas revelaram um magnífico navio, quase intacto, exceto por estar ligeiramente achatado e distorcido
devido ao seu enterramento. Inicialmente pensaram que o navio havia sido
enterrado em meados de 800 d.C., mas, segundo Stylegar, pesquisas
dendrocronológicas recentes demonstraram que seu sepultamento ocorreu no período entre 900 a 905 d.C.

Segundo Atkison, após a remoção da argila em que o navio estava
soterrado, emergiu uma nave de pouco mais de 23~m
de comprimento, 5,5~m de largura, cerca de 2~m de profundidade
da amurada até a parte inferior da quilha e com todos os equipamentos,
com um peso aproximado de 32~ton.

No que se refere ao material para sua confecção, a quilha, proa e o
cadastre de popa, bem como o leme, foram feitos de um só pedaço de carvalho. Seu mastro, medindo aproximadamente 12~m de altura, foi feito de pinho. Segundo Brøndsted, as carreiras de tábuas da nave
presas entre si, calafetadas com cordas alcatroadas e amarradas às
balizas por ramos de vime, deram ao navio uma grande elasticidade,
fazendo com que pudesse suportar mares revoltos.

Infelizmente a descoberta de Nikolas mostrou que o navio havia em algum
momento sido violado e seus tesouros saqueados, porém ainda permaneceram
restos de 32 escudos de guerra pintados de amarelo e preto,
que ficavam expostos nos flancos da embarcação. Em seu interior havia
uma câmara funerária, nela jazia um homem de meia idade com cerca de
1,80~m de altura, que, segundo Brown, padecera de gota.

O homem encontrado certamente deveria ter sido alguém de status elevado. 
Brown relata que, na década de 1920, o professor Anton Willem Brogger
afirmou tratar-se do rei Olavo Gudrodson, falecido em aproximadamente 900
d.C. Junto ao corpo, repousavam os esqueletos de doze cavalos todos com
arreios ornamentais, também estavam as ossadas de seis cães e os ossos e
penas de um pavão, uma ave rara para a região.

No interior da nave também foram encontradas seis camas, vestígios de lã
e seda entremeados de fios de ouro, e no local ainda estavam três pequenas
embarcações, remos, vários baús, barriletes, tonéis, canecas de madeira
e pratos, bem como um caldeirão de ferro e um armário com jogos de
tabuleiro.

Devido a engenhosidade do navio de Gokstad, em 1893 foi construída uma
réplica sua. Sob o comando do capitão norueguês Magnus Andersen, o navio
partiu da Suécia no dia 30 de abril, atravessou o Atlântico Norte e
aportou nos Estados Unidos em 13 de junho, demonstrando assim a
engenhosidade nórdica na criação de suas embarcações. Após a restauração
do navio de Gokstad, ele foi levado ao Museu dos Barcos Vikings em Oslo,
onde se encontra exposto atualmente para visitação.

\SIG{Marlon Ângelo Maltauro}

Ver também Embarcações; Navegação marítima; Noruega da Era Viking.

\begin{itemize}
\item \versal{ATKINSON}, Ian. \emph{Los barcos vikingos}. Barcelona: Akal, 1990, pp.
08-17.

\item \versal{BRØNDSTED}, Johannes. \emph{Os Vikings}. São Paulo: Hemus, s.d., pp.
116-118.

\item \versal{BROWN}, Dale W. (ed.). \emph{Os Vikings: intrépidos navegantes do Norte}.
São Paulo: Abril/Time Life, 1999, pp. 09-14.

\item \versal{STYLEGAR}, Frans-Arne. Gokstadfunnet. \emph{Store norske leksikon,} 2017.
\end{itemize}
\section{\versal{GOTLAND (GOTLÂNDIA)}}

Gotland consiste em uma ilha de origem calcária, de clima temperado e solo
raso, mas relativamente fértil, propício para a agricultura e o
pastoreio. Segundo a narrativa da \emph{Saga dos Gutas}
(\emph{Gutasaga}), a ilha foi colonizada pelo povo guta, população
aparentada culturalmente dos suíones (de onde advém o termo sueco) e dos
gotas. Alguns estudiosos sugerem que os gutas poderiam ser os ancestrais
dos godos ou um povo aparentado. Apesar de a \emph{Saga dos Gutas}
datar do século \versal{XIII}, sem definir historicamente quando se deu a
habitação da ilha, descobertas arqueológicas apontam que Gotland já
era habitada desde a Pré-história, pelo menos. No entanto, 
desconhece-se a regularidade de seu processo de povoamento.

Localizada a 90 km de distância da costa da Suécia, a ilha faz parte
atualmente do território sueco. Gotland é a maior ilha do mar
Báltico e da Suécia, possuindo um território de 2994
km\textsuperscript{2}. Embora Gotland compartilhasse a cultura
escandinava do continente, ainda assim sua população desenvolveu
características próprias, como no caso do dialeto local, o gútnico
antigo, uma variação regional do nórdico antigo oriental, falado no
território sueco.

Devido a essa relativa distância do continente, o acesso a várias rotas
mercantis da região, isso não apenas ocasionou em mudanças culturais,
mas também na constituição de uma classe política autônoma, pelo menos
em algumas épocas, do controle sueco. Tal autonomia política fica
evidente nas obras públicas e particulares de seus membros. Nesse ponto,
a importância de Gotland durante a Era Viking se deve ao seu papel como
centro comercial, suas estruturas militares e suas imensas riquezas.

No que diz respeito ao comércio, Helen Clarke aponta que foram achados
objetos de origem romana, incluindo armas e moedas datadas do século~\versal{V}.
Não se sabe se tais objetos chegaram à ilha através do comércio ou foi
resultado de saque. De qualquer forma, além de moedas romanas,
encontraram-se de períodos posteriores, já entre os séculos~{\versal{VI} ao \versal{X}},
moedas germânicas, frísias, eslavas, bizantinas e até árabes. Clarke
sublinha que existem dúvidas quanto a importância dos portos gotlandeses
no comércio báltico, pois não se sabe até onde ia a extensão de seus
contatos, mas a presença dessa variedade de moedas de prata na
região já é um reflexo do comércio de longa distância praticados pelos
nórdicos nos séculos~{\versal{IX e \versal{X}}.

Embora Gotland tenha conseguido providenciar sua subsistência agrícola e
pecuária, houve necessidade da importação de ferro, cobre, prata, ouro,
joias, âmbar, vidro, xisto, tecidos etc. No entanto, não era apenas a
riqueza que afluía para a ilha, Gotland também tornou-se um polo
artesanal. A ilha é notável por seu estilo artístico característico na
joalheria, ourivesaria e metalurgia, tendo-se encontrado em alguns
túmulos verdadeiras raridades, o que revela o alto nível do talento de
seus artesãos.

Por outro lado, toda essa movimentação comercial, apesar de ter levado
cidades como Visby (atual capital) e Paviken a se tornarem importantes
centros mercantis da Escandinávia, acabou por gerar rixas e
desavenças entre a elite da ilha, ao mesmo tempo em que também fazia
surgir o temor de possíveis invasões. Nesse ponto, Gotland se destaca
pela condição de que sua elite política delegou grandes somas de
dinheiro para a construção de cerca de cem fortificações entre fortes,
redutos, muros, muralhas etc. O maior forte da ilha localizado em
Torsburgen, ocupava uma área de 154 hectares, possuindo uma muralha com
7~m de altura e quase 2~km de extensão.

Sua riqueza arqueológica é imensa: além dos objetos
oriundos do comércio intercontinental, das dezenas de fortificações, a
ilha também é o local de milhares de túmulos e de pedras rúnicas
(\emph{runestones}). A respeito das pedras rúnicas, a arte gotlandesa
imprimiu em tais rochas algumas características particulares de sua
cultura, por isso alguns temas e formas são encontrados apenas na
ilha, como as diversas pedras pintadas com tema mitológicos, religiosos
e cenas do cotidiano. Não obstante, Gotland faz parte da região com o
maior número de pedras rúnicas na Escandinávia. Birgit Sawyer sublinha
que, das mais de 2.300 pedras rúnicas catalogadas, 89,2\% desse total
fica situado em território sueco e centenas dessas se encontram
em Gotland.

Por tais características, a ilha se tornou um importante centro econômico
durante os séculos~\versal{V} ao \versal{X}, embora pareça que seu alcance político não
tenha sido tão amplo, pois se desconhece soberanos gotlandeses que
participaram da expansão nórdica.

\SIG{Leandro Vilar Oliveira}

Ver também Birka; Comércio; Suécia da Era Viking.

\begin{itemize}
\item \versal{CARLSSON}, Dan. ``Ridanæs'': a Viking Age port of trade at Fröjel,
Gotland. In: \versal{BRINK}, Stefan (ed.). \emph{The Viking World}. London/New
York: Routledge, 2008, pp. 131-134.

\item \versal{CLARKE}, Helen. Sociedade, realeza e guerra. In: \versal{GRAHAM-CAMPBELL}, James.
\emph{Os Vikings}. Barcelona: Editora Folio \versal{S.A.} 2006, pp. 38-57.

\item \versal{SAWYER}, Birgit. \emph{The Viking Age rune-stones: custom and
commemoration in Early Medieval Scandinavia}. Oxford: Oxford University
Press, 2006.
\end{itemize}
\section{\versal{GRÁGÁS}}

As \emph{Grágás} ou as leis do \emph{Ganso Cinzento}, é como nos referimos
a uma coletânea de leis encontradas em mais de uma centena de códices,
alguns fragmentos de passagens e cópias de antigos manuscritos  
produzidos em diferentes momentos na Islândia. A primeira vez que o
termo \emph{Grágás} surgiu no contexto da cultura escrita islandesa foi
como parte de um inventário feito em 1548 sobre o arcebispado em
Skálholt. Apesar do termo surgir apenas no século~\versal{XVI}, acredita-se que o
conteúdo pertence a um período muito mais antigo da história da
Islândia.

De modo geral, a origem desses escritos pode ser datada do século~\versal{XIII},
com leis provenientes do fim do século~\versal{XI} e início do século~\versal{XII},
frações de tempo pertencentes aos períodos anteriores e posteriores ao
cristianismo na ilha. O acesso a essas leis pode ser feito através de
dois manuscritos que constituem o seu núcleo: o \emph{Livro Real},
\emph{Konungsbók}, e o \emph{Livro de Staðarhóll}, \emph{Staðarhólsbók}.

As diferenças entre esse corpo de leis e os outros exemplos encontrados
no restante da Escandinávia podem ser apontadas em vários aspectos: não
há preocupações com disposições militares, deveres e taxações sobre a
manutenção do território islandês ou ainda com os procedimentos
executivos das penas previstas nas leis. Ademais, casos de punições
físicas, incluindo a morte, não são levadas a cabo por ninguém em
especial, apenas autorizando e acobertando legalmente o vencedor do
caso, a punir ou matar a outra parte sem sofrer consequências a longo
prazo. Vale salientar que as vinganças familiares que tomam parte
significativa dos enredos das sagas dos islandeses, também ocupam seções
importantes das leis do \emph{Ganso Cinzento}.

O primeiro manuscrito (\emph{Konungsbók}), leva esse nome porque
pertenceu a Coroa Dinamarquesa. O segundo (\emph{Staðarhólsbók}), foi
assim nomeado em homenagem à fazenda onde foi encontrado no século \versal{XVI}.
Datados da metade do século \versal{XIII}, os manuscritos foram executados em
páginas de couro de gado e se encontram bem preservados, possibilitando
examinar a sua escrita com ornamentação policrômica, chamando a atenção
para o fato de que constituem materiais e técnicas construtivas bem
caros para a sua época.

Há menções sobre códigos escritos na Islândia anteriores aos escritos do
\emph{Ganso Cinzento}, a exemplo da passagem no \emph{Livro dos Islandeses},
\emph{Íslendingabók}, onde é dito que as leis islandesas foram pensadas
ainda na Era Viking, quando um colono de nome Úlfljót navegou até a
Noruega para conseguir um modelo pertinente às necessidades legais
islandesas, provavelmente se baseando em códigos legais aos quais um dos
exemplos são as \emph{Leis da Assembleia de Gula}, \emph{Gulaþingslǫg}, que
teriam sido adaptadas e conhecidas como as \emph{Leis de Úlfjót},
\emph{Úlfljótslǫg}.

Não temos razões para acreditar que esse acontecimento narrado no \emph{Livro
dos Islandeses}, que deve ter acontecido no início do século~\versal{X}, possa ter
ocorrido de fato, já que as \emph{Leis da Assembleia de Gula}, provavelmente
são posteriores ao estabelecimento da Assembleia Geral, \emph{Alþing},
porém, seja qual for o caso, a verdade é que se as \emph{Leis de Úlfjót}
existiram, não temos qualquer acesso na atualidade.

No período nos quais foram escritos, houve um interesse especial nas
reformas legais pertinentes às relações entre a Islândia e a Noruega,
porém não há nada que suporte a ideia de que tais manuscritos foram
produzidos como uma ação direta dos escritores legais noruegueses que
preceituaram normas para os colonos da ilha. Entretanto, podemos pensar
que esses islandeses estavam muito atentos ao aumento da intervenção que
a Coroa Norueguesa propunha sobre eles a partir de 1220.

É possível que alguns islandeses tenham sentido a necessidade de
registrar por escrito como eram as suas leis e seus costumes legais
antes da submissão à autoridade norueguesa. Adicionalmente, todo o
aparente caos do material encontrado, repetições, referências a outros
registros, inconsistências e desorganização textual, sugerem que esse
esforço tenha sido resultado da pressa em agregar todo esse conhecimento
nos códices dos quais o \emph{Livro Real} e o \emph{Livro de Staðarhóll} são exemplos
e provavelmente também estejam baseados em manuscritos anteriores.

Sobre o conteúdo das duas fontes citadas, em alguns momentos as
passagens encontradas são idênticas e em outras, apesar das semelhanças,
a construção se dá de modo diferente; também é possível encontrar algumas
leis que não existem no outro manuscrito. Tais diferenças são
reflexos das escolhas dos escribas que envolvem a preferência do
conteúdo e o acesso às fontes de sua época. Alguns pesquisadores
apontam, como uma possível origem dessas leis escritas, um códice
escrito pelo \emph{góði} islandês Hafliði Másson, de nome \emph{Compilado de
Hafliði} ou \emph{Hafliðaskrá}, cujo o manuscrito foi perdido.

Não necessariamente os manuscritos que contêm as leis do \emph{Ganso
Cinzento} foram copiados diretamente dessa compilação, mas sugerem que
devem ter se inspirado diretamente da memória evocada pelos oradores das
leis, uma vez que recitavam as leis anualmente, de modo que
provavelmente esse conhecimento legal deveria ser de um grupo reduzido
de pessoas que usavam as leis em favor próprio na hora de travar embates
jurídicos, advogando em favor de seu grupo familiar, pessoas próximas e
agregados, e não do domínio de toda a população islandesa.

Ainda que as leis do \emph{Ganso Cinzento} sejam um importante recurso para
nos situar acerca dos procedimentos jurídicos medievais islandeses,
oferecendo referências aos estudos das sagas islandesas (principalmente
nas \emph{Sagas dos Islandeses} e nas \emph{Sagas dos Sturlungos}), ilustrando com
muitas informações acerca das instituições sociais a administração da
ilha, elas não nos mostram como ambas as coisas de fato operavam em
conjunto.

Sabemos como as cortes, os encontros, os elementos sociais funcionavam,
sua periodicidade e composição, mas não como realmente eram conduzidos e
quais entraves poderiam surgir durante a sua orientação. Tais
preocupações são importantes quando refletimos acerca da realidade a
que se referem os manuscritos, e as possibilidades de investigação com
outras fontes, sobretudo literárias, tão ricas na Islândia.

É facilmente detectado também um corpo de leis consuetudinárias nas
diversas sagas, ausentes nos manuscritos, expondo assim um mundo legal,
ricamente composto por diretrizes que não estavam contidas nos
manuscritos aos quais temos acesso. Vale apontar que desse modo as leis
do \emph{Ganso Cinzento} representam uma parcela do que pode ter sido essa
vivência, mais um desejo enérgico de preservar, por uma elite social e
intelectual, a memória de uma época que ficou no passado e menos uma
constituição, ou uma carta, que englobe todos os costumes dos islandeses
durante a Era Viking.

\SIG{Pablo Gomes de Miranda}

Ver também Godi; Islândia da Era Viking; Sociedade.

\begin{itemize}
\item \versal{BYOCK}, Jesse. \emph{Viking Age Iceland}. London: Penguin Books, 2001.

\item \versal{DENNIS}, Andrew Ian. \emph{Grágás: an examination of the content and
technique of the old Icelandic Law Books, Focussed on Þingskapaþáttr
(the 'Assembly Section')}. Setembro, 1973. Tese (Doutorado em
Direito). Cambridge: Universidade de Cambridge, 1974.

\item \versal{JAKOBSSON}, Sverrir. The Territorialization of Power in the Icelandic
Commonwealth. In: \versal{BAGGE}, Sverre \emph{et al}. \emph{Statsutvikling i
Skandinavia i Middelalderen}. Oslo: Dreyers Forlag, 2012, pp.
101-118\textbf{.}

\item \versal{MCGLYNN}, Michael P. Orality in the Old Icelandic Grágás: legal formulae
in the Assembly Procedures Section. \emph{Neophilologus}, n. 93, 2009,
pp. 521-536.
\end{itemize}
\section{\versal{GRANDE ARMADA DANESA (866-878)}}

As invasões vikings à Inglaterra se iniciaram no final do século \versal{VIII} e
se mantiveram de forma esporádica, com algumas pausas de anos entre uma
expedição e outra até o ano de 850, quando o primeiro acampamento
nórdico de inverno foi estabelecido na ilha de Yahnet, na foz do Tâmisa.
Até então as expedições vinham durante o verão, saqueavam e iam embora,
retornando no ano seguinte. Isso dava tempo para os reinos saxões se
preparar para um próximo ataque. No entanto, a partir de 850, alguns
chefes dinamarqueses começaram a optar por permanecer o inverno na ilha e
prosseguir com suas incursões de pilhagem. Foi por essa época que
começaram a extorquir os saxões, cobrando-lhes o \emph{danegeld} sob a alegação de
que não os atacariam.

Por quinze anos essas práticas de acampamentos de inverno e extorsão
pelo \emph{danegeld} mantiveram-se, mas o ano de 865 foi diferente. De
acordo com a \emph{Crônica Anglo-saxã}, no ano de 865, a população de
Kent, na Ânglia Oriental, havia pagado o \emph{danegeld} como de costume,
mas, por motivos desconhecidos, durante a noite os dinamarqueses atacaram
a cidade, matando e roubando. No ano seguinte, para o horror da população
da Ânglia Oriental, um exército desembarcou na costa daquele pequeno
reino saxão.

A \emph{Crônica Anglo-saxã} refere-se àquele contingente militar pela
expressão \emph{micel here}, termo de difícil tradução, que em geral é
vertido como ``grande exército''. Embora a \emph{Crônica} não cite
quantos homens compunham aquela força de invasão, alguns historiadores
sugerem, com base na menção de navios e possivelmente pelo número de homens
que comportava cada embarcação, que o tal grande exército poderia
ter tido algo em torno de mil homens.

A chegada desse poderoso exército na costa da Ânglia Oriental seria fator
de mudança na história dos reinos saxões e escandinavos. Uma série de
campanhas, mantidas ao longo de doze anos, levaria ao surgimento dos
reinos nórdicos na Inglaterra.

Após permanecer algum tempo na Ânglia Oriental, a ponto de receber novos
tributos através do \emph{danegeld} e até mesmo mantimentos e cavalos, o
exército danês marchou para o norte, indo conquistar a cidade de York,
então capital da Nortúmbria, reino governado pelo rei Ælla \versal{II}. Com a
conquista de York, o exército marchou em direção ao Reino da Mércia,
montando acampamento de inverno próximo a Nottingham. No ano de 868,
travou-se batalha contra as tropas de Mércia, resultando na derrota destas.

O exército retornou para York e de lá marchou, no final de 869, de volta para a
Ânglia Oriental, assassinando naquele ano o rei Edmundo. Em seguida, as
tropas marcharam para o sul da Mércia e posteriormente adentraram o
Reino de Wessex. Após sofrerem algumas derrotas em Wessex, o grande
exército se retirou. Em 872, novos ataques voltaram a ocorrer: Londres
e Torksey foram saqueadas. Em 873, foi a vez de Rapton, e no ano seguinte
Cambridge foi alvo dos daneses, que de lá partiram para tentar
conquistar Wessex.

A segunda investida contra Wessex foi mais demorada, durando quatro
anos, mas apesar de uma série de batalhas, os daneses não conseguiram
derrotar o último reino saxão ainda livre. Com isso, em 878, após a
batalha de Ethandun, o rei Alfredo, o Grande de Wessex estabeleceu uma
trégua com o chefe Guthrun. Alfredo reconhecia o direito de Guthrun em
reinar sobre a Ânglia Oriental e se comprometia em manter o
\emph{danegeld}, em troca Guthrun comprometia-se em não invadir Wessex.
A partir de tal tratado, o Danelaw, termo que os saxões usavam para se
referir aos domínios daneses (dinamarqueses) na Inglaterra, teve início.

Os motivos para o ataque do grande exército ou grande armada não
são claros, inclusive se desconhece de quem teria sido a ideia para
empreender ousada campanha para conquistar territórios na Inglaterra.
Todavia, alguns relatos da época apontam que o motivo do ataque dos
nórdicos deveu-se ao intuito de vingar a morte de Ragnar Lothbrok. Três
supostos filhos do herói, Ivar Sem Ossos, Halfdan e Ubba, teriam
incentivado chefes dinamarqueses e noruegueses a formar uma coalizão.
Segundo a \emph{Saga de Ragnar Lothbrok}, o herói foi executado num poço
de cobras pelo rei Ælla da Nortúmbria e com isso seus três filhos teriam
liderado um poderoso exército para vingar a morte do pai e conquistar a
Inglaterra.

\SIG{Leandro Vilar Oliveira}

Ver também Dinamarca da Era Viking; Era Viking; Inglaterra da Era
Viking; Viking.

\begin{itemize}
\item \versal{ANÔNIMO}. \emph{The Anglo-saxon Chronicle}. Trad. Rev. James
Ingram. London: Everyman Press Edition, 1912.

\item \versal{GRAHAM-CAMPBELL}, James (org.). \emph{Os viking}s. Barcelona: Folio \versal{S.A.},
2006.

\item \versal{HEATH}, Ian; \versal{MCBRIDE}, Angus. \emph{The Vikings}. London: Osprey
Publishing, 1985 (Elite Series, vol. 3).

\item \versal{KEYNES}, Simon. The Vikings in England, c. 790-1016. In: \versal{SAWYER}, Peter
(ed.). \emph{The Oxford Illustrated History of the Vikings}. New York:
Oxford University Press, 1997, pp. 48-82.
\end{itemize}
\section{\versal{GRETTIS SAGA}}

A \emph{Grettis saga Ásmundarsonar} (\emph{Saga de Grettir, filho de Ásmundr}),
também conhecida como \emph{Grettis sterka saga} (\emph{Saga de Grettir, o
Forte}), é uma das mais famosas \emph{Íslendingasögur} juntamente com
\emph{Egils Skallagrímssonar saga}, a \emph{Njáls saga} e \emph{Laxdæla
saga}. No entanto, a \emph{Saga de Grettir, o Forte} difere destas outras sagas dos islandeses por fazer
uso mais frequente dos seres sobrenaturais (fantasmas, \emph{trolls},
\emph{berserkir}) e vários motivos folclóricos. Alguns autores a incluem na categoria de \emph{útilegumannasögur} (sagas dos foragidos), na qual também se inclui a \emph{Gisla Súrssonar} (\emph{Saga
de Gisli, filho de Surr}).

A \emph{Grettis saga} foi composta tardiamente, no início do século \versal{XIV},
por um autor desconhecido que coletou diversas fontes orais e escritas
sobre a figura histórica de Grettir Asmundarson, que provavelmente
nasceu por volta do ano 1000 em Bjarg (Islândia do Noroeste). Atualmente,
existem mais ou menos quatro pergaminhos da saga completos, todos
pertencentes aos séculos~\versal{XV}~e~\versal{XVI}: \versal{AM} 551 \versal{A} 4o, 4o \versal{AM} 556, \versal{AM} 152 fol. e
Delagardie 10 fol. Além disso, em torno de 40 manuscritos também são
preservados em papel, dentre os quais se destacam o \versal{AM} 150 fol. e \versal{AM} 151
fol.

A Grettis consiste de 93 capítulos divididos em duas seções principais:
a primeira seção se estende do capítulo 1 ao 82 e narra desde a
juventude de Grettir até sua morte na ilha de Drangey, que provavelmente
ocorreu entre os anos de 1030-1040. Por sua vez, essa primeira seção é
subdividida em duas partes: uma que conclui com a afirmação de Grettir
como foragido (capítulo 46) e outra que descreve seus dezenove anos de
banimento. Grettir, que foi mencionado pela primeira vez no capítulo 14,
mostra desde jovem um comportamento muito briguento (igual ao seu
próprio pai Ásmundr) culminando na morte de um servo, o que o
condena a três anos no exílio no capítulo 17.

Durante a sua estada na Noruega, Grettir luta contra um fantasma (capítulo 18), caracterizando seu primeiro encontro com seres sobrenaturais. Então mata o
malvado Bjǫrn e seu irmão, sendo, portanto, forçado a deixar a Noruega e
voltar para a Islândia. Na Islândia, retorna para lutar com outro
fantasma e sua vitória sobre o mesmo faz dele o homem mais forte deste país. No
entanto, antes de sua morte, o fantasma lança sobre Grettir uma tripla
maldição: que sua força não aumentará mais, que o infortúnio o
assombrará e que ficará com medo ao estar sozinho no escuro. Na
verdade, esse infortúnio o fará perder o favor do rei Olavo Haraldsson
da Noruega, que Grettir estava buscando em recompensa por sua bravura.
Quando volta à Islândia, depois de sua recente estadia na Noruega,
Grettir descobre que o seu pai e seu irmão Atli Ásmundr estão mortos e
ele foi declarado fora da lei, o que significa que qualquer um poderia
matá-lo com impunidade. Isso faz com que realize um longo período de
viagens e aventuras na Islândia para evitar cair nas mãos de seus
inimigos. Algumas dessas aventuras têm fortes origens
populares, além da presença de criaturas sobrenaturais, como Thorir,
metade homem metade \emph{troll}, com quem Grettir passa alguns invernos. A
partir do capítulo 69, encontramos Grettir junto com seu irmão Illugi e
um servo chamado Glaumr na ilha de Drangey, localizada no norte da
Islândia, onde permaneceu por três anos, até sua morte nas mãos de seus
inimigos como uma consequência indireta da maldição que a bruxa Þuríðr
tinha lhe lançado no capítulo 78.

A segunda seção principal da série, chamada \emph{Spesar þáttr} (``Relatos
curtos de Spes''), tem lugar em Bizâncio e se estende desde o capítulo 83
ao 93. Se trata de um þáttr com alguns elementos cômicos onde o
protagonista é Þorsteinn o Dromón, que vinga seu irmão Grettir
postumamente. Nesta seção têm sido vistas influências da lenda artúrica
de Tristão.

Além dessa influência arturiana, alguns especialistas também observaram
outras conexões possíveis entre a \emph{Grettis saga} e algumas famosas obras
da literatura europeia medieval. Assim, Fjalldal concebe de um
ponto de vista crítico vários paralelos entre a \emph{Grettis saga} e o
anglo-saxão \emph{Beowulf}, como, por exemplo, o episódio de matar o \emph{troll}
em Bárðardalr nos capítulos 64-66 da \emph{Grettis saga} e as façanhas de
Beowulf contra Grendel e sua mãe. Por sua vez, a aventura sexual de
Grettir em Reykir (capítulo 75) tem certas semelhanças com uma cena de
\emph{Decameron} de Boccaccio.

A exemplo de outros personagens famosos de muitas sagas islandesas,
Grettir também tem numerosos \emph{rímur}, como Magnus Laugum (século~\versal{XVIII}), que
dá a devida consideração para suas aventuras e destino.

\SIG{Mariano González Campo}

Ver também Literatura; Norreno; Sagas islandesas.

\begin{itemize}
\item \versal{FJALLDAL}, Magnús. \emph{The Long Arm of Coincidence. The Frustraed
Connection between Beowulf and Grettis saga.} Toronto: University of
Toronto Press, 1998.

\item \versal{GLENDINNING}, Robert J. \emph{Grettis saga} and European Literature in
the Late Middle Ages. \emph{Mosaic}, vol. 4, 1970, pp. 49-61.

\item \versal{HASTRUP}, Kirsten. Tracing Tradition: An Anthropological Perspective on
\emph{Grettis saga Ásmundarsonar}. In: \versal{LINDOW}, John \emph{et al}.
(eds.). \emph{Structure and Meaning in Old Norse Literature: New
Approaches to Textual Analysis and Literary Criticism}. Odense: Odense
University Press, pp. 281-313.

\item \versal{JÓNSSON}, Guðni (ed.). \emph{Grettis saga Ásmundarsonar}. Reykjavík: Hið
Íslenzka Fornritafélag, 1986.

\item \versal{KRAMARZ-BEIN}, Susanne. Der \emph{Spesar þáttr} der \emph{Grettis saga}.
Tristan-Spuren in der Isländersaga. In: \versal{BECK}, Heinrich; \versal{EBEL}, Else
(eds.). \emph{Studien zur Isländersaga. Festschrift für Rolf Heller}.
Berlim/New York: Walter de Gruyter, 2000, pp. 152-181.
\end{itemize}
\section{\versal{GROENLÂNDIA NÓRDICA}}

A Groenlândia passa a compor o mundo nórdico desde os registros das
terras relatadas por Gunnbjörn, filho de Úlfr Corvo, se pensarmos em uma
dinâmica literária, ou desde 1053 com uma carta papal. A partir desse
relato podemos pensar um início da presença nórdica, através de
Eiríkr Þorvaldsson ou Érico, o Vermelho (Eiríkr hinn
rauði), que em uma fuga da Islândia, devido à proscrição, chega à
Groenlândia, que passará cada vez mais a compor o mundo nórdico, sendo
fator fundamental para a presença nórdica na América do Norte.

Para compreendermos a Groenlândia nórdica, devemos ter por base o
\emph{Íslendingabók} (\emph{Livro dos Islandeses}), o \emph{Landnámabók} (\emph{Livro
da Colonização}) e as \emph{Sagas do Atlântico Norte}. Esse conjunto de fontes
não é a única possibilidade para se pensar sobre essa localidade, mas
acreditamos serem as fontes mais populares e amplas, no sentido de
entender o processo de descoberta e colonização da Groenlândia, como
também entender a visão que os nórdicos tinham daquilo que encontravam
nessa nova terra.

Os homens do Norte, que colonizaram essa região, dividiram-na em duas
habitações, que representam as localidades de colonização: uma na região
sul da ilha (\emph{Eystribyggð} -- a Habitação Oriental); outra
localizada mais a noroeste (\emph{Vestribyggð} -- a Habitação
Ocidental). A colonização da Groenlândia teve uma duração
mais específica: durou alguns séculos, diferentemente do que ocorrera
com a Islândia. A chegada Érico, o Vermelho, em 985, junto do ápice de uma
colônia com uma variação entre três e cinco mil habitantes e da fundação,
em 1126, da diocese de \emph{Garðar}, foram elementos ímpares, mas que
não fortes o suficiente para manter uma resiliência dos
assentamentos nórdicos na ilha. Em meados de 1350, a Habitação Ocidental
é abandonada, tanto pela pressão dos povos locais, quanto por grandes
forças climáticas, motivos que também levaram, em meados do
começo do século~\versal{XV}, à ocorrência do mesmo processo na Habitação
Oriental. Portanto, ao se falar de uma Groenlândia nórdica, fala-se de um
recorte temporal que se inicia nos fins do século~\versal{X} e que caminha até o
começo do século~\versal{XV}.

Partindo para uma análise das \emph{Sagas do Atlântico Norte}, temos relatos
similares, pois ambas as sagas trazem a questão da proscrição de Érico
como um fator que estimula sua saída da Islândia em busca de uma nova
terra relatada, à qual apenas catorze dos vinte e cinco navios que saíram da Islândia
chegaram. As \emph{Sagas dos Groenlandeses},
assim como o \emph{Landnámabók}, trazem uma questão familiar forte e
apresentam os sujeitos e suas famílias que assumem certas propriedades e
localidades na Groenlândia.

Outro fator que devemos levar em conta, é o nome da Groenlândia,
\emph{Grænland}, que significa Terra Verde, e que, segundo a explicação
da saga supracitada, se deve a uma estratégia de Érico para atrair
pessoas para a nova terra, pois a sua boa denominação seria facilitadora
para sua colonização.
Ari Þorgilsson (Ari hinn fróði -- Ari, o Sábio), o autor do
\emph{Íslendingabók} e do \emph{Landnámabók,} concorda com essa
prerrogativa da saga, algo que Adão de Bremen, na \emph{Gesta
Hammaburgensis ecclesiae pontificum} (\emph{História dos Bispos de Hamburgo e
Bremen}), vai se contrapor. Para este último, o nome Terra Verde se deve à
coloração do mar que rodeia a ilha, especialmente na região sul, em que
encostas verdejantes e arbustos são observados, portanto, o nome pode
não ter esse fundo tão comercial como relatado nas duas primeiras
fontes.

Ari, o Sábio, no capítulo sexto de sua obra, confirma Érico como
o descobridor da Groenlândia, assim como comenta sobre os habitantes da
ilha, algo que é fundamental para uma visão nórdica da Groenlândia. Eles são chamados de
\emph{skrælingjar}, que podemos pensar como esquimós, ou inuítes -- estes
últimos quando se fala mais da Groenlândia --, algo que vai ao encontro
dos relatos de outras fontes, que apresentam esses esquimós de várias
maneiras e com variados tipos de interação. Além disso, nos revela que
entre quatorze e quinze anos após o cristianismo chegar à Islândia (c.
1000) o processo de colonização movido por Érico se inicia.

A Groenlândia vai ter mais presença em fontes a partir
de 1053, quando encontramos sua citação em uma carta do Papa Leão \versal{IX}
para o arcebispo Adalberto de Bremen e posteriormente para o já citado
Adão de Bremen. Essa descoberta revela novas possibilidades, uma nova
terra com peculiaridades, que somente em 1261 irá aceitar a soberania da
Noruega e em 1380 a da Dinamarca, quando este primeiro país se une com o
último.

Pensar em uma Groenlândia nórdica é pensar em uma nova possibilidade que
surgiu no fim do século \versal{IX} e que trouxe uma dimensão ímpar ao permitir
as expedições para a América do Norte e o contato com
novos povos e culturas, assim como foi uma porta de entrada para o mundo
nórdico para aqueles que já não mais faziam parte das principais
regiões, seja por proscrição ou porque não se identificavam mais
politicamente com a localidade de onde vinham. Durante séculos, a ilha,
mesmo com sua paisagem gélida, poucas áreas de solo fértil e de uma
vegetação relativamente parca, foi um espaço de renovação do mundo
nórdico, um espaço que deixa claro o ímpeto de explorador e colonizador
dos nórdicos.

\SIG{José Lucas Cordeiro Fernandes}

Ver também Brattahlid; Esquimós e nórdicos; Leif Erikssson; Sagas do
Atlântico Norte.

\begin{itemize}
\item \versal{ANÔNIMO}: \emph{As três sagas Islandesas}. Tradução de Théo Moosburger.
Curitiba: Editora \versal{UFPR}, 2007.

\item \versal{ARNEBORG}, Jette. The Norse Settlements in Greenland. In: \versal{BRINK}, Stefan;
 \versal{PRICE}, Neil (eds.). \emph{The Viking world.} London: Routledge, 2012,
pp. 588-597.

\item \versal{BARNES}, Geraldine. \emph{Viking America: The First Millennium}.
Cambridge: D.S. Brewer, 2001.

\item \versal{BERGERSEN}, Robert. \emph{Vinland Bibliography: Writings Relating to the
Norse in Greenland and America}. Tromsø: University of Tromsø, 1997.

\item \versal{BREMEN}, Adam of. \emph{History of the Archbishops of Hamburg--Bremen}.
New York: Columbia University Press, 2002.

\item \versal{GWYN}, Jones.~\emph{La saga del Atlántico Norte: establecimiento de los
vikingos en Islandia, Groenlandia y América}. Barcelona: Oikos-Tau, \versal{S.A.}
Ediciones, 1992.

\item \versal{SHAFER}, John Douglas. \emph{Saga accounts of norse far-travellers}.
Durham: Durham University, 2010.

\item \versal{THORGILSSON}, Ari. The Book of Settlements. \emph{Landnámabók.} Trad.
Hermann Polsson and Paul Edwards. Canada: University of Manitoba Press,
2007.

\item \versal{THORGILSSON}, Ari; \versal{ANÔNIMO}. \emph{Íslendingabók, Kristni Saga: The book
of the icelanders, the story of the conversion}. Trad. Sion Gronlie.
Viking Society for Northern Research: University College of London,
2006.

\item \versal{VIGFUSSON}, Gudbrand; \versal{YORK POWELL}, F. \emph{Origines Islandicae: A
Collection of the More Important Sagas and Other Native Writings
Relating to the Settlement and Early History of Iceland}. Oxford:
Clarendon, 1905.
\end{itemize}
\section{\versal{GUDRID THORBJARNARDÓTTIR}}

Gudridur Thorbjarnardottir nasceu na Islândia, no final do século~\versal{X} e
era filha de fazendeiros. Uma de suas avós teria sido provavelmente uma
escrava, capturada na Irlanda ou na Escócia. As sagas narram que seu
primeiro amor era um jovem chamado Einar, mas seu pai recusou sua
permissão para se casar porque o pretendente era filho de um escravo.
Após esse incidente, viajou com o pai para a Groenlândia, para acompanhar
Érico, o Vermelho, que se instalou naquelas terras com mais pessoas a fim
de formar uma colônia. Em terras groenlandesas ela se casou com um dos
filhos de Érico, Thorstein. Após o matrimônio, Gudrid e Thorstein partem
em expedição com destino à América do Norte, onde o irmão do marido,
Leif Eriksson, esteve e tentou se instalar alguns anos antes. No tempo
que esteve na América do Norte, Gudridur teve um filho, mas, durante a
viagem de volta da família para a Groenlândia, Thorstein faleceu. Gudrid
permaneceu na Groenlândia e se casou novamente, desta vez com um comerciante
chamado Thorfinnr Karlsefni. Em 1010, lideraram outra expedição para a
América do Norte, com três navios e 160 colonos e, ao
desembarcarem, nomearam aquelas terras de Vínland. Enquanto estavam na
América do Norte, Gudrid deu à luz um filho, Snorri Thorfinnsson, que é
o primeiro europeu conhecido a nascer nas Américas. Eles voltaram para a
Groenlândia depois de dois anos, pois o número de nativos era superior ao
de colonos e a colônia não prosperava devido ao clima de insegurança em
que viviam. Posteriormente, o marido de Gudrid morreu, ela ficou viúva
pela segunda vez e seu filho herdou as terras do pai na Groenlândia.
Todos esses acontecimentos se passaram na mesma época em que tanto a
Islândia como a Groenlândia se tornaram cristãs, e possivelmente recém-convertida 
e acometida de um grande fervor na nova religião, Gudrid
decidiu fazer uma peregrinação. Já estava viúva, era detentora de uma
boa fortuna e seus filhos já tinham a sua própria família, nada a
prendia às terras groenlandesas.

Gudrid, então, decidiu mais uma vez partir. Nessa nova jornada viajou
para Roma, onde conheceu o Papa e depois voltou para a Groenlândia para a
igreja que seu filho tinha construído para ela na propriedade da família,
onde terminou os seus dias como freira. A viagem de peregrinação a Roma provavelmente não era tão
fisicamente perigosa e desgastante quanto a viagem marítima para a
América do Norte desconhecida, através do extremo oeste da Groenlândia e,
por volta do ano 1000, Gudrid Thorbjarnardottir foi seguramente a mulher
mais bem viajada do mundo, e permaneceria assim mais alguns anos.

Todas as informações que temos a respeito de Gudrid e de suas viagens,
tanto de colonização como de peregrinação, vêm da \emph{Saga dos Groenlandeses},
composta no século~\versal{XIV} e de autoria anônima. Nessa narrativa Gudrid é
descrita como uma mulher muito bela, de maneiras gentis e que é amada
por todos. As atitudes dessa mulher, que atravessou o mar em busca de
novas terras para colonização, não diferem em nada das atitudes
masculinas, pois todos estavam em busca de um novo local para se
estabelecer e corriam os mesmos riscos, mostrando que durante as
expedições de colonização todos estavam sujeitos às mesmas regras e,
principalmente, às mesmas condições, por vezes inóspitas, de vida.

Atualmente, a personagem Gudrid ainda desperta imensa curiosidade e
aguça a criatividade de artistas, como a da escritora escocesa Margaret
Elphinstone, que em 2001 escreveu um romance sobre Gudrid, intitulado
\emph{Gudrid -- The Sea Road}, e de Nancy Marie Brown, que escreveu uma
biografia sobre Gudrid intitulada \emph{The Far Traveler, Voyages of
Viking Woman}, publicada em 2007. Mais de mil anos depois, Gudrid ainda
inspira viagens.

\SIG{Luciana de Campos}

Ver também Aud; Freydis; Mulheres; Sociedade.

\begin{itemize}
\item \versal{JESCH}, Judith. \emph{Women in the Viking Age}. London: Boydell \& Brewer
Ltd, 1999.

\item \versal{JOCHENS}, Jenny. \emph{Women in Old Norse Society}. Ithaca:
\href{http://www.cornellpress.cornell.edu/publishers/?fa=publisher\&NameP=Cornell\%20University\%20Press}{Cornell
University Press}, 1995.

\item \versal{McLEOD}, Shane. Warriors, and women: the sex ratio of Norse migrants to
eastern England up to 900 \versal{AD}. \emph{Early Medieval Europe}, vol. 19, n. 3, 2011,
pp. 332-353.
\end{itemize}
\section{\versal{GUERRA E SIMBOLISMO}}

Bastante conhecidos por valorizarem suas proezas militares, para Keegan,
os nórdicos estavam entre os povos mais belicosos e resistentes que
tomaram de assalto o mundo europeu continental. Com sua disposição para
a luta corpo a corpo intensificada no século de disputas por terras que
precedeu sua era de viagens, os vikings constituíram uma atividade
bélica das mais famosas da Idade Média. Os seus feitos em batalhas, em
pirataria e nas expedições pelo mundo colaboraram para fazer a sua fama
até nossos dias. Algumas proezas e o ímpeto marcial estão destacados em
alguns relatos, assim como a figura do ``furor guerreiro'' no transcurso
de diversas batalhas, o que acabou fazendo com que fossem, além de
outros aspectos, caracterizados como povos bárbaros.

Organizada, devastadora e muitas vezes rápida: assim era a forma de
guerra praticada pelos escandinavos durante a Era Viking. Isso fica
claro na literatura que vai ser escrita sobre eles, sempre contendo
algum aspecto relacionado à guerra. Antes de se lançar em incursões, os
vikings se preparavam para os combates usando, entre outras coisas,
táticas militares que depois seriam vistas sendo utilizadas por exércitos na Era Moderna.


A preparação para a guerra se constituía desde considerações sobre o tipo de embarcação que seria
utilizada, passando pelos rituais oferecidos e favores pedidos aos
deuses, e chegando ao equipamento que seria usado. Usavam, principalmente nas
pilhagens, pequenas tropas com conhecimento prévio do terreno e do inimigo.
Os ataques relâmpagos e o sucesso nas pilhagens, todavia, eram possíveis não só
pela estratégia ou pelo conhecimento do terreno do inimigo, mas também
pelo tipo de embarcação usada pelos vikings nas incursões. O
\emph{langrskip} era um navio
longo, que chegava até 55 metros, e que era usado especificamente para a guerra. Com
velocidade média de até 18 km/h, era vantajoso no tamanho, velocidade e na
facilidade de transporte.

Além de estratégia e transporte, os armamentos usados na guerra também
eram fundamentais. Basicamente, os vikings usavam espadas, lanças, facas, arco e
flecha, machados, elmos e escudos. As espadas eram mais comumente
usadas por nobres, pelo alto custo pago para a forja, mas não se
restringia a eles -- a diferença se dava nos adornos
de cada uma, denotando o \emph{status} do guerreiro (as espadas ricamente
adornadas desde o punho até a bainha pertenciam aos guerreiros de
\emph{status} mais elevado). As espadas eram bastante eficazes não só
pelo alcance, mas pelo ferimento causado. Elas eram bens bastantes
preciosos e que iam muito além do caráter puramente bélico. Foram
intimamente associadas à justiça, soberania e poder, fazendo-se extremamente
significativas nas relações entre os homens que as portavam. Elas
definiam o valor de um homem, individual e
coletivamente. Algumas eram passadas de geração a geração e outras cremadas
junto com o guerreiro, segundo a crença de que todos os objetos queimados na pira
com seus donos seguiam com eles até Valhala, dando a oportunidade de o guerreiro morto em batalha lutar ao lado de Odin no Ragnarök.

Os guerreiros mantinham com as espadas uma relação que era de extrema
confiança e até respeito, já que toda a sobrevivência nas batalhas
advinha de seus armamentos, e principalmente porque para os povos
germânicos a batalha era uma questão individual. Daí algumas espadas
terem um nome (muitos nomes vão ser preservados em sagas e poemas) e até
mesmo uma personalidade, criando uma relação com o seu dono e chegando
mesmo a serem consideradas como uma extensão do próprio guerreiro. As
espadas podem ser consideradas um reflexo e até mesmo uma parte da
identidade desse povo, e vão estar sempre atreladas a cada dono. A
espada pode carregar vários discursos como, por exemplo, de ordem,
justiça e até mesmo desordem. Adquirem virtudes, personificam-se e podem
até encarnar países, dando corpo a valores ou ideais. Vão possuir vários
significados, costumes e seus donos terão um relacionamento íntimo e até
mesmo emocional com elas. Não poder contar com sua própria arma era uma
terrível maldição. A espada era vista como árbitro entre os homens. Até
nos casamentos germânicos realizados na Era Pré-cristã, ela estava
presente. Neles fincava-se uma espada em um tronco de árvore e o noivo
deveria retirá-la como símbolo de virilidade, associada à fertilidade
também.

As facas eram de uso cotidiano, sendo portadas pelas
mulheres. No campo de batalha, a eficácia se dava somente em combates
mais próximos homem a homem. As lanças eram usadas principalmente na
ofensiva. E como quase tudo que se fazia na guerra, ela também tinha
ligação com a religiosidade: logo no início da batalha arremessava-se uma
lança com o intuito de obter favores de Odin (deus associado à guerra).
Os escandinavos acreditavam que espadas, escudos e capacetes podiam ser
abençoados pelas divindades pagãs. O arco e a flecha, apesar de
marginalizados, tiveram bastante uso e importância nas batalhas por comporem
um armamento estratégico nas formações de batalha.

O machado era o armamento mais associado aos guerreiros vikings
por estar ligado aos \emph{berserkir} e ser bastante usado nas
incursões marítimas e na pirataria, ainda que não fosse tão utilizado nas
frentes de batalha. Elmos e escudos eram equipamentos de defesa. Os
primeiros eram em formato cônico, esférico e alguns apresentavam
proteção nasal, nada de asas ou chifres como se retrata frequentemente em \versal{HQ}s, livros,
revistas e filmes. Os escudos eram em formato circular, feitos de
madeira com uma faixa de ferro ao redor para dar maior segurança -- estes,
sim, são retrados com realismo na arte e mídia contemporânea. Nesse
âmbito de guerra, os \emph{berserkir} (\emph{berserkr} no singular) são os exemplos
mais famosos de guerreiros vikings. Estavam associados principalmente a
animais como o urso e o lobo (ao lobo se associavam os úlfhednar, úlfhedinn
no singular). O urso e o lobo foram vistos como símbolos de guerreiros
valentes. Os \emph{berserkir} lutavam com fúria assassina, possessos, sem qualquer proteção,
urrando e mordendo seus escudos tão ``loucos'' como lobos, avançando
contra seus inimigos usando machados (somente de uma face; era uma
arma sobretudo de ataque, sem defesa opcional), causavam um efeito
psicológico devastador no inimigo. Provocavam medo extremo com uma
agressão em estado puro, terrivelmente assustadora e de moldes
``suicidas''. O culto ao deus Odin (o próprio nome estava associado à
fúria, tanto no nórdico \emph{Ódr} quanto no germânico antigo
\emph{Wodan}), a fé em um deus xamânico, relacionado à magia, ao êxtase
e à metamorfose humana em animais, explicaria tal comportamento
agressivo. Não só os \emph{berserkir} causavam medo: outros
guerreiros traziam cabeças decapitadas de seus inimigos com o intuito de aterrorizar seus oponentes.

A cabeça possui um grande valor para muitos povos, simbolizando a
autoridade de governar, de ordenar e esclarecer. A cabeça simboliza a
força e o valor guerreiro do adversário e a decapitação garantia também
a morte desse mesmo adversário. Fala-se também de um poder de cura, em algumas
sociedades. Na literatura nórdica,
homens eram frequentemente decapitados em batalha ou em um ato de
vingança, após serem feitos prisioneiros. O culto ou a preservação de
cabeças humanas e crânios não foi observado somente entre os germânicos;
entre os celtas, as cabeças dos inimigos de grande valor eram
conservadas em azeites e trazidas em carros de guerra. Para eles,
simbolizava força e o valor do oponente que passava para quem a
possuísse.

Entre os nórdicos, a relação com cabeças decapitadas pode ser verificada em
algumas sagas. Muitas menções são feitas em contextos de guerra,
vingança ou busca de conhecimento. Em algumas, as cabeças falam com seus
portadores ou no campo de batalha, causando amedrontamento. Em uma
profecia na \emph{Njáls saga}, um ser sobrenatural evoca as imagens de
várias cabeças cortadas no campo de batalha. Usar cabeças cortadas como
meio adquirir conhecimento vem de uma tradição antiga de que Odin teria
consultado a cabeça de Mimir, que, segundo a \emph{Ynglinga saga}, teria
sido decapitada após este vanir ser feito prisioneiro. Após sua decapitação, a cabeça foi enviada de volta pra Odin, que a conservou com ervas para
que não apodrecesse e realizou uma magia para que ela lhe falasse sobre questões ocultas. O poder de uma cabeça cortada e sua
capacidade de falar após a separação do corpo foi observado tanto na
cultura germânica quanto na céltica.

Também germanos e celtas preservavam não só as cabeças, mas os crânios
dos mortos. Eles simbolizavam a sede do pensamento e do poder supremo.
Em algumas lendas europeias e asiáticas, os crânios humanos foram
considerados homólogos à abóbada celeste. No \emph{Grímnismál}, o crânio
do gigante Ymir se converteu, após sua morte, na abóbada do céu. Além da
busca por conhecimento, alguns crânios também foram
usados para vinganças. Em uma passagem da \emph{Saga dos Volsungos},
Gudrun utiliza os crânios de seus filhos com o rei Atli para se vingar
dele e humilhá-lo por tê-la utilizado para matar seus irmãos. O crânio
representa, ainda, o símbolo da mortalidade humana, mas também do que
sobrevive depois da morte, de modo que possuir o crânio do inimigo é mais que
um troféu, é a conquista do há de mais alto e de todo germe de
existência.

A importância desse uso da cabeça como sede de inteligência, juntamente
com seu uso como um troféu de batalha, trazendo a sorte e aumentando a
reputação de seu possuidor, é percebida nas primeiras tradições ligadas a
guerreiros e a batalhas entre os celtas e os germânicos; e ainda essa
concepção da cabeça como detentora de conhecimento e uso em campos de
batalha como forma de amedrontamento e troféus foi preservada na arte
(há muitas cabeças esculpidas e rostos semelhantes a máscaras como uma
força aterrorizante), nas sagas e nas lendas. Entre os povos germânicos,
a concepção do enforcado como meio de adquirir conhecimento oculto
ganhou maior proeminência, em relação aos celtas.

\SIG{Monicy Araujo Silva}

Ver também Espadas; Guerra e técnicas de combate; Religião.

\begin{itemize}
\item \versal{CHEVALIER}, Jean \& \versal{GHEERBRANT}, Alain.~\emph{Diccionario de los
Simbolos}. Barcelona. Editorial Herder, 1986.

\item \versal{DAVIDSON}, Hilda Roderick Ellis. \emph{Myths and symbols in pagan Europe:
early Scandinavian and Celtic religions}\textbf{.} Syracuse: Syracuse
University Press, 1988.

\item \versal{KEEGAN}, John. \emph{Uma história da Guerra}. São Paulo:
Companhia das Letras, 2006.

\item \versal{LANGER}, Johnni. Espadas míticas. In: \versal{LANGER}, Johnni (org.).
\emph{Dicionário de Mitologia Nórdica: mitos e ritos}. São Paulo: Hedra,
2015, pp. 169-172.

\item \versal{LANGER}, Johnni. \emph{Deuses, monstros, herois: ensaios de mitologia e
religião Viking}. Brasília: Editora Universidade de Brasília,
2009.

\item \versal{MIRANDA}, Pablo Gomes de. Berserkir\textbf{.} In. \versal{LANGER}, Johnni (org.).
\emph{Dicionário de Mitologia Nórdica: mitos e ritos}. São Paulo: Hedra,
2015, pp. 68-73.

\item \versal{PALAMIN}, Flávio Guadagnucci. \emph{O guerreiro Viking na Edda Poética:
religiões, mitos e heróis}\textbf{.} Maringá: Dissertação de Mestrado em
História, \versal{UEM}, Maringá, 2013.

\item \versal{SILVA}, Monicy Araujo. \emph{Fama, poder e prestígio: a espada na Saga
dos Volsungos e na Le Morte D´Arthur}. Monografia de bacharelado em
História. São Luís: \versal{UFMA}, 2014.
\end{itemize}
\section{\versal{GUERRA E TÉCNICAS DE COMBATE}}

Na cultura nórdica da Era Viking, como parte de uma cultura germânica,
a guerra ocupava função precípua, por meio da qual laços políticos,
econômicos e sociais eram estabelecidos ou refeitos a cada combate. A
sorte e a fortuna estavam dispostas no combate para aqueles que pudessem
sobreviver. Aos que não vivessem ao final do dia, um destino de fama
cobria o morto com uma glória que promovia o nome da família,
garantindo prestígio.

A favor dos vikings contava a capacidade da mobilidade proporcionada por
seus navios, com calado relativamente pequeno, bem como as marchas e
deslocamentos feitos em velocidades superiores a de forças oponentes no
mesmo período, em especial pela leveza dos equipamentos e a natureza de
suas ações.

A organização tribal das sociedades germânicas em um primeiro momento
foi a conformada das forças vikings em combate. Nos primeiros tempos da
Era Viking, os laços clânicos e tribais eram o elo que direcionavam os homens
e mulheres a seguirem em combate. Com o investimento em diversas áreas
do norte da Europa, em especial Irlanda e Inglaterra, esse tipo de
organização foi substituído por outro, construído ao redor do poder
político de chefes guerreiros ou grupos de homens de armas que ofereciam
serviços a quem pudesse pagar.

Tais grupos evoluíram em tamanho, treinamento e efetivo à medida que se
processaram as expedições e invasões vikings em diversas áreas. Um
desses grupos guerreiros foi formalizado através de um tratado entre os
rus de Kiev e o imperador bizantino em 874: era a Guarda
Varegue (Varangiana), uma unidade de elite do exército bizantino
subordinada diretamente ao imperador e encarregada de sua proteção,
valendo-se dos costumes germânicos de ligação entre o guerreiro e o seu
senhor, de fidelidade até a morte.

O treinamento dos guerreiros era feito por familiares ou, em casos mais
avançados, por membros experientes dos grupos guerreiros. Havia uma
série de armas ofensivas e defensivas à disposição e o
guerreiro em formação era treinado em todas, embora existisse,
naturalmente, uma preferência por um tipo de armamento com o qual se
obtivesse maior eficiência no manejo.

Espadas e machados eram as armas ofensivas principais, sendo as espadas
mais bem elaboradas garças aos custos envolvidos em sua
produção. Os machados eram mais baratos e ainda estavam
presentes no dia-a-dia, utilizados em diversas tarefas, o que
familiarizava muito o futuro guerreiro, pelo manejo diário, com as
funcionalidades da arma, que podia ser utilizada na mão ou,
dependendo do tamanho, arremessada, produzindo efeito devastador.

Assim como os machados, variando em tamanhos, as lanças eram outro tipo de
arma muito utilizada pelos vikings, sustentando formações de combate e
em arremessos, sendo populares como os machados, por causa do baixo
custo de produção. Apesar disso, lanças e machados não eram considerados
armas ``menores'' em importância em relação às espadas, sendo, assim como estas, decoradas
com motivos sagrados pagãos ou cristãos, e com papel
mitológico. Deuses portavam machados ou lanças como armas sagradas tanto quanto espadas.

O último grupo de armas ofensivas vikings eram arcos e flechas. Embora
não sejam muito conhecidos, existia o uso em larga escala por parte de
exércitos germânicos antes da Era Viking de arcos e flechas, para causar
baixas no inimigo, antes ou durante a batalha, e a Escandinávia não era
exceção nisto, ainda mais em combates navais, onde os arcos eram
decisivos para causar baixas nas tripulações inimigas e facilitar a
abordagem de navios hostis. Arcos também eram de grande valia em
expedições de pilhagem e saque, eliminando ou incapacitando um oponente
em relativa distância.

Em matéria de armamento defensivo, destacam-se os escudos e as proteções
individuais, como armaduras, completas ou não. Os escudos tinham
importância vital, e sem eles um homem não podia tomar parte em
expedições, pois só era permitido embarcar aqueles que portassem um
escudo. O escudo era composto basicamente por madeira, pintada em cores
diversas, com um pomo de ferro central, no qual o guerreiro o manejava,
podendo se valer do peso do corpo para sustentá-lo, quando da formação
defensiva mais clássica viking, a parede de escudos, no qual os
guerreiros colocavam os escudos lado a lado.

As armaduras, mais caras, estavam restritas a quem pudesse pagar pelo
fabrico, e o custo se refletia tanto na qualidade quanto na proteção
individual oferecida por elas, que variava do tronco e cabeça até uma
cobertura corporal completa. Os elmos possuíam formatos que ofereciam desde uma
proteção simples da parte superior da cabeça até uma proteção do todo do rosto e parte da nuca, tendo, normalmente, trama de cota de malha.

A proteção individual era composta por armaduras de couro, reforçadas ou
não com elementos de ferro, como pequenas placas ou anéis, tendo um
custo que permitia a aquisição por mais guerreiros. As mais
caras eram as cotas de malha, compostas de anéis de ferro entrelaçados,
cuja densidade oferecia proteção contra projéteis como flechas e dardos
ou ao menos reduzia os danos causados por estes. O fabrico de uma cota
de malha exigia alta especialização e capacidade, de maneira que seu
acesso era restrito. No decorrer da Era Viking, o conhecimento se
disseminou, e a ascensão de chefes guerreiros e reis promoveu um maior
acesso a elas, em especial na Normandia, um ducado do Reino de França
comandado por um chefe guerreiro viking a partir de 911.

Com a transformação dos exércitos vikings e com a ascensão desses chefes
guerreiros e reis, a sua mobilidade foi ressaltada, lançando
ataques por todo o litoral e redes fluviais da Europa no século \versal{IX} e
parte do \versal{X}. Nesses ataques, compostos tanto por expedições de pilhagem
quanto por invasões, o uso do equipamento por parte dos guerreiros
variava de acordo com a natureza da ação a ser executada.

Em caso de pilhagens, saques e razias, o armamento leve e ofensivo era
privilegiado, como machados, lanças e arcos e flechas. Escudos eram a
proteção individual escolhida, já que armaduras pesadas como as cotas de
malha reduziam a mobilidade e a velocidade de marcha dos grupos, quando em
terra. Em invasões, os vikings lançavam mão de todo o arsenal
disponível, já que a conquista e o estabelecimento de um domínio eram os
objetivos.

A despeito dessa mobilidade, os vikings não constituíam forças de
cavalaria, devido à escassez de cavalos na Escandinávia e a pouca
capacidade de suprir a forragem necessária a grandes forças de
cavaleiros, como os francos podiam fazer. Isso não quer dizer que eles
não soubessem utilizar os cavalos, como demonstrado em diversas ações na
Inglaterra e na França, mas o modo viking de fazer a guerra terrestre
era centrado na infantaria, seja como força de choque ou de inquietação
e desgaste do inimigo mediante o uso de projéteis.

A liberdade de ação concedida pela superior técnica naval, constituída
de conhecimentos de navegação mais do que por tecnologia, concedeu
aos vikings uma capacidade de deslocamento não vista na Europa até
tempos modernos. Valendo-se dos navios, havia projeção do poder viking
em diversas áreas.

Entretanto, o combate naval não diferia muito do combate terrestre e
buscava-se em muito reproduzi-lo, com a abordagem de embarcações e luta
entre as tripulações, com amarração de um navio ao outro. Havia pouca
margem de manobra, de modo que quando um barco quebrava a formação, fazia-o
normalmente para flanqueio da frota inimiga, para explorar uma brecha na
linha inimiga ou ainda para um ataque direto ao navio que seria a nau
capitânia da força inimiga.

E tanto em terra como no mar, predominava uma técnica de combate de
infantaria, típica do modo de guerra germânico e aprimorada pelos vikings:
a parede de escudos (\emph{skjaldborg}). A parede variava em tamanho de
acordo com o efetivo usado e com a intenção do comandante.
Basicamente, os guerreiros ombreavam uma linha juntos, sobrepondo
escudos de maneira a oferecer uma proteção que permitia avanço ou
defesa. A profundidade da parede variava de uma linha singular a várias,
com uso de lanceiros, tal como nas falanges gregas. A variação se dava
quando do ataque, normalmente um escalão de ataque seguia por trás da
parede de escudos, flanqueando ou apoiando a quebra da parede de escudos
quando entrava em choque com o inimigo.

Normalmente a formação de uma parede de escudos, tanto em ataque como em
defesa, era acompanhada de uma forte dose de ferocidade, com cantos e
gritos de guerra entoados, de maneira a fortalecer o moral dos
guerreiros e sustentar o duro e violento combate que se seguiria.

A outra formação tática era em uma disposição das forças em cunha
(triangular), com as linhas reforçadas em profundidade e protegidas por
forças nos flancos. Embora a lenda diga que a técnica foi ensinada pelo
deus Odin, é bem mais possível que ela tenha se desenvolvido com base
nos contatos e combates entre os germânicos continentais e os romanos,
que utilizavam uma formação do tipo, intitulada \emph{porcinum
capet}, que tinha finalidade ofensiva, buscando abrir uma brecha na
linha inimiga pelo choque.

Quanto ao reino da estratégia, os vikings se utilizavam de diversas
técnicas que não somente o combate frontal e brutal, como o estereótipo
medieval. Além da busca pela utilização do recurso de atacar com efeito
de surpresa, para provocar paralisia e destruição do dispositivo
inimigo, seja por combate ou fuga, é possível encontrar registros de
técnicas de desinformação, pela disseminação de boatos, marchas falsas,
uso de recursos como pedidos para enterrar chefes em cidades sitiadas,
entre outros.

Embora com uma reputação de ferozes combatentes, é importante perceber
que os vikings não eram provocadores de batalha. Suas forças tinham
qualidade, por treinamento, técnica e tecnologia, mas podiam ser
derrotadas, como o foram, por anglo-saxões e francos, para citar alguns.
Ainda havia sempre o recurso de buscar eliminar a liderança em batalha
ou por meio de ardis, como foi a tentativa viking de eliminar o rei
anglo-saxão Alfredo, o Grande, ao atacar seu palácio diretamente.

Um exército viking em geral fazia o possível para evitar batalha. Não se
trata de covardia, mas de estratégia e judiciosa aplicação de seu poder
combativo por parte de suas lideranças, uma vez que onde eles combatiam
rara era a chance de obter novos guerreiros para completar os vazios
abertos pelas baixas em batalha. Portanto, a escolha de lutar era
condicionada a uma grande certeza de vitória.

Isso foi reforçado pela utilização de fortificações por francos e
anglo-saxões para se defenderem das forças vikings, estabelecendo
sistemas defensivos que acabaram por inspirar as próprias técnicas
vikings de fortificação, como se pode observar nas linhas defensivas de
Trelleborg, na Dinamarca, criadas para deter uma invasão franca, que têm
aspectos semelhantes as dos \emph{burghs} anglo-saxões, estabelecidos
para conter os vikings.

\SIG{Sandro Teixeira Moita}

Ver também Armamento; Espada; Guerra e simbolismo; História da guerra.

\begin{itemize}
\item \versal{BRINK}, Stefan; \versal{PRICE}, Neil (eds.). \emph{The Viking World}. Abingdon:
Routledge, 2008.

\item \versal{HARRISON}, Mark; \versal{EMBLETON}, Gerry. \emph{Viking Hersir 793-1066 \versal{AD}.}
London: Osprey Publishing, 1993.

\item \versal{HEATH}, Ian; \versal{MCBRIDE}, Angus. \emph{The Vikings}. London: Osprey
Publishing, 1985.

\item \versal{HOLMAN}, Katherine. \emph{The \versal{A} to \versal{Z} of the Vikings}. Plymouth: The
Scarecrow Press, 2009.

\item \versal{SAWYER}, Peter (ed.). \emph{The Oxford Illustrated History of the
Vikings}. Oxford: Oxford University Press, 1997.

\item \versal{WINROTH}, Anders. The Age of the Vikings. Princeton: Princeton University
Press, 2014.
\end{itemize}
\section{\versal{GUERREIRAS NÓRDICAS}}

Existiram mulheres guerreiras na Era Viking? Com a popularidade da série
\emph{Vikings}, isso tornou-se uma ideia comum, visto que o tema aparece em
muitas sagas lendárias e o tema vem sendo investigado em indícios juntos
a corpos femininos de tumbas nórdicas.

O cronista Saxo Grammaticus enumerou várias personagens viris e bélicas
em sua obra, mas foi influenciado pelo referencial classicista e não
pela sociedade medieval. Tradicionalmente, o mundo nórdico foi concebido
pelos acadêmicos como sendo dominado totalmente por valores
masculinistas, mas recentemente essa visão tem mudado. Diversas
pesquisas vêm apontando a importância da mulher na sociedade nórdica,
não sendo apenas mães, concubinas ou escravas, mas ocupando importantes
papéis de atividades e ocupações muito além da esfera doméstica. Muitas
mulheres também foram artesãs, poetisas, curandeiras e sacerdotisas.
Algumas acompanharam expedições e jornadas colonizadoras para outros
países.

As fontes literárias nórdicas mencionam atividades de mulheres
guerreiras, mas o \emph{corpus} não é homogêneo e nem coerente. Nenhuma
das chamadas \emph{Íslendingasögur} (sagas dos islandeses ou de família)
menciona tal comportamento, ao contrário das \emph{fornaldarsögur}
(sagas lendárias), que são abundantes em descrições de mulheres armadas
e participando de lutas. Mas este último tipo de narrativa foi muito
influenciado pelo imaginário medieval e pelos antigos mitos, o que torna
difícil separar os elementos sócio-históricos do fantástico neste
subgrupo das sagas islandesas. Alguns pesquisadores acreditam que
mulheres podem ter participado de ações violentas, mas não era uma norma
na sociedade nórdica, sendo mais uma ação periférica ou mesmo marginal.
Algumas fontes não escandinavas mencionam atividades de nórdicas em
ataques ao mundo europeu, como a crônica irlandesa \emph{Cogadh Gáedhel
re Gallaibh} (século~\versal{XII}), que relata uma jovem ruiva (Inghen Ruaidh)
liderando um grupo de guerreiros nórdicos em ataques a Munster no século~\versal{X}.

Segundo o arqueólogo Leszek Gardela, a questão da autenticidade histórica
de mulheres guerreiras nórdicas é problemática, pois não existem
evidências objetivas nesse sentido. Ele realizou uma síntese sobre as
investigações arqueológicas envolvendo sepulturas femininas com
armamento até o presente momento. Em primeiro lugar, quase nunca foram
encontradas sepulturas femininas simples (com somente um corpo) com
espadas -- o principal símbolo marcial na Era Viking. Em algumas
sepulturas foram encontrados vestígios de machado, em outras de lança. É
possível que estes objetos não tenham sido utilizados em sentido prático
e marcial -- machados foram símbolos do culto a Thor e as lanças, a Odin
(ou associadas com bastões mágicos de ritos femininos). Não podemos
esquecer que sepulturas são demarcadoras de crenças em vida após a morte
-- a conexão entre objetos e religiosidade é muito grande. O tipo mais
comum de objetos belicosos encontrados nas sepulturas são facas e
punhais dos mais variados tipos e tamanhos -- algo totalmente condizente
com a maioria das sagas islandesas (especialmente as sagas de família).
Foram utilizadas tanto para uso doméstico e no cotidiano das fazendas,
quanto para matar pessoas (mesmo por mulheres, seja por vingança ou
proteção). Também foram encontrados vestígios de projéteis de arquearia
-- condizente com caça e defesa feminina das comunidades, não
necessariamente uso em guerra.

E Gardela também alerta para a problemática da interpretação dos
vestígios arqueológicos no contexto funerário -- não sendo ``espelhos''
da vida cotidiana (refletem muito mais as estruturas sociais, políticas e
religiosas da comunidade que enterrou o defunto do que a vida cotidiana
do defunto em si). Assim, ele afirma que os vestígios arqueológicos
ainda não podem auxiliar a definitivamente conceder um quadro afirmativo
para a questão de a mulher ter participado ativamente de batalhas e
expedições predatórias. Ele mesmo afirma que a questão pode ter um
contexto mais definido quando forem realizados exames osteológicos em
corpos nórdicos encontrados mais recentemente -- como nas mulheres
sármatas (pesquisa efetuada pela arqueóloga Davis-Kimball na região da
Ásia Central), onde foi comprovado, em exames dos ossos, que elas
estiveram diretamente em batalhas (verificando-se indícios como cortes, fraturas e sinais de
contusão causados por armamentos bélicos do período). Mas até o presente
momento, a arqueologia não comprova osteologicamente a questão para a
área escandinava, que ainda está em aberto.

Em recente publicação, uma equipe de pesquisadores liderados pela
arqueóloga Charlotte Hedenstierna-Jonson da Universidade de Uppsala
(Suécia), analisou as ossadas de um sepultamento de Birka (\versal{BJ} 581),
concluindo definitivamente que se trata das ossadas de uma mulher,
enterrada com diversos armamentos (espada, machado, lança, flechas,
faca, escudos e dois cavalos). Junto ao seu tórax, foi inserido um
tabuleiro com peças de jogos, que segundo a arqueóloga seria uma alusão
ao fato de que ela definiria as táticas e estratégias, ou seja, seria
uma líder no comando militar. A mulher teria 30 anos de idade e seria
muito alta, com cerca de 1,70~m de altura. O estudo fornece uma
nova compreensão da sociedade nórdica, suas construções sociais e normas
vigentes durante a Era Viking. Os resultados apontam uma sepultura de
alto \emph{status}, sendo a mulher enterrada possivelmente uma guerreira
feminina viking de alta posição (apesar de não possuir vestígios de
marcas de batalha em seus ossos), sugerindo que as mulheres pudessem ser
membros de alta posição na esfera dominada pelos homens.

\SIG{Johnni Langer}

Ver também Guerra e técnicas de combate; Lagertha; Mulheres; Sociedade.

\begin{itemize}
\item \versal{ANDERSEN}, Lise Præstgaard. On Valkyries, Shield-maidens and Other Armed
Women -- in Old Norse Sources and Saxo Grammaticus. In: \versal{SIMEK}, Rudolf \&
\versal{HEIZMANN}, Wilhelm (eds.). \emph{Mythological Women}: Studies in Memory
of Lotte Motz (1922-1997). Wien: Fassbaender, 2002, pp. 291-318.

\item \versal{GARDELA}, Leszek. Amazons of the viking world. \emph{Medieval Warfare} 7,
2017, pp. 08-15.

\item \versal{GARDELA}, Leszek. Mujeres poderosas en la Era Vikinga. \emph{Arqueología
e Historia}, vol. 13, 2017.

\item \versal{GARDELA}, Leszek. Warrior-women in Viking Age Scandinavia? A preliminary
archaeological study. \emph{Analecta Archaeologica Ressoviensia}, vol. 8,
2013, pp. 273-339.

\item \versal{HEDENSTIERNA-JONSON}, Charlotte et al. A female Viking warrior confirmed
by genomics. \emph{American Journal of Physical Anthropology}, 2017, pp.
1-8.

\item \versal{PRICE}, Neil \emph{et al}. Male-biased operational sex ratios and the
Viking phenomenon: an evolutionary anthropological perspective on Late
Iron Age Scandinavian raiding. \emph{Evolution \& Human Behavior}, vol. 38,
n. 3, 2017, pp. 315-324.

\item \versal{PRICE}, Neil. Women with weapons: in search of the female viking.
\emph{Jornadas de Arqueología y Cultura Vikinga}, Universidad de
Alicante, 2015.
\end{itemize}
\section{\versal{GUTA SAGA}}

A \emph{História dos gotlandeses} (\emph{Guta saga}) consiste em uma
narração, escrita em gotlandês antigo, da história da ilha báltica de
Gotlândia, no oeste da Suécia. Essa narração, que contém vários
elementos lendários, foi conservada em 8 folhas do manuscrito \emph{B64
Holmiensis} da Biblioteca Real Sueca (\emph{Kungliga Biblioteket}) e o
seu texto aparece imediatamente depois das \emph{Leis dos gotlandeses}
(\emph{Gutalag}), também recolhidas nesse mesmo manuscrito pertencente ao
século \versal{XIV}, apesar de a data original da composição da \emph{Guta saga}
parecer remontar-se ao anterior século \versal{XIII}. Não obstante, cabe
assinalar que o título desse texto não aparece neste manuscrito
conservado, pois foi dado por Carl Säve em 1852. Igualmente, a
tradicional divisão desta história em 4 capítulos usada pelos editores
modernos tampouco corresponde ao manuscrito original, ainda que resulte
útil para sua estruturação em uma série de sequências narrativas.

O texto da \emph{Guta saga} foi composto por cerca de 1.800 palavras e
relata, a partir de fontes orais (para os elementos lendários) e escritas
(para os elementos históricos e jurídicos), a história de Gotlândia, desde
o momento em que foi colonizada pela primeira vez por um tal de Þieluar, até
certos acontecimentos praticamente contemporâneos ao seu anônimo autor
ou compilador, possivelmente algum clérigo gotlandês. No texto também
somos informados de como a superpopulação da ilha forçou a migração de
uma parte de seus habitantes e como estes chegaram a estabelecer-se no
Império Bizantino. Em sequência, são descritos alguns antigos rituais
pagãos na ilha, sua cristianização e os laços estabelecidos com a
monarquia sueca. As últimas duas seções do texto narram com certo
detalhe as obrigações recíprocas entre os gotlandeses, o rei da Suécia e
o bispo de Linköping.

Além de ser um dos poucos textos em prosa de conteúdo não jurídico
procedente do antigo leste escandinavo, consideramos que a
particularidade da \emph{Guta saga} reside também em outros dos traços
principais de claro interesse para o germanista. Por um lado, de um
ponto de vista histórico e à margem dos numerosos dados referentes a
história medieval da própria Gotlândia, aparecem no capítulo~\versal{I} não
somente os interessantes dados sobre as migrações realizadas da
Gotlândia, mas também a sugestiva (e amplamente discutida) expressão
\emph{ok enn hafa þair sumt af varu mali}, que significa ``e todavia
possuem [os emigrantes de Gotlândia estabelecidos no Império
Bizantino] algo de nossa língua''. A partir de dados arqueológicos e
históricos, vários investigadores têm sustentado a ideia de que dita
expressão se refere à população de fala goda estabelecida próxima ao
Império Bizantino e do qual se situariam alguns núcleos na península da
Criméia, lugar de onde se manteve uma versão do gótico até o século
\versal{XVIII}. Assim mesmo se manteve a ideia, nem sempre
aceita, da origem escandinava dos godos, que já foi mantida no século~\versal{VI}
por Jordanes no capítulo \versal{IV} de seu \emph{De origine actibusque Getarum}
(\emph{Origem e gestas dos godos}). Por outro lado, de um ponto de vista
filológico alguns investigadores não têm duvidado assinalar o parentesco
existente entre o gotlandês antigo e o gótico. Não é fortuito que, de
todas as línguas escandinavas, o gotlandês é a que mais se assemelha ao
gótico. Existem, de fato, uma série de argumentos
linguísticos para contemplar o gotlandês como uma língua um tanto
diferenciada do sueco, dinamarquês, norueguês e islandês antigos. Entre
tais argumentos destacariam: (1) a retenção de alguns ditongos ali de
onde o sueco antigo já os havia perdido (p.ex., \emph{hoyra}); (2) a
palatalização e labialização de certas vogais em posições particulares
(\emph{dyma}. cfr. o sueco \emph{döma}); (3) a retenção do
{[}\emph{u}{]} breve do nórdico antigo comum alí onde o sueco possui uma
{[}\emph{o}{]} breve (\emph{gutar}, \emph{fulk}) e (4) a existência de
mais metafonias por {[}\emph{i}{]} que em outras línguas escandinavas
(\emph{segþi}. cfr. o islandês antigo \emph{sagði}). Isso faz do
gotlandês, apesar da relativamente pouca atenção de que tem gozado,
mais um elemento interessante para se ter em conta para a filologia
comparada das línguas germânicas, tanto antigas como modernas, pois uma
versão mais moderna do gotlandês ainda se segue falando na atualidade e
tem começado um processo de recuperação.

A \emph{Guta saga} possui uma
evidente carga ideológica que promove claramente a ideia de certa
independência e soberania gotlandesa. Parece ter sido o desfecho das já
mencionadas \emph{Gutalag} e composta para ajudar aos gotlandeses a estabelecer suas
relações com os vários poderes que dominaram o Báltico ao longo da Idade
Média, especialmente, Suécia, Dinamarca e a Liga Hanseática.

\SIG{Mariano González Campo}

Ver também Gotland (Gotlândia); Literatura; Norreno; Sagas islandesas;
Suécia da Era Viking.

\begin{itemize}
\item \versal{GUSTAVSON}, Herbert. \emph{Inledning till gutamålets studium}.
Visby, 1974, (Gotlandica, vol. 5).

\item \versal{MAILLEFER}, Jean Marie. Guta Saga: Histoire des Gotlandais: Introduction,
traduction, commentaires. \emph{Études germaniques}, vol. 40, 1985, pp.
131-140.

\item \versal{MITCHELL}, Stephen A. On the Composition and Function of \emph{Guta
Saga}. \emph{Arkiv för nordisk filologi}, vol. 99, 1984, pp. 151-174.

\item \versal{PEEL}, Christine (ed.). \emph{Guta Saga. The History of the Gotlanders}.
London: Viking Society for Northern Research-University College London,
1999.

\item \versal{STEARNS}, MacDonald. \emph{Crimean Gothic. Analysis and Etymology of the
Corpus}. Saratoga: Anma Libri, 1978.
\end{itemize}

\chapter{H \textarn{h}}

\section{\versal{HABITAÇÃO}}

Toda arquitetura, desde a monumental até a habitação mais prosaica ou
vernacular, é um produto cultural e, ao longo da história, sempre
despertou paixão, fascínio e admiração, enquanto tentava materializar as
necessidades, os desejos e as ambições humanas. Suas formas expressam,
às vezes dramaticamente, através das montoeiras de pedras, madeiras e
tijolos argamassados, as estruturas sociais, econômicas, ambientais e
políticas que as originaram. As habitações da Era Viking não fogem a
esses princípios fundamentais, e suas diferenças tipológicas provêm
diretamente da riquíssima diversidade encontrada no vasto território da
Escandinávia.

Muitas dessas habitações, assim como as características dos próprios
aldeamentos da Era Viking, remetem a povoações rurais produzidas
anteriormente, durante a chamada Idade do Ferro Primitiva. Nessa
época a agricultura e a criação de gado deviam ser os principais meios
de subsistência e as aldeias eram compostas por uma casa longa, com
moradia e estábulo sob o mesmo teto e outros edifícios menores,
provavelmente celeiros e oficinas. Essas casas longas de formato
retangular dispunham internamente de um corredor central com largura
variável, resguardado por telhados esconsos de duas águas cobertos por
um grosso amarrado de palha. Apesar da impossibilidade de se esboçar um
modelo preciso para essas habitações, os arqueólogos concordam que as
casas construídas ao longo dos períodos vindouros não podem ser
compreendidas sem uma prévia análise dessas primitivas edificações
produzidas ainda Antes da Era Comum.

Nada restou das habitações da Idade de Ferro Primitiva e sequer da
Era Viking, exceto alguns rastros deixados no solo, umas poucas
soleiras de pedras e buracos onde outrora existiam prováveis colunas de
madeira. Portanto, as análises e interpretações das provas
arqueológicas, recuperadas por intermédio das escavações, são alguns dos
métodos possíveis para tentar reconstruir uma pequena parcela desse
passado quase perdido. Impreterível considerar também, como ajuda
singular para (re)montar esse quebra-cabeça histórico, as
contribuições literárias advindas das famosas sagas, ou até mesmo dos
trechos da \emph{Edda}. Teva Vidal defende a tese que essas fontes literárias
às vezes apresentam um mundo mitológico, ou a-histórico, ou
pseudo-histórico, repleto de protagonistas fantásticos, mas que
contraditoriamente acontecem em um pano de fundo muito realista, com
locações e paisagens, em muitos casos, ainda hoje reconhecíveis. Segundo
ele, sítios selecionados de cada saga têm particular importância, pois
reúnem muitos detalhes descritivos das residências agrícolas ao longo de
suas ocorrências no texto.

Ao longo da Era Viking as habitações em sua grande maioria eram
simples, apropriavam-se de materiais obtidos no local e de técnicas
construtivas rudimentares, pois eram provavelmente feitas pelos próprios
moradores. As exceções eram as mansões dos aristocratas e dos
chefes vikings, que possuíam grandes dimensões. Segundo Campbell, alguns
edifícios escavados em Sædding, distrito de Esbjerg, sudoeste da Jutlândia
na Dinamarca, chegavam a alcançar 50~m de comprimento; na
aldeia de Borg, perto de Bøstad, norte da
Noruega, encontrou-se a ruína da maior casa viking conhecida atualmente, medindo
86~m de comprimento por 9~m de largura. Essas proporções
e as dimensões totais das habitações variavam muito, assim como as
tipologias arquitetônicas, principalmente em razão das características
regionais, posição social e dos recursos disponíveis no local -- algumas
transformações formais também acompanharam as diversas mudanças no
contexto cultural. Contudo, a forma básica mais comum dos edifícios
continuava sendo retangular alongada, por vezes com paredes curvas --
nestes casos a planta arquitetônica parecia fazer referência à curvatura
dos cascos dos famosos barcos vikings.

Símbolo austero da arte naval, a engenhosidade, a funcionalidade e a
extrema qualidade dos diferentes tipos de embarcações produzidas pelos
povos da Escandinávia comprovam o seu pleno domínio das técnicas de
carpintaria e marcenaria. O conhecimento e a destreza desses ofícios
tradicionais também foram apropriados como técnicas de edificação.
Grande parte das suas construções possuíam paredes de madeira,
ora cortadas em tábuas, ora em troncos inteiros empilhados
horizontalmente um sobre o outro com esquinas empalmadas. Não obstante,
a madeira não era a única matéria-prima utilizada nas construções, as
paredes também podiam ser construídas com um entrelaçado de varas, ripas
ou cipós recobertos por uma mistura de barro, água, esterco e fibras
vegetais. Além disso, podia-se utilizar também estereotomia, ou seja, alvenarias de
pedras cortadas, ou ainda uma combinação de camadas sobrepostas de
pedras assentadas sobre a turfa, ou apenas uma espessa camada de turfa.

Os telhados eram predominantemente de duas águas e podiam ser cobertos
por palha, turfa ou telhas de madeira -- em raros casos, dois desses
elementos eram combinados. A inclinação variava de acordo com a
tipologia construtiva e do material utilizado para cobertura. Algumas
vezes a peça da cumeeira podia ser arqueada, novamente fazendo
referência à quilha de uma embarcação. Nos casebres menores, o beiral do
telhado estendia-se até o chão, praticamente extinguindo as paredes
laterais.

O interior da maioria das casas possuía o chão de terra batida coberto
por juncos ou algum outro tipo de folhagem. Como possuíam poucas
aberturas -- algumas contavam apenas com uma única porta de acesso --, na
penumbra nebulosa do seu interior destacava-se o local destinado ao
fogo, o qual ocupava grande parte da área central da casa. Este era um
elemento estruturador não apenas do simplório esquema planimétrico da
edificação, mas da própria vida doméstica. Essa fogueira ou lareira
escandinava resumia-se a um simples recorte geométrico no solo, cingido
de pedras para isolar a chama no interior. O fogo mantido aceso durante
o dia e zelosamente alimentado para perdurar nas noites mais frias não
servia apenas para aquecer o interior, mas nele também eram preparados
os alimentos, pois era pouco usual que houvesse uma cozinha independente. Algumas
habitações tinham aberturas -- claraboias -- sobre a fogueira,
permitindo que parte da fumaça se dissipasse e gerando uma iluminação
zenital que facilitava a prática culinária e clareava esse salão
principal. A dimensão desse espaço era proporcional à dimensão da
edificação, pois esses salões podiam possuir uma ``nave'' ou ``secção''
única, ou nas casas maiores três ``naves'' ou ``secções'', sendo uma
principal e duas laterais mais estreitas. O que as definiam eram as
colunatas de tronco de árvores que percorriam todo o comprimento do
edifício e sustentavam o madeiramento do telhado. Nos salões mais
largos, por questões estruturais, as colunas de madeira emergiam do
centro dos salões formando essas divisões. Nos salões mais estreitos, as
estruturas de madeira ficavam embutidas nas paredes externas. Esta
última disposição proporcionava um espaço interno ininterrupto.

Compunha esse recinto ao redor da fogueira, encostadas nas paredes
laterais, volumes de madeira ou de pedra, ou ainda de terra aplanada --
reforçada internamente com vime -- cobertos com peles de animais que
serviam concomitantemente como bancos e camas. Neles os habitantes da
casa realizavam seus afazeres e relaxavam com jogos, músicas ou ouvindo
histórias. Enfim, um importante espaço de vivência, onde nas casas da
realeza e da alta aristocracia também aconteciam reuniões e encontros
políticos.

Poucos móveis integravam o interior da habitação. Tinha-se normalmente o tear, um
baú guarda-roupa e, nas casas maiores, uma mesa. A estrutura do
madeiramento do telhado também servia para pendurar objetos pessoais de
uso cotidiano e alimentos. Um mobiliário específico estava quase sempre
presente: a cadeira alta, um assento especial e normalmente entalhado,
destinado exclusivamente ao chefe da casa, amiúde posicionada na
cabeceira da habitação, mas movida de um local para o outro de acordo
com a ocasião. Cabe ressaltar os objetos trazidos de outras regiões,
adquiridos por meio comercial ou despojos das invasões, e que ajudavam a
compor a decoração.

As habitações podiam ter divisões internas, sendo a vivenda central com
a fogueira o maior compartimento, e os menores, um em cada extremidade,
proporcionavam espaço para armazenamento e áreas de trabalho para os
comerciantes e artesãos, como comprovam as escavações da antiga cidade
de Hedeby, localizada no norte da Alemanha, junto à atual fronteira com
a Dinamarca. Edificações maiores podiam ter ainda um segundo pavimento
ou uma espécie de mezanino, acessado por escadas de madeiras, conforme
citado em uma pequena passagem da saga \emph{Brennu-Njáls}, que
descreve que Gunnar dormia no cômodo acima do salão principal, junto com
a sua esposa e sua mãe. A literatura das sagas também menciona que as
mulheres se reuniam em uma parte específica da habitação, onde
realizavam suas tarefas e contavam histórias, porém não é possível
determinar com precisão a localização espacial desse cômodo. Os
lavatórios e banheiros eram normalmente estruturas
independentes, separadas a curta distância da moradia; contudo,
escavações arqueológicas do sítio Eiríksstaðir, no vale de Haukadalur, na
Islândia, indicam a existência de um lavatório interno. Em algumas casas
das aldeias agrícolas se preservava o hábito de abrigar os animais em
áreas determinadas do interior da edificação, cujo calor corporal
ajudava, inclusive, a aquecer o espaço;
outras, no entanto, podiam ter edifícios separados destinados aos
estábulos e para atividades específicas, como forjas e olarias.

A despeito de toda subjetividade interpretativa, é fato que tanto a
arqueologia quanto a literatura concorrem para fornecer um pálido
vislumbre da Era Viking, não obstante, suficientemente expressivo
para clarear e evidenciar a rica diversidade das habitações produzidas
ao longo deste período. Conforme descrito anteriormente, apesar de
possuírem algumas características ubíquas, dificilmente se consegue
formular um padrão ou modelo representativo dessas construções, a julgar
pela flexibilidade dos espaços, dos programas, das funções, para além da
necessidade de adaptação ao meio e as constantes mudanças nos costumes
habitacionais. Tudo isso contribui para ampliar a pluralidade tipológica
dessa arquitetura retentora de qualidades culturais específicas.

\SIG{João Batista da Silva Porto Junior}

Ver também Bóndi; Cotidiano; Cultura material; Patrimônio; Sociedade.

\begin{itemize}
\item \versal{ANDRÉN}, Anders. Places, Monuments, and Objects: The Past in Ancient
Scandinavia. \emph{Scandinavian Studies}, vol. 85, n. 3, 2013, pp.
267-281.

\item \versal{BERSON}, Bruno. A Contribution to the Study of the Medieval Icelandic
Farm: The Byres. \emph{Archaeologia Islandica} 2, 2002, pp. 37-64.

\item \versal{CAMPBELL}, James Graham. \emph{Grandes Civilizações do Passado:} \emph{Os
Vikings}. São Paulo: Editora Folio, 2006.

\item \versal{FAZIO}, Michael \emph{et al}. \emph{A História da Arquitetura Mundial}.
Porto Alegre: \versal{AMGH} Editora, 2011.

\item \versal{VIDAL}, Teva. \emph{Houses and Domestic Life in the Viking Age and
Medieval Period: Material Perspectives from Sagas and Archaeology}. PhD
thesis, University of Nottingham, 2013.
\end{itemize}
\section{\versal{HAROLDO DENTE AZUL (HARALDR GORMSSON)}}

Haroldo era filho do rei Gorm, o Velho e de Thyra Dannebod, tendo
nascido por volta do ano 935, em algum lugar da região da Jutlândia, na
Dinamarca. Nesse tempo, o país já estava praticamente unificado sob o
controle de um único soberano, algo possivelmente atestado desde o
século~{\versal{VIII} ou \versal{IX}}. Assim, no século~\versal{X}, a Dinamarca já dispunha de uma
hegemonia real, sendo que na época Gorm, o Velho, era o então monarca,
reinando a partir de Jelling, importante centro comercial na Jutlândia.
Sobre sua vida e feitos praticamente nada se sabe. Gorm faleceu por
volta do ano 958 ou 959, sendo sucedido por seu filho Haroldo. Em
homenagem aos pais, Haroldo ordenou que fosse erguida em Jelling uma
pedra rúnica.

Haroldo governou por quase trinta anos. Durante
seu longo reinado, Haroldo realizou três grandes feitos pelos quais
normalmente é lembrado: consolidou seu domínio na Dinamarca, impôs sua
autoridade à Noruega e ordenou que seu reino fosse cristianizado. Tais
façanhas constam na pedra rúnica de Jelling.

Para assegurar seu direito ao trono, Haroldo investiu massivamente na
construção de fortalezas pelo país. Quatro dessas fortificações foram
encontradas: Aggersborg e Kyrkat na Jutlândia, Trelleborg na Zelândia, e
Nonnebakken na Fiônia. No entanto, a existência de tais fortificações é
considerada por alguns historiadores hoje como um atestado que a época
fosse um período de recorrentes batalhas. James Graham-Campbell comenta
que, no século~\versal{X}, notou-se uma corrida na construção de fortificações nos
reinos escandinavos, reflexo de uma política exterior bastante delicada
e conturbada -- a invasão de reinos nessa época não era incomum.

Em 974, o imperador germânico Oto~\versal{II} atacou os domínios noruegueses, apossando-se
da importante cidade comercial de Hedeby e da muralha do Danevirke, territórios que ocupou por quase dez anos, até que, no ano de 983, Haroldo e seu aliado Mistivoi recuperaram os territórios perdidos.
Mistivoi ofereceu a mão de sua filha Thora em casamento ao rei
dinamarquês. Haroldo também ordenou que novos muros e defesas fossem
erguidos em Hedeby e a muralha do Danevirke fosse reforçada. Além da
construção de fortes e muralhas, outra façanha da engenharia de seu
governo foi a construção da grande ponte Ravning Enge.

Além de ter investido em grandes obras, Haroldo planejou um golpe de Estado
para conquistar a Noruega, ou pelo menos impor sua influência
indiretamente. Em um complô tramado entre Haroldo e o \emph{jarl} Haakon Sigurdsson, o rei Haroldo~\versal{II} da Noruega (961-970) foi destronado. Haakon
governou a Noruega de 970 a 995, atuando como vassalo de Haroldo~\versal{I} da
Dinamarca.

No que se refere a sua terceira grande façanha, a cristianização da
Dinamarca, ocorriam já no reinado de seu pai expedições missionárias ao país. Gorm, o Velho, porém, relutou em adotar a fé cristã, embora
tolerasse sua presença em seu reino. Haroldo teria sido convertido por
volta de 965, de acordo com o relato das \emph{Crônicas Saxônicas}, por
um missionário chamado Poppo. Com isso, o rei investiu na cristianização
de seu reino, promovendo a construção de igrejas.

Por volta de 985 ou 986, Haroldo sofreu um golpe de Estado, seu filho
Sueno Barba-bifurcada (c. 965-1014) destronou o pai e o enviou para o
exílio. Os motivos da traição não são claros, mas entre as hipóteses
consta que determinados grupos da nobreza e aristocracia estariam
descontentes com os altos gastos públicos e o aumento dos impostos
decretados por Haroldo para realização de suas obras. Além disso,
somar-se-ia às hipóteses sua imposição do cristianismo como a religião
oficial do Estado. Haroldo exilou-se em Jumme (atualmente Wolin, na
Polônia), falecendo por volta de 987.

Haroldo foi casado duas vezes: sua primeira esposa foi Gyrid
Olafsdottir, com quem teve dois filhos e duas filhas, sendo Sueno o
filho mais conhecido. Sua segunda esposa foi Thora Mistivisdattir, com a
qual não teve filhos. Quanto ao cognome ``dente azul''
(\emph{blátǫnn} em nórdico antigo), não se sabe ao certo de onde surgiu. O documento mais antigo a registrá-lo é a \emph{Crônica
Roskildense}, datada de 1140. Embora seja desconhecida a origem desse
curioso cognome, ele acabou sendo usado para nomear a famosa empresa de
rede sem fio Bluetooth, fundada em 1998.

\SIG{Leandro Vilar Oliveira}

Ver também: Dinamarca da Era Viking; Era Viking; Fortificações; Viking.

\begin{itemize}
\item \versal{GRAHAM-CAMPBELL}, James (org.). \emph{Os vikings}. Barcelona: Folio \versal{S.A.},
2006.

\item \versal{HOLMAN}, Katherine. \emph{Historical dictionary of the vikings}. Lanham:
Scarecrow Press Inc, 2003.

\item \versal{LOGAN}, F. Donald. \emph{The Vikings in History}. London/New York:
Routledge, 1991.

\item \versal{ROESDAHL}, Else. The emergence of Denmark and the reign of Harald
Bluetooth. In: \versal{BRINK}, Stefan; \versal{PRICE}, Neil (eds.). \emph{The Viking
World}. London/New York: Routledge, 2008, pp. 652-664.
\end{itemize}
\section{\versal{HAROLDO CABELOS BELOS (HARALDR HÁRFAGRI)}}

Haraldr Hárfagri, também conhecido como Harald Harfager, Haroldo Cabelos
Finos, ou Haroldo Cabelos Belos, era filho de Halfdan, o Negro, do qual herdou a associação na dinastia dos Ynglings e o domínio sobre a próspera
região de Vestfold, na Noruega. Com a reconquista de territórios tomados
por inimigos na região após a morte de Halfdan, somando-se à conquista sucessiva
de outros territórios, Harfager dominou toda a Noruega sob seu reinado,
tornando-se o primeiro rei do país unificado.

Devido à morte acidental do rei Halfdan por afogamento, Haroldo sucedeu seu pai quando tinha apenas dez anos, como relata o 
\emph{Heimskringla}, ou o conjunto de sagas dos reis da Noruega. Por
causa de sua pouca idade, ficou sob a tutela de Guthorm, seu tio do lado
materno, que cuidou também dos negócios reais, incluindo as
forças militares. Guthorm e Haroldo retomaram territórios tomados pelos
inimigos de Halfdan. Haroldo continuou, durante sua maturidade, sobre os outros territórios que
compunham a Noruega, tais como Orkadal, Gaulardal, Trondheim, Naumudal,
e Moer, vencendo em combate e subjugando outros reis e \emph{earls}, angariando
apoio militar e de recursos oriundos de impostos e favorecimentos
territoriais às aristocracias locais. Finalmente, por volta do ano 872
da era cristã, Haroldo enfrentou, na batalha de Haffrsjord, os últimos inimigos que ofereciam
resistência ao seu domínio. A partir desse momento, teve toda a Noruega sob seu reinado.

Como consequência da unificação da Noruega por Haroldo Cabelos Belos, houve
uma grande fuga em massa de inimigos e descontentes com seu domínio, culminando em uma colonização mais efetiva da Islândia e no domínio das
ilhas Hébridas e Órcadas na costa escocesa. No início de seu reinado
unificado, houve práticas de pilhagens da parte desses exilados, que atacavam o território norueguês a partir das regiões citadas. Para enfrentar o problema, Haroldo organizou grandes expedições de combate a essas ações de pirataria, principalmente as vindas das regiões das Hébridas e
Órcadas. Com o êxito das expedições, pôde estabalecer \emph{earls} nestas ilhas.
Como forma de manutenção de seu domínio sobre o território unificado da
Noruega, Haroldo Cabelos Belos fez uso de um sistema de coleta de impostos e
taxas sobre as terras, concedendo aos \emph{earls} e chefes locais de sua
confiança uma parte dos itens arrecadados, o que proporcionava a estes,
segundo as fontes escritas, uma grande riqueza. Em troca, essa aristrocracia deveria dar suporte militar à coroa.

Haroldo Cabelos Belos teve vida e reinado longos. Também muitas esposas, concubinas e filhos.
Sua saga no \emph{Heimskringla} conta que quando seus filhos estavam já adultos começaram
a ocorrer entre eles violentas disputas por territórios. Próximo de sua morte, que ocorreu em 930, Haroldo estabeleceu seu filho Érico Machado Sangrento no comando do reino. Postumamente a Haroldo, contudo,
Érico entrou em uma sangrenta disputa com seus irmãos, impondo um
sanguinário domínio, que terminou quando perdeu o apoio da aristocracia
para seu irmão Hakon, o Bom, deixando o país e tendo seu lugar ocupado
pelo último.

Embora tradicionalmente a historiografia especializada aponte
o nascimento de Haroldo para o ano 850, sua vitória em Haffrsfjord para
872, e sua morte para por volta de 933, não há um consenso sobre essas
datas. Alguns estudiosos sugerem, inclusive, que os eventos possam ter ocorrido
alguns anos à frente, e que a expansão de seu reinado sobre a Noruega só
foi possível graças à conquista do apoio das aristocracias locais de
cada região, as quais, apesar do domínio de Haroldo sobre elas, continuaram
obedecendo também aos seus conselhos de líderes e leis locais.

\SIG{Fábio Baldez Silva}

Ver também Era Viking; Noruega da Era Viking; Viking.

\begin{itemize}
\item \versal{BRØNDSTED}, Johannes. \emph{Os Vikings: história de uma fascinante
civilização}. São Paulo: Hemus, s.d.

\item \versal{GRAHAM-CAMPBELL}, James. \emph{Os Viquingues: Origens da Cultura
Escandinava}. Madrid: Del Prado, 1997, (vol. 1).

\item \versal{JONES}, Gwyn. \emph{A history of the vikings}. Oxford: Oxford University
Press, 1984.

\item \versal{STURLURSON}, Snorri. The Saga of Harald Fairhair. In\emph{: Heimskringla,
History of the kings of Norway.} Trad. Lee M.
Hollander. Austin: University of Texas Press, 2011.
\end{itemize}
\section{\versal{HAROLDO HARDRADA (HARALDR SIGURDSSON)}}

Haraldr Sigurdsson, também chamado de Harald Hardrade ou Haroldo Hardrada,
foi o último rei norueguês viking. Sua morte na batalha de Stamford
Bridge, no ano de 1066, na Inglaterra, é considerada como o marco final da
Era Viking. Haroldo nasceu em 1015 e era meio-irmão, por parte materna, de
Olavo Haraldson, o Santo. Hardrada, após sobreviver à batalha de
Stiklestad, em 1030 -- quando tinha apenas quinze anos de idade --, fugiu da
Noruega e passou a atuar
no exterior como pirata e mercenário. Após juntar riqueza e apoio
militar suficiente, retornou a seu país e conquistou o trono.
Empreendeu diversas campanhas militares com o objetivo de ampliar seus
domínios, mas que acabaram por levá-lo à morte em Stamford Bridge.

Após sobreviver à batalha de Stiklestad, na qual morreu seu irmão (o rei
Olavo Haraldson), Haroldo fugiu para a Suécia, onde reuniu parte do exército
derrotado de Olavo com o objetivo de um dia retornar à Noruega para
conquistar a posição de rei. Após o inverno de 1031, partiu para a
região de Novgorod, na Rússia, onde o rei Jarisleif o recebeu e o
abrigou. Jarisleif também concedeu a ele e seus homens autoridade de defesa
sobre seus domínios para atuarem como mercenários. Hardrada
permaneceu por um tempo considerável à serviço de Jarisleif, quando
conseguiu aprimorar suas habilidades em batalha. Depois, partiu para
Constantinopla, onde passou a empregar seus serviços como mercenário ao
Império Bizantino, tendo como missão combater ameaças como a pirataria
ao império. Tornou-se chefe da Guarda Varangiana do Império Bizantino e,
após conquistar muitas riquezas a serviço de Constantinopla, retornou para 
Novgorod em
1045, casando-se com a filha de Jarisleif.

Depois desses acontecimentos, Haroldo deixou a Rússia e passou a atuar em
expedições vikings contra as terras da Dinamarca. Para evitar mais saques a
seus domínios, Magnus Olafson, rei da Dinamarca e Noruega nesse momento, convidou 
Haroldo Hardrada para compartilhar seu reinado na
Noruega, momento no qual Haroldo passa a ocupar a posição de rei, em conjunto com
Magnus. Pouco tempo depois, Magnus morreu e Haroldo Hardrada passou a
exercer o reinado sobre a Noruega sozinho. Com a ascensão de Svein
Ulfson, na Dinamarca, Haroldo empreendeu campanhas militares contra o
país. Após muitas batalhas, chegou-se a um acordo no qual se decidiu que
Haroldo permaneceria rei da Noruega e Sueno rei da Dinamarca.

Após esses fatos, Haroldo Hardrada voltou suas atenções à Inglaterra,
partindo em uma campanha militar com o objetivo de conquistar o país.
Contou com uma frota de aproximadamente 200 navios de guerra, com
cerca de dez mil guerreiros sob as suas ordens, planejando,
primeiramente, conquistar a parte norte do território inglês, onde a
influência nórdica era maior. Empreendeu grandes saques e impôs
muitas baixas aos seus inimigos. Ao tomar conhecimento do ocorrido, o rei inglês Haroldo
Godwinson juntou suas forças rumo ao norte para confrontar Haroldo
Hardrada. Este, após obter várias vitórias sobre os ingleses, avançou para
Stamford Bridge, onde teve lugar uma violenta batalha entre os nórdicos
e os ingleses que resultou na morte de Hardrada, bem como na derrota de seu
exército. A partir de então, seus filhos Olavo e Magnus passaram a
compartilhar entre si o reino da Noruega.

Assim, a historiografia especializada aponta como o fim do período
viking a morte de Haroldo Hardrada, em 1066, considerando-o como o último
rei viking norueguês. Sua morte em Stamford Bridge marcou o
fim de uma era de expansão e saques dos povos escandinavos em diversas
áreas da Europa Ocidental -- como Rússia, Bizâncio e o Atlântico norte --, 
responsáveis, inclusive, por contatos com o mundo árabe e com uma parte da América do Norte.

\SIG{Fábio Baldez Silva}

Ver também Era Viking; Noruega da Era Viking; Viking.

\begin{itemize}
\item \versal{BRØNDSTED}, Johannes. \emph{Os Vikings: história de uma fascinante
civilização}. São Paulo: Hemus, s.d.

\item \versal{JONES}, Gwyn. \emph{A history of the vikings}. Oxford: Oxford University
Press, 1984.

\item \versal{SPRAGUE}, Martina. \emph{Norse warfare: unconventional battle strategies
of the ancient \emph{Vikings}}. New York: Hippocrene Books, 2007.

\item \versal{STURLURSON}, Snorri. The Saga of Harald Sigurtharson. In\emph{:
Heimskringla, History of the kings of Norway.} Trad.
Lee M. Hollander. Austin: University of Texas Press, 2011.
\end{itemize}
\section{\versal{HAUSTLONG}}

\emph{Haustlöng} (``a duração do outono'') é, nos termos de
Simek \& Pálsson, o título de um poema-escudo escáldico composto pelo
escaldo norueguês Þjóðólfr ór Hvíni (século~\versal{IX}). De acordo com o autor, o
título sugere que o poeta levou o período de um outono para compor o
poema. Este tem 20 estrofes, do tipo \emph{dróttkvætt} (``métrica da
corte''). De acordo com Ólason, essa métrica também é muito
utilizada nos \emph{drápa} e nos \emph{flokkur}, que são tipos de poemas de
elogio. Ross insere o \emph{Haustlöng} na categoria do subgênero poemas
pictóricos, que descrevem um objeto por meio de uma narrativa lendária e
mítica, muito parecida com a narrativa das poesias édicas. Þjóðólfr ór
Hvíni também compôs o poema \emph{Ynglingatal}, atribuído,
de acordo com Ross, ao subgênero 
mitológico e genealógico. O escaldo
também compôs duas \emph{lausavísur}. A ele são atribuídos também fragmentos
de um poema sobre Haroldo Cabelos Belos -- chamado \emph{Haraldskvæði} (ou
\emph{Hrafsnmál}) -- preservados nas sagas dos reis. Sobre sua pessoa,
sabe-se que tinha influência na corte do rei Haroldo Cabelos Belos e que
também elogiou o rei Rögnald heiðumhæri no poema
\emph{Ynglingatal}.

Esse poema descreve duas cenas mitológicas representadas em
um escudo que o poeta recebeu de um certo Þorleifr. As duas
representações são: aventura de Loki com o gigante Þjazi, que roubou a
deusa Iðunn; e a luta de Thor com o gigante Hrungnir. De acordo com Simek
\& Pálsson, o poeta preenche as cenas esboçadas no escudo
com seu próprio conhecimento mitológico.

O poema está preservados nas \emph{Eddas}, nos códices Regius (\versal{GKS} 2367
4°), Trajectinus (Traj 1374x), Wormianus (\versal{AM} 242 fol). Em outros codex
como, por exemplo, o Upsaliensis, há apenas fragmentos do poema. Com
relação à métrica do poema, ele tem 20 estrofes com 8 versos cada
(contidas em dois quartetos), três sílabas tônicas em cada verso, meia
rima interna \emph{skothending} nos versos ímpares (rimas com consoantes 
idênticas e vogais diferentes), rima completa
\emph{aðalhending} nos versos pares (rimas com vogais
idênticas junto a uma ou mais consoantes idênticas, em linhas
alternadas). O poema também tem aliteração e um troqueu no final dos
segundos semiversos. As seguintes rimas internas \emph{skothending}
existem no poema apresentado a seguir: 1. Þ\emph{\textbf{yrm}}ðit e
b\emph{\textbf{arm}}i, 3. Hr\emph{\textbf{ist}}usk
br\emph{\textbf{ust}}u, 5. m\emph{\textbf{jǫk}} e
hr\emph{\textbf{øk}}kva e 7. v\emph{\textbf{íg}}ligan e
v\emph{\textbf{ǫg}}na. As rimas internas \emph{aðalhending} são: 2.
s\emph{\textbf{\emph{ólg}}}num e d\emph{\textbf{\emph{ólg}}}i, 4.
br\emph{\textbf{\emph{ann}}} e m\emph{\textbf{\emph{ann}}}a; 6.
myrkb\emph{\textbf{\emph{ein}}}s e r\emph{\textbf{\emph{ein}}}ar e 8.
v\emph{\textbf{\emph{átt}}} e þ\emph{\textbf{\emph{átt}}}i

Estrofe 16 do poema: \emph{Baldrs of barmi þyrmðit þar solgnum manna
dolgi; bjǫrg bristusk ok berg brustu; upphiminn brann; frák Haka vagna
reinar myrkbeins vátt hrøkkva mjǫk móti; þá er þátti sinn vígligan
bana}. Tradução livre nossa: ``O irmão de Baldr não mostrou misericórdia
ao ganancioso inimigo dos homens; montanhas tremeram e rochas se
partiram; o céu ficou em chamas. Eu ouvi que a testemunha do osso negro
da terra das carruagens de Haki correu bruscamente para o outro lado
quando ele viu o matador bélico''.

Os \emph{kenning}, que também são elementos indispensáveis nesse tipo de
poesia, aparecem em grande quantidade. São eles nessa estrofe:

\emph{Barmi Baldrs} $=$ ``irmão de Baldr'' $=$ [\versal{THOR}]

\emph{Solgnum manna dolgi} $=$ ``ao inimigo ganancioso dos homens'' $=$ ``ao
[\versal{GIGANTE}]''. Referência ao gigante Hrungnir.

\emph{Haka vǫgna} $=$ ``das carruagens de Haki'' $=$ ``do [\versal{NAVIO}]''.
\emph{Haki} foi um famoso rei do mar escandinavo, mencionado na
\emph{Gesta Danorum} de Saxo Grammaticus e, também, em fontes do século~\versal{XIII}, 
sobretudo na \emph{saga dos Inglingos}, na \emph{saga dos
Volsungos} e no \emph{Nafnaþulur}.

\emph{Haka vǫgna reinar} $=$ ``do terreno gramado arado das carruagens de
Haki'' $=$ ``do terreno gramado arado do [\versal{NAVIO}]`` $=$ [\versal{MAR}]

\emph{Haka vǫgna reinar myrkbeins} $=$ ``do osso negro do terreno gramado
arado das carruagens de Haki'' $=$ ``do osso negro do [\versal{MAR}]'' $=$
[\versal{ROCHA}]

\emph{Haka vǫgna reinar myrkbeins váttr} $=$ ``testemunha do osso negro do
terreno gramado arado das carruagens de Haki'' $=$ ``testemunha da
[\versal{ROCHA}]'' $=$ [\versal{GIGANTE}]

\emph{Vígligan bani} $=$ ``o matador bélico'' $=$ [\versal{THOR}]

Portanto, a estrofe poderia ser reescrita da seguinte maneira: ``Thor não
mostrou misericórdia ao gigante Hrungnir; montanhas tremeram e rochas se
partiram; o céu ficou em chamas. Eu ouvi que o gigante Hrungnir correu
bruscamente para o outro lado quando ele viu Thor''. Para a determinação
dos \emph{kenningar}, foi o utilizado o livro de Egilsson (1931).

\SIG{Yuri Fabri Venancio}

Ver também Heiti; Kenning; Linguagem; Literatura; Norreno; Poesia
escáldica.

% ĴORGE : @FELIPE ver isso
\begin{itemize}
\item \versal{EGILSSON}, Sveinbjörn. \emph{Lexicon Poeticum Antiquæ Linguæ Septentrionalis. Ordbog over det norske-islandske Skjaldesprog. Forøget og udgivet for det kongelige nordiske Oldskriftselskab. 2 Udgave ved Finnur Jónsson}. København: \versal{S. L.} Møllers Bogtrykkeri, 1931.

\item \versal{GORDON}, Eric Valentine. \emph{An Introduction to Old Norse}. Oxford/New York: Oxford University Press, 1956.

\item \versal{ÓLASON}, Vésteinn. Old Icelandic Poetry. In: \versal{NEIJMANN}, Daisy. \emph{A History of Icelandic Literature}. Lincoln \& London: University of Nebraska Press, 2006, pp. 01--63.

\item \versal{ROSS}, Margaret Clunies. \emph{A History of Old Norse Poetry and Poetics}. Cambridge: \versal{D.S.} Brewer, 2005.

\item \versal{SIMEK}, Rudolf; \versal{PÁLSSON}, Hermann. \emph{Lexikon der Altnordischen Literatur}. Stuttgart: Kröner, 1987.
\end{itemize}

\section{\versal{HEDEBY}}

Atualmente situada no norte da Alemanha, próximo à fronteira da
Dinamarca, o sítio da antiga cidade de Hedeby hoje consiste em belos
prados agrícolas e alguns bosques, tudo diante das águas azuis do fiorde
Sehlei. Esse cenário bucólico oculta sob a terra algumas ruínas e
vestígios arqueológicos do que foi no passado uma das cidades mercantis
mais prósperas da Dinamarca e da Escandinávia. Por mais de dois séculos
Hedeby foi um importante centro econômico do sul da Escandinávia, ligado
com rotas e mercados germânicos, eslavos e francos.

As escavações do sítio de Hedeby somente se iniciaram no século~\versal{XX},
foram retomadas em meados dos anos 2000, e hoje a cidade já conta com um museu
local, que exibe os achados desenterrados. Ainda assim, grande parte do
sítio que compreendia a antiga cidade não foi escavado.

A data de origem de Hedeby (Heiðabýr, em nórdico antigo) é desconhecida.
Aponta-se que a cidade tenha começado como um vilarejo agrícola no
século~\versal{VIII}. Mas a menção mais antiga sobre Hedeby ou Haithabu (como
consta em algumas fontes) data de 808, quando o rei Godofredo da
Dinamarca, após destruir o mercado de Reric, estabeleceu alguns
comerciantes e artesãos no local, além de ordenar a construção da
muralha de Danevirke, que no século seguinte seria anexada à muralha de
Hedeby. Godofredo também ficou conhecido por declarar guerra ao
imperador Carlos Magno dos francos, daí Hedeby ser mencionada por
comerciantes francos que participavam do comércio na Germânia, tendo
sido obrigados a interromper o comércio devido à ameaça de guerra.

Não se sabe ao certo quando a cidade se tornou um centro mercantil, mas
isso ocorreu ainda no século~\versal{IX}. O antes vilarejo agrícola expandiu sua
área até as margens do fiorde Sehlei, formando inicialmente um pequeno porto. Graças ao porto, embarcações do Báltico e do mar do Norte
poderiam ali aportar. Não obstante, foram achadas algumas moedas de
prata cunhadas na própria cidade, datadas de 825 e baseadas em moedas
frísias. O achado de moedas no local atesta um comércio avançado, já que
normalmente os vikings não fizeram uso regular de moedas antes do século~\versal{X}. 
Para alguns historiadores, Hedeby e Ribe podem ter sido as primeiras
cidades escandinavas a terem cunhado suas próprias moedas.

Além da cunhagem de moedas, achados de formas e ferramentas de
metalurgia, ourivesaria e olaria, apontam que Hedeby também foi um
centro manufatureiro, tendo produzido objetos em ferro, bronze, prata e
ouro, talvez fabricado joias com metais preciosos, gemas e com vidro.
Além disso, encontraram-se vestígios de produção de cerâmica. Não
obstante, o cemitério da cidade também atesta o valor do local como rota
comercial, pois alguns dos objetos fúnebres achados nos túmulos são de
origem estrangeira, como moedas árabes. De fato, por volta de 965, um
comerciante árabe de nome Al-Turtushi visitou a cidade.

Não obstante, no século~\versal{X} a cidade já era cercada com muros de 10~m
de altura, que se estendiam em seu auge por mais de 1 km de extensão. A
cidade se estendia por pelo menos 24 hectares e possuiria uma
população estimada em mais de mil habitantes, número grande para os
padrões da época e do lugar. As escavações arqueológicas revelaram que
algumas das principais ruas da cidade eram pavimentadas com tábuas de
madeira. O fato de algumas ruas serem pavimentadas era atestado da
prosperidade do local, pois em geral as ruas eram de terra batida, mesmo
em grandes cidades. Além de possuir poços de abastecimento, o ribeiro
que cruzava a cidade era usado como escoador para os dejetos e água da
chuva.

Devido a sua localização ao sul da península da Jutlândia, já dentro do
continente em si, Hedeby foi um dos locais da Escandinávia mais próximos
da Europa cristianizada. De fato, emissários germânicos como Ebo,
arcebispo de Reims e São Oscar (Anscário de Hamburgo) visitaram a
Dinamarca entre 823-826. A primeira igreja de Hedeby teria sido erguida
nessa época. Por volta de 850, sob convite do rei Horik~\versal{I}, o Velho
(?-854), São Oscar fundou outra igreja em Hedeby e uma em Ribe.

Nota-se que Hedeby despontava não apenas como um centro comercial, mas
também religioso, por estar entre os primeiros a receber igrejas e a
semente do cristianismo. A importância econômica de Hedeby no século~\versal{X}
era tamanha que levou o imperador Oto~\versal{II} do Sacro Império
Romano Germânico a declarar guerra ao rei Haroldo Dente Azul. Em 974 a
cidade dinamarquesa foi tomada pelos alemães. Apenas em 983, o rei
Haroldo conseguiu reaver a cidade, ordenando que sua muralha fosse
reforçada e o Danevirke também.

No século \versal{XI} a cidade começou a entrar em crise, pois o desenvolvimento
de outras cidades começou a suplantar sua importância como polo
manufatureiro e comercial. No ano de 1050 os reinos da Noruega e
Dinamarca estavam em guerra novamente. Com isso o rei Haroldo~\versal{III}
da Noruega, ambicionando derrotar o rei dinamarquês Sueno~\versal{II}, ordenou que
Hedeby fosse atacada. A cidade foi saqueada, o povo assassinado e, por
fim, as casas queimadas. Um escaldo chamado Thorleik, o Belo, compôs
alguns versos falando sobre o saque e incêndio de Hedeby pelo exército
norueguês.

Mas apesar dessa quase total destruição, as casas foram reconstruídas e
Hedeby ainda resistiu por mais de uma década. No ano de 1066 um exército
eslavo vindo do leste atacou a cidade e a destruiu de vez. A população
sobrevivente acabou por abandonar a localidade.

Na popular série de televisão \emph{Vikings}, escrita por Michael Hirst,
a cidade de Hedeby surge propriamente na terceira temporada, apesar de
ser mencionada ainda na segunda. Mas diferente da realidade, a Hedeby da
série é uma vila governada pelo segundo marido de Lagertha, o \emph{earl}
Sigvard. Após sua morte Lagertha assume o controle de Hedeby, passando a
usar o título de \emph{earl} Ingstad. Nas temporadas seguintes a vila continua
a aparecer, mas sem invocar a pretensão de se tornar uma cidade, algo
assumido pela fictícia Kattegat, comunidade criada para série, situada
no sul da Noruega e cujo núcleo urbano é almejado por Lagertha e outros
senhores.

\SIG{Leandro Vilar Oliveira}

Ver também Comércio; Dinamarca da Era Viking; Era Viking; Viking.

\begin{itemize}
\item \versal{GRAHAM-CAMPBELL}, James (org.). \emph{Os vikings}. Barcelona: Editora
Folio \versal{S.A.} 2006.

\item \versal{GULLBEKK}, Svein H. Coinage and monetary economies. In: \versal{BRINK}, Stefan;
\versal{PRICE}, Neil (eds.). \emph{The Viking World}. London/New York: Routledge,
2008, pp. 159-169.

\item \versal{HILBERG}, Volker. Hedeby: an outline of its research history. In: \versal{BRINK},
Stefan; \versal{PRICE}, Neil (eds.). \emph{The Viking Worl}d. London/New York:
Routledge, 2008, pp. 101-111.

\item \versal{HOLMAN}, Katherine. \emph{Historical dictionary of the vikings}. Lanham:
Scarecrow Press Inc, 2003.

\item \versal{SINDBÆK}, Søren Michael. Local and long-distance exchange. In: \versal{BRINK},
Stefan; \versal{PRICE}, Neil (eds.). \emph{The Viking World}. London/New York:
Routledge, 2008, pp. 150-158.

\item \versal{SKRE}, Dagfinn (ed.). \emph{Means and Exchange}: dealing with Silver and
the Viking Age. Oslo: Aarhus University Press, 2007 (Kaupang Excavation
Project Publications Series, vol. 2).
\end{itemize}

\section{\versal{HEITI}}

\emph{Heitið} é uma figura de linguagem originada do verso de aliteração
germânico. De acordo com Simek, é um sinônimo poético
para um substantivo como, por exemplo, \emph{hjörr} para \emph{sverð}
``espada'' ou Grímnir para Odin. Tais sinônimos são entendidos como
floreios exóticos. A palavra \emph{heiti} significa ``nome'',
``denominação'' e é uma derivação do verbo no antigo nórdico
\emph{heita}, ``ser chamado''.

Neijmann afirma que \emph{heitin} são provavelmente
palavras arcaicas que sobreviveram apenas no uso poético, podem ser
metonímias (ing. \emph{surf}, \emph{wave} ou \emph{tide} para significar
\emph{ocean}), palavras comuns que têm outro significado na poesia (ing.
\emph{maid}) e nomes especiais para os deuses (\emph{Sigföður} ``pai da
vitória'', \emph{Valföður} ``pai dos caídos em batalha'', \emph{Grímnir}
``o que possui máscara e capacete'', \emph{Gangleri} ``o viajante'' para
Odin).

Ross afirma que \emph{heitin} (junto com
\emph{kenningarnar}) se encontram ocasionalmente na \emph{Edda poética}
e em outros tipos de poesia édica, mas em uma frequência mais esporádica
do que sistemática (da mesma maneira que ocorre em \emph{Beowulf}); por
outro lado, nas poesias com versos em \emph{dróttkvætt}, eles são muito
mais regulares. Também se encontram de maneira relativamente frequente
em poesias com versos em \emph{kviðuháttr.} Por serem poesias menos
elaboradas, não há tanta figura de linguagem como se encontra em poesias
com versos em \emph{dróttkvætt}.

No final de \emph{Skáldskaparmál}, a segunda parte do \emph{Edda em
prosa} de Snorri Sturluson, nos manuscritos \emph{Codex Regius},
\emph{Codex Trajectinus}, \emph{\versal{AM} 748 \versal{II} 4° \versal{C}}, 
\emph{\versal{AM} 748 \versal{I} b 4°
(\versal{A})} e \emph{\versal{AM} 757 a 4to (\versal{B})} encontram-se 106 estrofes que apresentam
uma lista de sinônimos, chamada de \emph{Þulur}.



\begin{table}[h]
\centering
\caption{Trecho em que se menciona os sinônimos para Thor em \emph{Þulur}:}
\label{my-label}
\begin{tabular}{ll}
\textit\{Þórr heitir Atli\}    & Þórr se chama “O Terrível”              \\
\textit\{ok Ásabragr\}         & e “Príncipe dos Æsir;                   \\
\textit\{sá es Ennilangr\}     & é “aquele com ampla testa”              \\
\textit\{ok Eindriði\}         & e “aquele que viaja sozinho”            \\
\textit\{Bjǫrn, Hlórriði\}     & Urso, o “Viajante Estrondoso”           \\
\textit\{ok Harðvéorr,\}       & e “O Forte Protetor”                    \\
\textit\{Vingþórr, Sǫnnungr,\} & “o Thor da batalha”, “O Verdadeiro”,    \\
\textit\{Véoðr ok Rymr.\}      & “O Guardião do Santuário” e “o Ruidoso”
\end{tabular}
\end{table}

A tradução acima teve base nas propostas de Simek (1984) para a tradução
dos \emph{heiti}.

\SIG{Yuri Fabri Venancio}

Ver também Inscrições rúnicas; Kenning; Linguagem; Literatura; Norreno;
Poesia éddica; Poesia escáldica.

% JORGE : @FELIPE
\begin{itemize}
\item \versal{MAGNÚSSON}, Ásgeir Blöndal. \emph{Íslensk orðsifjabók}. Reykjavík: Orðabók Háskólans, 2008.

\item \versal{NEIJMANN}, Daisy. \emph{A History of Icelandic Literature}. Lincoln \& London: University of Nebraska Press, 2006.

\item \versal{ROSS}, Margaret Clunies. \emph{A History of Old Norse Poetry and Poetics}. Cambridge: \versal{D.S.} Brewer, 2005.

\item \versal{SIMEK}, Rudolf. \emph{Lexikon der Germanischen Mythologie}. Stuttgart: Alfred Kröner, 1984.

\item \versal{SIMEK}, Rudolf; \versal{PÁLSSON}, Hermann. \emph{Lexikon der Altnordischen Literatur}. Stuttgart: Kröner, 1987.
\end{itemize}


\section{\versal{HELGÖ}}

Helgö consistiu numa pequena comunidade urbana de caráter mercantil e
manufatureiro da Suécia, construída antes da Era Viking
(séculos~\versal{VIII-XI}). Escavações arqueológicas na região encontraram
objetos e moldes datados do século~\versal{V} e até de períodos mais antigos.
Os historiadores e arqueólogos sugerem que Helgö tenha começado como uma
comunidade agrícola por volta do século~\versal{III}, época da Idade de Ferro
Romana (séculos~\versal{I-V}) na Escandinávia e tenha sido habitada regularmente
até o final do século~\versal{VIII}.

A disposição urbanística de Helgö revela em parte que a localidade não
consistiu num padrão comum do Período Viking, no qual se encontram
cidades cercadas por muralhas, com ruas mais ou menos delineadas, casas
próximas uma das outras etc. No caso de Helgö, sua disposição urbana foi
mais esparsa e centrada em pequenos núcleos, razão pela qual alguns arqueólogos
o chamaram de ``terraços'', uma vez que se constituíam de porções de terra aplainadas para servir de base para a construção de edificações longas e retangulares, separadas entre si por campos abertos ou agrícolas.

Para James Graham-Campbell, essa forma de disposição das residências,
oficinas e dependências lembra muito a imagem de comunidades agrícolas
suecas encontradas na região por aquele período, o que seria um
indicativo da origem rural de Helgö. Tal fato foi comum, pois ao
longo do período que aquela localidade foi habitada seguiu-se esse
modelo de construção e disposição das construções.

Não se sabe exatamente quando Helgö começou a despontar como polo
manufatureiro, mas os vestígios arqueológicos sugerem que isso ocorrera por volta do século~\versal{V}, pois, nessa época, conforme observa Torun Zachrisson, há mudanças nas técnicas de metalurgia na região da Escandinávia, influência de conhecimentos estrangeiros e da descoberta de minas e jazidas de ferro em algumas localidades, como no norte da Suécia.

Nesse caso, é importante salientar que a localização geográfica de Helgö
diz muito a respeito de seu papel como polo manufatureiro. A pequena
comunidade se situava na ilha de Ekeö, em meio ao lago Mälaren,
região conhecida por abrigar outras cidades importantes como Birka,
Sigtuna e Estocolmo. O lago Mälaren durante séculos foi um ponto de passagem
para pessoas, matérias-primas e mercadorias. Devido a sua localização
privilegiada e sua conexão tanto com o interior quanto com a costa, muitas comunidades
agrícolas e núcleos urbanos ali se desenvolveram.

Graças a tal condição, os artesãos de Helgö passaram a dispor de
matéria-prima e com isso começaram a se especializar nesse ofício. A
grande quantidade de objetos de bronze e ferro achados em Helgö, além
dos moldes e fôrmas usados para confeccioná-los, revela que, pelo menos
entre os séculos~{\versal{VI} e \versal{VII}}, Helgö tenha sido um profícuo polo
manufatureiro da região. Graham-Campbell aponta para o detalhe segundo o qual
uma das especialidades dos artesãos da cidade tenha sido a joalheria,
devido a grande quantidade de broches, colares e outras joias achadas.

Helgö também se destacou por objetos estrangeiros encontrados em sua
localidade, como uma coleção de moedas romanas de ouro, datadas dos
séculos~\versal{V} e \versal{VI}, um báculo de bronze de origem celta e uma surpreendente
estatueta asiática de Buda, achada na expedição de 1954, que na época
surpreendeu os arqueólogos. Além desses achados estrangeiros, que revelam
em parte a condição segundo a qual a população de Helgö tinha acesso a
mercadorias do exterior, outro aspecto curioso achado nas escavações
arqueológicas nos anos de 1950 foi a grande quantidade de estatuetas
douradas, chamadas de \emph{guldgubbar}, as quais representavam um
casal.

Segundo uma das linhas de interpretação, poderia ser uma referência ao deus
Freyr e sua esposa, a giganta Gerðr, como comenta Anne"-Söfi Gräslund. A
existência de muitas \emph{guldgubbar} levaram alguns estudiosos a 
questionarem se Helgö não poderia ter sido um local de culto pagão -- inclusive, Helgö significa ``ilha sagrada'' --, embora não haja nada
definitivo sobre essa hipótese. A cidade hoje em dia não mais existe,
mas a ilha continua a ser habitada.

\SIG{Leandro Vilar Oliviera}

Ver também Birka; Comércio; Suécia da Era Viking.

\begin{itemize}
\item \versal{ERIKSEN}, Marianne Hem \emph{et. al} (eds.). \emph{Viking Worlds:
Things, Spaces and Movement}. Oxford: Oxbow Books, 2014.

\item \versal{GRAHAM-CAMPBELL}, James. \emph{Os Vikings}. Barcelona: Editora Folio \versal{S.
A.}, 2006.

\item \versal{GRÄSLUND}, Anne-Sofie. The material culture of Old Norse Religion. In:
 \versal{BRINK}, Stefan; \versal{PRICE}, Neil (eds.). \emph{The Viking World}. London/New
York: Routledge, 2008, pp. 248-256.

\item \versal{ZACHRISSON}, Torun. Rotary querns and bread -- A social history of Iron
Age Sweden. \emph{AmS--Skrifter}, n. 24, 2014, pp. 181-191.
\end{itemize}
\section{\versal{HERÁLDICA}}

Os guerreiros, desde os tempos mais remotos da humanidade, buscavam marcar
e identificar seus feitos e hierarquias através de símbolos inscritos, geralmente, 
na pele de animal que usavam, ou mesmo numa arma mais trabalhada.
Na Antiguidade, os combatentes marcavam insígnias em armaduras e escudos, 
que serviam tanto para noticiar os outros de suas capacidades, como também para os identificar, como no caso do escudo de Aquiles. Registra-se, em Roma e Esparta, que o uso de
insígnias e símbolos de hierarquia servia como forma de demonstração
de feitos e capacidades. As insígnias apresentavam traços próprios
do seu autor. Não havia uma
regulamentação precisa que distinguisse o seu estilo e os seus traços.

Somente no medievo uma codificação desses símbolos e
insígnias passa a ser realizada, constituindo assim uma arte nova
responsável pela organização dos elementos: a heráldica. No
século~\versal{X}, com o crescimento das justas e de uma cultura de corte, os
brasões passaram por um refino e aperfeiçoamento, o que aumentou sua complexidade em
termos de características e estilos, bem como contribuiu para cercear seus usos para
estamentos específicos da sociedade. A Igreja, entre os séculos~\versal{X} e
\versal{XI}, voltou-se para a sistematização desses símbolos
com intenção clara de evitar um uso desmedido desses
símbolos, constituindo, assim, a heráldica. Afinal, se cada um pudesse criar 
sua própria identificação, a
tendência de existirem símbolos similares ou iguais seria bem maior,
o que descaracterizaria a necessidade do brasão, que é justamente a de identificar o seu
possuidor e evidenciar os seus feitos.

Vemos em múltiplos codexes medievais que é a partir da guerra 
que a heráldica se constitui. Ela foi sedimentada nas justas e
fez surgir profissões e ofícios novos ligados, intimamente, ao uso
de brasões, como é o caso do arauto e do heraldo (os quais apresentavam as armas e brasões
de seus representantes). Com o tempo, os símbolos tiveram sua função ampliada para além de
uma representação de família e seus feitos, tornando-se um instrumento
de recompensa real para os sujeitos responsáveis por grandes proezas.

Essa ampliação do uso da heráldica pelos mais diversos meios --
identificação (em selos, justas, batalhas etc.),
meio de recompensa, símbolo de legitimação e \emph{status} --, proporcionou mais destaque à 
heráldica, ampliando ainda mais as necessidades de
sistematização dos símbolos. Também tornou-se necessário lidar com alguns problemas que surgiam,
como, por exemplo, o problema das linhas bastardas (uma linha cortando o brasão, 
mas que variava dependendo da região).

Portanto, a heráldica é um conjunto de normas para armarias e
brasões com símbolos integrados que determinam a representação de
uma codificação específica. Por exemplo, um elmo no topo do brasão com
a viseira aberta poderia indicar uma nobreza solar, enquanto o elmo com a viseria fechada
indicaria uma
nobreza por disposição real. Além disso, com o avanço da heráldica,
surgiram brasões reais, gentílicos, eclesiásticos, dominiais e de
corporações ou profissionais, o que ampliou ainda mais a complexidade para
observar e compreender a codificação da heráldica. Algo semelhante ocorreu, posteriormente, com
a criação do \emph{Ex libris}, que representaria marcas de
autores ou instituições, bem como obras materiais escritas.

Logo, em meio a tanta demanda, surge uma organização específica que confere
sentido para cada ponto, como o fogo, que poderia significar busca
por glória ou um ardente guerreiro que busca feitos heroicos. Também a coloração poderia mudar o sentido, de tal modo que um brasão todo
prata representaria a completa pureza, por exemplo. Partes do
corpo, animais, plantas, objetos e utensílios aparecem nos brasões podendo
ter um ou mais sentidos, que se complexificam quando consideradas as
posições a partir das quais estão dispostas as representações, bem como as cores dos brasões.

Um escudo de armas tem uma regra específica de formas (sua longitude e
largura), que se apresentam em grande variedade. Tal escudo também possui uma Dextra e
uma Sinistra (que se refere aos lados direito e esquerdo do brasão, lembrando
que o brasão é algo vivo, de modo que esquerda e direita se definem no
brasão como se ele estivesse de frente para uma pessoa), Chefe e Contra
Chefe (parte superior e inferior), Honrarias, Suporte (elementos
colocados na parte interna dos brasões), Divisa, bem como outros elementos que
tornam ainda mais complexo o estudo da heráldica. Vale ainda mencionar que certas normas e técnicas
dos brasões variam de acordo com a região, assim como o corte dos
brasões (as secções que dividem um escudo de armas, como partido,
cortado, talhado, terçado etc.). Logo, para se entender heráldica,
importa reter que ela busca um entendimento de normas e regras que
variam com o tempo e o espaço. Requer um entendimento da coloração
dos símbolos e insígnias, bem como das formas, cortes e de toda uma combinação de
regras e elementos.

No caso da Escandinávia medieval, tem-se uma dinâmica específica. De
fato, como vimos, a heráldica é uma criação da Igreja no seio da Europa
Ocidental, que só exercerá o seu domínio com mais força na Escandinávia
pelos idos do século~\versal{XI}. Portanto, a Era Viking, compreendida
tradicionalmente como o período de 793 até 1066, não apresenta muito desses aspectos
da heráldica tradicional ligados às práticas da
cristandade, o que nos permite dizer que os vikings -- deixando de lado
a problemática quanto ao caráter étnico desse termo -- não usaram da heráldica.
Não obstante, em meados do século~\versal{XIII-XIV}, uma outra realidade se 
apresentava, na qual os múltiplos reinos e famílias da Escandinávia
passaram a ter uma maior representatividade em brasões com bases nos
modelos mais conhecidos de heráldica.

A Escandinávia tem toda uma dinâmica particular de
representação, com cores, símbolos, partições e usos próprios. Afinal,
cada sociedade produz traços específicos, ligados as suas respectivas práticas sociais e
culturais, de modo que a tentativa de unificação desses
símbolos (como ocorreu na Europa Ocidental medieval e depois na Escandinávia),
não pode perder de vista os traços culturais regionais que sempre orientam a
confecção de brasões.

A principal diferença de composição entre os elementos da Escandinávia e
da Europa Ocidental reside no fato de que os escudos da Escandinávia medieval eram
redondos, bem diferentes dos escudos ao modelo triangular reto.
Normalmente tendia a apresentar uma ou duas cores, possuindo poucas
divisões dentro de si. Alguns podiam ser traçados com formas circulares,
em modelo espiral com duas cores (modelo conhecido como Gyronny
Arrondi). Além disso, essas cores faziam parte de uma espécie
``heráldica regional'', podendo variar conforme as regiões da própria
Escandinávia. 

Portanto, podemos definir que não há um estilo claro e evidente de
heráldica na Escandinávia medieval até os séculos~\versal{XIII-XIV}. Contudo, apesar
de não comportar uma sistematização dentro dos moldes de definição que
apresentamos, isso não quer dizer que as sociedades da Escandinávia
medieval não usavam uma simbologia ou elementos iconográficos para
cumprir funções similares às dos escudos de armas da heráldica
da Europa Ocidental. De fato, a dinâmica de cores, de símbolos, de
animais, de objetos, entre outros, trarão toda uma peculiaridade
regional nos fins do século~\versal{XIII}, para construir de fato uma heráldica
medieval na Escandinávia.

\SIG{José Lucas Cordeiro Fernandes}

Ver também Escandinávia; Literatura; Era Viking; Viking.

\begin{itemize}

\item \versal{ABRANTES}, Marquês de. \emph{Introdução ao estudo da heráldica}. Lisboa:
Instituto de Cultura e Língua Portuguesa, 1992.

\item \versal{ANÔNIMO}. \emph{Heraldry for a Non-Heraldic Culture: Vikings and Coats
of Arms in the \versal{SCA}}. Disponível em:
\textless{}vikinganswerlady.com/vikheraldry.shtml\textgreater{}
Acesso: 9 Jun. 2017.

\item \versal{BANDEIRA}, Luís Stubbs Saldanha Monteiro. \emph{Vocabulário Heráldico}.
Lisboa: Edições Mama Sume, 1985.

\item \versal{CRAIGIE}, Maria-João de Nogueira Ferrão Vieira. \emph{Dicionário de
Bibliografia para Genealogistas}. Lisboa: Dislivro Histórica, 2006.

\item \versal{GUERRA}, Luiz de Figueiredo da. \emph{Manual do Brazão}. Viana, 1902.

\item \versal{LANGHANS}, Franz Paul de Almeida. \emph{Heráldica: Ciência de Temas
Vivos}. Lisboa: Fundação Nacional para a Alegria no Trabalho, 2 v., 1966.

\item \versal{RIBEIRO}, Joaquim Augusto Corrêa Leite. \emph{Tratado de Armaria:
technica e regras do brasão d'armas}. Lisboa: Empreza da Historia de
Portugal, 1907.

\item \versal{SEGRAIS}, René Le Juge de. \emph{Resumo da Ciência do Brasão}. Lisboa:
Livraria Bertrand, 1951.

\item \versal{SHARPTOOTH}, Thóra. \emph{Personal Display for Viking Age Personae: A
Primer for Use in the \versal{SCA}}. Disponível
em:\textless{}cs.vassar.edu/\textasciitilde{}capriest/display.html\textgreater{}
Acesso: 9 Jun. 2017.
\end{itemize}
\section{\versal{HIDROMEL}}

As fermentações naturais de frutas e folhas sempre foram observadas desde a
Pré-história e, ao longo dos séculos, aperfeiçoadas e adaptadas ao paladar
de cada região. O hidromel, um dos fermentados mais antigos de que se tem
notícia, é considerado a bebida dos vikings por excelência. Esse
estereótipo já cristalizado de que os nórdicos se embriagavam com
abundante hidromel ainda hoje é difundido sem se conhecer a fundo como e
em quais circunstâncias essa bebida rara e cara era consumida. O mel, na
Era Viking, era recolhido de colmeias selvagens. Portanto, conseguir mel
não era algo tão fácil, fato que o tornava um produto escasso, destinado
exclusivamente para o consumo em ocasiões especiais e, claro, para a
produção de hidromel. Um fermentado simples: mel, água, ervas aromáticas
ou frutas e uma levedura. Esta podia ser uma ``levedura selvagem''
presente no ar, que contaminaria a bebida provocando a fermentação
alcoólica e, depois de algum tempo, o hidromel estaria pronto para ser
consumido.

O hidromel (\emph{mjöð}), além de ser a bebida que proporcionava a
inspiração para a arte de se compor poesia, era também utilizado pelas
profetisas, pelos \emph{berserkir} -- guerreiros consagrados a Odin -- para
conseguirem atingir o êxtase e, consequentemente, o furor na batalha. Serviam também,
claro, para os líderes, os grandes guerreiros e os escaldos cantarem as
vitórias das batalhas e glorificarem o deus de um só olho. Mas, devido
ao seu ingrediente principal -- o mel -- ser raro e também bastante caro, o
seu consumo era destinado somente aos mais ricos e às grandes
comemorações de caráter religioso, político e guerreiro. O hidromel era
associado às festas no mundo dos deuses (o banquete de Égir,
\emph{Lokasenna} 1-65; a cuba mágica dos einherjar, \emph{Gylfaginning}
38), bem como a poesia e ao próprio Odin (\emph{Skáldskaparmál} 1).
Portanto, ao contrário da cerveja, o vinho e o hidromel tinham um
caráter muito mais sagrado, sendo destinados aos mais abastados.

Bebida por excelência dos deuses, guerreiros e chefes, o hidromel era
servido a todos os mortos em combate que adentravam o Valhala. As
valquírias, depois de escolherem os mortos de Odin no campo de batalha,
trajavam seus vestidos de trabalho e, com o cabelo preso com um nó
triplo ofereciam o corno cheio de hidromel como uma forma de boas vindas
àqueles que se mostraram corajosos e que agora irão desfrutar pela
eternidade da carne de porco e dessa bebida tão desejada.

As mulheres eram as responsáveis por fazerem e servirem o hidromel,
assim como eram pela cerveja consumida cotidianamente, durante as
festividades. A tarefa era feminina por excelência, pois eram as mulheres que
controlavam o fluxo de alimentos que entravam e saíam das despensas e
também eram responsáveis pelo cuidado com as ervas
utilizadas para dar sabor à bebida e auxiliar na sua fermentação.
Diferentemente do que se difunde atualmente, o hidromel da Era Viking não
possuía uma graduação alcoólica elevada, ficando entre os 4 e 8 graus. A
embriaguez advinda do consumo da bebida não era devido à graduação
alcoólica, mas sim à grande quantidade consumida: já que eram poucos os
consumidores, a quantidade destinada a eles era grande, e uma das
qualidades apreciadas em um grande líder e guerreiro era justamente a
gula e a embriaguez pela grande ingestão de bebida alcoólica. Quanto
mais se bebia e comia, mais valoroso era o homem.

O hidromel tem uma origem mítica. A bebida era guardada pela giganta
Gúnnlod e os deuses e homens não tinham acesso a ela. Odin, usando de
sua astúcia, seduziu a giganta e a amou por três dias. Depois,
metamorfoseado em águia roubou a preciosa bebida que além de saciar a
sede de deuses e guerreiros no \emph{post mortem}, inspirou os escaldos
a comporem a sua refinada poesia com as gotas que caíram do bico da
águia odínica sobre a suas frontes. O hidromel expelido pela cloaca da
ave inspirou os maus poetas a comporem a sua poesia medíocre.

Atualmente o hidromel é largamente consumido. Fabricado praticamente em
escala industrial, distancia-se muito da bebida consumida na Era
Viking, não só pelos ingredientes utilizados, como também pelo modo de
preparo e graduação alcoólica que ultrapassa a original. A bebida dos
deuses, dos poetas e dos guerreiros no Valhala ainda inspira canções e
poesia, bem como desperta curiosidade daqueles que querem provar na terra as
doçuras dos lábios de Gúnnlod.

\SIG{Luciana de Campos}

Ver também Alimentação; Cerveja; Festas e festins; Religião.

\begin{itemize}
\item \versal{CAMPOS}, Luciana de. A sacralidade que vem das taças: o uso de bebidas no
Mito e na Literatura Nórdica Medieval. \emph{Revista Brasileira de
História das Religiões}, vol. 23, 2015, pp. 97-107.

\item \versal{CAMPOS}, Luciana de \& \versal{LANGER}, Johnni. Brindando aos deuses:
representações de bebidas na Era Viking, no cinema e nos quadrinhos.
\emph{Revista de História Comparada} (\versal{UFRJ}), vol. 6, 2012, pp. 141-164.

\item \versal{HAGEN}, Ann. \emph{Anglo-Saxon food and drink}. London: Anglo Saxon Book,
2010.

\item \versal{WARD}, Christie. Alcoholic beverages and drinking customs of the Viking
Age. \emph{The Viking Answer Lady}, 2005. Disponível em:
\href{http://www.vikinganswerlady.com/drink.shtml}{\emph{http://www.vikinganswerlady.com/drink.shtml}}.
Acesso em 14/04/2017.
\end{itemize}

\section{\versal{HIGIENE E SAÚDE}}

A higiene e, consequentemente, a manutenção da saúde, era uma preocupação
das comunidades da Era Viking que faziam do banho, por exemplo, um
hábito cotidiano, assim como os cuidados com os cabelos e a pele. A
arqueologia e a literatura são fontes importantes para nos apresentar
mais elementos de como se davam esses cuidados corporais.

Os vikings viviam em um ambiente de frio intenso em determinadas épocas do ano,
o que obrigava as pessoas a se abrigarem, bem como dividirem o espaço com os
animais como cães, vacas, cabras e porcos, para que não sucumbissem e
também porque ofereceriam uma dose extra de calor a casa. Durante o
inverno, o banho só era possível em locais como a Islândia que, graças
às fontes termais, possuía água quente em abundância, permitindo que seus
habitantes construíssem piscinas para armazenar a água aquecida usada
para o banho da família. Na Suécia, Dinamarca e Noruega, não havia essas
fontes e os banhos eram tomados em tinas de madeira. No verão, quando
a temperatura subia vários graus, podiam usufruir dos rios, lagos e do
mar -- pois como exímios navegadores, eram excelentes nadadores. 
Para o banho, o sabão utilizado era mais delicado: na sua
composição era usada a banha, urina fresca, algumas ervas (como a
artemísia) e a urtiga. O banho era um ato importante no cotidiano e é
apresentado no capítulo 39 da \emph{Laxdæla saga}: Kjartan, enquanto
cortejava Guðrún, fez coincidir os seus banhos quentes nas piscinas
termais com os dela, fazendo com que encontros propositais acontecessem
entre os dois. Na Inglaterra, nativos perceberam que os invasores
nórdicos tomavam banhos regulares. John of Wallingford, o abade da
Abadia de St. Albans, escreveu em suas crônicas que os invasores
nórdicos pareciam muito mais atraentes para as mulheres, pois, ao
contrário dos homens anglo-saxões, eles penteavam seus cabelos
diariamente, tomavam banhos semanalmente e sempre lavavam suas roupas.
Um tratado negociado no ano 907 entre o Império Bizantino e os rus -- os
noruegueses da Suécia e da região leste do Báltico -- incluía uma
condição peculiar a todos os tratados já assinados: os bizantinos eram
obrigados a fornecer banhos para os rus sempre que eles desejassem.

Essas fontes termais também eram utilizadas para a lavagem de roupas. O
uso da água quente, além de tornar essa tarefa mais confortável, também
facilitava a limpeza das peças. O sabão utilizado nessa tarefa era
grosseiro, feito com sebo e urina velha. As roupas eram batidas com
pedaços de madeira para ajudar na retirada das sujeiras mais pesadas.

Os cabelos recebiam cuidados especiais. Eram lavados cuidadosamente com
água e untados com óleos perfumados para que ficassem brilhantes e
macios. Depois de lavados, os cabelos eram cuidadosamente penteados com
vários tipos de pentes. Estes podiam ser grossos ou finos e desembaraçavam os fios,
facilitando o trançado, dos mais simples aos mais complexos. Assim, as
mulheres mais ricas tinham ao seu dispor não somente pentes das mais
variadas espessuras e materiais (normalmente cunhados por artesãos experientes, 
possibilitando que seus cabelos recebessem uma espécie de tratamento ao serem
penteados), como também tinham servas para fazerem os trançados e os
penteados mais elaborados, tal qual as tranças usadas pela mulher de
Elling, da Dinamarca. Os homens possuíam, basicamente, os mesmos cuidados
capilares que as mulheres, fazendo uso de pentes diversos para manterem
os cabelos sempre arrumados segundo os padrões de beleza da época. Além dos cabelos longos, também possuíam barba que recebia cuidados,
sendo em alguns casos trançada e aparada.

As mulheres faziam uso de pedras grandes, redondas e lisas. Pressionavam-nas delicadamente sobre a pele do rosto para amenizar as rugas
e linhas de expressão. A mais fina banha de porco era misturada com
ervas (como a macelinha, por exemplo), compondo a fórmula de uma espécie
de creme hidratante facial da época. Algumas peças pequenas
confeccionadas com metal ou osso em forma de pequenos alicates, pinças e
cortadores eram utilizadas para cortar, lixar e limpar as unhas, bem como para limpar as orelhas.

Alguns estudos em ossadas da Era Viking sugerem que a boa saúde e
longevidade foram possíveis para ao menos parte da população. No mínimo as mais abastadas, que possuíam condições de, mesmo em circunstâncias
adversas, ter alimentos variados e em boa quantidade. Mas isso não quer
dizer que a vida das pessoas dessas comunidades estivesse livre de
doenças graves ou epidemias que sempre assombraram a existência
humana. A dieta da Idade Viking teve alimentos mais
grosseiros, menos refinados e praticamente sem
açúcar, a não ser aquela vinda do mel, sempre em pequenas quantidades. Em decorrência dessa
alimentação, os dentes apresentam um certo desgaste, mas poucos casos de
cárie. Nesse sentido, a saúde bucal geralmente era boa se compararmos com a dos ocidentais modernos.

Nas grandes cidades -- principalmente aquelas que funcionavam como
entrepostos comerciais, com a população maior e muito condensada --, o
saneamento era precário e muitas pessoas provavelmente sofreram com
doenças advindas da sujeira e da falta de higiene. A análise dos
sedimentos da cidade comercial de Birka continha ovos de parasitas
humanos. Os parasitas maduros teriam causado náuseas, diarreia e outras
doenças entre os moradores da cidade.

A boa saúde foi muitas vezes vista como uma extensão da boa sorte de
cada um ou de toda a comunidade. Portanto, muito dos tratamentos para
prevenir as doenças, bem como para garantir a boa saúde, consistiam em cantos e
encantos mágicos realizados por mulheres que manteriam a boa saúde.

\SIG{Luciana de Campos}

Ver também Cotidiano; Família; Mulheres.

\begin{itemize}
\item \versal{GRAHAM-CAMPBELL}, James. A vida doméstica. In: \emph{Os viquingues}.
Madrid: Del Prado, 1997, pp. 63-66.

\item \versal{CAMPOS}, Luciana de. Cosmética, plantas e saúde na Era Viking.
\emph{Youtube/Canal do \versal{NEVE}}, 2017.
\href{https://www.youtube.com/watch?v=4TDvmqRKjWc}{\emph{youtube.com/watch?v=4TDvmqRKjWc}}

\item \versal{NOUGIER}, Louis-René. Samstags ist badetag. \emph{Wikinger}. Hamburg:
Tesslof Verlag, 1983, pp. 36-37.
\end{itemize}
\section{\versal{HIRD}}

O termo \emph{hirð}, em nórdico antigo, denotava um estrato da sociedade
composto pelos ``nobres'', ``súditos'' e membros da corte dos reis
noruegueses (\emph{household}). Etimologicamente, a palavra \emph{hirð}
é oriunda do anglo-saxão \emph{hîrd}, que poderia indicar uma irmandade,
companhia ou hoste. Para melhor compreendermos tais grupos, recorremos à
\emph{hirðskrå} ou ``livro da \emph{hirð}'', um documento compilado
durante o reinado de Magnus Håkonsson (1217-1263), no qual constavam as
leis, direitos e obrigações da \emph{hirð}. A historiografia aponta
ainda que muitas das leis presentes na \emph{hirðskrå} podem remontar ao
reinado de Sverri (1184"-1202), tornando assim a \emph{hirðskrå} um
documento utilizado na compreensão não apenas da Noruega do século~\versal{XIII},
mas também de parte do século~\versal{XII} e da \emph{household} norueguesa de
séculos anteriores.

Inicialmente, a \emph{hirð} pode ser separada em três grandes grupos
(\emph{lǫguneyti}): \emph{hirðmen}, \emph{gestir} e \emph{kertisveinar}.
Cada um desses grupos, de acordo com o historiador Steinar Imsen, pode
ser entendido como um tipo de guilda ou corporação. Entre as suas
características, temos a obrigação de toda \emph{hirð} contribuir com um
terço de seu dízimo para um hospital no caso dos membros precisarem de
cuidados. Todavia, uma característica da \emph{hirð} que a afasta da
lógica de uma guilda em direção às suas raízes pré-cristãs pode ser
encontrada na forma da distribuição dos espólios de batalha, que eram
divididos igualmente entre todos os membros da \emph{hirð}, sendo que
nem mesmo o rei tinha privilégios sobre estes. Mesmo a figura do rei não
detinha poder pleno e inconteste sobre a \emph{hirð}, necessitando da
aprovação dos membros da mesma para suas decisões, assemelhando-se assim
a \emph{hirð} a uma irmandade.

A escolha dos membros é feita, sobretudo, a partir de famílias abastadas
e proeminentes na sociedade norueguesa, sem a obrigação da
hereditariedade, ainda que fosse comum certas famílias serem vistas com
frequência entre os \emph{lendmen}. Ademais, ser um membro da
\emph{hirð} incluía, além da posição de destaque social, o direito de
integrar um dos cargos de ofício real. O ingresso na \emph{hirð} dava-se
através do ritual conhecido como \emph{håndgang}, no qual o aspirante
jurava fidelidade ao rei, similar ao rito de homenagem, com a ressalva
de que não há caráter de vassalagem no processo. Entretanto, o ingresso
na irmandade era sujeito à aprovação dos outros membros, amparados
legalmente e com autoridade para negar o acesso do membro aspirante caso
assim desejassem.

Todos os membros da \emph{hirð} tinham alguns direitos comuns,
como a isenção de taxas para si e pelo menos um outro membro de sua
\emph{household}, bem como o julgamento por pares em casos de traição. A
resolução de disputas entre os \emph{hirðmen} eram feitas pelo rei ou na
assembleia da \emph{hirð} (\emph{hirðstefna}).

Os \emph{hirðmen} constituíam uma corporação diversa, organizada
hierarquicamente conforme o cargo e os direitos que possuíssem. Entre os
grupos que faziam parte dessa corporação destacamos aqui: os duques e
\emph{earls} acima de todos os outros; em seguida temos os
\emph{lendmen} -- que se distinguiam por receberem terras pertencentes
ao rei --; os detentores de cargos oficiais, como os \emph{stallar},
\emph{merkesman} e os chanceleres; e os \emph{skutilsvein} -- que
serviam o rei à mesa. Estes grupos compunham a \emph{hirðstiorar}, isto
é, os líderes da \emph{hirð}. Abaixo destes encontram-se os
\emph{hirðmen} comuns, sobre os quais há poucas informações disponíveis.

Dentre os \emph{hirðmen}, apenas os \emph{lendmen} recebiam uma pensão
régia (\emph{veyzla}) de 15 marcos ao ano exclusivamente em função de
sua posição na sociedade, pois na maioria dos casos eram membros de família abastadas e
de grande influência. Ademais, no final do século~\versal{XIII}, era direito
exclusivo dos \emph{lendmen} o de reter um corpo de homens cuja função
era a de protegê-los e proteger o rei, chamados de \emph{housecarls}.
Acrescentamos ainda que aqueles que ocupavam cargos de ofício na
\emph{hirð} também recebiam essa mesma quantia, todavia, por razão do
exercício de seus cargos.

Os \emph{gestir} (convidados) atuavam como uma força de policiamento,
espionagem, assassinato, bem como protetores do rei, serviço comparável
aos \emph{housecarls} de séculos anteriores, além de emissários régios.
Já os \emph{kertisveinar} serviam no salão de banquetes, à mesa do rei e
suas responsabilidades eram similares àquelas de pajens.

O século~\versal{XIV} vê profundas transformações no modelo da \emph{hirð} que
vimos até agora, levando a um subsequente desaparecimento da \emph{hirð}
na Noruega medieval. Ressaltamos aqui que é no século~\versal{XIV} que as
divisões dos \emph{hirðmen} dão lugar apenas à distinção entre
cavaleiros e escudeiros, enquanto que os \emph{lendmen}, por decreto do
rei Håkon~\versal{V} (1299"-1319) no ano de 1308, deixam de ser nomeados,
resultando no ocaso dessa corporação, seguidos dos \emph{gestir} e dos
\emph{kertisveinar} nas décadas seguintes. Ademais, mudanças na
estrutura de governo em função de acordos firmados com outros reinos
escandinavos acabaram por contribuir para que a \emph{hirð} fosse
desestruturada e, portanto, acabasse se transformando em um pequeno grupo
de nobres por conta de sua genealogia e não por seus serviços. A exceção
a isso são províncias norueguesas ao norte da Escócia, que retêm a
presença de homens da \emph{hirð} até o século~\versal{XVI}.

\SIG{Hiram Alem}

Ver também Era Viking; Guerra e técnicas de combate; Húskarl; Realeza;
Sociedade.


\begin{itemize}
\item \versal{BERGE}, Lawrence Gerhard
(trad.). \emph{Hirðskrá 1-37: a translation with notes}. Dissertação (Mestrado). Wisconsin: University of Wisconsin,
1968.

\item \emph{Hirðskrá}. In: \versal{KEYSER}, Rudolph; \versal{MUNCH}, Peter Andreas (eds.). \emph{Norges
gamle Love indtil 1387}. vol. 2, Oslo: Grondähl, 1848, pp. 387-450.

\item \versal{IMSEN}, Steinar. King Magnus and his Liegemen's `Hirdskrå': A Portrait of
the Norwegian Nobility in the 1270s. In: \versal{DUGGAN}, Anne J. (ed.).
\emph{Nobles and Nobility in Medieval Europe: Concepts, Origins,
Transformations}. Woodbridge: Boydell Press, 2000, pp. 205-220.

\item \versal{IMSEN}, Steinar. Earldom and Kingdom: Orkney in the Realm of Norway
1195-1379. In: \versal{WAUGH}, Doreen J. (ed.). \emph{The Faces of Orkney:
Saints, Skalds and Stones}. The Scottish Society for Northern Studies,
Edinburgh, 2003, pp. 65-80.

\item \versal{LARSON}, Laurence M. The Household of the Norwegian Kings in the
Thirteenth Century. \emph{The American Historical Review}, vol. 13, n.
3, 1908, pp. 459-479.
\end{itemize}
\section{\versal{HISTÓRIA DA GUERRA}}

Ver Armamento; Arquearia; Batalha de Bravalla; Batalha de Brunanburh;
Batalha de Clontarf; Batalha de Dyle; Batalha de Edington; Batalha de
Hafisfjord; Batalha de Maldon; Batalha de Stanford Bridge; Batalha de
Stiklestad; Espada; Fortificações; Grande armada danesa; Guerra e
religião; Guerra e técnicas de combate; Guerra e simbolismos; Guerreiras
nórdicas.

\section{\versal{HISTORIA DE ANTIQUITATE REGUM NORWAGIENSIUM}}

Trata-se da história do Reino da Noruega escrita em latim e narrada
desde Haroldo Cabelos Belos, do século~\versal{IX}, até Sigurd, o Cruzado, morto em
1130. Apresenta uma tradição oral sobre as sucessões dos reis
noruegueses e muito provavelmente consiste na primeira manifestação
escrita sobre esses reis, sendo considerada a mais antiga
biografia real norueguesa. Sua data de composição pode ser indicada
aproximadamente entre os anos 1177 e 1187.

Foi composta por Theodoricus Monachus, do qual muito pouco se sabe. Provavelmente pertenceu a uma comunidade beneditina, embora alguns
autores apresentem versões diferentes. Presumivelmente era norueguês, pois
afirma, no conteúdo da obra, que faz parte do mundo ali representado. Pode ser vinculado ao contexto da fundação da diocese de Nidaros em
1152, já que a obra foi dedicada ao arcebispo local, Eysteinn
Erlendsson, que esteve presente no arcebispado entre 1161-1188.

As características da
\emph{Historia de antiquitate
regum norwagiensium} são interessantes. Apenas metade de sua breve narrativa se ocupa com os anos da monarquia. Os dois personagens mais destacados, como em outras
crônicas, são Olavo Tryggvason e Olavo Haraldsson, ambos conhecidos por
promover a difusão do cristianismo nas terras norueguesas. Já no
prólogo, Theodoricus afirma que muitas das informações que utilizou
foram encontradas com islandeses que preservaram o seu conhecimento
histórico através de uma tradição de antigos poemas escáldicos.
Observa-se também uma tentativa, por parte do autor, em estabelecer
algumas datas durante a narrativa, que também comporta algumas digressões, como, por exemplo, a descoberta da Islândia (capítulo 3), o elogio ao
arcebispo de Nidaros, Eysteinn (capítulo 32), e a recordação de um tempo
violento que surgiu depois do reinado de Sigurd (capítulo 34).

O códice que apresenta a narrativa da \emph{Historia de antiquitate} foi
descoberto durante a década de 1620 por Jakob Kirchmann, em uma
biblioteca da cidade de Lübeck. Não tem uma data de composição
estabelecida. Posteriormente à descoberta, foram feitas algumas
transcrições durante o século~\versal{XVII} e uma \emph{editio princeps} em 1684
pelo próprio Jakob Kirchmann.

A narrativa é composta por um prólogo e 34 capítulos. Se inicia com a unificação da Noruega, datada por Theodoricus
do ano 852. Entre os capítulos 1-6, o autor da \emph{Historia de
antiquitate} aborda de forma breve os reis antes da presença do
cristianismo, desde Haroldo até Hakon \emph{Jarl}, ou seja,
aproximadamente entre os anos 865 e 995. Os capítulos 7-14 têm como
personagem central Olavo Tryggvason. Os reinados de Érico e Sueno
são mencionados brevemente, o que indica que se referem aproximadamente aos anos
995-1015. Sobre Tryggvason, este se destaca pela sua preocupação com
os aspectos do cristianismo. Olavo Haraldsson, importante rei norueguês
na conversão da Noruega para a cristandade, surge na narrativa entre os
capítulos 15 a 20, os quais se referem aos anos 1016-1028. Os capítulos
21 a 28 são dedicados aos reinos de Magnus Olafsson e Haraldo Hardruler.
Olavo Kyrre é o personagem central do capítulo 29 e seu filho, Magnus
Bareleg, ocupa o capítulo 31. A peregrinação para Jerusalém de seu outro
filho, Sigurd, é narrada no capítulo 33.

\SIG{Luciano José Vianna}

Ver também Fontes primárias; Historiografia e pseudo-história;
Literatura; Noruega da Era Viking.

\begin{itemize}
\item \versal{HOLMAN}, Katherine. \emph{Historical Dictionary of the Vikings}. Lanham,
Maryland, and Oxford: The Scarecrow Press, Inc. 2003, pp. 134-135.

\item \versal{JAKOBSSON}, Ármann. Royal Biography. In: \versal{McTURK}, Rory (ed.). \emph{A
Companion to Old Norse-Icelandic Literature and Culture}.
Malden/Oxford/Victoria: Blackwell Publishing Ltd, 2005, pp. 388-402.

\item \versal{LÖNNROTH}, Lars; \versal{OLÁSON}, Vésteinn; \versal{PILTZ} Anders. Literature. In: \versal{HELLE},
Knut (org.). \emph{The Cambridge History of Scandinavia,} vol. 1.
Prehistory to 1520. Cambridge: Cambridge University Press, 2008, pp.
487-520.

\item \versal{MORTENSEN}, Lars Boje. Theodoricus Monachus. In: \emph{Medieval Nordic
Literature in Latin. A Website of Authors and Anonymous Works} (c.
1100-1530). Disponível em:
\href{wiki.uib.no/medieval/index.php/Theodoricus_Monachus}{\emph{wiki.uib.no/medieval/index.php/Theodoricus\_Monachus}}.
Acesso em 18/06/2017.

\item \versal{THEODORICUS MONACHUS}. \emph{Historia de Antiquitate Regum Norwagiensium
(An account of the Ancient History of the Norwegian Kings)}. 
London: University College London, 1998, (Viking Society for Northern Research Text Series, 
vol. \versal{IX}).

\item \versal{WÜRTH}, Stefanie. Historiography and Pseudo-History. In: \versal{McTURK}, Rory
(ed.). \emph{A Companion to Old Norse-Icelandic Literature and Culture}.
Malden/Oxford/Victoria: Blackwell Publishing Ltd, 2005, pp. 155-173.
\end{itemize}
\section{\versal{HISTORIA NORWEGIAE}}

Trata-se de um texto escrito em latim, anônimo e que até hoje se encontra
incompleto. Composto aproximadamente entre 1150/1160 e 1175, apresenta
uma descrição geográfica do Reino da Noruega no contexto do século~\versal{XII}. É considerado um dos mais antigos documentos sobre a história
da Noruega. Além disso, narra a sequência e os feitos dos governantes
noruegueses, desde a lendária família dos reis Ynglingos até o santo
Olavo Haraldsson, com destaque para os reis Olavo Tryggvason (rei
entre os anos 995 e 1000) e Olavo Haraldsson (rei da Noruega entre 1016
e 1028), os quais são retratados na maior parte do documento (partes~{\versal{XVII} e \versal{XVIII}} do livro primeiro). Tal destaque vincula-se aos
comentários sobre a cristianização do território, com Tryggvason
convertendo a maior parte do território norueguês e Haraldsson dando
continuidade ao processo de conversão. Por conseguinte, a cristianização do
território representa um aspecto central na narrativa, principalmente a
região norte da Noruega, que, naquele contexto, apresentava algumas populações
locais que não haviam sido cristianizadas.

Provavelmente essa obra foi concebida nos círculos do governo episcopal, real, ou até em ambos. Embora seja de autor anônimo, é provável que este tenha pertencido a um destacado âmbito episcopal ou real e tenha
tido uma formação fora do Reino da Noruega, como na Saxônia ou na
Dinamarca. Uma parte do texto foi preservada, consubstanciada no prólogo
e em uma continuação que seria o primeiro livro (dividido em 18 partes).
Mas, possivelmente, o documento continha outros três ou quatro livros --
de acordo com a hipótese de Lars Boje Mortensen --, sendo considerada a
mais longa crônica latina composta na época no Reino da Noruega. Nas
palavras de Mortensen, uma ``empresa ambiciosa''.

A importância historiográfica da \emph{Historia norwegiae} está
fundamentada, ainda de acordo com Lars Boje Mortensen, à medida que é considerada a
primeira fonte para o conhecimento dos inícios da historiografia
norueguesa, além de fornecer um panorama valioso sobre o nascimento da
cultura literária na Noruega. O escrito foi conservado em três manuscritos
bem posteriores ao texto original (sendo, desse modo, cópias de outras cópias, de
acordo com \emph{stemma codicum} do texto): \emph{Edinburgh}, \emph{National Archives
of Scotland}, \emph{Dalhousie Muniments}, sob o registro~{\versal{GD} 45/31/\versal{I-II}}, copiado
na Escócia por volta de 1500; \emph{Stockholm}, \emph{Kungliga Biblioteket}, \versal{B} 17-\versal{II},
o qual foi copiado na Suécia, apresentando em seu conteúdo
principalmente leis suecas da primeira metade do século~\versal{XV}; e o
\emph{Stockholm}, \emph{Riksarkivet}, \versal{A} 8, cuja principal parte foi escrita na Suécia,
por volta de 1344, sendo que outras informações foram adicionadas
durante o século~\versal{XV}. Todos os códices apresentam outros textos. A
existência desse texto em diversos manuscritos indica a sua provável
importância institucional. Embora tenha sobrevivido somente uma parte do
que seria a introdução do texto, alguns autores trabalham com a ideia de
que ele expressa um sentimento de localidade vinculado aos aspectos
de centralização política ocorrida na Noruega durante o século~\versal{XII}.

O texto apresenta similaridades com outro escrito composto na época, a
\emph{Historia de antiquitate regum norwagensium}, composta por
Theodoricus Monachus e dedicada ao arcebispo Eysteinn Erlendsson de
Nidaros, embora não existam provas de que os dois autores dessas duas
obras mantiveram contato entre si. Na narrativa, são encontradas
referências a textos clássicos e medievais, além de fontes nórdicas.
Ademais, é provável que o autor anônimo tenha tido contato com outros
textos muito similares ao conteúdo apresentado pela \emph{Historia
norwegiae}, bem como tenha escrito o texto original em território norueguês.
Por exemplo, uma das fontes com as quais a \emph{Historia norwegiae} se
aproxima textualmente é a \emph{Gesta hammaburgensis ecclesiae
pontificum}, de Adão de Bremen, escrita em Hamburgo entre 1072 e 1075. A
semelhança textual para com esta fonte está precisamente voltada para a
introdução geográfica feita na \emph{Historia norwegiae}, na qual o autor
anônimo se preocupou em demonstrar o estado atual do cristianismo e do
paganismo no reino norueguês de sua época. Outra fonte que se aproxima
dos conteúdos da obra anônima é a obra de Honório de Autum, \emph{Imago
Mundi}, primeira obra de caráter enciclopédico a superar a obra de
Isidoro de Sevilha, as \emph{Etimologiae}. De acordo com Lars Boje
Mortensen, muitas frases encontradas na obra de Honório são também
encontradas no texto da \emph{Historia norwegiae}. Diversos
manuscritos da obra \emph{Imago Mundi} estavam em circulação no contexto
de composição da \emph{Historia norwegiae}. Sobre o local de composição
da obra, algumas possibilidades foram levantadas, tais como Lund, Bergen
ou Trondheim (antiga Nidaros), mas a hipótese mais indicada, devido a
diversas particularidades textuais, é a região leste da Noruega, na
época a província de Viken.

De acordo com a edição de Inger Ekrem e Lars Boje Mortensen -- que conta com a
tradução do texto para o inglês realizada por Peter Fischer (Copenhagen:
Museum Tusculanum Press, 2006) --, a narrativa está dividida em um prólogo
e uma parte do primeiro livro (dividido em 18 partes). A intenção do
autor da narrativa é clara e está exposta logo no prólogo: descrever a
extensão de uma região ampla, recriar a genealogia dos seus governantes
e abordar a presença do cristianismo e do paganismo. A narrativa da
\emph{Historia norwegiae} apresenta diversos assuntos, tais como: (\versal{I}) uma
descrição geográfica geral do reino; (\versal{II}) a região da costa; (\versal{III}) os
aspectos da região montanhosa do território; (\versal{IV}) a fronteira norte onde se encontravam os limites com os territórios pagãos; (\versal{V},
\versal{VI}, \versal{VII} e \versal{VIII}) as
diversas ilhas, dentre as quais algumas pagavam tributo ao rei da Noruega; a linhagem dos reis, as quais, de acordo com a
narrativa, foram originadas da Suécia; (\versal{IX}) as diversas sucessões ocorridas
no início da dinastia dos reis da Noruega até a presença final da
dinastia, tendo como cidade mais importante Uppsala, precisamente no
reinado de Olavo, conhecido como Tretelgje. A parte seguinte (\versal{X}),
apresenta o reinado dos reis já instalados em Vestfold, sendo o primeiro
deles Halvdan Hvitbein e o último Halfdan, o Negro. Em seguida
(\versal{XI}), a narrativa é retomada de forma mais densa,
apresentando diversas informações sobre os personagens, o primeiro dos
quais é Haroldo Cabelos Belos, assim como sua descendência, sendo sucedido
pelo seu filho mais velho, Érico Machado Sangrento, o qual se refugiou na
Inglaterra e morreu posteriormente (\versal{XII}). A narrativa segue com o
reinado de Hakon (\versal{XIII}), o retorno dos filhos de Érico Machado Sangrento, ou
seja, Haroldo, Sigurd e Gunnrod (\versal{XIV}). Na parte seguinte (\versal{XV}),
encontra-se a referência a Óláfr Tryggvason (rei da Noruega entre os
anos 995 e 1000), o qual é chamado de ``eterno rei da Noruega''
(\emph{perpetuum regem Norwegie}), assim como a presença do paganismo
com o rei Hakon (\versal{XVI}). As duas últimas partes são as mais extensas da
\emph{Historia norwegiae}, e são dedicadas a Óláfr Tryggvason (\versal{XVII}) e
Óláfr Haraldsson (\versal{XVIII}).

\SIG{Luciano José Vianna}

Ver também Fontes primárias; Historiografia e pseudo-história;
Literatura; Noruega da Era Viking.

\begin{itemize}
\item \versal{HAYWOOD}, John. \emph{Encyclopaedia of the Viking Age}. London: Thames \&
Hudson, 2000, p. 97.

\item \versal{EKREM}, Inger \& \versal{MORTENSEN}, Lars Boje (eds.). \emph{Historia Norwegie}.
Trad. Peter Fisher. Copenhagen: Museum Tusculanum Press, 2006,
pp. 49-105.

\item \versal{HOLMAN}, Katherine. \emph{Historical Dictionary of the Vikings}. Lanham,
Maryland, and Oxford: The Scarecrow Press, Inc. 2003, p. 135.

\item \versal{MORTENSEN}, Lars Boje. \emph{Historia Norwegie}. Disponível em:
\href{wiki.uib.no/medieval/index.php/Historia_Norwegie}{\emph{wiki.uib.no/medieval/index.php/Historia\_Norwegie}}.
Acesso em 19 de maio de 2017.

\item \versal{WÜRTH}, Stefanie. Historiography and Pseudo-History. In: \versal{McTURK}, Rory
(ed.). \emph{A Companion to Old Norse-Icelandic Literature and Culture}.
Malden/Oxford/Victoria: Blackwell Publishing Ltd, 2005, pp. 155-173.
\end{itemize}
\section{\versal{HISTORIOGRAFIA E PSEUDO-HISTÓRIA}}

Durante o medievo, a ação de recuperar o passado e utilizá-lo como um
\emph{locus} no qual seus aspectos seriam modificados para legitimar uma
situação política no presente foi constante. O passado e o presente
constantemente faziam parte do mesmo objeto textual, onde muitas vezes o
tempo pretérito era recuperado para servir de exemplo no presente,
principalmente com objetivos de legitimação política. Dessa forma, um
objeto formulado no passado poderia servir, posteriormente, para diferentes objetivos. O passado poderia ser recuperado ou até
modificado em seus diversos aspectos, tais como o textual, o
paleográfico, o codicológico ou o visual.

A produção de um objeto historiográfico dependia das circunstâncias
políticas e culturais de sua produção. Nesse sentido, alguns aspectos
gerais relacionados à historiografia medieval podem ser encontrados em
diferentes documentos, compostos em distintos contextos, os quais
comentamos a continuação. 1. \emph{Os conceitos de actor e auctor}.
Durante o medievo, os conceitos de \emph{actor} e \emph{auctor}
apresentavam definições distintas, significando, respectivamente, aquele
que produz um livro (ou o responsável pela sua produção) e aquele que tem a
autoridade (\emph{auctoritas}). 2. \emph{A pluralidade do conceito de
autoria}. A função do historiador no medievo era descrita de uma forma
muito mais ambígua e plural. Sua ação estava definida nos seguintes
verbos: \emph{compilare} (compilar), \emph{colligere} (reunir),
\emph{excerpere} (escolher), \emph{breviare} (sintetizar) e
\emph{redigere} (redigir). Considerando o aspecto único de cada produção
historiográfica no medievo, tal pluralidade do conceito de autoria
ajuda a compreender a razão pela qual cada objeto era considerado como uma
composição individual a partir do seu contexto de composição. 3. \emph{A
concepção linear da história}. Outra característica da historiografia
medieval é o aspecto de linearidade. Uma narrativa em forma linear, com
foco em um passado distante, longínquo e com um término voltado para um
presente, apresentava uma mentalidade contínua que estava de acordo com
os documentos que retratavam a formação de, por exemplo, uma linhagem ou
dinastia. 4. \emph{A lógica social do texto histórico}. De acordo com
Gabrielle M. Spiegel, todo estudo historiográfico que apresente a
perspectiva ``texto-contexto'' deve considerar a relação do texto com o
momento de composição, no qual o mundo histórico foi internalizado no
texto. De acordo com essa proposta, os textos históricos medievais não
devem ser compreendidos como documentos históricos pouco confiáveis:
eles pertencem a um contexto de composição e a partir deste adquirem um
significado. Nesse sentido, é necessário considerar a interação
``texto-contexto'' para descobrir o motivo da composição de um texto
historiográfico e, consequentemente, a intencionalidade do \emph{actor}
ou do patrocinador do texto. 5. \emph{Relação com o passado e a sua
representação}. O passado dinástico, a partir de uma perspectiva de
sucessão linear, demonstra uma continuidade fundamentada na legitimação,
por exemplo, de uma linhagem. Dessa forma, a fusão entre presente e
passado, reunidos na materialidade do documento, nos faz refletir sobre
a importância do tempo pretérito e sua relação com a contemporaneidade.
6. \emph{Função de legitimação}. Um gênero histórico possui
características próprias que eram conhecidas pelos medievais e por isso
eram selecionados dependendo da situação política a ser resolvida ou
recordada. 7. \emph{História e política}. A relação entre esses dois
âmbitos condensa a participação de todos os outros aspectos citados até
aqui, já que tais âmbitos fazem parte da principal característica da
historiografia medieval. Os gêneros históricos foram utilizados
constantemente em termos políticos, para legitimações políticas, para
fins políticos. O passado era o \emph{locus} através do qual poder-se-ia
transitar e encontrar soluções para situações no presente, seja através
de modificações do passado, seja através ``somente'' da recuperação de
suas informações que serviriam para serem utilizadas
no presente.

Após a composição original de um texto pelo seu \emph{auctor}, sobre o
qual este estabelecia sua \emph{auctoritas}, as posteriores reproduções
dependiam tão somente dos patrocinadores e dos \emph{actores}, os quais
atuavam de acordo com o contexto em que viviam: eliminavam,
acrescentavam e modificavam as informações de acordo com os seus
conhecimentos linguísticos, religiosos, morais, políticos e literários.
Portanto, quando compunham um manuscrito ou preparavam uma nova cópia,
os patrocinadores e \emph{actores} realizavam uma tarefa exaustiva e
introduziam no objeto a ser preparado informações que eram próprias do
seu tempo histórico.

No caso dos textos oriundos do norte da Europa, diferentes versões e
redações indicam que eles puderam ser adaptados pelos
novos \emph{actores} (com suas intenções específicas) ou para as diferentes
necessidades dos públicos finais, ou seja, da audiência. Por exemplo,
diversos autores afirmam que uma das principais obras
islandesas, o \emph{Landnámabók}, foi
manipulada em suas composições posteriores para legitimar reivindicações políticas que emergiram depois de sua
composição. Da mesma forma, ainda sobre as produções historiográficas, a
veracidade de uma história estava relacionada a diversos fatores, tais
como o tema abordado e a relação da escrita com a verdade.

Há uma diversidade de produtos historiográficos específicos oriundos do
território do norte europeu, dentre os quais encontra-se, por exemplo, as
sagas (que podem ser islandesas, lendárias, reais), as eddas (tanto em
prosa como em poesia), os registros escáldicos (antigas tradições,
narrativas heroicas, narrativas históricas, contos, folclore), as
gestas, os poemas (épicos, rúnicos) e a poesia (éddica, escáldica, pagã,
feminina). Tais produtos, oriundos da historiografia do norte da Europa
medieval, apresentam uma considerável diversidade (temática, material
etc.), com diferenças em seus
conteúdos e intencionalidades. Em um âmbito geral, a historiografia
nórdica apresenta algumas características particulares, como, por
exemplo, o foco na história contemporânea. Além disso, difere-se da
historiografia continental justamente por apresentar em seus textos
iniciais a escrita vernacular, desde os primeiros momentos do seu
surgimento. Uma característica geral é que a tradição oral e as
produções locais foram utilizadas como fontes para a composição
historiográfica. Tais tradições orais são classificadas como a
memória coletiva do território. No caso da historiografia islandesa, um
dos principais aspectos que a caracterizam é a ausência de versos épicos
em um sentido clássico e a presença de manifestações vernaculares desde
suas primeiras expressões. Os textos historiográficos desse âmbito
territorial estão entre os mais antigos trabalhos escritos na Islândia.
Além disso, as produções historiográficas islandesas apresentaram como
foco de atuação o passado imediato antes e depois da conversão ao
cristianismo ocorrida no território. Percebe-se nesses produtos uma
preocupação em mencionar autoridades escritas e testemunhas presenciais,
com o objetivo de destacar sua própria história. Nesse sentido, a
perspectiva genealógica foi muito utilizada, principalmente nos
primeiros textos elaborados durante o contexto da colonização,
demonstrando a consciência do vínculo territorial com a Noruega e, por conseguinte, estabelecendo o ato migratório como uma ação
criativa e legitimadora, o que posteriormente serviu para diluir
gradualmente o vínculo norueguês e afirmar um processo de construção de
identidade local. Por outro lado, a historiografia norueguesa apresentou
uma intensa utilização do latim em seus primórdios; porém,
gradativamente a escrita vernacular foi cada vez mais utilizada.

\SIG{Luciano José Vianna}

Ver também Fontes primárias; Islândia da Era Viking; Literatura; Noruega
da Era Viking; Suécia da Era Viking.

\begin{itemize}
\item \versal{AURELL}, Jaume. La historiografía medieval: siglos \versal{IX-XV}. In:
\emph{Comprender el pasado. Una historia de la escritura y el
pensamiento histórico} (Aurell, Jaume; Balmaceda, Catalina; Burke,
Peter; Soza, Felipe). Madrid: Ediciones Akal, 2013, pp. 95-142.

\item \versal{CHENU}, Marie-Dominique. Auctor, actor, autor. \emph{Archivum Latinitatis
Medii Aevi}, vol. 3, 1927, pp. 81-86.

\item \versal{MANRIQUE ANTÓN}, Teodoro. Ficción e historia en los primeros intentos
literarios de las letras islandesas: la representación del pasado.
\emph{Revista de Literatura Medieval}, vol. 24, 2012, pp. 141-153.

\item \versal{SPIEGEL}, Gabrielle M. History, Historicism and the Social Logic of the
Text. \emph{Speculum}, n. 65, vol. 1, 1990, pp. 59-86.

\item \versal{SPIEGEL}, Gabrielle M. \emph{Romancing the Past: the Rise of Vernacular
Prose Historiography in Thirteenth-Century France}. Berkeley/Los
Angeles/London: University of California Press, 1995.

\item \versal{SPIEGEL}, Gabrielle M. Theory into Practice: Reading Medieval Chronicles.
In: \versal{KOOPER}, Erik (ed.). \emph{The Medieval Chronicle}.
Amsterdam/Atlanta: Rodopi, 1999, pp. 01-12, vol. 1.

\item \versal{TEEUWEN}, Mariken. \emph{The Vocabulary of Intellectual Life in the
Middle Ages}. Turnhout: Brepols, 2003, pp. 222-223.

\item  \versal{WÜRTH}, Stefanie. Historiography and Pseudo-History. In: \versal{McTURK}, Rory
(ed.). \emph{A Companion to Old Norse-Icelandic Literature and Culture}.
Malden/Oxford/Victoria: Blackwell Publishing Ltd, 2005, pp. 155-173.
\end{itemize}
\section{\versal{HNEFATAFL}}

Entre os jogos de tabuleiro (\emph{tafl}) conhecidos pelos escandinavos
da Era Viking, provavelmente o que mais se destacou foi o
\emph{hnefatafl} (tabuleiro do rei), cujo nome se deve ao fato de que uma das
peças era chamada de rei. Ainda é jogado atualmente, possuindo, inclusive, versões virtuais para computadores e smartphones. No
entanto, apesar do \emph{hnefatafl} ainda ser conhecido, Philip Hingston
comenta que as maneiras pelas quais o jogamos hoje não são iguais àquelas dos nórdicos.

Não se sabe quando o jogo foi criado ou o local onde surgiu exatamente,
mas é fato que ele se tornou bastante popular na Escandinávia e, devido à expansão
nórdica, o \emph{hnefatafl} foi levado para outras terras como
Inglaterra, Escócia, Irlanda, etc. Em tais lugares o jogo
acabou, com o tempo, desenvolvendo novas regras.

O tabuleiro do \emph{hnefatafl}, como outros jogos de tabuleiro, possui o
formato quadrangular, sendo que suas dimensões variavam de lugar para
lugar e de acordo com o fabricante. Na Escandinávia, predominou as
versões de 11x11 e 13x13. Porém, na Irlanda e Escócia, encontrava-se a
versão 7x7. Na Inglaterra, as versões 9x9, 11x11 e 19x19 eram comuns.
Originalmente os tabuleiros eram feitos de madeira e as peças eram
talhadas em madeira ou em ossos. Não obstante, conhecem-se casos de peças feitas
de âmbar, pedra, metal etc.

O \emph{hnefatafl} era jogado por duas pessoas, sendo que um dos jogadores
era incumbido de proteger o rei, enquanto o outro devia liderar o
ataque para tomar capturá-lo. As peças eram divididas em
defensores (13 peças, incluindo o rei); e os
atacantes (26 peças). Todavia, tal quantidade de peças era correspondente
aos tabuleiros de, ao menos, 121 casas (11x11). Tabuleiros em versões
menores possuíam menos peças.

Posteriormente, as cores das peças foram divididas em brancas
(defensores) e vermelhas ou pretas (atacantes). Na Suécia moderna, se
popularizou o uso dos termos ``suecos'' para se referir às peças brancas
e ``moscovitas'' para se referir às peças vermelhas, em referência às
guerras travadas entre suecos e russos no século~\versal{XVIII}.

Quanto à disposição das peças, o rei se posiciona ao centro, no espaço
chamado de castelo, trono ou cidadela (a expressão varia dependendo do
país). Esse espaço somente pode ser ocupado pela peça do
rei. Sua guarda real, composta por 12 membros, cerca-o ou aparece
disposta em formato de cruz. Por sua vez, as peças vermelhas
dividem-se em quatro grupos de 6, situados em cada um dos lados do
tabuleiro.

O objetivo do jogo varia de acordo com o jogador. No caso do jogador
branco, este deve evitar que seu rei seja capturado, devendo
conduzi-lo a um dos quatro refúgios (\emph{burgs}), os quais consistem
nos espaços situados nas pontas do tabuleiro, as arestas. Por sua vez, o
jogador vermelho deverá impedir que o rei consiga escapar, tentando
capturá-lo. Para isso, o jogador vermelho deve eliminar a guarda real.

A eliminação das peças ocorre através dos flancos. No caso, todas as
peças se movem da mesma forma: apenas em linha reta, na
horizontal e na vertical, sem limite mínimo de espaços. Isso vale
também para o rei. Assim, duas peças de uma mesma cor devem
flanquear uma peça adversária por dois lados para que esta seja eliminada. No
caso do rei, este deve ser cercado pelos quatro lados para ser
eliminado. Enquanto o rei correr risco de ser capturado, o jogador
vermelho deve dizer \emph{raichi}, o equivalente ao xeque do xadrez.
Finalmente, quando o rei é capturado, o jogador vermelho diz
\emph{tuicha} (algo similar ao xeque-mate).

Ainda que, aparentemente, o \emph{hnefatafl} lembre o xadrez, eles são
bem diferentes entre si. O \emph{hnefatafl} se assemelha aos outros jogos de
tabuleiro como o \emph{Três em linha}, o \emph{Raposa e os Gansos}
(\emph{Halatafl}) e o \emph{Nine men's morris}. Estes consistem em jogos
de estratégia que requerem atenção e concentração, tanto quanto o xadrez.
Mesmo assim, eram jogos bem populares em seu tempo, como indicam as
representações iconográficas e menções em sagas como a \emph{Saga de
Hervör}.

\SIG{Leandro Vilar Oliveira}

Ver também Cotidiano; Jogos e esportes; Sociedade.

\begin{itemize}
\item \versal{ASHTON}, John C. `Linnaeus' Game of Tablut and its Relationship to the
Ancient Viking Game Hnefatafl. \emph{The Heroic Age: A Journal of Early
Medieval Northwestern Europe}, n. 13, 2010.

\item \versal{HINGSTON}, Philip. Evolving Players for an Ancient Game: Hnefatafl.
\emph{Proceedings of the 2007 \versal{IEEE} Symposium on Computational
Intelligence and Games,} 2007, pp. 168-174.

\item \versal{LAWRENCE}, David. A pictish origin ofr Hnefatafl. \emph{Board Games
Studies Journal}, n. 8, 2004, pp. 65-79.

\end{itemize}


\section{\versal{HÚSKARL}}

É um dos nomes dados ao guerreiro que serve próximo ao seu líder, muitas
vezes fazendo parte de sua comitiva e atuando como guarda-costas. A própria
construção da palavra \emph{húskarl} (\emph{húskarlar} no plural),
composta por ``karl'', expressão comum que designa o homem livre,
adicionado ao prefixo ``hús-'', casa, doméstico, indica o homem que
dorme debaixo do teto do seu senhor. É possível encontrar na língua
nórdica antiga outros exemplos de guerreiros que atuam como guardas,
como \emph{heimþegi}, ``recebedor da casa''. Os \emph{húskarlar} estão entre
os poucos casos de guerreiros que atuam em regime integral na cultura
escandinava, tendo em vista que a guerra era empregada de modo sazonal,
restando aos reis e caudilhos empregar um pequeno grupo de homens para o
serviço pessoal em sua comitiva. Os reis, os \emph{jarlar} e outras
figuras proeminentes da cultura guerreira escandinava, contavam com
grupos de profissionais, que podemos chamar de \emph{hirð}, e seus
integrantes os \emph{hirðmenn}, onde se encaixam os \emph{húskarlar}: um
grupo de guerreiros bem equipados e que podiam se dedicar com afinco ao
exercício guerreiro. Não estamos certos se os \emph{húskarlar} também
formaram a base de uma nobreza militar, ainda que tenham sido
representados enquanto parte de uma parcela privilegiada da sociedade
escandinava, também distantes de todo um corpo de homens livres, os
\emph{bændr}, que lutavam apenas sazonalmente.

Os \emph{húskarlar} são mencionados em duas pedras rúnicas na
província de Uppland, Suécia. Na~\versal{U} 330 uma mulher de nome Inga ergue o
monumento em pedra e uma ponte em memória de Ragnfastr, seu marido, e
indica Ǫzurr fora seu \emph{húskarl} (\textbf{× asur × uaʀ × huskarl ×
hans ×}). Na~\versal{U} 335, Holmi ergueu o monumento em pedra e também uma ponte
em memória de seu pai, Hæra que fora \emph{húskarl} de Sigrøðr
(\textbf{× uskarl × sifruþaʀ}).

Uma outra pedra rúnica na região de Södermanland, a \versal{S}ö 338, foi erguida em
memória a Þorsteinn pelos seus filhos, irmão, esposa e \emph{húskarlar}:
``Ketill e Bjǫrn ergueram esse monumento em memória de Þorsteinn,
seu pai; Ǫnundr em memória de seu irmão e os \emph{Húskarlar} em memória do
justo (?). Ketiley em memória de seu marido. Esses irmãos foram os
melhores entre os homens em terra e, no exterior, com o bando (\emph{Lið}),
trataram bem seus \emph{Húskarlar}. Ele morreu em batalha no Leste em Garðar,
líder do bando (Lið), o melhor dos conterrâneos (Landmanna)'' (\textbf{*
ketil : auk + biorn + þaiʀ + raistu + stain + þin[a] + at +
þourstain : faþur + sin + anuntr + at + bruþur + sin + auk :
hu[skar]laʀ + hifiʀ + iafna + ketilau at + buanta sin * bruþr uaʀu
þaʀ bistra mana : a : lanti auk : i liþi : uti : h(i)(l)(t)u sini
huska(r)la : ui- + han + fial + i + urustu + austr + i + garþum + lis +
furugi + lanmana + bestr}).

Na Runestone \versal{S}ö 238, ambos os irmãos são homenageados como líderes de
seus bandos (\emph{lið}), porém, quando exaltam seus feitos em terra (no
sentido de lar) ou no exterior, não fica claro se os seus homens
permaneceram guardando as terras ou se partiram em expedições, se as
inscrições indicam que foram pagos satisfatoriamente ou, ainda, se
receberam proteção de seus senhores. Þorsteinn e Ǫnundr eram chefes
caudilhos e provavelmente desfrutaram de posições sociais e riquezas
elevadas, porém, nada indica, nesse ou nos outros exemplos retirados das
pedras rúnicas, que eles eram aristocratas em algum sentido.
Pesquisadores, como Judith Jesch, acreditam que os \emph{húskarlar}
poderiam estar socialmente muito próximos de seus empregadores.

Além da documentação legal do Ducado da Normandia, na qual o termo aparece
brevemente como sobrenome de um Roger Huscaille, em 1263, a palavra \emph{húskarl} também
aparece com maior vigor na documentação inglesa antiga. No contexto da
campanha do rei Canuto, na \emph{Crônica Anglo-Saxã}, a rainha Emma
permanece em Winchester com os \emph{húskarlar} de seu filho.
Hardacanuto (filho de Canuto), por sua vez, destrói Worcestershire para
vingar seus dois \emph{húskarlar}, mortos durante a coleta de taxas em
Worcester. O bispo Cristiano e o \emph{earl} Osbearn foram a Ely
acompanhados por \emph{Densca Huscarles}, homens que podem ter servido
ao rei Sueno Úlfsson. Dessa maneira, podemos deduzir que as atividades
dos \emph{húskarlar} abrangiam, além da guarda e a guerra, o acompanhamento
em missões diplomáticas e a coleta de taxas. Canuto, Hardacanuto, Eduardo, o
Confessor e Haroldo Godwinson aparecem como exemplos no uso desses
homens em contextos não-militares, Bovi e Urk (homens de Hardacanuto que
também aparecem no \emph{Domesday Book} enquanto \emph{Ministri},) são
ambos testemunhas e receptores de terras.

\SIG{Pablo Gomes de Miranda}

Ver também Cultura material; Hird; Realeza; Sociedade; Viking.

\begin{itemize}
\item \versal{JESCH}, Judith. Skaldic Verse and Viking Semantics. In: \versal{FAULKES}, Anthony;
 \versal{PERKINS}, Richard (orgs.). \emph{Viking Revaluations}. Birmighan: Viking
Society for Northern Research, 1993, pp. 160-179.

\item \versal{JESCH}, Judith. \emph{Ships and Men in the Late Viking Age: the
vocabulary of runic inscriptions and skaldic verse}. Woodbridge: The
Boydell Press, 2001.

\item \versal{RENAUD}, Jean. Les Vikings et la Normandie. In: \versal{BATTAIL}, Jean-François;
 \versal{BATTAIL}, Marianne. \emph{Une Amitié Millénaire. Les relations entre la
France et la Suède à traver les âges}. Paris: Beauchesne, 1993, pp.
49-68.

\item \versal{SIGURĐSSON}, Jón Viðar. Kings, Earls and Chieftains. Rulers in Norway,
Orknet and Iceland c. 900-1300. In: \versal{STREINSLAND}, Gro; \versal{SIGURĐSSON}, Jón
Viðar; \versal{REKDAL}, Janerik. \emph{Ideology and Power in The Viking and
Middle Ages: Scandinavia, Iceland, Ireland, Orkney and The Faroes}.
Leiden: Brill, 2011, pp. 69-108.
\end{itemize}

\chapter{I \textarn{i}}
\section{\versal{IBN FADLAN}}

Foi enviada de Bagdá, em 921, uma missão do califa abássida al-Muqtadir
rumo à corte do rei dos povos búlgaros do Volga, Almis Iltäbär. No
século~\versal{X}, o reino da Bulgária do Volga havia enviado um pedido de apoio
financeiro, militar, religioso e intelectual ao califado abássida. Este,
em resposta, enviou a missão supracitada. Em 21 de junho de 921, a missão
diplomática liderada por Susan al-Rassi, um eunuco na corte do califa,
deixou Bagdá. Seu principal propósito era explicar a lei
islâmica aos povos búlgaros, recentemente convertidos, que viviam na
margem oriental do rio Volga, na atual Rússia, além de dar auxilio aos
búlgaros em uma resposta militar contra seus inimigos, os Khazars.

A missão diplomática se utilizou das estabelecidas rotas de caravanas em
direção a Bukhara, agora parte do Uzbequistão. Mas, em vez de seguir tal
rota até o leste, ela virou para o norte na região que hoje corresponde ao nordeste do
Irã. Saindo da cidade de Gurgan, perto do mar Cáspio, atravessaram
terras pertencentes a uma variedade de povos turcos, dentre os quais se destacavam os
khazar khaganate e os oghuz turks, na costa leste do mar Cáspio, os
pechenegs, no rio Ural, e os bashkirs, na região que agora corresponde à Rússia central.
Também os rus, na rota comercial do Volga.

Dentre os componentes dessa missão, encontrava-se Ibn Fadlan, responsável pela leitura da carta do califa ao rei dos búlgaros e pela
supervisão do trabalho dos faqih, homens peritos nas leis islâmicas.
Ibn Fadlan se tornaria responsável pela escrita do documento denominado
\emph{Risalya}, termo que, traduzido do árabe, significa relato. Trata-se de uma das
principais fontes textuais para o estudo da história, da etnogênese e da
formação política de muitas das tribos e povos que povoaram o leste
euro-asiático.

O documento original escrito por Fadlan foi perdido. Por muito tempo
encontrava-se preservada apenas uma versão transmitida pelo \emph{Léxico
geográfico}, escrita por Yäqütibn-Abdallah, na qual o autor declara
incluir trechos literais do \emph{Risala} (que, no período, já contava com
300 anos de produção) nos verbetes sobre: o rio Volga, denominado Atil;
os bashgird, povos indígenas da região de Bashkortostan, na atual
Turquia; os búlgaros; os khazars, povos seminômades da região correspondente à atual Turquia; a
região de Khwarezm; e os rus, povos do leste europeu e da região do
Volga. Dessa forma, para acessar os escritos de Fadlan, teríamos de nos basear em uma obra sobre a qual só se pode presumir ter sido escrita seguindo o manuscrito original, ou uma cópia deste.

Apenas em 1923, um manuscrito com 420 paginas foi
apontado como contendo uma versão considerada completa da obra de Ibn
Fadlan. Trata-se de manuscrito numerado como \versal{MS} 5229, datado do século~\versal{XIII} d.C. e descoberto por Zeki Validi
Togan na Biblioteca Central de Astan Quds Razavi, na cidade de Mashad, região do
atual Irã. Contudo, passagens adicionais não preservadas no \versal{MS} 5229 ainda
são apontadas para a obra do geógrafo persa Amin Razi denominada
\emph{Haft Iqlim}, Sete Climas.

O relato de Ibn Fadlan é dedicado, em grande parte, à descrição de
um povo que ele chamou de rus. Estes são identificados como vikings, pelos
historiadores que ganharam a alcunha historiográfica de normanistas, ou
como povos do leste eslavo, pelos historiadores que receberam a alcunha
de anti-normanistas. Algumas interpretações apontariam, ainda, para a
utilização do termo rus, pelos árabes, como uma definição de guerreiros e
mercantes que não levava necessariamente em consideração suas etnias. Os
rus, no relato de Fadlan, aparecem como comerciantes que se instalaram nas
margens do rio Volga, perto do acampamento de Bolğar. Eles são descritos
como sendo altos como palmeiras, com cabelos loiros, peles coradas e
tatuados. É dito pelo árabe que todos os homens se encontravam
constantemente armados com um machado, espada e faca longa.

A maior reverberação do relato de Ibn Fadlan, mesmo no campo acadêmico,
deveu-se ao trecho que se refere a dez dias de um ritual
funerário que envolveu festas, bebidas, músicas e sexo. Tudo isso com um
alto teor ritualístico, dedicado a um chefe local rus que havia
morrido. O ritual funerário teria como seu último momento a prática do
depósito e enterramento, mas se desenrolava por tão
longo tempo que tornava necessário, como primeiro ato, a construção de uma
edificação temporária para o depósito do morto. O depósito temporário do
morto, acompanhado por comidas, bebidas e instrumentos musicais, é
considerado um reflexo da crença na necessidade de um entretenimento para o morto durante esses dias.

Durante esses dez dias, nos diz Ibn Fadlan, eram consumidos dois terços
dos bens do morto. Entretanto, de acordo com Neil Price, provavelmente seriam dois
terços dos bens portáveis do morto e não de todas suas posses, uma
vez que os participantes do ritual eram homens em trânsito. Dos dois terços consumidos, um era destinado à fabricação de roupas extremamente adornadas
(que seriam depositadas com o chefe local) e o outro terço para a
fabricação de bebidas alcoólicas a serem consumidas durante os dez dias,
o que indica a existência de uma prática de intoxicação geral, que levaria todos a um estado de
sagrado frenesi durante o ritual.

Também foi apresentado por Ibn Fadlan o ritual conectado com a morte de uma escrava, com cerca de 14 ou 15 anos, que seria,
igualmente, um componente preciso na dramatização funerária com papéis
específicos a serem exercidos. Não sabemos se a escolha da escrava foi
forçada, mesmo considerando o relato de Ibn Fadlan, que nos diz que ela se voluntariou por acreditar
que acompanharia seu dono ao além-vida. O papel dessa escrava nesse ritual é marcado pela
sexualidade, pois, uma vez selecionada para acompanhar o chefe local,
passa a ser considerada sua esposa e a receber um tratamento
diferenciado durante todos os dez dias, mantendo, inclusive, relações com os homens
mais importantes do grupo no ato final antes de sua morte. A morte
dessa escrava seria realizada por uma mulher que Ibn Fadlan chamou de
Malak Al-Maut, nome que no islamismo corresponde ao anjo que separa a alma
do corpo do morto e é responsável por recolhê-lo em seu tempo
predestinado. Price, como um normanista, considera o nome uma
tradução para o termo ``escolhedor dos mortos'', mais conhecido como
Valkyrja.

O rito funerário descrito por Ibn Fadlan fornece-nos, dessa maneira,
muitos elementos que podem ser trabalhados pelos estudiosos do mundo
escandinavo. Não obstante, devemos sempre recordar que a multiplicidade de variações da
antiga religião nórdica não pode ser reduzida a um único relato ou a poucos vestígios arqueológicos, apenas. Sem dúvida alguma, o
relato acima apresentado enriquece nossa compreensão desse mundo nórdico e Ibn
Fadlan pode ser considerado uma das grandes fontes para a exploração da
multiplicidade dos povos nórdicos. Seu relato não deve ser relegado ao
esquecimento, mas ser comparado com as demais fontes que, se não forma apresentadas
em nosso verbete por questão de espaço e lógica de trabalho, devem constar nas análises mais aprofundadas pretendidas pelos
pesquisadores da área e pelos interessados em conhecer o mundo
escandinavo de maneira mais complexa.

\SIG{Munir Lutfe Ayoub}

Ver também Árabes e nórdicos; Funerais e enterros; Rus; Vikings.

\begin{itemize}
\item \versal{HRAUNDAL}, Thorir Jonsson. New Perspectives on Eastern Vikings/Rus in
Arabic Sources.~\emph{Viking and Medieval Scandinavia}, vol. 10, 2014,
pp. 65-97.

\item \versal{MIRANDA}, Pablo Gomes. Nenhum homem pode entender bem os rajputs de
outrora.~\emph{Medievalis}, vol. 1, n. 2, 2015, pp. 83-97.

\item \versal{MONTGOMERY}, James E. Ibn Fadlan and the Rusiyyah.~\emph{Journal of
Arabic and Islamic Studies}, vol. 3, 2000, pp. 01-25.

\item \versal{PRICE}, Neil. Mythic Acts: Material Narratives of the Dead in Viking Age
Scandinavia. In: \versal{RAUDVERE}, Catharina; \versal{SCHJODT}, Jens Peter (eds.).
\emph{More Than} \emph{Mythology. Narratives, Ritual Practices and
Regional Distribution in Pre-Christian Scandinavian Religions}. Lund:
Nordic Academic Press, 2012, pp. 13-46.
\end{itemize}
\section{\versal{IDADE DO FERRO GERMÂNICA}}

Idade do Ferro Germânica é o nome dado ao período que, na história do norte da Europa, abrange o intervalo entre os
séculos~{\versal{V} d.C. e \versal{VIII}} d.C. Antecede a
tal período a Idade do Ferro Romana, que compreende o intervalos entre os séculos~\versal{I}
d.C. e \versal{V} d.C., assim como se segue da Idade de Ferro Germânica o Período Viking, que
vai do século~\versal{VIII} d.C. ao século~\versal{XI} d.C. O período da Idade do Ferro Germânica é
ainda subdividido em duas partes, sendo a primeira denominada
período das migrações (que vai do século~\versal{V} d.C. ao século~\versal{VI} d.C.) e a
segunda (que vai dos séculos~{\versal{VI} d.C. ao \versal{VIII}} d.C.) denominada de
três formas diferentes, à depender do referencial geográfico. Constatam-se as denominações: período Vendel, na Suécia; período Merovíngio, na Noruega; e
Idade do Ferro Germânica Tardia, na Dinamarca.

As fontes literárias para o estudo da Escandinávia desse período são
compilações posteriores como, por exemplo, as \emph{Sagas}, escritas nos
séculos~{\versal{XII} e \versal{XIII}} d.C., frutos de relatos transmitidos
de forma oral durante um grande período. Tal oralidade gerou
modificações e supressões nas versões originais e se tornou um grande problema
para a utilização dessas fontes para a compreensão desse período. A
arqueologia aparece, dessa forma, como alternativa para o estudo dos
processos históricos ocorridos nessa época.

Os traços arqueológicos característicos desse período podem ser
observados pelas mudanças que, a partir do século~\versal{V} d.C., passariam a
ocorrer nos depósitos ritualísticos, compostos por inúmeros objetos
feitos de ouro, como anéis, braceletes, colares e bracteates, além de
pedaços não trabalhados do metal, como tiras e barras. No entanto,
quando tratamos de bronze e prata, o quadro se modifica: os únicos
objetos produzidos por esses metais e que se encontram associados com
os depósitos de ouro são os broches. A maior presença do ouro nos
depósitos demonstra uma primeira padronização dos cultos desse período,
nos quais metais como a prata e o bronze passariam a ter um caráter
secundário. Arqueólogos, como Dagfinn Skre, apontam a presença da prata
como metal de comércio a partir do século~\versal{VI} d.C. e a inexistência de
qualquer objeto de ouro como metal de troca ou de medida comercial para
o mesmo período.

Entretanto, desde o século~\versal{VI} d.C., os resquícios materiais passam a nos
indicar novas modificações que vão além da mudança tipológica dos
objetos depositados, apontando um maior investimento em materiais como o
ouro para depósitos praticados no momento de fundação de certas
edificações. Edificações que, pelas \emph{Sagas}, foram delimitadas como os
salões onde ocorriam festividades como casamentos, banquetes e
comemorações de vitórias em guerra.

A Idade do Ferro Germânica seria caracterizada, dessa maneira, como um período
de mudanças políticas e religiosas, bem como de formação de chefias
locais e pequenos reinados, processo que levaria à concentração de
determinados ritos -- como o de beber em honra aos deuses e aos
antepassados -- à mão desses aristocratas. Homens que pela execução dos
ritos promoviam suas alianças políticas (por meio da troca de
presentes) e legitimavam seus poderes, identificando-se como filhos
dos deuses e rememorando constantemente suas linhagens.

Podemos concluir, dessa forma, que a Idade do Ferro Germânica se caracteriza pela concentração de rituais em determinadas edificações
controladas por determinados personagens sociais. Assim, pelos depósitos
ritualísticos, podemos concluir que, desde o século~\versal{VI} d.C. e por todo o
Período Viking, existiria na Escandinávia uma conexão entre a antiga
religião nórdica e uma aristocracia. As edificações ritualísticas
surgiriam, portanto, como delimitação de um espaço sagrado que
conectaria os homens escandinavos com os deuses nórdicos pelo intermédio
de determinados homens que controlavam esses novos espaços sagrados.

\SIG{Munir Lutfe Ayoub}

Ver também Arqueologia da Era Viking; Era Viking; Escandinávia; Viking.

\begin{itemize}
\item \versal{HEDEAGER}, Lotte. \emph{Iron-Age Societies}. Cambridge: Three Cambridge
Center, 1992.

\item \versal{LARSSON}, Lars. The iron age ritual building at Uppåkra, southern
Sweden.~\emph{Antiquity}, vol. 81, n. 311, 2007, pp. 11-25.

\item \versal{MONTELIUS}, Oscar. Den nordiska jernålderns kronologi 1-3. \emph{Svenska
Fornhinnesförenings Tidskrift}, n. 10, 1897, pp. 55-130.

\item \versal{MONTELIUS}, Oscar. Den nordiska jernålderns kronologi 1-3. \emph{Svenska
Fornhinnesförenings Tidskrift}, vol. 3, n. 9, 1895, pp. 155-274.

\item  \versal{MÜLLER}, Sophus. \emph{Ordning af Danske Oldsager}. Copenhaguen:
Jernalderen, 1888-1895.

\item \versal{ROSS}, Margaret Clunies. The creation of Old Norse mythology. In: \versal{BRINK},
Stefan; \versal{PRICE}, Neil (eds.). \emph{The Viking World}. New York:
Routledge, 2007, p. 231-233.

\item \versal{SKRE}, Dagfinn. \emph{Means of Exchange: Dealing with Silver in the
Viking Age: Kaupang Excavation Project Publication Series}. Aarhus:
Aarhus University Press, 2008, vol. 2.
\end{itemize}
\section{\versal{ILHAS FAROÉ}}

Ilhas Faroé é o nome do arquipélago composto de vinte e duas ilhas
(situadas no Atlântico Norte) e que dista cerca de trezentos quilômetros
do norte da Escócia. Está localizado entre a Islândia e as
\emph{Shetland}. A primeira menção que temos do arquipélago nas fontes
escritas pode ser encontrada na obra \emph{De Mensura Orbis Terrae} (Da
Medição da Terra), do monge irlandês Dicuil, situada entre o início e a metade do
século~\versal{IX}. Segundo o relato de uma testemunha, Dicuil escreveu que
alguns eremitas vindos da \emph{Scotia}, provavelmente Irlanda, viveram
no arquipélago pelo menos um século antes, mas o haviam
abandonado em razão dos piratas nórdicos. Dicuil teria escrito também que a ilha seguia vazia de
anacoretas, mas ocupada por cordeiros e aves marinhas.

Mais informações escritas sobre o passado das Ilhas Faroé podem ser
encontradas na \emph{Færeyinga Saga} (Saga dos Faroeses), escrita na
Islândia séculos após o fim da Era Viking. Acerca disso, assinalamos que
os vários assentamentos no Atlântico Norte possuem, na memória
islandesa, um acontecimento que os une nas suas origens e os remete ao
período em que Haroldo Cabelos Belos empreendeu a campanha guerreira da
centralização de seu poder régio em torno da formação de um reino
norueguês emergente. Apesar de historicamente duvidoso, tal
acontecimento encontrou lugar cativo na cultura escrita islandesa e nas
sagas ali produzidas. Segundo a \emph{Saga dos Faroeses}, um certo Grim
Kamban, sobrenome que sugere origem celta, fugindo da perseguição do rei
Haroldo Cabelos Belos, se assentou nas Ilhas Faroé, o que deve ter
ocorrido no fim do século~\versal{IX}, data posterior a ocupação dos anacoretas
segundo relato de Dicuil. O nome Kamban também aparece nos registros de ``O
Livro dos Assentamentos'', \emph{Landnámabók}, como anotação da presença
do neto de Grim entre os primeiros colonizadores islandeses. A presença
do gentílico celta pode também nos remeter a uma rede de contatos e
assentamentos por todo o Atlântico Norte, na qual as Ilhas Faroé estão
inseridas.

As informações de um possível assentamento anterior ao nórdico na ilha
também são discutíveis. Embora vários artefatos de pedra com gravuras simples
(encontrados em Skúvoy), aliados aos topônimos que contêm o elemento
\emph{papa}, corroborem para sustentar o relato de Dicuil, nenhuma descoberta
arqueológica, de fato, concorda com a permanência dos monges na ilha.
Existem argumentos de que as amostras de pólen, coletadas na ilha de
Mykines, comprovem o cultivo de cereais durante o século~\versal{VII}, mas não há
consenso arqueológico sobre isso.

De fato, sabemos muito pouco sobre os colonos e suas atividades nas
Ilhas Faroé durante o Período Viking e mesmo posterior. Ao contrário do que ocorreu na Islândia, onde se desenvolveu uma vigorosa tradição historiográfica a partir do século~\versal{XIII}, (a própria \emph{Saga dos Faroeses} foi escrita na Islândia), no caso das Ilhas Faroé, a ausência de
uma cultura escrita que preservasse as narrativas e poemas cria sérias dificuldades. Uma inscrição rúnica datada do século~\versal{XI}
e localizada em Kirkjubøur, ilha de Streymoy, é tudo o que encontramos
no momento. E mesmo tal inscrição rúnica tampouco concede maiores
informações sobre o passado histórico do assentamento nórdico no
arquipélago. As outras inscrições rúnicas encontradas em artefatos de
pedra e madeira são posteriores a Era Viking.

Entre os exemplos de importantes sítios arqueológicos nas Ilhas Faroé,
podemos citar os de Kvívík e Tjørnuvík, em Streymoy. Também foram
encontrados registros arqueológicos referentes ao Período Viking em
Fuglafjørður e Leirvík, na ilha de Eysturoy, e em Sandur, na ilha de
Sandoy. Estes, junto com o sítio de Tjørnuvík, representam os únicos
enterramentos da Era Viking. Os achados de Kvívík e Fuglafjørður incluem
artefatos calculados como pertencentes ao fim do século~\versal{IX}, enquanto que
as provas de colonização são datadas do século~\versal{X}. Especificamente em
Kvívik foram encontrados resquícios de casas com fundações sólidas,
espaço para uma fogueira central e espaço para as brasas. Também foram
encontradas as fundações de um estábulo, desgastadas pela erosão
marítima. Semelhantes são os registros em Leirvík, onde, adicionalmente,
várias centenas de objetos foram encontrados, incluindo um recipiente de esteatite, contas
de vidro, pedras de amolar feita de xisto e alguns artefatos de madeira,
como um tabuleiro de jogo.

\SIG{Pablo Gomes de Miranda}

Ver também Færeyinga saga; Inglaterra da Era Viking; Irlanda da Era
Viking; Viking.

\begin{itemize}
\item \versal{GRAHAM-CAMPBELL}, James. \emph{Os Viquingues: origens da cultura
escandinava}. Barcelona: Folio, 2006.

\item \versal{CHURCH}, Mike J. \emph{et al}. The Vikings were not the first colonizers
of the Faroe Islands. \emph{Quaternary Science Reviews,} n. \textbf{77},
2013, pp. 228-232.

\item \versal{FISHER}, Ian; \versal{SCOTT}, Ian G. Early Medieval Sculpture From the Faroes: an
illustrated catalogue. In: \versal{SMITH}, Beverley Ballin; \versal{TAYLOR}, Simon;
\versal{WILLIAMS}, Gareth. \emph{West Over Sea -- studies in Scandinavian
Sea-Borne expansion and settlement before 1300}. Leiden: Brill, 2007, pp.
363-378.

\item \versal{STEFÁNSSON}, Magnús. The Norse Island Communities of the Western Ocean.
In: \versal{KNUT}, Helle. \emph{The Cambridge History of Scandinavia, vol. 1 --
Prehistory to 1520}. Cambridge: Cambridge University Press, 2006, pp.
202-220.
\end{itemize}
\section{\versal{INGLATERRA DA ERA VIKING}}

Convencionou-se chamar de Era Viking o período que corresponde ao
movimento migratório de povos escandinavos, entre os séculos~\versal{VIII-XI},
motivado por fatores diversos, como a expansão comercial no norte da
Europa no século~\versal{VIII}, desentendimentos entre as lideranças escandinavas
e a busca por glória e riqueza, por exemplo. Destacados pelas suas
habilidades enquanto navegadores, exploradores e colonizadores, ocuparam
as Ilhas Britânicas em meados do século \versal{IX}, influenciando aspectos da
cultura, organização política e social da ilha.

A maior parte dos grupos escandinavos que se deslocaram durante a Era
Viking compunha-se principalmente de membros de uma elite. Composta por pessoas
perseguidas em suas regiões de origem ou que buscavam riqueza, glória e
fama através de aventuras, tal elite constituía um grupo seleto de viajantes. Viajar por longas distâncias não era algo
simples e, consequentemente, requeria recursos para barcos, suprimentos
e pessoas.

No caso específico da Inglaterra, a Era Viking tem como marco inicial o
ataque nórdico ao mosteiro de Lindisfarne, na Northumbria, no ano de
793. O momento de seu encerramento, no entanto, ainda é alvo de algumas discussões.
Autores como Simon Keynes, por exemplo, consideram que o final da Era
Viking seria em 1016, momento em que a liderança escandinava Canuto
(1016-1035) instaurou um novo governo na Inglaterra, iniciando seu
domínio político à ilha. Já Julian Richards estende o período até o
ano de 1066, quando Haroldo Hardrada e Swein Strithison foram derrotados
com a chegada de Guilherme da Normandia, que viria a unificar a ilha sob
seu poder no mesmo ano.

Ambas as personagens estavam ligadas ao contexto das ocupações
escandinavas na Inglaterra. Swein era filho da irmã de Canuto, rei da
Dinamarca, Noruega e Inglaterra, entre os anos de 1016-1035. Haroldo
Hardrada deixou a disputa pelo trono %inglês???
logo em seguida e preferiu investir
na luta contra Swein Strithison pelo trono da Dinamarca. Dessa forma,
dependendo do foco direcionado à pesquisa, podem ser estabelecidos
marcos distintos para o final da Era Viking na Inglaterra.

De acordo com \versal{MS.E} e \versal{MS.F} da \emph{Crônica Anglo-Saxônica}, no ano de 793,
dragões de fogo foram avistados no céu, seguidos por um período de
grande fome. No mesmo ano, grupos nórdicos, identificados na
documentação apenas como homens pagãos (\emph{heðenra manna}), devastaram
a igreja de Lindisfarne, na Northumbria, norte da Inglaterra.

Fontes narrativas cristãs -- como os \emph{Annales Bertiniani} (\emph{Anais de
São Bertin}), escritos no século~\versal{IX}, e os \emph{Annales Vedastini} (\emph{Anais
de São Vaast}), do século~\versal{X} --, expressam com detalhes o ponto de vista das
sociedades cristãs europeias e o suposto horror que esses povos, caracterizados como cruéis e sanguinários, causavam
por onde passavam.

No momento da chegada dos nórdicos às Ilhas Britânicas, a
Inglaterra não era um reino unificado. Ao contrário, era formada por uma fragmentação de territórios diversos,
controlados por povos de origem germânica e governados por reis e
sub-reis. Dentre tais territórios, os mais proeminentes no século~\versal{IX} eram Northumbria, Mercia e
Wessex. Kent e o território dos Hwicce, por exemplo, seriam submetidos
ao reino de Wessex ao longo do século~\versal{IX}. Havia ainda reinos de origem
celta que também disputavam com os povos anglo-saxões por influência e
poder.

Dentre todos os reinos do período, o mais importante na construção da
Inglaterra é Wessex. Os \emph{Gewissæ}, de acordo com a \emph{Crônica},
representavam povo saxão que se fixou na parte alta do Tâmisa, em finais
do século~\versal{VI}. No momento da chegada dos escandinavos a Wessex, Ecgberth
(802-839) ocupava o trono como rei. Após sua morte, seu filho Æthelwulf
(839-858) assumiria e os ataques escandinavos intensificariam-se. Os filhos
de Æthelwulf assumiriam o comando de Wessex após a sua morte. Dentre todos eles, o que mais se destacou foi Alfred (871-899).

Durante o período alfrediano, os esforços do rei em conter o avanço
escandinavo foram expressivos. Para além da construção de fortificações
(\emph{buhrs}), Alfred investiu também em uma corte diversificada e
instruída, consolidando um projeto político-ideológico que
proporcionaria a fundação de um futuro reino da Inglaterra. A tradução
de obras do latim para o inglês antigo, bem como a compilação do texto da
\emph{Crônica}, iniciam-se no seio da sua corte. O título de rei dos ingleses
(\emph{rex Anglorum}) e rei dos anglo-saxões (\emph{rex Angulsaxonum})
passaram a ser utilizados em sua chancela, como uma proposta de unificar
os povos anglo-saxões (cristãos) contra os escandinavos (pagãos) a
partir da ideia de um povo eclesiástico, unidos em Cristo, proposto por
Agostinho e reafirmado por Beda (672-735).

Ao projeto político de Alfred dão continuidade seus filhos Edward
(899-924), como rei de Wessex, e Æthenflæd, que governava sozinha a
Mercia desde a morte do seu marido, em 911. Utilizando como base as
fortificações construídas pelo seu pai, Edward direcionava sua ofensiva
a partir de East Anglia e das terras a sudeste das midlands, enquanto
Æthenflæd concentrava seus ataques pelo norte e nordeste. Após a morte
de Æthenflæd, em 918, Edward assegurava sua posição na Mercia.

Podemos identificar desde Alfred, passando pelo reinado de Edward, uma
crescente do poder régio e da influência de Wessex sobre outras regiões
ao sul do Humber. No entanto, nenhum deles adquiriu o status de rei da
Inglaterra de fato.

Com a morte de Edward, seu filho Æthelstan (924-939) assumiria e é
considerado o primeiro rei dos ingleses. Foi durante o seu reinado que
as terras ao sul do Humber estiveram sob a influência de Wessex.
Centralizou a administração de tal maneira que nem seu pai ou seu
avô haviam feito.

É durante o reino de Æthelstan que o título de rei dos ingleses
(\emph{rex Anglorum}) passa a ser largamente utilizado para designar os
governantes de origem anglo-saxã na ilha. Cabe mencionar a vitória inglesa contra os
escandinavos na Batalha de Brunanburh, em 937, e a conquista do reino da
Northumbria, passando-o para controle anglo-saxão -- no qual
permaneceria até o ano de 934. Dessa forma, ao sul e ao norte do rio
Humber, a supremacia inglesa parecia triunfar.

Após a morte de Æthelstan, seu meio-irmão Edmund (939-946) ascendeu ao
trono. Em 939, Olavo Guthfrithson conquista o reino de York, submetendo-o
uma vez mais ao domínio escandinavo. As vitórias dos anos subsequentes a
Alfred, Edward e Æthelstan estão marcadas por constantes
reviravoltas escandinavas, paulatinas perdas e reconquistas
territoriais. Os reis subsequentes Eadred (946-957), Eadwig (957-959),
Edgar (959-975) e Edward, o mártir (975-978), governaram por curtos
períodos e de pouco tempo hábil dispuseram.

O último rei de origem inglesa, antes do controle escandinavo na ilha em
1016, foi Æthelred~\versal{II} (978-1013 e 1014-1016). Diferentemente de seus
antecessores, Æthelred não obteve o mesmo sucesso, ou, pelo menos, o
mesmo reconhecimento das gerações seguintes após sua morte. É normalmente
conhecido como \emph{the Unready}, o Despreparado, a partir do
vocábulo em \versal{OE} \emph{unrad}, ainda vigente no início do século~\versal{XIII}. No
caso, importante ressaltar que o termo \emph{un-ræd} faz referência a \emph{sem conselho} ou
\emph{mau conselho}, haja vista que a tradução de Æthelred é
literalmente \emph{nobre conselho}. Æthelred~\versal{II} acabou sendo
constantemente associado à figura de um rei incompetente e despreparado,
pois não apresentou a mesma habilidade -- ou, pelo menos, não a mesma
sorte -- que os seus antecessores em lidar com os ataques escandinavos.

Apesar de Æthelred ser constantemente associado à culpa pela vitória de Swein, à
ocupação de Canuto e ao estabelecimento de um reino escandinavo na
Inglaterra, a historiografia atualmente tem buscado rever o período em
questão e deixado de atribuir única e exclusivamente a ele o papel
de culpado.

Segundo Richard Abels, especialista em história militar da Inglaterra no
período, o sistema de fortificações iniciado por Alfred foi
bem-sucedido, mas acabou criando ilhas de poder régio -- a partir das
quais o rei podia, por exemplo, controlar a cunhagem de moedas e o
comércio fora da Inglaterra. Os \emph{buhrs}, conforme os anglo-saxões
dispersavam o poderio escandinavo, foram aos poucos perdendo sua
natureza militar de local de guarnições de tropas, transformando-se em
cidades atrativas economicamente, ao mesmo tempo que vulneráveis para
ataques escandinavos.

No século~\versal{XI}, Æthelred e seus conselheiros teriam observado a
necessidade de investir uma vez mais em fortificações militares, a
fim de se protegerem de possíveis ataques. Além dos \emph{buhrs}, havia
ainda outros gastos envolvidos na manutenção dessas fortificações
militares, tais como armamentos para os anglo-saxões -- cotas de malha,
elmos e armaduras -- e pagamentos de tributos aos escandinavos, sob
forma de pão, carne, vinho e cerveja. Todos esses esforços e sobretudo
o investimento gasto nos levam a crer que o reinado de Æthelred foi
marcado por certa prosperidade econômica, o que fez com que uma possível
invasão escandinava se tornasse atrativa uma vez mais.

Dentre os tributos pagos aos escandinavos estava o \emph{gafol}, imposto
pago de diferentes formas pelos ingleses (venda de terras e privilégios
e uso das reservas do tesouro), a fim de manter, sempre que necessário,
os escandinavos fora do território.

Com a invasão de Canuto e o estabelecimento de um reino escandinavo em
território inglês, encerra-se o período anglo-saxão na ilha, pelo menos
por ora. O retorno de um rei de origem anglo-saxã ocorrerá apenas com
Eduardo, o Confessor, em 1042, e se encerrará com a invasão normanda à
ilha, liderada por Guilherme da Normandia, em 1066.

\SIG{Isabela Dias de Albuquerque}

Ver também Anglo-saxões e nórdicos; Canuto, o Grande; Crônica
anglo-saxônica; Danelaw; Danevirke.

\begin{itemize}
\item \versal{ABELS}, Richard. \emph{Alfred the Great: War, Kingship and Culture in
Anglo-Saxon England}. Harlow: Longman, 1998.

\item \versal{ABELS}, Richard. English Logistics and military administration, 871-1066:
The Impact of the Viking Wars. In: \versal{NØRGÅRD}, Anne Jorgensen \& \versal{CLAUSEN},
Birthe L. In: \emph{Studies in archaeology \& history - Military
aspects of Scandinavian society in a European perspective, \versal{AD} 1-1300}.
Papers from an international research seminar at the Danish National
Museum, Copenhagen, 2-4 May, 1996. Copenhagen: National Museum, 1997. p.
257-265. Disponível em:
\href{http://deremilitari.org/2013/07/english-logistics-and-military-administration-871-1066-the-impact-of-the-viking-wars/}{\emph{http://deremilitari.org/2013/07/english-logistics-and-military-administration-871-1066-the-impact-of-the-viking-wars/}}.
Acesso em 27/10/2015.

\item \versal{FOOT}, Sarah. \emph{Æthelstan: The First King of England}. New Haven and
London: Yale University Press, 2011.

\item \versal{FOOT}, Sarah. The Making of Algelcynn: English Identity Before the Norman
Conquest. In: \emph{Transactions of the Royal Historical Society}. Sixth
Series, vol. 6, 1996.

\item \versal{JESCH}, Judith. \emph{Viking Diaspora}. London and New York: Routledge,
2015.

\item \versal{KEYNES}, Simon. A Tale of Two Kings: Alfred the Great and Æthelred the
Unready. \emph{Transactions of the Royal Historical Society}, vol. 36,
1986, pp. 195-217.

\item \versal{SAUL}, Nigel. \emph{The Oxford Illustrated History of Medieval England}.
Oxford: Oxford University Press, 2000.

\item \versal{YORKE}, Barbara. \emph{Kings and Kingdons of Early Anglo-Saxon England}.
Taylor \& Francis e-Library, 2003.

\item \versal{YORKE}, Barbara. \emph{Wessex in the Early Middle Ages}. London:
Leicester University Press, 1995.
\end{itemize}


\section{\versal{INSCRIÇÕES RÚNICAS}}


No antigo \emph{futhark}, período anterior à Era Viking, havia 24 runas
e cada som era representado por uma runa e cada runa representava apenas
um som, mas não completamente pois havia algumas redundâncias como, por
exemplo, as runas \textarc{I} 
\textbf{ë} e \textarc{\ng} 
\textbf{ŋ}, já que existiam as vogais
\textarc{e} \textbf{e} e
\href{https://en.wikipedia.org/wiki/\textarc{i}}{\emph{\textarc{i}}} \textbf{i} e a
combinação
\textarc{n}\textarc{g}
\textbf{ng}. De acordo com Spurkland, cada runa tinha um
nome comum e este não era aleatório: o som inicial desse nome era o
valor sonoro da runa. Assim, a runa
\href{https://en.wikipedia.org/wiki/\textarc{f}}{\emph{\textarc{f}}}, chamada de
*\emph{fehu} ``dinheiro, riqueza, gado'', tinha o valor sonoro
*\emph{f}. O termo científico para esse dispositivo mnemônico é
princípio acrofônico. No entanto, as runas
\href{https://en.wikipedia.org/wiki/\textarc{R}}{\emph{\textarc{\R}}} \textbf{R}
(*\emph{algiR} ``alce'') e
\href{https://en.wikipedia.org/wiki/\textarc{\ng}}{\emph{\textarc{\ng}}} \textbf{ŋ}
(*\emph{IngvaR} ``nome de um deus'') são exceções pois representam sons
que nunca aparecem no início de uma palavra.

Esse alfabeto rúnico foi utilizado na sincronia linguística, que é
conhecida por acadêmicos escandinavos como \emph{urnodisk},
\emph{Proto-Scandinavian,} pela língua inglesa, \emph{Early Runic} por
Nielsen e de ``rúnico primitivo'' em nossa proposta de tradução.
Tal sincronia existiu entre os anos 200 e 500
d.C., ou seja, séculos antes da Era Viking. De acordo com Nielsen,
as línguas germânicas do norte são descendentes diretas do
rúnico primitivo, diferentemente de outros trabalhos mais antigos como,
por exemplo, de Antonsen, que assume que o rúnico primitivo gerou
todas as línguas germânicas do noroeste, i.e., além das escandinavas, o
saxão antigo, o frísio antigo e o inglês antigo. Um exemplo de inscrição
nesse alfabeto é a famosa inscrição em um dos dois chifres de ouro
de Gallehus (400-450 d.C., sul da Jutlândia, Dinamarca).

Uma estela rúnica de período um pouco posterior, mas ainda antes da Era
Viking é a inscrição de Eggja (Sogndal, Noruega 700 d.C., código \versal{N
KJ}101). Ela não apresenta 24 runas, mas 21. Esta estela, e também as
estelas de Blekinge (Suécia, 625 d.C., códigos \versal{DR} 357, 358 e 359),
evidenciam uma transição gradual do rúnico primitivo para o nórdico da
Era Viking. De acordo com Spurkland, 
no período linguístico da composição desta estela havia duas runas
para a vogal \textbf{a}: a runa
\href{https://en.wikipedia.org/wiki/\textarc{a}}{\emph{\textarc{a}}}, que indica o
\textbf{a} nasal antes de /m/ e /n/, como em
\emph{\href{https://en.wikipedia.org/wiki/\textarc{l}}{\textarc{l}}\href{https://en.wikipedia.org/wiki/\textarc{a}}{\textarc{a}}\href{https://en.wikipedia.org/wiki/\textarc{n}}{\textarc{n}}\href{https://en.wikipedia.org/wiki/\textarc{d}}{\textarc{d}}\href{https://en.wikipedia.org/wiki/\textarc{e}}{\textarc{e}}}
(\textbf{lande}, ``terra'' no dativo), mesmo se a consoante seguinte
tenha caído como, como em
\emph{\href{https://en.wikipedia.org/wiki/\textarc{l}}{\textarc{l}}\href{https://en.wikipedia.org/wiki/\textarc{a}}{\textarc{a}}\href{https://en.wikipedia.org/wiki/\textarc{t}}{\textarc{t}}}
(\textbf{ląt}, ``terra'' no acusativo''), e também a runa \textara{j}, que
representa o \textbf{a} não nasalizado, como em
\emph{\href{https://en.wikipedia.org/wiki/\textarc{h}}{\textarc{h}}\href{https://en.wikipedia.org/wiki/\textarc{u}}{\textarc{u}}\href{https://en.wikipedia.org/wiki/\textarc{w}}{\textarc{w}}}\textara{j}\textarc{\R}
(\textbf{huwaR}, ``quem''). Além do mais, as runas redundantes do antigo
\emph{futhark} não aparecem nessa estela. Após 500 d.C. ocorreu, também,
tanto a queda do \textbf{j} inicial quanto a apócope do \textbf{a}
final, o que fez com que o nome *\emph{jāra} ``ano'', que representa a
runa \href{https://en.wikipedia.org/wiki/\textarc{j}}{\emph{\textarc{j}}} \textbf{j}, se
desenvolvesse para \emph{ár}, que consequentemente resultou na mudança
do valor sonoro da runa de *{[}j{]} para *{[}a{]}; além disso, a
aparência da runa também mudou.

Um desenvolvimento posterior no alfabeto rúnico é representado pela
inscrição rúnica no \emph{Crânio de Ribe} (Dinamarca, 720 d.C.), de
acordo com Spurkland, encontrado na cidade dinamarquesa de
Ribe. Esta cidade é considerada uma das mais antigas da Era Viking e foi
ponto estratégico para o contato com as Ilhas Britânicas e Europa
Ocidental. Nesta estela ocorreu uma redução de 21 para 16 runas. A
estela rúnica dinamarquesa de Gørlev, de Sjæland (850-900), de acordo
com Spurkland, também apresenta um alfabeto com 16 letras.
Nesta estela, há o nome Halvdan grafado como
% *\textara{j}\textarn{a}\textarc{l}\textarc{f}\textarn{a}\textarc{n}
% * \textarn{\textara{j} \textarn{a} \textarc{l} \textarc{f} \textarn{a} \textarc{n}}
% \textarn{h a l f t n}
(\textbf{Halftan}). Spurkland comenta que todas as runas
desta estela têm apenas um tronco vertical como, por exemplo, \textara{j}
\textbf{h} e \textbf{m}, diferentemente dos estágios anteriores que
tinham dois, i.e., \href{https://en.wikipedia.org/wiki/\textarc{h}}{\emph{\textarc{h}}} e
\href{https://en.wikipedia.org/wiki/\textarc{m}}{\emph{\textarc{m}}}, respectivamente.
Percebe-se também que a runa \textara{j} \textbf{h} nesse estágio tinha a mesma
forma que a runa \textara{j} \textbf{a} na estela de Eggja e no crânio de Ribe, no
entanto, este se transformou em \textarn{a}.

Uma simplificação ainda maior ocorre nas inscrições das lascas de
madeira de Haithabu (hoje em dia norte da Alemanha, no período cidade
dinamarquesa-sueca, 800 d.C.). De acordo com a enciclopédia \emph{Hele
Norges Leksikon}, Haithabu foi o ponto de comércio mais
importante na Era Viking, pois lá se encontravam comerciantes vindos do
mar do Norte e do Báltico. O local foi fundado por vikings dinamarqueses
em 804 e se desenvolveu rapidamente para uma cidade portuária e de
comércio; no entanto, no século~\versal{X} passou para as mãos suecas, em seguida
passou para o domínio alemão e, por fim, voltou para os dinamarqueses.
Antes de ser destruída em 1066, ela já havia perdido sua importância na
primeira parte do século~\versal{XI}.

Moltke afirma que o alfabeto dinamarquês de 16 runas, do
tipo da estela de Gørlev, assim como o alfabeto antecessor de 24 runas,
adaptados para serem cunhados em madeira, eram muito conservados para os
suecos que vieram para Haithabu. Portanto, com base no \emph{futhark}
dinamarquês, eles estabeleceram um \emph{futhark} significativamente
mais simples e prático. Outra inscrição rúnica de Haithabu, que
apresenta algumas runas suecas (a respeito das grafias de \textbf{m},
\textbf{a} e \textbf{n}), é a estela rúnica \versal{DR} 2 (935-950 d.C., de
acordo com o \emph{Nationalmuseet} de Copenhagen): \textart{\m} \textbf{m} em
\textarn{k}\href{https://en.wikipedia.org/wiki/\textarc{u}}{\emph{\textarc{u}}}\textart{\m} \textbf{kum}, parte
da palavra ``memória de morte'', em contraste com \textbf{m} em
\textarn{k}\emph{\href{https://en.wikipedia.org/wiki/\textarc{u}}{\textarc{u}}\href{https://en.wikipedia.org/wiki/\textarc{r}}{\textarc{r}}}\textarc{\R}
\textbf{kurmR}, nome ``Gormr'' na estela \versal{DR} 4 (do mesmo período); \textarc{n}
\textbf{n} em
\textarn{k}\textarc{n}\emph{\href{https://en.wikipedia.org/wiki/\textarc{u}}{\textarc{u}}\href{https://en.wikipedia.org/wiki/\textarc{b}}{\textarc{b}}\href{https://en.wikipedia.org/wiki/\textarc{u}}{\textarc{u}}}
\textbf{knubu} ``Gnupu'', em contraste com
\href{https://en.wikipedia.org/wiki/\textarc{n}}{\emph{\textarc{n}}} \textbf{n} em
\textarn{k}\emph{\href{https://en.wikipedia.org/wiki/\textarc{n}}{\textarc{n}}\href{https://en.wikipedia.org/wiki/\textarc{u}}{\textarc{u}}\href{https://en.wikipedia.org/wiki/\textarc{b}}{\textarc{b}}\href{https://en.wikipedia.org/wiki/\textarc{u}}{\textarc{u}}}
\textbf{knubu} na \versal{DR} 4 e \textart{a} \textbf{a} em
\textarn{k}\textart{a}\emph{\href{https://en.wikipedia.org/wiki/\textarc{r}}{\textarc{r}}\href{https://en.wikipedia.org/wiki/\textarc{\th}}{\textarc{\th}}\href{https://en.wikipedia.org/wiki/\textarc{i}}{\textarc{i}}}
\textbf{karþi} ``fez'' em contraste com
\textarn{k}\textarn{a}\emph{\href{https://en.wikipedia.org/wiki/\textarc{r}}{\textarc{r}}\href{https://en.wikipedia.org/wiki/\textarc{\th}}{\textarc{\th}}\href{https://en.wikipedia.org/wiki/\textarc{i}}{\textarc{i}}}
\textbf{karþi} na estela \versal{DR} 4. Tanto a estela rúnica \versal{DR} 2 quanto a \versal{DR} 4
foram erigidas a pedido da rainha dinamarquesa, mulher do rei sueco
Gnupa, em memória do filho deles, Sigtryg. Assim, a \versal{DR} 2 tem elementos
de runas suecas, ao passo que a \versal{DR} 4 é dinamarquesa.

Nestas runas de Haithabu percebemos uma simplificação ainda mais
avançada em relação à estela de Gørlev. Em contraponto com as runas \textarn{\A}
\textbf{ą} e
\textart{\b}
\textbf{b} (também com as variantes \textart{A} e
\textart{\b}
respectivamente), muitas runas têm galhos apenas de um lado do tronco,
ao passo que as runas de Gørlev tem galhos nos dois lados (\textart{t} \textbf{t}
em vez de \href{https://en.wikipedia.org/wiki/\textarc{t}}{\emph{\textarc{t}}}, e \textarc{n}
\textbf{n} em vez de \href{https://en.wikipedia.org/wiki/\textarc{n}}{\emph{\textarc{n}}}).
Outras runas como, por exemplo, \textart{\h} \textbf{h}, \textart{\R} \textbf{s}, \textart{\m} \textbf{m}
e \href{https://en.wikipedia.org/wiki/\textart{R}}{\emph{\textart{R}}} \textbf{R} ficaram
muito menos elaboradas


\begin{table}[]
\centering
\caption{O alfabeto do novo \emph{Futhark}, com as variantes citadas:}
\label{my-label}
\begin{tabular}{cccc}
f                            & u                             & þ                                            & ą                             \\
\textarc{f}                & \textarc{u}                 & \textarc{\th}                              & \textarn{\A}, \textart{A} \\\hline
r                            & k                             & H                                            & n                             \\
\textarc{r}                & \textarn{k}                 & \textara{j}, \textart{\h}                & \textarc{n}, \textarc{n}  \\\hline
ī                            & a                             & R                                            & s                             \\
\textart{i}                & \textarn{a}, \textart{a}  & \textarc{\R}, \textart{\R}               & \textarn{s}, \textart{s}  \\\hline
t                            & b                             & m                                            & l                             \\
\textarc{t}, \textart{t} & \textarc{b}, \textart{\b} & \textart{\m}, \textarn{m}, \textarn{m} & \textarc{l}                
\end{tabular}
\end{table}


Em vista disso, Spurkland afirma que o alfabeto rúnico da
Era Viking, com suas dezesseis runas, se desenvolveu em duas variantes:
a variante da estela de Gørlev, chamada de \emph{langkvistruner} ``runas
de ramos longos'' ou \emph{normalruner} ``runas normais'', ao passo que
a variante de Haithabu é chamada de \emph{kortvistruner/stuttruner}
``runas de ramos curtos. O autor também afirma que as runas de ramos
longos podem ser referidas como runas dinamarquesas e as de ramos
curtos, de sueco-norueguesas; e que tais denominações têm a ver com as
teorias sobre as origens e distribuição geográfica dos diferentes
alfabetos.

Um exemplo de inscrição rúnica norueguesa que apresenta a forma de ramos
longos é a estela de Valby:
\textarn{a}\href{https://en.wikipedia.org/wiki/\textarc{u}}{\emph{\textarc{u}}}\textarn{a}\href{https://en.wikipedia.org/wiki/\textarc{r}}{\emph{\textarc{r}}}þ\textarc{\R}
\href{https://en.wikipedia.org/wiki/\textarc{f}}{\emph{\textarc{f}}}\textarn{a}þ\href{https://en.wikipedia.org/wiki/\textarc{i}}{\emph{\textarc{i}}}
{[}\href{https://en.wikipedia.org/wiki/\textarc{u}}{\emph{\textarc{u}}}{]}\href{https://en.wikipedia.org/wiki/\textarc{l}}{\emph{\textarc{l}}}\textarc{\R}
\textbf{auarþR faþi (u)lR} ``Hávarðr pintou com fé''. A nossa tradução
está de acordo com a interpretação de Magnus Olsen, citado por Spurkland.
Esta estela é de Tjølling, Vestfold (século~\versal{IX}, código
\versal{N}140). Por outro lado, um exemplo de inscrição rúnica, também
norueguesa, contemporânea e de uma região muito próxima, mas que
apresenta a forma de ramos curtos, é a inscrição cunhada num pedaço de
madeira, que poderia ser um remo, encontrada nas ruínas do barco de
Oseberg (Oseberghaugen, Vestfold, 834 d.C.):
\emph{\href{https://en.wikipedia.org/wiki/\textarc{l}}{\textarc{l}}\href{https://en.wikipedia.org/wiki/\textarc{i}}{\textarc{i}}}\textart{t}\emph{\href{https://en.wikipedia.org/wiki/\textarc{i}}{\textarc{i}}\href{https://en.wikipedia.org/wiki/\textarc{l}}{\textarc{l}}\href{https://en.wikipedia.org/wiki/\textarc{u}}{\textarc{u}}\href{https://en.wikipedia.org/wiki/\textarc{i}}{\textarc{i}}}\textart{\R}\textart{\m}
\textbf{litiluism} ``idiota''. Spurkland entende que essa
inscrição tem as seguintes palavras e as representa com cognatas no
antigo nórdico: \emph{lítill} ``pequeno, pouco'' e \emph{víss}
``esperto, inteligente'', o que resultaria numa composição adjetival
\emph{lítilvíss} ``pouco inteligente, estúpido''. No entanto, faltaria
compreender o motivo de existir um \textart{\m} \textbf{m} sozinho no final desse
adjetivo. Em vista disso, o autor sugere que essa consoante representa
um ideógrafo: uma vez que a runa se chama \emph{*mannaz} ``homem'', ela
sozinha poderia representar o nome ``homem''; assim, o significado
completo seria ``homem idiota'', ``seu idiota''; outros interpretam como
uma frase mais filosófica, ``o homem sabe pouco'' ou ``pouco sabe o
homem''. Por volta do século~\versal{XI}, um alfabeto rúnico se desenvolve na
Noruega, que consiste de tanto runas de ramos curtos quanto de ramos
longos.

Um local onde a simplificação gráfica foi levada ao extremo é em
Hälsingland, Suécia e, assim muitas vezes são chamadas de runas de
Hälsing. Lá foram encontradas estelas rúnicas compostas apenas de ramos,
com troncos reduzidos ou omitidos completamente. Portanto, o que
distingue uma runa da outra é para que lado os ramos estão virados: para
direita ou para esquerda, para baixo ou para cima, ou se estão
posicionados na parte superior ou inferior da linha como, por exemplo,
em ´ \textbf{t} (originado a partir de \textart{t}) e ˋ \textbf{l} (originado a
partir de \href{https://en.wikipedia.org/wiki/\textarc{l}}{\emph{\textarc{l}}}).

No período transitório do antigo para o novo \emph{futhark} (entre séculos~{\versal{VII} e \versal{VIII}}), como apresentado no verbete \versal{Linguagem},
ocorreram modificações linguísticas: tanto o encurtamento de palavras
por conta da redução ou omissão de sílabas átonas finais, como também o
surgimento de novas vogais por conta de transformações vocálicas. Em
vista disso, novas vogais como /æ/, /ø/, /y/ e /ǫ/ surgiram ao mesmo
tempo que as palavras ficavam mais curtas. Spurkland aponta que
houve um descompasso entre as novas vogais e a representação gráfica,
pois, enquanto o número de vogais aumentava, o número de runas diminuía
como, por exemplo, o desaparecimento das runas que representavam
\textbf{ë}, \textbf{o}, \textbf{w}, \textbf{p}, \textbf{d} e \textbf{g}.
Como exemplo, o autor cita algumas palavras da inscrição rúnica do
Crânio de Ribe (770 d.C.), apresentado acima:
\emph{\href{https://en.wikipedia.org/wiki/\textarc{u}}{\textarc{u}}\href{https://en.wikipedia.org/wiki/\textarc{\th}}{\textarc{\th}}\href{https://en.wikipedia.org/wiki/\textarc{i}}{\textarc{i}}\href{https://en.wikipedia.org/wiki/\textarc{n}}{\textarc{n}}}
\textbf{uþin} ``Odin'' e
\emph{\href{https://en.wikipedia.org/wiki/\textarc{h}}{\textarc{h}}\href{https://en.wikipedia.org/wiki/\textarc{i}}{\textarc{i}}}\textara{j}\emph{\href{https://en.wikipedia.org/wiki/\textarc{l}}{\textarc{l}}\href{https://en.wikipedia.org/wiki/\textarc{b}}{\textarc{b}}}
\textbf{hialb} que pode significar tanto o verbo ``ajudar'' no tempo
passado (antigo nórdico \emph{hjalp}) quanto o substantivo ``ajuda'' (no
antigo nórdico \emph{hjǫlp}). O fato de o escultor ter utilizado a runa
\href{https://en.wikipedia.org/wiki/\textarc{u}}{\emph{\textarc{u}}} \textbf{u} em vez de
\href{https://en.wikipedia.org/wiki/\textarc{o}}{\emph{\textarc{o}}} \textbf{o}, na primeira
palavra, e \href{https://en.wikipedia.org/wiki/\textarc{b}}{\emph{\textarc{b}}} \textbf{b}
em vez de \href{https://en.wikipedia.org/wiki/\textarc{p}}{\emph{\textarc{p}}} \textbf{p},
na segunda palavra, mostra que tanto
\href{https://en.wikipedia.org/wiki/\textarc{o}}{\emph{\textarc{o}}} \textbf{o} quanto
\href{https://en.wikipedia.org/wiki/\textarc{p}}{\emph{\textarc{p}}} \textbf{p} não estavam
mais em uso.

Com tais desenvolvimentos, chegou um momento em que havia apenas quatro
runas vocálicas, \textart{A} \textbf{ą}, \textart{a} \textbf{a},
\href{https://en.wikipedia.org/wiki/\textarc{i}}{\emph{\textarc{i}}} \textbf{i} e
\href{https://en.wikipedia.org/wiki/\textarc{u}}{\emph{\textarc{u}}} \textbf{u}, que
representavam as vogais /ã/, /a/, /i/, /u/, /e/, /o/, /æ/, /y/, /ø/, /ǫ/
e, também, /w/ e /j/. Segundo Spurkland, é compreensível o
desaparecimento de runas redundantes, como
\href{https://en.wikipedia.org/wiki/	extarn{I}}{\emph{	extarn{I}}} \textbf{ë} e
\href{https://en.wikipedia.org/wiki/\textarc{\ng}}{\emph{\textarc{\ng}}} \textbf{ŋ}, em um
sistema que preza cada som para cada runa e cada runa para cada som e,
além do mais, também é concebível que tenha ocorrido uma simplificação
gráfica para facilitar na cunhagem; no entanto, a redução do número de
runas em um terço, ao passo que novos sons aparecem, é de difícil
explicação.

Ainda assim, Spurkland cita as seguintes explicações
que fazem referência à redução no número de runas no novo
\emph{futhark}: 1. isolação e declínio cultural (refutada por ele e
sugerida por Otto von Friesen (1918-19); 2. magia e divisões de grupos
rúnicos, i.e. três famílias com oito runas cada, teoria também refutada
pelo autor; 3. condições gráficas, 4. mudanças fonéticas e 5. mudanças
nos nomes das runas.

As três últimas explicações, de acordo com o autor, não podem ser vistas
de maneiras separadas, mas sim como complementares; ademais, elas não
devem ter tido o mesmo peso de importância nesse processo. Por exemplo,
não deve ter havido um anseio para a realização de uma simplificação
gráfica em que se estabelecesse um sistema de runas com apenas um
tronco; porém, no decorrer das modificações, a simplificação gráfica foi
um fator necessário. Spurkland afirma que o
principal fator para a transição do antigo para o novo \emph{futhark}
foi a mudança dos nomes das runas. Uma vez que cada runa tinha um nome,
que iniciava com o som que essa runa representava, se ocorressem mudanças
linguísticas estimuladas pelas leis fonéticas nesses nomes e, assim,
afetasse o início de seu nome, acarretaria na desconexão entre o nome da
runa e o som que ela representava. Exemplos que se mantiveram e que
desapareceram no novo \emph{Futhark}:
\href{https://en.wikipedia.org/wiki/\textarc{a}}{\emph{\textarc{a}}} \textbf{*\emph{ansuR}}
*{[}a{]} ``deidade'' (estágio 1) 
{\emph{\textarc{a}}} \textbf{*\emph{ãsuR}}
*{[}ą{]} (*a \textgreater{}*ą) (estágio 2)  \textarn{\A},\textart{A} ou \emph{\textbf{áss}}
*{[}a{]} (estágio3); \href{https://en.wikipedia.org/wiki/\textarc{j}}{\emph{\textarc{j}}}
\textbf{*\emph{jāra}} *{[}j{]} ``ano'' (estágio 1) % sinal estranho
\textara{j}
\textbf{*\emph{āra}} *{[}a{]} (*jā \textgreater{} *a) (estágio 2) % sinal estranho
\textarn{a},
ou \textart{a} \emph{\textbf{ár}} *{[}a{]} (estágio 3);
\href{https://en.wikipedia.org/wiki/\textarc{e}}{\emph{\textarc{e}}} \textbf{*ehwaR}
*{[}e{]} ``cavalo'' (estágio 1) % sinal estranho
\href{https://en.wikipedia.org/wiki/\textarc{e}}{\emph{\textarc{e}}} \textbf{jór} *{[}j{]}
(*eh \textgreater{} *jó-) (estágio 2) % sinal estranho
desapareceu (estágio 3);
\href{https://en.wikipedia.org/wiki/\textarc{w}}{\emph{\textarc{w}}} \textbf{*wunju}
*{[}w{]} ``felicidade'' (estágio 1) % sinal estranho
\href{https://en.wikipedia.org/wiki/\textarc{w}}{\emph{\textarc{w}}} \textbf{*unju} *(u)
{[}*wu- \textgreater{} *u{]} % sinal estranho
desapareceu (estágio 3);
\href{https://en.wikipedia.org/wiki/\textarc{u}}{\emph{\textarc{u}}} \textbf{*uruR} *{[}u{]}
``auroque'' (estágio 1) % sinal estranho
\href{https://en.wikipedia.org/wiki/\textarc{u}}{\emph{\textarc{u}}} \textbf{ur} *(u)
(estágio 2) % sinal estranho
\href{https://en.wikipedia.org/wiki/\textarc{u}}{\emph{\textarc{u}}}
\textbf{ur} *(u) (estágio 3);
\href{https://en.wikipedia.org/wiki/\textarc{o}}{\emph{\textarc{o}}} \textbf{*ōþila}
*{[}ō{]} ``propriedade'' (estágio 1) % sinal estranho 
desapareceu (seria \textbf{*øðil}
*(ø) (estágio 2);

A mudança fonética mais drástica ocorreu com *\emph{ansuR}. A síncope do
*{[}n{]} por causa do *{[}s{]} posterior fez com que o *{[}a{]} inicial
se tornasse nasal, *{[}ɐ̃{]}. Ocorre, posteriormente, o fechamento sonoro
(\emph{lydukning}), em que o *{[}ɐ̃{]} se fecha para *{[}a{]} por conta
do *{[}u{]} posterior, portanto *\emph{ásuR}. O mesmo *{[}u{]} é
sincopado e ocorre a assimilação progressiva -sR- \textgreater{} -ss-,
se formando, assim, \emph{áss.} Na transformação *\emph{ehwaR} para
\emph{jór,} testemunha-se um processo que chamamos de ruptura
(\emph{bryting}), que se trata do rompimento do *{[}e{]}, que se
transforma em *{[}ja{]}. A runa *ehwaR
(\href{https://en.wikipedia.org/wiki/\textarc{e}}{\emph{\textarc{e}}} \textbf{e}) não
entraria no novo sistema de runas por ter apenas um tronco. O nome da
runa \textbf{*}ōþila (\href{https://en.wikipedia.org/wiki/\textarc{o}}{\emph{\textarc{o}}}
\textbf{o}) provavelmente transformou-se em *\emph{øðil} em estágio
posterior, por meio de mutação vocálica causada pelo \emph{i}, e, assim
se esperaria que essa runa começasse a representar esse novo som
*{[}ø{]}, da mesma maneira que ocorreu no sistema anglo-saxão; no
entanto, ela sumiu. A runa *\emph{wunju} que virou *\emph{unju} também
sumiu por já existir a runa \emph{ur}.

Presenciamos, portanto, um desenvolvimento de cinco sons vocálicos,
*{[}i{]}/, *{[}e{]}, *{[}u{]}, *{[}o{]} e *{[}a{]}, para dez nos
estágios mais antigos do antigo nórdico (além das citadas, também
surgiram *{[}ɐ̃{]}, *{[}æ{]}, *{[}y{]},
*{[}ø{]} e *{[}ɒ{]}). Foi um
desenvolvimento muito mais crescente do que o que ocorreu, por exemplo,
no antigo inglês, no entanto, não havia um número suficiente de runas
para representar todos esses sons. Spurkland afirma que
não apenas a mudança dos nomes das runas foram fatores importantes para
a transição do antigo para o novo \emph{futhark}, mas também as mudanças
fonéticas das vogais. As únicas runas que permaneceram no novo
\emph{futhark} para representar vogais foram
\href{https://en.wikipedia.org/wiki/\textarc{i}}{\emph{\textarc{i}}} \textbf{i},
\href{https://en.wikipedia.org/wiki/\textarc{u}}{\emph{\textarc{u}}} \textbf{u}, \textarn{\A}
\textbf{ą} e \textarn{a} \textbf{a}, com algumas variantes.

Uma justificativa dada pelo autor frente à manutenção de poucas runas
vocálicas e a não criação de outras tem a ver com as terminações
flexionais do período. As terminações flexionais são partes muito
importantes das palavras, pois contêm informação gramatical, que faz com
que saibamos se uma palavra é um verbo, adjetivo, substantivo ou
pronome, se está no presente ou no passado, se é singular ou plural, se
é masculino, feminino ou neutro ou se está no caso nominativo,
acusativo, dativo ou genitivo. Portanto, para distinguir essas
diferentes formas de palavras, as quatro vogais *{[}i{]}, *{[}u{]},
*{[}a{]} e *{[}ɐ̃{]} eram muito importantes. As outras vogais, às quais
não eram atribuídas runas próprias, isto é, *{[}e{]}, *{[}o{]},
*{[}æ{]}, *{[}y{]}, *{[}ɒ{]} e *{[}æ{]}, apareciam principalmente no
interior de palavras, posição em que a distinção não é incisiva. Assim,
por exemplo, a runa \href{https://en.wikipedia.org/wiki/\textarc{i}}{\emph{\textarc{i}}}
\textbf{i} podia representar tanto *{[}i{]}, *{[}e{]} e *{[}æ{]} no
interior das palavras e, mesmo assim, o leitor conseguiria entenderia de
que elas se tratavam pelo contexto.

Um exemplo de runa na Era Viking norueguesa é pedra de Kuli (Kuløy em
Smøla, Nordmøre, Møre og Romsdal código N449). No entender de Asklak
Liestøl (1956), citado por Spurkland, lê-se:

% JORGE: @FELIPE
\begin{quote}
$\dagger$\textarc{\th}\textarc{u}\textarc{r}\textarc{i}\textarc{r}:\textart{a}\textarc{u}\textarn{k}:
\textara{j}\textart{a}\textarc{l}\textarc{u}\textart{a}\textarc{r}\textarc{\th}\textarc{r}:
\textarc{r}\textart{a}\textarc{i}\textart{\R}\textart{t}\textarc{u}\textart{\R}\textart{t}\textart{a}\textarc{i}\textarc{n}:
\textarc{\th}\textarc{i}\textarc{n}\textart{\R}\textarc{i}\textart{a}\textarc{f}\textart{t}
\textarc{u}{[}\textarc{l}{]}\textarc{f}{[}\textarc{l}{]}\textarc{i}\textarc{u}{[}\textart{t}{]} 

$\dagger$ \textbf{þurir:auk:haluarþr:raistustain:þinsiaftu{[}l{]}f{[}l{]}iu{[}t{]};}
\emph{Þórir ok Hallvarðr reistu stein eptir Ulfjót...;} ``Þórir e
Hallvarðr erigiram essa pedra para (em memória de) Ulfjótr ...''
\end{quote}

E também

\begin{quote}
$\dagger$ \textart{t}\textarc{u}\textart{a}\textarc{l}\textarc{f}.\textarc{u}\textarc{i}\textarc{n}
\textart{t}\textarc{r}\textara{j}\textart{a}{[}\textarc{f}{]}\textarc{\th}\textarc{i}:
{[}\textarn{k}{]}\textarc{r}\textarc{i}\textart{\R}{[}\textart{t}\textarc{i}\textarc{n}\textart{t}{]}
\textarc{u}\textarc{m}\textarc{r}:\textarc{u}\textarc{i}\textarc{r}\textarc{i}{[}\textart{t}{]}\textarc{i}
\textarc{n}\textarc{u}\textarc{r}\textarc{i}\textarn{k}\textarc{i}{]}

\textbf{tualf.uintr.ha{[}f{]}þi:{[}k{]}ris{[}tint{]}umr: uiri{[}t{]}
{]}inuriki} 

\emph{...Tólf vetr hafði kristindómr verit i Nóregi;}

``...Doze invernos o cristianismo tem estado na Noruega'' (traduções
nossas).
\end{quote}

De acordo com Spurkland, essa estela é muito especial
porque os escultores informaram que por doze invernos o cristianismo
tinha estado na Noruega quando ela foi erigida. Nela também aparece o
nome ``Noruega'' na Noruega; no entanto, essa palavra já havia sido
registrada anteriormente em uma estela dinamarquesa (\textbf{nuruiak,}
Jelling \versal{II}, 960 d.C., código \versal{DR} 42). A palavra ``cristianismo'', por sua
vez, tem seu primeiro testemunho nessa pedra. O autor coloca um
questionamento muito pertinente com relação a essa estela: ``Qual era a
consciência daqueles que moravam em Kuløya, Smøla, de que o cristianismo
tinha chegado na Noruega? Qual acontecimento que está sendo referido que
ocorreu há doze invernos?'' (p. 121). A comuna de Smøla se localiza na
região central e litorânea da Noruega.

A teoria mais apoiada por Spurkland é a do filólogo Nils
Hallan. Ele afirma que a inscrição faz referência a um evento do
reino e, assim, relaciona a pedra a um \emph{Thing} que ocorreu em
Moster (Bømlo, Hordaland), onde o Óláfr Tryggvason e o bispo Grímkell
determinaram uma lei que fazia com que o Cristianismo se tornasse a
religião do reino. A antiga lei de Gulating faz várias vezes referências
a tal determinação de Óláfr e Grímkell em Moster. Os historiadores
afirmam que isso ocorreu por volta de 1022 e 1024; portanto, a pedra
teria sido erigida entre 1034 e 1036. Também foram encontrados
fragmentos de madeira no local onde acredita-se que a pedra foi colocada
e que tais fragmentos parecem como restos de uma estrutura de ponte. Por
meio de uma análise dendrocronológica, determinou-se que a madeira
pertence a uma árvore tombada em 1034 e, assim, a estela de Kuli pode
ser datada de 1034 d.C. Assim, \emph{se} é o \emph{Thing} de Moster que
essa pedra faz referência, \emph{então} ele ocorreu em 1022 (grifos do
autor). Não há nenhum argumento decisivo com base histórica para ser
utilizado de maneira que possa refutar essa data.

\SIG{Yuri Fabri Venancio}

Ver também Linguagem; Literatura; Norreno; Poesia éddica; Poesia
escáldica; Sagas.

\begin{itemize}
\item \versal{ANTONSEN}, Elmer H. \emph{A Concise Grammar of the Older Runic
Inscriptions}. Tübingen: Max Niemeyer Verlag, 1975.

\item  \versal{Enciclopédia} Hele Norges Leksikon. \versal{B}. 6: \versal{H-I}. Oslo: Hjemmets bokforl,
1997, p. 29.

\item \versal{MOLTKE}, Erik. \emph{Runerne i Danmark og deres oprindelse}. København:
Forum, 1976.

\item \versal{NIELSEN}, Hans F. \emph{The early runic language of Scandinavia: studies
in Germanic dialect geography}. Heidelberg: Universitätsverlag Carl
Winter, 2000.

\item \versal{SPURKLAND}, Terje. \emph{I begynnelsen var fuþark. Norske runer og
runeinnskrifter}. Oslo: Landslaget for norskundervisning/Cappelen
akademisk forlag, 2001.

\item \versal{VENANCIO}, Yuri Fabri. \emph{Um estudo etimológico de internacionalismos:
cognatos nas línguas portuguesa e norueguesa}. Dissertação
(mestrado em Filologia e Língua portuguesa) -- Faculdade de Filosofia,
Letras e Ciências Humanas. São Paulo: Universidade de São Paulo (\versal{USP}),
2017.
\end{itemize}
\section{\versal{IRLANDA DA ERA VIKING}}

Os Escandinavos, popularmente conhecidos como vikings, aparecem muitas
vezes na historiografia com o epíteto de \emph{Ostmen} (``Homens do Leste'') ou
\emph{Loch lannaigh} (irlandês para ``povo da terra dos lagos''). Durante muito
tempo, a historiografia irlandesa apenas apontava a presença escandinava
de acordo com seus primeiros registros na Idade Média, ou seja, como
meros saqueadores. Essa imagem acabou se tornando durante muito tempo a
representação dominante na memória popular e nas representações gerais
sobre o tema na Irlanda, até que estudos mais recentes, na segunda metade
do século~\versal{XX}, mudaram essa perspectiva ao demonstrar que os vikings foram
bem mais relevantes para a História da Irlanda do que se pensava
anteriormente, introduzindo novas formas econômicas, urbanas e mesmo
linguísticas, visto que o próprio nome ``Irlanda'' é de origem nórdica.

Em verdade, de acordo com tal mudança de perspectiva é notório que
durante os quase 400 anos da presença escandinava no território irlandês,
eles tenham se tornado algo bem mais relevante do que meros saqueadores,
transformando-se aos poucos em fazendeiros, comerciantes, colonizadores
e formadores dos principais assentamentos na região.

Existe uma progressão entre as diferentes levas e invasões escandinavas
ao longo dos séculos. As primeiras levas ocorrem ao final do século~\versal{VIII}, 
mais especificamente em 793, quando, de acordo com o registro
presente nos Anais de Ulster (\emph{Annála Uladh}), ocorre ``O
Incêndio de Rechru pelos estrangeiros [\emph{heathens}] e Scí foi
sobrepujada e deixada apodrecer''. Sendo ``Rechru'' a ilha de Rathlin,
localizada ao norte da costa do condado de Antrim, e ``Scí'' a ilha de
Sky, situada nas ilhas hébridas escocesas.

Nesse primeiro momento, os ataques eram pontuais e voltados diretamente
para o saque. Nos primeiros 25 anos após o registro citado acima
ocorria, em média, um ataque viking por ano, de acordo com os anais. A
preferência em geral era o ataque feito aos monastérios e igrejas, pois
estes continham ouro, prata e outros objetos que apresentavam certo valor
de troca. Isso sem contar que, por conta do respeito local à fé cristã,
tais mosteiros e igrejas eram extremamente mal preparados para se defender
de uma possível invasão.

Como ressaltam Liam de Paor e Michael Richter, grande parte dos registros
escritos que temos sobre esse primeiro período de invasões foi feita
pelos próprios membros do clero, que eram, em geral, as potenciais
vítimas, o que torna a imagem que temos desses ataques um tanto
quanto mais terrível do que possivelmente sua realidade material
comprova.

De qualquer forma, os ataques eram constantes nesses primeiros anos e
grande parte das principais obras de arte cristã irlandesa do período
hoje se encontra não no território irlandês, mas em museus
noruegueses.

Essa informação é interessante, pois notamos que, ao menos nesse período
inicial, grande parte das invasões feitas à Irlanda veio de grupos de
origem norueguesa e não dinamarquesa. A invasão de grupos de origem dinamarquesa só acontecerá na segunda
metade do século~\versal{IX}. Os vikings de origem norueguesa eram chamados, nesse
contexto, de Finn-gaill, enquanto os de origem dinamarquesa passaram
posteriormente a serem chamados de Dub-gaill. Tal diferenciação entre
os grupos de invasores mostra que estes não agiam juntos, mas 
em levas separadas. O norte era ligeiramente mais atacado que a
região sul, o que permitiu que os antigos reis da região de Munster, ao
sul, se desenvolvessem mais que os reis das regiões ao norte da ilha
nesse período inicial das invasões estrangeiras.

E falamos aqui estrangeiras, pois, para os cronistas que relataram as
primeiras incursões escandinavas à Irlanda, era isso que eles eram
inicialmente: estrangeiros. Os anais irlandeses demonstram que a
primeira leva de invasões vikings à Irlanda ocorrem entre 795 (saque à ilha Lambay) 
e o ano 830. Nesse período, cerca de 25 saques são
registrados com clareza em anais irlandeses, como os \emph{Anais de Ulster}
(Annála Uladh). Eles têm seu início na costa norte da ilha e foram oriundos de
regiões escocesas, como as ilhas Orkney e as hébridas. Aos poucos, foram se espalhando. 
Esse movimento inicial provoca uma certa movimentação
migratória, como a ocorrida a partir da ilha de Iona, que, atacada
algumas vezes, obrigou seus monges a migrarem, entre 807 e 814, para Kells,
na Irlanda, onde fundaram um novo mosteiro.

Por volta de 823, toda a costa leste irlandesa estava sob 
ataques dos vikings. Como esclarece Richter, é interessante notar que,
apesar de o foco dos vikings ser o ataque aos mosteiros e grupos religiosos em
geral, algumas comunidades relacionadas ao culto irlandês da \emph{Céli
Dé} (prática ascética de alguns religiosos irlandeses que se abstém de
ações mundanas) foram poupadas, muito provavelmente por serem
propositalmente pobres e não oferecerem aos invasores riquezas com bom
valor de troca.

A mobilidade viking na Irlanda era algo notável e, por conta de sua
exímia prática de navegação, eles se tornaram imbatíveis nas primeiras
décadas de seus ataques. Eram mais rápidos e melhor preparados para a
navegação, não apenas na costa, mas também nos rios que os irlandeses
utilizavam. Com isso, o resultado da primeira leva logo se intensifica e
os vikings, a partir de 840, iniciam a segunda leva de suas invasões, quando
passam a se assentar na ilha, criando fortificações que logo se tornariam
cidades.

Grande parte das maiores cidades costeiras da Irlanda hoje são parte de
assentamentos vikings. Cidades como Wicklow, Arklow,
Wexford, Waterford, Cork e Limerick são fruto desta época. O mais
importante desses assentamentos, Dublin, também foi o primeiro de todos.
Como bem demonstram os \emph{Anais de Ulster}, já no ano 841, ``(...)
havia um acampamento naval em Duiblinn, do qual os Laigin e os Uí Néill
foram saqueados, tanto as localidades quanto as igrejas até Sliab
Bladma''. Tal acampamento Naval de \emph{Dubh Linn} (lagoa negra) se
tornaria a cidade de Dublin e já possuía sua importância estratégica
muitos séculos antes de se tornar a capital da República da Irlanda.

Dublin era tão importante que, em 851,
seria palco de disputa entre grupos escandinavos, quando
dinamarqueses liderados por Ivar, o Desossado, disputavam
a região com os noruegueses liderados por Olavo, o Alvo. Essa
disputa permanece por décadas em meio a conflitos e acordos, até que, por
volta de 870, Ivar se torna o grande chefe supremo da região. À
ocasião de sua morte, ele é descrito pelos escribas dos anais de Ulster
como ``Rei dos Nórdicos de toda a Irlanda e Bretanha'', prova
inconteste da grande influência que estes líderes vikings, bem como seus
assentamentos, já exerciam na Irlanda ao final do século~\versal{IX}.

No entanto, tal influência se aproveita de um ambiente propício para
o assentamento, pois, ao longo do século~\versal{IX}, os próprios irlandeses
também se encontravam em conflito interno. Esse conflito explica,
inclusive, o motivo pelo qual eles não organizaram uma defesa mais
efetiva contra esses estrangeiros invasores que aos poucos foram
se assentando na ilha. A Irlanda gaélica do século~\versal{IX} era composta por
uma série de pequenos reinos e províncias organizados em um complexo
sistema de reinos maiores e menores correlatos por área de influência.
Tradicionalmente, tais reinos eram divididos em duas grandes partes:
\emph{Leth Cuinn}, dominado pelos \emph{Uí Neill} da região de Tara, e
\emph{Leth Moga}, dominada pelos \emph{Eóganachta} de Cashel.

Aproveitando esse conflito, os vikings ocupam justamente o espaço
sudeste da ilha, onde o domínio, tanto dos \emph{Uí Neill} quanto dos
\emph{Eóganachta}, não estava consolidado. O efeito disso na dinâmica
política e cultural irlandesa da época é notável. Afinal, as
consequências desses saques (que, futuramente, se tornariam assentamentos), em meio aos
conflitos internos irlandeses, promoveram um período de intensa pilhagem,
destruição e trânsito de objetos e relíquias religiosas, seja pela
própria razia ou pelo comércio. Independente do sensacionalismo com o
qual os cronistas cristãos descreveram esses acontecimentos, é inegável
que parte da cultura material irlandesa desse período hoje se encontra
em museus dinamarqueses e noruegueses, sobretudo nestes últimos.

Para além da perda desses objetos, também outro fluxo material é notado
por conta de tal contexto: o transporte e entesouramento de livros, que
por questões de segurança foram levados pelos monges irlandeses para outros lugares, 
se concentrando, principalmente, no continente europeu. Outros tantos
manuscritos não encontraram um fim tão seguro, claro. Muitos foram
entesourados de maneira no mínimo peculiar, em lagos ou mesmo no mar,
destruindo grande parte da produção intelectual do período.

No entanto, a vida seguiu seu rumo e outras inovações também foram
estimuladas pelo mesmo período inicial da presença viking na Irlanda. É
nesse momento que igrejas de pedra, construídas com argamassa, passam a
substituir as antigas construções de madeira. Além disso, também é desse
período a construção das primeiras torres de sino em pedra, altas e
arredondadas, que não apenas modificavam a arquitetura, mas também a
dinâmica dessas comunidades religiosas. Claro que estas torres também
possuíam um propósito secundário e de vital importância para o momento,
que era o de refúgio, motivo pelo qual a grande maioria dessas torres
arredondadas possuíam suas portas alocadas a uma boa distância do solo,
além do seu próprio sino servir não apenas para informar a passagem das
horas, mas também como alerta de invasões. Também nesse período ocorre o
desenvolvimento da escultura em pedra, que, por conta do fluxo religioso
para o continente e a troca cultural com o renascimento artístico
carolíngio, inspirou a formação de gravações em pedra, como cruzes e
pilares com um detalhamento ímpar.

É nesse contexto que uma nova leva de vikings começa a surgir, a partir
de 914, corroborando, assim, com uma renovação das invasões escandinavas
na região. Define o que se conhece hoje como o segundo Período Viking
da Irlanda.

Tal segundo período tem início quando, em 914, uma grande frota vinda
do continente atraca em Waterford, se estabelecendo lá. A partir desse
centro, os invasores passaram atacar a região de Munster, ao sul da ilha.
Nessa mesma época, eles também se assentam no estuário do rio Shannon,
criando o contexto de surgimento do que hoje conhecemos por cidade de Limerick.

Ainda que os cronistas mais tardios foquem muito na chegada das embarcações 
nessas regiões, tratando-as como um grande advento, percebemos que a grande
inovação nesse caso é de caráter econômico, pois é nesse momento que surgem as
grandes cidades voltadas para as práticas comerciais, dentre as quais a mais
importante é a própria Dublin.

Após a consolidação do poder nórdico, em 919 -- quando tais vikings
oriundos da região norte da França devastaram Munster, fundaram Limerick
em 922 e derrotaram o Rei de Tara \emph{Niall Glúndub} em batalha --,
Dublin se tornaria o principal foco do poder nórdico na Irlanda pelas
duas décadas seguintes. Os reis de Dublin se tornariam figuras políticas
importantes ao longo do século~\versal{X}, ainda que grande parte de suas
atividades envolvessem manter o controle da região da Nortumbria e obter
autoridade sobre os demais centros vikings na Irlanda.

Na primeira metade do século~\versal{X}, Dublin e York eram centros
bem conectados. Foram controlados pela mesma família 
até 952, quando Olavo Cuarán foi forçado a deixar York, retornando a
Dublin. Nesse momento, os irlandeses passaram a contra-atacar e as
atividades escandinavas foram se concentrando mais em Dublin e seus
arredores, que, nessa época, era já um centro de ofícios e de manufatura,
com um comércio bem desenvolvido.

O papel dos nórdicos entra em crise na região na segunda metade do
século~\versal{X}, principalmente após a derrota na batalha de Tara, em 980,
evento que marcou as chamadas guerras das grandes dinastias e que
decretaria o fim da Era Viking na Irlanda.

Nessa época, a região de Munster era controlada pela dinastia
irlandesa de \emph{Eóganachta}, com sua sede em Cashel. Seu monopólio
sobre a região se estende por um grande período, desde 820, apesar de
serem mais fracos militarmente que os Uí Néill, visto que não
conseguiram conter tão bem as incursões vikings em suas terras. Vale
lembrar que, nesse momento, as populações vikings já se encontravam
integradas à cultura irlandesa, muito por conta das alianças políticas e
da crescente conversão ao cristianismo.

No entanto, é a partir da região dominada pela dinastia de
\emph{Eóganachta}, mais especificamente na parte oeste, que
surgiu um novo grupo que mudaria a forma como a disputa entre irlandeses
e vikings era encaminhada até então.

Conhecidos, a partir de 934, como \emph{Dál Cais}, tal dinastia se
expande consideravelmente. Quando atacaram Limerick, em 967, ficou evidente que
seu rei, \emph{Mathgamain mac Cennétic}, possuía uma grande ambição.
Apesar disso, aquele que fundaria uma nova dinastia de grandes reis seria seu irmão, Brian Boru, que o sucedeu como rei
em 976.

Brian Boru é hoje considerado uma das figuras lendárias da história
irlandesa, muitas vezes elevado ao \emph{status} de ``herói nacional''.
Entretanto, ao contrário do que a memória coletiva popular preserva, seu
interesse não era o de expulsar os vikings da Irlanda, mas de se tornar
o grande rei da Irlanda, mesmo que, após a batalha de Clontarf, os dois
objetivos estivessem muito próximos um do outro.

Em verdade, a carreira de Brian Boru é marcada por um crescimento meteórico. Ele
consegue desafiar a supremacia Uí Néill e sair em comitiva real por toda
a Irlanda sem maiores resistências. Contra ele estavam apenas as regiões de
Leinster e Dublin. Nessa tensão, inclusive, o rei Sitric, de
Dublin, buscou ajuda de Jarl Sigurd, de Orkney, dos vikings da ilha
de Man e do rei de Leinster, Máel Mórda. Brian Boru conseguiu com
suas forças vencê-los na batalha de Clontarf (1014), registrada no texto
literário e propagandístico compilado no \emph{Cogadh Gáedhel re
Gallaibh}, o qual nos descreve bem a vitória de Brian Boru contra seus
opositores.

Na batalha de Clontarf, Brian Boru, apesar de sair vitorioso, acaba
morrendo e consolidando, desde então, o seu título de Imperador dos Irlandeses
(\emph{Imperator Scotorum}). Clontarf marca,
convencionalmente, o fim das guerras das grandes dinastias e,
principalmente, o fim da Era Viking na Irlanda, que transformou
completamente a vida dos irlandeses ao mudar a estrutura social e
política da ilha, antes centralizada nas disputas internas e agora
controlada pelos centros comerciais e portos na costa leste.

\SIG{Erick Carvalho de Mello}

Ver também Brian Boru; Celtas e nórdicos; Dublin.

\begin{itemize}
\item \versal{CONNOLLY}, Sean J. \emph{The Oxford companion to Irish History}. Oxford:
Oxford University Press, 1998.

\item \versal{DOWNHAM}, Clare. Irish chronicles as a source for inter-Viking rivalry,
\versal{A.D.} 795-1014. \emph{Northern Scotland}, 26, 2006. pp. 51-63.

\item \versal{DOWNHAM}, Clare. \emph{Viking Kings of Britain and Ireland}. Edinburgh:
Dunedin Academic Press, 2007.

\item \versal{DOWNHAM}, Clare. \emph{``Hiberno-Norwegians'' and ``Anglo-Danes'':
Anachronistic ethnicities in Viking-Age England}, Mediaeval Scandinavia, vol.
19, 2009, pp. 139-169.

\item \versal{DUFFY}, Seán. \emph{Brian Boru and the Battle of Clontarf}. Dublin: Gill
Books, 2014.

\item  \versal{Ó CUÍV}, Brian. Ireland in the Eleventh and Twelfth Centuries c.
1000-1169\emph{.} In: \versal{MOODY}, Theodore \versal{W.} \& \versal{MARTIN}, Francis \versal{X.} \emph{The
Course of Irish History.} Cork: Mercier Press, 2011, pp. 107-122.

\item \versal{MAC NIOCAILL}, Gearóid. \emph{The medieval Irish annals}. Dublin: Dublin
Historical Association, 1975.

\item \versal{PAOR}, Liam de. The Age of the Viking Wars: 9\textsuperscript{th} and
10\textsuperscript{th} centuries\emph{.} In: \versal{MOODY}, Theodore \versal{W.} \&
 \versal{MARTIN}, Francis \versal{X.} \emph{The Course of Irish History}. Cork: Mercier
Press, 2011, pp. 91-106.

\item \versal{RICHTER}, Michael. \emph{Medieval Ireland: The Enduring Tradition}.
Dublin: Gill and Macmillan, 1988.
\end{itemize}
\section{\versal{ISLÂNDIA DA ERA VIKING}}

A Islândia é uma ilha dentro do Oceano Atlântico Norte, a 950 km da Noruega, com uma área de
103 mil quilômetros quadrados.
Possui um relevo acidentado, com muitas montanhas e nascentes de
águas quentes formadoras de gêiser. Apesar de pequena, a ilha possui uma
grande quantidade de atividade vulcânica: um total de 30 sistemas
vulcânicos. As temperaturas chegam aos -3º \versal{C} no inverno e aos 8º \versal{C} no verão,
formando um contraste entre o fogo dos vulcões e o frio constante uma das
características marcantes.

A ilha foi o destino de um processo de emigração da península
escandinava, motivado pela centralização do poder do leste da península durante o
reinado de Haroldo Cabelos Belos. As principais narrativas existentes
sobre esse período são as \emph{Íslendingabók} (\emph{Livro dos Islandeses}) e a
\emph{Landnámabók} (\emph{Livro dos Assentamentos}). O número estimado de
colonizadores iniciais varia de 311 a 436.

A autarquia islandesa foi instaurada na Thingvellir, em 874, quando
Ingólfur Arnarson se assentou na região. Foi oficializada em 930, com o
estabelecimento da primeira \emph{Althing.} Durante o processo de
assentamento na ilha, as terras foram divididas entre os novos colonos. A
\emph{Landnámabók} menciona 1500 nomes de fazendas e outros locais, de modo que o número
final de emigrados varia entre 4300 e 24000 pessoas. O processo de
migração massivo foi reduzido após 930, contudo as
migrações para a ilha nunca cessaram totalmente.

A população que colonizou a Islândia falava o nórdico antigo, 
idioma predominante na região da Escandinávia medieval durante os
séculos~\versal{IX} e \versal{XIII}. Era uma língua oriunda do germânico
nórdico, cujas modificações deram origem ao
nórdico antigo do oeste e, no século~\versal{XIII}, o islandês
antigo.

A política se organizava em torno de assembleias locais chamadas 
\emph{Thing}. Tais assembleias estavam sob o domínio de chefes locais,
chamados \emph{godar} (plural de \emph{godi}), os quais também eram
responsáveis pelo culto aos deuses, bem como pela construção do local de culto, o \emph{hof}. As disputas e decisões importantes eram levadas para a assembleia geral, chamada
\emph{Althing}. Já o território no qual o \emph{godi} exercia sua liderança era chamado de \emph{godord}. 
Os \emph{thingmenn} (homens da assembleia) apoiavam o \emph{godi} em suas decisões e eventuais conflitos,
enquanto o \emph{godi}, por sua vez, protegia os interesses de seus seguidores. Com o passar do tempo, o papel exercido pelo \emph{godi} foi perdendo o seu caráter
religioso, até se tornar totalmente secular. 

A cultura islandesa medieval produziu um dos conjuntos documentais mais
importantes da Era Viking: as sagas. Também chamadas de \emph{Sagas
Islandesas}, são um conjunto de obras escritas em formato de prosa que abordam 
uma miríade de aspectos da sociedade e cultura islandesa. 
Versam sobre elementos antigos (sagas lendárias); os reis
noruegueses (sagas reais); conflitos familiares e vendetas (sagas de
família); a vida de bispos e santos (sagas dos bispos); e histórias de
cavalaria (sagas cavaleirescas).

Mas as produções escritas islandesas não se resumem apenas às Sagas.
Outros documentos importantíssimos foram compilados, como as \emph{Eddas} (em
prosa e poética), a \emph{Landnámabók} e a \emph{Íslendingabók}. A região
também foi um terreno fértil para as poesias escáldicas, caracterizadas pelo corte
escandinavo e pela métrica de \emph{dróttkvaett}. Tais poesias eram
marcadas pela constante presença de \emph{kenninga} (plural de
\emph{kenning)}, um tipo de perífrase que emprega linguagem figurativa
para substituir substantivos simples. Um exemplo seria a substituição de
``espada'' por ``mordedora de pés''.

O fim da Era Viking coincide com a conversão oficial da Islândia. Em 995,
iniciou-se o reinado de Olavo~\versal{I}, na Noruega, paralelamente à expansão do
cristianismo pela Escandinávia. A Islândia acabaria se convertendo na
\emph{Althing} de 999, quando, em meio à pressão do rei Norueguês, a ilha
se viu obrigada a trocar os seus costumes antigos pelos novos costumes
impostos pela coroa norueguesa. Apesar da conversão coletiva ter sido
decidida na \emph{Althing}, nem todos islandeses modificaram seus
costumes imediatamente, sendo necessário uma constante presença
episcopal na região para sedimentar o cristianismo nos locais.

\SIG{André Araújo de Oliveira}

Ver também Althing; Godi; Sociedade; Thing.

\begin{itemize}
\item \versal{HOLMAN}, Katherine. \emph{Histocial Dictionaries of the Vikings}. Oxford:
The Scarecrow Press Inc., 2003.

\item \versal{SIGURÐSSON}, Jón Viðar. Iceland. In: \versal{BRINK}, Stefan; \versal{PRICE}, Neil (eds.).
\emph{The Viking World}. New York. Routledge, 2008, pp. 571-578.

\item  \versal{VÉISTEINSSON}, Orri. \emph{The Christianization of Iceland: Priest,
Power and social change 1000-1300}. Oxford: Oxford University Press,
2000.
\end{itemize}
\section{\versal{ÍSLENDINGABÓK}}

O \emph{Livro dos Islandeses}, também chamado \emph{Íslendingabók} (em nórdico
antigo) e \emph{Libellus Islandorum} (em latim), é uma obra em prosa
composta pelo padre islandês Ari Thorgilsson, no início do século~\versal{XII}. A
obra possuía originalmente duas versões: a mais recente (que sobreviveu a
passagem do tempo) e uma versão antiga, que vinha com informações sobre os
reis noruegueses.

O \emph{Livro dos Islandeses} é considerado um documento de
confiabilidade histórica, haja vista que o autor não apresentou elementos
sobrenaturais em sua narrativa, bem como mencionou vários indivíduos
pelo nome, de modo que se pode verificar todas as informações nele contidas. 
A obra narra, por meio da história oral, os principais eventos da
história da Islândia até aquele momento.

A obra é dividida em dez capítulos. No primeiro, narra a
colonização da ilha e a história de um monge chamado Papar, que vivia antes na região.
O segundo capítulo versa sobre a constituição da legislação islandesa. O
terceiro capítulo discorre sobre como a \emph{Althing} foi estabelecida na
Thingvellir. O quarto capítulo aborda a definição do calendário
islandês. O quinto explica como se deu a criação das assembleias
de quadrantes. O sexto capítulo relata a descoberta e a colonização da
Groenlândia. O sétimo capítulo aduz sobre a cristianização da Islândia na
Althing de 999 e os três últimos capítulos, por fim, são dedicados à exposição da
história dos Bispos e Faladores-da-lei da ilha.

\SIG{André Araújo de Oliveira}

Ver também Althing; Thing; Godi; Islândia na Era Viking.

\begin{itemize}
\item \versal{HOLMAN}, Katherine. \emph{Histocial Dictionaries of the Vikings}. Oxford:
The Scarecrow Press Inc., 2003.

\item \versal{SIGURÐSSON}, Jón Viðar. Iceland. In: \versal{BRINK}, Stefan; \versal{PRICE}, Neil (eds.).
\emph{The Viking World}. New York. Routledge, 2008, pp. 571-578.

\item  \versal{VÉISTEINSSON}, Orri. \emph{The Christianization of Iceland: Priest,
Power and social change 1000-1300}. Oxford: Oxford University Press,
2000.
\end{itemize}
\chapter{J \textarn{j} \textarc{j} \textart{j}}
\section{\versal{JELLING}}

Jelling localiza-se na região sudeste da atual Dinamarca, no condado de
Vejle. Durante o Período Viking, o local abrigou uma das realezas que
disputavam o poder dos territórios a oeste de Store Baelt. Tais territórios
eram, no período em questão, considerados parte da Dinamarca, ainda que hoje, com as atuais demarcações geográficas, não o sejam. Os monumentos de Jelling foram
erguidos pelo rei Gorm, o velho, e por seu filho, o rei Haroldo Dente Azul.
No local encontram-se vestígios arqueológicos de um grande
navio de pedra, dois grandes montes e duas \emph{pedras rúnicas}. A região é caracterizada pela presença de
monumentos tanto do período pré-cristão como 
do período cristão, o que possibilita demarcar a fase de transição da antiga
religião nórdica para o cristianismo.

A maior \emph{pedra rúnica}, erguida por Haroldo Dente Azul, contém três principais
celebrações: a primeira corresponde à conquista da Dinamarca e da Noruega; a
segunda ao processo de cristianização promovido pelo rei na Dinamarca; e a
terceira rende uma homenagem aos pais de Haroldo Dente Azul (Gorm, o velho, e a rainha
Thyra). Além das celebrações, tal pedra rúnica ainda contém duas grandes
imagens. Na primeira, pode ser observado o Cristo em posição de
crucificação, enredado em troncos de árvore, o que é apontado como um
paralelo ao poema \emph{Rúnatal}, presente na \emph{Edda poética},
na qual encontramos o mito de Odin enforcado em uma árvore. Na segunda imagem, encontra-se 
uma serpente se enrolando em outro animal, apontado por
arqueólogos como sendo um leão. Por sua vez, a menor \emph{pedra rúnica} foi
erguida pelo rei Gorm, o Velho, em homenagem à sua mulher, a rainha
Thyra. As duas \emph{pedras rúnicas} encontram-se hoje no lado sul da atual igreja
de Jelling.

A embarcação de pedra é apontada como tendo uma extensão que cobria a distância entre os dois
montes da região. Contava com cerca de 170 metros de comprimento, sendo a maior
embarcação de pedra localizada na Escandinávia. Um dos montículos foi
erguido sobre um antigo depósito funerário da Idade do Bronze. Acredita-se que em tal monumento
foi enterrado o rei Gorm, o velho, sendo o depósito
datado por estudos dendrocronológicos como sendo do ano de 959. O corpo de Gorm,
o velho, foi retirado do montículo norte ainda durante o século~\versal{X} e
reenterrado em uma sepultura sob a Igreja de Jelling. No montículo foram
encontrados apenas os bens depositados no momento do rito funerário como,
por exemplo, os armamentos do antigo rei. A mudança de localidade para o
depósito do corpo de Gorm pode ser interpretada como uma tentativa de
desvincular da imagem do rei a antiga religião pré-cristã, bem como de
buscar uma nova identidade cristianizada, provavelmente promovida
por seu filho Haroldo.

Ao sul do grande montículo do rei Gorm encontra-se outro montículo, 
apontado como local de deposito da rainha Thyra. No entanto,
escavações arqueológicas encontraram o monumento sem nenhum depósito
funerário, fato que ensejou um grande debate no interior da
historiografia de Jelling. O montículo sul foi escavado em 1992. Embaixo dele 
foi encontrada uma das pontas da embarcação que residia 
entre os dois montes. A outra ponta da embarcação
encontrava-se sob o montículo norte. O montículo sul foi datado por
métodos dendrocronológicos como sendo dos anos 970 e o líquen presente nas
pedras da embarcação abaixo dele sugeriu que estas haviam passado
cerca de 30 anos expostas antes da construção do montículo. Tal fato indicou 
uma conexão entre a embarcação e o montículo norte, posteriormente 
reorientada para o montículo ao sul.

O debate aumentaria com a indicação de uma embarcação em pedra
demarcando um depósito funerário abaixo de um montículo na região de
Baeke, datado também para 970. No depósito de Bake, uma \emph{pedra rúnica}
apontaria Thyra como esposa de Tue, o qual havia sido o responsável pelo
depósito da rainha. Arqueólogos atualmente apontam que, possivelmente, Thyra
havia se casado primeiro com Gorm e depois com Tue, sendo 
as duas construções monumentos rivais conclamando uma linhagem
que legitimaria o poder na região. A rivalidade estabelecida entre
Haroldo e Tue após a morte da rainha seria o elemento que poderia
explicar o monte vazio construído como elemento mnemônico de Thyra,
mesmo sem a presença de seu corpo. Buscando reforçar os laços que ela
teve com o rei Gorm -- reforço ideológico que seria conclamado com a
construção da grande \emph{pedra rúnica} de Haroldo --, o filho da antiga rainha
conectou suas linhagens aos seus pais e passou a fazer frente ao
último marido de sua falecida mãe.

\SIG{Munir Lutfe Ayoub}

Ver também Arqueologia da Era Viking; Cultura material; Dinamarca da Era
Viking.

\begin{itemize}
\item \versal{HOLST}, Mads Kähler \emph{et al}. The Late Viking-Age Royal Constructions
at Jelling, central Jutland, Denmark.~\emph{Praehistorische
Zeitschrift}, vol. 87, n. 2, 2013, pp. 474-504.

\item \versal{KROGH}, Knud J. The Royal Viking-Age Monuments at Jelling in the Light of
Recent Archaeological Excavations: A Preliminary Report.~\emph{Acta
Archaeologica}, vol. 53, 1982, pp. 183-216.

\item \versal{MOLTKE}, Erik. The Jelling monument in the light of the runic
inscriptions.~\emph{Mediaeval Scandinavia}, vol. 7, 1974, pp. 183-208.

\item \versal{SAWYER}, Birgit; \versal{SAWYER}, Peter. A Gormless History? The Jelling dynasty
revisited. In: \versal{HEIZZMAN}, Herausgegeben von Wilhelm; \versal{NAHL}, Astrid van
(eds.). \emph{Runica Germanica Mediaevalia}. Berlim: Walter de Gruyter,
2003, pp. 689-706.
\end{itemize}
\section{\versal{JOGOS E ESPORTES}}

Os vikings possuíam distintas formas de lazer, desde jogos de tabuleiro
até a prática de alguns esportes, como corrida e lutas. Pelo fato de
haver poucas cidades na Escandinávia, grande parte da população viking 
era rural, ainda que alguns, em função de suas viagens por outras terras, 
eventualmente adotassem outros costumes. 

Não obstante o estereótipo para o qual os vikings seriam um povo excessivamente 
bruto que se divertia exclusivamente pela violência, na Escandinávia a vida girava 
em torno do lar, fossem as casas de campo, as casas urbanas ou os salões. O 
cotidiano se desenvolvia no espaço residencial, sobretudo no inverno, que, 
dependendo da região da Escandinávia, podia ser bem longo. A realização de 
expedições de pirataria e invasão eram sazonais, dependendo da época. Os 
homens não ficavam todo o tempo treinando ou lutando. Realizavam outros 
afazeres, que incluíam o ócio. O mesmo ocorria com as mulheres, as quais 
também dispunham de momentos de lazer. As crianças nórdicas possuíam até mesmo brinquedos.

Pouco se sabe a respeito do passatempo e das brincadeiras das crianças nórdicas, 
uma vez que quase nada foi escrito sobre a infância desses povos.
Os brinquedos, conhecidos graças às
descobertas arqueológicas, geralmente eram feitos
de madeira e osso, sendo alguns de metal. Eram miniaturas de animais, frequentemente de cavalos. 
No entanto, foram encontrados também
barquinhos, bonecos, e espadas de madeira. 
Estas eram usadas pelos meninos para brincar ou receber treinamento, mas, 
possivelmente, também serviam como objetos de decoração.

Além do uso de brinquedos, as crianças também jogavam jogos de tabuleiro,
dançavam, cantavam, aprendiam a tocar instrumentos musicais, bem como
realizavam outras atividades consideradas como passatempo. As
meninas, por exemplo, desde cedo eram instruídas na arte da costura e fiação. Os
meninos, por sua vez, poderiam aprender a lutar, caçar e realizar alguns afazeres domésticos.

Em razão da falta de registros, se desconhece quais seriam exatamente as brincadeiras de criança. 
Leszek Gardela comenta, porém, que
as crianças da época, tais como as de hoje em dia, poderiam fazer uso de diferentes
tipos de objetos para suas brincadeiras: pedras, galhos, barro, ossos, neve etc.

Um passatempo compartilhado pelas crianças e adultos eram os jogos de
dados e os jogos de tabuleiro. Os dados (\emph{teningar}) não eram
necessariamente cúbicos, mas oblongos. Normalmente, os jogos com dados
envolviam algum tipo de aposta. Outros jogos faziam uso de tabuleiros
(\emph{tafl}), dentre os quais dois se destacam: o (\emph{Halatafl}), semelhante ao 
atual Três em linha; e o \emph{Nine
men's morris}, que lembra o Raposa e os Gansos. 
Ambos são jogos de estratégia simples, sendo o tabuleiro conformado por
espaços e linhas e o objetivo obter as peças do
adversário. Um dos jogos de tabuleiro mais popular entre os vikings foi
o \emph{Hnefatafl} (``Tabuleiro do rei''), que apresenta dinâmica similar a dos jogos
anteriores, embora utilizasse uma peça diferente, representando o
rei, como no xadrez.

O \emph{hnefatafl} foi mencionado em algumas sagas, como na \emph{Saga de
Hervör} e na \emph{Saga do rei Heiðrekr}. As regras do jogo em sua versão original não são
propriamente conhecidas, mas hoje ele é jogado com outras
regras. Sabe-se que o \emph{hnefatafl} era jogado por duas pessoas. Um
dos jogadores controlava o rei e seu exército, o qual possuía a
obrigação de defender o monarca. O rei ficava situado no centro do
tabuleiro, rodeado por seus soldados, enquanto o inimigo se situava nas
bordas.

É possível que houvessem outros jogos de tabuleiro, além dos daqueles por nós conhecidos. 
Não obstante, os vikings também
praticavam atividades ao ar livre. Algumas dessas atividades estavam
relacionadas ao treinamento militar e à educação física. Os nórdicos
praticavam uma luta chamada \emph{glima} (similar à luta
greco-romana), cujo objetivo era derrubar o adversário ao chão,
valendo-se de movimentos de pura força. Tal luta poderia ser parte do
treinamento ou praticada apenas para diversão, o que envolvia
a elaboração de apostas. A \emph{Grettis saga Ásmundarsonar} menciona a organização de
uma \emph{glima} pelos irmãos Hjalti e Thorbjorn, na Islândia, durante
uma assembleia de verão.

Durante a primavera, o verão e até mesmo o outono, os vikings também
praticavam natação, que se tornava prática ainda mais corriqueira em terras 
de clima mais ameno. No poema \emph{Beowulf} (c. 1000),
o herói é descrito como um exímio nadador. Beowulf narra como suas
façanhas ter nadado quilômetros em mar aberto. Na \emph{Saga Laxdæla}
é mencionada uma competição de natação no rio Nid, na Noruega.

Outro esporte era o tiro ao alvo, praticado seja com arco e
flecha (forma mais comum), seja através de arremesso de machado ou de dardo. As
sagas, ao cantarem as habilidades dos heróis, descrevem alguns como exímios
arqueiros, outros como verdadeiros
atletas. Na \emph{Saga Njáll}, o herói Gunnar de Hliðarendi é descrito
como um exímio arqueiro, esquiador, lutador e corredor
veloz, capaz de saltar a própria altura.

De fato, algumas sagas relatam personagens participando de competições
de nado, corrida, luta, salto etc. Gardela comenta que corridas de esqui 
podem ter ocorrido com frequência na Noruega e Suécia, ou, ao menos, 
o uso de esqui para passeio. O deus Ullr, divindade pouco influente no panteão
nórdico, era descrito como usuário de esquis. A giganta Skadi, esposa do
deus Njörd, também foi mencionada como usuária de esquis. Talvez o uso destes
estivesse mais voltado para a locomoção ou a caça e não
necessariamente para o esporte.

Somando-se a tais esportes, haviam os jogos de bola (\emph{knattleikr}),
os quais, segundo Régis Boyer, lembrariam o beisebol ou o críquete. Nesses
jogos, os membros de uma equipe teriam que arremessar uma pequena bola um para outro, 
evitando que os adversários obtivessem a bola. As regras
dos \emph{knattleikr} não são conhecidas, ainda que brevemente
mencionados em algumas sagas. Sabe-se que algumas partidas
terminaram em briga, como no caso relatado na \emph{Egils saga
Skalla-Grímssonar}.

Pelo que as fontes sugerem, os jogos de \emph{knattleikr} eram
considerados esportes de força e impróprios para as mulheres, sendo
praticados por crianças e adultos apenas do sexo masculino. Pouco se 
sabe sobre o direito das mulheres de praticar
esportes, embora se saiba que, além do \emph{knattleikr}, a \emph{glima}
também era praticada apenas por homens.

Além dessas atividades, encontravam-se também a
falcoaria e a caça, mas eram restritas à elite. Por mais que a caça consistisse num ato de
sobrevivência, povos em diferentes épocas a praticavam como esporte
e lazer. No caso dos escandinavos, isso não foi diferente. Mas, além da
caça, havia também uma atividade singular, a rinha de cavalos
(\emph{hestaat} ou \emph{hestavlg}). Cavalos mais agressivos eram
incitados a lutar num cercado.

Todavia, uma característica curiosa do combate de cavalos é que, em
alguns casos, estes usavam chifres como forma de causar maior dano
durante o combate. Outra medida para deixar os animais agressivos era
introduzir uma égua no cio para forçar dois garanhões a lutarem como se 
estivessem disputando a fêmea.

A rinha de cavalos foi uma atração bem popular na Islândia, conforme atestam
sagas como a \emph{Brennu-Njáls saga} e \emph{Grettis saga
Ásmundarsonar}. O evento atraia pessoas de todas as partes da ilha
para participar ou assistir. Na Noruega, o \emph{hestavlg} também foi
popular, mas acabou sendo proibido mais cedo graças ao cristianismo, que
considerava tal luta comportamento bárbaro.

\SIG{Leandro Vilar Oliveira}

Ver também Cotidiano; Cultura material; Hnefatafl; Sociedade.

\begin{itemize}
\item \versal{BELL}, Robert C. \emph{Board and table games from many civilizations}.
New York: Dover Publications Inc., 1979. pp. 75-78.

\item \versal{BOYER}, Régis. \emph{La vida cotidiana de los vikingos: 800-1050}.
Barcelona: José J. de Olañeta, Editor,2000, pp. 223-229.

\item \versal{GARDELA}, Leszek. What the Vikings did for fun? Sports and pastimes in
medieval northern Europe. \emph{World Archaeology}, vol. 44, n. 2, 2012,
pp. 234-247.

\item \versal{GRAHAM-CAMPBELL}, James (org.). \emph{Os viking}s. Barcelona: Folio \versal{S.A.},
2006, pp. 64-65.
\end{itemize}
\section{\versal{JOIAS E OURIVESARIA}}

As joias sempre foram um adorno muito apreciado tanto por homens como
por mulheres de todas as camadas da sociedade durante a Era Viking. Assumiam
a forma de pulseira, braceletes, colares e broches. Algumas joias eram
usadas como simples ornamento e poderiam indicar o \emph{status} social de
quem as usava. Outros itens, como broches, tinham, além da função
estética, uma função prática de prender roupas. Além disso, haviam joias
com valor simbólico, como os martelos de Thor, que além de serem
objetos estéticos, ainda tinham a função de representar as
crenças daqueles que os usavam. As joias poderiam ser elaboradas com
vários materiais, como madeira, vidro, âmbar, bronze, ouro e prata. As
peças de joalharia eram muitas vezes decoradas com desenhos geométricos,
entrelaçados elaborados e cabeças de animais.

Aparentemente, os nórdicos, de início, não usavam brincos, ainda que conhecessem
essa peça. Com o tempo, principalmente as mulheres passaram
a usar brincos cotidianamente e eles passaram a ser confeccionados com uma
grande riqueza de detalhes, a partir da utilização de materiais como a prata e contas
de vidro. Acredita-se que os brincos passaram a ser confeccionados e utilizados
depois que foram comercializados em
expedições às terras eslavas, pois nesses locais os brincos já eram
usados frequentemente. Essas peças eram ricamente elaboradas a partir da utilização de
uma técnica de ourivesaria chamada filigrana, que consiste em entrelaçar
e depois soldar de maneira extremamente delicada finíssimos fios de
ouro e prata, podendo ou não formar desenhos de formas geométricas, de
animais, flores ou mesmo apresentar um efeito rendado.
A técnica da filigrana também era utilizada na elaboração dos vários
tipos de broches, colares, anéis contas e demais joias. Assim como
a confecção de pentes exigia artesãos com técnicas especializadas, o
mesmo ocorria com a elaboração das joias. A ourivesaria exigia não só
mão de obra especializadas, como também artesãos que conhecessem as
melhores ligas metálicas que garantiriam mais durabilidade e
beleza à peça. Muitos ourives também dominavam a técnica de trabalhar com o
vidro e elaboravam contas coloridas usadas nos colares e brincos.
As contas de cor azul foram encontradas em alguns túmulos, o que pode indicar
que essa cor estava relacionada aos funerais. Além de dominar todas
essas técnicas, os ourives também deviam conhecer e trabalhar
muito bem o âmbar, que era muito apreciado, bem como as pedras preciosas
utilizadas em menor escala e que aparecem em joias muito raras. Estas ou pertenciam 
aos mais abastados, ou poderiam ser produtos de
transações comerciais, ou ainda originárias de algum saque realizado em
terras estrangeiras.

Alguns torques e colares eram elaborados com fios de prata trançada, que 
conferiam resistência e beleza, bem como
atribuíam às peças a condição de símbolos do \emph{status} social de quem as
usava. O mesmo pode ser dito dos broches usados pelas mulheres para
segurarem as alças de seus \emph{upon-dress}. As mulheres mais ricas usavam
broches de prata ricamente trabalhados e, ao lado de seus colares feitos
com contas de vidro, âmbar, conchas raras e metais preciosos, podiam
ostentar sua riqueza com tais objetos. O molho de chaves que descia
pelo \emph{upon-dress}, seguro por uma corrente de metal trabalhado, também
era trabalho de ourivesaria e outro símbolo de poder e \emph{status}
feminino.

Podemos inferir que a arte da ourivesaria na Era Viking recebeu
influências da ourivesaria celta da Irlanda, do mundo eslavo e também de
Bizâncio, que, além de fornecer matérias primas que não eram encontradas
na Escandinávia, também foi fundamental na difusão de técnicas mais
complexas para trabalhar o metal. Tais técnicas conferiram a ourivesaria nórdica
uma grande singularidade quando comparada às outras ourivesarias da
época.

\SIG{Luciana de Campos}

Ver também Cotidiano; Ferreiros e ferraria; Metalurgia; Mulheres.

\begin{itemize}
\item \versal{ARMBRUSTER}, Barbara R. Tools and techniques of Viking age goldsmith´s.
\emph{\versal{I} Symposium Internacional sobre tecnologia del oro antiguo}.
Madrid: Europa y América, 2002.

\item \versal{GRAHAM-CAMPBELL}, James. \emph{The Cuerdale Hoard and related Viking-Age
silver and gold from Britain and Ireland in the British Museum}. London:
British Museum Research Publication no. 185, 2013.

\item \versal{HAYEUR SMITH}, Michèle.~\emph{Draupnir's Sweat and Mardöll's tears: An
Archaeology of Jewelry, Gender and Identity in Viking Age
Iceland.}~British Archaeological Reports, John and Erica Hedges Ltd \&
Archaeopress, Oxford, 2004.
\end{itemize}
\section{\versal{JORVIK}}

Jorvík era o nome em nórdico antigo pelo qual os vikings se referiam à
cidade inglesa de York. Na segunda metade do século~\versal{IX}, York foi
conquistada pelos dinamarqueses, tornando-se por vários anos a capital
de um reino viking. A cidade de York foi fundada no ano 71 pelos
romanos, originando-se a partir do forte Eboracum, localizado na
confluência dos rios Ouse e Foss. Uma povoação surgiu em
torno do forte romano. Localizada a 60 km da costa, a cidade possui
ligação com o mar através do rio Ouse, que deságua no estuário Humber.
Isso permitia que embarcações pudessem facilmente aportar em York e
depois se dirigir para o mar.

A história de York entre os séculos~{\versal{V} e \versal{VII}} é desconhecida, pois marca a
época da saída dos romanos da Bretanha e a chegada dos saxões a ilha. No
começo do século~\versal{VII}, Eoforwic (como era referida nas fontes da época)
era uma cidade de pequeno porte variando entre mil a dois mil
habitantes. Nessa época, a cidade contava com uma catedral e um
seminário. Era referida como um dos principais epicentros religiosos
do Reino da Nortúmbria e se tornou arquidiocese em 735. Um dos
mais proeminentes estudantes do seminário de York foi o monge Alcuíno
(735-804), conhecido por ter servido na corte do imperador Carlos Magno.

No século~\versal{VII}, York, que ainda era conhecida pelo nome saxão de Eoforwic,
já apresentava um mercado de produtos artesanais e agrícolas, embora se
desconheça sua importância econômica para a região. A história daquela
pequena cidade mudaria drasticamente no século~\versal{IX} com as invasões dos
dinamarqueses (daneses). No ano de 866, o Grande Exército Danês
aportou na costa da Ânglia Oriental e, a partir dali, marchou até a
Nortúmbria, levando o então rei Ælla~\versal{II} a se defender. Com a morte do
rei da Nortúmbria, a região foi capturada pelos daneses no ano seguinte.
Eoforwic passou, então, a ser chamada de Jorvik.

Após a tomada de Jorvik pelos vikings, estes decidiram colocar no trono
um ``rei marionete'', elegendo um nobre local de nome Ecbert, o qual
governou até 872, quando foi assassinado. Em seu lugar, assumiu Ricsige,
que, por sua vez, permaneceu cerca de quatro anos no trono, sendo
substituído por Halfdan, suposto filho de Ragnar Lothbrok.

Em 878, a Nortúmbria passou a fazer parte do Danelaw (nome dado aos
territórios ocupados pelos daneses), constituindo o reino do norte desse
território. Até 954, o Reino de Jorvik
foi disputado pelos saxões, dinamarqueses, noruegueses e irlandeses. De
fato, houve alternâncias de poder nesse período, quando, em dados momentos,
reis saxões governaram brevemente, até serem depostos por senhores
vikings.

Todavia, não obstante tais intrigas políticas, quando Jorvik esteve
sob governo nórdico, este favoreceu a cidade, pois favoreceu sua participação no comércio
escandinavo. Katherine Holman observa que Jorvik tornou-se um importante
polo manufatureiro. Tal importância podia ser verificada no nome das ruas, 
já que algumas delas faziam referência aos
produtos ali fabricados. Assim, encontramos ruas cujos nomes se referiam
a produção de vidro, metal, joalheria com âmbar, prata e ouro,
carpintaria, tecelagem etc. Uma das ruas mais conhecidas era a chamada
rua Coppergate, escavada entre 1976-1981.

No caso da rua Coppergate, seu nome remetia ao fato de concentrar um
grande número de artesão de objetos em cobre, sobretudo fabricantes
de copos, taças, pratos etc. Richard Hall comenta que a grande
quantidade de material encontrado em Coppergate pode indicar que
a cidade fazia parte de alguma rota comercial para exportação de tais
utensílios. Mas, além de tais produtos em cobre e madeira, os artesãos de
Jorvik também eram especializados na produção de facas, ferramentas,
acessórios de vestuário, joalheria, calçados, roupas, móveis etc.

Richard Hall salienta, ainda, que, durante o governo escandinavo, York
voltou a ser um importante centro comercial, exportando produtos para a
Irlanda, sobretudo para Dublin (cidade ocupada pelos vikings) e 
outros mercados do Danelaw, além da região da Escandinávia e talvez até em áreas mais distantes. Não
obstante, a cidade também recebia mercadorias estrangeiras, como seda
bizantina, vinhos germânicos e peças de âmbar advindas do Báltico,
bastante requisitadas para a joalheira.

O comércio prosperou a tal ponto que alguns monarcas passaram a cunhar
suas próprias moedas, como foi o caso de Érico Machado Sangrento, o
último rei viking de Jorvik. Após seu assassinato, em 954, os saxões
retomaram o controle da cidade, a qual manteve sua produção manufatureira 
pelo restante do século e até mesmo acordos comerciais com os
daneses por certo tempo.

No ano de 1066, data da conquista normanda da Inglaterra, a população de
York era estimada em 15 mil habitantes, consistindo na segunda maior
cidade do país. Tal crescimento vertiginoso em parte se deveu à
prosperidade econômica e à segurança que a cidade dispôs nos duzentos anos anteriores.

Mas todas essas informações somente foram possíveis graças à
arqueologia. O solo lamacento da cidade contribuiu para preservar
objetos e vestígios de mais de mil anos atrás. A partir das escavações
realizadas na moderna York, não apenas se coletou um grande número de
artefatos da Era Viking, mas também se encontrou vestígios de casas e de
ruas. Jorvik foi um aglomerado urbano com casas de madeira (uma ao lado da
outra), cujos telhados eram revestidos com palha para ajudar no
aquecimento durante o inverno. Além disso, as casas possuíam quintais,
nos quais as pessoas podiam cultivar hortas e criar animais como
galinhas, porcos e ovelhas.

As ruas eram, em sua maioria, de terra batida, ainda que algumas fossem pavimentadas com troncos.
A cidade possuía muros e postos de guarda. Seu porto foi ampliado para
comportar a grande quantidade de embarcações que ali chegavam, o que também se devia ao fato
de a cidade ter se tornado não apenas um polo manufatureiro, mas um centro de
escoamento de matéria-prima. As fazendas e vilas nos arredores forneciam
seus produtos à cidade, como também adquiriam os produtos dela. Naquele
tempo, a Inglaterra era uma fabricante de lã considerável.

\SIG{Leandro Vilar Oliveira}

Ver também Canuto, o Grande; Comércio; Inglaterra da Era Viking.

\begin{itemize}
\item \versal{HADLEY}, Dawn M. \emph{The northern Danelaw: its social structure: c.
800-1100}. New York: Leicester University Press, 2000.

\item \versal{HALL}, Richard. York. In: \versal{BRINK}, Stefan; \versal{PRICE}, Neil (eds.). \emph{The
Viking World}. London/New York: Routledge, 2008, pp. 379-384.

\item \versal{HALL}, Richard. \emph{Exploring the World of the Vikings}. London: Thames
\& Hudson, 2007.

\item \versal{HOLMAN}, Katherine. \emph{Historical dictionary of the vikings}. Lanham:
Scarecrow Press Inc, 2003.
\end{itemize}


\chapter{K \textarn{k} \textarc{k}}
\section{\versal{KAUPANG}}

Kaupang é uma das mais antigas regiões urbanas na Escandinávia. Localiza-se no condado
de Vestfold, atual Noruega, sendo fundada por
volta de 800 e abandonada em meados do século~\versal{X}. A primeira fonte que 
apresenta um relato sobre Kaupang consiste nos relatos do viajante norueguês
Ohthere ao rei Alfred, do reino anglo-saxão de Wessex. Ela está presente na
tradução do livro de Paulus Orosius para o inglês antigo, intitulada
\emph{Historiarum Adversum Paganos Libri} e cuja versão
anglo-saxônica é datada do século~\versal{X}. No relato de Ohthere, a região é
denominada Sciringes heal, mas alguns arqueólogos, como Dagfinn Skre, 
apontam a localidade como correspondente à suprarregião de Skiringssal, mencionada nas sagas
\emph{Ynglinga Saga} e \emph{Fagrskinna} como sendo uma suprarregião
conectada às origens da realeza norueguesa, da qual faria parte a
localidade de Kaupang. A suprarregião de Skiringssal é apontada como
sendo a atual municipalidade de Tjolling. A última fonte que aponta essa
conexão foi um registro hospitalar de 1445, que, emitido em Tonsberg,
apresenta o nome Skirisall, reforçando a conexão da suprarregião com a
antiga paróquia de Tjolling, da qual Tonsberg fazia parte.

Skiringssal, por sua condição de suprarregião, tornou-se uma das
localidades centrais, nas quais se mesclavam
as atividades de produção de artefatos, comércio, residência aristocrática,
realização de rituais e reuniões nas \emph{Things}. Por apresentarem tal aglutinação 
de atividades variadas, os locais centrais se tornaram pontos estratégicos para escavações
arqueológicas. Especificamente em Skiringssal, as escavações encontraram um local 
de assembleia, um lago sagrado, um salão, um grandioso cemitério e, ainda, uma cidade marcada
por sua produção manufatureira e seu comércio, além da descoberta de
artefatos que demonstram a conexão do mundo viking com os mundos árabe e franco.

As escavações de Kaupang se deram em momentos diferentes e realizadas por
arqueólogos com métodos diversos de escavação e registro. O
primeiro desses momentos é marcado pela escavação de Nicolay Nicolaysen,
que, em 1867, explorou os depósitos funerários de Nordre Kaupang. Foi seguido
por Charlotte Blindheim, que iniciou suas escavações em 1947 e fez sua
última publicação sobre a região em 1999, inaugurando não apenas a
exploração de diversos cemitérios de Kaupang (dentre estes, os cemitérios
de Hagejordet, Sondre kaupang, Lamoya e Bikjholberget), mas também
iniciando a exploração das regiões de assentamento de Kaupang que não
haviam sido exploradas anteriormente. As últimas pesquisas e escavações
da região contaram com a liderança do arqueólogo Dagfinn Skre. Iniciaram
suas campanhas em 1997 e sua última publicação, sob a liderança da
arqueóloga Unn Pedersen, ocorreu em 2016.

As escavações de Kaupang revelaram um total de quatro possíveis
habitações e delimitaram seus locais de queima, bem como os poços de instalação
de seus postes de sustentação. No que se refere aos artefatos, revelou 
a presença de moedas árabes, moedas de ouro de Dorestad, centenas de contas de
vidro, joias de ouro e bronze, cerâmica, armamentos e muitos artefatos
provenientes da produção manufatureira. O surgimento de locais centrais 
no mundo escandinavo, como Skiringssal, é uma das marcas do período
viking, demarcado, dentre outros fatores, por um crescente número de
regiões especializadas que representavam a intensificação do comercio e
da manufatura.

Os locais centrais possibilitam, por conseguinte, o estudos das mudanças 
na estrutura geográfica, que, por sua vez, revelam a intensificação 
de atividades que, anteriormente, ocorriam em escalas
reduzidas dentro das fazendas, a fim de suprir necessidades locais, ou
mesmo de maneira não fixada, como a atividade dos
ferreiros. Estes, por exemplo, após a intensificação do comércio, deixaram de 
circular pelas mais diversas regiões e passaram a se fixar em determinados locais
especializados, os quais aglutinavam os mais diversos personagens sociais:
aristocratas, ferreiros, tecelões, vidreiros e comerciantes. Tal mudança
levou, inclusive, à alteração da forma de produção
manufatureira, que deixou de ser voltada para produção de artefatos únicos
e satisfação de necessidades especificas para
adaptar-se a uma produção em serie de itens idênticos, que seriam
distribuídos não apenas nas localidades de imediação, mas de modo
a atingir e prover uma circulação de bens que atingiam regiões cada vez
mais distantes. Demandava-se uma produção capaz de suprir a alta da demanda 
gerada pelo progressivo arrefecimento do antigo processo de produção de circulação da mão de obra.

\SIG{Munir Lutfe Ayoub}

Ver também Arqueologia da era Viking; Noruega da Era Viking;
Escandinávia; Viking.

\begin{itemize}
\item \versal{PEDERSEN}, Unn.~\emph{I smeltedigelen: finsmedene i vikingtidsbyen
Kaupang}. Tese de Doutorado. Institutt for arkeologi, konservering og
historie, Det humanistiske fakultet, Universitetet i Oslo, 2010.

\item \versal{PRICE}, Neil. Mythic Acts: Material Narratives of the Dead in Viking Age
Scandinavia. In: \versal{RAUDVERE}, Catharina; \versal{SCHJODT}, Jens Peter (eds.).
\emph{More Than} \emph{Mythology. Narratives, Ritual Practices and
Regional Distribution in Pre-Christian Scandinavian Religions}. Lund:
Nordic Academic Press, 2012, pp. 13-46.

\item \versal{SKRE}, Dagfinn. \emph{Means of Exchange: Dealing with Silver in the
Viking Age}. Aarhus:
Aarhus University Press, 2008, (Kaupang Excavation Project Publication Series, vol. 2).

\item \versal{STYLEGAR}, Frans-Arne. The Kaupang Cemeteries Revisited. In: \versal{SKRE},
Dagfinn (ed.). \emph{Kaupang in Skiringssal}. Aarhus: Aarhus University Press, 2007, pp.
65-126. (Kaupang Excavation Project
Publication Series, vol. 1).
\end{itemize}
\section{\versal{KENNING}}

Define-se \emph{kenning} como uma perífrase que substitui uma
denominação simples por uma outra muito mais complexa, com mais palavras e
elementos de expressão. O \emph{kenning} é o elemento mais
complexo da poesia escáldica.

Com relação a sua utilização, Simek \& Pálsson afirmam
que a ocorrência do \emph{kenning} é mais registrada na poesia de elogio, 
enquanto na poesia heroica e éddica ela é usada mais esparsamente. Além de ser uma
variação ornamental da expressão poética, o \emph{kenning} também
enfatiza no texto a passagem por meios poéticos e pela integração do
passado, sempre com componentes mitológicos, que não apenas inserem o
poema no mundo pagão, mas também são necessários para entendê-lo. De
acordo com Ólason, as alusões mitológicas dos
\emph{kenningar} adicionam à poesia uma dimensão especial e cria uma
percepção de que o mundo dos deuses e dos homens são paralelos: a
``deusa do cálice'' ``mulher'' e o ``Thor do navio'' significa
``homem'', por exemplo. No entanto, com a chegada do cristianismo, esse modo de
pensar ficou obsoleto, se não herético. Além do mais, muitos
\emph{kenningar}, mas não todos, têm componentes metafóricos: \emph{unda
gjalfr} como ``mar das dores'' ou [\versal{SANGUE}]. Uma vez que cada componente
de um \emph{kenning} pode ser substituído por um outro \emph{kenning}
completo, as figuras estendidas resultantes devem ser traduzidas para
que faça sentido dentro da frase: \emph{unda gjalfrs eldi} (``fogo do
mar das feridas'', ``fogo do [\versal{SANGUE}''] ou [\versal{ESPADA}]). Ainda que seu efeito
sempre pareça deliberadamente caótico, Ólason afirma que o \emph{kenning}
cria um contraste notável à regularidade estrita da métrica.

A respeito da estrutura do \emph{kenning}, Simek \& Pálsson
afirmam que ele é constituído principalmente por dois componentes:
uma palavra base (\emph{stofnorð}) e uma palavra modificadora
(\emph{kenniorð}). Mas essa estrutura pode ser ampliada. Poole
afirma que tais componentes quase sempre são \emph{heiti}.
Portanto, de acordo com Simek \& Pálsson, a forma mais simples de formar
um \emph{kenning} é por meio de uma composição (palavra base somada a
palavra modificadora) ou por meio da representação de duas palavras, na qual a
primeira representaria a palavra base e a outra a palavra modificadora. Tal forma 
foi denominada de \emph{einkennt}. Mas também é possível que haja uma
palavra modificadora com dois componentes e, por conseguinte, o \emph{kenning}
passa a ter três componentes, sendo chamado de \emph{tvíkennt}. Se
for ainda mais longo, é chamado de \emph{rekit}.

Poole acrescenta que o tipo ``composição'' é formado pela junção de
duas palavras, sendo determinante o primeiro termo da composição:
\emph{skýrann} ``salão da nuvem'' [\versal{CÉU}]; ao passo que a formação com
duas palavras se dá pelo uso do genitivo: \emph{leggjar íss} ``gelo do
braço'' [\versal{ANEL, BRACELETE DE PRATA}]. O autor complementa ao afirmar
que o último exemplo, ``gelo do braço'', seria possível de ser
compreendido no período da composição, porque as pessoas relacionariam as
cores do gelo e da prata como citado por Snorri em
\emph{Skáldskaparmál}: \emph{Gull er kallat í kenningum eldr handar eða
liðs eða leggjar þvíat þat er rautt, en silfr eða svell eða héla þvíat
þat er þvítt} (``o ouro é chamado nos \emph{kenningar} de fogo da mão ou
fogo dos membros ou fogo da perna porque é vermelho; mas o prata é
chamada de neve, gelo ou cristal de gelo porque é branco'') (edição de
\versal{FAULKES}, 1998, p. 61). Óleson, por sua vez, assume que a
criação dos \emph{kenningar} é baseada em sistemas: a ideia de ouro como
fogo ou luz, que não se apaga na água, fornece a base para inúmeras
formações em que o fogo é um componente e o líquido é outro: \emph{Rinar
bál} ``a fogueira do Reno'' [\versal{OURO}] (\emph{Háttatal}, estrofe 91),
\emph{bála elfar} ``das fogueiras do rio'' [\versal{DO OURO}] (\emph{Erfidrápa
Óláfs helga} de Sigvatr Þórðarson) etc.

Como exemplo de um \emph{tvíkennt}, que tem os dois determinantes citados
por Poole, podemos mencionar o desenvolvimento do \emph{kenning leggjar
íss}: \emph{(lýsi)brekku leggjar íss} ``declive, terra luminoso do gelo
do braço'' = ``declive, terra luminoso'' do [\versal{ANEL, BRACELETE DE
PRATA}] = [\versal{MULHER}]. O autor também expõe alguns exemplos de
\emph{kenningar} incompletos como: \emph{Hlín}, que é um
\emph{heiti} para Frigg e também pode ser um \emph{kenning} incompleto
para ``mulher''. Neste caso, considera-se incompleto 
porque há muitos \emph{kenningar} formados com \emph{Hlín} e outro elemento 
que resultam em [\versal{MULHER}]. É possível verificá-los na obra de Egilsson.
Também existem \emph{kenningar} que inserem um elemento adjetival,
como no exemplo acima com \emph{lýsi}, bem como um elemento derivado
de verbo: \emph{beiði-Týr} ``desejando, demandando Týr''.

Ólason questiona se a audiência nos salões reais
entenderia, de fato, a complexidade dos \emph{kenningar}. Em sua argumentação,
cita T. S. Elliot: \emph{Genuine poetry can communicate before it's
understood}. O som e a fúria de um poema
de batalha seria, assim, contemplado por uma audiência antes da mensagem 
ser completamente decodificada por ela. Também é possível assumir que a
audiência sentia satisfação em decifrar a complexidade da mensagem dos
poemas escáldicos, bem como em saber que fazia parte de um grupo seleto
com acesso a mensagens sagradas e secretas. Assume-se, portanto, de
acordo com o autor, que a audiência acostumada com a poesia escáldica
sabia basicamente o que esperar. Não obstante, a proposta da arte
escáldica seria também surpreendê-la com novas combinações. A respeito das
sintaxes, o autor afirma que, embora pareçam muito complexas, 
seguem certos padrões e regras. Por fim, Óleson considera que uma
audiência qualificada poderia entender o poema na primeira audição.

Uns dos \emph{kenningar} mais antigos que se tem registro são aqueles da estela
rúnica de Karlevi (ilha de Öland, Suécia, código \versal{Ö}l 1, fim do século~\versal{X}, 
Era Viking). No poema, encontra-se: \emph{draugr dolga þrúðar},
que significa ``árvore ou feitor da Trudr (da deusa) da hostilidade,
inimizade, batalha'' = ``árvore ou feitor da [\versal{BATALHA}]'' =
[\versal{GUERREIRO}]. Trudr é um dos filhos de Thor e a palavra
\emph{draugr} aparece em outros \emph{kenningar} com o sentido de homem,
guerreiro: \emph{draugr ørlygis} ``árvore/feitor da batalha``
(Ragnarsdrápa, 8), \emph{draugar brimis} ``árvores/feitores da espada''
(Anon \versal{XII}, \versal{C}10) etc., como apresentados por Egilsson.

Outro \emph{kenning} registrado na estela rúnica de Karlevi é:
\emph{reið-Viðurr Endils iǫrmungrundar} ``Viðurr da carruagem da vasta
terra de Endill (rei do mar)''. A vasta terra de Endill, o rei do mar, é
o mar. Assim, ``Viðurr da carruagem do [\versal{MAR}]'' = Viðurr do
[\versal{NAVIO}] = [\versal{MARINHEIRO}]. Para decifrar esses \emph{kenningar}
tivemos como base o livro de Egilsson (1931).

\SIG{Yuri Fabri Venancio}

Ver também Inscrições rúnicas; Heiti; Linguagem; Literatura; Norreno;
Poesia éddica; Poesia escáldica.

\begin{itemize}
\item \versal{EGILSSON}, Sveinbjörn. \emph{Lexicon Poeticum Antiquæ Linguæ
Septentrionalis. Ordbog over det norske-islandske Skjaldesprog. Forøget
og udgivet for det kongelige nordiske Oldskriftselskab}. 2 Udgave ved
Finnur Jónsson. København: S. L. Møllers Bogtrykkeri, 1931

\item \versal{FAULKES}, Anthony. \emph{Edda. Skáldskaparmál. 1. Introduction, Text and
Notes}. London: Viking Society for Northern Research, 1998.

\item  \versal{ÓLASON}, Vésteinn. Old Icelandic Poetry. In: \versal{NEIJMANN}, Daisy. \emph{A
History of Icelandic Literature}. Lincoln/London: University of
NebraskaPress, 2006, pp. 01-63.

\item \versal{POOLE}, Russell. Metre and Metrics. In: \versal{McTURK}, Rory (ed.). \emph{A
Companion to Old Norse-Icelandic Literature}. Malden/Oxford/Victoria:
Blackwell Publishing Ltd, 2005, pp. 265-284.

\item \versal{SIMEK}, Rudolf; \versal{PÁLSSON}, Hermann. \emph{Lexikon der altnodischen
Literatur}. Stuttgart: Alfred Kröner, 1987.
\end{itemize}
\section{\versal{KIEV}}

O principado de Kiev, com capital na cidade homônima, foi durante
muito tempo o principado e a cidade mais importantes do território que a
historiografia hoje denomina como Rus. Localizada na base do rio Dniepre,
na região florestal onde atualmente se encontram o sul da Rússia e o
nordeste da Ucrânia (local que os escandinavos chamavam de
\emph{Garđariki}), a cidade em si foi fundada pelo lendário Kii, conforme
a tradição presente na \emph{Crônica dos Anos Passados}. Não há datação
na \emph{Crônica} sobre esse evento, mas, segundo Jonathan Shepard, é
possível que o território só tenha sido povoado a partir do século~\versal{VIII},
conforme indicam os vestígios arqueológicos dos eslavos que lá habitavam.
Inicialmente, a área que comporia a futura cidade estava sob o domínio
do Império da Khazária, mas os escandinavos tomaram o controle a partir
do século~\versal{IX} e lá se estabeleceram.

A área que corresponde a Kiev estava localizada em terra fértil ao
longo do rio Dniepre, que, a partir do século~\versal{X}, se tornou uma importante
rota comercial que conectava os nórdicos e a cidade de \emph{Mikligardr}
(Constantinopla), além de ser utilizada pelos bretões
descendentes de escandinavos. O comércio era basicamente de produtos de
luxo, conforme registram os tratados presentes na \emph{Crônica}. Os varegues
forneciam cera, mel, peles e escravos, enquanto os bizantinos forneciam
ouro, seda, frutas e vinho. O controle do comércio de peles, em especial,
transformou Kiev em uma potência econômica regional, conforme Janet Martin. 
Ao redor de Kiev habitavam diversos povos seminômades eslavos
e turcomanos, com destaque para os Pechenegues, que entrariam em conflito
muitas vezes com a futura Rus de Kiev e atrapalhariam o comércio com os
bizantinos.

A \emph{Crônica} afirma que os primeiros varegues a governar Kiev foram
os semiléndários Askold e Dir, dois guerreiros a serviço de Riurik, de
862 a 882. A fonte registra que eles não tinham parentesco com o líder
nórdico, mas chama-os de boiardos, o que implica dizer que eles faziam
parte da aristocracia militar e estavam somente abaixo de Riurik na
hierarquia interna. Apesar da datação da \emph{Crônica} ser
questionável, é geralmente aceito pela historiografia que dois varegues
governaram Kiev antes de serem mortos por Oleg, o Profeta (882-912),
tutor de Igor Riurikovich (913-945). A presença escandinava na elite de
Kiev, especialmente na elite militar, continuou nos governos de Olga
(945-964), Sviatoslav~\versal{I}~Igorevich (964-972) e Iaropolk Sviatoslavich
(972-978). Os tratados de paz entre Rus e Bizâncio, presentes nas entradas de 944, 
mencionam diversos militares escandinavos que ocupavam
posições importantes, como Sveneld, Blud e os enviados de Igor, além de Olga.
Diversos objetos de origem escandinava datados desse período foram
encontrados ao redor de Kiev, sugerindo a manutenção de relações
comerciais e assentamento de mais nórdicos em Rus.

A primazia de Kiev entre os demais principados de Rus foi consolidada
com o batismo de Vladimir~\versal{I}~Sviatoslavich (978-1015) e eventual expansão
do cristianismo, sendo fortalecida com a centralização política de Iaroslav
Vladimirovich, o Sábio (1016-1018, 1019-1054), também conhecido em
fontes nórdicas como Jarizleif. O Principado de Kiev continuou sendo
mais importante de Rus até meados do século~\versal{XII}, quando os principados
do norte, sobretudo Suzdália, começaram a acumular maior poder e
importância política. Mesmo enfraquecida, a possibilidade de governar
Kiev ainda era bastante cobiçada pelos outros príncipes de Rus, ainda
que o principal ramo da dinastia Riuríkida tenha preferido permanecer no
norte em meados do século~\versal{XII}. Em 1240, Kiev foi invadida pela Horda de
Ouro do mongol Batu Khan, encerrando definitivamente, segundo os
historiadores, o período conhecido como Rus de Kiev.

\SIG{Leandro César Santana Neves}

Ver também: Crônica dos Anos Passados; Mikligardr; Novgorod; Olga de
Kiev; Rus; Rússia da Era Viking; Varegues; Vladimir~\versal{I} de Kiev.

\begin{itemize}
\item \versal{DUCZKO}, Wladsyslaw. \emph{Viking Rus: studies on the presence of
Scandinavians in Eastern Europe}. Leiden: Koninklijke Brill \versal{NV}, 2004.

\item \versal{FRANKLIN}, Simon; \versal{SHEPARD}, Jonathan. \emph{The Emergence of Rus
750-1200}. Essex: Longman, 1996.

\item \versal{MARTIN}, Janet. \emph{Treasures of the Land of Darkness: The Fur Trade
and its Significance to Medieval Russia}. Cambridge: Cambridge
University Press, 1986.

\item \versal{NOONAN}, Thomas S. European Russia, c. 500-1050. In: \versal{REUTER}, Timothy.
\emph{The New Cambridge Medieval History -- Volumen \versal{III} c. 900-c. 1024}.
Cambridge: Cambridge University Press, 2008, pp. 487-514.
\end{itemize}
\chapter{L \textarn{l}}
\section{\versal{LAGERTHA}}

Atualmente, Lagertha tornou-se uma personagem bastante popular devido a
sua representação mais recente num seriado de televisão. Entretanto, o
imaginário que retrata Lagertha como uma guerreira viking já era pintado pelos
românticos desde o século~\versal{XIX}. Em casos como esse, as artes contribuíram para
tornar Lagertha uma personagem cada vez mais conhecida e até mesmo
um exemplo de mulher viking guerreira, gerando inclusive certas indagações
acerca de sua existência real, bem como se haveria entre os vikings o
costume de mulheres irem à guerra.

A única fonte conhecida acerca de Lagertha é a \emph{História Danesa}
(\emph{Gesta Danorum}), escrita pelo clérigo Saxo Grammaticus no século~\versal{XII}. 
A obra, da qual se conhece apenas nove volumes, narra de forma breve
a história de Lagertha no último volume. Segundo o relato de Saxo,
Lagertha, Lagerda, Lathgherta, Laðgerða, era uma ``donzela de escudo''
(\emph{skjaldmö}), termo que gera certa controvérsia atualmente, pois há quem defenda que
as tais ``donzelas de escudo'' realmente possam ter sido reais e não apenas
lendas ou mitos.

Nesse caso, a \emph{skyaldmö}, segundo informa Saxo, tratar-se-ia de uma
mulher virgem, brava e audaz, que trajaria roupas masculinas de
guerreiro e participaria de batalhas. Seria uma espécie de amazona
nórdica. No caso de Lagertha, ela se tornou uma ``donzela de escudo'' por
necessidade e não por vocação. De acordo com o Gesta Danorum, ela foi a
primeira esposa do famoso herói Ragnar Lothbrok, suposto rei da Noruega
ou Dinamarca, que teria vivido no século~\versal{IX}.

De acordo com o relato da \emph{Gesta Danorum}, o avô de Ragnar, o rei
Siward (ou Sigurd), que vivia na Noruega, foi assassinado pelo rei Frø
(ou Frodo) dos Suenos (ou \emph{svear}), governante do povo que habitava o atual
território da Suécia, que viria a originar os suecos. Saxo não explica
claramente o motivo da contenda entre os dois senhores. De qualquer
modo, o monarca sueco invadiu o território de Siward, vencendo seu
exército e o matou. Quanto às mulheres (familiares,
damas de corte e escravas), Frø ordenou que fossem humilhadas
publicamente e, para isso, as enviou para um prostíbulo.

A notícia da invasão das terras de Siward, sua derrota, sua morte e 
humilhação de suas mulheres chegaram até seu neto, Ragnar Lothbrok, que,
ao saber dessa grande ofensa, reuniu um exército e declarou guerra ao
rei Frø. Novamente Saxo não nos passa detalhes, mas aduz que parte
das mulheres cativas, incluindo Lagertha, decidiram se unir à
campanha de Ragnar para se vingarem. As mulheres trocaram os vestidos
por trajes masculinos e empunharam armas. Lagertha se destacou
entre todas devido a sua bravura e habilidade como guerreira enquanto
lutava com seus cabelos esvoaçantes no campo de batalha. Tal condição
atraiu os olhares de Ragnar.

Após a vitória sobre o exército do rei Frø, Ragnar, apaixonado por
Lagertha, decidiu procurá-la, indo até a casa dela. Ao chegar na
residência de Lagertha, Ragnar deparou-se com um urso e um lobo. A
presença de tais animais lembra algumas histórias nas quais a
``donzela'' era protegida por guardiães, fossem estes humanos, animais
ou monstros. Após matar o urso e ferir gravemente o lobo, Ragnar
finalmente pôde adentrar à residência de Lagertha e pedi-la em
casamento.

Ragnar e Lagertha tiveram três filhos: um menino chamado Fridleif e duas
meninas, cujos nomes não são citados. Lagertha, ao se tornar mãe e esposa,
teria deixado seu lado guerreiro momentâneo, embora o relato não informe nada
quanto a sua origem e passado. Pelo que Saxo Grammaticus informou,
Lagertha nem sempre teria sido uma guerreira, mas tornou-se uma devido a
ocasião da morte do rei Siward e a ameaça do rei Frø.

Assim, quando ela se casou com Ragnar Lothbrok, perdeu sua condição
de ``donzela de escudo''. Três anos depois, após algumas aventuras,
Ragnar acabou conhecendo a princesa Thora Borgarhjört
(Þora), filha do rei Herodd da Gotland. Para disputar a mão em
casamento da princesa, Ragnar se divorciou de Lagertha. Posteriormente
venceu o desafiou do rei Herodd de Gotland e pôde se casar com Thora. No
caso de Lagertha, Saxo informa que ela se casou novamente, porém, o nome
do segundo marido não é mencionado.

Saxo prossegue o relato de Lagertha de forma breve, expondo que, ao lado
do novo marido, organizou uma frota de 120 navios para ajudar na
Batalha de Laneus, na qual faleceu um dos filhos de Ragnar, chamado
Siward ou Sigurd. Graças ao auxilio de Lagertha, a batalha foi vencida.
Após retornar para casa, ela e o marido brigaram. Ainda que os motivos de 
tal briga não sejam conhecidos, a \emph{Gesta Danorum} sugere a possibilidade de Lagertha 
ter matado o marido e assumido suas terras. Após tal acontecimento, a história de Lagertha não é mais
narrada.

Ainda que o \emph{Gesta Danorum} faça referências a acontecimentos históricos, 
ele também narra acontecimentos lendários, razão pela qual Lagertha 
é considerada uma lenda ou uma personagem literária. No entanto, Hilda Davidson, em seu
comentário a respeito do \emph{Gesta Danorum}, apontou que Lagertha
pode ser uma lenda baseada em alguma mulher real, possivelmente
Hlaðgerðr, mulher que, no século~\versal{VI}, enviou vinte navios para
auxiliar o rei Halfdan dos Scylding. Outra hipótese é que Lagertha tenha
sido inspirada em Luitgarde de Vermandois, esposa do duque Guilherme Espada Longa e
duquesa normanda na época em que a
Normandia era uma província franca colonizada por vikings.

Em outra hipótese apontada por Laia San José Beltrán, Lagertha teria
sido inspirada na rainha Thorgerd (Þorgerd), esposa do rei Hakoon da
Noruega (937-995), que morava em Hladir, no vale de Gualardar. Este,
segundo Saxo, teria sido um dos lugares que viveu Lagertha. Outra teoria,
também mencionada por San José Beltrán, aduz que Lagertha pode
ter sido baseada em uma deidade similar às valquírias que aparecem em
algumas sagas, como a \emph{Saga Jomsviking}, a \emph{Saga de Njáls} e a
\emph{Þorleifs Þáttr jarlsskálds}. Não obstante, Lagertha também poderia
ser um dos nomes utilizados para criar a personagem da deusa Freyja.

Para Franesca Zappatore, existe a possibilidade de Lagertha ser uma construção
de Saxo Grammaticus e não ter sido baseada em mulheres reais, 
já que o autor era conhecedor da literatura clássica,
na qual se encontram histórias de amazonas. De qualquer forma, tais
possíveis referências são hipóteses inconclusivas, pois até hoje não se
encontrou evidências concretas de mulheres vikings atuando no campo de
batalha, ou mesmo da existência de tropas femininas. Lagertha também pode ser
uma personagem literária inspirada em algumas valquírias dos mitos,
especialmente a valquíria Brunhilde, esposa do herói Sigurd, cuja
história é narrada na \emph{Saga dos Volsungos} e em alguns poemas da
\emph{Edda poética}.

Entre as representações artísticas baseadas na personagem, a mais recente
diz respeito ao seriado ``Vikings'' do History Channel,
estreado em 2013. A série retrata Lagertha (Katheryn Winnick) como esposa
de Ragnar Lothbrok (Travis Fimmel) e mãe de Björn e Gyda. A
Lagertha da série foi baseada na versão do Saxo, mas com algumas
diferenças. Na trama da série, Ragnar não é descendente de
reis, sendo retratado como um fazendeiro bravo, petulante, esperto e com espírito de
liderança. O passado de Ragnar e de Lagertha não é mencionado na série, porém, no
final da primeira temporada, o casamento de ambos já apresentava desgaste.
A partir da segunda temporada, Lagertha inicia sua ascensão social.

De acordo com o relato de Saxo Grammaticus, a história de Lagertha
termina com ela se tornando uma mulher ingrata e traiçoeira, pois
assassinou o próprio marido para lhe usurpar o nome e a propriedade. Tal
característica foi aproveitada na série, pois Lagertha mata seu segundo
marido (Earl Sigvad) e toma o controle da vila de Hedeby, 
assumindo com o nome de Earl Ingstad. Hedeby foi uma cidade real, 
situada no sul da Dinamarca, tendo sido um importante núcleo urbano e polo 
comercial, mas, na série, não passa de uma vila. Ainda assim, é cobiçada por 
alguns, como Kalf, que na terceira temporada passa a disputar 
com Lagertha o controle de Hedeby.

Não obstante, a Lagertha da série manteve várias características da
personagem descrita por Saxo Grammaticus. Ambas são mulheres audazes,
exímias guerreiras, belas, ambiciosas, traiçoeiras e de temperamento forte. 
A Lagertha da série participa das expedições à Inglaterra e à França, como
também não mede esforços para remover de seu caminho as ameaças, mesmo que
para isso tenha que sujar suas próprias mãos com sangue. Nesse ponto a
Lagertha da série se revela maquiavélica e traiçoeira, algo observável
nas temporadas terceira e quarta.

\SIG{Leandro Vilar Oliveira}

Ver também Guerreiras nórdicas; Literatura; Mulheres.

\begin{itemize}
\item \versal{DONTAINE}, François. História e ficção em Vikings. \emph{Notícias
Asgardianas}, n. 10, 2015, p. 88-94.

\item \versal{GRAMMATICUS}, Saxo. \emph{The history of Danes}, books \versal{I-IX}. Ed.
Hilda Ellis Davidson. Trad. by Peter Fisher. Woodbridge: \versal{D. S.}
Brewer, 1979.

\item \versal{HOLMAN}, Katherine. \emph{Historical dictionary of the vikings}. Lanham:
Scarecrow Press Inc., 2003.

\item \versal{PUCHALSKA}, Joanna Kataryzna. Vikings Television Series: When History and
Myth Intermingle. \emph{The Polish Journal of the Arts and Culture},
vol. 15, n. 3, 2015, pp. 89-105.

\item \versal{SAN JOSÉ BELTRÁN}, Laia. Análisis histórico de la serie Vikingos de
History Channel. In: \emph{Los Vikingos en la Historia}, 2. \versal{HUM}-165:
Patrimônio, Cultura y Ciências Medievales. Granada: Universidad de
Granada, España, 2015, pp. 25-72.

\item \versal{ZAPPATORE}, Francesca. \emph{Maiden warriors in Old Norse Literature}.
Dissertação de Mestrado, Universidade de Bolonha, s.d.
\end{itemize}
\section{\versal{LANDNÁMABÓK}}

\emph{Landnámabók}, livro dos assentamentos, detalha a
colonização da Islândia pelos noruegueses nos séculos~\versal{IX}~e~\versal{X}. 
Atribui-se a composição do livro a Ari Thorgilsson, no começo do século~\versal{XII}.
Ari Thorgilsson, conhecido como o pai da história da Islândia, foi o
primeiro homem a escrever em vernacular. Além da suposição que atribui 
a ele a escrita da \emph{Landnámabók}, sabe-se que ele escreveu o \emph{Livro dos Islandeses},
\emph{Íslendingabók}.

A obra apresenta um total de 430 colonizadores diferentes, 3500 nomes
pessoais e 1500 nomes de fazendas registradas em suas folhas. Tais informações
foram muito úteis, posteriormente, aos compiladores de sagas.
Existe um total de cinco versões diferentes do livro. Três foram escritas
ou complicadas no período medieval e duas compiladas no século~\versal{XVII}.
\emph{Sturlubók} é atribuído a Sturla Thórdarson, mas a única versão
existente de sua redação se encontra em um manuscrito do século~\versal{XVII}.
\emph{Hauksbók} é a obra de Haukr Erlendson, escrita entre 1306-1308,
mas hoje se encontra preservada em manuscrito do século \versal{XVII}. Haukr aparenta ter
usado \emph{Sturlubók} e uma versão mais antiga, agora perdida, chamada
\emph{Styrmisbók}. \emph{Melabók} foi escrito no começo do século~\versal{XIV} e
é a mais antiga das versões preservadas, embora só restem fragmentos do
texto original. \emph{Skardsárbók} é uma compilação do século~\versal{XVII}
baseada na \emph{Hauksbók} e \emph{Sturlubók}. Por fim, a
\emph{Thórdarbók}, que é outra compilação do século~\versal{XVII}, mas
baseada na \emph{Skardsárbók} e \emph{Melabók}.

\SIG{André Araújo de Oliveira}

Ver também Althing; Thing; Godi; Islândia na Era Viking; Íslendigabok.

\begin{itemize}
\item \versal{HOLMAN}, Katherine. \emph{Histocial Dictionaries of the Vikings}. Oxford:
The Scarecrow Press Inc., 2003.

\item \versal{SIGURÐSSON}, Jón Viðar. Iceland. In: \versal{BRINK}, Stefan; \versal{PRICE}, Neil (eds.).
\emph{The Viking World}. New York. Routledge, 2008, pp. 571-578.

\item  \versal{VÉISTEINSSON}, Orri. \emph{The Christianization of Iceland}: Priest,
Power and social change 1000-1300. Oxford: Oxford University Press,
2000.
\end{itemize}
\section{\versal{L´ANSE-AUX-MEADOWS}}

\emph{L'Anse-aux-Méduses}, em francês, ou Enseada das Águas-vivas, em
português, é o nome dado a um dos sítios arqueológicos mais importantes
no âmbito dos estudos da presença nórdica na América. É considerado o sítio com
presença nórdica mais distante na América, depois dos sítios
arqueológicos da Groenlândia. A pequena comunidade que reside no sítio
está localizada na região de Terra Nova (\emph{Newfoundland}) e
Labrador, ambas províncias do extremo oeste do Canadá, em que o
explorador~Leifr Eriksson (\emph{Leifr Eiríksson}) teria residido como
um ponto do processo de colonização da região por parte de europeus.

Em 1960, o norueguês Helge Ingstad e a arqueóloga~Anne-Stine Ingstad,
foram responsáveis por encontrar na região os primeiros achados
arqueológicos que davam maiores ares de comprovação da presença nórdica
na América do Norte, corroborando com muito que fora apresentado por
diversas fontes literárias. A descoberta desse casal de pesquisadores,
se inicia com o termo Vínland, terra das vinhas, que seria o nome
que as fontes literárias e escritas nos apresentam como o nome dado
pelos nórdicos para a região americana. Partindo desta premissa, os
pesquisadores buscaram regiões prováveis de vinhas e com base nas
referências escritas e literárias, considerando a região próxima de
Massachusetts e do Canadá, como possibilidades para o local dessa
presença nórdica.

O casal viajou por meses pela costa da região mapeada antes de chegarem
à ponta Norte da ilha e na aldeia de L'Anse aux Meadows. Partindo de
entrevistas e explorações, eles começaram a pautar a presença dos
assentamentos nórdicos na região, algo que era bem incerto neste início
de 1960. Desses relatos eles descobriram sobre formas retangulares
peculiares que poderiam ser estruturas diferentes, em uma região ao sul
de L'Anse aux Meadows, a região de \emph{Epaves Bay}. A partir daí,
usando por base o conhecimento da formação sobre as fazendas na
Groenlândia, irá se iniciar um longo trabalho de exploração arqueológica
que resgatará verdadeiros tesouros e trazer para o debate da presença
nórdica na América uma força cada vez mais ampla para a aceitação dessa
presença.

Com o início das escavações e explorações, a equipe arqueológica irá
encontrar cerca de oito grandes fundações, que continham: casas, forja,
serragem, regiões para culinária e outras fundações de mais variadas
funcionalidades. Vários foram os objetos encontrados na escavação, em
que os achados residiam: pregos de ferro artesanais -- que colocavam em
suspensão a teoria de que as comunidades locais haviam construído as
estruturas --, um fuso de pedra sabonete para amolação de fios, uma
tábua de barco, um recipiente feito de casca de vidoeiro, pinos de
bronze, peças de ferro, acessórios de cozinha e as cascas de abóbora (do
tipo
\href{https://en.wikipedia.org/wiki/Cucurbita_moschata}{\emph{Cucurbita
moschata}}).

A partir desses objetos, várias pesquisas e perspectivas puderam ser
ponderadas e facilitar um estudo sobre os nórdicos na América. Um dos
alfinetes anelados encontrado no sítio, era largamente usado na Irlanda,
que tinha sua origem de fundação nórdica e que se consolidará a partir
do século \versal{IX}. Este alfinete que poderia ser usado tanto para capas como
para aportes no ombro, revela muito de uma trajetória dos viajantes
escandinavos, mostrando toda uma circularidade de culturas e bens
materiais que rodavam dentro das regiões nórdicas devido as suas viagens
e incursões.

Apesar destas descobertas materiais, de fato havia uma aproximação
dessas descobertas com a literatura? As Sagas do Descobrimento da
América foram amplamente usadas, virando inspiração para um dos
primeiros filmes sobre os nórdicos (\emph{The Viking} de 1928, dirigido
por Roy William Neill), inclusive, mas também revelou-se muitos estudos,
tanto para desacreditar da fonte, devido a sua linguagem e do
``compromisso'' literário ser um compromisso outrém, como também aqueles
que acreditavam que um registro arqueológico já iria corroborar para com
a construção dessas narrativas. Devido ao nome Terra da Vinhas, os
pesquisadores pensavam em terras mais ao Sul, em que vinhas e vegetais
diferenciados cresciam, e que por isso os nórdicos denominaram tal
localidade, pelo contato com esse tipo de flora. Logo, devido é esse
tipo de flora não ser a presente em L'Anse aux Meadows, culminou em uma
descrença desse relato, ampliando o sentido hiperbólico e literário das
narrativas das sagas. O que trouxe uma mudança nessa questão foi a
presença da
\href{https://en.wikipedia.org/wiki/Cucurbita_moschata}{\emph{Cucurbita
moschata}}, que não cresce tão ao Norte, o que culminou na teoria de que
os nórdicos exploram as terras americanas, e que L'Anse aux Meadows
seria um assentamento estratégico para os nórdicos negociaram com os
\emph{skrælingjar} (nome que os nórdicos davam para os povos encontrados
na América do Norte e Groenlândia, os esquimós -- s. \emph{skrælingi}).

A partir disso, datações de radiocarbono determinaram que o material
encontrado seria de meados do ano mil, justamente no tempo em que Leifr
Eriksson havia viajado para a América, de acordo com as narrativas
literárias/escritas e suas datações. As escavações em L'Anse aux Meadows
trouxeram provas mais definitivas para a questão do ``descobrimento'' da
América do Norte, e trouxeram bases arqueológicas para afirmar que os
nórdicos estiveram na América quase quinhentos anos antes de Colombo,
trazendo uma nova concepção para a história da humanidade. Sua
importância é tanta, que no ano de 1978 a comunidade/aldeia se tornou
Patrimônio da Humanidade pela Organização das Nações Unidas para a
Educação, a Ciência e a Cultura (\versal{UNESCO}).

\SIG{José Lucas Cordeiro Fernandes}

Ver também Sagas do Atlântico Norte; Leif Eriksson; Vínland.

% JORGE: @FELIPE
\begin{itemize}
\item \versal{Arneborg}, Jette. The Norse Settlement in Greenland: The Initial Period
in Written Sources and in Archaeology. In: Wawn, Andrew; Sigurðardóttir,
Þórunn (eds.). \emph{Approaches to Vinland: Proceedings of a conference
on the written and archaeological sources for the Norse settlements in
the North-Atlantic region and exploration of America}. Reykjavik:
Sigurdur Nordal Institute, 2001.

\item \versal{Barnes}, Geraldine. \emph{Viking America: The First Millennium}.
Cambridge: D.S. Brewer, 2001.

\item \versal{Bergersen}, Robert. \emph{Vinland Bibliography: Writings Relating to the
Norse in Greenland and America}. Tromsø: University of Tromsø, 1997.

\item \versal{Fanning}, Thomas. \emph{Viking Age Ringed Pins from Dublin: Medieval
Dublin Excavations 1962-81}. Dublin: Royal Irish Academy, s. \versal{B},
1994, vol. 4.

\item \versal{Ingstad}, Helge; \versal{Ingstad}, Anne Stine. \emph{The Discovery of a Norse
Settlement in America: Excavations of Norse Settlement in L'Anse aux
Meadows, Newfoundland}. New York: Checkmark Books, 2001.

\item \versal{JONES}, Gwyn. \emph{The Norse Atlantic Saga: Being the Norse Voyages of
Discovery and Settlement to Iceland, Greenland, and North America}.
Oxford/New York: Oxford University Press, 1986.

\item \versal{LANGER}, Johnni. Vikings, cultura e religião: o mito arqueológico nórdico
dos Estados Unidos. \emph{O Olho da História}, n. 18, 2012.

\item \versal{POHL}, Frederick J. \emph{The Viking Settlements of North America}. New
York: Clarkson N. Potter, Inc. The Science News-Letter, 1972.

\item \versal{SANTOS}, André Luiz Campelo dos. \emph{Vikings na terra nova: uma
análise acerca do imaginário nórdico na América}. Monografia (Bacharelado
em História) - Universidade Federal do Ceará, 2013.

\item \versal{The Science News-Letter}. Viking Ruins Found. \emph{The Science
News-Letter.} Estados Unidos:
\href{https://www.jstor.org/publisher/sciserv?refreqid=excelsior\%3Af41fc9e76fcaf675e16e7a82cde0cc7a}{Society
for Science \& the Public}, vol. 84, n. 20, nov. 16, 1963, p. 306.

\item  \versal{WALLACE}, Birgitta. The Norse in Newfoundland: L'Anse aux Meadows and
Vinland. \emph{Newfoundland Studies}, vol. 19, n. 1, 2003, pp. 06-43.

\end{itemize}
\section{\versal{LAPÔNIA DA ERA VIKING}}

Situada na Fenoescândia, no norte das penínsulas Escandinava e de Kola,
está a \emph{Sápmi} -- a
terra dos Sámi, único grupo étnico reconhecido como aborígenes europeus.
O termo ``Lapônia'' parece ter surgido a partir de \emph{lapp} --
palavra que, no século~\versal{XII} d.C., descrevia ``aquele que pratica atividades
econômicas `lapônicas''', como a pesca, a caça e a criação de renas na região
ártica. Ambas as denominações foram cunhadas por forasteiros e são atualmente vistas como
depreciativas. O termo \emph{Sápmi}
começou a se espalhar sobretudo após os anos 1970, momento em que os
estudos Sámi tomaram nova direção, englobando essas populações
que, por meio das letras, passaram a se autodesignar. As divisões
geopolíticas atuais, como ``Lapônia Norueguesa, Sueca e Finlandesa'', são
uma herança dos contatos coloniais entre os Estados nórdicos
modernos -- sobretudo a partir do século~\versal{XVI} d.C. -- e as populações
locais da \emph{Sápmi}.

Os assentamentos Sámi, na Fenoescândia, começaram por volta dos séculos~\versal{XI} e
\versal{X} a.C., seguindo duas direções: a do oceano Ártico e a da região da atual
Finlândia. Até o século~\versal{XI} d.C., essas populações se expandiram para o
Norte da Noruega e Suécia. A região de Áltá, na Noruega, tem a maior
quantidade de pinturas rupestres Sámi. Os principais temas, assim como
os tambores \emph{noiadi}, são a caça, a pesca, os animais e os rituais
religiosos. Esse material conjunto forma a base para os estudos sobre a
cosmologia e religiosidade Sámi.

No Período Viking, surgem as primeiras vilas de inverno -- locais de
fácil acesso para mercadores e coletores de impostos. Na \emph{Historia
Norwegie}, datada da segunda metade do século~\versal{XII} d.C., há uma descrição
dos ganhos Sámi e sua relação com o monarca norueguês que tinha
condições de impor tributos aos \emph{finneskaten}. Nos séculos
seguintes, com a ascensão dos Estados modernos nórdicos, a tensão pela
região aumenta. No século~\versal{XIV} d.C., os conflitos entre Suécia e Rússia se
iniciam e a primeira resolução viria em 1595, com o Tratado de Teusina.
Porém, é só em 1617 -- com o Tratado de Stolbova -- que a definição
atual das fronteiras seria estabelecida. Com o aumento dos
interesses econômicos e políticos pela região, surge um dos primeiros
mapas que retrata o escopo geográfico da \emph{Sápmi}, feito pelo
holandês Jan Huyghen (1594).

Poucos anos depois, em 1601, os contemporâneos e compatriotas de Hyughen -- van
Linchoten e Simon van Salingen -- elaboraram outras cartas que,
além de referenciarem os limites geográficos, traziam algumas representações das
populações que habitavam a região. No caso de van Salingen, sua
\emph{Lappia par Norwegie} é especialmente interessante, pois o
cartógrafo holandês foi patrocinado pelo monarca dinamarquês Cristiano~\versal{IV},
que, em troca, esperava solidificar suas reivindicações territoriais
sobre a região e, consequentemente, promover sua ocupação e exploração.

No caso sueco, no século~\versal{XVII} d.C., houve interferência ativa na
\emph{Sápmi}: trabalhos missionários luteranos são enviados à Lapônia a
partir da década de 30, seguido por algumas escolas
idealizadas por Johannes Skytte. Além disso, a coroa fomenta a
colonização da região por meio da promulgação da \emph{Lappland Bill} em
1673. É do mesmo ano a obra \emph{Lapponia}, de Johannes Schefferus, professor da Universidade
de Uppsala. Esta é, provavelmente, o primeiro trabalho
que tratou exclusivamente da geografia da \emph{Sápmi}, além dos costumes e
maneiras de viver dos habitantes da região. Considerado um
``lapologista'', Schefferus é um dos responsáveis pela divulgação
global da região como a terra do lendário e fascínio.

A partir do século~\versal{XVIII} d.C., a curiosidade pelo exótico,
somada aos impulsos científicos do Iluminismo, fomentaram expedições e
relatos sobre a região pautadas na experiência pessoal, em crenças
comuns e nas influências literárias anteriores (como Schefferus).
Os suecos Carlos Lineu e Nicolaus Hackzell, os franceses Jean-François
Regnard e Réginald Outhier, o norueguês Knud Leem e o lombardo Giuseppe
Arcebi são exemplos de personalidades que empreenderam viagens e produziram
relatos sobre a geografia e os povos locais.

No século~\versal{XIX} d.C., com o advento das noções de Lars Laestedius 
e da teoria das raças que circulava pela Europa, houve uma diminuição
do caráter ``místico'' da Sámi 
(que, todavia, não deixa de existir. Inclusive, no século~\versal{XX},
é esse resquício de místico e selvagem que deveria ser domesticado
pelo cristianismo, civilização e campesinato). A atividade agrícola e o
pastoreio itinerante de renas selvagens passam a ser entendidas como a
salvação do extremo norte. No fim desse mesmo século, a política frente aos
Sámi (e à demarcação de terra, debate que surge com a ocupação das
\emph{siidas} por colonos suecos no século~\versal{XVII} e ``resolvido'' apenas
no século~\versal{XX}) é isolacionista. Havia a necessidade de preservar, de
forma intacta, aquela sociedade. O lema dessa campanha isolacionista
era: ``\emph{Lapp skall vara lapp}'': [um] lapão será [um]
lapão.

Nos séculos~\versal{XX}~e~\versal{XXI}, algumas dessas crenças ``lapológicas'' sobre o exotismo da
\emph{Sápmi} e seus habitantes continuaram, ajudndo a
reforçar os discursos turísticos da região. As noções de diferença
cultural, mundo primitivo e selvageria são muito utilizadas pela
indústria turística. De forma contraditória, essa reprodução de antigas
ideias sobre a região e os Sámi raramente é a imagem que eles têm
de si. O debate sobre o direito à terra e a outros recursos ainda continua
em disputa, visto que por muito tempo a maior parte das resoluções foram
tomadas de ``cima para baixo'', sem, necessariamente, levar em
consideração as reivindicações dos povos locais. Assim, há muito que se
estudar e debater sobre a história dos Sámi e a relação desigual com os
Estados modernos nórdicos e suas implicações históricas.

\SIG{Vítor Bianconi Menini}

Ver também Finlândia da Era Viking; Sámi, fínicos e nórdicos.

\begin{itemize}
\item \versal{BROADBENT}, Noel D.~\emph{Lapps and Labyrinths: Saami
Prehistory, Colonization and Cultural Resilience}. Whasington D.C: Arctic
Studies Center, National Museum of Natural History, 2010.

\item  \versal{FÜR}, Gunlog.~\emph{Colonialism in the Margins: Cultural
Encounters in New Sweden and Lapland}. Leiden: Brill, 2006 (The Atlantic
World - Europe, Africa and the Americas, 1500 - 1800, vol. \versal{IX}).

\item \versal{KENT}, Neil.~\emph{The Sámi Peoples of the North: A Social and
Cultural History}. London: Hurst \& Company, Lapps and labyrinths, 2014.

\item \versal{LEHTOLA}, Veli-Pekka.~\emph{The Sámi People: Traditions in
transition}. Fair Banks: University of Alaska Press, 2004.
\end{itemize}
\section{\versal{LAXDAELA SAGA}}

\emph{Laxdæla saga} é uma saga pertencente ao subgênero das
``sagas dos islandeses'' -- \emph{Íslendingasögur} -- datada aproximadamente
de meados do século~\versal{XIII}, quando o gênero de literatura cavalheiresca
estava em seu apogeu na Europa continental. A saga foi conservada em
numerosos manuscritos, que, frequentemente, são estudados em dois grupos. A
versão integral mais antiga da saga está inserida no códice denominado
Möðruvallabók (\versal{AM} 132 fol.), possivelmente redigido entre os anos
1330-1370. O fragmento mais
importante e antigo do segundo grupo procede de meados do século~\versal{XIII} e
se diferencia dos fragmentos do primeiro grupo à medida que inclui os dez
capítulos que, na opinião dos especialistas, não foram escritos
pelo suposto autor original da saga. Tais capítulos constituíram uma
unidade narrativa de menor importância, introduzida na trama da obra
para aprofundar a figura de Bolli, um dos principais personagens da
saga, conhecido como \emph{Bolla þáttr Bollasonar}.

A \emph{Laxdæla saga} é uma saga anônima, cuja autoria foi atribuída no
decorrer dos anos a um dos personagens mais ilustres da ilha, como é o
caso do poeta Óláfr hvítaskáld (†1259), sobrinho de Snorri
Sturlusson. Sem dúvida, nas últimas décadas, não tem sido poucas as
vozes que tem defendido a ideia de uma autoria feminina da saga, devido a requintada representação
psicológica de Guðrún, sua personagem principal. É o caso de Guðrún Nordal, por exemplo.
Também se sustenta que o autor poderia ter sido
alguém como Sturla Þórdarsson (†1284), outro sobrinho de Snorri
Sturluson, político de grande influência, poeta e autor da \emph{Hákonar
saga} e a \emph{Magnúss saga}.

Os principais acontecimentos narrados pela saga demarcam fatos
importantes da história da ilha: o início da colonização, a formação da
assembleia geral (ou \emph{Alþingi}) no ano de 930 e a conversão ao
cristianismo no ano 1000. Não obstante, passados os primeiros
capítulos nos quais são enunciadas as vicissitudes dos primeiros
colonizadores noruegueses (como Björn ou seu filho Ketill, o Chato), a narrativa
da Laxdæla percorre alguns caminhos pouco habituais na literatura
islandesa da época.

Na saga estão refletidas com nitidez as tendências românticas
procedentes do continente, sobretudo na apresentação da história de
amor entre Guðrún Osvifsdóttir e Kjartan Ólafsson. Não deixa de lado,
entretanto, a grandeza trágica própria de outras sagas consideradas como
o auge do gênero, como a \emph{Egils saga} ou a \emph{Njáls saga}. Na
\emph{Laxdæla saga}, convergem elementos pertencentes às tradições
literárias que deram origem ao que se pode chamar ``renascimento
nórdico''. A \emph{Laxdæla saga} é
uma história que representa como nenhuma outra a multiplicidade do
ambiente literário islandês. Os traços de tal multiplicidade podem ser encontrados tanto nos possíveis paralelismos, apontados por muitos investigadores,
entre o personagem de Guðrún e a personagem Brynhildr dos poemas épicos da Edda
Maior, quanto no gosto do autor da \emph{Laxdæla saga} pela literatura de caráter cortesão, que
naquela época começava a ser traduzida na Noruega. Porém, a \emph{Laxdæla saga} não é somente um fiel reflexo dessas
correntes literárias, senão da disputa entre a fé dos antepassados e a
nova religião dos livros. Em suas páginas têm lugar tradições de origem
pagã -- como as profecias postas na boca de Gestr, quem interpreta os
sonhos de Guðrún --, que, todavia, convivem com o anúncio da chegada da nova
religião que \emph{at miklu sé háleitari} (``é muito mais sublime''), cujo auge se expressa
nas referências da supremacia do cristianismo na
morte de Kjartan.

A \emph{Laxdæla saga} é, pois, um dos exemplos mais acabados de uma
época da literatura islandesa em que estão representadas tanto as obras
originalmente concebidas em língua vernacular -- cuja função principal era
a de refletir e glorificar a história da Islândia e de seus personagens
mais destacados, em alguns casos da perspectiva das classes dominantes
do país --, como as obras traduzidas de originais latinos e franceses,
que são boa prova da permeabilidade dos literatos islandeses a
influências, modelos e temas de origem estrangeira.

\SIG{Teodoro Manrique Antón}

Ver também Islândia da Era Viking; Linguagem; Literatura; Poesia
escáldica; Sagas islandesas.

\begin{itemize}
\item \versal{AUERBACH}, Loren. Female Experience and Authorial Intention in Laxdœla
saga. In: \emph{Saga-Book}. Viking Society for Northern
Research, University College London, 1998-2001, pp. 30-53 (vol. 25).

\item \versal{JAKOBSSON} Ármann. Laxdæla Dreaming: A Saga Heroine Invents Her Own Life.
In: \emph{Leeds Studies in English,} n. 39, 2008, pp. 33-51.

\item \versal{MUNDT}, Marina. Sturla Þórdarson und die Laxdæla saga. In: \emph{Skrifter
fra instituttene for nordisk språk og litteratur ved universitetene I
Bergen, Oslo og Trondheim}, vol. 4. Bergen: Universitetsforlaget, 1969.

\item \versal{NORDAL} Guðrun. Text in Time: the Making of Laxdœla. In: \versal{NORDVIG}, Asger
Mathias Valentin; \versal{TORFING}, Lisbeth H. \emph{et al}. (eds.). \emph{The
15th International Saga Conference: Sagas and the Use of the Past},
Århus, 2012.

\item \versal{VANHERPEN}, Sofie. Letters in the margin: Female provenance of
\emph{Laxdæla saga} manuscripts on Flatey. Paper presented at the
16\textsuperscript{th} Saga Conference Sagas and Space, Zürich and
Basel, Switzerland, ago. 2015.
\end{itemize}
\section{\versal{LEIF ERIKSSON}}

Leifr Eiríksson, ou \emph{Leifr hinn heppni} (Leifr, o Sortudo), é
um dos nórdicos mais famosos da cultura nórdica, considerado o
descobridor das terras na América do Norte e responsável pela primeira tentativa de colonização da região. Leifr é filho de \emph{Eiríkr Þorvaldsson},
ou Érico, o Vermelho (\emph{Eiríkr hinn rauði}). Este foi
responsável pela colonização da Groenlândia, evidenciando o caráter explorador que marcou a linhagem da família.

A fama de Leifr é tributária das Sagas
do Atlântico Norte, que nos revelam tanto sobre o descobrimento da
Groenlândia, como também da América do Norte. Na saga dos groenlandeses --
uma das Sagas do Atlântico Norte --, encontra-se o relato sobre como Leifr
inicia sua empreitada voltada para a descoberta de novas terras. De fato, tal
saga nos revela que ele segue os relatos de viagem de Bjarni
Herjúlfsson, o qual já havia avistado terra desconhecida. Leifr contrata
a tripulação de trinta e cinco homens de Bjarni para intentar fazer o
que este não fez, a saber, pisar na nova terra.

Leifr encontrou a terceira e última terra avistada por Bjarni: ``Conosco não
aconteceu, quanto a esta terra, como aconteceu com Bjarni, de não termos
pisado em terra. Agora darei um nome à terra e hei de chamá-la de
Helluland [Terra da Placa da Rocha]'' (\versal{Anônimo}, 2007a,
p. 64). Essa terra -- possivelmente a região de Labrador, no Canadá,
conforme indicam os achados de L'Anse aux Meadows -- trouxe novas formas culturais para o mundo
nórdico. Após isso ele encontra outra terra, plana e
com florestas, destoante da paisagem gélida da anterior. Chamou-a de
Markland, ou Terra Coberta por Florestas. Esta seria uma região mais
ao sudeste de Labrador (c. 1000).

Leifr e seu bando viajaram por alguns dias, conhecendo novos lugares, até finalmente
se concentrarem na exploração da terra: ``Agora eu quero ter nosso bando
dividido em dois e quero ter a terra explorada, e uma metade do bando
ficará em casa, enquanto a outra metade explorará a terra [...]''
(\versal{Anônimo}, 2007a, p. 65). Além dessa exploração, Leifr construiria
casas no local, transformando sua descoberta em um assentamento na América do
Norte.

Leifr, que ``[...] era um homem grande e forte, o homem mais
imponente de se ver, esperto, bom e justo em todos os aspectos''
(\versal{Anônimo}, 2007a, p. 66), após receber relatos sobre as parreiras e uvas de
Tyrkir (seu pai de criação), resolveu chamar a nova terra de
Vínland, Terra das Vinhas/Parreiras. Após um tempo de exploração, 
Leifr parte de volta para a Groenlândia,
levando consigo novas experiências e relatos, assentando as bases
para outras viagens, como a do explorador Thorvaldr, a tentativa de Thorsteinn
e a exploração de Thorfínnr Karlsefni.

Ademais, a \emph{Saga de Eiríkr, o Vermelho}, nos revela o grande apreço deste para
com Óláfr Tryggvason, um dos grandes
reis da Noruega, razão pela qual haveria incumbido Leifr de levar o cristianismo para
Groenlândia.  Este atendeu a solicitação de Érico e foi mais além, levando a fé cristã também para as terras recém descobertas da América do Norte.

Leifr é, sem dúvida, um dos maiores ícones nórdicos no mundo
contemporâneo. Sua figura foi retratada em estátuas, monumentos, músicas e selos, além de 
ser referenciada em feriado nacional (o ``\emph{Leif Erikson Day}'', no dia nove de outubro, uma
homenagem do presidente Lyndon B. Johnson e do congresso americano para
o primeiro europeu nas Américas). Leifr marcou a história da
Groenlândia como um líder forte, principal responsável pela
cristianização da localidade. Desde seu nascimento, em 970, até
sua morte, em meados de 1019-1125 (as sagas e as demais fontes não
mencionam a data precisa de sua morte), foi um
símbolo de uma linhagem de exploradores vikings.

\SIG{José Lucas Cordeiro Fernandes}

Ver também Brathahlid; Sagas do Atlântico Norte; Vínland.

\begin{itemize}
\item \versal{ANÔNIMO}. A Saga do Groenlandeses. In: \emph{As Três Sagas Islandesas}.
Trad. Théo Moosburger. Curitiba: Editora \versal{UFPR}, 2007a.

\item \versal{ANÔNIMO}. A Saga de Eiríkr Vermelho. In: \emph{As Três Sagas Islandesas}.
Trad. Théo Moosburger. Curitiba: Editora \versal{UFPR}, 2007b.

\item \versal{ARNEBORG}, Jette. The Norse Settlements in Greenland. In: \versal{BRINK}, Stefan;
\versal{PRICE}, Neil (eds.). \emph{The Viking world}. London: Routledge, 2012,
pp. 588-597.

\item \versal{FERNANDES}, José Lucas Cordeiro;~\versal{CARDOSO}, Gleudson Passos;~\versal{SANTOS}, André
Luiz Campelo dos. A descoberta do horizonte: a cristianização dos
Vikings na América. \emph{Revista Brasileira de História das Religiões},
vol. 8, 2015, pp. 109-124.

\item \versal{GWYN}, Jones.~\emph{La saga del Atlántico Norte: establecimiento de los
vikingos en Islandia, Groenlandia y América}. Barcelona: Oikos-Tau, \versal{S.A.}
Ediciones, 1992.

\item \versal{RAFNSSON}, Sveinbjörn. The Atlantic Islands. In: \versal{SAWYER}, Peter (ed.).
\emph{The Oxford Illustrated History of the Vikings}. Oxford: Oxford
University Press, 2001, pp. 110-133.

\item \versal{SHAFER}, John Douglas. \emph{Saga accounts of norse far-travellers}.
Durham: Durham University, 2010.

\item \versal{THORGILSSON}, Ari; \versal{ANÔNIMO}. \emph{Íslendingabók, Kristni Saga: The book
of the icelanders, the story of the conversion}. Trad. Sion Gronlie.
Viking Society for Northern Research: University College of London,
2006.

\item \versal{UMBRICH}, Andrew. \emph{Early Religious Practice in Norse Greenland:
From the Period of Settlement to the 12 th Century}. Reykjavík:
Universidade da Islândia, 2012.
\end{itemize}
\section{\versal{LINDHOLM HØJE}}

Lindholm Høje é um dos principais locais de assentamento e depósito funerário
do Período Viking. Foram encontradas na região 682
sepulturas, situadas no monte de Voerbjerg, ao norte de Aalborg, atual
Dinamarca. A disposição do cemitério no monte onde se
encontra é fundamental para sua adequada compreensão. No topo do monte,
encontram-se os depósitos funerários mais antigos, datados do século~\versal{V}. 
Conforme se desce o monte, encontram-se os depósitos mais
contemporâneos, sendo os últimos datados do século~\versal{XI}. A questão
visibilidade era de suma importância. Os
habitantes da região se identificavam uns aos outros a partir da associação com
as linhagens representadas pelos corpos ali depositados. Ao mesmo tempo, a
manutenção da região e a preservação de suas memórias serviam para construir e
reconstruir as compreensões dos ancestrais.

Os depósitos funerários da região são demarcados externamente
por padrões de pedra e pelas formas das embarcações, com seus contornos 
triangulares, circulares ou ovais. Econtram-se também na região
depósitos funerários datados do Período Pré-viking. A partir dos padrões
de depósito, estabeleceu-se as embarcações com contornos triangulares como
associadas ao gênero masculino, enquanto os
contornos ovais ou circulares estariam associados ao gênero
feminino. Mesclam-se na região depósitos de inumação e cremação,
práticas estas que variavam em conformidade com o período histórico, cabendo anotar que a
prática da cremação tronou-se mais frequente que a inumação durante o Período Viking.

Com o tempo, a areia trazida à região pelos ventos oriundos da costa oeste cobriu
as bases do monte de Voerbjerg, soterrando o assentamento. Este
seria abandonado durante o século~\versal{XIII} e os depósitos só viriam a ser
descobertos em 1889, antes das primeiras escavações, ocorridas
em 1952. O soterramento favoreceu a depredação da região. Durante o
século~\versal{XIX}, ela foi local de extração de pedras para a construção de vias da região, o que contribuiu para danificar as estruturas externas dos depósitos funerários.
Em meio a tais depredações, os depósitos funerários do Período
Viking localizados na região mais baixa do monte sofreram depreciação maior do que os mais antigos, correspondentes à Idade do Ferro
germânica, que, por se encontrarem mais ao topo, foram
melhor conservados.

O local do assentamento foi revelado pelas pesquisas
arqueológicas. Situava-se às margens do canal de Limfjord, que dividia
a península da Jutlândia durante o Período Viking. O único ponto possível 
para atravessar o canal de Limfjord localizava-se em sua parte mais a oeste do fjord de
Aggersund. Graças a sua localização geográfica, o assentamento de Aalborg
tornou-se, inegavelmente, região de circulação de pessoas e bens. Os achados 
arqueológicos encontraram, inclusive, a presença de moedas árabes, dentre muitos outros artefatos. Um broche com estilo de Urne, datado do século~\versal{XI} e 
localizado em um dos depósitos funerários, foi
identificado como o modelo para cópias de bronze produzidas
por um joalheiro em Lund, atual Suécia, durante o século~\versal{XII}. Achados
dessa natureza indicam a relação da região com o leste escandinavo, com
os mares do báltico (local onde preponderaram os contatos com os povos
árabes) e com a região Lund.

\SIG{Munir Lutfe Ayoub}

Ver também Arqueologia da Era Viking; Cemitério de Borre; Dinamarca da
Era Viking; Funerais e enterros; Sepultamentos.

\begin{itemize}
\item \versal{BIRKEDAL}, Peter; \versal{JOHANSEN}, Erik. The eastern Limfjord in the Germanic
Iron Age and the Viking Period. \emph{Acta archaeological}, vol. 71, n.
1, 2000, pp. 25-33.

\item \versal{CHRISTIANSEN}, Torben Tier. Detektorfund og bebyggelse- Det ostlige
Limfjordsomrade i ynge jernalder og Vikingetid. \emph{Kuml}, vol. 57, n.
57, 2008, pp. 101-138.

\item \versal{LANGER}, Johnni. Erfi: As Práticas Funerárias na Escandinávia Viking e
Suas Representações. \emph{Brathair}, vol. 5, n. 1, 2012, pp. 114-127.

\item \versal{LERCHE}, Grith. Additional comments on the Lindholm Høje field.
\emph{Tools and Tillage}, vol. 4, n. 2, 1981, pp. 110-116.
\end{itemize}
\section{\versal{LINDISFARNE}}

Lindisfarne é uma pequena ilha situada na costa nordeste da Inglaterra,
no condado de Northumberland, o qual no começo da Idade Média
compreendia o Reino da Nortúmbria. A ilha é acessível por rota terrestre
durante a maré baixa. Mas, embora sua pequena dimensão territorial e seu
clima frio não fossem propícios para o desenvolvimento agrícola e rural na região,
Lindisfarne se destacou por seu mosteiro.

O norte da Inglaterra demorou a ser conquistado pelos romanos, quando seria 
efetivamente cristianizado. Inclusive, a Escócia, que ficava para além
da Muralha de Adriano, foi considerada pelos romanos como 
território hostil, pagão e bárbaro. Com o passar do tempo, os senhores
anglos que já haviam sido cristianizados decidiram expandir o
cristianismo pelo norte da Inglaterra e adentrar a Escócia.

Em 635, Santo Aidan (?-651) viajou para a Nortúmbria a convite do rei
Osvaldo, coroado em 634. O novo monarca tinha interesse de expandir a
sua fé pelo reino e, para isso, convidou o monge irlandês Aidan, o qual
à época vivia na ilha de Iona, na Escócia. Santo Aidan aceitou o
convite e se mudou para Lindisfarne, onde fundou um importante mosteiro
(nomeado posteriormente como Mosteiro de Santo Cuteberto), o qual ficou
conhecido por ser um centro copista da região, produzindo inclusive
cópias com ricas iluminuras dos Evangelhos.

Nos anos seguintes, o mosteiro prosperou e se tornou respeitado pelo
missionarismo na Nortúmbria e Mércia. Também era conhecido pelo seu trabalho
literário. Entretanto, o ano de 793 traria um acontecimento chocante
para a ilha. Nessa data, segundo narra os relatos históricos da \emph{Crônica Anglo-Saxã}
(\emph{Anglo-Saxon Chronicle}) -- redigidos a partir do
século~\versal{IX} --, o mosteiro de Lindisfarne foi alvo de uma terrível invasão
ocasionada por guerreiros brutos, vindos do mar. Um dos trechos da obra
diz o seguinte: ``793: neste ano apareceram presságios terríveis na
Nortúmbria, que assustaram muito as pessoas. Consistiam em imensos
torvelinhos e relâmpagos, e viam-se dragões chamejantes voando pelo ar.
Aqueles sinais foram imediatamente seguidos por uma época de grande
fome, e pouco depois, em 8 de junho do mesmo ano, os homens pagãos
destruíram a igreja de Deus em Lindisfarne, saqueando e matando''.

Esse breve relato é considerado o primeiro registro anglo-saxão que se
conhece acerca de um ataque viking a Lindisfarne. O acontecimento foi
tão impactante que ainda hoje alguns historiadores o consideram como o
marco de início da Era Viking (séculos~\versal{VIII-XI}). Foi a
primeira expedição de pilhagem viking a respeito da qual que se tem notícia.

Relatos saxões posteriores consideraram que o ataque a Lindisfarne teria sido
o prenúncio do Juízo Final, pois aqueles pagãos (que foram
considerados mais demônios do que homens) profanaram o mosteiro,
roubaram, destruíram e mataram. Tal ataque tornou-se ainda mais chocante
para a época por se tratar de um massacre ocorrido num mosteiro, contra
clérigos desarmados. O monge beneditino Alcuíno de York (c. 735-804), em
carta escrita ao rei Æthereld da Nortúmbria, mencionou que a destruição
do Mosteiro de São Cuteberto, em Lindisfarne, era um castigo de Deus,
devido ao fato do povo ter se distanciado de seus mandamentos. Tal ideia
foi mantida por muito tempo.

Quanto à descrição da terrível tempestade e supostos dragões que
voavam pelo céu, se cogitou, inicialmente, que poderia se referir 
à figura de proa das embarcações escandinavas. Mas, nas
últimas décadas, os escandinavistas apontaram que poderiam ser referências
a auroras boreais. Inclusive, uma dessas auroras teria ocorrido no ano de
793 e tal fenômeno, em algumas áreas da Europa, era visto como a ação de
dragões.

Não obstante, hoje se sabe que o ataque a Lindisfarne não foi
a primeira incursão nórdica ao arquipélago bretão, tampouco a primeira
atividade dos vikings fora da Escandinávia. O historiador James
Graham-Campbell aponta que possivelmente os nórdicos já mantinham
contatos comerciais com os saxões desde o século~\versal{VII}, pelo menos.

\SIG{Leandro Vilar Oliveira}

Ver também Era Viking; Inglaterra da Era Viking; Viking.

\begin{itemize}
\item \versal{ANÔNIMO}. \emph{The Anglo-saxon Chronicle}. Trad. Rev. James
Ingram. London: Everyman Press Edition, 1912.

\item \versal{ARBMAN}, Holger. \emph{Os Vikings}. Lisboa: Editorial Verbo, 1967.

\item \versal{BEARD}, Darren. Astronomical references in the Anglo-saxon Chronicles.
\emph{J. Br. Astronomic Association}, vol. 115, n. 5, 2005, pp. 261-264.

\item \versal{GRAHAM-CAMPBELL}, James (org.). \emph{Os vikings}. Barcelona: Folio \versal{S.A.},
2006.

\item \versal{HOLMAN}, Katherine. \emph{Historical dictionary of the vikings}. Lanham:
Scarecrow Press Inc, 2003.

\item \versal{KEYNES}, Simon. The Vikings in England, c. 790-1016. In: \versal{SAWYER}, Peter
(ed.). \emph{The Oxford Illustrated History of the Vikings}. New York:
Oxford University Press, 1997, pp. 48-82.
\end{itemize}
\section{\versal{LINGUAGEM}}

Ao estudar a Era Viking nos deparamos com um outro desafio, que é a
coleta dos testemunhos linguísticos. Duas principais fontes podem ser
utilizadas para um estudo linguístico: as fontes concretas e a
reconstrução linguística. A primeira, como o próprio nome já nos diz, é
possibilitada por meio de documentos, objetos arqueológicos etc.; a
segunda é obtida mediante o método comparativo entre as características
linguísticas, preferencialmente documentadas, de fases posteriores dos
dialetos da mesma família. Por intermédio de tal comparação, surge uma
forma reconstruída, que seria, teoricamente, o embrião das formas que
viriam a surgir nos dialetos posteriores. Neste verbete, apresentaremos
apenas as fontes linguísticas da Era Viking.

As
sincronias anteriores à Era Viking (750-800 d.C.-1050 d.C.) seriam o
Germânico do \emph{continuum} norte-oeste (- 200 d.C.), o Rúnico
Primitivo (200-500 d.C.) e a Era das Síncopes (500-750 d.C.); ao passo
que as sincronias posteriores seriam o Nórdico Antigo (1050-1350 d.C.,
que teria uma divisão entre nórdico antigo do oeste, norueguês e
islandês antigos, nórdico antigo do leste, dinamarquês e sueco
antigos).

De acordo com Spurkland, o principal meio de registro é a
pedra. O conteúdo da mensagem, na maioria das vezes, visa a memória de
alguém que faleceu. Tais inscrições não eram colocadas apenas sobre o
túmulo, mas também em lugares públicos ou sobre pontes e vias para que
os transeuntes pudessem lê-las. O autor também afirma que no período do
Rúnico Primitivo (200-500 d.C.), a prática era mais comum na Noruega.
Porém, no final desse período, a prática sofreu um declínio na Noruega,
passando a ser comum no Sul da Suécia. Na Era Viking, no entanto, foi
a Suécia que começou a produzir estelas rúnicas em grande escala. Tal
tendência começou no Sul e se espalhou para o Norte, alcançando seu auge
em Uppland, local em que há registro de mais de mil estelas rúnicas da
Era Viking e do início da Idade Média. Tal número é muito alto, sobretudo se
comparado às cerca de cinquentas estelas rúnicas da Era Viking
encontradas na Noruega.

No que tange ao modelo sintático clássico de escritura, Spurkland
esclarece: ``\versal{X} erigiu essa estela em memória de \versal{Y}'' como, por
exemplo, podemos observar na estela rúnica da comuna de Søgne, no condado de
Vest-Agder (código \versal{N}211, entre 1000 e 1050 d.C.):
\textart{a}\textarc{u}\textarc{i}\textarc{n}\textart{t}\textarc{r}:\textarc{r}\textarc{i}\textart{\R}\textart{t}\textarc{i}:\textart{\R}\textart{t}\textarc{i}\textarc{n}:\textarc{\th}\textarc{i}\textarc{n}\textart{a}:\textart{a}\textarc{f}\textart{t}\textarc{i}\textarc{r}:\textarn{k}\textarc{u}\textarc{n}\textarc{u}\textart{a}\textart{t}:\textart{\R}\textarc{u}\textarc{n}\textart{\R}\textarc{n}
\textbf{auintr:risti:stin:þina:aftir:kunuat:sunsn}. Lê-se, em nórdico antigo
\emph{Eyvindr reisti stein þenna eptir Gunnhvat, sun sinn}
(``Eyvindr erigiu essa pedra em memória de Gunnhvatr, seu filho'', trad.
nossa). Essa estela rúnica está escrita no alfabeto novo \emph{futhark}.
O antigo \emph{futhark} sofreu muitos
desenvolvimentos até chegar nesse novo modelo e suas variantes (consulte
o verbete \versal{Alfabeto rúnico}). As estelas rúnicas registradas por
meio do novo \emph{futhark} representam apenas uma parte de toda a
complexidade linguística do período, uma vez que testemunhos escritos
não representam a fala com todas as suas variantes. Na
Era Viking, como em qualquer outro período, havia dialetos. Ademais,
apenas as características linguísticas de pessoas alfabetizadas são
refletidas nas estelas rúnicas em disposição nos dias de hoje.

Uma característica linguística desenvolvida nesse período é o emprego do ``artigo definido
enclítico'', que não existente nas outras línguas germânicas. De acordo
com Skard, não é claro qual pronome deu origem a esse
artigo, mas nos documentos mais antigos aparecem as formas \emph{inn} e
\emph{enn}, como, por exemplo, \emph{karl inn} (``o homem'') e \emph{Ormr
inn langi} (``a grande serpente''). O autor afirma que, na poesia escáldica,
o artigo definido enclítico deve ter surgido por volta de 900 d.C. e
passou a ser usado na língua falada desde então. Isso não quer dizer que
haja documentos escritos nesse período que comprovem tal afirmação. Esta
só pode ser considerada hipoteticamente, por conta das cópias de manuscritos
encontradas séculos depois. Johnsen afirma que a estela
rúnica \emph{Ekilla bro} (Bålsta, Uppland, Suécia, \versal{U}644, 1000-1040 d.C.) é a
mais antiga com esse artigo definido enclítico: [...]
\textara{j}\textarn{a}\textarc{n}:\textarc{f}\textarc{i}\textarc{l}:\textarn{a}\textarc{u}\textarn{s}\textarc{t}\textarc{\th}:\textarc{i}\textarn{k}\textarc{u}\textarn{a}\textarc{r}\textarc{i}\textarn{k}\textarc{u}\textarc{\th}\textara{j}\textart{\h}\textarn{a}\textarc{b}\textarc{i}\textarn{\A}\textarc{n}\textarc{t}\textarc{i}\textarc{n}\textarc{i}
\textbf{[...] han:fil:austr:miþ:ikuari kuþ heabi ontini}. Tal
inscrição teria a seguinte transcrição em antigo nórdico, de acordo com
a agência de herança cultural \emph{Swedish National Heritage Board}:
\emph{[...] Hann fell austr með Ingvari. Guð hialpi andinni}, com a
seguinte tradução proposta por nós: ``Ele sucumbiu no Leste com Ingvarr.
Deus ajude o espírito (dele)''. Vemos aqui a utilização do substantivo
\emph{ontini} (\emph{andinni}, ``o espírito''), que é a forma definida
enclítica do gênero feminino no caso dativo, embora no nórdico antigo e
islandês moderno a palavra seja de gênero masculino (i.e.
\emph{andanum}). Na Noruega, em Aust-Agder, também há a inscrição
\emph{Storhedder}, de 1100 d.C., com o artigo definido enclítico, mas
estaria fora do período da Era Viking.

Skard conclui que o posicionamento enclítico do artigo não
chegou, por exemplo, no Oeste e no Sul da Jutlândia (Dinamarca), porque
lá haveriam formas como \emph{æ mann} (``o homem''). De acordo com
Brøndum-Nielsen, não há registros do artigo definido
enclítico em runas dinamarquesas.

Uma segunda diferença dialetal que se iniciou em período muito
anterior (início do rúnico primitivo) e que teve consequências nos
dialetos da Era Viking é a mutação vocálica causada pela vogal ``a''
não acentuada. Esta influencia as vogais átonas ``u'' e ``i'' no
meio da palavra, resultando nos sons /o/ e /e/. Trata-se, pois, de uma
assimilação regressiva à distância. Tal desenvolvimento, que ocorreu
nos dialetos germânicos, não correu em algumas partes da Escandinávia.
De acordo com Seip, esse processo ocorreu mais na parte
Oeste: \emph{golv} (``chão''), \emph{bod} (``mensagem''), \emph{hol} (``oco''),
\emph{skote} (``atirado''), \emph{broten} (``quebrado'') e \emph{kolle}
(``topo de uma montanha''). No entanto, no escandinavo do Leste há
\emph{gulv} (``chão'' no dinamarquês), \emph{bud} (dinamarquês e sueco), \emph{hul} (dinamarquês)
e \emph{skudt} (dinamarquês), \emph{bruten} (``quebrado''), \emph{kulle} (``topo''),
\emph{skudt} (dinamarquês), \emph{brutit} (sueco) e \emph{kulle} (sueco).

Uma terceira diferença dialetal são as mutações vocálicas causadas
tanto pelos sons /i/ e /u/, que também são assimilações
regressivas à distância e, da mesma maneira, causaram consequências
dialetais na Era Viking. A mutação vocálica causada pelo som /u/, a
propósito, ocorreu exatamente nesse período. De
acordo com Skard, a primeira não ocorreu em toda a Escandinávia e a
segunda teve mais força no islandês e menos no sueco e dinamarquês.
O autor também afirma que a primeira foi propagada fora da área
nórdica, com exceção do gótico, ao passo que a segunda é um
fenômeno exclusivamente nórdico. O som /i/, que também pode
ser representado por ``i'' ou ``j'', influenciou de maneira regressiva as
três vogais longas traseiras ``u'', ``o'' e ``a'', que se
transformaram em três novos fonemas: ``y'', ``ø'' e ``æ''.
Exemplo: a palavra germânica \emph{flutjan} (``mover-se'') $>$
norueguês/islandês antigo \emph{flytja} ≅ dinamarquês antigo
\emph{flytje}; O termo germânico \emph{gastimaz} (dativo plural, tema em \emph{i},
``aos convidados'') $>$ \emph{gastimz} $>$
\emph{gestimz} $>$ \emph{gestimR} ≈
\emph{\href{https://en.wikipedia.org/wiki/\textarc{g}}{\textarc{g}}\href{https://en.wikipedia.org/wiki/\textarc{e}}{\textarc{e}}\href{https://en.wikipedia.org/wiki/	extarn{I}}{	extarn{I}}\href{https://en.wikipedia.org/wiki/\textarc{t}}{\textarc{t}}\href{https://en.wikipedia.org/wiki/\textarc{u}}{\textarc{u}}\href{https://en.wikipedia.org/wiki/\textarc{m}}{\textarc{m}}}\textarc{\R}
\textbf{gestumR} (estela rúnica de Stentoften, Blekinge, Suécia, 600-650
d.C.) $>$ antigo nórdico \emph{gestum}. À propósito, todas as
classes substantivais se convergiram para a terminação dat.pl.
\emph{-um}, por isso a variante ``im\versal{R}'' ≈ \textbf{um\versal{R}} (-\emph{umaz}
$>$ -umR $>$), originária dos temas em
\emph{u} ou de temas consonantais.

Como há casos em que a mutação causada pelo /i/ não ocorre, Skard
formula quatro períodos em que esse desenvolvimento
ocorreu na história das línguas escandinavas e Kock (1911-1916). 
Dois exemplos dessa mutação são: germânico \emph{fots} (``pé'')
$>$ rúnico primitivo \emph{fotR} $>$ antigo
nórdico \emph{fotr}. Mas o germânico \emph{fōtiz} (``pés'') \textgreater{}
rúnico primitivo *\emph{fōtiR} \textgreater{} antigo nórdico
\emph{fætr}; germ. *\emph{farō} ``eu viajo/vou'', *\emph{farizi} ``você
viaja/vai`` e *\emph{faridi} ``ele viaja/vai''; no entanto, em antigo
norueguês/islandês é \emph{fer}, \emph{ferr} e \emph{ferr},
respectivamente. A mutação nos verbos fortes da classe \versal{VI} no tempo
presente singular ocorreu especialmente no Escandinavo do Oeste; de
acordo com Enger \& Conzett (2016, p. 258), a única forma herdada é a da
segunda pessoa, em que ocorreu a mutação causada pelo *[i], quedas
vocálicas e transformação do -\emph{z}- em -\emph{r}-; e a esta forma as
duas outras se assimilaram.

A respeito da mutação vocálica causada pelo *[u], desenvolvimento
nórdico por excelência, mas também com muito mais efeito na parte Oeste, 
o som *{[}u{]}
influencia na ocorrência do arredondamento da vogal anterior
(especialmente o som *[a] curto ou longo) e, assim, como resultado
surge um som *[ɒ] aberto, escrito \emph{ǫ} ou \emph{ǫ́} (rúnico
primitivo \emph{*landu} \textgreater{} antigo nórdico \emph{lǫnd}
``terras''), em vista disso, há palavras como \emph{barn} ``criança'' e
\emph{bǫrn ``crianças''. }

A mutação causada pelo \versal{\emph{R}} final também é um processo que ocorreu
na Era Viking. Essa consoante, no entanto, diferentemente das
anteriores, causa a mutação por assimilação de contato em uma vogal
anterior. Esse som se desenvolveu do antigo *{[}z{]}, que por sua vez,
de acordo com a Lei de Verner, se desenvolveu do *{[}s{]}. Este processo
ocorreu apenas na parte oeste e em alguns dialetos suecos. Skard 
exemplifica com germ. *\emph{glaza} ``vidro'' \textgreater{}
rúnico primitivo *\emph{glaRa} \textgreater{} \versal{AI} e \versal{AN} \emph{gler} (em
Din. \emph{glar; glarmester} ``vidraceiro''); germ. *s\emph{ū}-
\textasciitilde{} *\emph{suw}- ``porca'' \textgreater{} rúnico primitivo
*\emph{suR} \textgreater{} antigo islandês e norueguês \emph{sýr}, mas
dinamarquês e sueco \emph{so} (*ū \textgreater{} *ō). As formas do
rúnico primitivo são propostas por Bjorvand \& Lindeman.

A ruptura vocálica, que também já havia desde a Era das Síncopes, é um
processo que afetou, no entanto, mais os dialetos do leste do que do
oeste. A vogal \emph{a} ou \emph{u} não acentuada
no final da palavra, que pode sofrer apócope ou não, causa influência no
\emph{e} curto na sílaba tônica de maneira que ele se ditonga. O novo
ditongo se aproxima da vogal que causou a ruptura: *{[}ea{]}
\textgreater{} *{[}ia{]}, com influência do *{[}a{]}, e *{[}eu{]}
\textgreater{} *{[}eo{]} \textgreater{} *{[}io{]}, com influência do
*{[}u{]}. Por isso há no antigo norueguês \emph{stela}, no antigo
dinamarquês \emph{stiælæ} e no antigo sueco \emph{stiala} ou
\emph{stiælæ} ``roubar'', mas há, por exemplo, em antigo
islandês/norueguês \emph{hjarta} e em antigo dinamarquês \emph{hjerte.}
Noreen, cita três períodos de ruptura vocálica, entre
650-900 d.C., por volta de 900 d.C., e após 900 d.C.

No entanto, não se pode falar em reais limites dialetais, pois eles não
são bem recortados e um exemplo disso é a monotongação, que é típica do
Escandinavo do Leste. A respeito desse processo, os sons *{[}ai{]} e
*{[}au{]} (também *{[}øy{]}) do germânico, grafados como \emph{ai} e
\emph{au} no Rúnico Primitivo, se monotongam em \emph{e} e \emph{o}
longos, *{[}ē{]} e *{[}ō{]}. Este processo ocorreu no saxão e frísio
antigos, no Noroeste da Alemanha, na Dinamarca e na Suécia. Na Noruega,
o fenômeno não foi muito relevante: ocorreu com mais frequência antes de
consoantes longas ou geminadas como, por exemplo, na palavra superlativa
\emph{mestr} (\versal{RP} \textless{} *\emph{maistaz}) ``o/a mais'', mas não
ocorreu na forma comparativa \emph{meiri} (\versal{RP} \textless{} *\emph{maizo})
``o/a mais que''. Na Islândia, por fim, o fenômeno não teve nenhuma
influência. Um exemplo da transformação de \emph{ai} para \emph{e} seria
pode ser dado com a palavra ``pedra'': \textarn{s}\textarc{t}\textarc{a}\textarc{i}\textarc{n}\textarc{a} (\textbf{staina}, Tune,
Noruega, \versal{N KJ}72 \versal{U}, 400 d.C.) e \textbf{{[}s{]}tain{[}a{]}} na estela
rúnica sueca Rö (Bohuslän, 400 d.C., código \versal{B}o \versal{KJ}73 \versal{U}), ambas no caso
acusativo singular. Na Era Viking, há testemunhos de \textart{\R}\textart{t}\textart{a}\textarc{i}\textarc{n}
\textbf{stain} (no acusativo singular, Pedra Kuli, Edøy, Møre og
Romsdal, Noruega, entre 1000 e 1030 d.C., código \versal{N}449); por outro lado,
tanto na estela rúnica sueca da igreja Angarn (Uppland século~\versal{XI}, código
\versal{U}201) quanto na estela rúnica dinamarquesa Asferg, Aarhus (970-1020
d.C.) há a palavra \textarn{s}\textarc{t}\textarc{i}\textarc{n} \textbf{stin}, masculina no caso acusativo). No
entanto, diferentemente do esperado, o ditongo se manteve na Gotlândia e
se monotongou na parte leste da Noruega.

Outros processos, citados pelo autor, que se iniciaram na sincronia da
Era das Síncopes e deram continuidade na sincronia da Era Viking são as
assimilações consonantais. São elas: \emph{mp} \textgreater{} \emph{pp},
\emph{nt} \textgreater{} \emph{tt} e \emph{nk} \textgreater{} \emph{kk};
essas três assimilações foram realizadas muito mais nas línguas
escandinavas do oeste do que do leste, em que tal processo ocorre com
mais frequência no final de palavras. Há também assimilações que
ocorreram por toda a Escandinávia: \emph{ht} \textgreater{} \emph{tt},
\emph{lþ} e \emph{nþ} para \emph{ll;} \emph{lR}, \emph{nR}, \emph{sR}
para \emph{ll}, \emph{nn}, \emph{ss}, respectivamente. Um exemplo:
rúnico primitivo
\emph{\href{https://en.wikipedia.org/wiki/\textarc{h}}{\textarc{h}}\href{https://en.wikipedia.org/wiki/\textarc{a}}{\textarc{a}}\href{https://en.wikipedia.org/wiki/\textarc{i}}{\textarc{i}}\href{https://en.wikipedia.org/wiki/\textarc{t}}{\textarc{t}}\href{https://en.wikipedia.org/wiki/\textarc{i}}{\textarc{i}}\href{https://en.wikipedia.org/wiki/\textarc{n}}{\textarc{n}}\href{https://en.wikipedia.org/wiki/\textarc{a}}{\textarc{a}}\href{https://en.wikipedia.org/wiki/\textarc{\R}}{\textarc{\R}}}
\textbf{haitinaR} ``chamado'', na estela de Kalleby, Suécia, século~\versal{V}
\textgreater{} *\emph{heitinR} \textgreater{} \emph{heitinn}. As quedas
de algumas consoantes também ocorreram nesse período. Tanto Seip
quanto Skard citam a queda das consoantes \emph{h}
e \emph{v}. A primeira, em posição inicial e a segunda em posição
inicial e antes de \emph{r}, que certamente ocorreu no século~\versal{IX}. O autor
também cita a síncope do \emph{v} após sílabas longas: nos tempos
literários encontra-se tanto \emph{Noregr} (\textless{}
*\emph{Norðvegr}) e, a forma mais rara, \emph{Norvegr} ``Noruega''.

Um sufixo para a formação de agentes nominais que aparece nesse período
é o -\emph{are}, ou também -\emph{ari}, possivelmente via inglês antigo,
que emprestou do latim -\emph{arius} num período ainda anterior:
\emph{leikarar} ``músicos, comediantes'' (Torbjørn Honrklove). Com
relação a empréstimos no período podemos citar: \emph{stræti} ``rua''
(do inglês antigo, pois há no Beowulf \textless{} latim strata),
\emph{bátr} ``bote'' (do antigo frísio, ao lado da forma herdada
\emph{beitr}, que aparece mais em antigas poesias; a forma no germânico
é *\emph{baitaz}. Pressupõe que é um empréstimo porque o ditongo germ.
*{[}ai{]} não se transforma em *{[}ā{]} no rúnico primitivo, mas em
*{[}ei{]}. É também possível que a palavra do antigo nórdico
\emph{frjádagr} ``sexta-feira'' seja um empréstimo do antigo frísio
(\textless{} *\emph{fríadagr}, do antigo frísio \emph{frīadei}
\textless{} germ. *\emph{frijjōz} + *\emph{dagaz} ''dia de Frigg''),
caso contrário ela deveria ser \emph{*friggjar-dagr}, uma vez que o
germânico -\emph{jj}- se transforma em -\emph{ggj}- em nórdico (compare
germ. *\emph{tvajjǫ}, genitivo de `` dois'' \textgreater{} antigo
nórdico \emph{tveggja}, antigo saxão \emph{tweio} e antigo alto alemão
\emph{zweio}).

A respeito da sintaxe, Mørck afirma que, uma vez que
a quantidade de inscrições não é muito grande, a interpretação sintática
é sempre um desafio e, além do mais, por serem muito curtas tais
inscrições, não é possível dizer muito sobre as suas construções
sintáticas: quase sempre são constituídas de apenas palavras soltas ou
de dois elementos de oração, encontra-se poucas orações adverbiais e
verbo no infinitivo. Assim, a maioria das sentenças rúnicas são orações
principais com apenas verbos conjugados e com eventuais orações
subordinadas adverbiais ou nominais. Exemplo de uma sentença por meio da
estela rúnica de Alstad~\versal{I} (Hof, Oppland, Noruega, 1000 d.C., código
\versal{N}61):
\emph{\href{https://en.wikipedia.org/wiki/\textarc{i}}{\textarc{i}}\href{https://en.wikipedia.org/wiki/\textarc{u}}{\textarc{u}}\href{https://en.wikipedia.org/wiki/\textarc{r}}{\textarc{r}}\href{https://en.wikipedia.org/wiki/\textarc{u}}{\textarc{u}}\href{https://en.wikipedia.org/wiki/\textarc{n}}{\textarc{n}}}:\href{https://en.wikipedia.org/wiki/\textarc{r}}{\emph{\textarc{r}}}\textarn{a}\href{https://en.wikipedia.org/wiki/\textarc{i}}{\emph{\textarc{i}}}\textarn{s}{[}\href{https://en.wikipedia.org/wiki/\textarc{t}}{\emph{\textarc{t}}}{]}\href{https://en.wikipedia.org/wiki/\textarc{i}}{\emph{\textarc{i}}}{[}:{]}\textarn{s}{[}\href{https://en.wikipedia.org/wiki/\textarc{t}}{\emph{\textarc{t}}}{]}\textarn{a}\emph{\href{https://en.wikipedia.org/wiki/\textarc{i}}{\textarc{i}}\href{https://en.wikipedia.org/wiki/\textarc{n}}{\textarc{n}}}:\emph{\href{https://en.wikipedia.org/wiki/\textarc{\th}}{\textarc{\th}}\href{https://en.wikipedia.org/wiki/\textarc{i}}{\textarc{i}}\href{https://en.wikipedia.org/wiki/\textarc{n}}{\textarc{n}}}\textarn{a}:\textarn{a}\href{https://en.wikipedia.org/wiki/\textarc{f}}{\emph{\textarc{f}}}{[}\href{https://en.wikipedia.org/wiki/\textarc{t}}{\emph{\textarc{t}}}{]}\href{https://en.wikipedia.org/wiki/\textarc{i}}{\emph{\textarc{i}}}\textarc{\R}{[}:{]}\textarn{a}\href{https://en.wikipedia.org/wiki/\textarc{u}}{\emph{\textarc{u}}}\textarn{a}\emph{\href{https://en.wikipedia.org/wiki/\textarc{u}}{\textarc{u}}\href{https://en.wikipedia.org/wiki/\textarc{n}}{\textarc{n}}}\textbf{iurun:rais{[}t{]}i{[}:{]}s{[}t{]}ain:Þina:af{[}t{]}iR{[}:{]}auaun,}
{[}\emph{Jórunn reisti stein þenna eptir}{]}, ``Jorunn erigiu essa
estela em memória de'' (trad. nossa).

\SIG{Yuri Fabri Venancio}

Ver também Heiti; Kenning; Linguagem; Literatura; Norreno; Poesia
éddica; Poesia escáldica.

\begin{itemize}
\item \versal{ANTONSEN}, Elmer H. \emph{A Concise Grammar of the Older Runic
Inscriptions}. Tübingen: Max Niemeyer Verlag, 1975.

\item \versal{BJORVAND}, Harald; \versal{LINDEMAN}, Fredrik O. \emph{Våre arveord: etymologisk
ordbok}. Oslo: Instituttet for sammenlignende kulturforskning, 2000.

\item \versal{ENGER}, H. O.; \versal{CONZETT}, P. Morfologi. Fonologi. In: \versal{SANDØY}, Helge (org.).
\emph{Norsk Språkhistorie. Mønster}. Oslo: Novus Forlag, 2016, pp.
215-315.

\item \versal{KOCK}, Axel. \emph{Umlaut und Brechung im altschwedischen. Eine Übersicht}.
Lund: C.W.K. Gleerup, 1911-1916, (Lunds universitets årsskrift).

\item \versal{KRISTOFFERSEN}, Gjert; \versal{TORP}, Arne. Fonologi. In: \versal{SANDØY}, Helge (org.).
\emph{Norsk Språkhistorie. Mønster}. Oslo: Novus Forlag, 2016, pp.
101-212.

\item \versal{JOHNSEN}, Ingrid S. \emph{Stuttruner i vikingtidens innskrifter}. Oslo:
Universitetsforlaget, 1968.

\item  \versal{MØRCK}, Endre. Syntax. Fonologi. In: \versal{SANDØY}, Helge (org.). \emph{Norsk
Språkhistorie. Mønster}. Oslo: Novus Forlag, 2016, pp. 319-445.

\item \versal{NOREEN}, Adolf. Altnordische Grammatik~\versal{I}. Altisländische und
altnorwegische Grammatik. 4. Auflage. In \versal{HTML} umgearbeitet von Aldrea de
Leeuw van Weenen, 2010. Disponível em:
\href{http://www.arnastofnun.is/solofile/1016380}{\emph{http://www.arnastofnun.is/solofile/1016380}}. Acesso em: 4 dez. 2017.

\item \versal{SEIP}, Didrik A. \emph{Norsk språkhistorie til omkring 1370}. Oslo:
Aschehoug, 1955.

\item \versal{SKARD}, Vemund. \emph{Norsk språkhistorie. Band 1: til 1523}. Oslo:
Universitetsforlaget, 1973.

\item \versal{VENANCIO}, Yuri Fabri. \emph{Um estudo etimológico de internacionalismos:
cognatos nas línguas portuguesa e norueguesa}. Dissertação
(mestrado em Filologia e Língua portuguesa) -- Faculdade de Filosofia,
Letras e Ciências Humanas. São Paulo: Universidade de São Paulo (\versal{USP}),
2017.

\item \versal{VIKØR}, Lars S.; \versal{TORP}, Arne. \emph{Hovuddrag i norsk språkhistorie}.
Oslo: Ad Notam Gyldendal, 1995.
\end{itemize}
\section{\versal{LITERATURA}}

A literatura da Era Viking pode ser apreciada por meio de dois tipos de
fontes: as constituídas no período e as constituídas em período
posterior, cujo conteúdo supostamente foi composto, talvez oralmente, na
Era Viking e, por isso, retrata a expressão literária da época. A
primeira é encontrada apenas em inscrições rúnicas gravadas em pedras; a
segunda, por sua vez, pode ser encontrada nas escrituras em
pergaminho.

De acordo com Jesch, mesmo que haja convincentes
indicações da continuidade de formas poéticas desde a Era Viking até
períodos muito posteriores, as evidências materiais dessa Era é limitada
a um pequeno número de inscrições grafadas em estelas rúnicas, normalmente destinadas
a homenagear alguém. Em vista disso, a autora afirma ser
necessário considerar evidências escritas em épocas posteriores
(principalmente na Islândia medieval) para que seja possível apreciar a
completude da expressão poética daquele período. Portanto, tanto as
sagas (que contêm narrativas sobre o cotidiano de reis, chefes, amigos
etc.) quanto a poesia encontrada nos manuscritos não seriam literaturas
compostas na Era Viking, mas em período posterior, que remetem
ou não à Era Viking. Assim, em termos métricos e linguísticos, a poesia
dos vikings é apenas um pedaço de uma história muito mais longa da
poesia escandinava. Esta pode ser delimitada de pelo menos 400 d.C. até cerca
de 1500 d.C.

Considera-se tal história literária a partir de 400 d.C. em razão da presença, nesse período,
de inscrições compostas em verso aliterativo, que tem como
principal modelo o \emph{fornyrðislag}. A aliteração é a repetição de
sons consonantais em início de palavra ou sílaba tônica e o verso
aliterativo consiste de dois semiversos (curtos), separados por uma
cesura. Há uma ou duas letras aliterativas no primeiro semiverso, que
aliteram com a primeira sílaba tônica no segundo semiverso
(\emph{Merriam-Webster's Encyclopedia of Literature}, p. 36). Duas
estelas rúnicas que, de acordo com Jesch, apresentam
esse verso aliterativo são a inscrição de Tune (Østfold, Noruega, 400
d.C.), na métrica \emph{ljóðaháttr}, e a inscrição rúnica do
\emph{Chifre de Ouro de Gallehus} (Gallehus, Møgeltønder, Jutlândia
Dinamarca, século~\versal{V}), na métrica \emph{fornyrðislag}: Ek Hlewagastiz
Holtijaz//horna tawido (em rúnico primitivo) ``Eu, HlewagastiR, filho de
HoltiR, fiz o chifre`` (tradução nossa).

Nessa inscrição há quatro sílabas tônicas, marcadas por \versal{X}; o \versal{Y} indica
uma sílaba tônica secundária: as três primeiras fazem aliteração com o
som /\emph{h}/: \emph{hlewagastiz}, \emph{holtijaz} e \emph{horna}.
Esse padrão é encontrado em versos de períodos muito posteriores.

Após a conversão da Islândia ao cristianismo, a escrita em alfabeto
latino, a pena, a tinta e o pergaminho foram inseridos no país e,
consequentemente, houve um florescimento de uma cultura literária. De
acordo com Whaley, os temas históricos foram um dos
primeiros motivos para o despertar do interesse literário. Uma
prova disso é a obra \emph{Íslendingabók} (\emph{Livro dos Islandeses}), de
Ari Þorgilsson, que é a obra mais antiga da Islândia medieval. Tómasson
reforça a importância dada pela Islândia ao seu
passado histórico com o fato de que muitos dos maiores intelectuais
islandeses eram conhecidos como \emph{inn fróði} (``o sábio, o erudito'')
ou também pela forma com sentido aproximado \emph{inn vitri}.
Ademais, Whaley afirma que a
Islândia, nos primeiros séculos, chegou até mesmo a exportar
historiadores para atuarem como escaldos. Estes, por volta do ano
1000, se tornaram os principais provedores de propaganda e comemorações
poéticas nas cortes reais da Escandinávia.

A respeito das manifestações poéticas, Jesch complementa
ao afirmar que enquanto a obra \emph{Íslendingabók}, junto com as sagas
islandesas, é um testemunho da novidade e da distinção da Islândia com
sua cultura literária, a poesia, por sua vez, sempre testifica seus laços remotos com
a história e a cultura da Noruega, que seria a pátria originária dos
islandeses. Ross afirma que uma farta quantidade de poesias dos
mais diferentes tipos está preservada nos manuscritos islandeses do
século~\versal{XIII} e dos séculos posteriores. 
Ademais, segundo Jesch, muitas dessas poesias foram compostas
no momento da escrita ou pelo menos no período literário que se inicia a partir do
século~\versal{XII}. No entanto, ainda assim, é bem evidente que uma grande parcela
delas, que estão preservadas nos manuscritos, têm suas origens na Era
Viking. Também é bem possível que algumas dessas antigas poesias
sejam reproduções fieis à poesia oral dos vikings.

Seip considera que, embora as poesias escáldicas tenham
suas origens em períodos mais antigos do que os dos manuscritos que as
contêm, elas provavelmente se mantiveram relativamente intocadas, uma
vez que eram compostas de acordo com um esquema métrico muito restrito.
Por conta disso e até certo ponto, é possível reconstruir as formas
originais de palavras a partir dos versos. Por fim, Jesch
atesta que o principal problema é identificar quais versos islandeses
medievais se originaram na Era Viking e, então, determinar quão fieis
eles são comparados aos antecedentes orais.

Em vista disso, concordando com Jesch, pode-se afirmar
que uma importante atividade literária da Islândia era preservar e
registrar tradições orais antigas: históricas, mitológicas e poéticas.
Gunnell e Whaley dividem a literatura medieval islandesa
em duas categorias principais: poesia éddica e poesia
escáldica. No entanto, Jesch admite
que, apesar das categorizações binárias acarretarem em simplificações
de um \emph{corpus} muito maior e mais diversificado, elas são úteis
para refletir sobre as origens e contextos da poesia na Era Viking, bem como sua
transmissão para o período literário.

A poesia éddica tem esse nome por conta de um manuscrito geralmente
conhecido como \emph{Codex Regius da Poesia Éddica} (embora tenha
emprestado esse nome da \emph{Edda}, de Snorri Sturluson, que também é
uma fonte medieval importante da poesia antiga). Ela é quase
sempre anônima, com temas e lendas antigas, baseada em histórias
germânicas da era da migração na Europa continental e não em proezas
históricas de reis escandinavos. Neckel \& Kuhn,
Larrington e Hallberg afirmam que tal manuscrito,
composto na Islândia em 1270, contém uma coleção, ou até mesmo uma
antologia de vinte e nove poemas compreendidos como édicos, que tratam de
temas míticos e heroicos, apresentando uma variedade de métricas, incluindo a
\emph{fornyrðislag} e a \emph{ljoðaháttr}.

Jesch também questiona a idade desses poemas e se eles se
originaram na Era Viking, perguntas que são muito pertinentes. Também
afirma que, no século~\versal{XIII}, mesmo com a Islândia já cristianizada, os copistas
ainda registravam poemas de conteúdo pagão correspondentes a uma era pré-cristã
provida tanto de elementos mitológicos escandinavos como também de heróis do
período da Migração. A autora faz uma conexão desse período com a Era
Viking, uma vez que há estelas antigas que testemunham o
conhecimento tanto da forma quanto do conteúdo dos versos édicos. Um
exemplo dado por ela é a estela Rök 
(Östergötland, Suécia, código \versal{Ö}g 136, 800 d.C., portanto Era
Viking). Abaixo um trecho da estela, que
está inscrita em novo \emph{Futhark} de ramos curtos: \textbf{[...]
raiþiaukR hin þurmuþi stiliR flutna strontu hraiþmaraR sitiR nu o kuta
sinum skialti ub fatlaþR skati marika [...]} (``Teodorico, o audaz,
rei dos guerreiros do mar, reinou as costas do mar de Reid. Ele senta
agora armado em seu cavalo gótico, com o escudo atado, o chefe dos
Mærings'').

De acordo com Jesch, esse trecho da estela de Rök se encontra
na métrica \emph{fornyrðislag}, ``a métrica das declarações antigas'', e
faz uma alusão a lendas heroicas. Como apresentado no verbete
\versal{Poesia éddica}, o primeiro verso longo tem aliteração em \emph{Þ}, que é
a primeira sílaba tônica que cai em um substantivo. A primeira
semiestrofe começa com \emph{Reð} e a segunda, com \emph{Sitir}. Cada
semiverso é separado por \versal{ӀӀ} e também há aliteração em cada um dos quatro
pares aliterativos, que formam quatro versos longos: \emph{þ},
\emph{st}, \emph{g} e \emph{sk}, respectivamente. Nessa métrica, há um
número variável de sílabas, razão pela qual elas não são
contadas.

Wessén e Gade, no entanto, consideram
que essa inscrição está composta na métrica \emph{kviðuháttr} (``métrica
do poema''), que foi muito importante, particularmente entre
os séculos~\versal{IX} e \versal{X}. Após a métrica
\emph{dróttkvætt}, a métrica \emph{kviðuháttr} é a mais importante
desenvolvida a partir do \emph{fornyrðislag,} e tem a característica de
alternar versos de três e quatro sílabas em linhas ímpares e pares,
respectivamente, sem rítmo regular.
Essa métrica tem similaridades
tanto com os versos édicos quanto com os versos escáldicos, mas é melhor
classificada como escáldica, pois, embora dispense rima interna, as
sílabas são contadas e há alguns \emph{kennings}.

No poema da estela de Rök, percebemos aliterações entre \emph{þ} no
primeiro par, \emph{st} no segundo par, \emph{s} no terceiro par e
\emph{sk} no quarto par. Embora seja possível encontrar versos com
formas métricas na era das imigrações, particularmente na poderosa
expressão da feitiçaria (bem forjada e arcaicamente obscura) e embora
também seja possível vislumbrar uma arte poética desenvolvida a partir
de caracteres mágicos (e míticos) por trás de um número de inscrições
nos períodos anteriores à Era Viking, nenhuma estela rúnica nos passa um
\emph{insight} tão profundo da literatura do mundo antigo como essa.

Com relação ao conteúdo, Jesch afirma que Teodorico é um
chefe famoso dos Ostrogodos (século~\versal{V/VI}). No entanto, saber o motivo pelo
qual ele está numa pedra sueca três séculos depois é uma tarefa difícil.
A autora afirma que ele se encaixa no padrão dos heróis do período da
migração, que são celebrados em versos édicos, como Átila, o huno,
Gunther e o burgúndio, que aparecem em vários poemas lendários na segunda
parte do \emph{Codex Regius}. A palavra \emph{Reid} do poema faz
referência a Reidgotlândia, país citado tanto nas sagas escandinavas
quanto no poema \emph{Widsith}, escrito em inglês antigo, que retrata os
povos, reis e heróis da Europa. Na enciclopédia sueca \emph{Nordisk
Familjebok}, afirma-se que \emph{Reidgoter} é uma antiga
denominação para ostrogodos e que \emph{Reiðgotaland}, na \emph{Edda de
Snorri}, é parte a Jutlândia e parte a área continental na Dinamarca e
Suécia, ao contrário das ilhas, chamadas de \emph{Eygotaland}.

A estela rúnica de Rök testemunha que o tipo de poesia encontrado no
\emph{Codex Regius} era conhecido na Era Viking. No entanto, não
significa que os poemas dessa antologia são daquele período.
Por conseguinte, em concordância com Williams,
é importante levar em consideração as poesias em estelas rúnicas justamente 
porque elas se atrelam completamente ao resto do
\emph{corpus} do antigo nórdico.

Com relação aos manuscritos medievais (que não podem ser atribuídos com
segurança a um período anterior ao ano 1200 d.C.), para que eles alcancem
a Era Viking, seria necessário assumir a existência de uma tradição mais antiga de
manuscritos (e, portanto, não rastreável) ou um período de
transmissão oral, ou provavelmente os dois. De certo modo, é possível
admitir que muitos dos materiais nos poemas édicos -- ou seja, as estórias
dos deuses e heróis, o vocabulário conceitual, as ideologias e crenças --
têm origem em tempos bem mais antigos. Contudo, acolher a teoria de
que os poemas, do modo que estão preservados, são bem antigos,
dependeria da existência de uma vasta transmissão oral das formas fixas
de poemas que são, na verdade, soltos em suas estruturas, o que não
corresponderia com o que sabemos sobre poesia oral de outras culturas.
Portanto,
de acordo com Meulengracht \& Sørensen, é mais possível que os
poemas édicos sejam retrabalhos em vários períodos a partir de materiais do
``reservatório'' do conhecimento cultural antigo. Assim, de acordo com a autora, os poemas édicos
sobreviventes representam uma prática cultural da Era Viking, não sendo
necessariamente textos da Era Viking na forma com a qual se encontram nos
manuscritos.

Jesch afirma que a poesia escáldica é mais facilmente traçada até suas
origens na Era Viking. O termo ``poesia escáldica''também é utilizado muito mais
amplamente para fazer referência aos mais variados tipos de poesia islandesa (e
escandinava) do que à poesia éddica. Um outro ponto relevante é a importância dada ao nome do
poeta autor, informação que, no entanto, não é sempre
confiável. Igualmente relevante é a importância atribuída ao contexto histórico e literário para a composição do poema,
diferentemente da poesia éddica, que é anônima e remete a eventos
antigos, atemporais e de comum bagagem cultural. Além disso, a autora
afirma que a preservação dessas informações auxiliares sobre os versos
escáldicos está relacionada à sua transmissão e, diferentemente da Edda
Poética (uma antologia) ou da Edda de Snorri (um manual com citações
ilustrativas), os manuscritos que preservam os poemas escáldicos os citam
em um contexto narrativo e de modo a indicar seu contexto
cronológico, geográfico e social, sempre remetido à Era Viking. Todavia, 
ainda há um problema na datação.

Ainda que os poemas escáldicos se encontrem em manuscritos do século~\versal{XIII}
ou de séculos posteriores, grande parte deles é provavelmente um produto
da Era Viking, composta e executada em um contexto oral. A resposta para a
pergunta sobre como essas expressões orais foram transmitidas e como
elas sobreviveram até o período literário é incerta,
mas, de acordo com a autora, o historiador e poeta medieval Snorri
Sturluson, a quem se atribui a autoria do \emph{Edda em Prosa}, 
pode nos dar uma guia para essa resposta.

No prólogo do \emph{Heimskringla}, Snorri comenta de onde ele toma seus
exemplos. No início do parágrafo ele escreve: \emph{Með Haraldi konungi
váru skáld, ok kunna menn enn kvæði þeirra ok allra konunga kvæði,
þeirra er síðan hafa verit at Nóregi} (``Com o rei Haraldr havia escaldos
e as pessoas ainda conhecem seus poemas e os poemas de todos os reis que
estiveram na Noruega até então``, tradução nossa). Logo em seguida:
\emph{Ok tókum vér þar mest dœmi af því, er sagt er í þeim kvæðum, er
kveðin váru fyrir sjálfum höfðingjunum eða sonum þeirra [...]} (``E
nós tomamos como evidência o que está dito nesses poemas, coisas que
foram recitadas antes dos próprios chefes e de seus filhos'', tradução
nossa). Por fim, na última linha do prólogo: \emph{En kvæðin þykkja mér
sízt or stað fœrð, ef þau eru rétt kveðin ok skynsamliga upp tekin}
(``Mas parece-me que é bem pouco provável que os poemas estejam
corrompidos, desde que eles sejam corretamente compostos e interpretados
cuidadosamente'', tradução nossa). Em tais trechos está a origem da ideia --
muito difundida entre os críticos literários -- de que a forma dos versos escáldicos é uma
garantia de sua preservação mais ou menos exata na tradição oral até o
surgimento da cultura literária, que permitia registrar essa oralidade
em papel. Ademais, os versos escáldicos são muito mais fixos do que outras métricas
germânicas e têm a intenção de serem memoráveis na forma exata em que
foram compostos originalmente. Assim, a
complexidade métrica dos versos escáldicos não permite erros, o que
descaracterizaria o poema. Esse raciocínio está de acordo com o que Snorri considerou em
sua obra: o poema não estará ``corrupto'' se for ``composto
corretamente'' e ``interpretado cuidadosamente''.

Na estela rúnica Karlevi (ilha de Öland, Suécia, código \versal{Ö}l 1, fim do
século~\versal{X}, portanto Era Viking) encontra-se uma inscrição de um poema
escáldico, composto na métrica \emph{dróttkvætt} (``métrica da corte''):
\emph{S[t]æ[inn] [sa]s[i] es sattr æftiR Sibba Goða, sun
Fuldars, en hans liðisatti at....} (``[Essa pedra foi] erigida em memória
de Sibbi, o bom, filho de Fuldarr, e seu séquito alocado em ....'',
tradução nossa).

De acordo com Williams, essa é a estrofe mais antiga
registrada na métrica \emph{dróttkvæt}, assim como a única estrofe
completa para a qual se tem a fonte original. Esse
poema apresenta as características da métrica \emph{dróttkvætt} porque ele tem
oito versos, contidos em dois quartetos, com três sílabas tônicas em
cada verso. Os versos ímpares têm rima interna incompleta
\emph{skothending}, que rima palavras com uma ou mais consoantes
idênticas e vogais diferentes. Os versos pares tem a rima completa
\emph{aðalhending}, que rima palavras que têm tanto vogais idênticas
quanto uma ou mais consoantes idênticas. Além disso, essa métrica
também tem uma aliteração impecável e um troqueu no final dos segundos
semiversos. As rimas
internas estão apresentadas em itálico/negrito e as aliterações em
negrito. Além do exemplo acima, há ainda aproximadamente setenta
inscrições rúnicas com formas métricas (grande parte fragmentos). Entretanto, 
muitas delas são formações pós Era Viking, como, por exemplo, as
inscrições de Bryggen, Bergen, cunhadas em madeira e osso no século~\versal{XIV}.
Dessas inscrições, entre cinco e seis são mais antigas, em antigo
\emph{futhark}, enquanto as restantes estão em novo \emph{futhark}.
Outros exemplos de poesia em estela rúnica são
dados por Jansson (1987, p. 131-143).

Na diferenciação entre a poesia éddica e escáldica, há, portanto, de
acordo com Ross, os fatores métricos (um
\emph{dróttkvætt} bem feito, que é sempre escáldico, é muito diferente
em várias maneiras de um \emph{fornyrðislag}, típico da poesia éddica);
o fator temático, que pode tratar de um louvor a um líder histórico ou
amigo, na poesia escáldica, mas de antigos contos heroicos, na poesia
éddica; a voz poética narrativa (que, na poesia escáldica, pode ser de um
poeta engajado que observa as ações do aqui e agora ou que elogia as
qualidades de um homem, conhecidas pela audiência, enquanto na poesa éddica um narrador
apagado que relata uma narrativa já provavelmente
conhecida e ouvida pela audiência).

\SIG{Yuri Fabri Venancio}

Ver também Heiti; Inscrições rúnicas; Kenning; Linguagem; Norreno;
Poesia éddica; Poesia escáldica.

\begin{itemize}
\item \versal{ENCICLOPEDIA} Merriam-Webster's Encyclopedia of Literature.
Springfield/Massachusetts: Merriam-Wester, Incorporated, Publishers,
1995.

\item \versal{JANSSON}, Sven B. F. \emph{Runes in Sweden}. Stockholm: Gidlunds, 1987.

\item \versal{JESCH}, Judith. Runes. In: \versal{BRINK}, Stefan; \versal{PRICE}, Neil (eds.). \emph{The
Viking World}. London/New York: Routledge, 2008, pp. 291-298.

\versal{ÓLASON}, Vésteinn. Old Icelandic Poetry. In: \versal{NEIJMANN}, Daisy. \emph{A
History of Icelandic Literature}. Lincoln/London: University of Nebraska
Press, 2006, pp. 01-63.

\item \versal{POOLE}, Russell. Metre and Metrics. In: \versal{McTURK}, Rory (ed.). \emph{A
Companion to Old Norse-Icelandic Literature}. Malden/Oxford/Victoria:
Blackwell Publishing Ltd, 2005, pp. 265-284.

\item \versal{ROSS}, Margaret Clunies. \emph{A History of Old Norse Poetry and
Poetics}. Cambridge: \versal{D. S.} Brewer, 2005.

\item \versal{WHALEY}, Diana. \emph{A Useful Past}: Historical Writing in Medieval
Iceland. In: Old Icelandic Literature and Society (Cambridge Studies in
Medieval Literature 42). Cambridge: Cambridge University Press, 2000,
pp. 161-202.

\item \versal{WILLIAMS}, Henrik. Runes. In:
\versal{BRINK}, Stefan; \versal{PRICE}, Neil (eds.). \emph{The Viking World}. London/New
York: Routledge, 2008, pp. 281-290.
\end{itemize}

\chapter{M \textarn{m} \textarc{m} \textart{m}}
\section{\versal{MAR BÁLTICO DA ERA VIKING}}

Adão de Bremen parece ser o primeiro a usar o termo \emph{mare Balticum}
para tratar da região circunscrita pelo Báltico e os povos que lá
viviam. Além do nome usado por Adão de Bremen, Olaus Magnus utiliza o
termo \emph{mare Gothicum}, para tratar da costa sul desse oceano em sua
\emph{Historia de gentibus septentrionalibus}. No século~\versal{XVI},
cartógrafos holandeses popularizam o termo \emph{Oostzee}, que tinha como
significado o próprio mar. Com o aumento de interesse político (e
diplomático) pela região, houve um crescente empenho em historicizar o
mar e os povos da região. Podemos falar da existência de diversos
Bálticos que estão em constante reinvenção e redefinição. No curso do
tempo, as balizas de análises históricas se alteram e a semelhança
física espacial parece ser a única amarra entre os diferentes contextos.
Em outras palavras, existe o Báltico de Adão de Bremen, como existe o
Báltico do \emph{Stortmaktstiden} (sueco do século~\versal{XVII}), o da União
Soviética do século~\versal{XX} e o Báltico da União Europeia. É claro que, no
pouco espaço que temos para abordar -- de maneira introdutória -- o
tema, seria impossível transcorrer todos esses ``Bálticos''. Por isso,
dado o recorte da obra, voltaremos nossa prefação ao período em que, em função do comércio, 
a região se torna uma zona de contato para diferentes povos(séculos~\versal{VII} 
ao \versal{XIII} d.C.), como os frísios, bálticos, fínicos,
eslavos, árabes, judeus, anglo-saxões, germânicos e outros.

Os três principais grupos que ocuparam a região (nórdicos, eslavos e
bálticos) agiam como mediadores no contato entre o mundo muçulmano, o
oceano báltico e mares do Norte. Alguns historiadores comparam,
inclusive, a esfera social e os aspectos da religiosidade desses povos: os
três, de forma geral, organizavam-se em tribos e os chefes locais
lutavam entre si em busca de influências. No campo da religiosidade, se
os escandinavos tinham Thor como deus do trovão, os eslavos e bálticos
tinham, respectivamente, Perun e Perkunas. No entanto, suas práticas
religiosas parecem divergir. Com isso, não pretendemos estabelecer uma
``ancestralidade comum'' ao culto a esses deuses, mas simplesmente
atestar semelhanças quanto a suas crenças.

Os eslavos migraram para a localidade por volta do século~\versal{VIII} d.C. e,
na região do Elba, os poloneses passaram a se organizar por volta da
década de 60 do século~\versal{X} d.C. A consolidação de poder na área pode ser
atestada pelos diversos achados arqueológicos de comunidades que se
instalaram na região, assim como fortificações erguidas, como Poznań, Ląd,
Gniezno, Giecz e outras. Já os povos bálticos, originalmente, se
instalaram em uma região maior, que fora reduzida a partir do século~\versal{I}
d.C., graças às invasões germânicas e eslavas nas regiões que compreendem
a Pomerânia oriental, o noroeste da Rússia, parte do alto rio Volga,
assim como o norte da Ucrânia. No norte, os povos balto-fínicos, que
pertencem ao tronco linguístico fino-urálico, se instalam no Nordeste da
Rússia, no Golfo da Finlândia (finns e carélios), no Norte da
Escandinávia (sámi) e no sul do Báltico (estonianos e letos). Por fim,
os dinamarqueses e suíones, que, a partir do Báltico, atingiram outras
áreas tais como as ilhas britânicas, o império Franco, Constantinopla, o
mundo árabe, além das áreas próximas ao rio Volga e ao mar Cáspio. Esses
contatos se deram, basicamente, de duas formas: por meio de trocas
comerciais e pela pilhagem. No império Bizantino, por exemplo, esses
homens do norte serviram à guarda palaciana imperial ou como mercadores
pelas rotas imperiais.

A região Báltica, portanto, pode ser considerada um centro de trocas
multiétnicas onde várias zonas de contato (troca) foram estabelecidas.
Como exemplo, citamos Hedeby (Dinamarca), Birka (Suécia), Ralswiek
(Alemanha) e Wolin (Polônia). Produtos como ornamentos de ouro, grãos,
cavalos, ceras, peles, mel, armas, tecidos e artigos de artesãos da
região, além do comércio de escravos que ali circulavam, possuíam grande
valor intrínseco, o que tornava o transporte e o comércio para médias e
longas distâncias -- assim como a pilhagem -- atividades altamente
lucrativas. Até o século~\versal{XII} d.C., os dinamarqueses figuraram como um dos
principais agentes da região. Do ponto de vista estratégico-espacial, o
Báltico é uma região que traz grandes benefícios àqueles que têm a
capacidade de controlar as rotas marítimas ou negá-las aos inimigos. A
fundação da Liga Hanseática transfere o controle nórdico para a região
continental da Europa. Essa aliança militar e cartel comercial de
cidades têm o eixo Novgorod-Talín-Lübeck-Hamburgo-Bruges-Londres como um
dos mais lucrativos. Essas cidades, anteriormente não tão fortes, cooperaram para
fortalecer seus interesses (de comércio e proteção de caravanas) e
acabaram por criar um ator político de grande peso na região, que perduraria
até o século~\versal{XVI} d.C.

\SIG{Vítor Bianconi Menini}

Ver também Escandinávia; Finlândia da Era Viking; Sámi, fínicos e
nórdicos; Suécia da Era Viking.

\begin{itemize}
\item \versal{BRAND}, Hanno; \versal{MULLER}, Leos.~\emph{The Dynamics of economic culture in
the North Sea and Baltic Region (ca. 1250-1700): Proceedings
of the Esbjerg and Stockholm workshops} (2003-2004). Hilversum:
Verloren, 2007, vol. 1.

\item \versal{ENDRE}, Bojtár.~\emph{Foreword to the Past: A Cultural History
of the Baltic People}. New York: Central European University Press, 1999.

\item \versal{KIRBY}, David.~\emph{Northern Europe in the Early Modern
Period: The Baltic World}, 1492-1772. New York: Routledge,
2013.

\item \versal{NORTH}, Michael.~\emph{The Baltic: A History}. Cambridge:
Harvard University Press, 2015.
\end{itemize}
\section{\versal{MEDICINA E BOTÂNICA MÁGICA}}

Durante a Era Viking, a maioria da população confiava em alguma mulher
da família que conhecia plantas e ervas, bem como o modo de empregá-las de maneira
correta para cada sintoma ou doença. Tal papel poderia ser desempenhado também por
uma pessoa mais velha da comunidade que detinha esse conhecimento. Não havia médicos,
apenas pessoas que dominavam as artes da cura. As artes médicas e
curativas foram largamente empregadas e praticadas por pessoas que se
dedicavam aos cuidados para com os enfermos e feridos. Os remédios eram
elaborados com ervas medicinais nativas da área nórdica. Os tratamentos
médicos eram rudimentares e consistiam basicamente em cuidados simples,
como a limpeza das feridas, realizadas até que estas cicatrizassem. O uso
de bandagens e de cataplasmas também era comum em praticamente todos os
ferimentos. A preparação dos remédios era feita exclusivamente à base de
plantas e de gordura animal para a preparação de unguentos e pomadas.
Outro ramo dessa medicina rudimentar era a obstetrícia, pois, em muitos
casos, o parto era extremamente problemático e difícil, podendo acarretar
a morte da mãe, da criança ou de ambas. As parteiras possuíam um
grande conhecimento sobre doenças e tratamentos, ou seja, eram também
curandeiras experientes e auxiliavam a todos que necessitavam. Alguns
estudos de restos de esqueletos da Era Viking apontam evidências de que
algumas fraturas se curavam muito bem, fossem elas em costelas, em
ossos dos braços ou das pernas. Existem ainda estudos osteológicos que
mostram evidências de que membros quebrados foram manipulados para
permitir que os ossos se regenerassem mais rapidamente.

Em algumas áreas mais densamente povoadas -- as cidades que serviam
como entrepostos comerciais, por exemplo --, as epidemias ocorriam com certa frequência
devido à grande circulação de pessoas vindas das mais diversas regiões.
Doenças comuns à época, como a varíola, a disenteria e a lepra eram
fatais, pois os tratamentos eram somente paliativos, não existindo uma
cura para elas. A água não potável era uma das grandes responsáveis pela
proliferação de doenças e, por essa razão, o consumo da cerveja era muito
grande em praticamente todas as camadas sociais. Portanto, os nórdicos
enfrentavam essas doenças com resignação, já que pouco poderia ser feito
para controlá-las e uma das únicas armas contra elas seria justamente o
uso da magia.

Alguns esqueletos mostram que pelo menos algumas pessoas viveram até uma
idade mais avançada na Era Viking, embora apresentassem doenças degenerativas
e desgaste nas articulações, indicativos comuns da velhice. Na saga de Egil, por
exemplo, mencionam-se as condições de saúde inerentes à velhice, como
cegueira e surdez. O capítulo 85 da saga narra que Egill, quando ficou
velho, tornou-se frágil, com pernas rígidas e com visão e audição falhas. 
Porém, seu talento para compor poesia parece
não ter sido afetado, haja vista que compôs poemas zombando de suas
enfermidades na velhice. Acredita-se que, nessa época, ele teria mais de
oitenta anos.

As feridas de batalhas eram fatais e as práticas do que chamamos hoje de
``primeiros socorros'' não existiam, a não ser pela presença de algumas
mulheres curandeiras que cuidavam dos feridos, oferecendo conforto e
curativos da época, como pode ser comprovado pelo capítulo 234
da \emph{Óláfs saga}. Þormóðr foi ferido por uma flecha. Depois que a
batalha terminou, ele saiu do campo e entrou na cabana onde as mulheres
estavam cuidando dos feridos. Uma das mulheres examinou a ferida e viu a
ponta da flecha de ferro, mas não conseguiu determinar o seu caminho, tampouco
saber quais os órgãos internos ela havia atingido. Ela deu a Þormóðr
um caldo quente, contendo alho-poró, cebolas e outras ervas. Se, depois
de comer, o paciente pudesse cheirar o caldo da ferida, ela saberia identificar que partes
vitais haviam sido afetadas e se a ferida era fatal. Þormóðr recusou o
caldo. Em vez disso, ele pediu à mulher para cortar a ferida e assim
expor a ponta da flecha. Ele segurou a flecha com pinças e puxou-o para
fora. Vendo as fibras gordurosas que aderiram à cabeça da seta, Þormóðr
disse: ``Veja o quão bem o rei mantém seus homens. Quanta gordura há no
meu coração'', e morreu.

A magia também foi largamente empregada para a cura dos ferimentos em
batalha, como podemos constatar na descrição do banquete sacrificial
narrado no capítulo 22 da \emph{Kormáks saga}: Þórd's aconselhou
Þorvarðr que, para acelerar sua recuperação, fosse até uma colina próxima
e lá derramasse o sangue de um touro morto por Kormákr, de modo a oferecer
um banquete da carne do touro para os elfos que habitavam a colina. A
magia também foi usada para que uma ferida se curasse. No capítulo 57 da
saga de \emph{Laxdæla saga}, Eiður diz que uma ferida infligida por sua
espada Sköfnung não se curaria, a menos que se esfregasse nela uma pedra
que antes havia sido esfregada em sua espada.

As plantas eram largamente empregadas no cotidiano. Muitas possuíam um
uso mágico ou eram utilizadas para curar feridas e doenças.
Algumas espécies foram domesticadas, sendo cultivadas em hortas
próximas das casas para serem utilizadas \emph{in natura} durante a
primavera e o verão. Durante o inverno, eram utilizadas secas (eram 
colhidas, postas para secar, armazenadas, penduradas nas
paredes e pendentes dos telhados ou ainda em pequenos potes de cerâmica).
Para um detalhamento das ervas utilizadas pelos nórdicos, consultar o
verbete \versal{Plantas mágicas} do {Dicionário de Mitologia Nórdica}.

\SIG{Luciana de Campos}

Ver também Agricultura; Alimentação; Cotidiano; Cultura material.

\begin{itemize}
\item \versal{CAMPOS}, Luciana de. Cosmética, plantas e saúde na Era Viking.
\emph{Youtube/Canal do \versal{NEVE}}, 2017. Disponível em:
\href{https://www.youtube.com/watch?v=4TDvmqRKjWc}
{\emph{https://www.youtube.com/watch?v=4TDvmqRKjWc}}. Acesso em: 4 dez. 2017.

\item \versal{CAMPOS}, Luciana de. Plantas mágicas. In: \versal{LANGER}, Johnni (org.).
\emph{Dicionário de Mitologia Nórdica}. São Paulo: Hedra, 2015, pp.
373-377.

\item \versal{NATIONALMUSEET}. \emph{Herbs, spices and vegetables in the viking
period}. Kobenhavn, 2017. Disponível em:
\href{http://en.natmus.dk/historical-knowledge/denmark/prehistoric-period-until-1050-ad/the-viking-age/food/herbs-spices-and-vegetables/}{\emph{http://en.natmus.dk/historical-knowledge/denmark/prehistoric-period-until-1050-ad/the-viking-age/food/herbs-spices-and-vegetables/}}. Acesso em: 4. dez. 2017.

\item \versal{NATURAL HISTORY MUSEUM}. \emph{The Viking Garden}. Disponível em:
\href{https://www.nhm.uio.no/english/visiting/botanical-garden/the-viking-garden/}{\emph{https://www.nhm.uio.no/english/visiting/botanical-garden/the-viking-garden/}}. Acesso em: 4 dez. 2017.
\end{itemize}
\section{\versal{METALURGIA}}

É a palavra usada para definir o conjunto de procedimentos e técnicas
para tornar o metal adequado para a forja de algum tipo de objeto. Os
processos metalúrgicos incluem a extração do metal da natureza (quer
ele esteja sozinho ou combinado com outros minerais), a fundição para a
sua purificação, bem como os tratamentos térmicos e químicos necessários para
que, posteriormente, ele seja trabalhado por um ferreiro (\emph{craddock}). Muito
antes de descobrir as técnicas metalúrgicas para trabalhar com os
minerais ferrosos terrestres, o homem aprendeu a forjar primeiro o ferro
dos meteoritos com martelos de sílex (Eliade). Juntamente com os raios que caíam
de encontro ao chão durante as tempestades, os meteoritos que cruzavam o céu
eram vistos como sinal da fúria de divindades que habitavam planos
superiores e castigavam os homens -- nesses casos, faziam-se
associações diretas ao deus Thor e, dessa forma, os primeiros ferreiros
perceberam as propriedades especiais dessas rochas, que traziam consigo a
sacralidade celeste.

Após a descoberta dos minerais ferrosos, o homem percebeu que o metal
sagrado dos meteoros também podia ser extraído do ventre da Terra, fato que acabou por atribuir
ao trabalho do ferreiro/mineiro uma dimensão obstétrica (ajudava o planeta a parir a
matéria-prima das ferramentas e armas). Aproveitando-se de tal condição, 
os ferreiros não repassavam seus conhecimentos para qualquer
pessoa, sendo preciso mais do que simples vontade para se tornar um ferreiro. Os
poucos que conseguiam fazer parte desse seleto círculo eram submetidos a
vários ritos de iniciação. O trabalho -- transmutação da matéria -- era considerado sagrado.

A Idade do Ferro germânica se estende entre os anos 400 e 800 da Era
Comum. Nesse período foram desenvolvidas muitas das técnicas de
trabalho com ferro e outros metais que seriam empregadas nos anos
posteriores, durante o que se convencionou chamar de Era Viking.

Desde o final da década de 1990, escavações arqueológicas
(conduzidas pela Universidade de Oslo) nas ruínas de uma igreja 
localizada em Hurdalfeitas, na Noruega, 
têm obtido resultados importantes. Foram encontradas
amostras materiais da Idade do Ferro, com destaque para 
as forjas utilizadas na obtenção de ferro cru. Junto a essas pequenas forjas foram
encontradas amostras de ferro e escória -- parte do minério de ferro
que não é utilizada para fabricação de ferramentas -- em quantidade
suficiente para evidenciar que naquele local se fazia redução do
minério de ferro. O diâmetro dos fornos achados não ultrapassa os 50 cm.
Tais fornos tinham formato cilíndrico e dificilmente ultrapassavam 1 m de altura. Era
estruturado por paredes de 5 cm de diâmetro, compostas por areia e
argila. Por fora, possivelmente havia um reforço de madeira e pedras. No
interior desse pequeno forno intercalavam-se camadas de carvão e de
minério de ferro. Havia, na base, um orifício para injeção de ar por meio
de foles, para que o fogo atingisse temperaturas mais altas. Após
atingida a temperatura almejada (que, de acordo com os estudos, variava entre
1200º\versal{C} e 1500º\versal{C}), fazia-se um orifício na lateral do forno para recolher
o ferro que escorria em um cadinho de bronze. Eventualmente, deixava-se que
o ferro cair no solo para, posteriormente, recolher as pelotas com uma tenaz. Estas eram
reunidas, trabalhadas com martelo e
bigorna e, por fim, transformadas em barras.

Bergstol também aponta conexões diretas entre a metalurgia, a magia e os
rituais fúnebres na Era do Ferro escandinava. Em muitas inumações foram
encontradas grandes quantidades de escória, o que, segundo o estudioso, 
apontaria para uma concepção do fogo como um poder transformador e libertador:
liberta o ferro da escória, a carne dos ossos e, talvez, a alma
do corpo. Deve-se sempre levar em conta que, para os povos escandinavos
pré-cristãos, a metalurgia não era um conhecimento estritamente técnico.
Havia muitos aspectos místicos que deveriam ser seguidos e respeitados
para que um homem pudesse se tornar um ferreiro.

\SIG{Michel Roger Boaes Ferreira}

Ver também Armamento; Cultura material; Espada; Ferreiros e ferraria;
Sociedade.

\begin{itemize}
\item \versal{BERGSTOL}, Jostein. Iron Technology and Magic in Iron Age Norway. In:
\emph{Metals and Society}. Lisboa: \versal{BAR} International Series 1061, 2000,
pp. 77-82.

\item \versal{COLINS}, John. \emph{The European Iron Age}. London: Routledge, 1984.

\item \versal{CRADDOCK}, Paul T. ``Mining and Metallurgy''. In:~\versal{OLESON}, John
Peter~(ed.).~\emph{The Oxford Handbook of Engineering and Technology in
the Classical World}. Oxford: Oxford University Press, 2008, pp. 93-120.

\item \versal{ELIADE}, Mircea. \emph{Ferreiros e Alquimistas}. Rio de Janeiro: Zahar
Editores, 1979.~

\item \versal{HEDEAGER}, Lotte. \emph{Iron Age Myth and Materiality: An Archaeology of
Scandinavia ad 400-1000}. New York: Routledge, 2011.
\end{itemize}
\section{\versal{MIKLIGARDR (BIZÂNCIO)}}

A cidade de Constantinopla, capital do Império Bizantino, era conhecida
como Mikligardr entre os escandinavos. A nomenclatura provém das
palavras \emph{mikill} (grande), e \emph{garđr} (fortaleza, moradia),
aludindo ao fato da cidade ser fortificada. Outros nomes nórdicos para
Constantinopla são Mikligarthr ou Miklagarđr. Os bizantinos eram
conhecidos pelos escandinavos como gregos, enquanto estes chamavam os
vikings que vinham da rota do rio Dniepre em direção a Constantinopla de
``varegues'' (\emph{varangoi}), mesma denominação utilizada alguns anos
mais tarde por Rus de Kiev. Há outras denominações, como \emph{Rhos} (que
se referia aos nórdicos que se estabeleceram em Rus) e os
\emph{Koulpingoi} (nórdicos da região do Báltico, mas que também podem
ser finos, termo mais tardio), que interagiram diretamente com a cidade
imperial.

É possível que interações comerciais entre escandinavos e gregos tenham ocorrido
desde o início do século~\versal{IX}, ainda que, conforme Fedir Androshchuk, mercadorias
bizantinas datadas do século~\versal{VI} tenham sido encontradas na região da
Suécia. Talvez o principal elo entre a Escandinávia da Era Viking e
o Bizâncio, a rota do rio Dniepre foi provavelmente utilizada a partir do
início do século~\versal{X} e intensificada conforme os nórdicos se estabeleciam
em Rus. A tradição historiográfica atribuiu ao caminho percorrido por
esses nórdicos o nome de Rota dos Varegues aos Gregos, localizada 
entre o mar Negro e o mar Báltico. O uso dessa
terminologia implica uma relação unilateral, mas ambos os lados eram
beneficiados pelas trocas. Ademais, a Rota dos Varegues aos Gregos 
não era a principal rota comercial utilizada pelos vikings no Oriente. 
Jonathan Shepard afirma, ainda, que a
rota não era utilizada exclusivamente para a troca entre os varegues e
os gregos, pois também envolvia os bretões.

Muitos dos escandinavos que foram para Constantinopla permaneceram na
cidade como uma força militar mercenária a serviço do Imperador
bizantino na chamada Guarda Varegue. De acordo com Sigfús Blöndal, à
princípio, faziam parte da guarda somente os nórdicos Rus, mas eventualmente mais
escandinavos, como noruegueses e suecos, aderiram à guarda. A partir de
Basílio~\versal{II} (960-1025), mais escandinavos passaram a emigrar para
Bizâncio com o intuito de participarem do exército, entre eles Haroldo
Hardrada (1046-1066), futuro rei norueguês. É possível que os
escandinavos tenham ajudado os bizantinos em batalhas anteriores ou em
outros tipos de prestação de serviços ainda no século~\versal{IX}, dada a presença dos
Rus da Suécia na corte de
Carolíngia, junto ao enviados de Constantinopla, 
conforme descrita nos \emph{Annales Bertiniani.}

Os bizantinos mantiveram uma relação especial com os escandinavos de
Rus. Apesar do evento narrado nos \emph{Annales Bertiniani}, vários
conflitos foram documentados entre os Rus e os bizantinos, como o ataque
a Constantinopla em 860, conforme descrito pela \emph{Crônica dos Anos
Passados}. Esta também conta diversos conflitos a partir do
estabelecimento da dinastia Riuríkida em Kiev, sobretudo em 907, 941, e
971. Tratados de paz mostram que alianças militares eram bastante
frequentes. Constantinopla foi diretamente responsável pela
cristianização de Rus de Kiev, seja pelo batismo de Olga (945-964) e
de Vladimir \versal{I} Sviatoslavich (980-1015) (patrocinados pelo Império),
seja pela existência de missões evangelizadoras desde o século~\versal{IX}, quando o
Patriarca Fócio de Constantinopla falou sobre o sucesso da missão
bizantina de 867 em suas homilias. A arqueologia mostra também que
varegues que retornaram de Constantinopla para Rus foram convertidos ao
cristianismo de rito grego. Diversos pingentes em forma de crucifixo foram
encontrados ao longo de Rus em túmulos pertencentes a uma elite
guerreira escandinava, causando o efeito que Oleksiy Tolochko chama de
Cristandade Varegue (\emph{Varangian Christianity}), no qual a elite
cristã influencia a eventual conversão de Rus.

\SIG{Leandro César Santana Neves}

Ver também: Crônica dos Anos Passados; Kiev; Olga de Kiev; Rus; Rússia
da Era Viking; Varegues, Vladimir~\versal{I} de Kiev.

\begin{itemize}
\item \versal{ANDROSHCHUK}, Fedir. \emph{What does material evidence tell us about contacts
between Byzantium and the Viking world c. 800-1000?} In: \versal{ANDROSHCHUK},
Fedir; \versal{SHEPARD}, Jonathan; \versal{WHITE}, Monica (eds.). \emph{The Byzantium and
Viking World}. Uppsala: Uppsala Universitet, 2016, pp. 91-116.

\item \versal{AUDY}, Florent. \emph{How were Byzantine coins used in Viking-Age Scandinavia?}
In: \versal{ANDROSHCHUK}, Fedir; \versal{SHEPARD}, Jonathan; \versal{WHITE}, Monica (eds.).
\emph{The Byzantium and Viking World}. Uppsala: Uppsala Universitet,
2016, pp. 141-168.

\item \versal{BLÖNDAL}, Sigfús. \emph{The Varangians of Byzantium}. Trad.
Benedikt S. Benediktz. Cambridge: Cambridge University Press, 1978.

\item \versal{SHEPARD}, Jonathan. ``From the Bosporus to the British Isles: the way from
the Greeks to the Varangians''. In: \versal{DZHAKSON}, Tatiana N. \emph{Drevneishie
Gosudarstva Vostochnoi Evropy [Os Estados Mais Antigos do Leste
da Europa]}.~Moscou: Indrik, 2010, pp. 15-42.

\item \versal{SHEPARD}, Jonathan. \emph{The Viking Rus and Byzantium}. In: \versal{BRINK}, Stefan;
\versal{PRICE}, Neil (eds.). \emph{The Viking World.} London: Routledge, 2008,
pp. 476-516.

\item \versal{TOLOCHKO}, Oleksiy. ``Varangian Christianity in Tenth-century Rus''. In:
\versal{GARIPZANOV}, Ildar; \versal{TOLOCHKO}, Oleksiy (orgs.). \emph{Early Christianity
on the Way from the Varangians to the Greeks (Ruthenica
Supplementum 4)}. Kiev: Ruthenica, 2011, pp. 58-69.
\end{itemize}
\section{\versal{MOBILIÁRIO}}

O mobiliário das casas vikings era simples e pouco diversificado. Sinais
de requinte eram visíveis apenas nas casas dos muitos ricos. Como grande
parte da população era rural e os lares nas cidades eram pequenos, a
existência de móveis era bem limitada. As casas e salões eram, em geral,
retangulares, o que acabava por influenciar na disposição da
mobília.

Ao invés de mesas com muitas cadeiras, optava-se por longos e largos
bancos (\emph{lokrekkja}). Nas casas humildes, os bancos poderiam ser
colados na parede, preenchidos com terra e revestidos com vime. Em
outras casas, os bancos eram totalmente de madeira. Escavações
arqueológicas encontraram em lares dos séculos~\versal{X} e \versal{XI} 
(em Dublin, York, Oseberg e Novgorod), bancos com encosto, apoio de braço e
entalhamentos de figuras geométricas ou de animais. Em ocasiões
solenes, os bancos eram forrados com palha ou tecido.

Por serem longos, os bancos acabavam servindo de cama. Na
maioria das casas não havia camas e as pessoas dormiam nos bancos ou em
leitos forrados sobre o chão. Porém, em casas ricas, haviam
camas ricamente adornadas, como ilustra, por exemplo, o exemplar de três camas
achado no navio túmulo de Oseberg, na Noruega. Tais camas poderiam ser,
inclusive, importadas e serem cobertas com tecidos finos.

O chefe do lar possuía um assento mais alto, uma cadeira ou cadeirão
(\emph{öndvegi}) que destacava sua presença. No caso dos salões, o
\emph{jarl} ou rei possuía um cadeirão ou trono de madeira
para si e para sua esposa. Na maioria das casas tradicionais escandinavas
a existência de cadeiras era algo raro, haja vista que tais
povos preferiam o uso de bancos, que se adequavam melhor à disposição da casa. Mas em
lares maiores ou mais abastados encontraram-se vestígios de cadeiras,
algumas delas apresentando, inclusive, trabalho em marcenaria.

A respeito das mesas, dependendo do espaço da residência, havia
pequenas mesas de alvenaria ou banquetas em torno do fogo central,
usadas para guardar os utensílios de cozinha e realizar o preparo de
alimentos. Além de tais superfícies de apoio, encontravam-se também
mesas de madeira de diferentes tamanhos, dependendo da extensão da casa.
Nos salões, os quais eram espaços de convivência pública da comunidade
perante seu chefe, era comum haver várias mesas (ou longas mesas) para
reuniões e banquetes.

A existência de outros móveis também era diminuta, como no caso de baús
e caixas para guardar roupas, objetos, armas, joias etc. Em alguns
casos, registra-se a presença de armários ou estantes de canto
(\emph{klefi}), usados para guardar utensílios domésticos, outros
objetos e até víveres. No caso das casas que não dispunham de armários
ou estantes, os objetos, ferramentas, armas, animais de caça, ervas
etc. eram pendurados nas colunas, paredes, vigas etc., através de
ganchos e cordas. Achavam-se também aparelhos de tear, pequenos bancos
de madeira, os quais poderiam ser carregados facilmente. Também encontravam-se banquetas
móveis, usadas para se fazer algum serviço manual.

\SIG{Leandro Vilar Oliveira}

Ver também Cotidiano; Habitação; Patrimônio; Sociedade.

\begin{itemize}
\item \versal{BOYER}, Régis. \emph{La vida cotidiana de los vikingos: 800-1050}.
Barcelona: José J. de Olañeta, Editor, 2000.

\item \versal{CLARK}, Helen. A vida quotidiana. In: \versal{GRAHAM-CAMPBELL}, James (org.).
\emph{Os vikings}. Barcelona: Folio \versal{S.A.}, 2006, pp. 70-72.

\item \versal{GUNNARSDOTTIR}, Sunnifa. \emph{Viking age furniture: the archaeological
evidence}, 2013.
\end{itemize}
\section{\versal{MOEDAS E CUNHAGEM}}

Para discutir dinheiro e seu uso, o conceito de valor deve ser
apresentado. Tem como raiz o termo protogermânico \emph{Werða}, que
significa ``correspondente a'' e surge a partir da comparação com
outro objeto. O sistema monetário-\emph{commodity} da Escandinávia
medieval tem, em relação aos dias atuais, pelo menos duas grandes diferenças.
A primeira reside no fato de hoje existir uma única unidade financeira, enquanto que, 
na Escandinávia -- assim como outras localidades no período -- existiam várias 
medidas de valor (que nem
sempre eram expressas moedas metálicas cortadas em formatos arredondados).
A segunda diferença recai sobre a forma: se hoje utilizamos o que
chamamos de moeda fiduciária (aquela que não possui lastro em metais
preciosos e nem valor intrínseco), na Escandinávia medieval prevaleceu o
uso do dinheiro com valor intrínseco que, muitas vezes, era a prata. O
valor que se atribuía à prata -- como em outros tempos se atribuiu ao sal
-- associava-se com sua função de préstimo utilitário (a prata serve para o ferreiro fazer
consertos, por exemplo), bem como a algumas de suas características, como sua portabilidade e
fácil manuseio. Tais elementos influenciaram na adoção desse material como moeda.

Se olharmos para o sistema de trocas de mercadorias (chamado como sistema de
\emph{commodities} pela literatura) tendo a atual economia monetizada como
referência, somos inclinados a caracterizar as dinâmicas da primeira forma
como primitivas. No entanto, o uso de dinheiro e plataformas monetárias
depende das relações que as sociedades da época possuem com a troca. Se
levarmos em consideração o tripé econômico produção-troca-consumo,
que tomou inúmeras formas no decorrer da história, concluiremos que a
economia da Escandinávia medieval, assim como de outros lugares, deve
ser compreendida por suas variações e complexidades. Um exemplo seria o
próprio uso de moedas como pagamento: a moeda, cunhada \emph{in loco} ou
vinda de outros lugares, é um dos vários meios de se efetuar um
pagamento.

As primeiras cunhagens foram feitas na Escandinávia (em Ribe, atual Dinamarca) 
e datam da metade do século~\versal{VIII} d.C. Já no início do século~\versal{IX} d.C. as moedas
teriam chegado à Hedeby e, por volta do ano mil, à Suécia e Noruega. Há
uma série de enterramentos em Kaupang e Birka -- datados do século~\versal{IX}
d.C. -- que atestam a existência de pagamento via prata na região.
Existem também enterramentos datados do século~\versal{X} d.C. -- em boa parte da
Escandinávia -- com prata cortada. Contudo, esses achados, assim como parte da
literatura sobre o tema, levantam dois problemas: o primeiro seria uma
espécie de anacronismo que se assenta na atribuição de uma importância excessiva
à prata e à moeda como meio de pagamento, perdendo de vista as diferenças entre 
a sociedade atual e a Escandinávia medieval; o segundo reside no grande papel
conferido à troca de presentes que ocorriam nas sociedades escandinavas. Existem
diversos estudos que tratam da troca de presentes na Era Viking como uma
forma de forjar e manter relações sociais. Todavia, de um ponto de
vista mais pragmático, existia a necessidade da efetuação de trocas cujo
intuito é adquirir objetos -- sobretudo quando se considera o contexto rural 
dos povos escandinavos.

A Escandinávia teve contato, no primeiro milênio, com moedas romanas de ouro e
prata. Há uma série de enterramentos do período entre os séculos~\versal{III} 
d.C. e \versal{VI} d.C. que atestam a presença de tais moedas na região. Podemos
identificar quatro momentos que demarcam diferentes grupos de moedas romanas que fluíram para
a Escandinávia, passando antes pelo noroeste e norte da região do
Báltico: o primeiro momento corresponde aos \emph{denarii}, que saíram do Império entre 160-194 d.C.
para Götland, Bornholm, Jutlândia, Fiônia e Zelândia; o segundo
momento corresponde ao ouro (na forma de \emph{aurei}, \emph{solidi} e
medalhões) fluindo para Fiônia, Zelândia e Jutlândia, desde o final do 
século~\versal{III} d.C. até o século~\versal{IV} d.C.; o terceiro corresponde à prata, 
em formato de \emph{siliquae} ou em lingotes, que flui para Fiônia, Zelândia e Jutlândia 
desde o final do século~\versal{IV} d.C. até o século~\versal{V} d.C.; por fim, 
o quarto corresponde aos \emph{solidi} -- e talvez lingotes de ouro --,
que, durante os séculos~\versal{V} d.C. e \versal{VI} d.C., chegaram à Öland e depois
à Bornholm e à Götland.

O papel das moedas romanas exportadas além do Reno e Danúbio -- como
tributo ou forma de pagamento de bens e serviços -- mudam drasticamente
ao chegar às comunidades germânicas e bálticas, pois os escandinavos
compõem, naquele momento, uma sociedade sem mercado econômico como o
romano. Ou seja, a relação que os escandinavos tinham com a moeda romana
era menos econômica e mais simbólica. Se houve uso
econômico das moedas, foi muito restrito. As relações entre os povos do
norte com cunhagens (locais ou estrangeiras) mudam ao longo do final do
período antigo (que alguns chamam de Antiguidade Tardia) e medieval.

A presença dos comerciantes francos em Ribe, Birka, Kaupang e Hedeby 
(século~\versal{IX} d.C.) pode ter ajudado na aceitação e implementação do uso de
prata como forma de pagamento, visto que outras localidades visitadas
pelos frísios também pagavam e aceitavam prata como dinheiro. Além da
presença franca, a vida citadina favoreceu o desenvolvimento do uso da
prata como dinheiro, uma vez que as dinâmicas urbanas dependem da
aquisição de comida e não de sua produção. No entanto, é difícil
definir a proporção exata do uso de prata como dinheiro nas relações 
econômicas de tais regiões. Sabemos que, no
século~\versal{X} d.C., há um aumento no uso desse meio de pagamento no espaço
urbano. Na Noruega, por exemplo, é apenas entre o final do século~\versal{XII}
d.C. e o século~\versal{XIV} d.C. que temos registros mais complexos sobre o uso de
prata como moeda.

No caso norueguês, no século~\versal{X} d.C., não há cunhagem de moedas próprias.
Não há nenhuma forma de controle central sobre o uso de dinheiro. Érico~\versal{I} 
(ou Machado Sangrento) emitiu moedas na Nortúmbria e Haroldo~\versal{I}
(ou Dente Azul) tentou criar um sistema ``nacional'' de cunhagens que funcionou de
forma parcial. No entanto, a literatura costuma reconhecer
Olavo Tryggvason (r. 995-1000 d.C.) como o primeiro a estabelecer uma
cunhagem norueguesa da qual se tem evidências de distribuição. Achados em
Gotland, Escânia, Pomerania e Uplândia corroboram com a tese de que a
prata cunhada na Noruega atingia regiões mais ao sul. Outros regentes
como Olavo~\versal{II}, da Noruega (r. 1015-1028 d.C.), e Haroldo~\versal{III} (r. 1047-1066
d.C.) também estabeleceram cunhagens que atingiram regiões forasteiras.
Sendo assim, a partir da análise da Noruega, pode-se apontar a Era Viking 
como o meio termo entre o desenvolvimento de sistemas de cunhagem, visto que, durante os
séculos~\versal{VIII} e \versal{XI} d.C., se percebe um aumento significativo -- em
quantidade e intensidade -- do uso de moedas que no início (em Vestfold,
por exemplo) eram ``importadas'' e que durante os anos foi se tornando
tarefa dos regentes locais.

Outro caso que chama a atenção é o da Islândia. Na primeira metade do
século~\versal{XI} d.C., a maioria dos achados são de moedas de outros lugares
que não a Escandinávia (árabes e francos principalmente). Já na segunda
metade do século, existem achados de moedas do reinado de Haraldo~\versal{III}, da
Noruega. No entanto, no século~\versal{XII} d.C., a prata perde seu lugar como a
principal unidade de valor para o \emph{vaðmál} -- espécie de tecido de
lã não tingido. Essa substituição provavelmente se deu em função da
escassez de prata na Islândia a partir do século \versal{XI} d.C.. O uso do
\emph{vaðmál} -- item muito conhecido e aceito na ilha -- não excluiu a
importância da prata, que passa a ser o lastro da nova unidade de valor. Se
compararmos o caso islandês com o de outras regiões, vemos que o uso de
moedas é até razoável (menor que Noruega, por exemplo, mas ainda
significativo). Não há cunhagem local, mas moedas estrangeiras são
utilizadas. Os achados na ilha pertencem aos mais diversos contextos,
como mercados, fazendas, covas e, principalmente, \emph{Things}. Por
ser um lugar com presença de testemunhas, as \emph{Things} parecem ser o
cenário ideal para uso de moedas como forma de executar diversas
transações (pagamento de débitos e fianças, por exemplo) durante uma
assembleia.

\SIG{Vítor Bianconi Menini}

Ver também Comércio; Dirhem; Thing.

\begin{itemize}
\item \versal{BURSCHE}, Aleksander. Circulation of Roman Coinage in Northern Europe in
Late Antiquity. In: \emph{Histoire \& mesure}. Disponível em:
\href{http://histoiremesure.revues.org/886}{\emph{http://histoiremesure.revues.org/886}}.
Acesso em 16/04/2017.

\item \versal{BURSCHE}, Aleksander. \emph{Later Roman barbarian contacts in Central
Europe: numismatic evidence}. Berlim: Gebr. Mann, 1996.

\item \versal{GRAHAM-CAMPBELL}, James; \versal{SINDBÆK}, Søren; \versal{WILLIAMS}, Gareth (orgs.).
\emph{Silver Economies, Monetisation and Society in Scandinavia \versal{AD}
800-1100.} Aarhus: Aarhus University Press,~2011.

\item \versal{HEDEAGER}, Lotte. \emph{Iron Age Myth and Materiality: An Archaeology of
Scandinavia \versal{AD} 400-1000}. New York: Routledge, 2011.

\item \versal{LIND}, Lennart. \emph{Roman denarii: Hoards and stray finds in
Sweden}, Stockholm Numismatic Institute, Stockholm University, 2013, vols. 1 e
2.

\item \versal{NAISMITH}, Rory; \versal{ALLEN}, Martin; \versal{SCREEN}, Elina (orgs.). \emph{Early
Medieval Monetary History: Studies in Memory of Mark Blackburn}.
Farnham: Ashgate, 2014.

\item \versal{SIGMUNDSSON}, Svarar (org.). \emph{Viking Settlements \& Viking Society.
Proceedings from the 16th Viking Congress held in Reykjavík and Reykholt
in Iceland in August 2009}. Reykjavik: University of Iceland Press,
2011.

\item \versal{SKAARE}, Kolbjørn. \emph{Coins and coinage in Viking-age
Norway: the establishment of a national coinage in Norway in
the \versal{XI} century, with a survey of the preceding currency history}. Oslo:
Universitetsforlage, 1976.
\end{itemize}
\section{\versal{MORKINSKINNA}}

Trata-se de um manuscrito composto na Islândia em nórdico antigo,
aproximadamente em 1220, no qual é encontrado uma compilação sobre a
vida dos reis da Noruega. É considerado por alguns autores como a
primeira coleção de compêndios de sagas dos reis noruegueses,
acompanhada por obras similares, como a \emph{Fagrskinna} e a
\emph{Heimskringla}, compostas anos depois. Pertence, portanto, à
tradição da \emph{konungasögur}. Seu conteúdo se estende entre os anos
de 1030 e 1157, abrangendo o período da morte de Olavo, o Santo, até o
reinado dos filhos de Haroldo Gilli. Em comparação com as sagas
anteriores, apresenta uma narrativa mais detalhada.

O autor da \emph{Morkinskinna} foi um islandês que conhecia muito bem a
história da Noruega. Na obra, percebe-se uma certa preocupação com os
islandeses na corte norueguesa e a narrativa está repleta de anedotas
sobre os islandeses. Embora a identidade do autor ainda permaneça no anonimato,
a preocupação com os poetas que faziam parte da corte norueguesa
talvez indique que era, também, um poeta ou um biógrafo real.

A estrutura da \emph{Morkinskinna} difere de obras de autores como
Snorri Sturlusson, bem como de outros escritores de sagas do mesmo contexto,
apresentando uma característica mais próxima a autores de romances. O
autor está interessado nas características da realeza e, principalmente,
nas virtudes dos reis, focando nos contextos em que havia dois reis
concomitantes no reino da Noruega. Embora se trate de um compêndio de
sagas de reis da Noruega, o \emph{Morkinskinna} recebeu uma significativa 
atenção por parte da crítica. Além de apresentar uma lista de reis, sua estrutura
apresenta pequenas histórias sobre os islandeses conhecidas como
\emph{þættir}, além de digressões anedóticas.

Alguns autores, como Anderson e Gade, destacam que a quantidade de
inserções desses \emph{þættir} na narrativa da \emph{Morkinskinna}
indica uma certa falta de interesse cronológico por parte do autor da
obra. Mesmo assim, embora apresentem essa diversidade estrutural, alguns
autores destacam que os \emph{þættir} estão perfeitamente incluídos
dentro da narrativa do documento e apresentam uma função de
\emph{exempla} (tal como observamos nos documentos continentais
ocidentais do período), voltada para o desenvolvimento da figura do rei
na sociedade e para formas de ensino relativo ao comportamento nas
cortes. Ademais, na narrativa também se percebe a preocupação
em demonstrar a relação do rei para com os seus homens, no
sentido de aceitar seus conselhos. Apresenta, nesse sentido, um certo
modelo comportamental no âmbito da corte.

\SIG{Luciano José Vianna}

Ver também Fontes primárias; Historiografia e pseudo-história; Noruega
da Era Viking.

\begin{itemize}
\item \versal{ANDERSSON}, Theodore; \versal{GADE}, Kari Ellen. \emph{Morkinskinna: The Earliest
Icelandic Chronicle of the Norwegian Kings (1030-1157)}. Ithaca: Cornell
University Press, 2000.

\item \versal{HOLMAN}, Katherine. \emph{Historical Dictionary of the Vikings}. Lanham,
Maryland, and Oxford: The Scarecrow Press, Inc. 2003, p. 192.

\item \versal{JAKOBSON}, Ármann. Royal Biography. In: \versal{McTURK}, Rory (ed.). \emph{A
Companion to Old Norse-Icelandic Literature and Culture}. Oxford:
Blackwell Publishing, 2005, pp. 388-402.

\item \versal{MIRANDA}, Pablo Gomes de. Sagas reais (\emph{Konungasögur}). In: \versal{LANGER},
Johnni (org.). \emph{Dicionário de mitologia nórdica. Símbolos, mitos e
ritos}. São Paulo: Hedra, 2015, pp. 445-447.

\item \versal{OLÁSON}, Vésteinn. Family sagas. In: \versal{McTURK}, Rory (ed.). \emph{A
Companion to Old Norse-Icelandic Literature and Culture}. Oxford:
Blackwell Publishing, 2005, pp. 101-118.

\item \versal{ROWE}, Elizabeth Ashmane; \versal{HARRIS}, Joseph. Short Prose Narrative
(\emph{páttr}). In: \versal{McTURK}, Rory (ed.). \emph{A Companion to Old
Norse-Icelandic Literature and Culture}. Oxford: Blackwell Publishing,
2005, pp. 462-478.
\end{itemize}
\section{\versal{MULHERES}}

As mulheres na sociedade nórdica possuíam papéis bem distintos. Elas eram
comandadas pelos homens e de cada uma eram esperados determinados
comportamentos que não podiam ser contrariados sem que houvesse
algum tipo de punição. As mulheres não participaram de trocas comerciais
ou de incursões, embora tenham participado claramente em viagens de
exploração e assentamentos em lugares como Islândia e Vínland. As
responsabilidades das mulheres sempre foram claramente definidas como
domésticas. Tanto homens como mulheres que se aventuravam em tarefas e
atividades não condizentes com o seu sexo eram execrados. Alguns
desses comportamentos eram, inclusive, estritamente proibidos por lei.
Segundo as \emph{Grágás}, as mulheres usavam roupas masculinas, cortavam 
o cabelo curto e transportavam armas. Elas viviam sob a
autoridade de seu pai e, na ausência deste, ficavam sob
a tutela dos irmãos ou parentes próximos enquanto fossem solteiras
e do marido, depois de casadas.
Elas possuíam uma liberdade limitada para dispor de seus bens. 
Eram proibidas de participar da maioria das atividades
políticas, não podendo exercer a função de chefe, de juiz ou mesmo
servir como testemunha. Em hipótese alguma podiam ter voz em uma
\emph{Thing}.

Mas, em contrapartida, as mulheres eram respeitadas e possuíam uma
grande liberdade, especialmente quando comparadas com as mulheres de outras sociedades
europeias da época. Elas conseguiam administrar as finanças da família e
podiam supervisionar a fazenda na ausência de seu marido, exercendo sua
autoridade frente aos servos e escravos sem ser contestada. Na viuvez,
elas podiam ser ricas, importantes proprietárias de terras e bens, 
podendo dispor de toda a sua fortuna como bem desejassem. Também existiam
leis que protegiam as mulheres de várias atitudes masculinas indesejadas,
como beijos forçados e estupros.

As sagas são fontes fundamentais para entendermos o papel das mulheres
na sociedade nórdica e um bom exemplo do poder feminino é o da
personagem Aud, a de ``mente profunda'', da \emph{Laxdæla saga}. Ela
abandona a Noruega para viver com a família em terras escocesas, mas,
devido aos conflitos enfrentados, parte de sua família é morta. Diante disso, Aud
providenciou a construção de um navio, reuniu sua
família (e agregados) e partiu para a Islândia. Uma vez na Islândia, ela
reivindicou terras e constituiu uma fazenda. Ao longo dos anos, 
distribuiu porções de suas propriedades, além de ter
arranjado casamentos para suas filhas. Em suma, Aud assumiu todas as
responsabilidades normalmente designadas ao marido. Quando morreu, Aud
foi colocada em um navio que lhe serviu de túmulo, uma honra
normalmente reservada apenas para os homens mais poderosos e ricos.

Mas nem só de mulheres ricas e poderosas com poder de comando era
constituída a sociedade nórdica. No cotidiano, as mulheres possuíam uma
alta carga de trabalho, pois, na vida de homens, mulheres e crianças que
habitam uma região com invernos rigorosos, a luta pela sobrevivência
exigia muito de todos. No dia a dia, as responsabilidades das mulheres
estavam restritas ao mundo doméstico e à manutenção da vida. Tarefas como
a preparação dos alimentos, lavagem das roupas, cuidados infantis,
cardação, fiação e tecelagem eram executadas dentro das casas
diariamente. As tarefas da fazenda incluíam a ordenha e a preparação
dos derivados de leite como por exemplo o \emph{skyr}, uma espécie de
queijo, era tarefa exclusivamente feminina. A linha divisória entre as
responsabilidades masculinas e femininas normalmente estava localizada
na entrada da casa. As mulheres estavam encarregadas de tudo o que corresponde 
ao interior enquanto os homens tinham responsabilidade por tudo o que 
corresponde ao exterior.

A maioria das sagas de família islandesas narram os feitos masculinos e,
muito provavelmente, foram escritas por homens. As mulheres, nessas
narrativas, desempenhavam apenas papéis secundários, sendo descritas como
possuidoras de uma personalidade forte e marcante. Essas personagens
femininas são reverenciadas pela sua beleza, mas principalmente pela sua
sabedoria. Muitos dos traços de caráter considerados positivos nos
homens, como, por exemplo, um forte senso de honra, coragem e valentia,
são também considerados traços positivos nas mulheres.

Uma outra característica das mulheres nas sagas, assim como na vida
social, é o papel de incitadoras. As mulheres frequentemente estimulavam
os homens a agir (a se vingar, por exemplo) quando estes estavam desestimulados
frente a uma situação difícil. As mulheres, em algumas situações,
mostram-se mais ávidas para proteger a honra da família, devido ao seu
papel passivo. Sem poder partir para a ação, encontravam nas
palavras de estímulo a sua arma.

Uma mulher poderia usar a ameaça de divórcio como um meio para estimular
seu marido a agir. O divórcio era relativamente fácil e poderia resultar
em grandes encargos financeiros para o marido, portanto a melhor
alternativa para o homem, nesse caso, seria escutar a sua esposa,
evitando assim a perda de bens. 

As mulheres também eram hábeis praticantes de magia. Em casos
específicos, essas habilidades eram consideradas como um grande mal, de modo que
algumas praticantes eram banidas e até mortas por fazerem uso da magia.
A magia era considerada uma prática essencialmente feminina e, caso um
homem se aventurasse pelas searas da magia, seria considerado como um
infame. Mas, em alguns casos específicos, o uso da magia não era visto
como algo negativo. Pelo contrário, era considerado necessário para livrar a
comunidade das dificuldades e mazelas causadas pela carestia. A
personagem Thorbjorg, da \emph{Saga de Eiríkr}, é descrita como uma mulher velha e
sábia, a mais nova de nove irmãs, que possuía a capacidade de prever o
futuro e fazer profecias. Tais atributos lhe concediam um \emph{status} de
prestígio nessa comunidade groenlandesa, que atravessava um momento
crítico devido a ausência de caça e, portanto, vivia uma época de
extrema fome. Thorbjorg foi convidada a prever o destino
da comunidade, que naquele momento enxergou em seus poderes mágicos uma
saída para a crise em que viviam. Além da magia de caráter propiciatório ou
conjuratório, as mulheres encontravam na fofoca e na propagação de
boatos, quase sempre difamatórios, uma forma alternativa de poder, que
não necessitaria de força física, habilidade em armas ou de parentesco com
famílias abastadas. As mulheres, ricas ou pobres, encontravam na magia e
na fofoca os seus nichos de poder mais fortes e eficazes.

As mulheres não toleravam nenhum tipo de galanteio que ocorresse sem o
seu consentimento ou que as forçassem a fazer algo que elas não tivessem
vontade, como, por exemplo, beijos forçados em locais públicos. O homem
que cometesse qualquer ato que contrariasse o desejo da mulher seria
condenado a pagar uma espécie de multa indenizatória para a família da
mulher ofendida. Era uma desonra grave para um homem ferir uma mulher,
mesmo acidentalmente, em um ataque a uma casa. Se, por acaso, a casa fosse
queimada para matar os ocupantes, as mulheres e as crianças podiam sair
sem sofrer qualquer tipo de agressão. Mesmo as chacotas mais agressivas
ou mesmo atos de violência simulada (como, por exemplo, ameaçar com uma faca ou
qualquer outro objeto) eram também atos inaceitáveis e seriam reprovados
não só pela família, mas por toda a comunidade.

A exceção a essas regras de proteção e respeito às mulheres aconteceu
durante as incursões de pirataria e comércio, pois durante esses ataques
as mulheres eram rotineiramente levadas como saque para serem vendidas
como escravas e, portanto, estavam sujeitas a todo e qualquer tipo de
violência. É preciso lembrar que essas mulheres não eram nórdicas, mas
estrangeiras, geralmente irlandesas ou eslavas. É igualmente importante salientar
que o estupro de mulheres, que poderia ser parte da violência típica de
uma batalha ou ataque, não era muito praticado pelos vikings durante 
suas invasões, principalmente se comparado com outros invasores europeus da
época, como os carolíngios. A violação era um crime hediondo e altamente
repudiado pela sociedade nórdica.

Portanto, podemos observar que as mulheres, mesmo não podendo participar
das assembleias que discutiam os problemas mais graves da comunidade (ou mesmo 
portar armas), aprendiam a se defender e a defender aqueles que estavam sob sua
guarda e proteção. Na ausência dos homens, podiam contar com leis que as
protegiam e, caso alguém atentasse contra a sua honra e a sua vida, a
punição era inevitável. As mulheres nórdicas gozavam de visibilidade,
importância e respeito dentro da comunidade, sendo sempre vistas como
essenciais à manutenção da vida de todos.

\SIG{Luciana de Campos}

Ver também Aud; Freydis; Gudrun; Guerreiras nórdicas; Estupro; Sexo e
sexualidade; Sociedade.

\begin{itemize}
\item \versal{ANDERSON}, Sara \& \versal{SWENSON}, Karen. \emph{The Cold Counsel: The Women in
Old Norse Literature and Myth}. London: Routledge, 2002.

\item \versal{FRIDRIKSDÓTTIR}, Jóhanna Katrín. \emph{Women in Old Norse Literature:
bodies, words, and power}. London: \versal{AIAA}, 2013.

\item \versal{JESCH}, Judith. \emph{Women in the Viking Age}. London: Boydell \& Brewer
Ltd, 1999.

\item \versal{JOCHENS}, Jenny. \emph{Women in Old Norse Society}. Ithaca:
\href{http://www.cornellpress.cornell.edu/publishers/?fa=publisher\&NameP=Cornell\%20University\%20Press}{Cornell
University Press}, 1995.

\item \versal{LEE}, Christina. Viking Age women. In: \versal{HARDING}, Stephen. \emph{In search
of vikings}. London: \versal{CRC} Press, 2015, pp. 60-70.
\end{itemize}
\section{\versal{MÚSICA}}

Não fazemos ideia de como era a música na Era Viking. Temos acesso à
cultura material, que cada vez mais nos revela sobre os instrumentos
musicais. A cultura escrita nos assegura que a música não só existiu, mas
era parte importante da vida e do cotidiano dos povos da
Escandinávia.

Entre os poemas da Edda poética, o deus Heimdall, apesar de não estar
relacionado à música, toca a sua trombeta Gjallarhorn, dando início ao
crepúsculo dos deuses, evento que, segundo o poema ``A Profecia da
Vidente'' (\emph{Vǫluspá}), também contou com um pastor, de nome Eggthér, tocando
a sua harpa. A harpa é um instrumento famoso nas narrativas heroicas
germânicas conectadas ao mundo escandinavo. Em Beowulf, o instrumento
está sempre animando as festas em Heorot. Dois poemas sobre os feitos de
Átila narram como o rei Gunnar morre com as mãos atadas e em um poço
com serpentes, tocando a sua harpa com os seus pés. Questiona-se,
adicionalmente, se poemas como ``A Canção da Gróa'' e ``Os Versos das Lanças''
(\emph{Grottasǫgr} e \emph{Darraðarljóð}) não seriam canções próprias
para o trabalho.

Nas sagas islandesas, os exemplos são vários e não caberiam neste verbete
todas as citações possíveis. Vamos nos ater a uma passagem em particular,
na saga do rei Haroldo Severo (Haraldr Harðráði), que consta na compilação
de sagas nomeada de \emph{Morkinskinna}. Nesta, o rei conta orgulhoso que, 
dentre as suas várias habilidades, possui conhecimento nas rimas escáldicas, é
sendo capaz de entender
tanto os poemas quanto as técnicas de harpa. O \emph{jarl} Rognvald Kali 
Kolsson, na \emph{Saga dos Colonos das Ilhas
Órcades} (\emph{Orkneyinga saga}), se gaba de possuir as mesmas habilidades.

Não obstante os vestígios dos instrumentos musicais 
encontrados na região datando desde a Era do Bronze, como o lur -- uma
espécie de trompa de bronze de procedência celta, encontrado em
Brudevælte, na Dinamarca --, só podemos vislumbrar o registro de uma
melodia no \emph{Codex Runicus} (documento escrito em torno do 
século~\versal{XIV}, no qual está escrito o que deve ter sido parte de uma balada):
\emph{Drømde mik em drøm i nat um, silki ok ærlik pæl}, usualmente
traduzido como ``Eu sonhei um sonho ontem a noite, de seda e rico
tecido''. Infelizmente, estamos em um período posterior a Era Viking e
nada parecido é encontrado antes.

Por outro lado, é possível encontrar alguns testemunhos oriundos do
trabalho de viajantes, emissários ou escritores de outras culturas,
contemporâneos a Era Viking. De Hedeby, no século~\versal{X}, Ibrahim ibn Yaqub
al-Tartushi narra: ``nunca escutei nada mais horrendo que o canto
dos eslévicos [habitantes de Schleswig ou Eslévico]. É um zumbido vindo
direto de suas gargantas, que é pior que o latido de cães''. Ibn Fadland
também apresenta o seu testemunho na \emph{Risala}. Ao acompanhar um
funeral de um chefe guerreiro, nos diz Ibn Fadland que os homens consumiam bebidas
por dez dias, engajavam em intercurso sexual com as mulheres e tocavam
instrumentos musicais. 

Analisando brevemente o relato de al-Tartushi, podemos conceber um tipo
de canto gutural, normalmente associado à cultura sámi, mongol ou, no
geral, de populações siberianas. É possível, ainda, que mais de uma
pessoa cantasse ao mesmo tempo, mas isso são apenas conjecturas. Devemos
lembrar que esses viajantes possuíam uma educação cosmopolita e
estavam acostumados a gostos extremamente refinados dentro dos seus próprios
padrões culturais, de modo que não devemos ler essas fontes sem qualquer
tipo de crítica.

Alcuíno escreve em carta endereçada ao bispo de Lindisfarne aduzindo que os
homens não deveriam escutar o som da cítara. Essa
passagem está em acordo com uma descrição muito curiosa de Adão de
Bremen sobre a música tocada no templo de Uppsala, na qual é dito que os
cânticos são numerosos e obscenos, sendo melhor não falar sobre
eles. Fato é que a música era um componente vital para as relações
mágico-religiosas na Escandinávia pré-cristã, costume atestado nas
sagas, como na \emph{Saga de Érico, o Vermelho} (\emph{Eiríks saga Rauða}),
na qual uma garota precisou cantar um feitiço junto a uma feiticeira
itinerante para que os espíritos trouxessem benesses à comunidade.

Sobre os instrumentos, já foram encontradas trompas de bronze -- os lures -- 
sendo as mais antigas datadas do século~\versal{X} a.C. Geralmente são
simples, com uma campana decorada com depressões e fixada em um corpo
longo e curvo, terminando em um bocal. Alguns lures também possuem
chocalhos em uma das suas extremidades, composto de placas que batem
umas nas outras. Geralmente esses lures eram curvos, com o propósito de
serem carregados com maior facilidade. Encontramos também correntes
afixadas. Provavelmente, as trompas de bronze cumpriam funções religiosas, pois as
encontramos em pares e como oferendas nos depósitos em pântanos. Também
podemos identificá-las na pintura de diversas estelas e paredões, o que
reforça esse papel.

Da Era Viking podemos encontrar lures retos, feitos em madeira,
principalmente a partir da Bétula. Podemos apontar, como exemplo, os vestígios
encontrados no sepultamento de Oseberg. Não sabemos com
certeza se esses eram instrumentos utilizados para fazer música ou
apenas em atividades agropastoris. Tal atividade, na
Escandinávia, era acompanhada pelo uso de um instrumento muito parecido com tais lures (ou
com a trombeta de bétula), datados a partir do século~\versal{X} e 
utilizados até data recente.

Outros exemplos de aerófonos são as diversas flautas feitas do chifre de
vaca. O exemplo encontrado em Västerby, na Suécia, possui um bocal e
quatro furos para encaixar os dedos. Outra, em Värmland, também na
Suécia, foi encontrada com cinco furos. Todas essas flautas são simples,
possuem menos de trinta centímetros e um bocal feito de madeira.
Vestígios de flautas de ossos também são encontradas durante a Era
Viking em abundância, feitas dos ossos de vacas, cervos e pássaros de
grande porte. Elas vão de pequenos flautins de dois furos, como o achado
em Birka, às flautas maiores com diversos furos. A de Aahrus possuía
sete.

Duas flautas muito únicas devem ser mencionadas aqui: uma Flauta-de-Pã, 
encontrada em York e datada do século~\versal{X}, feita em um bloco retangular de
madeira, com cinco furos feitos na vertical com diferentes profundidades,
criando diferentes tons. A Flauta de Falster data da segunda
metade do século~\versal{XI} e só foi encontrado um tubo de madeira com cinco
furos, sugerindo que ele tenha sido parte de um instrumento maior. Foram
sugeridos que essa flauta poderia ser parte de uma Gaita-de-Foles, mas
as reconstruções modernas adicionaram um bocal e o seu som lembra muito
um Oboé, ou um \emph{Hornpipe}.

Foram achados no navio de Oseberg um conjunto de cinco chocalhos: uma
série de anéis de metais interligados e presos a dois bastões. Não obstante
o provável uso mágico-religioso, não estamos certos se tais artefatos
foram utilizados como instrumentos musicais. Outros chocalhos
desse tipo também foram encontrados em Stövernhaugen e em Akershus e são
datados do período entre o século~\versal{IX} e \versal{XI}. Ambos consistem em anéis presos em torno
de um anel maior. O chocalho de Stövernhaugen ocupava o topo de um
bastão de um metro e setenta. Se o conjunto todo fosse batido no chão,
poderia gerar um ritmo baseado na pancada desse bastão e o retinir do
chocalho. O exemplo de Akershus lembra adereços similares achados em
arreios para cavalos.

Surpreendentemente, não encontramos qualquer indício de tambores entre
as sociedades germânicas na Era Viking. Sabemos que outras etnias
próximas possuíam esses instrumentos, como os tambores \emph{Goavddis} e
\emph{Gievrie} das culturas Sámis. Vale
mencionar, por outro lado, que alguns pesquisadores consideram a batida
em escudos como indício de música. É o caso no poema Elogio a Ragnar
(\emph{Ragnarsdrápa}), mas o melhor exemplo é, novamente, fornecido pelo relato de Ibn
Fadlan, que, no funeral, relata o sacrifício de uma escrava no qual, 
com a finalidade de abafar seus gritos,
os homens batiam nos seus escudos.

Sobre os instrumentos de corda encontrados na Escandinávia e
pertencentes a Era Viking, podemos citar copiosos exemplos encontrados
na cultura escrita e na cultura material. Nem sempre são fáceis de
identificar. Em uma escultura na catedral de Nidaros, em Trondheim, um
homem toca um instrumento composto por uma tábua de madeira e três
cordas, o que sugere ser algum tipo de \emph{jouhikantele}, uma cítara
importada da Finlândia. Trata-se, porém, de uma hipótese ainda objeto de
debate entre os pesquisadores. A rabeca e o giga (\emph{fiðla} e
\emph{gigja}), por exemplo, são mencionados várias vezes nas sagas
islandesas, mas nenhum exemplar foi encontrado. Supomos apenas que
seriam instrumentos estrangeiros trazidos por artistas itinerantes que
passaram a compor as cortes dos reis escandinavos, haja vista que o
processo de formação dos reinos escandinavos significou, entre outras
coisas, uma integração à cultura da Europa continental.

As harpas e liras, entre os instrumentos de cordas, são um caso à parte. 
Uma reconstrução da Lira de Sutton Hoo foi feita a partir dos
fragmentos arqueológicos encontrados e nos revelam um instrumento
construído a partir de tábuas de madeira, com uma cavidade no centro
cortado por seis cordas. Essas cordas são presas por tarraxas de madeira
em um lado e unidas do outro. A lira de Sutton Hoo, apesar de ser
Anglo-Saxônica e se assemelhar a outras liras da Europa continental, nos
oferece um vislumbre de como podem ter sido as liras nórdicas, tanto
pela proximidade das regiões, como em comparação com as liras
encontradas mais tarde em território escandinavo. A lira de Kravik,
encontrada em Oslo e datada do século~\versal{XIII}, nos oferece um paralelo
interessante, mostrando como as diferenças com a lira Anglo-Saxônica
foram poucas: uma corda a mais e pequenas alterações no corpo do
instrumento.

\SIG{Pablo Gomes de Miranda}

Ver também Cotidiano; Literatura; Sociedade.

\begin{itemize}
\item \versal{BIRDSAGEL}, John. Music and Musical Instruments. In: \versal{KIRSTEN}, Wolf;
\versal{PULSIANO}, Phillip (orgs.). \emph{Medieval Scandinavia: An Encyclopedia}.
New York: Garland, 1993. pp. 420-423.

\item \versal{GRINDE}, Nils. \emph{A History of Norwegian Music}. Lincoln: University
of Nebraska Press, 1991.

\item \versal{HORTON}, John. \emph{Scandinavian Music: a short history}. London: Faber,
1963.

\item \versal{TSUKAMOTO}, Chihiro. \emph{What Did They Sound Like? Reconstructing the music
of the Viking Age}. Dissertação (Mestrado em Estudos
Islandeses Medievais). Reykjavik: Universidade da Islândia, 2017.
\end{itemize}
\chapter{N \textarn{n} \textart{n}}
\section{\versal{NAVEGAÇÃO MARÍTIMA}}

\emph{Conceito geral}: navegação é a arte de determinar ou manter um
navio em uma direção específica, ou, ainda, a arte de dirigir uma embarcação em
qualquer tipo de situação marítima. Isso implica a determinação da
posição e da direção da viagem em qualquer momento e em qualquer local. As
técnicas de navegação dos nórdicos durante a Era Viking ainda carecem de
maiores pesquisas, mas diversos estudos já demonstram um certo nível de
sofisticação, bem como o uso de orientações astronômicas e de equipamentos.
Desde os anos 1950, temos duas teorias interpretativas da navegação
nórdica: a que defende um conhecimento empírico (baseado na tradição
oral e astronomia, mas sem o uso de qualquer tipo de instrumental, a
exemplo dos argumentos de Jan Bill e Richard Hall), com o predomínio da
cabotagem; e outra que argumenta em favor do uso de equipamentos.
Pesquisadores mais recentes, por sua vez, defendem uma fusão entre as
duas interpretações. As técnicas de navegação da Era Viking foram
desenvolvidas primeiramente nas águas da Escandinávia e, posteriormente,
levadas para outras áreas da Europa, basicamente através de
memorizações.

Os nórdicos utilizavam os mesmos métodos em suas extensivas incursões em
diferentes regiões, das ilhas britânicas até a Islândia, Groenlândia e
América.

\emph{Rotas e direção}: Apesar da distância mais curta entre a Noruega e
a Groenlândia ser de 1.500 km e Islândia à Groenlândia ser 560 km, os
nórdicos preferiam normalmente seguir as rotas mais longas, beirando a
Islândia pelo Sul e evitando os icebergs pelo norte. Provavelmente os
primeiros marinheiros não perdiam nunca a vista da terra, realizando
preferencialmente uma navegação de cabotagem, seguindo de ilha em ilha.
Isso fazia com que a viagem demorasse muito mais do uma navegação em
linha reta pelo mar. Mas evitava que a embarcação se perdesse no oceano.

\emph{Orientação por pássaros cativos}: para evitar os dias e as noites
nubladas (que impedem a localização pelo Sol e pelas estrelas), existem
referências de que eram utilizados pássaros como localização náutica.
Segundo o \emph{Landnamabók}, o primeiro homem a navegar pela Islândia,
Flóki Vilgerðarson, levou em sua viagem vários corvos porque não sabia o
caminho exato para seguir. Quando não via mais a costa, Floki soltou um
dos pássaros, que imediatamente retornou para a ilha Feroé. Mais tarde,
soltou um segundo pássaro, que voou alto, mas retornou ao navio. No dia
seguinte, soltaram um terceiro pássaro, que voou até um ponto de
horizonte e foi seguido pela embarcação. Logo descobriram a Islândia.
Para Jan Bill, essa passagem foi influenciada pelo referencial cristão
(remete à narrativa de Noé), mas não invalida a interpretação do uso de
pássaros como auxiliares na navegação nórdica.

\emph{Orientação por instrução e cognição}: Quando uma tripulação já
havia seguido por certa rota, o piloto dava instruções a outros
navegadores, dispensando o uso de pássaros nas viagens. Bjarni
Herjólfsson (segundo a \emph{Saga dos Groenlandeses}, o primeiro europeu
a avistar a América), quando chegou à terra dos bosques, supôs que não
se tratava da Groenlândia, pois, apesar de estar perdido, as instruções
que havia recebido permitiriam encontrar seu caminho por 800 quilômetros
em mar desconhecido, até as terras que jamais havia conhecido antes. Por
sua vez, foi capaz de transmitir as informações necessárias para que Leif
Ericsson e os habitantes da Groenlândia pudessem chegar até a nova terra.
Esse tipo de instrução apenas descrevia claramente os pontos de referência
próximos das novas terras, assim como a latitude em que
apareciam: por exemplo, a montanha mais alta de uma costa. Segundo o
pesquisador Ian Atkinson, um modelo de instruções para navegação da
Islândia para a Groenlândia seria, a partir de Snaefelli (a colina nevada) -- o ponto
mais alto da costa islandesa e o último ponto antes de desaparecer no
horizonte --, continuar navegando até ver Blaserk (a camisa negra), uma
montanha de 3.500 metros da Groenlândia. Entre uma montanha a outra se
passaria mais ou menos um dia. Também o marinheiro daria outras
informações de referências sobre o largo da costa groenlandesa, assim
como sobre a profundidade das águas de cada ponto, medidas com corda e
peso.

Os pesquisadores Indruszewski e Godal definiram os padrões necessários
para a criação de um mapa cognitivo na mente do navegador nórdico: 1.
Primeiro reconhece-se os pontos mais peculiares da paisagem terrestre e
marinha que, por sua vez, são definidas essencialmente pelas condições do clima e o tempo de
observação (dia ou noite); 2. O segundo passo é transformar a paisagem
em marcas para navegação -- um penhasco com cerca de 150 metros de
altura que pode ser visível a 20 km de distância de uma costa; 3. O
passo seguinte é transferir os pontos náuticos para uma orientação mais geral e
complexa, baseada em referências objetivas (observações astronômicas) e
subjetivas (instinto humano); 4. Finalmente, essas informações devem se
transformar em dados para outras pessoas

\emph{Orientação por animais}: Um marinheiro experiente podia conseguir
pontos de referências em alto mar, como uma zona a meio dia de viagem
ao sul da Islândia, onde se concentram grandes grupos de baleias para
comer. Também pássaros migratórios, em seus voos anuais, podem propiciar
boas informações, pois seguem sempre a mesma rota (é o caso dos gansos
selvagens que voam entre a Inglaterra e a Islândia). Algumas aves
marinhas não são vistas nas Shetland, mas são comuns na Islândia, por
exemplo.

\emph{Orientação por estrelas}: O principal ponto de referência estelar
de quase todos os marinheiros do hemisfério norte desde a Antiguidade é
a estrela Polaris, alfa da constelação Ursa Menor, situada quase
exatamente no Norte Celeste. Ela praticamente não se
move, ao contrário de grande parte da abóbada celeste que gira em seu
redor. Mas, como a posição da Polaris é variável conforme a época do ano,
torna-se importante a utilização de algum tipo de instrumento para medir sua altura no
céu e, consequentemente, a latitude do local, definida sempre em relação aos pontos
cardeais. Segundo o pesquisador Ian Atkinson, o navegador nórdico
utilizava um bastão vertical (ou mesmo o mastro do navio) para marcar o
ponto onde aparecia a estrela enquanto ainda está próximo da terra. Depois, em alto
mar, voltava a utilizar o bastão e, se a estrela aparecia no mesmo
ponto, significava que a embracação estava na mesma latitude. Se surgia em um ponto mais alto,
significava que a embarcação estava numa latitude maior, mais próxima
do Pólo Norte geográfico. Em terra, tal método permite determinar a
latitude em um raio de 24 km.

Não temos conhecimento de nenhum tipo de instrumento nórdico para esta
finalidade (nem nas fontes literárias), mas sabemos que os navegantes do
mundo islâmico utilizavam, durante o século~\versal{IX} d.C., um 
equipamento denominado de \emph{kamal} (criado anteriormente pelos hindus e chineses),
que consistia em um pequeno quadrado de madeira com um barbante preso
ao centro. A tábua era estendida na distância do braço do piloto e o fio
esticado até o rosto, de modo que a estrela era observada pelo canto superior do 
\emph{kamal}, enquanto o canto inferior era nivelado com o horizonte. O
ângulo formado pela linha da estrela e a linha do horizonte permitia definir a
latitude do observador.

Em 1996, Engstrom e Nykanen propuseram, na revista \emph{Fornvännen}, que
os cata-ventos inseridos nas proas das embarcações nórdicas podem ter
sido utilizados como instrumentos para determinar a posição do Sol e das
estrelas, funcionando quase como uma espécie de quadrante (hipótese
proposta, inicialmente, em 1975, por Svend Larsen). Na navegação noturna, tal quadrante
poderia determinar a altura da estrela acima do horizonte e, consequentemente, 
a latitude do navio. A mesma revista publicou uma
contestação dessa pesquisa, elaborada por Arne Christensen, em 1998, mas que apresenta caráter
muito superficial. De qualquer modo, o conhecimento sobre instrumentos
nórdicos utilizados na navegação ainda carece de mais investigações e
debates.

A maior limitação do suposto método envolvendo a Polaris reside no fato
de que, no período do verão, as estrelas não são visíveis nas regiões árticas
devido ao fenômeno do Sol da meia noite. Testamos no programa
\emph{Stellarium} duas latitudes: na Islândia (\versal{N} 65°), Polaris não foi
visível em nenhuma hora do dia ou noite na data de 6 de junho do ano 1000 d.C.;
já na região de Sept-Iles, no Canadá (\versal{N} 51°), a estrela foi
visível durante a noite na mesma data. Outras estrelas também podem ter sido
utilizadas pelos navegantes antigos como indicadoras de direções
marítimas, como Altair (Leste), Antares (Sudeste) e Capela (Nordeste).

\emph{Orientação pelo Sol}: Certamente o Sol foi o principal referencial
para determinar a localização e o direcionamento das empreitadas
náuticas pelo Atlântico Norte. O curso aparente do astro rei pelos céus
de oriente para ocidente depende da altura do observador e da época do
ano. A única direção fixa, sem levar em conta a época do ano e a altura
do Sol, é constituída quando este se encontra no zênite (no ponto mais alto do
céu), ao meio dia. Não é fácil calcular as direções quando o Sol se
encontra em outro ponto, mas é possível fazer isso nas viagens de curta
duração, com poucas alterações de latitude, recorrendo aos conhecimentos
sobre os movimentos do Sol memorizados em terra antes do embarque.
Segundo pesquisadores, como Thirslund e Vebaek, os nórdicos puderam
realizar cálculos bem mais precisos de posicionamento empregando gnômons
de madeira (ver verbete \versal{Bússola solar}), provavelmente com uma margem de
±5°. Também o emprego das famosas pedras solares, mencionadas em várias
sagas islandesas, pode ter auxiliado os navegantes a localizar o Sol,
especialmente em tempo nublado (ver verbete \versal{Pedra solar}).

\SIG{Johnni Langer}

Ver também Astronomia; Bússola solar; Embarcações; Mar Báltico; Oseberg;
Pedra solar; Sagas do Atlântico Norte.

\begin{itemize}
\item \versal{BILL}, Jan. Navigation. In: \versal{SAWYER}, Peter (ed.). \emph{The Oxford
Illustrated History of the Vikings}. Oxford: Oxford University Press,
1997, pp. 197"-199.

\item \versal{ENGSTROM}, Jan \& \versal{NYKANEN}, Panu. \emph{New interpretations of Viking Age
weathervanes}. Fornvännen, vol. 91, n. 3, 1996, pp. 137-142.

\item \versal{INDRUSZEWSKI}, George \& \versal{GODAL}, Jon. Maritime skills and astronomical
knowledge in the Viking Age Baltic Sea. \emph{Studia Mythologica
Slavica}, vol. 9, 2006, pp. 15-39.

\item \versal{KARLSEN}, Leif. \emph{Secrets of the viking navigators}. Seattle: One
Earth Press, 2003.

\item \versal{THIRSLUND}, Soren. \emph{Viking navigation}. Oslo: Viking Ship Museum,
2007.
\end{itemize}
\section{\versal{NJÁLS SAGA}}

A \emph{Njáls saga} ou \emph{Saga de Njáll, o queimado} é a mais extensa
e complexa das \emph{Íslendingasögur}. É uma obra anônima que foi
composta em princípios da segunda metade do século~\versal{XIII}, quando ainda
estava muito recente a perda da independência  para os noruegueses e
o desmantelamento do chamado ``estado livre islandês''. A saga está
conservada em uma grande quantidade de manuscritos dos séculos~\versal{XIV-XVII},
tanto em pergaminho como em papel (os mais tardios), sendo o mais antigo 
denominado \versal{AM} 468 4to, \emph{circa} 1320 d.C. Apesar dessa
riqueza documental, nenhum dos manuscritos preserva a saga em sua forma completa.
Somente seis contêm o que os especialistas consideram os núcleos
temáticos principais da saga, tornando
extremamente complicada a reconstrução do que se supõe como seu texto
original. Graças ao novo projeto do \emph{Árni
Magnússon Institute for Icelandic Studies} -- \emph{The Variance of Njáls
saga} --, se tem aprofundado (e, possivelmente, superado) a classificação de
Einar Ólafur Sveinsson, de 1953, por meio do exame das diferentes versões e
manuscritos da saga em perspectiva linguística, filológica e literária.
A importância da \emph{Njáls saga} no conjunto da literatura islandesa
medieval, sobretudo no subgênero das Sagas Islandesas, foi tal
que acabou por dar origem a múltiplas tradições, de caráter secundário, em
forma de lendas, baladas, cabendo mencionar também as estrofes \emph{rímur}, que versa sobre
os principais personagens e acontecimentos da saga, alguns dos quais
estão contidos em manuscritos datados dos séculos~\versal{XVI}~e~\versal{XVII}.

A \emph{Njáls saga} nos oferece
uma das imagens mais completas da Islândia da época heroica, visto que
os fatos descritos provavelmente foram sucedidos entre os anos
960-1020. Ainda que seja evidente que seu autor se baseou em tradições
orais, não parece menos certo que também teve ao seu alcance obras do
mesmo gênero, como a \emph{Eyrbyggja saga}, a \emph{Laxdæla saga} ou
a \emph{Ljósvetninga saga}, além de outras sagas de diferentes temáticas como a \emph{Gauks
saga Trandilssonar}. Os principais problemas em torno da \emph{Njáls
saga} giram precisamente em torno de sua historicidade, das fontes
utilizadas para a sua confecção, do que se supõe serem os 
componentes originais da saga, assim como da intenção do autor de elaborar
uma obra de caráter cristão.

A trama da saga aparece resumida com perfeição no verbete que Vésteinn
Ólason produziu há alguns anos para a enciclopédia \emph{Medieval
Scandinavia}, de Ph. Pulsiano, na qual se alega que a \emph{Njáls saga} é a
trágica história da amizade entre dois homens, Gunnar Hámundarson e
Njáll Þorgeirsson, cujos acontecimentos têm lugar na Islândia, nos países escandinavos,
na Irlanda e nas Ilhas Britânicas. Essa dispersão geográfica, com a
consequente introdução de novos personagens e situações, faz com que a
saga tenha sido por vezes considerada uma mera sucessão de pequenas
unidades temáticas, sem muita conexão entre elas. Não obstante, uma leitura mais
profunda atesta, sem dúvida, que os dois grandes temas da
obra, o destino e o elo mantenedor da honra (\emph{i.e.}, as disputas
entre as famílias -- esposas -- dos personagens principais) são o fio
condutor que nos leva ao clímax da saga, ao incêndio da granja de Njáll
e sua morte ao não querer abandoná-la.

O autor da \emph{Njáls saga} tem um interesse especial por apresentar os
personagens principais imersos em disputas legais no \emph{Alþingi}, a
Assembleia Geral, por uma variedade de razões. Por um lado, estaria o
intento descritivo -- não sempre isento de erros -- de um sistema judicial
imperfeito. Por outro, a de proporcionar à audiência uma caracterização
dos personagens, especialmente nas situações em que, da disputa, se
desprende uma diminuição ou um aumento da honra. Não
devemos duvidar, entretanto, que o texto seja também um veículo
privilegiado para o autor expressar suas opiniões -- críticas --
sobre a sociedade em que vivia, bem como para mostrar as diferentes
maneiras pelas quais se podia chegar a um acordo, seja por uma decisão na
Assembleia, seja por um acordo privado entre as partes ou mesmo pela
decisão de uma das partes em fazer justiça com suas próprias mãos. É 
nesse ponto, precisamente, que
devemos buscar a interpretação correta da saga em meio ao debate se a
\emph{Njáls saga} representa uma evidência da incapacidade das leis para
a manutenção da paz, ou se, pelo contrário, podemos extrair dela uma visão
mais positiva de uma sociedade de homens honestos que procuram evitar a
desintegração do corpo social. Seria o caso do protagonista, ou de Hall
Þorsteinsson, que, ao final da saga, renuncia a compensação que teria
direito por seu filho, que havia sido assassinado quando tentava separar
os contendores na batalha campal no \emph{Alþingi}.

Os sonhos proféticos e a ideia de destino são, junto às disputas legais,
os eixos em torno dos quais giram as vidas dos
personagens principais. Em alguns casos -- com os sonhos de Gunnarr e
Flosi, ou as predições de Njáll sobre o comportamento futuro de
sua esposa Bergthora e de outros --, tais eixos são utilizados pelo autor 
somente como uma
ferramenta narrativa para antecipar acontecimentos para a sua audiência.
No que se refere à aparente proeminência das ideias pré-cristãs do
destino e da predestinação, a opinião mais
compartilhada entre os críticos é a de que o autor, preocupado com sua satanização 
e com a condenação de seus antepassados, haveria
integrado o remanescente pré-cristão da religião popular presente na
saga (como as \emph{fylgjur}, ou personagens como o feiticeiro
Svanr) com a trama teológica cristã presidida pela ideia da
providência. Todavia, a intenção do autor de integrar a
herança pagã da ilha com o cristianismo militante de boa parte das
elites da época não foi impedimento para que,
na obra, pudessem ser reconhecidos nítidos ecos das sagradas escrituras,
como o sermão da montanha (ou do ``Evangelho segundo São Marcos''),
assim como da \emph{Niðrstigningar saga} e outros textos de caráter
homilético. Retido isso, deve-se ter clareza que a
\emph{Njáls saga} não deve ser comparada à literatura de caráter
alegórico, tão habitual entre os textos cristãos mais citados, dado que
nela não podemos encontrar as chaves para interpretar a mensagem ou
mensagens que seus autores intentavam transmitir.

\SIG{Teodoro Manrique Antón}

Ver também Linguagem; Literatura; Norreno; Poesia escáldica; Sagas
islandesas.

\begin{itemize}
\item \versal{HAMER}, Andrew. Njals saga and its Christian Background: A Study of
Narrative Method. \emph{Germania Latina \versal{VIII} (Mediaevalia Groningana New)},
Peeters, 2014.

\item  \versal{LÖNNROTH}, Lars. \emph{Njáls saga: A Critical Introduction}. Berkeley,
Los Ángeles, London: University of California Press, 1976.

\item  \versal{McTURK}, Rory. The supernatural in Njáls saga: a narratological approach.
In: \versal{HINES}, John; \versal{SLAY}, Desmond (eds.). \emph{Introductory Essays on
Egils saga and Njáls saga}. London: Viking Society for Northern
Research, 1991, pp. 102-124.

\item \versal{MILLER}, William Ian. \emph{Why Is Your Axe Bloody? A Reading of Njáls
Saga}. New York: Oxford University Press, 2014.

\item \versal{NORDAL}, Guðrún. The Dialogue between Audience and Text: The Variants in
Verse Citations in Njáls saga's Manuscripts. In: \versal{MUNDAL}, Else;
\versal{WELLENDORF}, Jonas (eds.). \emph{Oral Art Forms and their Passage into
Writing}. Copenhagen: Museum Tusculanum, 2008, pp. 185-202.
\end{itemize}
\section{\versal{NORMANDIA}}

A Normandia, terra de ``gente muito inquieta'' segundo o cronista
germânico Otto de Freising (c. 1114-1158), passou por importantes
mutações político-sociais ao longo dos séculos~\versal{IX}-\versal{XI}. Até o século~\versal{IX}, a
região era parte da Nêustria, porção ocidental do mundo franco, mas a
chegada dos nórdicos iria impactar, decisivamente, o curso de sua
história. Não por acaso, o termo ``Normandia'' significa ``terra dos
homens do norte'' (\emph{Northmannia}, \emph{Nordmannia}), pois a
expressão ``homens do norte'' (\emph{northmanni, nordmanni}) era a forma
pela qual os francos se referiam aos escandinavos.

A primeira aparição viking no litoral da Nêustria ocorreu em 820. Porém,
as incursões começaram apenas após 841, ano em que Ruão foi queimada.
Nos anos seguintes, a cidade novamente foi pilhada e os escandinavos
ocuparam o vale do baixo Sena. A partir da segunda metade do século~\versal{IX},
os assentamentos nórdicos se multiplicaram e os conflitos com os
carolíngios eram frequentes. Em versão muito difundida pelos manuais
historiográficos, tal contexto de animosidade teria feito com que o
nórdico Rollo e o rei franco Carlos, o Simples, assinassem o Tratado de
Saint-Clair-sur-Epte (911), pelo qual uma parte da Nêustria seria
entregue aos vikings (a futura ``Normandia''). Entretanto, os episódios
que cercam essa concessão permanecem um mistério, pois as informações
coetâneas são inexistentes. De acordo com Lesley Abrams, a dimensão
exata do poder dos vikings naquela região e período é obscura, o que
dificulta uma reconstituição do ``nascimento'' da Normandia.

O clérigo Dudo de Saint-Quentin (c. 965-1026) é a nossa única fonte de informação
sobre o Tratado, que não encontra confirmação fora da Normandia. Nas
condições do pacto, Rollo cessaria os ataques, se converteria ao
cristianismo e prestaria ``homenagem'' a Carlos; este, por sua vez,
ofereceria sua filha Gisela em casamento ao viking e cederia parte de
seu território. Na realidade, a concessão do monarca carolíngio
referia-se apenas à região já ocupada pelos vikings, ou seja, as terras
ao redor de Ruão (praticamente, a Alta Normandia atual). Algumas regiões
a oeste seriam anexadas em 924 e 933, quando Rollo e seus sucessores
absorveram outros vikings, muitos dos quais de origem dinamarquesa. A
procedência de Rollo, contudo, ainda é discutida pelos especialistas,
porque, embora Dudo mencione que ele era filho de um príncipe
dinamarquês, algumas sagas o consideram um norueguês.

Não sabemos também ao certo a natureza do poder que emergiu após 911, já
que o título ``duque'' (\emph{dux}), em referência ao governante normando,
se consolida na documentação apenas no início do século~\versal{XI}. Seja como
for, a fundação da Normandia impulsionou uma rápida fusão étnica e
cultural entre normandos e francos, que resultaria, entre outras coisas,
em gerações conduzidas ao cristianismo e às línguas \emph{d'oïl}. Como
nos informa Jean Renaud, o nórdico foi paulatinamente esquecido, mas não sem
antes deixar marcas que podem ser observadas no francês atual (p. ex.
\emph{homard} [``lagosta''], do nórdico \emph{humarr}) e na
toponímia normanda (por exemplo, \emph{Carquebut}, do nórdico \emph{kirkja}
[``igreja''] + \emph{býr} [``aldeia'']).

Para o estudo sobre a fixação dos vikings naquela porção da Nêustria, o
vocabulário, a toponímia e as informações das fontes escritas são
fundamentais, pois, conforme Else Roesdahl, até o momento foram achados
pouquíssimos vestígios arqueológicos da presença escandinava. De forma
significativa, os pesquisadores ainda não encontraram nenhuma sepultura
masculina, inscrição rúnica ou escultura viking. As joias e os objetos
de arte (como o ``martelo de Thor'' achado recentemente na Alta
Normandia) são raros e a maior parte das espadas descobertas estava em
rios, sugerindo que elas foram perdidas nas incursões. Esse contexto
parece indicar que muitos vikings já tinham abraçado o cristianismo e
esquecido o alfabeto rúnico, bem como as técnicas artísticas de seus ancestrais.
É provável, também, que a população local tenha rejeitado a linguagem e
a arte escandinava por considerá-las indecifrável e bárbara.

Após sua fundação, a Normandia continuou com a ampliação de seu poder
político. Ela chegou a ser atacada, em 944-945, pelo rei carolíngio 
Luís~\versal{IV} e por Hugo, o Grande. Na década de 960, foi atacada 
pelos condes de Anjou,
Flandres e Blois-Chartres, aliados do rei Lotário, mas Ricardo~\versal{I}
(942-996) conseguiu defendê-la com sucesso. As relações entre normandos
e francos ao longo do século~\versal{X} eram muitas vezes conflituosas, o que
pode ser percebido nas acusações que foram lançadas. Naquela época, o
monge Richer de Reims (c. 940-998) afirmou que Ricardo~\versal{I} havia sido um
``líder de piratas''. Mais tarde, Ademar de Chabannes (c. 989-1034)
conta que, enquanto estava no leito de morte, Rollo teria ordenado que
escravos cristãos fossem sacrificados em nome de Odin e Thor.

A exteriorização da religiosidade normanda preocupava os clérigos, que
se deparavam várias vezes com novos escandinavos estabelecidos na
região, muitos dos quais ainda pagãos. O fato, porém, é que essa
presença não interferiu seriamente no processo de cristianização da
Normandia, cujos duques já patrocinavam a reconstrução e ampliação da
rede monástica e episcopal. Roberto~\versal{I} (1027-1035), por exemplo,
reconciliou-se com a Igreja e devolveu as propriedades que havia
confiscado antes de partir numa peregrinação até Jerusalém (1035). Seu
sucessor, Guilherme~\versal{I} (1035-1087), chegou a participar do movimento
reformista, promovendo um concílio da Paz de Deus em Caen (1047).

Nesse processo de integração, certos costumes escandinavos podiam ser
encontrados na legislação normanda dos séculos~\versal{XI-XII}, como a pena do
exílio (``banimento''), sempre acompanhada pelo confisco dos bens do
condenado e com a função de preservar a ordem social. Já um exemplo da
fusão entre francos e normandos aparece na famosa Tapeçaria de Bayeux
(fim do século~\versal{XI}), na qual o exército de Guilherme~\versal{I} é chamado, por
duas vezes, de ``franco'' (\emph{franci}). Mesmo assim, os normandos
ainda não tinham perdido suas ligações com a \emph{Scandia}. Tanto é assim que
Ricardo~\versal{I} e Ricardo~\versal{II} (996-1026) tinham aliados escandinavos. Devemos
ainda lembrar que moedas normandas da primeira metade do século~\versal{XI} foram
encontradas na Escandinávia e Ruão tornou-se uma referência no mundo
nórdico: foi nela que Óláfr Haraldsson -- futuro monarca norueguês --
recebeu o batismo (1013).

Embora tenham pilhado muitas cidades durante suas investidas, os vikings
contribuíram para o desenvolvimento urbano da Normandia ao longo dos
séculos~\versal{IX-XI}. Naquele contexto, nasceram núcleos citadinos em Alençon,
Caen, Falaise, Cherburgo e Valognes, muitos deles favorecidos pela
atuação dos próprios duques normandos. O poderio político-econômico
acentuou-se após a conquista da Inglaterra anglo-saxônica (1066),
sobretudo pela transferência dos espólios para a Normandia. Tal contexto
de prosperidade pode ser observado no posto alfandegário perto de
Cherburgo, cuja receita entre 1049 e 1093 se multiplicou em quatorze vezes.
No entanto, a relação entre os territórios, como destaca Cassandra
Potts, era muito complexa e deve ser matizada: vários aristocratas
normandos não tinham participado da expedição, da mesma forma que outros
não se interessavam pela Inglaterra e não queriam ver as duas regiões
controladas pelo mesmo governante.

A anexação da Inglaterra não se mostrou suficiente para conter o ímpeto
aventureiro normando. Ao longo do século~\versal{XI}, cavaleiros zarparam do
ducado para combater no sul da Itália e na Sicília, onde alguns serviram
como mercenários. Conseguiram tomar o poder e, depois, fundaram o Reino
Normando da Sicília. Além disso, guerreiros normandos ``tomaram a Cruz''
para combater na Terra Santa. A Normandia foi uma das regiões que mais
alimentou os efetivos nas Cruzadas, o que, segundo Marc Bloch, estaria
relacionado à ``relativa paz'' desfrutada nesse principado
``notavelmente centralizado''. É preciso salientar, contudo, que as
conquistas normandas no Mediterrâneo e no Oriente tiveram poucas
consequências no ducado, exceto no sentido de potencializar o prestígio
e a reputação de seus cavaleiros.

Poder-se-ia encerrar esta breve síntese sobre o nascimento e
consolidação da Normandia, no ano de 1066, com a Batalha de Hastings, na
Inglaterra. Tal episódio é escolhido por muitos historiadores para
marcar o fim da Era Viking. O problema é que o evento não interrompeu o
\emph{continuum} histórico da Normandia; na verdade, ele apenas
estabeleceu uma ligação -- nem tão profunda, como vimos -- entre as duas
regiões até 1144, ano em que o ducado foi conquistado pelo conde
Geoffrey de Anjou. Outros historiadores, entre eles David Bates, afirmam
que a independência política e o poderio normando de colonização e
conquista prosseguiram efetivamente até 1204, quando o território foi
anexado pelo rei capetíngio Filipe Augusto.

\SIG{Guilherme Queiroz de Souza}

Ver também França na Era Viking; Rollo; Vikings na França.

\begin{itemize}
\item \versal{ABRAMS}, Lesley. Early Normandy. \emph{Anglo-Norman Studies}, vol. 35,
2013, pp. 45-64.

\item \versal{BATES}, David. The Rise and Fall of Normandy, c. 911-1204. In: \versal{BATES},
David; \versal{CURRY}, Anne (eds.). \emph{England and Normandy in the Middle
Ages}. London: Hambledon Press, 1994, pp. 19-36.

\item \versal{BLOCH}, Marc. \emph{A Sociedade Feudal}. Lisboa: Edições 70, 1979.

\item \versal{BREESE}, Lauren Wood. The Persistence of Scandinavian Connections in
Normandy in the Tenth and Early Eleventh Centuries. \emph{Viator}, vol.
8, 1977, pp. 47-62.

\item \versal{DUBY}, Georges. \emph{A Idade Média na França (987-1460). De Hugo Capeto
a Joana D'Arc}. Rio de Janeiro: Jorge Zahar Ed., 1992.

\item \versal{POTTS}, Cassandra. Normandy, 911-1144. In: \versal{HARPER-BILL}, Christopher; \versal{VAN HOUTS}, Elisabeth (eds.). \emph{A Companion to the Anglo-Norman World}.
Woodbridge: Boydell, 2003, pp. 19-42.

\item \versal{RENAUD}, Jean. The Duchy of Normandy. In: \versal{BRINK}, Stefan; \versal{PRICE}, Neil
(eds.). \emph{The Viking World}. London: Routledge, 2008, pp. 453-457.

\item \versal{RENAUD}, Jean. Les Vikings et la Normandie. In: \versal{BATTAIL}, Jean-François;
 \versal{BATTAIL}, Marianne. \emph{Une Amitié Millénaire. Les relations entre la
France et la Suède à travers les âges}. Paris: Beauchesne, 1993, pp.
49-68.

\item \versal{ROESDAHL}, Else. What may we expect? On the problem of Vikings and
archaeology in Normandy. In: \versal{FLAMBARD HÉRICHER}, Anne-Marie (ed.).
\emph{La progression des Vikings, des raids à la colonisation}. Ruão:
Publications de l'Université de Rouen, Cahiers du \versal{GRHIS}, 2003, n. 14,
pp. 207-214.
\end{itemize}
\section{\versal{NORRENO (NÓRDICO ANTIGO)}}

O norreno (\emph{norrœna}) é uma antiga língua indo-europeia pertencente
à família germano-nórdica. Em vários documentos nórdicos medievais
também aparece a denominação \emph{dǫnsk tunga} (língua danesa) para se
referir à antiga língua nórdica comum, utilizada em princípios da Era
Viking, na Dinamarca, no sul e no centro da Noruega e Suécia, incluindo
Bornholm, Gotlândia e Åland. O termo \emph{dǫnsk tunga} também parece
remeter a uma espécie de identidade pan-escandinava primogênita
(Anderson, 2000). Em alguns países de línguas neolatinas, como Espanha e
Portugal, é frequente o emprego do termo "nórdico antiguo'' e "nórdico
antigo'', respectivamente. Em francês, se usa habitualmente
\emph{norrois} e, no italiano, \emph{norreno}.

Devido à expansão viking a partir do século~\versal{VIII}, o norreno começou a
ser empregado também em locais como Islândia, Ilhas Feroé, Ilhas
Orcadas, Ilhas Shetland, Groenlândia, Normandia, zonas costeiras da
Finlândia e Estônia, assim como algumas partes da Rússia. Nas Ilhas Britânicas,
o norreno conviveu muito estreitamente com o inglês antigo. 
Chegaram até a produzir numerosos empréstimos linguísticos do nórdico antigo
ao inglês (\emph{window} $=$ \emph{vindauga}, \emph{guest} $=$ \emph{gestr},
\emph{thrall} $=$ \emph{þræll}, etc).

Apesar da sensação de unidade que o termo norreno pode oferecer, é certo
que já na época viking existia uma série de variedades dialetais que,
em termos gerais, podem classificar-se em norreno ocidental (norueguês e
islandês antigos), norreno oriental (danês e sueco antigos) e gotlandês
antigo, usado principalmente na ilha báltica de Gotlândia e,
provavelmente, apresenta similaridades com a antiga língua gótica.

De um ponto de vista gramatical, o norreno é uma língua altamente
flexível, que conta com quatro casos (nominativo, acusativo, dativo e
genitivo), três gêneros (masculino, feminino e neutro), dois números
(singular e plural) para substantivos, adjetivos e verbos e três
(singular, dual e plural) para pronomes pessoais. À parte os tempos
passado, presente e futuro (este último com ajuda normalmente de verbos
auxiliares), os verbos possuem, ademais, um modo indicativo e outro
subjuntivo, sofrendo numerosas mutações vocálicas, especialmente nos
verbos irregulares. Por sua vez, os adjetivos podem ser fortes ou
fracos. Em geral, a sintaxe norrena se caracteriza por possuir a
estrutura \versal{SVO} (Sujeito-Verbo-Objeto). A fonética norrena tem sofrido
numerosas mudanças no tempo (por exemplo, muitas vogais largas foram
convertidas em ditongos), porém, no geral, sempre tem conservado o acento
na primeira sílaba.

De um ponto de vista histórico, o norreno pode dividir-se em três
períodos principais:

- Período pré-viking (antes do século~\versal{VIII}): trata-se do denominado
proto-nórdico, quer dizer, uma evolução dialetal nórdica do
proto-germânico. Foram conservados testemunhos dessa primitiva língua em
algumas inscrições rúnicas pertencentes ao antigo \emph{fuþark}. Um
exemplo de proto-nórdico é a palavra \emph{gastiz}, que, no posterior
Período Viking, se converterá em \emph{gestr}, com a correspondente
mutação vocálica, e muda no marcador de caso nominativo singular
masculino.

- Período Viking (séculos~\versal{VIII-XI}): Desse período, para o qual se poderia
aplicar a denominação específica \emph{dǫnsk tunga}, procedem
numerosas inscrições rúnicas realizadas com o alfabeto mais simplificado
do \emph{fuþark} jovem. Também se têm conservadas palavras ou vestígios
da língua norrena do Período Viking em vários poemas éddicos e
escáldicos, em topônimos e antropônimos, em documentos latinos escritos
na Escandinávia e em empréstimos de outras línguas. É importante
assinalar aqui que a língua empregada nas posteriores sagas nórdicas não
são, a rigor, ``a língua dos vikings'', apesar do que se propaga em
numerosos artigos de divulgação popular.

- Período medieval ou pós-viking (séculos~\versal{XII-XIV}): Esse período se
caracteriza fundamentalmente pela adoção do alfabeto latino (que, não
obstante, conviveu com o rúnico durante vários séculos) e uma grande
produção de textos literários, históricos e mesmo científicos, que, em
muitos casos, denotam uma notável influência da tradição clássica e
cristã continental. Por outra parte, a distinção dialetal entre norreno
ocidental e oriental se tornou mais aguda. O islandês antigo se converte,
nesse período, na língua literária por excelência e na variante
linguística que mais logrou conservar diversos elementos sintáticos,
morfológicos e léxicos do norreno padrão frente a variados fenômenos
sociolinguísticos e de outra índole, como, por exemplo, a considerável
influência do baixo alemão (\emph{Niederdeutsch}) na Dinamarca, Noruega
e Suécia, graças à expansão da Liga Hanseática pela Escandinávia.

A partir dos séculos~\versal{XV} e \versal{XVI}, predominam as distintas línguas nacionais
frente ao norreno mais ou menos comum dos séculos passados. Além da
divisão entre línguas nórdicas ocidentais e orientais, se pode falar
também de línguas nórdicas continentais (sueco, danês, norueguês) e
insulares (islandês, feroês e o extinto \emph{norn} das islas Órcades e
Shetland). As línguas insulares (especialmente o islandês) são as que
mais possuem recursos gramaticais do antigo norreno das que foram conservadas até hoje.



\begin{table}[h]
\centering
\caption{Tabela comparativa das principais línguas nórdicas atuais a partir da
palavra norrena \emph{dauði} (morte):}
\label{my-label}
\begin{tabular}{lllllll}
Norreno        & Islandês (moderno) & Feroês         & Sueco        & Dinamarquês  & Norueguês (bokmål) & Norueguês (nynorsk) \\
\textit{dauði} & \textit{dauði}     & \textit{deyði} & \textit{Död} & \textit{Død} & \textit{død}       & \textit{daude}     
\end{tabular}
\end{table}

\SIG{Mariano González Campo}

Ver também Linguagem; Literatura; Poesia éddica; Poesia escáldica.

\begin{itemize}
\item \versal{ANDERSON}, Carl Edlund. The Danish Tongue and Scandinavian Identity.
\emph{Mid-American Medieval Association (\versal{MAMA}) Annual Conference}.
Tulsa: \versal{OK}, \versal{USA}, 2000.

\item \versal{BARNES}, Michael. \emph{A New Introduction to Old Norse \versal{I} (Grammar)}.
London: Viking Society for Northern Research, 2007.

\item \versal{FERNÁNDEZ ÁLVAREZ}, María Pilar. \emph{Antiguo islandés. Historia y
lengua}. Madrid: Ediciones Clásicas, 1999.

\item \versal{HAUGEN}, Einar. \emph{The Scandinavian Languages. An introduction to
their History}. Cambridge: Harvard University Press, 1976.

\item  \versal{HÆGSTAD}, Marius. \emph{Vestnordiske maalføre fyre 1350}. Oslo:
J. Dybwad, 1906-1942 (2 vols.).
\end{itemize}
\section{\versal{NORUEGA DA ERA VIKING}}

A Noruega é entrecortada por estreitos fiordes que se estendem por
vários quilômetros entre montanhas íngremes. Possui uma quantidade
elevada de ilhas próximas à costa, além de terreno montanhoso que sempre
dificultou a viagem por terra. Durante os séculos de ocupação, o mar era
o meio mais viável para a comunicação entre os povos que viviam nessa
região.

É também da convivência humana com as suas características naturais,
amplamente ligada aos espaços hídricos, que advém o seu nome,
``o caminho do Norte''. Estamos nos referindo a uma rota de navegação ao
longo da própria costa norueguesa, iniciada em Skagerrak ou Kattegat,
passando por Lindesnes e ao norte, onde as ilhotas em torno de Tromsø,
no extremo norte, podem ter representado algum tipo de fim ou limite
dessa rota, conforme indica uma das fontes mais antigas sobre a Noruega, o
relato de Öttar.

Foi anglicizado como Ohthere, termo anterior à inscrição rúnica
encontrada nas \emph{pedras rúnicas} de Jelling (segunda metade do século~\versal{X}),
nas quais, em uma de suas faces, podemos ver a inscrição
\emph{nuruiak}. O relato de tal anglicização pode ser encontrado em uma tradução para o
inglês antigo da \emph{História Contra os Pagãos}, de Paulo Orósio, cujo
os sete livros, encomendados pelo próprio rei Alfredo, recebeu um
acréscimo de tópicos geográficos sobre o norte da Europa, enriquecidos
com os relatos de diversos viajantes. Nela, podemos ler, além de Óttar, também
o relato de Wulfstan.

O termo \emph{Norðweg} é mencionado de acordo com a obra traduzida, na qual
consta que, em 890, Óttar foi recebido na corte do rei Alfredo, em Wessex,
onde descreveu suas empreitadas comerciais na Escandinávia. Disse ser o
mais ao norte dos homens do Norte, relatando, entre outras coisas, seu
contato com o povo Sámi (descrito em inglês antigo como \emph{Finnas}),
em Hálogaland, norte da Noruega. No contato com o podo Sámi, Ottar 
coletava diversos artigos na
forma de tributos (peles, penas, presas e cordas de morsas ou
focas) para, em seguida, vendê-los nos mercados de Hedeby e Kaupang. Pouco
é narrado sobre sua morada em Halogaland, mas as pesquisas arqueológicas
perto de Tromsø encontraram vários vestígios materiais, provavelmente de
assentamentos da Era Viking, que revelam intensa atividade em regiões
como Senja e Kvaløya.

Nesse momento, o contato com os povos germânicos da Escandinávia já
estava bem estabelecido e é certo que algum acordo existia para uma produção
contínua de tais itens. O uso do termo \emph{Finn} no corpo das
narrativas, incluindo amplas passagens do \emph{Heimskringla} (escritos
do século~\versal{XIII} sobre os reis noruegueses), se deve justamente ao caráter
dessas relações interculturais entre os povos germânicos da
Escandinávia, os finlandeses orientais, certas culturas Sámi e os Lapões
(como categoria econômica). Não obstante, o uso do termo também se deve à categorização das diversas
culturas, genericamente, como Finn (justamente pela permeabilidade
dessas culturas).

O poema elegíaco \emph{Haraldskvæði}, composto pelo poeta Thórbjǫrn
Hornklofi ao rei Haroldo Cabelos Belos (século~\versal{IX}), faz alusão ao termo
``Senhor dos Noruegueses'' (\emph{Dróttinn Nórðmanna}), indicando a disseminação do uso da
expressão. Contudo, apesar de as
sagas islandesas -- principalmente naquelas classificadas como Sagas dos
Reis (\emph{Konungasögur}) -- considerarem a empreitada do rei Haroldo Cabelos Belos
como um marco fundante do reino da Noruega, não há muitas evidências que corroborem para isso. 
Provavelmente, o que havia no início da Era Viking
eram pequenos territórios dominados por grupos étnicos diversos que
nomeavam as regiões habitadas.

Boa parte das informações escritas sobre a Noruega da Era Viking advém
de manuscritos do século~\versal{XIII} que narram as Sagas dos Reis,
\emph{Konungasögur}, biograficamente os monarcas noruegueses, bem como a
consolidação de seu reino. Na origem dessa monarquia, estariam os
antigos e lendários reis \emph{Ynglingos}, que habitaram a antiga
Uppsala. O território norueguês foi unificado por Haroldo
Cabelos Belos (Haraldr Hárfagri) e a sua linhagem permaneceu no poder
durante todo o medievo, com uma breve interrupção quando as chefias de
outro centro de poder situado no Norte, os \emph{jarlar} de Lade (região
próxima das atuais Trøndelag e Hålogaland), aproveitaram a discórdia entre os filhos e netos de
Cabelos-Belos. As investigações, entretanto, apontam para uma situação
bem diferente.

Segundo os poucos versos da poesia escáldica, contemporâneos do reinado
de Haroldo Cabelos Belos, a situação foi diferente. Este deve ter mantido
de fato o controle sobre territórios ao sudoeste e centro da Noruega, mantendo
alguma autoridade formal sobre as regiões ao longo da rota de
navegação que já havia sido mencionada por Óttar, na corte do rei
Alfredo. Os dois filhos de Cabelos Belos, Hákon Bondoso (Hákon Góði) e
Érico Machado Sangrento (Eiríkr Blóðøx), reinaram até pouco depois da
metade do século~\versal{X}, por volta de 960, mas logo a influência dinamarquesa
modificou o panorama político da Noruega.

É sintomática a pontuação nas sagas acerca da autoridade dos reis
dinamarqueses ao sul, que ocasionaram conflitos de interesses. Tal
influência durou até a primeira metade do século~\versal{XI}, quando o rei
Haroldo Severo (Haraldr Hárðraða) voltou a atuar politicamente na
região. Viken, no sul da Noruega, não foi parte dos domínios de
Cabelos Belos e podemos observar nos \emph{Anais Reais Francos} (\emph{Annales
regni Francorum}), o relato da ocupação do rei Godofredo e seus
descendentes na região. Os filhos de Érico assumiram as posses
territoriais de seu avô, sendo a autoridade depois entregue aos
\emph{jarlar} de Lade. O \emph{jarl} Hákon Sigurðsson se submeteu como
súdito do rei dinamarquês Haroldo Dente Azul (Haraldr Blátǫnn) até a sua
morte, provavelmente em 995, quando o seu filho, o \emph{jarl} Érico
Hákonarson (Eiríkr Hákonarson), assumiu o posto, vindo a cooperar com o
reinado dinamarquês até o fim de sua carreira, quando se tornou o
\emph{jarl} do rei Canuto, o Grande (Knútr, inn Ríki) no reino da
Nortúmbria.

Os reis dinamarqueses também providenciaram uma oportunidade única para
os noruegueses da Era Viking reunirem riquezas pessoais: a participação nas empreitadas guerreiras
contra Inglaterra.
Curiosamente, o desenrolar dos reinados de Olavo Tryggvason (\emph{Óláfr
Tryggvason}) e dos irmãos Olavo Haraldsson (Óláfr Haraldsson, também
conhecido como São Olavo) e Haroldo Severo estão ligados a tais
oportunidades. Eram homens que foram treinados no
exterior, angariavam tesouros e, no retorno à Noruega, podiam convencer seus futuros
seguidores a apoiar suas pretensões monárquicas. Entre os séculos~\versal{X} e
\versal{XI}, os reis escandinavos lideraram saques vikings, algo que os
predecessores dos séculos anteriores não haviam feito. Muitos líderes
vikings desse período parecem ter sido exilados, se contentando com o
que pudessem ganhar na Europa cristã ou na Rússia.

Olavo Tryggvason e Olavo Haraldsson, ambos reis missionários, foram
afortunados e se tornaram reis da Noruega após carreira de saqueadores
vikings no exterior. Olavo Tryggvason foi um líder guerreiro que desempenhou
papel fundamental na formação do reino da Noruega. Seus efêmeros
cinco anos à frente da conquista dos territórios noruegueses (além de sua
vida missionária) só foram possíveis graças a sua longa carreira viking, que
lhe proporcionou ganhos materiais suficientes para possibilitar a
expansão de seu reino a partir da região de Viken. Já Olavo Haraldsson 
fundou importantes centros religiosos
na Noruega, aproximando cada vez mais o seu reino dos padrões
das monarquias estabelecidas na Europa Central. Apesar da resistência
das chefias guerreiras que comandavam as costas orientais, dos conflitos
com os \emph{jarlar} de Lade e do crescente poder de Canuto que formava
um império norte-atlântico, o cristianismo acabou contribuindo com a
educação e burocratização necessárias para a modernização palaciana,
que, por sua vez, sustentou reinados mais pacíficos, como o de
Olavo Tranquilo (Óláfr Kyrre).

No fim da Era Viking, Haroldo Severo, que reinou entre 1046 e 1066 --
dividindo o trono com seu sobrinho, Magno, o Bondoso (Magnús Óláfsson,
ou Magnús inn Góði) --, iniciou sua carreira fugindo para o Oriente,
integrando a Guarda Varegue e se incumbindo de vigiar as rotas
marítimas bizantinas. Vindo a falecer na batalha da Ponte de Stamford,
esse rei clamava para si a autoridade de terras na Noruega, Dinamarca e
Inglaterra, mostrando que o poderio político do crescente reino
norueguês já ultrapassava suas fronteiras naturais e se expandia sobre a
Europa setentrional.

Apesar disso, nos séculos seguintes a Noruega atravessaria momentos
contraditórios. A presença cada vez maior da Igreja contribuiu para um
incremento no aparato burocrático da monarquia. Ao mesmo tempo, a Igreja,
enquanto instituição, passou a ser dona de largas porções de terras, se
tornando cada vez mais independente da aristocracia. Ademais, tal
arrojo não impediu que a realeza ainda se apresentasse com sérios
problemas institucionais. A falta de clareza na sucessão real resultou
em uma guerra civil entre os séculos~\versal{XII}~e~\versal{XIII}. Por outro lado,
ainda no século~\versal{XIII}, uma \emph{Pax Norvegica} floresceu mediante a
reestruturação da \emph{Hirð} e dos \emph{Ármenn} (outrora guardas
pessoais do rei, que passaram a ocupar cargos administrativos), o avanço
na anexação dos territórios ao norte e à aproximação formal de várias
colônias através de acordos jurídicos e comerciais.

\SIG{Pablo Gomes de Miranda}

Ver também Era Viking; Oseberg; Viking.

\begin{itemize}
\item \versal{BAGGE}, Sverre. \emph{From Viking Stronghold to Christian Kingdom: state
formation in Norway, c. 900-1350}. Copenhagen: Museum Tusculanum Press,
2010.

\item \versal{HELLE}, Knut (org.). The Norwegian Kingdom: succession disputes and
consolidation. In: \versal{HELLE}, Knut (org.). \emph{The Cambridge History of
Scandinavia}. Cambridge: Cambridge
University Press, 2006, pp. 369-391 (vol. 1 -- Prehistory to 1520).

\item \versal{KRAG}, Claus. The Early Unification of Norway. In: \versal{HELLE}, Knut (org.).
\emph{The Cambridge History of Scandinavia}. Cambridge: Cambridge University Press, 2006, pp. 184-201 (vol. 1 -- Prehistory to 1520).

\item \versal{KRAG}, Claus. The Creation of Norway. In: \versal{BRINK}, Stefan; \versal{PRICE}, Neil
(eds.). \emph{The Viking World}. New York: Routledge, 2008, pp. 645-651.
\end{itemize}
\section{\versal{NOVGOROD}}

A cidade e futuro Principado de Novgorod, também conhecida pela
historiografia como \emph{Velikii Novgorod} (Grande Novgorod) e
\emph{Gospodin Velikii Novgorod} (Senhor Novgorod, o Grande; nomenclatura
mais tardia), foi uma das principais cidades fundadas pelos vikings na
Rússia Europeia, juntamente com Kiev. Teve grande importância comercial e
política desde sua fundação. Em fontes escandinavas, o termo
\emph{Garđariki} é utilizado para referir-se à cidade, embora o mesmo termo possa se
referir igualmente a Kiev. \emph{Holmgardr}, significando ilha ou
península de acordo com Wladislaw Duczko, também poderia se tratar de
Novgorod, mas não se sabe os limites do uso desse termo. Atualmente, está
localizada no Oblasto de Novgorod, no noroeste da Rússia.

A geografia de Novgorod é bastante diferente quando comparada a Kiev. Novgorod está
situada nas zonas florestais do norte da Rússia Europeia, em região
repleta de pântanos, porém não muito fértil. O rio Volkhov e seus
tributários Volkovets e Zhilotug cercavam a área e desaguava no lago
Ílmen. Inicialmente, algumas tribos fino-úgricas habitavam a região e,
com o tempo, alguns povos eslavos se assentaram no local. Um dos
primeiros assentamentos varegues em massa na região foi em Riurikovo
Gorodische (Castro de Riurik), no século~\versal{X}, uma região praticamente
``ilhada'' localizada entre os tributários do Volkhov. Nas fontes
escandinavas, o território é chamado de \emph{Holmgardr} e, no mesmo
século do assentamento, tornou-se a cidade de Novgorod. Vestígios
arqueológicos apontam para a presença nórdica no local desde o século~\versal{IX},
conforme sustenta Valentin Ianin, com achados de moedas árabes e bizantinas que
possivelmente foram transportadas para região por escandinavos.

De acordo com Duczko, os achados arqueológicos de Riurikovo Gorodische
condizem com o assentamento de nórdicos da região da Suécia, pois alguns
itens encontrados não aparecem no resto de Rus, mas são comuns na
Escandinávia, sobretudo em Birka na Suécia. Entre eles, encontram-se um
pingente em formato de uma cabeça de dragão e outro em formato de uma
mulher com um vestido longo. É possível que Riurikovo Gorodische tenha
sido uma espécie de ``primeira capital dos varegues'' que se assentaram em
Rus, conforme apontam alguns historiadores, sobretudo aqueles afiliados ao marxismo.
Mas autores como Jonathan Shepard e Wladislaw Duczko discordam, devido à falta de evidências de assentamento nórdico na região
antes da segunda metade do século~\versal{IX}. Ainda assim, algumas fontes
parecem mostrar que havia uma entidade política autônoma em Novgorod
antes do assentamento dos varegues em Kiev. É o caso de uma fonte árabe do
início do século~\versal{X}, a qual revela que os Rus viviam em uma terra ilhada e que
seu entidade política era chamada de \emph{khagan}. Para alguns monarcas escandinavos,
especialmente os noruegueses, Novgorod serviu de refúgio, como no caso dos reis noruegueses Olavo Tryggvason (995-1000) e Haroldo
Hardrada (1046-1066). Vale mencionar, uma igreja foi construída em
homenagem a Santo Olavo Haraldsson (1015-1028), em Novgorod, pouco após a
sua morte.

Novgorod estava ligada diretamente tanto com a rota do Mar Negro (a
chamada Rota dos Varegues aos Gregos) quanto com a rota do Volga de
comércio entre escandinavos e árabes. O comércio de peles foi
fundamental para a consolidação dos laços comerciais e do poder do
principado. Segundo Janet Martin, Novgorod era uma das maiores cidades
produtoras de peles. Estas eram exportadas para diferentes locais como
Constantinopla, Suécia e o Califado Abássida, estando presente na
maioria dos mercados europeus. Martin também afirma que Novgorod atuava
como um local de estocagem de mercadorias vindas de outros territórios.
Os escandinavos que lá aportavam poderiam obter produtos de luxo tanto
bizantinos e árabes quanto frísios e bretões. O comércio de peles em
Novgorod sobreviveu até o século~\versal{XII}, comercializando, durante o período, quase
exclusivamente com os escandinavos. Entrou em decadência com o
fortalecimento de Suzdália e sua entrada no mercado.

Não há fontes contemporâneas de Rus sobre a história de Novgorod e
fontes estrangeiras mencionam-na superficialmente. Muito do que se
assume sobre Novgorod vêm de fontes muito posteriores, como uma edição
posterior da \emph{Primeira Crônica de Novgorod} (\emph{Novgoródskaia
Piérvaia Liétopis}), que menciona o passado da cidade, mas cuja edição
principal é datada do século~\versal{XIII} e não contém entradas anteriores a
1016. As fontes principais são legadas pela arqueologia e pelos poucos relatos
presentes na \emph{Crônica dos Anos Passados}. A fonte ainda atesta que,
após concordar em ser o governante das tribos eslavas, o lendário Riurik
estabeleceu-se em Novgorod após permanecer por um tempo em Staraia Ladoga.
Após a morte de Riurik, Oleg, o Profeta (882-912), assumiu o controle de
Novgorod, mas eventualmente foi para o sul, com a provável intenção de
centralizar o poder em Kiev.

É possível que exercer a função de príncipe de Novgorod tenha sido um
pré-requisito para eventualmente se tornar príncipe de Kiev, ao menos
antes da reforma do sistema de sucessão por Iaroslav Vladimirovich, o
Sábio (1016-1018, 1019-1054). Oleg, Vladimir~\versal{I} Sviatoslavich,
de Kiev (980-1015), e o próprio Iaroslav foram príncipes de Novgorod
antes de assumirem o trono kievano. Embora fosse estabelecido que os
príncipes da dinastia Riuríkida devessem assumir o governo de Novgorod
com a reforma de Iaroslav, existiram diversas instâncias nas quais a
aristocracia e o \emph{veche} (assembleia popular) expulsaram o príncipe
destinado. A partir dos séculos~\versal{XI} e \versal{XII}, Novgorod começou a ganhar uma
maior autonomia política e, eventualmente, se tornou o que,
na opinião de muitos historiadores (incluindo Ianin), se poderia chamar de ``república''
(considerando o
funcionamento de suas instituições políticas e o fato de o poder das assembleias
de boiardos e populares serem mais poderosas que o poder do príncipe). Todavia, há
de se ter cautela com a utilização de tal termo em sua acepção moderna. 
É também notável que Novgorod não foi afetada pelos ataques
mongóis do século~\versal{XIII}.

\SIG{Leandro César Santana Neves}

Ver também: Crônica dos Anos Passados; Kiev; Mikligardr; Rus; Rússia da
Era Viking; Staraia Ladoga; Varegues; Vladimir~\versal{I} de Kiev.

\begin{itemize}
\item \versal{BIRNBAUM}, Henrik. \emph{Lord Novgorod the Great: Essays in the History
and Culture of a Medieval City-State}. Columbus: Slavica Publishers,
1981.

\item \versal{DUCZKO}, Wladsyslaw. \emph{Viking Rus: studies on the presence of
Scandinavians in Eastern Europe}. Leiden: Koninklijke Brill \versal{NV}, 2004.

\item \versal{IANIN}, Valentin L. Medieval Novgorod. In: \versal{PERRIE}, Maureen (org.).
\emph{The Cambridge History of Russia}. Cambridge: Cambridge University Press, 2006, pp. 188-210 (vol. 1: From Early Rus' to 1689).

\item \versal{FRANKLIN}, Simon; \versal{SHEPARD}, Jonathan. \emph{The Emergence of Rus
750-1200}, Essex: Longman, 1996.

\item \versal{MARTIN}, Janet. \emph{Treasures of the Land of Darkness: The Fur Trade
and its Significance to Medieval Russia}. Cambridge: Cambridge
University Press, 1986.
\end{itemize}
\chapter{O \textarn{o} \textarc{o} \textart{o}}
\section{\versal{OLAVO HARALDSSON}}

Rei na Noruega entre 1015 e 1028, é uma figura fundacional que
contribuiu de forma marcante para a independência e identidade do país,
tornando-se o santo padroeiro deste e, desse modo, um dos maiores símbolos
medievais da autonomização norueguesa face às pretensões dinamarqueses.
Foi aliás um mártir dessa causa, tendo dado a vida por ela na batalha de
Stiklestad, em 1030, quando tentou retomar o trono dois anos após ter
sido expulso do país.

A tradição lendária faz dele descendente de Haroldo Cabelos Belos, primeiro
rei de uma Noruega unida, o que o colocava como possível candidato ao trono
numa época em que sucessão era um jogo aberto e a primogenitura
estava longe de ser a regra. Mas, à semelhança do seu homônimo e
antecessor, Olavo Tryggvason, a juventude do futuro monarca foi passada
em incursões vikings, tendo supostamente lutado e pilhado nas ilhas
britânicas antes de se virar para sul e passar um inverno na Normandia.
Terá sido por essa altura que, segundo a tradição, atacou a costa ibérica e
chegou perto do estreito de Gibraltar, de onde queria partir até
Terra Santa, mas um sonho disse-lhe para voltar para trás e regressar à
Noruega, onde seria rei para sempre. Uma lenda tardia e nada mais, mas
que foi bastante útil nos anos que se seguiram à batalha de Stiklestad.

Na Normandia, diz a tradição, foi batizado em torno de 1013. A
decisão não teria sido extemporânea nem movida por uma fé pessoal, mas
teria também sido fruto de anos de contato direto com a nova religião, bem como da
observação do papel desta na administração dos reinos europeus. Tal postura não
era rara e tampouco foi excepcional na vida de Olavo, marcada pela
acumulação de saques e seguidores, que depois foram usados na investida pela conquista do trono
norueguês. A subida ao poder de Olavo Haraldsson, em 1015, permitiu-lhe
assegurar a cristianização do país, fazendo-se acompanhar nesse processo
por um clérigo de nome Grímkell, talvez de origem ou formação inglesa. Tal clérigo
foi também bispo real e conselheiro pessoal do monarca. No fundo, era uma
transposição para a Noruega de um modelo político e administrativo da
Europa cristã.

Se de início Olavo foi popular por afirmar a independência do reino --
tendo aproveitado a morte inesperada de Svein, da Dinamarca, para
subtrair o território norueguês à esfera dinamarquesa --, o tempo e o
estilo de seu governo autoritário tornaram-no impopular, o que acabou culminando na sua expulsão
da Noruega em 1028, quando Canuto, filho de Svein, enviou um
exército para repor a autoridade da sua família. Olavo fugiu e
refugiou-se no que é hoje a Rússia, tentando um regresso dois anos mais
tarde, tentativa esta que falhou e o levou à morte. Mas, porque o
martírio enobrece a memória de uma pessoa (mais ainda quando ela é
sucedida por governantes ainda mais impopulares e conta com amigos
influentes dispostos a construir uma narrativa heroica), o culto a Olavo
começou a ganhar forma apenas um ano depois do seu falecimento.

Em 1031, o seu corpo foi transladado para a igreja de S. Clemente, em
Trondheim, e a partir daí a sua popularidade não parou de crescer,
sendo ainda ajudada por restos mortais bem preservados -- como convém a um santo --,
uma lista crescente de milagres e o patrocínio do bispo Grímkell, bem como de
vários poetas, alguns destes antigos companheiros de Olavo, que pintaram a
memória do monarca em tons hagiográficos. Terá sido nessa altura que
surgiu a lenda que contava como ele navegou até ao estreito de Gibraltar
e quis ir até Jerusalém, sinal de sincera fé cristã, mas, conforme mencionado, teria recebido
um sonho que lhe prometia o trono norueguês em perpetuidade. Era
uma narrativa fictícia, mas conveniente, sobretudo porque deslegitimava o controle
dinamarquês do país, o qual terminaria em 1035, quando Magnus, filho de
Olavo, assumiu o trono norueguês e foi co-monarca, juntamente com o seu tio Haroldo
Hardrada.

Ambos os homens tinham a seu favor não apenas a legitimidade de sangue,
mas também o argumento religioso de estarem ligados ao santo padroeiro e
rei perpétuo da Noruega, o que conferia-lhes uma aura de poder e autoridade.
A mesma aura que, mais de cem anos depois, em 1163, Magnus~\versal{V} tentou manter,
declarando o território norueguês como feudo de São Olavo e jurando
administrá-lo enquanto representante e vassalo do santo. O país
confundia-se, assim, com uma figura fundadora a quem foi dedicado um culto
que chegou a cruzar as fronteiras.

\SIG{Hélio Pires}

Ver também Era Viking; Noruega da Era Viking; Viking.

\begin{itemize}
\item \versal{BRINK}, Stefan. Christianization and the early Church. In: \versal{BRINK}, Stefan;
 \versal{PRICE}, Neil (eds.). \emph{The Viking World}. London/New York: Routledge,
2010, pp. 621-628.

\item \versal{CHRISTIANSEN}, Eric. \emph{The Norsemen in the Viking Age}. Oxford:
Blackwell, 2006.

\item \versal{KRAUG}, Claus. The creation of Norway. In: \versal{BRINK}, Stefan; \versal{PRICE}, Neil
(eds.). \emph{The Viking World}. London/New York: Routledge, 2010, pp.
645-651.

\item \versal{PIRES}, Hélio. Nem Tui, nem Gibraltar: Oláfr Haraldsson e a Península
Ibérica. \emph{En la España Medieval} 38, 2015, pp. 313-328.
\end{itemize}
\section{\versal{OLAVO TRYGGVASON}}

Olavo Tryggvason (968-1000) foi o rei da Noruega, de 995 até 1000, quando
marcou seu nome na história pelo seu papel fundamental na expansão
do cristianismo pelo norte europeu. As informações sobre o seu reinado
podem ser encontradas em diversas documentações, sendo a mais antiga a
\emph{Gesta Hammaburgensis ecclesiae pontificum}, escrita pelo Cronista
alemão Adão de Bremen, em 1076. Existem algumas sagas que falam sobre a
vida do rei Olavo, escritas por Oddr Snorrason e Gunlaugr Leifsson,
respectivamente: a \emph{Óláfs saga Tryggvasonar} e a \emph{Óláfs saga
Tryggvasonar}. A terceira saga é a \emph{Heimskringla}, de Snorri
Sturlusson, que usou a saga de Oddr Snorrason como principal fonte de
inspiração. Há também a \emph{Ólafs saga Tryggvasonar em mesta}, que
se inspirou no material produzido por Oddr Snorrason e Gunlaudr
Leifsson.

A infância de Olavo Tryggvason é um elemento de debate, pois diferentes
documentações divergem a esse respeito. A \emph{Historia Norwegiae} diz que
ele nasceu no arquipélago das Órcades, ao norte da Escócia, enquanto a
\emph{Àgrip af Nóregskonungasögum} narra a versão que Olavo, com três anos
de idade, fugiu juntamente com sua mãe para as Órcades. A versão com mais
informações sobre sua infância está presente na \emph{Heimskringla}, que
possui informações ausentes nas outras documentações.

A sua juventude foi passada na Estónia e na corte de Vladimir~\versal{I}, na
atual Rússia. Ainda jovem, iniciaria sua carreira como
viking e seus saques à Inglaterra são registrados nas \emph{Anglo-Saxon
Chronicle}, em 991. Ele lutou na batalha de Maldon e, posteriormente, foi
pago em uma boa quantia de \emph{danegeld}. Três anos após a batalha de
Maldon, em 994, Olavo se converteria ao cristianismo, sendo batizado pelo
Santo Alfege da Cantuária.

Um ano após a sua conversão, Olavo retorna a Noruega e toma o poder de
Haakon Sigurdsson. Haaakon era o \emph{jarl} de Late e a liderança efetiva na
Noruega, apesar de ser um vassalo do rei dinamarquês, Haroldo
Dente Azul. Ambos, Haroldo Dente Azul e Haakon Sigurdsson, realizaram
um complô para retirar do trono o rei anterior, Haroldo Capa Cinzenta.
Olavo Tryggvason aproveitou a impopularidade do rei Haakon perante os
cristãos e os senhores locais para tomar o poder por meio de uma rebelião.

No poder real, Olavo Tryggvason transferiu o centro do poder para
Trondheim e iniciou o processo de conversão do restante da Noruega,
assim como das Ilhas Faroé e Islândia. Para aumentar seu poder, Olavo
pediu Sigrid Storrada em casamento, com a condição dela se converter, mas
seu pedido foi rejeitado, haja vista que ela não desejava abandonar a sua fé na
religião nórdica. Após a rejeição e conflito com Sigrid, Olavo se casou
com Thyre, irmã de Sueno Barba Bifurcada. Barba Bifurcada, posteriormente,
se casaria com Sigrid, exponenciando uma inimizade entre os dois monarcas.
Essa inimizade levaria ao conflito em Svöld, onde Olavo morreria
confrontando os exércitos suecos e dinamarqueses.

\SIG{André Araújo de Oliveira}

Ver também: Thing; Godi; Islândia na Era Viking; Realeza; Viking.

\begin{itemize}
\item \versal{HOLMAN}, Katherine. \emph{Histocial Dictionaries of the Vikings}. Oxford:
The Scarecrow Press Inc., 2003.

\item \versal{LINDKVIST}, Thomas. Early political organisation, Introductory survey.
In: \versal{HELLE}, Knut (org.). \emph{The Cambridge History of Scandinavia}.
Cambridge: University of Cambridge Press, 2003, pp. 160-167 (vol. 1).

\item \versal{SIGURÐSSON}, Jón Viðar. Iceland. In: \versal{BRINK}, Stefan; \versal{PRICE}, Neil (eds.).
\emph{The Viking World}. New York. Routledge, 2008, pp. 571-578.

\item  \versal{VÉISTEINSSON}, Orri. \emph{The Christianization of Iceland: Priest,
Power and social change 1000-1300}. Oxford: Oxford University Press,
2000.
\end{itemize}
\section{\versal{OLGA DE KIEV}}

Princesa Olga de Kiev governou Rus de Kiev como regente entre 945 e 964,
após a morte de seu marido Igor e enquanto seu filho, Sviatoslav Igorevich
(964-972), era menor de idade. Ela era avó do príncipe Vladimir
Sviatoslavich~\versal{I}, de Kiev, e, conforme a tradição, sua criada Malucha criou
o príncipe. Não se sabe exatamente a data do nascimento de Olga, já que
sua primeira menção na \emph{Crônica dos Anos Passados} está na entrada
do ano de 903. O \emph{Livro da Genealogia Real}, fonte escrita no
século~\versal{XIV}, afirma que Olga teria nascido em 890, hipótese hoje
contestada por diversos autores, como Constantin Zuckerman. Este argumenta
que uma idade avançada de Olga prejudicaria alguns feitos
posteriores de sua vida, como a dupla viagem à Constantinopla nas décadas
de 940 e 950. Olga teria cerca de 60-70 anos de idade nesses eventos e
seria improvável que uma idosa tenha feito a difícil jornada de Kiev
para Constantinopla, sobretudo se considerados os constantes ataques de tribos eslavas das
estepes russas no caminho entre ambas as cidades.

Ainda sobre a entrada de 903, esta diz que Olga seria natural de Pskov,
uma cidade atualmente localizada no noroeste da Rússia. É muito mais
provável que Olga fosse escandinava ou descendente de escandinavos, já
que seu nome possui uma óbvia semelhança com o nome nórdico
\emph{Helga}, ou mesmo com o nome \emph{Allogia}, presente na \emph{Saga de Olaf
Tryggvason}, que alguns autores associam à regente. Uma teoria sobre o
passado de Olga que corrobora com sua origem escandinava é de Roman
Kovalev. Este afirma que Olga seria uma sacerdotisa da deusa
nórdica Freyja antes de ser batizada, fundamentando sua argumentação 
a partir dos símbolos de realeza
encontrados próximos ao possível túmulo de Olga. Tais símbolos -- uma chave e um falcão --
possuem relação direta com a deusa
nórdica. Essa representação estaria presente de maneira implícita na
\emph{Crônica} e em fontes posteriores sobre Olga.

O governo de Olga enquanto regente começou em 944 ou 945, segunda entrada
de acordo com a \emph{Crônica}. Ainda conforme esta
fonte, Igor, marido de Olga, foi persuadido por seu séquito a cobrar
mais tributos dos derevlianos, uma tribo eslava localizada ao oeste do
rio Dniepre, na atual Ucrânia, pois o séquito do varegue Sveneld tinha
mais regalias. Os derevlianos reagiram à uma nova coleta e assassinaram
o príncipe. Conforme fonte bizantina, Igor fora amarrado junto a
troncos e tivera seu corpo partido ao meio. Alguns enviados dos
derevlianos, então, foram até Olga e pediram para que ela se casasse com
o príncipe deles. Olga aceitou, desde que viessem no dia seguinte e
ordenassem aos kievanos para que fossem carregados em um barco em
direção ao seu castelo. Os enviados
assim o fizeram, mas foram enterrados vivos pelos kievanos a mando da
própria Olga. Este foi o início de uma das narrativas mais curiosas
presentes na \emph{Crônica}: o conto da ``vingança de Olga''.

A vingança continuou após Olga insistir para os derevlianos enviarem os
seus homens mais importantes para a discussão do casamento proposto
anteriormente. A regente recebeu os novos enviados e preparou-lhes um
banho nas termas de sua residência. Quando os enviados entraram no
banho, Olga ordenou que a porta fosse trancada e os derevlianos foram
queimados vivos. A fase seguinte da vingança foi a aniquilação das forças
militares derevlianas, enquanto Olga visitava a capital Iskorosten para
organizar um funeral ao seu marido Igor, morto no local. Houve uma
celebração após o funeral, na qual Olga ordenou que os derevlianos
bebessem uma grande quantidade de hidromel, enquanto os militares de Rus
permaneciam sóbrios. Aproveitando que os guerreiros inimigos estavam
embriagados, Olga comandou o assassinato dos derevlianos.

A última parte da vingança consistiu na destruição de Iskorosten, feita de uma
maneira bastante peculiar. Ao questionar os derevlianos sobre o motivo
de ainda resistirem à tributação dos Rus e de ainda protegerem a sua
capital após um longo sítio, Olga prometeu-lhes que os deixaria em paz
se pagassem um único tributo: três pardais e três
pombas. Os derevlianos aceitaram, mas Olga ainda não tinha terminado sua
vingança. Ela ordenou que em cada pássaro fosse amarrado um pedaço de
pano e enxofre. Os pássaros voltaram para
Iskorosten, causando um incêndio na cidade. Olga escravizou os derevlianos e 
impôs-lhes um tributo maior do que o
proposto pelo seu marido. Esse tipo de sítio de cidade, com a prática do envio 
de pássaros para causar incêndio, é recorrente na
chamada ``literatura épica medieval''. Ocorre, por exemplo, na saga do
rei norueguês Haroldo Hardrada (1046-1066), quando o monarca fez uso de
pássaros com cera e enxofre para destruir uma cidade na Sicília.

É possível que, dado o aparente exagero das punições descritas na
\emph{Crônica}, os atos da vingança não tenham ocorrido ou, se realmente
aconteceram, tiveram alguma ligação com a morte de Igor. 
Aleksandr Koptev, por exemplo, interpreta os eventos como etapas de funerais típicos
dos nobres de Rus, tais como os descritos por Ibn Fadlan. A vingança seria, para o autor,
a representação do funeral de Igor. Mas Koptev não deixa
claro se acredita ou não na veracidade do ato. Um autor que acredita
no massacre de Olga contra os derevlianos é Yuriy Dyba, embora este
argumente que as ações de Olga não foram necessariamente vingativas, mas uma medida
necessária para o controle de rotas comerciais com o oeste, assim como para o
estabelecimento de postos comerciais e terras de caça, conforme descrito nas
entradas de 947. Seja como for, a campanha de Olga contra os Derevlianos
deve ter sido guardada em lugar especial na memória dos habitantes de
Rus, pois a narrativa foi eventualmente preservada na \emph{Crônica}, o
que, conforme Simon Franklin, mostra a aceitação de tal comportamento
feminino -- claramente herança do passado nórdico -- por uma Rus
pós-cristianização.

Olga também é famosa por ser a primeira governante de Rus cuja conversão
ao cristianismo de rito grego é comprovada por fontes. Embora o
cristianismo já fosse praticado entre os varegues de Rus desde o século~\versal{IX} --
conforme o Patriarca de Constantinopla Fócio -- e ainda que haja teorias
sobre Igor ser ``secretamente'' um cristão, foi Olga a primeira que
assumiu a religião do Império Bizantino. O batismo de Olga é um dos
tópicos mais debatidos sobre a regente e mesmo sendo a data de 957
a mais aceita pela historiografia, existem várias hipóteses indicando que o
evento pode ter ocorrido entre 944 e 961. De acordo com a \emph{Crônica dos Anos Passados},
imediatamente após ser
batizada, Olga demonstrou astúcia ao recusar um convite do Imperador Constantino~\versal{VII}
Porfirogênito para ser Imperatriz e ``governar Constantinopla ao seu
lado'', alegando que, após o batismo, ela era filha do Imperador e um 
possível casamento seria incesto. Assim como a história de vingança
presente na mesma fonte, não há provas se esse diálogo realmente
ocorreu.

Olga não conseguiu cristianizar Rus, tarefa que coube a seu neto
Vladimir Sviatoslavich, em 988. Devido à falta de fontes que atestem 
uma tentativa de cristianização, é possível que Olga mal tenha tentado
introduzir o cristianismo de rito Grego em Rus, mesmo que hagiografias
posteriores afirmem que ela patrocinou a construção de diversas Igrejas
em Kiev. Seu pedido de bispos para o rei alemão Oto~\versal{I}, em 959, corrobora
com a apatia de Olga com relação à propagação do cristianismo na visão de alguns
autores. Na \emph{Crônica dos Anos Passados}, o único ato de propagação
de sua nova fé estaria ligado às sucessivas tentativas de conversão de seu
filho Sviatoslav, que permaneceu pagão pois seu séquito ``riria'' dele.
Alguns autores, sobretudo soviéticos marxistas, interpretam essa
passagem como uma ``reação pagã'' e pressão por parte da elite militar de
Rus. Olga faleceu em 969 e foi oficialmente canonizada
no século~\versal{XVI}. Hoje ela é comemorada como santa ``igual aos apóstolos''
pela Igreja Ortodoxa, no dia 11 de julho.

\SIG{Leandro César Santana Neves}

Ver também: Crônica dos Anos Passados; Mikligardr; Kiev; Rus; Rússia da
Era Viking; Varegues; Vladimir de Kiev.

\begin{itemize}
\item \versal{BUTLER}, Francis. A Woman of Words: Pagan Ol'ga in the Mirror of Germanic
Europe. \emph{Slavic Review}, vol. 63, n. 4, 2004, pp. 771-793.

\item \versal{DYBA}, Yuriy R. Administrative and Urban Reforms by Princess Olga:
Geography, Historical and Economic Background. \emph{Latvijas arhīvi /
Latvijas Nacionālais arhīvs}. Galv. red. \versal{V}. Pētersone, n. 1-2, 2013, pp.
30-71.

\item \versal{FRANKLIN}, Simon; \versal{SHEPARD}, Jonathan. \emph{The Emergence of Rus
750-1200}. Essex: Longman, 1996.

\item \versal{KARPOV}, Aleksey Iu. \emph{Kniaguinia Olga} [\emph{Princesa Olga}].
Moscou: Molodaia Gvardia, 2012.

\item \versal{KOVALEV}, Roman K. Grand Princess Olga of Rus' Shows the Bird: Her
`Christian Falcon' Emblem. \emph{Russian History}, vol. 39, 2012, pp.
460-517.

\item \versal{KOPTEV}, Aleksandr. Ritual and History: Pagan Rites in the Story of the
Princess' Revenge (the Russian Primary Chronicle, under 945-946).
\emph{\versal{MIRATOR}}, vol. 11, n. 1, 2010, pp. 01-54.
\end{itemize}
\section{\versal{OSEBERG}}

Em agosto de 1903, o arqueólogo norueguês Gabriel Gustafsson recebeu a
visita de Oskar Rom, dono de uma fazenda chamada Revehaugen, em Oseberg,
na região de Vestfold, atual Noruega. Rom trazia consigo uma peça de
madeira talhada e decorada que havia encontrado enquanto retirava uma
montanha de seu terreno para alargar seus campos. Gustafsson, no
primeiro momento, se mostrou cético com a visita de Rom, mas na hora em
que o fazendeiro lhe mostrou o artefato que havia recolhido em seus
campos, o arqueólogo imediatamente identificou a peça de madeira como
pertencente ao padrão de representação de figuras zoomórficas do Período
Viking. Dois dias depois, o arqueólogo visitou a fazenda e escavou uma
pequena trincheira provisória que o persuadiu da importância e do
tamanho da descoberta. Mas, naquela ocasião, tal trincheira não poderia ser
executada devido tanto ao inverno que se aproximava quanto às preparações
financeiras que ainda deveriam ser realizadas para uma escavação daquele
porte. Gustafsson, então, fechou a trincheira inicial para proteger a
descoberta da neve e do frio e começou a
preparação de sua equipe. Também começou a levantar as finanças necessárias 
para a realização da escavação.

A escavação da embarcação de Oseberg teria, assim, seu início no verão de
1904, com os trabalhos realizados pelo professor Gabriel Gustafsson. As
primeiras indicações tratariam as dimensões do monte funerário que foi
apontado como tendo 40~m de diâmetro e 2,5~m de altura (o 
arqueólogo aponta uma altura original de 6,5~m para o monte, que
com o tempo cedeu e desmoronou para o tamanho atual). O verão
daquele ano foi seco, o que era uma boa notícia para a equipe, 
uma vez que tornava a escavação muito mais fácil. Mas fazia-se
necessário uma construção provisória que represasse o rio, possibilitando que
mangueiras fossem colocadas para trazer a água e umidificar os
artefatos de madeira que ficariam expostos a
intempéries. Mesmo depois de embalados em 397 pacotes, os artefatos ficariam
submersos em tanques de água até receberem os tratamentos necessários
para sua conservação. Logo, a embarcação seria trazida à tona e ficaria claro
que esta se encontrava quebrada, com sua composição muito
distorcida devido à pressão do monte de pedra, terra e turfa. 
A parte inferior da embarcação havia sido pressionada contra a
base do local de depósito, cuja composição era basicamente de barro e,
por esse motivo, havia cedido facilmente, quebrando a quilha da embarcação
no meio e forçando para cima as madeiras que compunham a câmara do
depósito funerário.

Contudo, o mesmo monte que danificou a embarcação também
preservou em grande medida os achados, uma vez que as condições do
depósito (feito em terreno argiloso, com a parte de seu bordo
para além do mastro, coberta por uma câmara de madeira, sendo a
maior parte dos objetos depositados sob o teto desta câmara, que
contava ainda com uma grande montanha de pedras e com uma cobertura
de uma camada espessa de turfa) que geraram, por fim, uma pressão fundamental para 
privar os artefatos do contato com o oxigênio, isolando-os dos aspectos de
decomposição. A reconstrução da embarcação, no entanto, lembrou a
montagem de um grande quebra-cabeças. Foi deixada ao encargo do
engenheiro naval Fredrik Johannessen, responsável por identificar e
demarcar cada uma das mais de duas mil peças que compunham o navio. No
dia 5 de novembro, a escavação estava completa e as peças se encaminhavam
para Oslo, onde a reconstrução iria começar.

Entre os artefatos, foram escavados: uma embarcação com cerca de vinte
metros de comprimento, construída para abrigar cerca de quinze pares de
remos (e que apresentava alguns desgastes que sugeriam que
havia sido utilizada um bom tempo antes de servir para o depósito
funerário); ossos humanos; uma grande quantidade de objetos de madeira,
incluindo um vagão; tapeçaria; cordas; elementos têxteis; e os esqueletos
de quinze cavalos, quatro cachorros e dois bois. A embarcação de Oseberg
foi, assim, rapidamente detectada como uma embarcação funerária comparável
a outras da região de Vestfold, como as de Tune e Gokstad. Mas a de
Oseberg se destacava pela preservação de seus achados, que incluíam
objetos de madeira e outros objetos orgânicos. Devido à sua importância,
a preservação dos achados era de extrema urgência e a reconstrução
da embarcação ainda demandaria muito tempo e trabalho. Boa parte do
carvalho utilizado para a construção da embarcação estava razoavelmente
preservado e, assim, pôde ser submetido a um processo com óleo de linhaça
e ácido carbólico durante o prolongado tempo de secagem. Os objetos
feitos de ferro foram submetidos à secagem e, em seguida, foram
cozinhados em parafina para evitar o processo de ferrugem. Os de bronze
foram secos e laqueados para evitar decomposições. As cordas da
embarcação foram tratadas com glicerina e os artefatos de couro
foram tratados com óleos específicos. Os artefatos
têxteis apresentavam algumas dificuldades particulares: a lã e a seda
haviam se preservado bem devido à argila, mas o linho havia coagulado em
um amontoado de camadas sobrepostas praticamente impossíveis de serem
separadas.

O arqueólogo Gabriel Gustafsson, por sua vez, embarcou em uma expedição
pelos mais diversos museus da Europa para pesquisar as últimas técnicas
aplicadas na preservação de artefatos. Ele retornou com a ideia
de saturar a madeira em uma solução de água e alúmen, sendo o alúmen
depois lavado e retirado da parte externa da peça, tornando-a apta
a ser exposta ao processo de secagem. Uma vez seca, a madeira ainda
sofreria um processo de revestimento com óleo de linhaça e uma camada de
laca para seu revestimento. O alúmen que permaneceria na parte interna
dos artefatos, com o tempo, se cristalizaria, criando camadas externas a
madeira e gerando, assim, uma estrutura que a protegeria das
contrações e expansões naturais de seu substrato. Contudo, tal técnica, 
por Gustafsson como a melhor, com o tempo foi
apresentando problemas. O alúmen craquelou e os artefatos de madeira se
tornaram, assim, muito delicados e de difícil manuseio. Os artefatos se tornaram, 
também, extremamente sensíveis à variação de temperatura e umidade.
Qualquer variação muito brusca desses fatores poderia
ocasionar um processo reverso da cristalização do alúmen, o que
estouraria os artefatos de madeira.

Os tratamentos supramencionados auxiliaram na preservação de mais de
90\% do carvalho original na reconstrução da embarcação, assim como 
de mais da metade dos pregos usados pelos construtores do
Período Viking. Os postes da proa e da popa da embarcação, assim como seu leme,
foram torcidos devido à pressão do monte funerário. Houve muita
ansiedade para o restauro dessas partes do navio, que também foram
cozidas no vapor e submetidas à pressão de maquinários que conseguiram
endireitar as partes com sucesso. A parte superior do poste traseiro
da embarcação, no entanto, foi exposta às intempéries, provavelmente causadas
pela escavação, em algum momento desconhecido, de outro buraco
no monte funerário. Tal parte superior do poste traseiro se tornou
a única parte nova e totalmente restaurada da embarcação. Foi
desenhada como um rabo de dragão, acompanhando, assim, a cabeça do
dragão que se encontrava esculpida no poste dianteiro. A reconstrução
dos postes da embarcação de Oseberg utilizou como guia uma cena
encontrada na Tapeçaria de Bayeux, onde esses navios foram retratados em
momentos de invasão.

\SIG{Munir Lutfe Ayoub}

Ver também Arqueologia da Era Viking; Embarcações; Noruega da Era
Viking; Sepultamentos e enterros.

\begin{itemize}
\item \versal{ARDWILL-NORDBLADH}, Elizabeth. Re-Arranging History. In: \versal{HAMIKALIS},
Yannis; \versal{PLUCIENNIK}, Mark; \versal{TARLOW}, Sarah. \emph{Thinking through the
body: Archaeologies of Corporeality}. New York: Plenum Publishers, 2002,
pp. 201-216.

\item \versal{FERGUSON}, Robert. \emph{The Vikings}. New York: Penguin Books, 2009.

\item \versal{GULDBERG}, Gustav Adolph. Om Osebergskipets menneskeknokler fra den yngre
järnaldern. \emph{Norsk Magazin for Laegevidenskapen}, 1907, pp.
1385-1397.

\item \versal{GUSTAFSSON}, Gabriel. Notes on a decorated becket from the Oseberg find.
\emph{Saga Book of the Viking Club \versal{V}}, part \versal{II}, 1906, pp. 297-299.
\end{itemize}
\chapter{P \textarc{p}}
\section{\versal{PATRIMÔNIO}}

Patrimônio poderia ser resumidamente definido como um conjunto de bens
materiais, imateriais e naturais, de valor excepcional e reconhecido
interesse, que ajudam a narrar a história de um povo, bem como sua relação com o
meio ambiente. Um importante legado herdado do passado, reconhecido no
presente e transmitido para gerações vindouras, abrangendo por isso
todas as temporalidades, sem permitir generalizações. No entanto, segundo
Françoise Choay, as transferências semânticas e as muitas
modificações tornaram ``patrimônio'' um conceito nômade, preso 
em estratos de significados, bem como repleto
de ambiguidades e contradições.

O infindável debate acadêmico acerca do patrimônio se estende por outros
conceitos igualmente amplos e complexos, tais como: Bens culturais
(divididos em materiais e imateriais -- ou intangíveis -- e bens naturais
-- ou ambientais). Aditam-se ainda os conceitos de monumentos, conjuntos
e locais de interesse, para além de outros ainda mais dúbios, como
preservação, conservação, tombamento etc. As definições jurídicas
desses conceitos também variam de um lugar para o outro, de modo a
melhor se ajustarem as legislações e assegurarem a legalidade dos
processos patrimoniais.

Historicamente, as primeiras preocupações com determinadas heranças
culturais foram esboçadas na Itália durante a passagem da Idade Média
para Idade Moderna, quando, na renascença do século~\versal{XV}, os vestígios de
um grandioso passado clássico passaram a ser orgulhosamente exibidos. No
entanto, somente no final do século~\versal{XVIII} que, na França, a partir da
reação dos enciclopedistas ao vandalismo que se seguiu à Revolução de
1789, foi adquirido, em nome do interesse público, a proteção legal de
determinados bens, aos quais foi conferida a capacidade de representarem
toda nação. Durante o século~\versal{XIX}, na França, na Inglaterra, na
Alemanha e em outras nações europeias surgiram instituições,
predominantemente públicas (mas também privadas) que começaram a
elaborar leis de conservação e de restauração de monumentos, de modo a
tentar estruturar uma prática preservacionista. Por fim, a destruição
causada pelos conflitos armados das duas grandes Guerras Mundiais do
século~\versal{XX} demonstrou a preciosidade, mas também a extrema fragilidade do
patrimônio, levando a criação, em 16 de novembro de 1945, da Organização
das Nações Unidas para Educação, Ciência e Cultura (United Nations
Educational, Scientific and Cultural Organization -- \versal{UNESCO}). Entre
suas muitas atribuições, coube à \versal{UNESCO} incentivar os países a
cooperarem na conservação do patrimônio, criando ao longo das suas
convenções a famosa Lista do Patrimônio Mundial, responsável por incentivar a
preservação de bens considerados significativos para a humanidade. Em
suma, a conformação de um patrimônio comum, partilhado entre todos e
cuja proteção é uma responsabilidade de cooperação da comunidade
internacional.

Apesar da Lista do Patrimônio Mundial crescer em ritmo constante, a
indissociável relação entre o conceito de cultura e patrimônio faz com
que não exista um consenso das resoluções ou das escolhas
institucionalizadas pela \versal{UNESCO}. Muitas são as críticas direcionadas
para o processo de homogeneização, padronização, ou ainda de uma
globalização cultural, apesar das tentativas das novas políticas em
apreenderem e atribuírem mais valor a então intitulada
``diversidade cultural'' em âmbito local, nacional e
transnacional. Ainda assim, o conceito de patrimônio estabelecido pela
\versal{UNESCO} permanece como um dos mais abrangentes, servindo, algumas vezes,
como uma base de referência global.

Todas essas controversas atribuídas às diferentes interpretações e
acepções do Patrimônio, bem como de demais conceitos a ele associados,
multiplicam-se para o estudo das heranças culturais escandinavas da Era
Viking. Porquanto sua análise remete a um vasto e extremamente
diversificado universo de bens encontrados na Dinamarca, Noruega e
Suécia, ou, num sentido mais amplo, abrangendo também a Finlândia,
Islândia e as Ilhas Féroe ou Ilhas Faroé(s). Apesar desse extenso
alcance geográfico, tal levantamento patrimonial ainda estaria
incompleto se desconsiderasse os importantes achados da Inglaterra,
Escócia e Irlanda, além de achados de muitos outros países que extrapolam o limite
europeu, por onde os intrépidos navegadores nórdicos deixaram algum
vestígio de sua passagem. Ou seja, uma dimensão e, principalmente, uma
pluralidade impossível de abranger, cabendo aqui -- como único recurso
possível -- apenas apontar alguns direcionamentos para uma abreviada
compreensão desse abundante patrimônio escandinavo.

O interesse e a definição de todo esse patrimônio começaram no século~\versal{XIX}, 
quando as Sagas Islandesas foram traduzidas, impressas e tornadas
acessíveis. Também nesse período foram descobertas as primeiras embarcações
nórdicas. A partir daí, embalados
pelas óperas wagnerianas, a Era Viking começou a despertar uma
fascinação mundial. Esse processo foi seguido de perto pelo considerável
desenvolvimento da arqueologia no norte da Europa.

Uma das primeiras embarcações encontradas em melhor estado de
conservação foi o Barco de Gokstad, datado século~\versal{IX} e achado em 1880,
no sudeste da Noruega. Este era provavelmente o modelo de barco usado
pelos vikings para atravessar os mares em direção às terras ocidentais.
Atualmente esse famoso bem material está exposto ao público, juntamente
com o barco de Oseberg -- encontrado posteriormente em 1904 --, no Museu
dos Barcos Vikings de Oslo, na península de Bygdøy.

Desde então, as imagens dos grandes guerreiros navegando pelos mares do
norte voltaram a permear o imaginário ocidental, contudo, agora muito
mais pitorescos que ameaçadores. Suas heranças materiais vão desde os
pequenos objetos utilitários, como pentes, recipientes de madeiras,
caldeirões e pás de ferro para cozinhar, até grandes construções como as
fortificações circulares estilo Trelleborg. Cabe mencionar, ainda, os sítios arqueológicos que
preservam o traçado de povoações inteiras, como em Birka; um assentamento
comercial fortificado, localizado na ilha de Björkö, no lago Mälaren, na
Suécia; Ribe, no lado oeste da península da Jutlândia, na Dinamarca; e
Hedeby, localizada no norte da Alemanha, junto à atual fronteira com a
Dinamarca. Nesta última, outro importante remanescente arqueológico
atesta a capacidade técnica dos construtores escandinavos, a
\emph{Danevirke}, originalmente chamada \emph{Danavirki}. Literalmente
``obra dos dinamarqueses'', a \emph{Danevirke} era um misto de estrada e muralha de terra,
acompanhada por um fosso, que se estendia por mais de trinta quilômetros
entre o limite ocidental da Jutlândia e Schleswig, no Mar Báltico.
Algumas das suas partes ainda hoje estão visíveis, com altura variável
entre três a seis metros. Contudo, originalmente, o talude de terra
deveria ser ainda mais alto e coroado por uma paliçada de madeira,
constituindo uma verdadeira fronteira ao sul da Dinamarca.

Patrimônio arquetípico da Escandinávia são as célebres pedras rúnicas.
Espalhadas por todo o território, elas são concomitantemente monumentos
artísticos e documentos literários. Algumas narram e ilustram o lento
processo de conversão ao cristianismo. Nessas pedras, normalmente
esculpidas em baixo-relevo, o uso de cor era extremamente necessário,
pois permitia destacar o motivo decorativo das inscrições. Por isso,
combinavam diferentes tons a fim de criar contrastes e maior nitidez.
Algumas cidades, como Jelling (localizada na região
sudeste, no condado de Vejle, na Dinamarca), conseguem reunir verdadeiros
complexos dessas pedras.
Igualmente importantes são os conjuntos de pedras ou estelas gravadas e
pintadas da ilha de Gotland, situada no báltico sueco.

Bens materiais de dimensões menores, como broches ovalados, fíbulas,
colares, pulseiras, pingentes, enfeites e adornos de bronze ou outros
metais precisos, excedem nos museus europeus. Muitos desses artefatos
podiam combinar cores e materiais diferentes. Eram decorados com
motivos típicos da arte escandinava, baseados em vários animais
estilizados, desenhos de plantas ou entrelaçados de laços e fitas.
Dificilmente aparecem representações humanas. Para Duby,
trata-se de uma arte de grande vitalidade, imaginação e audácia, comum
ao conjunto da Escandinávia, enquanto suas formas se aplicam também a
outros objetos usuais: barcos, edifícios, móveis, arreios, taças para
beber e outras coisas. Campbell, por sua vez, afirma que a
arte viking estava aberta a uma influência -- filtrada -- da Europa
ocidental. Ele a divide em seis estilos sucessivos: Oseberg, Borre,
Jelling, Mammen, Ringerike e Urnes, porém alerta que um novo estilo não
substituiria imediatamente o antigo.

Extremamente comuns também são as diversas moedas que comprovam o vigor
econômico daquela sociedade, bem como as armas de guerra, tais como espadas,
facas, machados e pontas de lanças. Alguns desses objetos possuíam uma
rica ornamentação, principalmente aqueles que cumpriam funções
cerimoniais. Inserido nesse repertório de bens materiais bélicos, cabe
destacar o Elmo de Gjermundbu, encontrado em 1947, na fazenda
homônima, localizada na comuna de Ringerike, no condado de Buskerud, na
Noruega. Esse importante e raro achado arqueológico, apesar de não estar
bem conservado, contribuiu para desmistificar a equivocada imagem --
quase mítica -- dos vikings usando elmos com chifres. Sapatos e
fragmentos dos panos das vestimentas são mais raros. Todos esses bens da
cultura material expressam a qualidade artística da produção
escandinava. 

No entanto, o patrimônio da Era Viking não se resume aos
artefatos produzidos por eles, pois também abrangem aqueles trazidos de outras
regiões, adquiridos por meio comercial ou pilhagens. Segundo Duby, 
não existem remanescentes dos edifícios
religiosos da época pagã, mas cabe considerar os grandes outeiros
funerários e comemorativos, pertencentes a uma arquitetura monumental.
Pois, na época pré-cristã, coexistiam enterros por cremação e inumação. Na
cremação, após as incinerações, os locais eram cobertos com pedras,
algumas vezes em forma de barco, gerando as esplêndidas
``embarcações-túmulos'', como as que fazem parte do cemitério
arqueológico de Lindholm Høje, ao norte da Jutlândia, na Dinamarca.

Para além do patrimônio natural e dos remanescentes da cultura
material, muitas são as tradições, gestos e hábitos herdados da Era
Viking pelos povos escandinavos e da Europa insular. Todos constituem um
importante patrimônio cultural intangível, presente em toda a intensidade
na vida social e nas diversas manifestações culturais, como ritos,
festas, crenças, jogos, músicas, danças, poesias e outras inúmeras
criações artísticas.

\SIG{João Batista da Silva Porto Junior}

Ver também Cultura material; Era Viking; Escandinávia; Fortalezas;
Habitação; Viking.

\begin{itemize}
\item \versal{CAMPBELL}, James Graham. \emph{Grandes Civilizações do Passado: Os
Vikings}. São Paulo: Editora Folio, 2006.

\item \versal{CHOAY}, Françoise. \emph{A Alegoria do Patrimônio}. São Paulo: Editora
Unesp, 2001.

\item \versal{DUBY}, Georges. \emph{História Artística da Europa: A Idade Média -- Tomo~\versal{I}}. 
Rio de Janeiro: Paz e Terra, 2002.

\item \versal{STEFÁNSDÓTTIR}, Agnes \& \versal{MALÜCK}, Matthias. \emph{Viking Age Sites in
Northern Europe: A transnational serial nomination to \versal{UNESCO}´s World
Heritage List}. Islândia: Prentmet, 2014.
\end{itemize}
\section{\versal{PEDRA SOLAR}}

A pedra solar (em nórdico: \emph{solársteinn}), também popularmente
denominada de pedra solar viking, é um tipo de mineral supostamente
utilizado pelos antigos escandinavos como material para auxilio na
navegação marítima. Citada em algumas fontes medievais, ela foi
popularizada recentemente por novas descobertas e pela sua aparição na
série televisiva \emph{Vikings}.

A pedra solar é referenciada em \emph{Rauðúlfs þáttr} (século~\versal{XIII}), 
pela qual se poderia localizar o Sol mesmo em dias
nublados. Ela também é mencionada na \emph{Hrafns saga
Sveinbjarnarsonar} (século~\versal{XIII}) e em inventários monásticos (séculos~\versal{XIV-XV}). 
Alguns pesquisadores alegam que a natureza alegórica dessas
fontes concederia um sentido simbólico à pedra solar, mas, por outro lado,
pesquisas empíricas e novas descobertas vem permitindo considerar que
realmente foi um objeto real utilizado na Era Viking.

O pioneiro dos estudos empíricos com a pedra solar foi o arqueólogo
dinamarquês Thorkild Ramskou, em 1967. Ele acreditava que ela poderia ser
um mineral islandês (a cordiorita), que polarizaria a luz quando o Sol
estivesse oculto por nuvens, auxiliando os nórdicos em navegações em
alto mar durante a Era Viking.

Uma equipe multidisciplinar liderada por Guy Ropars (Universidade de
Rennes), em 2011, determinou que a pedra solar poderia identificar a
direção do Sol a olho nu, dentro de condições turbulentas e
crepusculares. Os testes foram realizados com espato (cristal de calcita
transparente) da Islândia. O processo consistira em mover a pedra
através do campo visual até revelar um padrão entópico amarelado ao
olho. Um outro modo alternativo é inserir um ponto no alto do cristal,
de modo que, quando se olha para o lado de baixo, dois pontos aparecem,
porque a luz é ``despolarizada'' e fraturada ao longo de diferentes
eixos. O cristal pode então ser girado até que os dois pontos tenham a
mesma luminosidade. O ângulo da face superior mostra, dessa maneira, a
direção do Sol.

Os experimentos da equipe liderada por Guy Ropars levaram em conta o
encontro de um cristal de calcita junto a instrumentos de navegação de
um navio britânico naufragado no século~\versal{XVI}. O fragmento foi descoberto
em 2002, no canal da Mancha, e atualmente faz parte do acervo do Museu
Alderney (França). Análises químicas revelaram que se trata de espato da
Islândia. Segundo os pesquisadores, os navegadores renascentistas ainda
utilizariam esse método de orientação pelo fato da grande massa metálica
transportada pela embarcação (os canhões) afetar a funcionalidade da
bússola magnética.

Outra equipe com pesquisadores húngaros e suecos propôs que os nórdicos
teriam utilizado conjuntamente a bússola e a pedra solar, mesmo durante
a noite, após o crepúsculo, para orientação. Também
alguns experimentos náuticos comprovaram a praticidade do equipamento.

Apesar de não existirem descobertas arqueológicas que evidenciem
diretamente a utilização de pedras solares como instrumentos de
navegação, pesquisas recentes apontam o conhecimento de calcita pelos
nórdicos durante a Era Viking, como as encontradas no assentamento de
Annagasson na Irlanda.

\SIG{Johnni Langer}

Ver também Astronomia; Bússola solar; Embarcações; Navegação marítima.

\begin{itemize}
\item \versal{BERNÁTH}, Balázs \emph{et al}. How could the Viking Suncompass be used
withsunstones before and aftersunset? \emph{Proceedings of the Royal
Society}, vol. 470, 2014, pp. 01-18.

\item \versal{GANNON}, Megan. First evidence of viking-Like Sunstone found. \emph{Live
Science}, 6 de março de 2013.

\item \versal{KEMP}, Martin (dir.). \emph{Viking sun stone}. National Geographic
Channels, 2012, documentário, 23 min.

\item \versal{RAMSKOU}, Thorkild. Solstenen. \emph{Skalk}, n. 2, 1967, pp. 16-17.

\item \versal{ROPARS}, Guy \emph{et al}. A depolarizer as a possible precise sunstone
for Viking navigation by polarized sky light. \emph{Proceedings of the
Royal Society}, 2011, pp. 01-14.

\item \versal{SZÁZ}, Dénes \emph{et al}. Adjustment errors of sunstones in the first
step of sky-polarimetric Viking navigation: studies with dichroic
cordierite/tourmaline and birrefringente calcite crystals. \emph{Royal
Society Open Science}, n. 3, 2016, pp. 02-21.
\end{itemize}


\section{\versal{PENTES}}

Os pentes, bem como outros itens ligados aos cuidados com o corpo e à higiene pessoal, eram
apreciados como uma espécie de objetos de luxo, símbolos do
\emph{status} social. Em sepulturas do século~\versal{IX} escavadas na ilha de Gotland, 
Suécia, foram encontrados pentes confeccionados em
osso, artisticamente trabalhados, junto a esqueletos femininos e
masculinos. Portanto, esses objetos eram utilizados tanto por homens
como por mulheres e eram símbolos de higiene e limpeza. Podiam também ser símbolos
religiosos.

Na Era Viking, tais utensílios, devido a sua importância e 
beleza, eram guardados com muito cuidado pelos seus donos, muitas vezes
em estojos especiais, entalhados em madeira e decorados
com filigranas. Isso demonstra a importância que esses objetos possuíam
na época.

Os pentes eram feitos por artesãos especializados, que os produziam
em vários tamanhos e formatos, com dentes mais grossos ou
mais finos, estreitos, mais largos, pequenos e grandes. A variedade de
tamanhos e da espessura dos dentes variava, bem como o
material utilizado na confecção.

No caso dos pentes maiores, o cabo era composto por até oito pedaços e 
os dentes eram entalhados um a um por meio de ferramentas
especiais. Tais pentes tinham uma medida que
variava entre 15 e 20~cm e eram feitos com chifres de
alce. O chifre era serrado em vários pedaços e tamanhos diferentes. Cada pedaço
seria utilizado para elaborar as diferentes partes do pente. Havia
pentes com dentes dos dois lados, entalhados em uma única
peça de osso de alce. Tinham dentes mais largos em uma das faces
e mais finos na outra, não ultrapassando os 15~cm. 
Começaram a ser utilizados a partir do século~\versal{XI}. Os pentes
mais finos e estreitos com dentes longos e finos eram feitos com ossos
do metatarso de gado e utilizados para pentear pequenas e longas mechas
de cabelo. Todos esses pentes eram utilizados por homens e mulheres,
confeccionados manualmente por artesãos especializados, e não podiam ser
considerados apenas objetos de higiene pessoal, pois eram verdadeiras joias
que simbolizavam poder.

\SIG{Luciana de Campos}

Ver também Cotidiano; Higiene e saúde; Mulheres; Sociedade.

\begin{itemize}
\item \versal{ASHBY}, Steve. Viking combs. \emph{University of York}. Disponível em:
\href{https://www.york.ac.uk/research/themes/viking-combs/}{\emph{https://www.york.ac.uk/research/themes/viking-combs/}}. Acesso em 4 dez. 2017.

\item \versal{CARLSSON}, Dan. Combs and comb making in the Viking Age and Middle Age.
\emph{The Vikings}, 2004, pp. 01-10.

\item \versal{HAYWOOD}, John. Combs. In: \emph{Enclycopaedia of the Viking Age}.
London: Thames and Hudson, 2000, p. 48.
\end{itemize}
\section{\versal{PERSONAGENS LITERÁRIAS E HISTÓRICAS}}

Ver Aud (Unn), a de mente profunda; Brian Boru; Canuto, o Grande; Egil
Skallagrimsson; Érico Machado Sangrento (Erik Haraldsson); Érico, o Vermelho;
Freydis Eiríksdóttir; Gudrid Thorbjarnardóttir; Hakon Sigudsson; Hakon
Haraldssoni; Haroldo Dente Azul (Haraldr Gormsson); Haroldo Cabelos Belos (Haraldr Hárfagri);
Haroldo Hardrada (Haraldr Sigurdsson); Ibn Fadlan; Lagertha; Leif Eriksson; Olavo
Haraldsson; Olavo Tryggvason; Olga de Kiev; Ragnar Lodbrok; Rollo;
Vladimir~\versal{I} de Kiev.

\section{\versal{POESIA ÉDDICA}}

A métrica \emph{fornyrðislag}, de acordo com Ólason, é a
mais comum utilizada nos poemas éddicos. Essa métrica se desenvolveu a
partir das longas linhas aliterativas germânicas, encontradas no
repertório cultural medieval de todos os povos de língua germânica, cuja
a poesia vernácula sobreviveu. No entanto, diferentemente dos versos comuns germânicos, que não
tem limites de estrofe (exemplo, \emph{Bewoful} no inglês antigo e
\emph{Hildebrandslied} no antigo alto alemão), a poesia éddica é
construída em estrofes, que são quase sempre de tamanho irregular.

As estrofes do
\emph{fornyrðislag} são, de acordo com a autora, de tamanho irregular. Já os versos, embora com
número variável de sílabas, são geralmente considerados menores do que
os versos germânicos antigos por conta das quedas vocálicas em 
sílabas que não carregavam o acento tônico. Ólason
e Ross complementam ao afirmar que as
estrofes tem oito semiversos (versos curtos), separados por cesura e
interligados por aliteração. Portanto, quatro pares aliterativos. Em
vista disso, cada meia-estrofe, chamada \emph{vísuhelmingur}, tem dois
versos longos e dois pares aliterativos. Não há regras estritas que
controlam o número e a distribuição das sílabas átonas, embora o número
delas seja entre duas e três em um único verso curto, dependendo da
quantidade de sílabas tônicas. De acordo com Ross, tal
aliteração entre os dois semiversos se forma entre uma ou duas sílabas
no semiverso~\versal{A}, e uma, que seria a primeira sílaba acentuada, no
semiverso~\versal{B}. Poole complementa ao afirmar que se a tônica
primária cair em um substantivo ou em um adjetivo, é essa tônica que
deve levar a aliteração.

Eduard Sievers classifica os semiversos do \emph{fornyrðislag} em
cinco tipos. O sistema dessa classificação opera com três níveis de
acentuação: tônica primária, anotada como (/); tônica secundária,
anotada como (/); e tônica mínima, ou átona, anotada como
(x). Por meio dessas anotações, resume-se as categorias dos semiversos
da seguinte maneira: \versal{A} como /x/x; \versal{B} como x/x/; \versal{C} como x //x ou
x/\textbackslash{}x; \versal{D} como //\textbackslash{}x ou //x\textbackslash{} e
\versal{E} como /\textbackslash{}x/. Apenas pouquíssimos versos em
\emph{fornyrðislag} refusam entrar em uma dessas cinco categorias como,
por exemplo, \emph{``Freyju at kvæn''}, do poema \emph{Þrymskviða}.
Poole complementa afirmando que
consoantes iniciais em uma sílaba tônica é suficiente para a aliteração,
exceto nos clusters \emph{sp}, \emph{st} e \emph{sk}, casos em que todo
o cluster é necessário. As vogais iniciais em sílabas tônicas aliteram
entre si e com o \emph{j}. Além do mais, sílabas átonas não entram nesse
esquema e, por conta disso, palavras, como \emph{ek} (``eu''), não têm
função estrutural.

O autor exemplifica com o poema éddico \emph{Oddrúnargrátr} (``Lamento de
Oddrún''), encontrado no \emph{Codex Regius}. Ele segue o
\emph{Guðrunarkviða~\versal{III}} e precede o \emph{Atlaviða}: \emph{Opt undrumk
þat \textbf{ӀӀ} hví ek eptir mák; línvengis Bil \textbf{ӀӀ} lífi halda;
er ek ógnhvǫtum \textbf{ӀӀ} unna þóttumk; sverda deili, sem \textbf{ӀӀ}
sjálfri mér}; trecho citado e normalizado por Neckel \& Kuhn, no qual a marcação \emph{\textbf{II}} indica a quebra do verso.
Tradução proposta: ``Mulher! Eu
sempre me pergunto o porquê de levar a vida para frente, quando me
parece que amei o corajoso guerreiro, provedor de espadas, como amava a
mim mesmo''.

De acordo com Poole, o esquema métrico desse poema é: 1.
//x\textbackslash{} \emph{\textbf{ӀӀ}} xx/x \versal{DB}; 2. /\textbackslash{}x/
\emph{\textbf{ӀӀ}} /x/x \versal{EA}; 3. xx/\textbackslash{}x \emph{\textbf{ӀӀ}}
/x/x \versal{CA}; 4. /x/x \emph{\textbf{ӀӀ}} x/x/ \versal{AB}. Nesse trecho há também um
\emph{kenning} para ``mulher'': \emph{língenvis Bil} ``\emph{Asynjor} do
covil das cobras'' (consulte o verbete \versal{Kenning}).

Ross exemplifica a métrica \emph{fornyrðislag}
com uma estrofe do poema \emph{Atlakviða}, que é um poema sobre
Átila, o huno, registrado na Edda poética. Como a maioria dos poemas
éddicos, o poema é anônimo e tem como tema uma antiga lenda:
\emph{\textbf{A}tli sendi \textbf{ӀӀ á}r til Gunnars; \textbf{k}unnan
segg at ríða, \textbf{ӀӀ} \textbf{K}néfrǫðr vár heitinn; at
\textbf{g}ǫðum kom hann \textbf{G}júka \textbf{ӀӀ} ok at
\textbf{G}unnars hǫllo; \textbf{b}ekkjum aringreypum \textbf{ӀӀ} ok at
\textbf{b}jóri svásum}; normalizado por Neckel \& Kuhn. Tradução proposta: ``Átila enviou um
mensageiro a Gunnar, um homem instruído para cavalgar, Knéfrǫðr era
chamado; veio ao pátio de Gjúki e ao salão de Gunnar, para os bancos
envolvidos na lareira, para a cerveja agradável''.

De acordo com Ross, o narrador se oculta da narração e
descreve um evento assumindo que a audiência o conheça, ao mesmo
tempo em que embriaga esse trecho com um suspense dramático, sobretudo no momento em que o
mensageiro entra no mundo acolhedor do salão germânico com a intenção de
causar desordem. Esse mundo é bem típico e os epítetos (``bancos
envolvidos na lareira'' e `` cerveja agradável''), bem como as formas em
genitivo -- que causam uma semântica de ``posse'', ``pertencente a'',
(``pátio de Gjúki'' e ``salão de Gunnar'') --, sugerem um ambiente
conhecido, no qual, no entanto, a discórdia está prestes a ser
desencadeada. Portanto, de acordo com a autora, esse poema é muito
parecido com a poesia heroica em inglês antigo e com o que está
preservado nos versos heroicos em antigo alto alemão.

Com relação à métrica, trata-se de uma estrofe com quatro pares de
versos aliterativos: a primeira sílaba tônica do segundo semiverso
(parte~\versal{B}) se alitera com o primeiro semiverso (parte~\versal{A}). No entanto, há
um descompasso no número de aliterações em cada verso longo, uma vez que,
no terceiro verso longo, se aliteram três palavras. Os versos longos
também têm números variáveis de sílabas e, consequentemente, o número de
sílaba átonas é variado.

A prática ocasional de regularmente adicionar uma ou duas sílabas átonas
ao verso do \emph{fornyrðislag} resultou no \emph{málaháttr}. Poole 
afirma que, geralmente, há pelo menos
cinco sílabas por semiverso, às vezes seis. Embora essa métrica não seja
encontrada com muita frequência (apenas esporadicamente dentro de alguns
poemas em \emph{fornyrðislag}), ela, no entanto, 
aparece sozinha no poema éddico \emph{Atlamál in grænlenzku}.

Exemplificamos essa métrica com a estrofe 79:
\emph{\textbf{T}}ók ek þeira hjörtu, \emph{\textbf{ӀӀ}} ok á
\emph{\textbf{t}}eini steikðak; \emph{\textbf{s}}elda ek þér
\emph{\textbf{s}}íðan, \emph{\textbf{ӀӀ s}}agðak, at kalfs væri:
\emph{\textbf{e}}inn þú því \emph{\textbf{o}}llir, \emph{\textbf{ӀӀ}}
ekki réttu leifa, \emph{\textbf{t}}öggtu \emph{\textbf{t}}íðliga,
\emph{\textbf{ӀӀ t}}rúðir vel jöxlum. Tradução de conteúdo proposta:
``Eu tirei os corações deles; em um espeto os cozinhei; e então os dei
para ti; disse que eram de bezerros; tu comeste tudo sozinha, tu não
deixaste nada; com voracidade tu mastigastes, os dentes estavam
ocupados''.

Na maioria dos semiversos há cinco sílabas, com exceção, por exemplo,
dos quatro primeiros. Entretanto, como considerado acima,
sílabas átonas não entram nesse esquema e, por conta disso, se
considerarmos as palavras átonas \emph{ek} (``eu''), \emph{ok} (``e'') e
\emph{at} (``que'') como sem função estrutural, teremos cinco sílabas nos
semiversos em que elas aparecem. Os semiversos ímpares, com exceção do
primeiro, têm duas palavras que aliteram com uma palavra dos semiversos
pares correspondentes. Esse poema é o mais longo dos poemas heroicos da
\emph{Edda poética} e conta a mesma história de \emph{Atlakviða}, mas expressa
uma visão de mundo e inspirações artísticas totalmente diferentes.
O poema tem algumas partes hediondas como, por
exemplo, na estrofe em que Guðrún descreve seus planos de matar seus
filhos, o que realiza em seguida. Outro exemplo se encontra na estrofe citada acima,
na qual Guðrún come o coração de seus filhos em um espeto.

Além das métricas \emph{fornyrðislag} e \emph{málaháttr}, também há a
métrica \emph{ljóðaháttr} (``métrica da canção'') nos poemas éddicos, que
era mais frequentemente utilizada em poesia de diálogos gnômicas.
De acordo com os autores, nessa métrica, cada
estrofe é dividida em duas semiestrofes, que têm um par de semiversos
cada, conectados por aliteração, somado a um verso longo com três sílabas
tônicas e aliteração interna. Poole complementa ao
afirmar que a métrica tem, portanto, uma estrutura de três partes. O verso
longo não tem cesura e contém duas ou até três sílabas tônicas com,
frequentemente, uma sílaba tônica secundária. Tais características
são encontradas no exemplo abaixo:

Estrofe 57 do poema éddico \emph{Hávamál}, da \emph{Edda Poética}: 
\textbf{M}eðalsnotr \emph{\textbf{ӀӀ}} skyli
\textbf{m}anna hverr, æva til \textbf{s}notr \textbf{s}é /
\textbf{ö}rlög sín \emph{\textbf{ӀӀ}} viti \textbf{e}ngi fyrir, þeim er
\textbf{s}orgalau\textbf{s}astr \textbf{s}efi. Tradução de conteúdo
proposta: ``Mediano deveria cada homem ser, nunca muito sábio; seu
futuro, ninguém sabe de antemão, a esse a mente fica mais despreocupada.
'' (tradução nossa).

\SIG{Yuri Fabri Venancio}

Ver também Heiti; Kenning; Linguagem; Literatura; Norreno; Poesia
escáldica.

\begin{itemize}
  \item  \versal{JÓNSSON}, Finnur. \emph{Sæmundar-Edda}. Reykjavik: Kostnaðarmaður,
Sigurður Kristánsson, 1905.

\item  \versal{ÓLASON}, Vésteinn. Old Icelandic Poetry. In: \versal{NEIJMANN}, Daisy. \emph{A
History of Icelandic Literature}. Lincoln/London: University of Nebraska
Press, 2006, pp. 01-63.

\item \versal{POOLE}, Russell. Metre and Metrics. In: \versal{McTURK}, Rory (ed.). \emph{A
Companion to Old Norse-Icelandic Literature}. Malden/Oxford/Victoria:
Blackwell Publishing Ltd, 2005, pp. 265-284.

\item \versal{ROSS}, Margaret Clunies. \emph{A History of Old Norse Poetry and
Poetics}. Cambridge: D. S. Brewer, 2005.
\end{itemize}
\section{\versal{POESIA ESCÁLDICA}}

A poesia escáldica era composta
por poetas profissionais islandeses, os escaldos. Estes viajavam
entre as cortes dos reis e nobres escandinavos, ao mesmo tempo em que
compunham e executavam versos de elogio e louvor aos seus anfitriões.
Apesar dos islandeses terem dominado essa arte e monopolizado a função
de poeta da corte na Escandinávia, esse tipo de poesia de elogio já era
composta na Noruega antes da colonização da Islândia.
Em um dos manuscritos da \emph{Edda Poética} e, também, em um
manuscrito do \emph{Heimskringla}, há um catálogo de escaldos ou poetas
da corte, chamado de \emph{Skáldatal}, arranjados cronologicamente de
acordo com os reis e magnatas elogiados pelos poetas e, como primeira
menção, está Ragnar Lóðbrók, o legendário viking dinamarquês do século~\versal{IX},
que tinha como escaldo o Bragi Boddason. Muitos outros reis antigos,
dinamarqueses e suecos, são
mencionados nessa obra. No entanto, não há preservada nenhuma poesia que
elogia tais reis dinamarqueses e suecos antes da era dourada da poesia
cortês islandesa (por volta de 1000 d.C.). Ademais, duvida-se que o
elogio do poema \emph{Ragnardrápa}, atribuído a Bragi, de fato seja um
elogio a Rágnar Lóðbrók. A respeito da Noruega, há muito mais evidências
da prática da poesia de elogio em estilo escáldico nas cortes dos reis
noruegueses a partir do período de Haroldo Cabelos
Belos (Haraldr Hárfagri), cujos escaldos eram também noruegueses. No entanto, após esse
período, o único escaldo norueguês conhecido é \emph{Eyvindur skáldaspillir},
que elogiou o rei Hákon, o bom (960 d.C.). Por fim, foram os islandeses
que dominaram a poesia de elogio a partir do fim do século~\versal{X}. Os reis
noruegueses aceitaram poemas de elogio de islandeses por todo o século~\versal{XIII}, 
mas talvez mais por razões diplomáticas do que pela apreciação da
obra.

A poesia escáldica pode ser dividida em duas categorias principais:
poesia cortês e poesia pessoal. A poesia cortês era
composta para príncipes e outros governantes e inclui, além do verso
laudatório, a poesia genealógica e mitológica; a poesia pessoal, por
outro lado, é formada principalmente de estrofes soltas, os
\emph{lausavísur} (``versos soltos''). A poesia de elogio, que faz parte
da poesia cortês, pode ser dividida em dois tipos: \emph{drápa},
considerado o de mais prestígio e que se tornou o único tipo utilizado
para elogiar os reis; e o \emph{flokkur}. Ambas
utilizam como métrica principal o \emph{dróttkvætt} (``métrica da
corte''). O autor também caracteriza o \emph{drápur} como um poema de
três seções, sendo a do meio dividida em subseções com refrão
(\emph{stef}) e \emph{flokkur} (``grupo''), como um tipo de poesia que
não tem nenhuma regra formal de composição acima do nível da estrofe.

Ross também menciona os subgêneros \emph{erfidrápa} (de
\emph{erfi}, ``funeral'', ``banquete''), ou balada memorial, utilizada
para ser recitada em funerais ou celebrações de morte de algum rei ou
\emph{jarl}. É o caso, por exemplo, da \emph{Glægonskviða} (``Canção do
silêncio'') de Þórarin loftunga em memória ao rei norueguês Olavo~\versal{II}, da
Noruega, e endereçada ao rei Svein Knútsson. Segundo Ohlmarks, o
\emph{erfidrápa} foi a origem de toda a arte escáldica.
O segundo tipo são poemas genealógicos, como o
\emph{Ynglingatal} (``Listas dos Ynglingar'', 900 d.C. de Þjóðólfr of
Hvinir), o \emph{Háleygjatal} (``Lista dos homens de Hálogaland'', 986
d.C., de Eyvindr Skáldaspillir) destinado ao \emph{jarl} Hákon
Sigurðarson (considerado o último grande governante pagão da Noruega, após sua
vitória sobre os Jómsvíkingar, um grupo de guerreiros que tinha base na
costa báltica) e, por por último, o poema \emph{Nóregskonungatal} (``lista
dos reis noruegueses'', 1180 d.C. do islandês Jón Loptsson, preservado
no manuscrito \emph{Flateyjarbók}). Um terceiro subgênero engloba os
poemas pictóricos, que descrevem objetos. Neles a
narrativa mítica e lendária é primordial e, por isso, não tiveram lugar no
mundo após a implementação do cristianismo. Os poemas pictóricos eram
poemas compostos de um escaldo para seu patrão. O escaldo compunha um
poema que descrevia visualmente o objeto, o qual se tratava de um presente
recebido do patrão. Este esperava um poema como retribuição. Tal
subgênero, por apresentar narrativa mítica e lendária, manifesta uma
continuidade com as narrativas míticas e heroicas das métricas éddicas,
mas transportadas para um mundo cortês. Portanto, pode-se afirmar que às
vezes a poesia escáldica e éddica tratavam de um mesmo tipo de tema.
Dois bons exemplos desse subgênero são os poemas \emph{Haustlǫng}, de
Þjóðólfr of Hvinir, e \emph{Rágnarsdrápa}, de Bragi Boddasson.

Com relação à preservação da poesia escáldica, Ólason
afirma que ela pode ser considerada acidental, pois se deu à medida que
as poesias foram
citadas como material de origem, ou que foram incorporadas, por outras razões, em
sagas de natureza mais ou menos históricas, ou mesmo à medida que eram
utilizadas como exemplo em tratados. Com relação aos textos
normalizados, encontrados na maioria das edições, eles foram
estabelecidos por uma conflação de textos provenientes de mais de um manuscrito e, às
vezes, por conjecturas baseadas em emendas nos registros dos
manuscritos.

A respeito das métricas, além da métrica \emph{dróttkvætt} (métrica da
corte), que é a mais utilizada nas poesias escáldicas, há também a
\emph{kviðúháttr} (métrica do poema), a \emph{hrynhenda}
(\emph{hrynhent, hrynjandi háttr}) -- métrica fluída -- e a
\emph{runhenda} (com rima no final ou \emph{runhendr háttr}).
Comentaremos sobre tais métricas a seguir.

Ross afirma que o desenvolvimento mais importante do
\emph{fornyrðislag}, típico das poesias éddicas, além do próprio
\emph{dróttkvætt}, foi o \emph{kviðúháttr}, característico por alternar
versos de três e quatro sílabas, em versos ímpares e pares,
respectivamente, e com falta de ritmo regular. Como apontado no verbete
\versal{Literatura na Era Viking}, essa métrica tem similaridades tanto com os
versos édicos quanto com os versos escáldicos, mas é melhor classificada
como escáldica pois, embora dispense rima interna, as sílabas são
contadas e há alguns \emph{kenningar}. Poole afirma que
as regras que controlam o esquema silábico e que são aplicadas apenas a
uma minoria de poemas na métrica \emph{fornyrðislag} são muito mais
documentadas e explicitadas na métrica \emph{kviðuháttr}. Tais regras
são dependentes do comprimento silábico, que pode ser longo ou curto.
Uma sílaba longa consiste de uma vogal longa (exemplo, \emph{mér}, ``a
mim'') ou um ditongo seguido por uma ou mais consoantes (por exemplo, a
primeira sílaba de \emph{reykr}, ``fumaça''). A sílaba também pode ser
longa se possuir uma vogal curta seguida de um grupo consonantal ou uma
consoante geminada (por exemplo, as primeiras sílabas de \emph{marg-ir},
``muitos'' no masculino, e \emph{drekk-a}, ``beber''). Uma sílaba
será curta se houver uma vogal curta seguida por uma única
consoante (por exemplo, \emph{bað}, ``banho'') ou uma vogal longa seguida por
uma outra vogal sem a intervenção de uma consoante (segunda sílaba de
\emph{lofgró-inn}, ``crescido das folhas'').

O autor exemplifica a métrica \emph{kviðúháttr} com uma estrofe do poema
\emph{Ævikviða}, contido na \emph{Saga de Grettir}. Versão de Jónsson:
1. \emph{Sǫ\textbf{g}ðu mér, \textbf{ӀӀ} þau's
Si\textbf{g}arr veitti;} 2. \emph{\textbf{m}ægða laun \textbf{ӀӀ m}argir
hœfa;} 3. \emph{unz lo\textbf{f}gróinn \textbf{ӀӀ} lau\textbf{f}i
sœmðar;} 4. \emph{\textbf{r}eyni\textbf{r}unn \textbf{ӀӀ r}ekkar fundu.}
Tradução: ``Muitos disseram que a recompensa dos parentes, pagas pelo
Sigarr, seria adequada para mim [i.e. pendurando]; até os homens
encontrarem o arbusto de sorveira-brava, crescida louvavelmente com
folhagem de honra [Þorbjǫrn]'' (tradução nossa, com base em Poole). No
segundo semiverso do primeiro verso percebe-se que há cinco sílabas e não quatro.
Árnasson, afirma que nesse caso deve-se considerar
a ocorrência de uma resolução, em que um par de sílabas, das quais tanto a primeira quanto a
segunda são curtas, equivale metricamente como uma sílaba longa; e da
\emph{neutralização}, em que um enclítico (\emph{es}, mais tarde
\emph{er}; pronome relativo, ou \emph{ek}, ``eu'') se torna não silábico
ao perder a vogal (\emph{þau es} $>$ \emph{þau's}). Ross
também exemplifica a métrica \emph{kviðuháttr} com o poema escáldico
\emph{Arinbjarnarkviða} (``Poema sobre Arinbjǫrn'', um chefe local) de
Egill Skallagrímsson (960 d.C., manuscrito Mǫðruvallabók, \versal{AM} 123 fol.,
de 1350), no qual está registrado no folio 99v, após o texto da saga de
Egil: 1. \emph{Þat \textbf{a}lls heri \textbf{ӀӀ} at \textbf{u}ndri
gefsk; 2. hvé hann \textbf{u}rþjóð \textbf{ӀӀ au}ði gnœgir; 3. en
\textbf{G}rjótbjǫrn \textbf{ӀӀ} of \textbf{g}œddan hefr; 4.
\textbf{F}reyr ok Njǫrðr \textbf{ӀӀ} at \textbf{f}járafli.~}Versão em
prosa de Ross: \emph{þat gefsk at undri allsheri hvé hann gœgir urþóð
auði, en Freyr ok Njǫrðr hefr of gœddan Grjótbjǫrn at fjárafli}.
Tradução de conteúdo proposta: ``é de se maravilhar por todo o mundo
como ele é capaz de prover riquezas ao povo, mas Frey e Njord dotaram o
urso das pedras ($=$ Arinbjǫrn) com o poder da riqueza''.

Percebe-se que, nesse poema, há similaridades tanto com a poesia éddica
quanto com a escáldica, mas, de acordo com Ross, é melhor
classificada como escáldica, pois, embora não haja rimas internas, é uma
métrica que conta as sílabas e que também apresenta alguns
\emph{kenningar} ou formações parecidas com \emph{kenningar}, como em
\emph{Grjótbjǫrn}. Nessa palavra, há um dispositivo escáldico
\emph{ofljóst} (``muito claro'') de trocadilho: Arinbjǫrn é chamado
de ``urso da pedra''. Para entender o trocadilho, é apenas necessário
substituir \emph{grjót} (``pedras'') por \emph{arinn} (``lareira''
feita de pedras), enquanto \emph{bjǫrn} (``urso'') seria comum para as duas.
Contudo, diferentemente da poesia escáldica, a ordem das palavras é
simples e as duas partes da estrofe não são sintaticamente discretas.
Também é visível uma alternância que varia de três a quatro sílabas. Outros
poemas nessa métrica são: \emph{Sonatorrek} (``A Perda Árdua dos
Filho'', consulte o verbete \versal{Sonatorrek}), também de Egill
Skallagrímsson; \emph{Ynglingatal} (``Listas dos Ynglingar'');
\emph{Háleygjatal} (``Lista dos homens de Hálogaland'') e o
\emph{Nóregskonungatal} (``Lista dos reis da Noruega''), citados
anteriormente. Para mais exemplos, consulte Faulkes (1999, p. 83 para
uma lista) e Fidjestøl (1982, p. 175-177).

De acordo com Williams, a estrofe em \emph{dróttkvætt}
mais antiga se encontra na estela rúnica de Karlevi (Karlevi, ilha de
Öland, Suécia, código Ög 1, 1000 d.C.), assim como a única estrofe
completa do texto original.

Com relação à métrica, ela tem oito versos, contidos em dois quartetos e
três sílabas tônicas em cada verso. Os versos ímpares têm rima interna
incompleta \emph{skothending}, que rima palavras com uma ou mais
consoantes idênticas e vogais diferentes; ao passo que os versos pares
tem a rima completa \emph{aðalhending}, que rima palavras que têm tanto
vogais idênticas quanto uma ou mais consoantes idênticas. Ademais, o
poema tem aliteração e um troqueu no último par silábico do verso.
Todavia, os pares
silábicos anteriores podem ser de um dos tipos da métrica
\emph{fornyrðislag} apresentados por Sievers.

A terceira métrica a ser comentada, a \emph{hrynhenda} (\emph{hrynhent,
hrynjandi háttr}, ``métrica fluída''), que é representada principalmente
pelo escaldo Arnórr Þórðarson (século~\versal{XI} d.C.), foi escolhida como a
principal métrica para compor poemas religiosos. De acordo com Attwood,
o poema de elogio do escaldo ao rei Magnus Óláfsson é o
único exemplo existente dessa métrica, que é um desenvolvimento do
\emph{dróttkvætt}, uma vez que amplia as seis sílabas em um verso de
três sílabas tônicas para oito sílabas, das quais quatro são acentuadas.
Além do mais, de acordo com Whaley, embora a estrutura
rítmica e assonante dos poemas na métrica \emph{dróttkvætt} seja
mantida, os versos na métrica \emph{hrynhenda} têm uma cadência
descendente, que possivelmente surgiu por conta da influência da métrica
trocaica utilizada nas sequências litúrgicas e hinos do mundo latino
(\emph{apud} \versal{ATWOOD}, 2005, p. 49). Assim, a ampliação para essa métrica
foi importante para o desenvolvimento da poesia cristã.

Exemplificaremos essa métrica com o poema \emph{Hrynhenda}, de Arnórr
Þórðarson: 1. Lj\textbf{ót}u dreif á lypting \textbf{út}an; 2.
l\textbf{\emph{auð}}ri -- bifðisk goll et r\textbf{\emph{auð}}a; 3.
f\textbf{ast}ligr hneigði fúru ge\textbf{yst}ri; 4.
f\textbf{\emph{ýr}}is garmr -- ok skeiðar st\textbf{\emph{ýr}}i; 5.
St\textbf{irð}um helzt umb Stafangr n\textbf{orð}an; 6.
st\textbf{\emph{ál}}um -- bifðusk fyrir \textbf{\emph{ál}}ar; 7.
\textbf{upp}i glóðu élmars t\textbf{ypp}i; 8. \textbf{\emph{Eld}}i glík
-- í Danav\textbf{\emph{eld}}i. Versão em prosa proposta por Whaley
(2009): \emph{Ljótu lauðri dreif útan á lypting ok stýri skeiðar; it
rauða goll bifðisk; fastligr garmr fýris hneigði geystri fúru. Helzt
stirðum stólum norðan umb Stafangr í Danaveldi; álar bifðusk fyrir;
typpi élmars glóðu uppi glík eldi}. Tradução proposta por nós: ``a
imunda espuma esguichou contra a popa e contra o leme do navio; o
dourado vermelho estremeceu. O persistente cão de caça do pinheiro
inclinou o pinheiro violento. Você manejou as robustas proas a partir do
Norte, passando por Stavanger, até o reino dos daneses. As correntes
estremeceram a dianteira. Os topos dos mastros do corcel tempestuoso
brilhavam como fogo''. O poema, portanto, também apresenta
\emph{kenningar}: \emph{fastligr garmr fýris} (``persistente cão de caça
do pinheiro'' $=$ [\versal{VENTO}]), \emph{geystri fúru} (``pinheiro violento'' $=$
[\versal{NAVIO}]) e \emph{élmarr} (``corcel tempestuoso/da tempestade`` $=$
[\versal{NAVIO}]).

As primeiras quatro sílabas (ou dois primeiros pares silábicos), de
acordo com Poole, coincidem em um dos cinco tipos
silábicos da métrica \emph{fornyrðislag} apresentados por Sievers, o que
também ocorre na métrica \emph{dróttkvætt}, como apresentado
anteriormente. Porém, as duas sílabas finais (ou o par silábico
final) formam um troqueu, assim como também na métrica
\emph{dróttkvætt}. A diferença ocorre no terceiro par silábico, que tem
sempre a forma /x, de acordo com os modelos de Sievers, em que o /
representa um acento primário e o x, um acento mínimo. Em vista disso,
esse poema fica exclusivamente no par silábico do tipo~\versal{A} de Sievers:
/x/x /x/x /x/x /x/x. Poole afirma que, ao contrário do
\emph{dróttkvætt}, o desenvolvimento do \emph{hrynhenda} constitui um
claro e muito fascinante caso de hibridismo, pois essa métrica exibe a
influência do tetrâmetro trocaico das sequências litúrgicas e hinos do
mundo latino.

A última métrica a ser considerada é a \emph{runhenda}. De
acordo com Poole, tal métrica apresenta uma
característica que não existe na poesia escáldica em geral: a rima
final. Como exemplo mais antigo, há o poema \emph{Hǫfuðlausn} (``o resgate
da cabeça''), de Egill Skallagrimssón (século~\versal{X}). As rimas, de acordo com o
autor, se formam às vezes entre palavras masculinas, femininas, às vezes
em pares, às vezes abrangendo quatro versos etc. Muitas das rimas são
do tipo completa (\emph{aðalhendingar}), mas algumas são incompletas
(\emph{skothendingar}). Gísli Súrsson é um poeta a quem se
atribui, mas não com toda certeza, alguns poemas na métrica
\emph{dróttkvætt} com rimas finais. O escaldo \emph{Rǫgnvaldr jarl}
também compôs um \emph{lausavísa} com essa característica. Como exemplo,
o autor apresenta uma estrofe do poema Lilja, de Eysteinn Ásgrímsson
(século~\versal{XIV}), que seria um poema já muito além do período da Era Viking.

\SIG{Yuri Fabri Venancio}

Ver também Inscrições rúnicas; Heiti; Kenning; Linguagem; Literatura;
Norreno; Poesia éddica.

\begin{itemize}
\item \versal{ATTWOOD}, Katrina. Christian Poetry. In: \versal{McTURK}, Rory (ed.). \emph{A
Companion to Old Norse-Icelandic Literature}. Malden/Oxford/Victoria:
Blackwell Publishing Ltd, 2005, pp. 43-63

\item \versal{FAULKES}, Anthony. \emph{Snorri Sturluson Edda. Háttatal}. London: Viking
Society for Northern Research, 1999.

\item \versal{FIDJESTØL}, Bjarne. \emph{Det norrøne fyrstediktet}. Øvre Ervik: Alvheim
\& Eide, 1982
(\href{https://bibsys-almaprimo.hosted.exlibrisgroup.com/primo_library/libweb/action/search.do?vl(freeText0)=Nordisk+institutts+skriftserie+\%2F+Universitetet+i+Bergen\&vl(108994631UI0)=lsr32\&vl(108994632UI1)=all_items\&fn=search\&tab=default_tab\&mode=Basic\&vid=UBO\&scp.scps=scope\%3A(SC_OPEN_ACCESS)\%2Cscope\%3A(DUO)\%2Cscope\%3A(\%22UBO\%22)\%2Cprimo_central_multiple_fe\&ct=lateralLinking}{\emph{Nordisk
institutts skriftserie / Universitetet i Bergen}}).

\item \versal{JANSSON}, Sven B. F. \emph{Runes in Sweden}. Stockholm: Gidlunds, 1987.

\item  \versal{JÓNSSON}, Finnur. \emph{Den norsk-islandske Skjaldedigtning}. \emph{B.
Rettet Tekst}. \versal{I}. København: Rosenkilde og Bagger, 1973.

\item  \versal{ÓLASON}, Vésteinn. Old Icelandic Poetry. In: \versal{NEIJMANN}, Daisy. \emph{A
History of Icelandic Literature}. Lincoln/London: University of
NebraskaPress, 2006, pp. 01-63.

\item \versal{POOLE}, Russell. Metre and Metrics. In: \versal{McTURK}, Rory (ed.). \emph{A
Companion to Old Norse-Icelandic Literature}. Malden/Oxford/Victoria:
Blackwell Publishing Ltd, 2005, pp. 265-284.

\item \versal{ROSS}, Margaret Clunies. \emph{A History of Old Norse Poetry and
Poetics}. Cambridge: D. S. Brewer, 2005.

\item \versal{WHALEY}, Diana. (Introduction to) Arnórr jarlaskáld Þórðarson,~Hrynhenda,
Magnússdrápa. In: \versal{GADE}, Kari E. (ed.).~\emph{Poetry from the Kings'
Sagas 2: From c. 1035 to c. 1300. Skaldic Poetry of the Scandinavian
Middle Ages 2}. Turnhout: Brepols, Turnhout, 2009, pp. 181-206.

\item \versal{WILLIAMS}, Henrik. Runes. In: \versal{BRINK}, Stefan; \versal{PRICE}, Neil (eds.).
\emph{The Viking World}. London/New York: Routledge, 2008, pp. 281-290.
\end{itemize}
\section{\versal{POVOS E ETNIAS}}

Ver Anglo-saxões e nórdicos; Árabes e vikings; Celtas e nórdicos; Godos;
Esquimós e nórdicos; Finns e nórdicos; Rus; Sámi; Fínicos e nórdicos;
Viking.

\chapter{R \textarn{r}}
\section{\versal{RAGNAR LODBROK}}

Ragnar Lodbrok (Ragnar Lóðbrok), ou Ragnar Calças Felpudas, foi um rei
lendário ou semilendário do Período Viking, muito popular nas narrativas
medievais escandinavas e ligado a relevantes casas reais, como as
dinastias \emph{Ynglinga} e \emph{Munsö}. Apesar de de ser certo
que tal rei não existiu fora das representações encontradas nas narrativas míticas, 
o debate acadêmico ainda gravita entre a possibilidade do
personagem mitológico ser o resultado de uma amálgama entre vários personagens
históricos, ou produto do imaginário escandinavo da Era Viking e,
principalmente, de épocas posteriores.

As narrativas em torno de Ragnar Lodbrok contribuíram muito para a
imagem romântica que possuímos dos vikings, haja vista que o personagem
possuiu uma vida pautada pela aventura nos mares, pelo saque aos reinos
cristãos e pela violência em combate e em morte, ideais também atribuídos
aos seus filhos. Mas a sua figura também contribui para constituir um
exemplo de como os escandinavos na Idade Média imaginaram o seu passado
heroico. A Saga de Ragnar Lodbrok, \emph{Ragnars saga Lóðbrokar}, por
exemplo, foi escrita na Islândia por volta da primeira metade do 
século~\versal{XIII} e narra a vida de um homem que teria vivido séculos antes. No
entanto, ela já havia sido incorporada ao passado dinamarquês pelos
escritos de Saxo Grammaticus, os Feitos dos Daneses, \emph{Gesta
Danorum}.

Ragnar é uma figura que pode ser sintetizada da seguinte maneira:
primeiro, temos um chefe guerreiro de nome Ragnar, que morreu após o saque
em Paris, em 845, sendo lembrado pela abadia de Saint-Germain-des-Près;
depois, precisamos lembrar que \emph{Lothkona} ou \emph{Lodbroka}, não
só era nome de um espírito tutelar feminino, como também possivelmente 
a mãe -- enquanto conceito polissêmico -- de uma família de
saqueadores que agiram entre os reinos Anglo-saxões e a Francia, entre
860 e 870, conhecidos séculos depois como os filhos de Lodbrok; agora
pensemos numa mescla de narrativas sobre princesas e dragões (este
último elemento muito popular em heróis germânicos, de Beowulf a Sigurð);
somemos a isso uma invasão histórica ao reino da Nortúmbria, onde o rei
Ӕlla faleceu, e a morte em um poço com víboras, nos mesmos moldes de
Gunnar na Saga dos Vǫlsungos, \emph{Vǫlsunga saga}.

Ao adicionarmos conquistas de territórios e a formação de um império
norte-atlântico, conflitos políticos e códigos legais, temos o rei
Regnerus que ocupa a narrativa do nono livro dos \emph{Feitos dos Daneses}.
Retiremos boa parte do conteúdo político, temos a Saga de Ragnar
Lodbrok.

Ademais, os detalhes da sua vida lendária revelam um caso espetacular
de circularidade das fontes orais e escritas, de modo que podemos visitar uma
miscelânea composta por cronistas normandos. Inclusive, o registro mais
antigo desse personagem pode ser encontrado nos Feitos dos Duques
Normandos (\emph{Gesta Normannorum Ducum}), de Guilherme de Jumièges,
que menciona um rei de origem anglo-escandinava de nome \emph{Lothbroc},
o qual havia forçado o seu filho Bjórn Flanco de Ferro ao exílio a fim
de que o jovem organizasse sua própria vida de pilhagens.

Outras fontes medievais que podemos citar são: o \emph{Conto dos Filhos de
Ragnar} (\emph{Ragnarssona þáttr}), as já mencionadas no \emph{Feitos dos
Daneses} (\emph{Gesta Danorum}), da pena de Saxo Grammaticus e a \emph{Saga de
Ragnar Lodbrok} (\emph{Ragnars saga Lóðbrokar}), além dos poemas conhecidos como os ``Ditos
de Kraka'' (\emph{Krákumál}) e o ``Elogio a Ragnar'' (\emph{Ragnarsdrápa}).
Excetuando a última fonte -- que é atribuída ao poeta Bragi Boddason, o
Velho, e datada em torno do século~\versal{IX} --, todos os detalhes que possuímos de
sua vida advém de fontes tardias e de larga inspiração mitológica, como
já mencionado.

Narrativas sobre a sua vida lendária contam que ele teria sido filho de
Sigurd Hringr (Anel) e casado com diferentes princesas, com 
as quais teve diversos filhos que também se lançaram ao mundo
heroico do imaginário viking. Ragnar teria sido, ainda, parente de
Godofredo, rei dinamarquês do século~\versal{IX}, também distinguido por sua
conduta heroica, o que pode levar a crer que a figura de Horik (filho de Godofredo) 
tenha inspirado alguns elementos de seus feitos. Outras
personalidades do medievo escandinavo que podem ter contribuído para a
construção da lenda de Ragnar Lodbrok são Ragnfrid, Reginherus e
Ragnall (provavelmente Rangvald, \emph{jarl} de Mœrr).

Sua alcunha está ligada ao modelo de calças utilizadas para enfrentar e
vencer duas serpentes que impediam qualquer aproximação entre o herói e
a princesa Thora Borgahjort. É dito que essa princesa sueca havia criado
duas serpentes que seu pai trouxera de uma caça, mas que haviam crescido
desmedidamente e passaram a ser consideradas uma ameaça. Nenhum dos seus
pretendentes havia conseguido superar tal obstáculo, até que Ragnar se
prontificou. O herói usou calças felpudas para se proteger do veneno das
serpentes e as endureceu com areia, ou jogando-as em água gélida (as
fontes divergem).

Outros feitos lendários são creditados ao herói e, entre eles, os
mais celebrados que podemos mencionar são os saques no mar Báltico, o
famoso Cerco à Paris em 845 (onde conseguiu o exorbitante pagamento de
sete toneladas de prata), o famoso \emph{danegeld} e a sua
invencibilidade em batalha, que perdurou por toda a sua vida guerreira,
sendo vencido apenas no fim pelo rei Aella~\versal{II}, da Nortúmbria. Herdeiro das
possessões de seu pai (Sigurð Hring, rei dos suecos e conquistador de
reinos na Dinamarca), Lodbrok também teria governado algumas regiões na
Inglaterra e Finlândia.

Durante a sua vida dividida entre batalhas e feitos heroicos, Ragnar
Lodbrok casou-se com Laðgerða, ou Lagertha, chefe guerreira que se
distinguia na batalha pelo vigor e pelos cabelos soltos que a
identificavam na luta. Siward, avô do herói, morreu em batalha contra um
rei de nome Frø, que, vitorioso, humilhou as mulheres que pertenciam a
família do morto, fazendo delas prostitutas. Fugindo da humilhação,
algumas mulheres pegaram em armas e foram vitais no momento em que
Lodbrok trouxe os seus guerreiros para vingar o avô em campo de batalha,
contra o rei Frø.

Essa guerreira (ou Dama de Escudo, como normalmente são chamadas as
mulheres belicosas nas fontes mitológicas escandinavas) propunha um
teste aos seus pretendentes: lutar contra duas feras que serviam
como suas guardiãs. Lodbrok vence o combate contra um urso e um cão, passando a
viver feliz, por algum tempo, com a guerreira. Apesar disso, enraivecido
por ter sido acossado pelas bestas, Ragnar abandona a mulher com quem
teve duas filhas (de nome não registrados na escrita) e um filho chamado
Fridleif.

Após realizar os feitos que lhe renderam o casamento com Thora
Borgarhjort, Ragnar teve dois filhos de nomes Agnar e Érico. Infelizmente,
Thora adoeceu e morreu, restando ao herói prantear o falecimento de sua
jovem esposa no mar. Após a morte da esposa, Lodbrok passou mais uma vez a praticar o
saque onde pudesse. Tornou a casar, quando propôs um desafio a Aslaug de
vir ao seu encontro vestida e não vestida, comendo e não comendo,
sozinha e não sozinha. Aslaug compareceu ao encontro vestida em uma rede de
pesca, comendo uma cebola e trazendo um cão.

Aslaug, também chamada Kráka, era a filha dos heróis lendários Sigurð, o
Matador de Dragão, e Brynhilda, com quem ele teve boa parte
dos seus filhos mais famosos: Bjórn Flanco de Ferro, Ivar Sem-Ossos,
Camisa-Branca, Ragnvald, Sigurð Serpente no Olho, Hingwar (possivelmente
outro nome para Ivar), Halfdan e Ubba, além de três filhas que
costuraram o estandarte do corvo em um único dia. Tais filhos vão ser
citados ou não, dependendo da fonte consultada. Mas, enquanto os
detalhes das vidas desses homens são produtos das narrativas lendárias --
nas quais pelo menos Ivar Sem Ossos (Hingwar), Halfdan e Ubba serão citados
como líderes da Grande Armada Danesa --, de modo geral, admite-se que os
filhos de Ragnar Lodbrok possivelmente teriam sido figuras históricas.
Aslaug tinha a habilidade de falar com os pássaros, como o pai. A
serpente no olho de Sigurð, um de seus filhos, apenas reforça a ideia de sua
origem nobre.

Já em idade avançada, Ragnar começou a ansiar novamente por aventuras e,
temendo que os feitos dos seus filhos ultrapassassem os seus, preparou
suas embarcações. As notícias acerca de suas intenções chegaram aos homens de posses, 
que passaram a proteger os seus tesouros, temendo o saque viking. Aslaug lhe
alertou que a empreitada estava destinada ao fracasso e, ao ouvir do
marido que atacaria com dois navios a fim de aumentar a grandeza de seus
feitos, lhe entregou uma cota de malha encantada com a finalidade de
oferecer proteção.

Os dois navios foram destroçados pela tempestade quando se aproximavam da
Inglaterra e Ragnar, junto aos poucos homens de seu bando que
sobreviveram, passaram a saquear o território a pé, parados apenas
pelo rei Ӕlla, na Nortúmbria. Capturado vivo e se recusando a dizer a sua
verdadeira identidade, Ragnar foi morto em um poço de serpentes. As
notícias chegaram aos seus filhos, que prometem vingança. Esta se realiza
sob a forma da ``Águia de Sangue'', uma execução lendária e sem
evidências que tenha de fato existido.

Ragnar foi jogado em um poço com serpentes pelo rei Ӕlla e morreu quando 
retiraram à força sua armadura, sendo posteriormente vingado pelos seus
filhos. A tradição de heróis que encontram o mesmo fim também conta com
Gunnar, morto por Atli. Sua figura gravada em madeira -- na qual aparece 
tocando harpa com os pés, como descrito no poema ``Os Ditos de Atli'' (\emph{Atlamál}) -- 
pode ser encontrada na Noruega a partir do século~\versal{XII}. De
maneira geral, nas representações funerárias ou memoriais da Era Viking,
é possível encontrar figuras antropomórficas presas por serpentes ou
seres serpentiformes. Ainda que essas recorrências pareçam estar
interligadas, não é possível tecer grandes conjecturas. De Vries, por
exemplo, presume que a ligação entre a morte de Ragnar e as serpentes
pode ter surgido nos assentamentos dinamarqueses no norte da Inglaterra.
Também aponta que as representações de homens e serpentes podem ter servido como
prova da história de Ragnar, mas essa é uma posição difícil de ser
sustentada pelos pesquisadores atualmente.

Não poderíamos deixar de falar sobre a importância de Ragnar Lodbrok na
cultura moderna, com ênfase na exposição da sua figura através do
seriado \emph{Vikings}, produzido por Michael Hirst e exibido através do
canal de televisão History. O seriado se inspira brevemente nos
relatos mitológicos, contemplando as fontes apontadas aqui anteriormente
e posicionando o herói no epicentro de importantes eventos da Era Viking,
como o saque a Lindisfarne, o cerco a Paris e o acossamento dos centros
de poderes das Ilhas Britânicas.

Sem maiores comprometimentos com a fidedignidade histórica, o seriado
prefere representar um Ragnar crível, que, mesmo distante dos elementos
mitológicos originais, possa agradar ao público com elementos
familiares ao imaginário formado sobre os vikings nos últimos séculos.
Os séculos que separam os acontecimentos anteriormente citados e
outros pontos centrais para a narrativa do seriado, como o
estabelecimento do ducado da Normandia por Rollo (Hrólf Ganger), o
reinado de Ecbert, entre outros, nos servem como um lembrete da
discrepância entre as fontes históricas e o roteiro elaborado para os
episódios do seriado.

Os produtores da série também apostam na identificação dos espectadores
com uma pretensa atmosfera religiosa pagã, em oposição ao universo
cristão. A figura de Ragnar Lodbrok, interpretado por Travis Fimmel, nos
é apresentada recitando versos do \emph{Hávamál} e frequentando o templo
de Uppsala, situado no alto de uma montanha, dentro de um bosque (quando, em verdade,
as descrições e vestígios referem-se a uma planície). A representação do templo apresenta clara
inspiração nas igrejas de madeira escandinavas. Cabe mencionar ainda a música
moderna como trilha sonora, que, todavia, soa estranhamente familiar ao que o público conecta
como algo ``viking''.

Se funciona essa ponte entre o anacronismo pertinente aos
produtores da série e as expectativas criadas pelo público que espera
por algo ``genuinamente viking'', isso se dá na longa elaboração do herói como
parte do modelo de nobre pagão na cultura medieval, ou de bárbaro
romantizado na literatura vitoriana. Nada mais chocante que a cena na qual
o ritual da Águia de Sangue é encenado: as costelas do \emph{jarl} Borg
são retiradas, seus órgãos internos expostos e, enquanto nenhum gemido
foi proferido, a entrada ao Valhalla foi garantida pelo suplício. Uma
punição celebrada dentro de uma pretensa ideia de ``legitimidade
viking'', mas só mencionada tardiamente em fontes de larga inspiração
mitológica e sem nenhuma base arqueológica. Dessa maneira, o Ragnar
Lodbrok de Michael Hirst se afasta dos clichês mais grosseiros sobre o
mundo viking, mas não consegue se distanciar dos anseios e das
pretensões do público ao qual se destina, que elege quais elementos
aprova como pertinente ao perfil desse personagem.

\SIG{Pablo Gomes de Miranda}

Ver também Dinamarca da Era Viking; Viking; Vikings na televisão.

\begin{itemize}
\item \versal{DE VRIES}, Jan. Die Westnordische Tradition Der Sage Von Ragnar Lodbrok.
\emph{Zeitschrift Für Deutsche Philologie}, n. 53, 1928, pp. 257-302.

\item \versal{KACANI}, Ryal Hall. \emph{Ragnar Lothbrok and the semi-legendary history
of Denmark}. Senior Thesis, Brandeis University, 2015.

\item \versal{MAWER}, Allen. Ragnar Lóthbrok and His Sons. \emph{Saga Book of the
Viking Club}, vol. 6, 1909, pp. 68-89.

\item  \versal{McTURK}, Rory. \emph{Studies in Ragnar's Saga Lodbrokar}. Oxford: Society
for the Study of Medieval Languages and Literature, 1991.

\item \versal{SAWYER}, Peter H. \emph{Kings and Vikings: Scandinavia and Europe \versal{AD}
700"-1100}. London: Methuen, 1982.

\item \versal{TROMANS}, Dominic. The Making of a Legend: The Saga of Ragnar Lothbrock
and the \versal{TV} series Vikings. \emph{Academia.edu}, 2015, pp. 01-13.
\end{itemize}
\section{\versal{REALEZA} }

Dados coletados pela arqueologia -- em especial por conta das descobertas
de túmulos, de evidências nas inscrições rúnicas e relatos de fontes
escritas -- mostram a estratificação da sociedade escandinava na Era
Viking. Antes desse período, o poder já se concentrava nas mãos de
famílias dinásticas. A riqueza se dava principalmente pela posse de
terras e pelos produtos dela extraídos, os quais se convertiam em
impostos a seus senhores. Existia uma hierarquia social na qual os pequenos
reinos dessas regiões eram controladas pelas aristocracias locais que
tomavam decisões baseadas em conselhos (ou \emph{Things}). 
Com o passar do tempo e a centralização do poder
real na Escandinávia viking, os tributos passaram a ser recolhidos pelos 
representantes dos reis, aos quais eram conferidos grandes
poderes (uma característica da formação monárquica). As
evoluções das monarquias ocorrem aproximadamente no século~\versal{XI},
quando se verifica também mais evidências da produção de moedas pelos
reis nórdicos.

Assim, antes da Era Viking havia vários reinos na Escandinávia, embora
não se saiba com exatidão a extensão desses reinos. As informações sobre
os processos de unificação dos territórios nos países escandinavos se
encontram hoje limitadas. Sabe-se que, provavelmente, regiões da Dinamarca,
até o ano 800 d.C., já estariam sujeitas a um único rei. O mesmo deve ter ocorrido na Noruega,
no fim do século \versal{IX}. A maior parte do território atual da
Suécia foi unificada antes do século~\versal{XII}.

A monarquia era hereditária, porém a sucessão não era algo garantido,
pois havia muitas disputas entre os descendentes. Ocorriam muitos
casos de pretendentes ao reinado que passavam por longos períodos no
exílio acumulando riquezas e séquito militar, para enfim retornarem às suas
terras e conquistarem o direito monárquico. 
Por vezes, também, se firmavam acordos que visavam um governo
conjunto entre mais de um monarca ou mesmo a divisão do território em
regiões nas quais cada pretendente exerceria o domínio como senhor. Para
obter o comando real, um indivíduo não precisaria necessariamente estar
conectado a uma área geográfica em particular. Existiram muitos
líderes guerreiros vikings em expedições que carregavam o título de rei
sem possuir nenhuma base de poder territorial estabelecido em suas
terras natais.

O poder também estava longe de ser centralizado nos reinos escandinavos
vikings. Havia um grau de relativa independência das regiões que
compunham os reinos, que mantinham seus próprios costumes e
leis. As aristocracias locais possuíam grande influência, apesar desta 
ter diminuído com o advento das administrações reais em uma
maior escala. Ainda assim, a manutenção do domínio real dependia muito
de uma boa interação com os aristocratas. As monarquias escandinavas da
Era Viking tinham uma conexão muito intrínseca com a religião e os
mitos. A guerra possuía um papel importantíssimo na construção do
reinado, visto o forte caráter militar que permeava as alianças entre os
reis e a classe aristocrata. A guerra e a obtenção de saques eram de
suma importância na obtenção de recursos, bem como na
distribuição dos produtos no interior da sociedade. A generosidade do rei conforme este
distribuía os produtos para a aristocracia era um elemento chave na
manutenção das alianças. Os reis distribuíam bens e poder. Seus aliados aristocratas deveriam em troca fornecer homens
para a guerra.

\SIG{Fábio Baldez Silva}

Ver também Canuto, o Grande; Genealogia; Hird; Sociedade; Viking.

\begin{itemize}
\item \versal{BRINK}, Stefan. Law and society: Polities and legal customs in Viking
Scandinavia. In: \versal{BRINK}, Stefan; \versal{PRICE}, Neil. (eds.). \emph{The
Viking world}. New York: Routledge, 2008, pp. 23-31.

\item \versal{GRAHAM-CAMPBELL}, James. \emph{Os Viquingues: Origens da Cultura
Escandinava}. Madrid: Del Prado, 1997 (vol. 1).

\item \versal{ROESDAHL}, Else. \emph{The Vikings}. London: Penguin Books, 1998.

\item \versal{STEINSLAND}, Gro. Ideology and power in the Viking and Middle Ages
Scandinavia, Iceland, Ireland, Orkney and the Faeroes. In:
\versal{STEINSLAND}, Gro; \versal{SIGURDSSON}, Jón \versal{V}; \versal{REKDAL}, Jan \versal{E}; \versal{BEUERMANN}, Ian
(orgs.). \emph{Ideology and power in the Viking and Middle Ages}.
Boston: Brill, 2011, pp. 01-14.
\end{itemize}
\section{\versal{REGIÕES E PERÍODOS HISTÓRICOS}}

Ver Arqueologia da Era Viking; Dinamarca da Era Viking; Era Viking;
Expansão nórdica; Finlândia da Era Viking; França na Era Viking;
Inglaterra da Era Viking; Irlanda da Era Viking; Islândia da Era Viking;
Noruega da Era Viking; Reino da Dinamarca; Rússia da Era Viking; Suécia
da Era Viking.

\section{\versal{RELIGIÃO}}

\emph{Conceito}: O termo religião é um conceito atualmente cercado de
debates conceituais e teóricos, sem um consenso uniforme na academia.
Suas raízes no mundo romano clássico e cristão -- no sentido de culto e
reverência aos deuses --, chegando ao racionalismo iluminista, o tornam 
um conceito eminentemente relacionado à história intelectual
europeia, adaptado e transferido para outras culturas e épocas. No mundo
nórdico pré-cristão, a exemplo de outras áreas culturais, não existia um
termo linguístico específico para religião. Os escandinavistas
contemporâneos utilizam majoritariamente o termo Religião Nórdica Antiga
para uma série de crenças, práticas e rituais mantidos durante a Era
Viking. O tradicional termo paganismo nórdico vem sendo criticado devido
a sua forte carga pejorativa nas fontes primárias, enquanto a utilização
do termo \emph{forn siðr} (costume) como substituto de religião pelos
acadêmicos atuais foi efêmera. Para o pesquisador Thomas DuBois, a
religião durante o período pré-cristão era de natureza étnica e teve um
grande dinamismo, com variações sociais e regionais, bem como múltiplas trocas e
intercâmbios com culturas não escandinavas. Ainda segundo ele, existiram
comunidades de fé baseadas tanto em culturas quanto em instituições
sociais, sendo a experiência religiosa nórdica estruturada como uma
visão de mundo -- classificando seres, paisagens e situações. Outros
acadêmicos vêm percebendo a Religião Nórdica Antiga como fortemente
integrada na vida social, econômica e política do
Período Viking. Para o arqueólogo Mike Parker Pearson, a Religião
Nórdica Antiga pode ser percebida por duas perspectivas: numa perspectiva
continental (no contexto do ano mil d.C.), onde ela imita muitos dos
aspectos dos rituais da Idade do Ferro; e em outra perspectiva,
pré-histórica, na qual estas práticas podem ser percebidas como o produto final
de uma variação regional de um pan-paganismo, cuja prática se estendia
das ilhas britânicas até a Alemanha e Escandinávia.

\emph{Fontes}: As fontes primárias para o estudo da Religião Nórdica
Antiga são literárias (as Eddas, as sagas islandesas, as crônicas
históricas e de colonização, crônicas estrangeiras e a poesia escáldica)
e arqueológicas (monumentos, pedras rúnicas, textos rúnicos e latinos,
esculturas, cenas de tapeçaria, depósitos funerários e cemitérios). Em
especial, as sagas islandesas contêm diversas referências à
religiosidade nórdica antiga, incluindo rituais, profecias, atos de
piedade, práticas e crenças mágicas, maldições, referências a templos e
espaços sagrados, entre outros.

Em algumas pesquisas vêm sendo reempregados, para compensar certas
lacunas e escassez de fontes originais da área nórdica, alguns métodos comparativos
com religiões de outras culturas e áreas, como a finlandesa, a báltica,
a celta, a oriental e a euroasiática de forma geral.

\emph{Teologia e mitos}: A teologia nórdica envolvia uma série de
divindades, divididas em dois grupos (\emph{Aesir} e \emph{Vanir}).
Algumas divindades femininas (\emph{dísir}) ocupavam papel central no
mundo privado, enquanto forças sobrenaturais (\emph{álfar}) eram
seres inferiores em conexão com os Vanir. Outras categorias são os
\emph{jotnar} (gigantes) e os \emph{dvergar} (anões). As narrativas
mitológicas expressavam o complexo relacionamento entre deuses, gigantes
e homens. As comunidades acreditavam em vários deuses e seres
sobrenaturais, mas geralmente poucos ou somente uma única deidade
recebia maior atenção na esfera individual. A relação entre homem e
divindade variava de temor e medo a uma ligação de profunda amizade
(\emph{ástvínr}). Dentro do contexto de uma religião sem dogmas,
doutrinas, organização e centralização, os mitos são a principal
expressão religiosa de mundo, emoções, bem como de ideias e valores sobre a
natureza, as localidades divinas e o homem. Essas narrativas orais
acompanhavam os ritos e as dramatizações, além de serem incluídas na arte e
cultura material. Os mitos explicam as origens do universo e servem de
modelo tanto para os cultos quando para o comportamento dos indivíduos
nos grupos.

\emph{Autoridades religiosas}: Não existiam sacerdotes profissionais,
como os druidas dos povos celtas ou o sacerdócio hereditário dos
indo-iranianos. Funções ritualísticas de diversos tipos eram realizados
por pessoas de diferentes ocupações e papéis na sociedade. Reis e
líderes eram conhecidos por suas importantes ocupações em banquetes
sacrificiais públicos. Na Islândia, existia a instituição do \emph{goði},
um papel de liderança que combinava funções políticas, jurídicas e
religiosas. Outro tipo de personalidade que tinha algum tipo de função
religiosa era o \emph{Þulr} (orador). Nos atos de comunicações
com o além, tanto homens quanto mulheres desempenhavam papéis
importantes, mas as mulheres possuíam mais valor no tocante às previsões
do futuro. A \emph{völva} era a profetisa que frequentemente era
requisitada para atender situações reais de infortúnios e crises
sociais.

Reis (\emph{konungar}) e nobres (\emph{jarlar}) tinham performance central em
festivais cerimoniais e em santuários. Em \emph{Hákonar saga}, Snorri
descreve um ritual efetuado em Trøndelag, conduzido por Sigurd
Hlada-Jarl. Durante o cerimonial, Sigurd abençoou o fogo no local e brindou
aos deuses, invocando a fórmula \emph{Til árs ok friðar} (``Para uma boa
estação e a paz''). Essa ideia de que a autoridade real era fundamentada
em elementos religiosos foi denominada de realeza sagrada e vem sendo
questionada e debatida intensamente. Mais recentemente, a pesquisadora
Gro Steinsland vem demonstrando a ligação entre mito e rito nos papéis
de liderança política, com especial interesse na hierogamia entre Freyr e Gerd e
suas ressignificações na cultura material (plaquetas de ouro com imagens
destes personagens míticos foram encontrados em salões reais). A giganta
Gerd seria uma representação do território assumido pelo rei. Ela seria
a força primitiva, enquanto que o rei encarnaria o papel de fertilizador
e de mantenedor da ordem (como o deus Freyr).

\emph{Ritos e cultos}: O culto, no mundo pré-cristão, tinha um papel de
mediação e estruturação da unidade coletiva. Existiam diversos tipos de
rituais. Os banquetes sacrificiais (\emph{blótveizlur}) ocupavam um
papel proeminente nos grandes festivais sazonais, contando com a participação de
um grande número de pessoas. Os rituais familiares eram frequentemente
feitos em fazendas, a exemplo do \emph{álfablót}, sendo os ancestrais os
mais antigos temas de adoração nas regiões nórdicas. Outros tipos de
ritos eram frequentes em certas épocas da vida das pessoas e das
comunidades, como nascimentos, iniciações, casamentos e funerais. Uma
grande quantidade de vestígios de poços de cozinha, datados da Idade do
Bronze até o ano 1000 d.C. foram encontrados na Escandinávia,
utilizados em banquetes e rituais tanto de famílias quando da comunidade
de forma mais ampla. Na área islandesa eram frequentes a utilização do
consumo de carne de cavalo. Outros tipos de atividades com implicações
ritualizadas eram lutas entre guerreiros, salto sobre fogo, hóquei,
natação, arremesso de objetos, andar sobre remos, competições de canto,
danças com máscaras etc.

As pesquisas arqueológicas também encontraram uma grande quantidade de
vestígios de rituais relacionados com a construção ou inauguração de
habitações, bem como com construções variadas na Escandinávia do Período Viking. Os rituais
consistiram na consagração com bebidas e o depósito dos vasilhames
(potes, vasos e copos) em áreas com fogo (aqui identificado com o seu
poder sobrenatural e transformador). As áreas mais ricas desse tipo de
material são o sul da Suécia e o norte da Dinamarca, algumas delas
associadas com depósitos de cerâmica e construções.

Analisando diversos tipos de vestígios materiais obtidos na tradição
ritualística nórdica, Anne Carlie considerou diversas mudanças
diacrônicas: inicialmente, na Idade do Bronze, temos depósitos agrários,
com diversos machados, cerâmicas e ossos animais. Posteriormente, no
período das migrações (séculos~\versal{VI} a \versal{VIII} d.C.), surgem depósitos não
agrários, com vestígios de armamentos e sepultamentos. No início da Era
Viking até o final do medievo, aumentam os depósitos mágicos: objetos
antigos (como machados neolíticos e fósseis) são encontrados junto a
ossos animais e humanos, moedas e pingentes/amuletos com símbolos
odínicos (aves e serpentes).

Os rituais funerários são muito pesquisados atualmente pelos
escandinavistas, devido à sua riqueza de material arqueológico. Eles são
expressões de atividades específicas (guerra, negociações, caça, atrações pessoais),
funcionando também como elementos de identidade social em uma rede de relações
híbridas. Muitos são conectados diretamente aos cultos odínicos, a
exemplo do funeral descrito por Idn Fadlan. Outro cronista árabe, Ibn
Rustah, também mencionou uma elaborada câmara sepulcral de um líder
nórdico da Rússia, com depósitos de comida, bebidas, vasilhames e
moedas. Segundo esse cronista, a esposa do chefe foi colocada viva
dentro da sepultura. Para o arqueólogo britânico Neil Price, os funerais
nórdicos não consistiam apenas de rituais, mas também de performances e
dramatizações de narrativas míticas. Esses atos passavam ao público
presente várias mensagens de cunho social e religioso. Mesmo os animais
presentes -- geralmente sacrificados e dispostos no local -- executam
papéis em um drama funerário. Embora os atores não estivessem presentes
na cena final, desempenham o papel principal: o de confirmarem a
sepultura como uma moradia.

\emph{Espaço ritual}: Locais de culto envolviam diversos tipos de sítios
naturais, como montanhas, arvoredos, campinas, ilhas, lagos e rios.
Nesses locais, diferentes tipos de construções e monumentos tinham
intenções religiosas: alinhamentos de pedras em forma de embarcações ou
círculos; pedras erigidas com runas; lareiras para propósitos cultuais.
Sacrifícios, invocações, bênçãos e ações de graças eram efetuadas em
lugares reservados para tais fins. A única palavra em nórdico antigo
que possuía uma inequívoca denotação de local de culto é \emph{vé}, mas
também existiam outras com sentido semelhante, como \emph{lundr},
\emph{akr} e \emph{hof}.

\emph{Templos e locais sagrados}: A famosa descrição do templo pagão de
Uppsala (Suécia), realizada por Adão de Bremen, conteve diversos
referenciais cristãos. Escavações arqueológicas no local indicaram
que não existiu um grande templo como descrito pelo cronista, mas um
grande salão real, utilizado para fins cerimoniais. Locais muito
semelhantes foram também encontrados em Mære (Noruega), Järrestad
(Suécia) e Helgö (Suécia). Apesar de tradicionalmente os pesquisadores
argumentarem que não existiram templos ou construções especializadas
para fins rituais na Era Viking, recentes descobertas vêm demonstrando
que, além dos praticados ao ar livre, cultos ocorriam também nesses
espaços fechados. Em Borg (Noruega), uma pequena casa foi encontrada,
situada numa elevação rochosa, construída com soleiras e paredes de
madeira. Ela foi erigida junto a um jardim pavimentado, cobrindo uma
área de cerca de 1.000 m². A redor dessa construção foram encontrados
inúmeros ossos de animais (cachorros, cavalos e javalis) e também
amuletos circulares de metal, ligados com pingentes do martelo do deus
Thor.

Recentemente outro templo foi escavado, no sítio de Uppåkra (Suéca),
entre os anos de 2000 e 2004. As pesquisas revelaram uma construção
utilizada entre os séculos~\versal{VI} e \versal{X}, com forma e estrutura muito
semelhante às posteriores igrejas de aduelas da Noruega do período
cristão. No local, foram encontrados vestígios de bracteatas,
\emph{gullgubber} (placas de ouro com representações humanas,
associadas à hierogamia entre o deus Freyr e a giganta Gerd), e diversos
vestígios de armamentos quebrados, como escudos e pontas de lança. Essa
última prática também foi verificada nos pântanos dinamarqueses da Idade
do Ferro, indicando uma continuidade de práticas religiosas na
Escandinávia. Mas os achados mais espetaculares do templo foram um copo
de prata e bronze com ornamentos de ouro e uma sofisticada bacia de
vidro.

Vestígios de sítios rituais ao ar livre foram encontrados em Frösön
(Suécia). Entre os ossos de animais domésticos e selvagens, foram
achados vários remanescentes de ursos -- um animal associado diretamente
à marcialidade e ao deus Odin, o que levou os pesquisadores a
acreditarem que se tratava de um sítio sacrificial, possivelmente uma
sepultura. Os corpos foram depositados junto a árvores que cresciam na
época dos sacrifícios, o que conduz a uma comparação direta com o relato
de Adão de Bremen e a Tapeçaria de Oseberg, que apresenta imagens de
enforcados numa grande árvore.

Diversos tipos de pesquisas vêm apontando implicações cosmológicas em
estruturas e localidades nórdicas. Para Lars Larsson, o templo de Uppåkra
foi construído representando aspectos cósmicos e sociais: seus depósitos
de pontas de lanças invocava o salão do Valhalla (repleto de guerreiros
renascidos), enquanto os postes centrais são uma alusão à árvore
Yggdrasill. Um desses postes foi recoberto com figuras de ouro,
conectando simbolicamente o local com o bosque de Glasir e o Valhalla.
Outro local, a fortificação de Ismantorp (Suécia), foi identificado
pelo arqueólogo Andres Andrén como também possuindo conotações
cosmológicas: seus nove portões seria uma alusão ao numero sagrado de
Odin e aos nove mundos, enquanto o poste central teria sido uma
referência à Yggdrassill. Tendo um caráter de legitimação de uma nova
ordem militar no final do período de migrações, Ismantorp também foi
integrada à uma nova concepção de liderança política, que se utilizou de
referenciais mitológicos e religiosos.

A arqueóloga Lotte Hedeager também acredita em implicações cosmológicas
na estrutura arquitetônica do sítio de Gudme (Dinamarca): ele
teria sido um modelo paradigmático de Asgard, ou seja, um centro de
culto ao deus Odin. Sua arquitetura seria baseada (em termos de
imaginário artístico) ao que se acreditava ser a moradia dos
deuses. Do mesmo modo que o trono de Odin (Gladsheim), o trono do rei
ficava em uma posição central e mais elevada no centro de Gudme. Nesse
caso, o objeto também servia como suporte para a autoridade real.

\emph{Sacrifícios}: A forma mais comum de imolações na Escandinávia é a
utilização de animais domésticos e selvagens. Eles são vistos como um
meio de comunicação com alguma divindade e cada animal geralmente é
conectado a diferentes seres divinos. Assim, bois, porcos e javalis são
sacrificados para Freyr e Freyja, enquanto cabras e bodes para Thor. Os porcos
simbolizavam o poder bem-sucedido das reproduções e da fertilidade em
geral (mesmo as humanas). Já o bode representava tanto o humor quanto a
imagem sagrada do sexo na imaginação pré-cristã. Outros animais
sacrificados são imediatamente consumidos em banquetes, como os cavalos
na Islândia (conectados a Odin e ao mundo da nobreza). Esse ato possuía
um significado de ratificação de leis ou funcionava como agregador das relações
entre as comunidades. Outro animal odínico é o urso, relacionado aos
ancestrais míticos dos clãs e identificado à bravura dos guerreiros,
os quais se transformam no animal em diversas sagas, demonstrando a
popularidade do urso enquanto ser portando poder animista.

A presença de sacrifícios humanos, especialmente associados a funerais,
não é fácil de ser detectada diretamente, mas, apesar disso, diversos
vestígios foram encontrados e são debatidos pelos especialistas. Um
grande número de sepulturas da Era Viking contém indivíduos que foram
claramente decapitados, esfaqueados ou enforcados com as mãos amarradas.
Exemplos famosos são o homem enterrado em Stengade (Dinamarca), abaixo
de outro corpo masculino coberto com uma pesada lança. Outro exemplo é uma
sepultura de Birka (Suécia), com o corpo de um jovem ao lado de um homem
de idade avançada também coberto com lanças. As fontes clássicas (como
Tácito e Diododo da Sicília), já descreviam sacrifícios de guerreiros,
capturados pelos antigos germanos e dedicados a Mercúrio (possivelmente
Wotan). Os sacrifícios eram realizados pelo trespassar de uma lança ou enforcamento, ambos
relacionados com o auto-sacrifício do próprio Odin, que utilizou esses
dois métodos conjugados. Em outro sítio, Borg (Suécia), foram
encontrados depósitos junto ao grande salão, datados da Era Viking, com
vestígios de ossos animais, incluindo dez cachorros decapitados. O
escavador do local, A. Nielsen, concluiu que se tratava de um templo
dedicado a Freyr ou Freyja. Em Trelleborg (Dinamarca), um fosso revelou
ossos de crianças junto a porcos, vacas, cabras e cachorros. Em Repton
(Inglaterra), ao redor do chefe morto, foram encontrados ossos de
vítimas jovens. A fonte mais famosa sobre sacrifícios em funerais é a do
árabe Ibn Fadlan, que relatou uma imolação durante os funerais de um
líder dos nórdicos situados na área do Volga, durante o século~\versal{X} d.C.

Outros tipos de sacrifícios humanos detectados pela arqueologia são
diversas imolações encontradas em pântanos escandinavos. Os sacrifícios são datados da
Idade do Ferro e diversos especialistas acreditam que estavam
conectados com o culto do deus Wotan/Odin. Em crônicas históricas do
medievo, também surgem descrições semelhantes. No relato de Adão de
Bremen, a respeito de práticas religiosas pré-cristãs, menciona-se
sacrifício periódico no templo de Uppsala, Suécia, realizado a cada nove
anos e contendo nove representantes de cada espécie, incluindo seres
humanos, que eram enforcados em uma árvore situada ao lado do templo. No
\emph{Heimskringla}, Snorri Sturluson descreve o rei sueco Aun
sacrificando seus próprios filhos para aplacar um período de longa fome,
fato idêntico ao sacrifício do próprio rei Domaldi, descrito na mesma fonte.
Obviamente aqui temos algumas filtragens realizadas pelo cristianismo,
que mantinha um referencial moralista sobre estas antigas práticas.

\emph{Artefatos religiosos}: Alguns dos conhecidos objetos com intenções
supostamente religiosas são estatuetas antropomorfas de madeira e metal.
Um dos mais famosos exemplos é o objeto de bronze encontrado em
Rallinge, com 7 cm, representando uma figura masculina com pênis ereto
(geralmente interpretado como sendo o deus Freyr). Outra estatueta
(Eyraland) representa um homem portando barba e um martelo, identificado
com Thor. Algumas figuras de barba também são associadas a esse deus,
como as encontradas em Suécia, Islândia e Ucrânia. Por sua vez, o deus
Odin é identificado a outras estatuetas e esculturas sem um dos olhos,
como as de Lindby, Tisso e Uppakra.

As pessoas utilizavam amuletos contra doenças e perigos, bem como para proteção
diante das adversidades da vida. Muitas vezes esses objetos possuíam uma relação
direta com os poderes de alguma divindade. O martelo de Thor, por
exemplo, era um objeto comumente encontrado em sepulturas, fortificações
e locais sagrados. Ele continha significados mágicos e de proteção. Alguns
pingentes em formato de machado, feitos de âmbar, parecem ter sido
utilizados em ritos funerários com propósitos semelhantes. Miniaturas da
lança de Odin, Gungnir, são conhecidas de muitas localidades da Suécia.
Outros objetos, como representações de valquírias, também são atrelados ao
culto de Odin e foram encontrados em sepulturas de \emph{volvas}. Muitos
outros tipos de amuletos foram descobertos na Escandinávia da Era
Viking, como pingentes representando escudos com espirais, tronos e
serpentes. O primeiro possuiria ligação com o culto ao Sol e a
fertilidade, enquanto o segundo pode estar relacionado tanto a Thor como
a Odin (ambos possuem tronos). A serpente é um dos símbolos religiosos
mais difundidos entre os povos indo-europeus e entre os nórdicos possuía
vários significados, entre os quais o renascimento e a vida, sem esquecer de sua
relação com o xamanismo de Odin.

\emph{Religiosidade popular}: Algumas expressões da fé nórdica separam
claramente a crença em seres superiores (os deuses e deusas) dos seres
conectados com o mundo rural, as regiões provincianas e os espíritos da
terra. Desse modo, algumas dessas expressões aproximam-se do folclore,
sendo especialmente relacionadas com os elfos, gigantes, anões e \emph{trolls}.

De acordo com a visão corrente sobre destino no mundo nórdico, cada
indivíduo e família recebia certa quantidade de sorte, em termos
tanto materiais quanto abstratos. As ideias de sorte e azar eram usadas para
explicar as situações correntes, as estratificações sociais e para
entender porque uma família era mais rica do que outra. Sorte era
considerada um fato certo da vida, mas às vezes se tentava obtê-la por
meio mágico ou de encantamentos.

\emph{Influências religiosas externas}: Diversas pesquisas apontam
influências estrangeiras na religião nórdica, de um período que remonta
ao início das migrações germânicas até o final da Era Viking. Segundo
Hilda Davidson, Anders Kalliff e Olof Sundqvist, o culto ao deus Odin
sofreu assimilações do culto oriental de Mitra, que penetrou na área
germânica com a expansão dos exércitos romanos. Ambos possuem uma
estreita ligação com alguns animais, como corvo, cachorro e serpentes,
além de estreita relação com a ideologia militar e aspectos da
morte. O motivo iconográfico da morte de um touro pelo deus, inexistente
na Era Viking e central ao mitraismo, tem sido identificado pelos
pesquisadores como tendo ocorrido no período das migrações e surgido,
supostamente, em bracteatas, nas quais algumas representações portam um
touro junto a suásticas e uma figura masculina com corvos e armas
(Wotan/Odin).

Também existem evidências de influências da área céltica, especialmente
da Irlanda, onde encontramos símbolos e narrativas que foram adicionadas à
oralidade e iconografia nórdica. Os especialistas tradicionalmente
acreditavam que o mundo escandinavo tinha influenciado os povos sámi e
finlandeses, mas, atualmente, se percebe que houve, assim como na área báltica, 
trocas culturais e religiosas entre ambos, num movimento de circularidade 
frequente. Outra influência, especialmente forte no
final da Era Viking, são advindas do cristianismo. Diversos acadêmicos
acreditam que alguns indícios fortemente cristãos presente nas fontes
literárias como a Edda Poética não tenham sido criados no momento que
as narrativas foram preservadas por escrito. Teriam, porém, penetrado na
oralidade pagã ainda quando esta era atuante e cercada de pessoas
convertidas, o que se denomina hoje de \emph{interpretativo norroena}.

\SIG{Johnni Langer}

\SIG{Ver também Folclore; Genealogia; Guerra e simbolismos.}

\begin{itemize}
\item \versal{CARLIE}, Anne. Ancient building cults. In: \versal{ANDRÉN}, Anders; \versal{JENNBERT},
Kristina; \versal{RAUDVERE}, Catharina (eds.). \emph{Old Norse religion in
long-term perspectives}. Lund: Nordic Academic Press, 2006, pp. 206-211.

\item \versal{DAVIDSON}, \emph{The lost beliefs of Northern Europe}. London: Routledge,
2001.

\item \versal{DUBOIS}, Thomas. \emph{Nordic religios in the Viking Age}. Philadelphia:
University of Pennsylvania Press, 1999.

\item \versal{GRASLUND}, Anne-Sofie. The material culture of Old Norse religion. In:
 \versal{BRINK}, Stefan; \versal{PRICE}, Neil (eds.). \emph{The viking World}. London:
Routledge, 2008, pp. 249-256.

\item \versal{HEDEAGER}, Lotte. \emph{Iron age myth and materiality}. London:
Routledge, 2011.

\item \versal{LANGER}, Johnni. A religião Nórdica Antiga: conceitos e métodos de
pesquisa. \emph{Rever}, vol. 16, 2016, pp. 118-143.

\item \versal{LANGER}, Johnni. Paganismo nórdico. In: \versal{LANGER}, Johnni (org.).
\emph{Dicionário de Mitologia Nórdica}. São Paulo: Hedra, 2015, pp.
357-361.

\item \versal{LANGER}, Johnni. \emph{Fé Nórdica: mito e religião na Escandinávia
Medieval}. João Pessoa: Editora da \versal{UFPB}, 2015.

\item \versal{LANGER}, Johnni. \emph{Na trilha dos vikings: ensaios de religiosidade
nórdica}. João Pessoa: Editora da \versal{UFPB}, 2015.

\item \versal{RAUDVERE}, Catharina \& \versal{SCHJODT}, Peter (orgs.). \emph{More than
mythology: narratives, ritual practices and regional distribution in
pre-christian scandinavian religions}. Lund: Nordic Academic Press, 2012.
\end{itemize}
\section{\versal{RÖK STONE}}

Uma das mais famosas pedras rúnicas, apontada como a maior inscrição rúnica
pré-cristã já encontrada, contendo 760 caracteres. Localiza-se hoje ao
lado da igreja de Rök, em Östergötland, na atual Suécia. Considerada 
como um dos primeiros escritos literários suecos, marcando
o início da literatura no país, a pedra de Rök foi encontrada
na parede da igreja, que havia sido construída no século~\versal{XII}. A partir do
século~\versal{XIX}, os estudos passaram a anotar como prática comum do
período a utilização de antigas pedras rúnicas para a construção das antigas
igrejas. A pedra foi gravada por volta do século~\versal{IX}, fato indicado por
estudiosos ao constatarem a utilização do denominado alfabeto rúnico de ramo curto.
Ela contém gravação em todas as suas faces, com exceção da sua base, que foi
utilizada como apoio e se encontrava, por conseguinte, enterrada no
solo. Os escritos encontram-se preservados até os dias de hoje, com
exceção de poucas partes danificadas, que, todavia, não prejudicam sua leitura,
compreensão e tradução.

A pedra rúnica de Rök é considerada única devido ao conteúdo de sua
inscrição, que se refere ao rei Ostrogodo, imperador da Roma ocidental,
Teodorico, o Grande. Refere-se também à valquíria Gunnr (termo que pode ser traduzido
como batalha) e ao deus do trovão, Thor. Contudo, muito do que se
encontra nas inscrições é de difícil compreensão em função de fatores
como a utilização de Kennings (recurso de figuração poética muito comum
dos poemas antigos escandinavos). Historiadores, como Lars
Lönnroth, dividiram as inscrições em três partes de tamanhos praticamente
equivalentes. Tais partes seriam marcadas por conterem duas questões em cada uma, 
bem como uma resposta para
cada duas questões. Essa forma se assemelha a da denominada Greppamini,
constituindo espécie de jogo de adivinhar poético, muito utilizado na Edda em prosa.

\SIG{Munir Lutfe Ayoub}

Ver também Arqueologia da Era Viking; Suécia da Era Viking.

\begin{itemize}
\item \versal{ANDRÉN}, Anders; \versal{JENNBERT}, Kristina; \versal{RAUDVERE}, Catharina. Old Norse
Religion Some problems and prospects. In: \versal{ANDRÉN}, Anders; \versal{JENNBERT},
Kristina; \versal{RAUDVERE}, Catharina (eds.). \emph{Old Norse Religion in
long-term perspectives: origin, changes, and interactions}. Lund: Nordic
Academic Press, 2006, pp. 11-15.

\item \versal{KORTLANDT}, Frederik \emph{et al}. Early Runic consonants and the origin
of the younger futhark.~\emph{\versal{NOWELE}: North-Western European Language
Evolution}, vol. 43, 2003, p. 6.

\item \versal{LIESTØL}, Aslak. The emergence of the Viking runes.~\emph{Michigan
Germanic Studies}, vol. 7, 1981, pp. 107-116.

\item \versal{SAMPLONIUS}, Kees. Rex non reditvrvs. Notes on Theodoric and the
Rök-Stone.~\emph{Amsterdamer Beiträge zur älteren Germanistik}, vol. 37,
1993, p. 21.
\end{itemize}
\section{\versal{ROLLO}}

Foi um notório chefe nórdico que conquistou grande prestígio na França
do século~\versal{X}, tornando-se um nobre do Império Franco. Nascido com o nome
de Hrólf Röngvaldsson, mas também chamado de Göngu-Hrólfr em alguns
relatos, acabou se tornando mais conhecido em vida e na história pelo
apelido de Rollo (``Andarilho''), possível latinização de Hrólf.

O epíteto de Andarilho é incerto. Alguns historiadores assinalam, com
base em algumas sagas que mencionam várias viagens de Rollo, que seu apelido
adviria dessa sua condição de viajante. Por outro lado, a
\emph{Göngu-Hrólfr Saga}, manuscrito de autoria anônima e datado do
século~\versal{XIV}, assinala que o apelido de Andarilho se devia ao fato de
Rollo ser descrito como sendo um homem grande e robusto, de modo que não
haveria cavalo que conseguisse carregá-lo, razão pela qual tinha que viajar a pé. Essa
saga também narra as aventuras de Rollo na Escandinávia, Inglaterra e
Rússia.

A origem de Rollo não é exata, mas se sabe que os franceses se referiam a ele como
sendo dinamarquês. Não obstante, de acordo com o \emph{Heimskringla}
(\versal{XIII}), Rollo seria norueguês, filho do \emph{jarl} Röngvald Eysteinsson
de Møre (também conhecido como Röngvald, o Sábio), que abandonou
a Noruega por volta da década de 870 devido a desavenças com o rei
Haroldo Cabelos Belos (c. 850-932). Rollo, que já era nascido nesse tempo,
acompanhou o pai e a família no exílio. Credita-se a Röngvald a
colonização das Ilhas Shetland e Orkney, ao norte da Escócia.

Apesar de ter nascido no século~\versal{IX}, Rollo somente começou a se destacar
na história no século seguinte. Os relatos da
\emph{Historia Normannorum} (História Normanda) -- crônica redigida no
século~\versal{XI}, pelo monge de Vermandois, Dudo de St. Quentin (c. 960-1043?) --
versam, nos livros 2, 3 e 4, sobre a vida de Rollo, desde seu exílio da
Dinamarca até seu governo como duque. Segundo a crônica \emph{Andarilho},
após ter partido em exílio com a família, Rollo teria viajado pela Inglaterra, 
Flandres (atualmente na Bélgica) para, finalmente, se estabelecer na
França (\emph{Francia}), onde já havia acampamentos nórdicos
permanentes.

Sabe-se que, nos anos de 885-886, os chefes Siegfried e Gorm lideraram
ataques à Paris e, na ocasião, o conde Odo (c. 852-898) esteve à frente
do comando do exército da cidade. No entanto, não se tem certeza se
Rollo teria participado dessa expedição ou se somente chegou à França anos
depois. De qualquer forma, sabe-se que, em data incerta, Rollo se mudou para a
França, passando alguns anos atuando em pilhagens e ataques no
reino franco. Todavia, no ano de 911, Rollo, que já se apresentava como
um chefe, fechou acordo com o rei Carlos~\versal{III}, o Simples (879-929).

A França, desde 799, era alvo de incursões vikings, as quais se acentuaram
após a década de 830. Em 845, Paris foi saqueada pela primeira vez. Na época
de Carlos, o Simples -- ainda que seus antecessores (Odo e Carlos, o Gordo)
tenham resistido às invasões vikings no rio Sena -- novos ataques 
continuavam a ocorrer não apenas nas terras
percorridas pelo Sena, mas também em outras áreas do império. Assim, por
motivos não totalmente definidos, Rollo e o rei Carlos fecharam um
acordo.

Segundo informam as crônicas francas, após a Batalha de Chartres,
ocorrida em 911, o chefe viking Rollon (nome que é grafado nas
fontes francas) assinou com o rei Carlos, o Simples, um acordo em St.
Clair-sur-Epte. Nos termos desse acordo, o rei franco cedia terras entre os rios
Epte, Risle, Bresle, Avre e Dives, na região da antiga Província de Ruão
(Rouen), hoje parte da Normandia. Ali, Rollo poderia instituir seu feudo,
sob o compromisso de defender o reino de novas invasões vikings. Além de
receber um feudo e a missão de defesa do reino, Rollo também se casou
com a princesa Gisla ou Gisela, filha mais velha do rei. A união não
gerou descendência.

De início, Rollo manteve sua palavra e defendeu aquelas terras. Mas, à
medida que ganhava cada vez mais notoriedade e poder, começou por
conta própria a conquistar os territórios vizinhos. Por mais que isso
tivesse irritado os nobres -- que se queixaram ao rei pela atitude daquele
normando (termo pelo qual os francos se referiam aos vikings) --, Carlos, o
Simples, dependia da proteção e contatos de Rollo para assegurar o
noroeste do reino. O \emph{Historia Normannorum} não justifica porque o
rei não repreendeu seu genro por tais atos.

Em 918, segundo consta um documento eclesiástico da época, Rollo foi
batizado, convertendo-se ao cristianismo e adotando o nome de Roberto.
Apesar de adotar o nome franco, Rollo ainda continuava a ser referido
pelo seu apelido nórdico. À medida que conquistava terras, adotou o
título de Conde de Ruão (\emph{Rúðujarl}), passando a distribuir terras
e riquezas para seus homens de confiança. Além disso, os topônimos do
noroeste da França trazem influências da língua nórdica, o que atesta a
colonização da região pelos vikings.

Apesar da toponímia conservar vestígios do idioma nórdico antigo,
os normandos acabaram, com o tempo, se convertendo ao cristianismo, bem como
adotando a língua e os costumes francos. Rollo detém o mérito de ter criado 
uma ``colônia'' nórdica
não por meio de uma invasão propriamente -- como nos casos das colônias nórdicas na Inglaterra e na Irlanda --, mas através de um acordo que não foi
desfeito pelo rei Carlos \versal{III} e nem pelos seus sucessores.

Em 924 ou 925, Rollo abdicou do governo de seu feudo em favor de seu
filho Guilherme Espada Longa (?-943). Os motivos para ter feito isso não
são conclusivos. De acordo com os relatos da \emph{Historia
Normannorum}, Guilherme era filho de Poppa de Bayeux, segunda esposa de
Rollo. Após a morte de Gisla, em data desconhecida, Rollo casou-se com
Poppa, filha de Berengário~\versal{II}, conde de Bayeux.

Rollo viveria até mais ou menos os anos de 930 ou 932. Através de seu
filho Guilherme Espada Longa, em 924 as terras de Bayeux, Exmes e Sées
foram anexadas. Em 933, foi a vez de Cotentin e Avranchin. As dimensões
territoriais da Normandia foram estabelecidas por essa época. Todavia, o
território normando apenas se firmou politicamente no final do século~\versal{X},
com Ricardo~\versal{I}, da Normandia (933-996) -- neto de Rollo -- e cognado de
Ricardo, o Destemido (Ricardo Sans-Peur), este considerado o primeiro
legítimo duque da Normandia.

Na ficção, Rollo é personagem de algumas histórias interessantes. Uma das
mais antigas é a \emph{Göngu-Hrólfr Saga}, escrita no século~\versal{XIV}, na
Islândia, que conta a história de uma viagem de Rollo para a Rússia, até
a corte da Princesa Ingigerd. No século~\versal{XVII}, foi lançada uma peça
teatral inglesa, intitulada \emph{Rollo Duke of Normandy}. Também
conhecida como \emph{The Bloody Brother}, pois nessa história Rollo
teria assassinado seu irmão chamado Otto.

Mais recentemente, Rollo, interpretado pelo ator britânico Clive Standen,
tornou-se personagem recorrente na série \emph{Vikings}, criada por
Michael Hirst no ano de 2013. Nesse seriado, a história de Rollo foi
mesclada com a ficção. Ele se tornou irmão do lendário herói Ragnar
Lothbrok, além de ser um dos personagens que mostra mudanças em sua
condição social, pois inicia sua trajetória como simples irmão de um fazendeiro da Noruega
para se tornar Duque da Normandia na França (apesar de ele, historicamente,
ter sido conde, e não duque, propriamente).

Embora seja retratado como um homem alto, forte e bravo, Rollo não
possui fama. Nesse ponto, ele inveja seu irmão Ragnar por sua sagacidade,
intuição, determinação, por sua bela esposa Lagertha e seus filhos.
Tais condições serviram de motivo para que, nas duas primeiras temporadas, Rollo
entrasse em conflito com Ragnar, chegando a traí-lo. No entanto, o
personagem se aproxima de sua versão histórica a partir do final da
terceira temporada. Nesse momento, Ragnar decide comandar o primeiro
ataque à Paris e, ao término da temporada, Rollo decide se aliar aos francos, 
traindo novamente seu irmão e seu povo. Na temporada quatro, Rollo
é apresentado como senhor da Normandia, assumindo seu papel histórico,
apesar do anacronismo da série, pois o enredo se passa no
século~\versal{IX}, desconsiderando que, historicamente, Rollo se tornou um senhor
franco apenas em 911.

\SIG{Leandro Vilar Oliveira}

Ver também França na Era Viking; Normandia; Vikings na França.

\begin{itemize}
\item \versal{CHIBNALL}, Marjorie. \emph{The Normans}. Oxford: Blackwell Publishing,
2006.

\item \versal{HOLMAN}, Katherine. \emph{Historical dictionary of the vikings}. Lanham:
Scarecrow Press Inc, 2003.

\item \versal{LOGAN}, F. Donald. \emph{The Vikings in History}. London/New York:
Routledge, 1991.

\item \versal{NELSON}, Janet L. The Frankish Empire. In: \versal{SAWYER}, Peter (ed.). \emph{The
Oxford Illustrated History of the Vikings}. New York: Oxford University
Press, 1997, pp. 19-47.

\item \versal{PULSIANO}, Phillip; \versal{WOLF}, Kirsten (eds.). \emph{Medieval Scandinavia: an
encyclopedia}. New York/London: Garland Publishing, Inc. 1993.

\item \versal{SAN JOSÉ BELTRÁN}, Laia. Análisis histórico de la serie Vikingos de
History Channel. In: \emph{Los Vikingos en la Historia}, 2. \versal{HUM}-165:
Patrimônio, Cultura y Ciências Medievales. Universidad de Granada,
Granada, España, 2015, pp. 25-72.

\item \versal{RENAUD}, Jean. The Dutch of Normandy. In: \versal{BRINK}, Stefan; \versal{PRICE}, Neil
(eds.). \emph{The Viking World}. London/New York: Routledge, 2008, pp.
453-457.
\end{itemize}
\section{\versal{ROSKILDE}}

A escavação das cinco embarcações denominadas Skudelev foi o marco
inicial da arqueologia marítima na Dinamarca. Tais embarcações
encontravam-se naufragadas nas vias fluviais de Peberreden, na região de
Skudelev, a 20 km do Fjord de Roskilde, atual ilha de Zealand, na
Dinamarca. Em 1924, foi recuperada a quilha do que ficaria depois
conhecido como Skuldelev 1, achado reportado pelo Museu Nacional da
Dinamarca. Em 1956, mergulhadores recuperaram outro pedaço da embarcação,
que seria levada também ao museu. Tal fato culminou em um projeto
arqueológico de larga escala na região.

O bloqueio resultante do naufrágio das embarcações e de um
preenchimento de pedras -- que se situava aproximadamente no ponto intermediário da
profundidade de 40 km do Fjord, que corta a ilha de Zealand em direção
norte-sul --, é considerado estratégia de bloqueio de embarcações inimigas
que poderiam tentar atacar a cidade de Roskilde. Tal estratégia teve sua
primeira fase de desenvolvimento entre os anos 1070 e 1090, com o
naufrágio do Skuldelev 1 (grande cargueiro), do Skuldelev 3 (pequeno
cargueiro) e do Skuldelev 5 (embarcação de guerra de médio porte). A
segunda parte, que seria desenvolvida entre os anos 1100 e 1140 -- com
o objeto de reforçar o bloqueio --, foi realizada pelo naufrágio de uma
grande embarcação de guerra (que, inicialmente, foi confundida com duas
embarcações e, por esse motivo, denominada de Skuldelev 2/4) e pelo
Skuldelev 6, um pequeno cargueiro. Todas as embarcações tiveram suas
cordas e equipamentos retirados antes do naufrágio. A área havia sido
preenchida por pedras, de forma que as embarcações se encontravam escondidas.

O trabalho arqueológico teve início em 1957, quando se iniciou
um processo de observação e compreensão da área, que contou até mesmo
com mergulhos estratégicos nas partes em que a água se fazia mais profunda. O trabalho
incluía mapear a extensão da localidade coberta pelas pedras e,
finalmente, removê-las. Skuldelev 1 se fazia visível nos locais onde os
pescadores haviam penetrado a barreira de pedras para tornar o canal
navegável. Contudo, ainda maiores proporções da Skuldelev 1 viria à tona
graças às escavações.

Em 1958, o trabalho continuou. Foram encontradas a embarcação Skuldelev 3 -- a pequena
embarcação de cargo -- e o
madeiramento do que, naquele momento, acreditava-se ser da Skuldelev 4, 
depois percebido como sendo pertencente a Skuldelev 2, razão pela qual a embarcação
ficou conhecida como Skuldelev 2/4.

As escavações de 1959, por sua vez, revelariam ainda mais duas
embarcações, as Skuldelevs 5 e 6, que não vieram à tona por meio do trabalho dos
mergulhadores, como as demais, mas por um processo de rastreamento de
superfície que aumentava a eficiência do processo
arqueológico. Contudo, tornar-se-ia evidente, após certo tempo, que seria
impossível erguer tais embarcações a partir do trabalho de mergulhadores.
Surgia, assim, o objetivo de demarcar a área
precisa das embarcações para que estas pudessem ser represadas. Tal empreitada
exigiu a arrecadação de grandes fundos, que só seria finalmente concretizada em 1962.
As embarcações acabariam sendo escavadas em terra seca.

Após a recuperação das embarcações, os arqueólogos passaram a se dedicar às 
reconstruções. Em 1969, seria finalmente construído, em Roskilde, o Museu
de Barcos Vikings, que abrigaria as cinco Skuldelevs. Posteriormente, tais
embarcações motivariam estudos de arqueologia experimental, que dariam
origem às embarcações denominadas Saga Siglar e Roar Ege
(respectivamente, replicas do Skuldelev 1 e do Skuldelev 3). A \emph{Saga Siglar} seria
construída em 1983 para a realização de uma circunavegação e acabaria
naufragada em 1992, durante uma tempestade no mediterrâneo. Roar Ege
seria, por sua vez, construída em 1982 e encontra-se até hoje nos portos
do Museu de Barcos Vikings.

\SIG{Munir Lutfe Ayoub}

Ver também Arqueologia da Era Viking; Dinamarca da Era Viking;
Embarcações; Navegação Marítima; Oseberg.

\begin{itemize}
\item \versal{BONDE}, Niels; \versal{STYLEGAR}, Frans-Arne. Roskilde 6--et langskib fra
Norge--Proveniens og alder.~\emph{Kuml}, vol. 60, n. 60, 2011, pp.
247-261.

\item \versal{CROOME}, Angela. The Viking Ship Museum at Roskilde: expansion uncovers
nine more early ships; and advances experimental ocean-sailing
plans.~\emph{The International Journal of Nautical Archaeology}, vol.
28, n. 4, 1999, pp. 382-393.

\item \versal{CRUMLIN-PEDERSEN}, Ole. Aspects of Viking-Age Shipbuilding: In the Light
of the Construction and Trials of the Skuldelev Ship-Replicas Saga
Siglar and Roar Ege.~\emph{Journal of Danish Archaeology}, vol. 5, n. 1,
1986, pp. 209-228.

\item \versal{OLSEN}, Olaf; \versal{CRUMLIN-PEDERSEN}, Ole. The Skuldelev Ships: A Preliminary
Report on Underwater Excavations in Roskilde Fjord, Zealand.~\emph{Acta
Archeologia}, 1958, pp. 161-175.
\end{itemize}
\section{\versal{RUS}}

Na historiografia, o termo Rus pode ter três significados
distintos, conforme o contexto que envolve o seu uso. Pode significar o nome
dado aos nórdicos que se assentaram ao longo da região onde hoje estão
localizados o noroeste e sudoeste da Rússia, bem como o norte da Ucrânia; a
unidade política e administrativa com base no território que esses
nórdicos conquistaram e nos quais se sedentarizaram; ou os descendentes de tais
nórdicos que passaram a residir em tal unidade política. O termo, em si, é de
enorme complexidade e a sua verdadeira origem filológica ainda é
desconhecida. As poucas evidências arqueológicas existentes também são
inconclusivas sobre a natureza da palavra. Não se sabe ao certo se a
nomenclatura provém do germânico, eslavo ou fino-úgrico. Tampouco se sabe a quem
exatamente se referia, se aos escandinavos ou aos eslavos. Por causa
desse problema, desde o século~\versal{XVIII} existe um debate acadêmico
fervoroso sobre as origens do
termo Rus, conhecido como Controvérsia Normanista.

Na Controvérsia Normanista, há, primariamente, duas posições sobre a
origem da terminologia, tanto de um ponto de vista étnico (o mais
enfatizado) quanto de um ponto de vista geográfico. São as posições 
normanista e antinormanista. Os
normanistas afirmam -- baseados, principalmente, nas entradas da \emph{Crônica dos Anos
Passados} e em outras fontes -- que a origem do termo Rus como referência ao povo e ao
território seria fruto de uma influência exclusivamente escandinava. 
Do lado oposto e fortemente influenciado
pelo nacionalismo russo -- sobretudo durante o século \versal{XVIII}, bem como durante a 
vigência da União
Soviética --, os antinormanistas acreditam que o termo englobava somente
eslavos e não teve qualquer tipo de influência
escandinava. Atualmente, uma versão híbrida, preferível para esse tipo de
debate, é aceita por alguns
especialistas, como André Muceniecks. Os autores adeptos a essa terceira versão afirmam que rus
consistia um grupo multiétnico de guerreiros, cuja maioria (incluindo a liderança)
seria escandinava, mas do qual também fariam parte eslavos, finos e turcomanos.
De um termo ocupacional, ele acabou por se tornar um termo étnico.

Mesmo assim, a proeminência dos escandinavos era inegável entre os rus.
Nas fontes que tratam do tema, uma das primeiras menções ao termo rus,
como \emph{Rhos}, vem dos \emph{Annales Bertiniani}, uma fonte do
Império Carolíngio datada da primeira metade do século~\versal{IX}. Nesta, a palavra
se refere aos vikings provenientes da Suécia, mas cujo líder é conhecido
pela denominação turcomana \emph{khagan}. Fontes bizantinas, como o
tratado \emph{De Administrando Imperio}, também tratam do termo em uma
perspectiva que privilegia a visão normanista, referindo-se tanto aos
rus quanto aos seus domínios através de uma nomenclatura de origem
nórdica, diferenciando-os dos eslavos que habitavam essa região. Mas,
possivelmente, o uso contemporâneo do termo mais relevante para 
descobrir sua origem provém de fontes do Oriente muçulmano, que são
contraditórias entre si. Se, por um lado, autores como Ibn Fadlan e Ibn
Rusta diferenciam os \emph{rūs} (rus) dos \emph{saqāliba} (eslavos), por
outro lado, Ibn Khordadbeh, que é anterior aos muçulmanos supracitados,
não os diferencia e trata os \emph{rūs} como uma parte dos
\emph{saqāliba}.

Conforme Przemysław Urbańczyk, somente a partir da cristianização de Rus
por Vladimir~\versal{I} Sviatoslavich de Kiev (980-1015) que o termo Rus mudou
semanticamente, designando não mais guerreiros nórdicos, mas eslavos
orientais cristãos. Como território, Rus é geralmente entendida como uma
associação de principados habitados pelo povo rus, cujos membros eram ligados entre
si a partir de laços dinásticos (pois todos faziam parte da dinastia Riuríkida). 
Entre os séculos~\versal{IX} e \versal{XIII}, faziam parte desse território
o oeste da Rússia, a Ucrânia, a Bielorrússia
e a Moldávia. A nomenclatura ``Rus de Kiev'', derivada do uso
geográfico do termo, é utilizada para denominar o período entre os
séculos~\versal{IX} (com a chegada dos varegues) e \versal{XIII} (com a tomada de Kiev pelos
mongóis), no qual o Principado de Kiev era o preponderante entre os
demais.

\SIG{Leandro César Santana Neves}

Ver também Crônica dos Anos Passados; Kiev; Novgorod; Olga de Kiev;
Rússia da Era Viking; Staraia Ladoga; Varegues; Vladimir~\versal{I} de Kiev

\begin{itemize}
\item \versal{DUCZKO}, Wladsyslaw. \emph{Viking Rus: studies on the presence of
Scandinavians in Eastern Europe}. Leiden: Koninklijke Brill \versal{NV}, 2004.

\item \versal{FRANKLIN}, Simon; \versal{SHEPARD}, Jonathan. \emph{The Emergence of Rus
750-1200}. Essex: Longman, 1996.

\item \versal{MUCENIECKS}, André Szczawlinska. \emph{Austrvegr e Garđaríki -
(re)significações do leste na Escandinávia tardo-medieval}. Tese de
Doutorado em História Social. São Paulo: Faculdade de Filosofia, Letras
e Ciências Humanas, \versal{USP}, 2014.

\item \versal{PRITSAK}, Omeljan. The Origin of Rus. \emph{Russian Review}, vol. 36, n.
3, 1977, pp. 249-273.

\item \versal{STANG}, Håkon. \emph{The Naming of Russia}. Oslo: Meddelelser, 1996.

\item \versal{URBAŃCZYK}, Przemysław. Who were the early Rus? In: \versal{MAKAROV, N. A.};
 \versal{LEONTIEV, A. E.} \emph{Rus v \versal{IX-XII} vekákh: Óbchtchestvo, Gosudárstvo,
Kultúra} [\emph{Rus nos séculos \versal{IX-XII}: Sociedade, Estado,
Cultura}]. Moscou-Vologda: Driévnosti Siévera, 2014, pp. 228-233.
\end{itemize}
\section{\versal{RÚSSIA DA ERA VIKING}}

A Rússia que conhecemos hoje não é a mesma que foi conhecida pelos vikings
a partir do século~\versal{IX}. Não se estendia muito para o leste
além dos Urais e era composta por partes de alguns outros países
contemporâneos, como a Ucrânia e Belarus. Por questões de abrangência, Thomas
Noonan define a área que compunha a Rússia da Era Viking de Rússia
Europeia, ainda que outros termos sejam
utilizados por especialistas, como o próprio nome Rússia, Europa Oriental
(no caso dos pesquisadores russos), ou Planície Russa. Ainda conforme
Noonan, a Rússia Europeia pode ser dividida em cinco áreas diferenciadas
por sua geografia e pelas atividades econômicas: as estepes do sul da
Rússia/Ucrânia; a área florestal ucraniana; a área florestal no centro e
norte russo, região fronteiriça ao mar Báltico; e a região da tundra, no extremo
norte da Rússia. A \emph{Crônica dos Anos Passados} lista diversos povos
eslavos que residiam na área, como os polianos, derevlianos e radimichi.
Todavia, havia também povos de origem turcomana e fino-úgrica.

A população que habitava a Rússia da Era Viking não se resumia aos nórdicos recém-chegados e aos
povos eslavos. Entre os muitos povos, havia os Búlgaros do Volga, ou
Búlgaros Negros, como eram conhecidos pelos bizantinos. Estes eram de
origem turcomana e se instalaram ao norte da Rússia, especificamente
ao longo do rio Volga. Mas a Bulgária do Volga (não confundir com o
Império da Bulgária no leste da Europa) também era composta por etnias
da região da Finlândia e povos eslavos locais. Embora inicialmente
subjugada ao Caganato da Khazária, com a conversão do território ao islamismo por
volta de 900, a região da Bulgária
do Volga passou a conviver com um grande influxo de mercadores e
mercadorias vindos do Califado Abássida, passando a ser a principal ``rival'' comercial de Kiev entre o
final do século~\versal{IX} e o início do século~\versal{X}. Muito do poder da Bulgária do
Volga vinha, assim como no caso de Rus, da posição geográfica privilegiada e do
domínio de uma da rotas do rio homônimo, onde havia intenso comércio. Os
búlgaros do Volga agiam como intermediários entre os nórdicos e o
Califado Abássida no comércio de diversos produtos, como seda, ornamentos
e prata.

Uma outra parte da Rússia moderna que manteve relações com os eslavos e,
eventualmente, com os nórdicos que aportaram em Kiev era o vasto Caganato
da Khazária, onde os kházaros, um povo de origem turcomana, se
assentaram desde o século~\versal{VII}. A Khazária encontrava-se na região do
Cáucaso. Conforme a \emph{Crônica}, os povos eslavos que pediram a
liderança de Riúrik e seus irmãos estavam previamente subjugados pelos
kházaros. É provável esse relato seja verdadeiro, pois, entre os séculos~\versal{VI} e \versal{X}, os
kházaros formavam uma poderosa força regional e um poderoso aliado do
Império Bizantino. A Khazária, ao contrário das outras partes da Rússia,
localizava-se dentro da área onde predominava um solo de cor escura,
bastante fértil. De acordo com Thomas Noonan, os kházaros também foram
importantes para a existência de uma grande quantidade de \emph{dirhams}
de prata (moeda utilizada pelo Califado Abássida) em Rus durante os
séculos~\versal{IX} e \versal{X}. Tal importância se deve aos contatos dos 
kházaros tanto com a Bulgária do Volga quanto
com o Califado, bem como devido a sua posição de mediador entre este, os escandinavos
e os búlgaros do Volga. Esse quadro sugere que as relações comerciais entre
Rus e a Khazária ainda permaneceram fortes mesmo após o assentamento
nórdico.

A parte mais notável da Rússia da Era Viking seria o território
conhecido pelos escandinavos como \emph{Garđaríki} (``reino das cidades'').
Neste, estavam localizadas Kiev e Novgorod, que se
tornariam, a partir do assentamento veregue, os principais centros econômicos, 
políticos e culturais. Kiev, localizada onde hoje se encontra a
cidade homônima na Ucrânia, foi fundamental no controle da rota
comercial entre os nórdicos e os bizantinos. Nóvgorod, sendo hoje a
cidade de Velikii Novgorod, na Rússia, conviveu com a criação de 
diversos postos comerciais
nórdicos, principalmente para o comércio de peles e
escravos. Outros pontos de assentamento nórdico são Staraia Ladoga
(próxima aos lagos Ilmen e Ladoga, no noroeste da Rússia) e Pskov
(próximo do Lago Pskov, no noroeste da Rússia), ambos na área florestal
próxima ao Mar Báltico. Os escandinavos que se fixaram nesses locais
eram conhecidos como ``Rus'' nas fontes estrangeiras, embora seja possível
que a terminologia fosse de conotação ocupacional e não étnica.

De acordo com Jonathan Shepard, a busca pela prata proveniente do
Califado Abássida motivou os nórdicos -- conhecidos como varegues pela
\emph{Crônica dos Anos Passados} e por fontes bizantinas -- a se
aventurarem pela Rússia Europeia. Os escandinavos chegaram primeiro em
Staraia Ladoga, em meados do século~\versal{VIII}, conforme achados arqueológicos
de pertences, como ferramentas e pentes. Alguns historiadores, como
Wladyslaw Duczko, classificam Staraia Ladoga como a primeira capital do
\emph{Khaganato Rus}, uma entidade política que, alegadamente, precedeu Rus de
Kiev. Os nórdicos partiram de Staraia Ladoga para Novgorod, especialmente para a região de
Riurikovo Gorodische, onde ocorreu o assentamento e o controle das rotas
do rio Volkhov. A partir do século~\versal{IX}, alguns varegues decidiram seguir
em direção ao rio Dniepre, no sul, motivados por vantagens advindas
de comércio com os bizantinos, bem como pelo impedimento de seguir pelo
rio Volga em função da presença dos búlgaros. A \emph{Crônica} conta que os eslavos
dessa região pediram aos escandinavos que os governassem no ano de 862,
mas a arqueologia mostra que os varegues somente começaram a se assentar
em Kiev a partir do século~\versal{X}.

Além da \emph{Crônica dos Anos Passados}, não há muitas outras fontes
escritas que tratam sobre a organização política da Rússia da Era
Viking. Mas, uma vez que a narrativa da \emph{Crônica} versa quase exclusivamente sobre Kiev
e as outras fontes disponíveis são posteriores, torna-se difícil saber quem foram os
governantes dos futuros principados de Rus de Kiev. É possível que os varegues formaram a elite
político-administrativa e militar desses locais, assim como ocorreu na
cidade próxima ao Dniepre. No tratado de 945, há
diversos nomes nórdicos representando príncipes de Rus. Os escandinavos
que não permaneceram em Rus continuaram mantendo boas relações com os
que ficaram, participando ativamente do comércio e oferecendo sua ajuda
como guerreiros mercenários nas diversas ocasiões em que os Rus entraram
em conflito com algum território ou contra os povos eslavos das estepes. 
Mesmo após a cristianização de Rus, no final do
século~\versal{X}, os laços entre ambos permaneciam fortes, com a permanência da
ajuda militar, bem como de casamentos entre os príncipes de Rus e a realeza
escandinava. Como exemplo, menciona-se o matrimônio entre Iaroslav Vladimirovich, o
Sábio (1016-1018, 1019-1054), e Ingigerth, filha do rei Olavo
Skötkonung~(995-1022), da Suécia.

\SIG{Leandro César Santana Neves}

Ver também Crônica dos Anos Passados; Kiev; Mikligardr; Novgorod; Olga
de Kiev; Rus; Staraia Ladoga; Varegues; Vladimir~\versal{I} de Kiev.

\begin{itemize}
\item \versal{ANDROSHCHUK}, Fjodor. The Vikings in the east. In: \versal{BRINK}, Stefan; \versal{PRICE},
Neil (eds.). \emph{The Viking World}. London: Routledge, 2008, pp.
517-542.

\item \versal{DUCZKO}, Wladsyslaw. \emph{Viking Rus: studies on the presence of
Scandinavians in Eastern Europe}. Leiden: Koninklijke Brill \versal{NV}, 2004.

\item \versal{FRANKLIN}, Simon; \versal{SHEPARD}, Jonathan. \emph{The Emergence of Rus
750-1200}. Essex: Longman, 1996.

\item \versal{MOREIRA}, Fabrício de Paula Gomes. \emph{A constituição político-cultural
da autoridade dos príncipes Rus´ entre os séculos \versal{X} e \versal{XII}}. Dissertação
de Mestrado em História. Mariana: Programa de Pós-Graduação em História
‒ \versal{UFOP}, 2014.

\item \versal{NOONAN}, Thomas S. European Russia, c. 500-1050. In: \versal{REUTER}, Timothy.
\emph{The New Cambridge Medieval History -- Volumen \versal{III} c. 900-c. 1024}.
Cambridge: Cambridge University Press, 2008, pp. 487-514.

\item \versal{SHEPARD}, Jonathan. The Viking Rus and Byzantium. In: \versal{BRINK}, Stefan;
\versal{PRICE}, Neil (eds.). \emph{The Viking World}. London: Routledge, 2008,
pp. 476-516.
\end{itemize}
\chapter{S \textarn{s} \textarc{s} \textart{s}}
\section{\versal{SAGAS DO ATLÂNTICO NORTE}}

Termo usado para se referir às sagas do descobrimento da América
ou às \emph{Vínland sagas} (\emph{Sagas das Terras das Vinhas}). Buscando narrar as explorações
nórdicas que culminaram na descoberta de novas terras na América, essas
sagas são divididas em duas: a \emph{Eiríks saga rauða} -- conhecida também como a
\emph{Saga de Eiríkr, o Vermelho} -- e a \emph{Grœnlendinga saga},
ou \emph{Saga dos Groenlandeses}.

A primeira delas foi escrita em meados do século~\versal{XIII} e preservada em dois manuscritos que contêm versões um
pouco diferentes: \emph{Hauksbók}(século~\versal{XIV}) e \emph{Skálholtsbók} (século~\versal{XV}). Filólogos modernos acreditam que o \emph{Skálholtsbók} é
uma versão mais próxima da composição original. Apesar da saga original
ter sido escrita no começo do século~\versal{XIII}, ela acaba narrando eventos
ocorridos no intervalo de 990 até 1030. Dualidades temporais são comuns nas sagas islandesas, 
nas quais há o tempo de tessitura documental
e de preservação e o tempo em que a narrativa ocorre. Obviamente, qualquer fonte sempre falará mais de seu tempo do que sobre tempo a que se
refere; muitos elementos das narrativas das sagas do Atlântico
Norte estão arraigados pela estética própria ao autor cristão que as
escreveu. Mas não somente: essa saga é também um produto claro de força da tradição
oral da sociedade islandesa, sendo um produto desses relatos.

Em sua narrativa, ela busca mostrar como se deu a descoberta e o
processo de assentamento da Groenlândia, focando a primeira parte de sua
narrativa na vida e ações de Érico, o Vermelho. Relata-se como
a tendência de proscrição da família é um fator estimulador da
narrativa, e que a impossibilidade de permanência na Noruega e depois
na Islândia será um fator motriz para o descobrimento amplo e a
exploração da Groenlândia.

Érico, com base em relatos de Gunnbjörn, parte em direção a novas
terras que teriam sidos vistas por este último. Durante um período de
cerca de dois anos, Érico habita e explora a região que denomina de
Terra Verde (significado de Groenlândia), até retornar com as notícias de
novas terras e possiblidades para a Islândia. Após um período, consegue reunir família e aliados e parte para a Groenlândia, com cerca de vinte e cinco
navios, dos quais, de acordo com o \emph{Íslendingabók},
apenas quatorze conseguiram chegar à nova terra, e ``Isso se deu quinze
invernos antes de cristianismo ser tomado como lei na Islândia
{[}\emph{c.1000}{]}'' (Anônimo, 2007b, p. 90-91, grifo nosso).

Após isso, a narrativa apresenta traços de formações familiares de novas
vindas para a Groenlândia, assim como a formação de novos núcleos
familiares, assemelhando-se às \emph{Íslendingasögur}, as sagas de
família. A saga também relata os percalços da vida dos
indivíduos da região, que passavam por fome e dificuldades comerciais.
Traz também traços de paganismo e magia, mesmo que preenchidos por
uma óptica cristã de composição. A narrativa chega em um segundo momento,
onde falará sobre Leifr, filho de Érico, apresentando sua viagem para a
corte do rei Olavo Tryggvason, assim como sua conversão para o
cristianismo, até o seu retorno ao fiorde de seu pai. A narrativa segue
com inserção de novos personagens e de novas motivações para viajar para
Vínland, culminando nos relatos de tentativa e fracasso de
colonização dessa nova terra.

A segunda saga tem um início semelhante, focando na narrativa sobre
Érico e apresentando elementos ocorridos de formas bastante similares da
primeira. E de fato, essa tendência de proximidade ocorre muito pelo
tempo de tessitura de ambas se dar no começo do século~\versal{XIII}, e
narrar espaços temporais bem similares. Mas os relatos em que se
baseiam, a forma estética de composição e a escolha do desenvolvimento da
narrativa são diferentes, e por muitas vezes trazem contradições e
disputas entre essas duas fontes.

A \emph{Saga dos Groenlandeses}, como o nome indica, tratará da
narrativa dos indivíduos que habitam essa terra, e de fato é isso que a saga
apresenta: traz um maior grau de detalhe sobre o processo
de ``descobrimento'' da América do Norte, revelando novos sujeitos,
novas narrativas, como um reflexo da diferença de relatos da tradição
oral que cada autor absorveu. Exemplo que pode ser observado em um
fator: Bjarni, na \emph{Saga dos Groenlandeses}, é apresentado como o primeiro
a visualizar a América do Norte, o que na saga de Érico é colocado como
algo feito por Leifr.

Os detalhamentos e narrativas sobre a presença nórdica na
América do Norte são bem mais amplos nessa última saga em relação a
anterior, tendo início após uma rápida introdução e apresentação de elementos
estéticos típicos do gênero. Além de narrar a viagem de Leifr, a saga
trata da viagem de Karlsefni (provavelmente aquele que teve o
círculo familiar responsável pelos relatos que compõem a narrativa e/ou
sujeitos que tinha uma ``predileção'' pelo seu papel na história),
mostrando elementos das viagens, do assentamento e dos conflitos com os
povos locais da região, os \emph{skrælingjar --} esquimós. Um grande
elemento desta saga em detrimento da outra é que a narrativa sobre uma
circularidade de conhecimento sobre a Vínland é bem mais
divulgada e apresentada, revelando o modo como as expedições
e explorações desta terra eram parte de um imaginário nórdico.

Trazendo dentro de si incontáveis elementos que envolvem a dinâmica do descobrimento, as duas sagas são peças 
fundamentais em um estudo sobre América e sobre a literatura medieval, sendo um conjunto de fontes
muito populares e de circularidade dentro da própria dinâmica medieval -- o que fica 
claro com a existência do \emph{Grœnlendinga þáttr}, o conto
dos groenlandeses (``conto'' é uma tradução aproximada para \emph{þáttr} --
p. \emph{þættir} --, o qual representa pequenas narrativas,
independentes ou complementares a outros textos), ampliando ainda mais a
dimensão do estudo sobre essas sagas do Atlântico Norte. De fato, elas
são ``{[}...{]} um raro momento em que o homem, a despeito de ter
ultrapassado todos os seus limites imagináveis, \emph{perceber-se}
pequeno demais para sua vontade e a sua curiosidade'' (Moosburger, 2007,
p. 137).

\SIG{José Lucas Cordeiro Fernandes}

Ver também Brathahlid; Groenlândia nórdica; Leif Eriksson; Vínland.

\begin{itemize}
\item \versal{ANÔNIMO}. A Saga do Groenlandeses. In: \emph{As três sagas Islandesas.}
Trad. Théo Moosburger. Curitiba: Editora \versal{UFPR}, 2007a.

\item \versal{ANÔNIMO}. A Saga de Eiríkr Vermelho. In: \emph{As três sagas Islandesas.}
Trad. Théo Moosburger. Curitiba: Editora \versal{UFPR}, 2007b.

\item \versal{ANÔNIMO}. Eirik the Red's saga. In: \versal{HREISSON}, Viðar; \versal{COOK}, Robert;
 \versal{GUNNELL}, Terry; \versal{KUNZ}, Keneva; \versal{SCUDDER}, Bernard (eds.). \emph{The
Complete Sagas of Icelanders.} Reykjavík, Islândia: Leifur Eiríksson
Publishing, 1997, pp. 01-18 (vol. 1).

\item \versal{ANÔNIMO}. The Saga of the Greenlanders. In: \versal{HREISSON}, Viðar; \versal{COOK},
Robert; \versal{GUNNELL}, Terry; \versal{KUNZ}, Keneva; \versal{SCUDDER}, Bernard (eds.). \emph{The
Complete Sagas of Icelanders.} Reykjavík, Islândia: Leifur Eiríksson
Publishing, 1997, pp. 19-32 (vol. 1).

\item \versal{ANÔNIMO}. The Tale of the Greenlanders. In: \versal{HREISSON}, Viðar; \versal{COOK},
Robert; \versal{GUNNELL}, Terry; \versal{KUNZ}, Keneva; \versal{SCUDDER}, Bernard (eds.). \emph{The
Complete Sagas of Icelanders.} Reykjavík, Islândia: Leifur Eiríksson
Publishing, 1997, pp. 372-382 (vol. 5).

\item \versal{JONES}, Gwyn. \emph{The Norse Atlantic Saga}: Being the Norse Voyages of
Discovery and Settlement to Iceland, Greenland, and North America.
Oxford and New York: Oxford University Press, 1986.

\item \versal{MOOSBURGER}, Théo de Borba. Posfácio. In: \emph{As três sagas
Islandesas.} Curitiba: Editora \versal{UFPR}, 2007.

\versal{O'DONOGHUE}, Heather. \emph{Old norse-Icelandic Literature: a short
introduction}. Hoboken: Blackwell Publisher, 2005.

\item \versal{ROSS}, Margaret Clunies (ed.). \emph{Old Icelandic Literature and
Society}. Cambridge: Cambridge University Press, 2000.

\item \versal{SIGURÐSSON}, Gísli. \emph{The medieval Icelandic saga and oral
tradition: a discourse on method}. London: Harvard University Press,
2004.

\item \versal{THORGILSSON}, Ari; \versal{ANÔNIMO}. \emph{Íslendingabók, Kristni Saga: The book
of the icelanders, the story of the conversion}. Trad. Sion Gronlie.
Viking Society for Northern Research: University College of London,
2006.
\end{itemize}
\section{\versal{SAGAS ISLANDESAS}}

Ver Egils saga; Eyrbyggja saga; Færeyinga saga; Flateyjarbók; Grettis
saga; Guta saga; Laxdaela saga; Njáls saga; Sagas do Atlântico Norte.

\section{\versal{SÁMI, FÍNICOS E NÓRDICOS}}

Em algumas sagas islandesas, os noruegueses que visitavam
Jotunheim viajam para o norte. Noções de perigo, frio extremo,
fome e longa escuridão invernal são associadas, nesses relatos nórdicos,
ao norte. Esse ponto geográfico mítico é, também, o lar dos \emph{sami}
e \emph{fínicos:} vistos pelos nórdicos (e tratados nas sagas) como
diferentes. Se tomarmos o escopo temporal da Era Viking e do período de
conversão ao cristianismo (c. 800-1300 d.C.), vê-se que entre esses povos
que habitavam a Fenoescândia houve uma simbiose cultural pautada no
comércio, mas também em atritos (principalmente saques) e intercâmbios
religiosos.

Pode-se datar a chegada dos povos fino-urálicos na Escandinávia em 3.300 a.C.~ 
e com a chegada dos povos indo-europeus, por volta de 2.700 a.C.,
formou-se a cultura Kiukainen. A partir de 2.000 a.C. pode-se falar em
etnogênese balto-fínica: quando há a separação entre os povos
proto-fínicos, situados na costa, que se dedicavam a atividade
agropastoril, e os povos proto-sami, situados no interior do continente,
dedicados à caça e coleta. É importante ressaltar que, mesmo após a
divisão, esses povos se desenvolveram de forma paralela. Com efeito,
existiam afinidades linguísticas e econômicas entre os sami
e balto-fínicos,uma vez que bens, pessoas e tradições religiosas
circulavam entre essas diferentes culturas por meio do comércio e troca
de diferentes ganhos.

Em textos medievais já influenciados pelo cristianismo, como as sagas,
essa assimilação entre sami e fínicos é um dos pontos centrais para que se entenda
o uso do termo (impreciso) \emph{finnar} como designante de
ambos. Conforme Thomas DuBois, a noção medieval existente sobre os povos
não germânicos estava calcada em um entendimento de que eram
entidades étnicas unificadas que haviam migrado em massa para as margens
da civilização escandinava como intrusos, onde seriam cerceados pelo
poderio militar dos nórdicos e o eventual domínio cristão.

Esses contatos, pacíficos ou não, influenciaram diretamente na maneira
com que se produziu a imagem dos sami e fínicos nas sagas. A coleta de
impostos, parte das relações entre os sami e noruegueses que ocorria no
inverno, é datada do século~\versal{I} d.C. O comércio pode ser remontado ao
século~\versal{III} d.C, aparecendo o Norte e seus habitantes como potenciais
vítimas aos saques nórdicos, já que não prestariam assentuada resistência. É
interessante notar, também, que os nórdicos competiam com os fínicos
pelo acesso às peles e produtos provenientes da região sami, já que para
os nórdicos esses produtos eram importantes, pois poderiam fazer parte
do plantel de bens a serem enviados para a Inglaterra e trocados por
mel, trigo e tecidos. Vale notar que esse sistema de trocas só funciona porque ambos se beneficiavam da transferência do excedente produzido e da recepção de produtos considerados valiosos.

A questão sazonal é importante para discutir o contato entre nórdicos e
\emph{finnar}, pois parte dos arqueólogos associa o uso de estradas ao
período invernal. Nas sagas, existem algumas menções às caravanas de
nórdicos, que se dirigem ao norte durante o inverno, como Haroldo Cabelos Belos e a comitiva de dezenove homens de Thorolf. A estrada de
Adamvalldá, região sueca situada no vale Arjeplog, na divisa com a
Noruega, é um exemplo de elo entre os assentamentos sami aos nórdicos.
Estudiosos argumentam que a estrada não teria sido construída pelos e
para os povos locais, mas por nórdicos que desejavam estabelecer contato
com a região ártica e seus habitantes.

Escavações e mapeamentos da região apontam que o caminho era marcado por
pedras para facilitar o reconhecimento, já que viajar pelas montanhas
durante o inverno do norte da Escandinávia não era tarefa
simples. Argumenta-se ainda que esse tipo de empreitada -- organizar uma
estrada que pudesse ser usada e bem demarcada -- se dava em função de demarcações
políticas de grupos interessados em seu uso e que, para cumpri-las, seria
necessário um tipo de organização social mais eficiente, ausente nas
comunidades sami. Portanto, é plausível que instituições como a Igreja e
os reinos da Suécia e Noruega teriam planejado, financiado e construído
a estrada.

Já vimos que os povos sami e fínicos tinham relações entre si e com os
povos nórdicos, mas de qual forma os primeiros eram vistos pelos
últimos? Embora componham tipos diferentes de texto, as sagas trazem
informações interessantes sobre os sami (e fínicos) pela óptica nórdica.
É claro que essas descrições não podem ser tomadas como verdadeiras,
pois há uma clara separação entre os nórdicos e outros povos --
especialmente os \emph{finnar} -- que não deixa de ser uma forma de
incorporar no outro o que é temido ou indesejável, tornando-o
patológico. Sendo assim, esses relatos são interessantes para pensarmos
as relações de poder entre os diferentes grupos que circulavam pela
região.

A má reputação do norte, enquanto espaço incompatível para sustentar a
vida, parece ser o ponto central que associa a malícia aos \emph{finnar}:
à concepção de que a qualidade do espaço afeta as pessoas soma"-se a
ideia de um povo não civilizado, conectado com a terra e dotado de
habilidades mágicas e idolatrias pagãs. Em oposição aos nórdicos, o
arquétipo \emph{finnar} aparece como negativo em diversas sagas, com
algumas modificações.

A única habilidade que aparece associada aos fínicos é, também,
dada à condição das terras do norte: esquiar. Na \emph{Gesta Danorum} o
nome \emph{skritfínnss} (``esquiadores fínicos'') é usado para descrever
aquele povo, enquanto que na \emph{Griðamál}, o termo que aparece é
\emph{finnr skrí}ð\emph{r}. Na \emph{Saga de Santo Olavo}, capítulo 83, há uma
breve descrição do ``pequeno Fin'':

\begin{quote}
Havia um homem do planalto chamado Fin, o pequeno, e alguns dizem que era
da raça finlandesa. Era um homem pequeno singular, tão ágil com os pés
que nenhum cavalo o venceria. Era, também, um arqueiro e corredor com
raquetes de neve bem treinado.
\end{quote}

Fora isso, os \emph{finnar} são retratados como usuários de roupas feitas de
pele animal, habitantes de tendas e que tinham predileção por manteigas
e gorduras animais -- produtos trocados com os germânicos da região, dada
sua indisponibilidade nas terras setentrionais.

As principais características, do ponto de vista nórdico, dos \emph{finnar} apresentada nos textos medievais e que, de certa forma, foram legados à
posterioridade, são a bruxaria e a feitiçaria. Essa reputação parece dúbia e
está entre a admiração/respeito e o medo/trepidação. Há menções a um
\emph{noiadi} sami, habilidoso nas artes mágicas que possuía a
capacidade de prever o futuro, além de ser tratado como líder por outros
sami. Mas, os casos mais famosos que associam bruxaria aos sami estão,
também, vinculados à figura da mulher. Na \emph{Ynglinga saga}, por exemplo,
existem relatos sobre o envolvimento dos reis Vanlandi e Agni com belas
mulheres (bruxas) sami que, por vingança, acabam por rogar feitiços que
levam a morte dos reis.

Nas sagas existem diversas referências a mulheres de origem sami que ganham poder
e influenciam reis e príncipes graças à pura magia. O caso de
Haroldo Cabelos Belos, retratado na \emph{Heimskringla}, também é
interessante. O norueguês é vítima de um feitiço que transforma
Snæfriðr, a filha do rei sami Svási, em sua única obsessão e a condição
imposta por Svási para que Haroldo pudesse deitar-se com ela era o
casamento. Outro exemplo na \emph{Heimskringla} seria o de Gunhildr,
treinada nas artes mágicas sami por dois exímios caçadores bruxos, que
se casa com Érico Machado Sangrento.

Dada sua conexão com a terra, os \emph{finnar} nunca são associados com
cidades, e sim com florestas escuras do norte. Cabe, ainda, uma ressalva
sobre os sufixos utilizados em nórdico antigo para descrever a região
dos \emph{finnar}. Ao adicionar \emph{mǫrk} (marca) ao
prefixo \emph{Finn} temos \emph{Finnmǫrk} (Finamarca)
que denota área fronteiriça e pouco habitada. Se o sufixo usado for \emph{land}
(terra), denota-se terra habitada. Essas ressalvas são importantes, pois
nos atentam à análise daquele discurso. No capítulo quatorze da \emph{Saga de Egil Skallagrimsson}, por exemplo, há uma descrição da \emph{Finnmark} (Finamarca):

\begin{quote}
{[}\ldots{}{]} ao norte se encontra a Finamarca, onde há distritos
perdidos pouco povoados; alguns em vales, outros próximos a lagos que
são enormes e próximos a eles encontram-se extensivas florestas.
\end{quote}

O uso do termo Finamarca pode ter sido usado na passagem como forma de
respaldar a ideia de uma região pouco povoada, cercada de lagos e
florestas que, portanto, seria a residência do outro. Ou seja, os
sufixos \emph{mǫrk} ou \emph{land} são usados em
situações diferentes a partir da relação que o autor quer estabelecer
entre os nórdicos e nortistas.

Os nomes dos \emph{finnar} citados nas sagas, geralmente, são associados
a fenômenos naturais como a neve (como Drífa e Snær) -- outra forma de
reafirmar, por parte dos nórdicos, a conexão entre os indivíduos do
norte e a terra. Quando aparecem sem nomes, recebem descrições
relacionadas à sua etnia. Existem, por exemplo, passagens em que a
associação entre sami e \emph{trolls} é feita, ou seja, usa-se o termo
\emph{troll} como um sinônimo pejorativo para \emph{finnar.} Há, ainda,
termos como \emph{halftroll}, \emph{halfbergrisi} e \emph{halffrisi}
(meio"-\emph{troll}, meio"-gigante da montanha e meio"-gigante) usados para
difamar a origem mestiça (nórdica x \emph{finnar}) de um personagem. O
termo \emph{hálfinnr} também aparece e expressa um indivíduo
``meio-sami'', no entanto, o adjetivo só é empregado quando o personagem
tem um pai norueguês e uma mãe sami, o que pode nos ajudar a entender as
relações de poder existente entre esses grupos.

Enquanto um produto de autoria cristã do século \versal{XIII}~, as sagas
enfatizam o caráter estranho e negativo dos povos do norte: uma vez que
são considerados como um grupo marginal pelas elites cristãs nórdicas, os
\emph{finnar} pertencem ao grupo do ``outro''. Tendo em vista o
contexto de produção cristã das sagas, classificar os \emph{finnar} como
um grupo que vivia entre o mundo conhecido civilizado e a periferia
desconhecida é um encaixe perfeito, visto que na dicotômica cosmovisão
cristã não há espaço para os pagãos, logo, eles deveriam ser
convertidos. Portanto, seria muito simplista resumir o contato entre
nórdicos e \emph{finnar} a partir da dicotomia caçador-coletor nômade
\emph{versus} nórdico agricultor sedentário, como feita pelas sagas. A
relação simbiótica dos povos que circulavam a Fenoescândia nos parece
ter variado, por linhas tênues, entre a cooperação, coexistência e
violência ocasional -- que aumenta na medida que a cristianização dos
reinos nórdicos toma espaço -- marcada tanto pelas vias positivas (como
o comércio) quanto por conflitos e relações desiguais.

\SIG{Vítor Bianconi Menini}

Ver também Finlândia da Era Viking; Mar Báltico da Era Viking; Suécia da
Era Viking.

\begin{itemize}
\item \versal{AALTO}, Sirpa. \emph{Categorizing Otherness in the Kings' Sagas}.
Joensuu: University of Eastern Finland, 2010.

\item \versal{BERGMAN}, Ingela \emph{et al}. Stones in the snow: a Norse fur traders'
road into Sami country.~\emph{Antiquity}, vol. 81, n. 312, 2007, pp. 397-408.

\item \versal{BROADBENT}, Noel D.~\emph{Lapps and Labyrinths: Saami
Prehistory, Colonization and Cultural Resilience}. Washington \versal{D.C.}: Arctic
Studies Center, National Museum of Natural History, 2010.

\item \versal{DEANGELO}, Jeremy. The North and the Depiction of the ``Finnar'' in the
Icelandic Sagas\emph{\textbf{.~}Scandinavian
Studies}\textbf{,~}Champaign, vol. 82, n. 3, 2010, pp. 257-286.

\item \versal{GREEN}, William Charles. \emph{The Story of Egil Skallagrimsson}, London:
Elliot Stock, 1893. Disponível em:
\textless{}http://sagadb.org/egils\_saga.en\textgreater{}. Acesso em
17/05/2017.

\item \versal{KENT}, Neil.~\emph{The Sámi Peoples of the North: A Social and
Cultural History}. London: Hurst \& Company, Lapps and labyrinths, 2014.

\item \versal{LEHTOLA}, Veli-Pekka. \emph{The Sámi People: Traditions in transition}.
Fair Banks: University of Alaska Press, 2004.

\item  \versal{PÁLSSON}, Hermann. The Sami People in Old Norse
Literature.~\emph{Nordlit}\textbf{,~}UiT The Arctic University of
Norway, {[}s.l.{]}, vol. 3, n. 1, 1999, pp. 29-53.

\item \versal{STURLUSON}, Snorri.~\emph{Heimskringla: The Chronicle of the
Kings of Norway}. Disponível em:
\textless{}https://www.gutenberg.org/files/598/598-h/598-h.htm\#link2H\_4\_0316\textgreater{}.
Acesso em: 25/05/2017.
\end{itemize}

\section{\versal{SEPULTAMENTOS}}

Embora a imagem icônica das sepulturas nórdicas antigas seja a de um
barco cremado no mar ou depositado sob uma colina fúnebre, carregado com
oferendas e restos mortais, a verdade é que o registro arqueológico é
bastante mais rico e revela uma grande diversidade nas formas de
sepultamento, não sendo possível dizer que havia um modelo universal na
Escandinávia antiga. Ao contrário, o que se sabe atualmente sugere grandes
variações nos costumes consoantes à região, comunidade ou período
histórico, para não falar da posição ou posses das pessoas
envolvidas.

Entenda-se que a ideia popular de um barco enterrado sob uma colina
artificial não está incorreta, pois conhecem-se vários exemplos. O mais
famoso é a sepultura de Oseberg, datada de c. 830 e escavada no sul da
Noruega no início do século \versal{XX}, que consistia numa embarcação de grandes
dimensões -- 21.5 metros de comprimento por 5.1 de largura -- e onde
foram depositados os restos mortais de duas mulheres, acompanhadas de
múltiplas oferendas. Entre elas contam-se não só animais como cavalos,
cães e bois, mas também trenós, uma carroça, figuras em madeira para a
proa da embarcação, mobília, têxteis e ainda comida e utensílios de
cozinha, num espólio que indica tratar-se de uma sepultura de estatuto
social elevado, provavelmente régio. Mas seria errado pensar que este
tipo de sepultamento era a norma, não só porque a maioria das pessoas
não teria posses para pagar um túmulo de iguais riqueza e dimensão, mas
também porque em outras regiões os costumes podiam ser outros, por vezes
mesmo radicalmente diferentes.

Um caso particular é o das ilhas Åland, localizadas entre a Suécia e a
Finlândia, onde a prática comum passava pela cremação dos mortos,
deposição das cinzas num pote de cerâmica e por fim a colocação de uma
pequena pata de animal em barro no topo do recipiente. O costume é de
tal forma único que não só fornece um exemplo claro de regionalismo
fúnebre, como ajuda a identificar os padrões migratórios dos nativos das
Åland, já que foram encontrados alguns sepultamentos idênticos ao longo
dos rios Volga e Kljaz'ma. E numa ilha sueca mais a sul, a Öland, alguns
defuntos foram sepultados juntamente com fósseis de amonoides, não sendo
seguro se havia nisso um simbolismo maior ou se era apenas uma opção por
gosto pessoal. A análise dos sepultamentos, portanto, passa não só pelas tradições
comunitárias mas também pela dimensão individual das práticas.

Outro caso peculiar, mas que encontra algum eco nas fontes escritas, é o
de Bogla, na região sueca de Småland, onde uma pequena colina local foi
usada para depositar restos mortais, alguns dos quais terão sido
inseridos em ranhuras ou aberturas naturais da elevação. A prática faz
lembrar o conteúdo da \emph{Eyrbyggja saga}, um texto tardio e islandês
e por isso distante de Bogla, mas onde é contado que alguns mortos
habitam o interior de uma montanha, ideia que pode ter feito parte das
crenças dos habitantes da Småland. E a provar a enorme diversidade de
tipos de sepultamento na Escandinávia antiga, mesmo ao nível local, a
colina de Bogla exibe um conjunto de práticas diferentes, de cremação a
mera deposição dos corpos, com ou sem monumentos, com conjuntos de
pedras dispostas de forma circular ou triangular, campas retangulares ou
pequenas colinas artificiais e até a reutilização de sepulturas ou
partes delas, havendo ainda sinais de convivência ou transição
religiosa. Um cenário complexo, muito para lá da simplicidade ou
clichês.

Se Bogla pode refletir as práticas de uma comunidade rural, já mais a
norte, também na Suécia, o posto comercial de Birka mantém a nota de
complexidade, mas acrescida de um requinte que talvez tenha algo a ver
com o ambiente cosmopolita daquele que foi um dos primeiros centros
urbanos nórdicos. O cemitério da antiga povoação conta com cerca de duas
mil pequenas colinas, muitas delas erguidas sobre camadas de cremação,
mas o costume que se destaca no registro arqueológico é o uso de caixões
ou de câmaras fúnebres. Trata-se de uma prática invulgar na Suécia
central, mas que revela alguma fusão de elementos nativos e importados,
o que não surpreende, já que Birka integrava as redes de comércio
transregional dos séculos \versal{VIII} a \versal{X}. E tem-se um vislumbre do que será
talvez um lado mais afetivo das práticas fúnebres por via de algumas das
câmaras, como a Bj.834, onde, a par de cavalos, armas e utensílios
domésticos, a sepultura continha o corpo de uma mulher sentada ao colo
de um homem.

Se estes exemplos revelam sepultamentos sem recurso a embarcações, há
outros casos em que o barco continua ausente, mas a sua forma está
claramente presente. É assim no cemitério de Limfjord, na Dinamarca,
onde o comum parece ter sido a cremação dos corpos dentro de uma área
delimitada por pedras que, com frequência, desenham o contorno de uma
embarcação. Encontra-se a mesma prática na ilha sueca de Gotland e
também na Suécia central, no que é um reflexo da importância dos barcos
na antiga cultura nórdica. Os próprios deuses refletem essa realidade,
com Freyr, filho do próspero barqueiro divino que é Njord, a ter o
melhor dos navios, diz a estrofe 43 do poema éddico \emph{Grímnismál}. E
é na \emph{Edda} de Snorri que encontramos a descrição do funeral de
Balder, onde o filho de Odin é cremado numa embarcação. É uma imagem
clássica, icônica mesmo, mas que como se viu pode induzir em erro se
assumida como universal, embora não deixe de ser factual como uma de
várias formas de sepultamento na Escandinávia antiga.

O registro arqueológico dessa prática é mais escasso por óbvia
destruição do barco no processo de cremação, mas o relato feito pelo
árabe Ibn Fadlan, que encontrou um grupo de vikings nas margens do Volga
em 922, deixou-nos uma descrição que corrobora até certo ponto o modelo
mitológico, já que refere a deposição do corpo no interior do navio
juntamente com oferendas animais e inanimadas e o sacrifício de uma
escrava que é também cremada. E quando tudo estava já consumido pelas
chamas, o relato menciona ainda a construção de uma colina artificial
sobre as cinzas e o erguer de um poste de madeira com uma inscrição que
referia o nome do morto. É possível que a narrativa mitológica
transmitida ou criada por Snorri na \emph{Edda} seja um reflexo de
práticas como as descritas por Ibn Fadlan, que, embora não fossem a
totalidade dos costumes fúnebres escandinavos, eram certamente parte
deles.

A elevação de um poste de madeira no local do enterro tem paralelo em
sepulturas conhecidas, embora sem cremação, tendo-se usado o mastro do
próprio navio no caso em que ele foi apenas coberto com uma colina de
terra. Em alguns casos podiam acrescentar-se arranjos monumentais de
pedras, fosse em círculo em torno do túmulo -- como na ilha de Groix, na
Bretanha -- fosse em linha ao gênero de uma via processional. E há ainda
o caso de Anundshög, na Suécia, onde em um mesmo espaço relativamente
reduzido encontram-se colinas, dois conjuntos de pedras dispostas em
forma de embarcação e uma inscrição em runas, talvez relacionada com um
defunto ou pelo menos em memória de alguém chamado Heðinn. Também aqui,
no entanto, há que chamar a atenção para o fato de que nem todas as
pessoas teriam posses ou estatuto para tal monumentalidade. A maioria
dos nórdicos teria sepulturas mais simples e efêmeras, senão
mesmo meras valas, e são precisamente essas as que deixam menos
vestígios arqueológicos.

Faz-se patente o uso de oferendas fúnebres, como
comida, utensílios domésticos ou armas, e ainda sacrifícios animais,
como cavalos ou cães, a que se podem juntar opções menos comuns, como
pavões ou corujas. Mas há também registro arqueológico de vítimas
humanas, masculinas e femininas, mortas ora por decapitação,
estrangulamento, esfaqueamento ou quebra do pescoço, não se sabe se de
forma voluntária ou forçada. É o caso de uma das sepulturas de Birka,
onde o corpo decapitado de um jovem foi encontrado junto a outro de um
homem mais velho, ou de um túmulo próximo de Roskilde, na Dinamarca,
onde uma mulher foi enterrada juntamente com uma vítima humana masculina
cujo pescoço foi partido. Estes e outros casos fornecem um paralelo
físico para o episódio mitológico ou o relato escrito de Ibn Fadlan,
dando-lhes substância histórica.

Se algumas destas formas de sepultamento sugerem influência cristã, em
especial no caso de campas retangulares ou caixões, aconselha-se cautela
antes de se chegar a essa conclusão. Afinal, a simplicidade de algumas
sepulturas pode dever-se ao baixo estatuto social dos defuntos, que
podem ter tido um enterro singelo, desprovido de qualquer
monumentalidade e com uma colina fúnebre insipiente que o tempo apagou
com facilidade. Mesmo a ausência de oferendas, que é um traço
característico de uma sepultura cristã -- incluindo nas narrativas das
\emph{Sagas dos Islandeses} -- deve ser encarada com algum cuidado, pois também
se pode ter ficado a dever às poucas posses ou estatuto reduzido do
morto.

\SIG{Hélio Pires}

Ver também Funerais e enterros; Religião; Sociedade.

\begin{itemize}
\item \versal{AMBROSIANI}, Björn. Birka. In: \versal{BRINK}, Stefan; \versal{PRICE}, Neil (eds.).
\emph{The Viking World}. London/New York: Routledge, 2010, pp. 94-100.

\item \versal{ARTELIUS}, Tore; \versal{KRISTENSSON}, Anna. The universe container: projections
of religious meaning in a Viking Age burial-ground in northern Småland.
In: \versal{ANDRÉN}, Anders; \versal{JENNBERT}, Kristina; \versal{RAUDVERE}, Catharina (eds.).
\emph{Old Norse religion in long-term perspectives}. Lund: Northern
Academic Press, 2006, pp. 147-152.

\item \versal{BENNETT}, Lisa. Burial practices as sites of cultural memory in the
\emph{Íslendingasögur}. \emph{Viking and Medieval Scandinavia} 10, 2014,
pp. 27-52.

\item \versal{LUNDE}, Paul; \versal{STONE}, Caroline. \emph{Ibn Fadlan. Ibn Fadlan and the Land
of Darkness: Arab Travelers in the Far North}. London: Penguin, 2012.

\item \versal{PRICE}, Neil. Dying and the Dead. In: \versal{BRINK}, Stefan; \versal{PRICE}, Neil (eds.).
\emph{The Viking World}. London/New York: Routledge, 2010, pp. 257-273.
\end{itemize}
\section{\versal{SEXO E SEXUALIDADE}}

\emph{Conceito geral}: O comportamento sexual nórdico na Escandinávia
Medieval possuía um padrão duplo, tanto antes quanto depois da
cristianização. Segundo John Haywood, a sexualidade masculina poderia
ser concedida a mulheres adequadamente disponíveis (geralmente com alto
status social), enquanto a sexualidade feminina era vista como possessão
da sua família e sempre que possível, controlada. Isso não significa que
as necessidades sexuais das mulheres não fossem levadas em conta. O sexo
era considerado central para o casamento, e o fracasso de um homem em
corresponder às expectativas sexuais de sua esposa era motivo de
divórcio.

\emph{Fontes sobre a sexualidade nórdica antiga}: Segundo Jenny
Joschens, existem três principais grupos de fontes primárias para se
estudar a sexualidade: as leis, originalmente formadas durante o
paganismo, mas modificadas após a cristianização, especialmente o código
\emph{Grágás}; as sagas contemporâneas (que oferecem um quadro da
sociedade cristã dos séculos \versal{XII} e \versal{XIII}); as sagas de família,
tradicionalmente um material para se estudar a sociedade nórdica antiga,
mas com alguns problemas, visto que as informações são fragmentárias e
foram também filtradas pelo referencial cristão (a exemplo da ausência
de violência contra as mulheres e descrições de múltiplos parceiros
sexuais).

Os escritores europeus contemporâneos descreviam os homens escandinavos
da Era viking como tendo uma sexualidade relaxada. No século \versal{XI} o
cronista alemão Adão de Bremen reclamou das infidelidades do rei Svein
Estrithson, mas teve certa inclinação para perdoá-las, pois
considerava isso um lapso endógeno tanto dos dinamarqueses quanto dos
seus vizinhos suecos.

\emph{Escravas, concubinas e sexo}: As garotas escravas e as mulheres
nascidas livres que careciam de tutores masculinos, eram impelidas para
o sexo casual ou para relacionamentos extraconjugais a longo prazo.
Mulheres escravas eram obtidas em mercados de toda a Europa. Na
\emph{Laxdæla saga} 5 foi descrita a compra de uma bela escrava chamada
Melkorka durante a viagem do protagonista, com a qual acaba se
envolvendo amorosamente. O código de leis \emph{Grágás} 1 menciona que
um homem tem direito de comprar uma escrava para seu prazer corporal.

O cronista árabe Ibn Fadlan registrou no \emph{Risala} (920 d.C.)
relações casuais de nórdicos com uma escrava durante o funeral de um
chefe na região do Volga. Também leis islandesas do final da Era Viking
comentavam que garotas escravas eram vendidas para finalidades sexuais
por preços elevados. Uma descrição da \emph{Fljótsdæla saga} 6 sugere
que as escravas atendiam favores sexuais de visitantes.

Servas ocupavam um estrato social entre as escravas e as mulheres livres
e diversas narrativas indicam sua disponibilidade sexual entre
fazendeiros e viajantes. Na \emph{Grettis saga} 7 foi narrado que
Gretir, ao passar a noite em uma fazenda, teve um encontro sexual com
uma garota serviçal (\emph{griðkona}), enquanto a filha do fazendeiro
escapou.

As concubinas eram costumeiras na sociedade nórdica. Adão de Bremen
relata que os homens possuíam duas ou três ao mesmo tempo. As concubinas
geralmente eram de estratos sociais inferiores e eram beneficiadas tendo
relações com homens de categorias mais elevadas. Mas ela não poderia se
tornar esposa oficial, tendo que ser tolerada pela mulher do homem em
questão. Muitas concubinas provinham da escravidão.

Segundo Jenny Jochens, o fenômeno da prostituição foi essencialmente
urbano. Como a Islândia medieval foi baseada no mundo rural, ela
desconheceu essa prática.

\emph{Sexo, crime e tabu}: Para o historiador John Haywood o
comportamento sexual dos homens com relação às escravas e servas não era
estendido para outras mulheres da sociedade, como esposas, filhas,
irmãs, mães e mães adotivas protegidas pela honra familiar e caso alguém
fosse pego em flagrante delito, receberia severas penalidades. Os
assédios feitos por um homem a uma mulher solteira eram vistos com desprezo, e
se não seguissem uma proposta formal de casamento, o pai ou o tutor da
mulher poderia buscar vingança de sangue. Crimes sexuais atacavam diretamente a honra
da família da mulher.

O adultério era considerado pelas leis um crime mais grave que a
fornicação em si, mas era uma ocorrência contumaz naquela época.
Embora o adultério fosse cometido tanto por homens como mulheres, estas
últimas eram punidas mais severamente, especialmente após a
conversão cristã. Nesse momento, os homens culpavam as mulheres por seu
adultério, argumentando que "ela não foi fiel a mim''. Uma mulher podia
legalmente divorciar-se de seu marido durante a Era viking (e o seu dote
ser restituído) se o seu marido não a satisfizesse na cama, se ele
preferisse os homens ou se gostasse de usar roupas femininas.

Era esperado das mulheres solteiras que fossem virgens quando casassem,
segundo o que é mencionado em diversas sagas islandesas. Uma solteira
era um bem por meio do qual a família poderia obter muitas riquezas. O
preço político da noiva poderia trazer muitas alianças com outras
famílias poderosas. Já no casamento, a sociedade almejava que ela fosse
fiel ao seu marido. Na \emph{Edda Poética} é comum verificarmos diversos
tipos de insultos contra o comportamento promíscuo ou incestuoso de deusas
e personagens femininas, uma clara influência dos padrões sociais
vigentes. Outra razão do controle da sexualidade feminina se dava sobre a
natalidade; o risco de obter filhos ilegítimos poderia
significar muitas dificuldades para a família da mulher. As mulheres que
evitavam a gravidez não recebiam punição. Nos casos de estupro ou assédio, também não eram estigmatizadas.

A homossexualidade feminina e masculina, o incesto e a bestialidade
(zoofilia) eram consideradas ofensas altamente passíveis de punição.

\emph{Conceito de beleza e erotismo}: Ao contrário do mundo clássico,
não ocorreram distinções entre a beleza de homens e mulheres. Os mesmos
termos são usados para ambos os sexos. A descrição da beleza masculina
ou feminina nem sempre tinha um cunho erótico. A beleza feminina muitas
vezes era baseada na brancura dos braços das garotas e mulheres ou no
comprimento dos seus cabelos. A nudez praticamente não é mencionada nas
sagas islandesas, com algumas exceções -- como Freydís Eiríksdóttir, no
momento em que enfrenta alguns indígenas, mas sem cunho erótico. Uma
mulher grávida era considerada ``pesada'' e ficava mais ``leve e limpa''
após o parto. Também passava por cerimônias de limpeza e reentrada na
sociedade.

\emph{Romance}: Caso uma mulher tivesse interesse em um homem e fosse
correspondida, eles sentavam juntos e trocavam beijos (\emph{hana
kyssir}). Ele coloca a cabeça no colo dela e ela lava o seu cabelo. Em
recintos fechados, eles bebem no mesmo corno. Quando uma mulher casada
queria algo com o marido, ela colocava as mãos sobre seu pescoço, para
convencê-lo. Se um homem não aparecesse para seu casamento, era dito que
ele ``fugiu da vagina'' e ela ``fugiu do pênis''.

Um tema muito presente nas sagas islandesas (em 15 sagas, citado mais de
20 vezes) é a ``visita do amor ilícito'', o qual, segundo Jenny
Joschens, possivelmente tem raízes no mundo pagão. Essa relação de amor
ocorre quando um homem solteiro visita regularmente uma jovem, mas sem
realizar uma proposta formal de casamento, o que acaba gerando oposições
por parte da família e do tutor da moça. Em algumas sagas (como
\emph{Kormáks saga} 8) ocorrem assassinatos após a interdição, mas
geralmente os homens que assediam as moças são mortos pelo protagonista
ou herói da saga. Alguns pesquisadores (como Rolf Heller) negam esse
tipo de narrativa como sendo histórica, mas Jochens reconhece que o tema
era conhecido pelos autores cristãos e pode ter raízes pré-cristãs,
especialmente em se tratando de temas relacionados às relações
masculinas fora do casamento.

Em alguns momentos, as sagas islandesas descrevem iniciativas sexuais
partindo de mulheres jovens (\emph{Hallfreðar saga} 8; \emph{Vatnsdæla
saga} 8). A \emph{Ljósvetninga saga} 10 descreve a narrativa de uma
garota chamada Fridgerdr, que encontrava prazer na companhia de homens
problemáticos. A sexualidade é quase sempre implícita nas sagas, tendo
poucas referências de iniciativa feminina, sendo a maioria dos casos de
encontros sexuais como resultado de ações masculinas.

\emph{Pornografia}: As únicas descrições pornográficas presentes nas
fontes literárias em nórdico antigo são três capítulos da \emph{Bósa
saga ok Herrauðs} (7, 11 e 13). No primeiro caso, os personagens Bosi e
Heraud estão de passagem pelo interior e encontram uma casa de
camponeses, onde encontram abrigo para uma tormenta (cap. 7). Ali mora
uma senhora idosa e sua filha, muito atraente, que os recebe e troca
suas roupas molhadas, além de trazer cervejas para os dois. Pela noite,
Bosi vai na cama da garota e ela pergunta o que ele queria. Bosi responde
que quer tornar o seu ``conde'' duro com ela. A garota pergunta onde
está o conde e manuseia seu pênis, dizendo que ele é um monstro tão duro
quanto uma árvore e fizeram sexo várias vezes pela noite. Os dois amigos
partiram e chegaram então em Gautland (cap. 9) e se depararam novamente
com outra casa humilde pelo interior, onde um homem os recebeu (e ele
também tinha uma filha muito bonita). Ele aborda a garota pela noite,
ela pergunta o que ele queria e o mesmo responde que gostaria de regar o
seu potro no vinho dela. Ela segura o seu pênis, o acaricia e afirma que
sua cabeça não está bem colocada, mas que se tiver alguma coisa para
beber, vai melhorar. Ele ``afoga o potro'' diversas vezes, até que ela
reclama que sua fenda e até a sua cama estão muito molhadas. Bosi
responde que o potro ficou ``bêbado e vomitou'', pois estava ``doente de
cerveja''. Eles ainda se divertiram em cima e embaixo e a garota afirmou
que nunca havia montado um potro tão facilmente. A terceira descrição
(cap. 13) repete a mesma situação, Bosi e seu amigo chegam em uma casa
simples de camponeses, onde a filha solteira do casal é extremamente
bela. Ele aborda a moça durante a noite e ela pergunta qual a sua
intenção. Ele fala que quer introduzir um anel em sua cavidade. Após
introduzir o seu pênis no sexo da garota, ela responde que este prazer é
o mesmo que beber hidromel fresco.

Estas três narrativas constantes da saga foram influenciadas pelas cenas
mais picantes e sexuais dos \emph{fabliaux,} contos escritos na França
durante os séculos \versal{XII} e \versal{XIV} com forte ironia e crítica social. Todas as
três narrativas foram transcorridas na região de Bjarmaland (Finlândia).
Segundo Jenny Joschens, elas se referem não à forma de como a
sexualidade era vivida durante o paganismo, mas como ela foi percebida
durante o século \versal{XIV}. A sexualidade de Bosi é descrita de forma
grosseira e direta, concedendo pouco espaço para a imaginação do leitor.
Em todas as narrativas é Bosi que toma a iniciativa, apesar da imensa
satisfação das personagens femininas.

Algumas das metáforas utilizadas na \emph{Bósa saga ok Herrauðs} também
foram utilizadas na poesia escáldica como sinônimo para pênis, como o
cavalo e espada. Na \emph{Bjarnar saga}, o personagem Þórðr Kolbeinsson
declama um poema afirmando que deu prazer para a sua mulher (Oddný
eykyndill), fazendo o seu ``remo'' crescer duro dentro da vagina dela.

\emph{Sexualidade feminina}: As descrições de mulheres pagãs na
literatura nórdica geralmente não seguiram o modelo de desvio cristão,
perversidade sexual ou amoralidade. São muito mais frequentes as
descrições de beleza feminina do que apetite sexual. O modelo das sagas
é de personagens femininos sendo manejadoras do desejo masculino. Em
algumas obras nativas, observa-se certa influência dos romances da
literatura continental, como em \emph{Hávarðar saga Ísfirðings} e
\emph{Sörla þáttur}. Nas sagas contemporâneas, a mulher é
responsabilizada por quebrar os papéis estabelecidos pelo casamento na
sociedade pagã, mas agora modificados pela Igreja.

Um dos únicos casos de ninfomania registrado na literatura nórdica
medieval é o da rainha norueguesa Gunnhildr Gormsdóttir. Segundo a
\emph{Heimskringla}, Gunnhildr viveu certo tempo com os sámi
finlandeses, no norte escandinavo, com os quais aprendeu artes mágicas,
que em troca solicitavam favores sexuais. Após seduzir Érico, incitou-o a
matar estes sámi. A rainha foi amante do islandês Hrut Herjolfsson, que
era muito mais novo que ela. Em público, ela demonstrava seu amor sem
reservas. Mas também se enamorou de outro islandês, Ólafur Höskuldsson.

\emph{Virilidade e satisfação sexual}: Segundo Carl Phelpstead, os
guerreiros nórdicos eram sexualmente inseguros e angustiados com o
tamanho do pênis. Ser ridicularizado pelas mulheres comprometia sua
masculinidade, afetando diretamente sua vida social. O tamanho do pênis
indicava também o seu status social. O pênis demonstrava posição social
e também reafirmação e estabelecimento social pela agressão fálica. Um
homem com problemas no pênis não podia ter posição social dominante. O
tamanho do pênis também podia causar embaraços, como na \emph{Grettis
saga}, onde o personagem Grettir é surpreendido dormindo nu por duas
servas, que riem dele ao observar que seu pênis era muito pequeno. Ele
responde para as duas na forma de um poema, alegando que seu pênis ainda
era pequeno mas iria crescer e causar muitos problemas para a deusa do
sexo, Freyja. Mas o tamanho do órgão sexual não era suficiente -- era
necessário muita virilidade e desempenho. Na \emph{Saga de Njál}, uma
rainha ciumenta amaldiçoa o guerreiro Hrutr Herjolfsson pela enorme
ereção quando ele fez sexo com sua noiva.

Mulheres insatisfeitas com homens faziam estes perderem seu prestígio
político e social. Existiam hierarquias sociais conectadas a metáforas
de penetração na linguagem, como espadas, pênis e língua. Assim, quem
penetra com palavras, armas e a fala são os poderosos (homens) nos
fracos (fêmeas). No cenário social, a sexualidade prevê um simbolismo
para a dominação e submissão permitindo o \emph{status quo} na
sociedade.

\emph{Promiscuidade}: Fontes do século \versal{XIII} indicam o prevalecimento de
uniões informais e a tolerância de múltiplas companhias sexuais por
concubinagem. Esse comportamento era típico da realeza norueguesa. As
sagas reais revelam casos de pais que ofereciam suas filhas para favores
sexuais para reis dinamarqueses e noruegueses. A prática continuou após
a cristianização, especialmente na Noruega, causando problemas na
sucessão dinástica, o que ocasionou o aceite do papel eclesiástico da
monogamia para legitimar a sucessão. Persistindo na aristocracia da
Noruega e Islândia, essa prática sugere que originalmente a sexualidade
pagã era constituída fortemente por múltiplos parceiros. Também a
referência de concubinas e crianças ilegítimas nas sagas de famílias
sugere a existência de múltiplos parceiros entre fazendeiros, um
problema que frequentemente criava tensões.

\emph{Difamação sexual}: Mulheres promíscuas e heterossexuais
incestuosos eram denominados de \emph{ergi}. As sagas islandesas não
discutem ou referenciam mulheres homossexuais, mas o código de leis
religiosas da Islândia (Stock. Perg. 4to no. 15), datado do final do
século \versal{XII}, menciona e proíbe práticas com severas penitências, como o
homossexualismo masculino e o sexo com animais. A possibilidade de
mulheres fazerem sexo umas com as outras não fazia parte do mundo
narrativos dos autores das sagas. O substantivo \emph{ergi} ou os seus
adjetivos raramente eram usados para descrever mulheres, mas quando
aplicados, podiam significar ninfomania. Entre os homens, o adjetivo
\emph{ergi} comumente significava efeminado, indicando fraqueza e covardia.
Também a perda da virilidade masculina pode ser considerada um sinal de
\emph{ergi}, como na velhice.

Uma das formas mais espetaculares de difamação sexual era o \emph{níð},
presente na literatura e nos códigos de leis. Na \emph{Njáls saga} 123 o
personagem Flosi é acusado de ser pervertido por um trol. Mas a
acusação recai apenas sobre o homem violado pelo ato sexual masculino
(na metáfora de difamação). A prática da sodomia (consentida ou não) na
Era viking é algo que a pesquisa não consegue demarcar com eficiência,
sendo as acusações disso pelas fontes literárias totalmente simbólicas e
morais. Segundo o pesquisador Preben Sørensen existiriam três
significados para o termo \emph{arg/ragr:} perversidade sexual (ser
penetrado analmente); versado em feitiçaria; covarde/efeminado.

\SIG{Johnni Langer}

Ver também Estupro; Mulheres; Sociedade.

\begin{itemize}
\item \versal{CLOVER}, Carol J. Regardless of Sex: Men, Women, and Power in Early
Northern Europe. \emph{Representations} n. 44, 1993, pp. 01-28.

\item \versal{GADE}, Kare Ellen. Penile Puns: Personal Names and Phallic Symbols in
Skaldic Poetry. \emph{Essays in Medieval Studies} 6, 1990, pp. 57-65.

\item \versal{HAYWOOD}, John. Attitudes to sex. In: \emph{Encyclopaedia of the Viking
Age}. London: Thames and Hudson, 2000, p. 169.

\item \versal{JACOBSEN}, Grethe. Sexual Irregularities in Medieval Scandinavia. In:
 \versal{BULLOUGHS}, Vera \emph{et al} (ed.) \emph{Sexual Practices and the
Medieval Church}. Buffalo: Prometheus Books, 1982, pp. 72-85.

\item \versal{JOCHENS}, Jenny. The Church and Sexuality in Medieval Iceland.
\emph{Journal of Medieval History} 6, 1980, pp. 377-392.

\item \versal{JOCHENS}, Jenny M. The Illicit Love Visit: An Archaeology of Old Norse
Sexuality, \emph{\versal{JHS}} 1, 1991, pp. 357-392.

\item \versal{PHELPSTEAD}, Carl. Size Matters: penile problems in saga of icelanders.
\emph{Exemplaria} 19 (3), 2007, pp. 420-437.
\end{itemize}
\section{\versal{SIGTUNA}}

Localizada no leste da Suécia, ao norte do lago Mälaren, Sigtuna foi
fundada por volta da década de 980, substituindo a cidade de Birka como
centro econômico e político daquela região. Sua fundação é
tradicionalmente atribuída à iniciativa do rei Érico, o Vitorioso (Eiríkr
inn sigrsæli). A cidade no século~\versal{XI} despontou como centro comercial,
manufatureiro e político, pois se tornou sede do governo de Érico e de
seu filho Olavo, o Tesoureiro (Olof Skötkonung\textbf{)}.

A cidade foi erguida ao longo de uma grande via central em formato de \versal{S},
chamada de \emph{stora gatan}, que percorre o sentido leste-oeste
próximo à margem do lago. As casas em formato retangular ficavam
situadas de ambos os lados da avenida principal. Estima-se que a cidade
nos seus primeiros anos dispunha de cem casas. Com o tempo novas
moradias foram erguidas e novas ruas surgiram. Nesse sentido, Sigtuna
diferenciava do plano urbanístico de outras cidades nórdicas, as quais
adotavam um modelo mais ou menos circular e cercado por muros.

As escavações arqueológicas, iniciadas na segunda metade do século~\versal{XIX},
e continuadas ao longo do \versal{XX}, encontraram grande variedade de objetos,
feitos de distintos materiais e até mesmo importados, o que revela que
Sigtuna possuía contatos comerciais distantes, pois foram encontradas
joias de vidro, oriundas do leste europeu. James Graham-Campbell
assinala que oficinas escavadas em torno do salão real, revelam
construções amplas com armazéns, além de terem sido encontrados em
diversos substratos entre os séculos \versal{XI} e \versal{XIII}, objetos feitos de metal,
vidro, osso, chifre e até restos de tecidos.

Em geral a diversidade de matéria-prima não era algo comum na
Escandinávia, somente importantes polos manufatureiros e comerciais
dispunham de acesso a diferentes matérias-primas. Tal condição atesta
que Sigtuna de fato foi uma cidade economicamente importante no começo
da Baixa Idade Média. E isso também se reflete na condição de que a
cidade foi produtora de moedas, outra característica rara na
Escandinávia, pois pouquíssimas cidades possuíam casas da moeda.

No caso das moedas, algumas delas datadas do século \versal{XI}, apresentavam
cunhadas em alfabeto latino às palavras Siht, Stnete e Situn, que de
acordo com Jonas Ros, consistem em variações e abreviações do nome
Sigtuna. O rei Olavo, o Tesoureiro (c. 995-1022), após obter vitória
sobre o rei norueguês Olavo Tryggvason, com a ajuda do rei dinamarquês
Suevo Barba-bifurcada, ordenou que moedas comemorativas fossem cunhadas
em Sigtuna, cidade usada como capital real desde a época de seu pai.

As moedas cunhadas no governo do rei Olavo trazem inscrições que o
atestava como ``rei dos suevos'' e ``governador de Götar''. Seu filho e
sucessor Anund Jacob (c. 1022-1030/5) também continuou a ordenar a
cunhagem de moedas em Sigtuna. Posteriormente a produção foi suspensa,
sendo retomada pelo rei Canuto Eriksson (1167-1196). O fato de três reis
suecos terem ao longo do século \versal{XI} e \versal{XII} cunhado suas próprias moedas em
Sigtuna revela a importância não apenas econômica da cidade, mas também
política.

Entretanto, outro dado curioso de algumas dessas moedas era a referência
a cruzes, pois Olavo, o Tesoureiro, foi um rei cristão que incentivou a
expansão do cristianismo na Suécia. Nesse caso, James Graham-Campbell
salienta que um dos destaques arquitetônicos da cidade de Sigtuna diz
respeito a sua grande quantidade de igrejas. Estima-se que boa parte
desses templos foi erguida ainda no século \versal{XI}. O número de sete igrejas,
somado a sepulturas cristãs, revela que Sigtuna ainda em seus primórdios
já constituía em uma cidade cristianizada. Salientando que o processo de
cristianização na Suécia foi o mais tardio a começar, se comparado com a
Noruega e a Dinamarca.

Sigtuna continuou como um importante centro econômico e político até o
final do século \versal{XII}, quando foi atacada por invasores eslavos, que
queimaram a cidade. Mas posteriormente um novo núcleo urbano foi erguido
nas proximidades, consistindo na atual cidade de Sigtuna. Nesse sentido,
alguns historiadores preferem usar Velha Sigtuna (\emph{Fornsigtuna})
para se referir à cidade original, que foi destruída durante o governo
de Canuto Eriksson. O papel político de Sigtuna foi substituído com o
tempo por Gamla Uppsala, e sua função econômica foi perdendo espaço e
importância para Estocolmo.

\SIG{Leandro Vilar Oliveira}

Ver também Birka; Comércio; Suécia da Era Viking.

\begin{itemize}
\item \versal{GRAHAM-CAMPBELL}, James (org.). \emph{Os vikings}. Barcelona: Editora
Folio \versal{S.A.} 2006.

\item \versal{HOLMAN}, Katherine\emph{. Historical dictionary of the vikings}. Lanham:
Scarecrow Press Inc, 2003.

\item \versal{ROS}, Jonas. Sigtuna. In: \versal{BRINK}, Stefan; \versal{PRICE}, Neil (eds.). \emph{The
Viking World}. London/New York: Routledge, 2008, pp. 140-144.
\end{itemize}
\section{\versal{SIMBOLISMO ANIMAL}}

A Arqueologia, desde sua consolidação como disciplina independente no
século \versal{XIX}, vem se debruçando sobre as questões religiosas e seus
aspectos materiais. Com os trabalhos de André Leroi-Gourhan, em 1964, a
arte rupestre foi compreendida como o único meio de compreender o
simbolismo das sociedades paleolíticas, permitindo encontrar os limites
do uso cotidiano e simbólico. Desde então, a Arqueologia voltada a
religião foi-se multiplicando e se especializando até que em 1989 o
arqueólogo André Debord denomina esta subárea de \emph{Archéologie
religieuse.}

Nesta subárea, também chamada de Arqueologia da Religião em português,
os pesquisadores propõem um debate entre as fontes materiais e
literárias, de forma a encontrar um ponto de equilíbrio, onde os textos
rúnicos e a iconografia estão de um lado e a oralidade das eddas, sagas
e crônicas estão do outro. Desta maneira, analisar os simbolismos na
arte viking é dialogar com os rituais, os costumes, os mitos e a
cosmologia, estando sempre atento aos elementos da lenta transição para
o cristianismo na área nórdica.

Uma forma de se vislumbrar a cultura nórdica através da Arqueologia é a
análise dos elementos simbólicos em utensílios, ferramentas, joias,
armas, roupas, construções, ou seja, na cultura material. Muito do que
sobreviveu ao tempo nos mostra uma pista do gosto nórdico para a arte,
além de revelar um pouco de sua religiosidade, evidenciando a intrínseca
relação do escandinavo medieval com sua simbologia.

A iconografia nórdica estava recheada de símbolos que se relacionam
diretamente, ou indiretamente, com os deuses e outras narrativas
míticas. Muitas delas possuem uma origem possível de ser traçada desde
os petróglifos da Idade do Bronze, mas ainda assim, possuem um
significado muito próprio da cultura nórdica. Os animais compõem uma
importantíssima parcela deles, pois foram largamente utilizados para
expressar seus distintos estilos artísticos. Características físicas e
comportamentais dos mais diversos animais, tanto selvagens quanto
domésticos, foram exploradas pelos artistas escandinavos, trazendo junto
de suas aparências uma série de informações subjetivas simbólicas.

As aves possuem um simbolismo em comum entre as várias espécies. De modo
geral, são consideradas fontes de conhecimento; poderiam fazer um
homem mais sábio, além do que, com a habilidade de viajar entre os
planos, estavam como intermediárias entre os deuses, os humanos e os
mortos. Através delas, se podia ter uma proteção mágica,
alcançar o mundo dos deuses e barganhar a vida ou a morte de alguém.
Logo sua associação ao poder foi apropriada por uma elite social que
necessita de legitimação para assegurar sua posição e então tornaram-se
signos de sabedoria, de favor divino e de nobreza.

Quanto a presença de mamíferos na iconografia nórdica, seu
simbolismo se mostra complexo e extremamente vasto. Os bovinos, por
exemplo, possuem uma origem iconográfica já na Idade do Bronze e estavam
associados à fertilidade através da agricultura, pecuária e até pela
vertente de virilidade, o que lhe concedia um certo sentido bélico,
devido seus atributos de força física e seus chifres pontiagudos, o que
contribuiu para o equívoco do uso de elmos com cornos por vikings.
Contudo, na Era Viking, ele já não é representado com frequência e seu
sentido fica mais restrito as questões econômicas, demonstrando a
fortuna acumulada de um proprietário de terras. Esta relação é
evidenciada pela própria palavra nórdica para gado, \emph{Fé} (ou
\emph{Fehu}, que em proto-nórdico quer dizer dinheiro, gado e riqueza).

Como outros exemplos dessa pluralidade, temos as múltiplas facetas do
cavalo nos mitos, sendo esse um elemento transicional entre os limites
do doméstico e do selvagem e outras fronteiras cósmicas, bem como
simbolicamente associados à marcialidade, à virilidade e à fertilidade.
A seu turno, os lobos são a representação de um dos maiores temores dos
homens escandinavos, a morte, mas constituem grande fonte de inspiração
para seu culto guerreiro. Através do domínio destas e outras feras, como
os ursos, demonstram um desejo de manter segura a ordem cósmica e ter
algum controle sobre ela, representada, material e mentalmente, pelas
forças da natureza e do destino.

Ainda acerca da ordem cósmica, temos os ofídios, que, segundo as
narrativas míticas, possuem papéis essenciais para a estabilidade e para
o caos. Além disso, também representam o submundo, tanto marinho quanto
terreno, sendo habitantes e guardiãs desses meios. Serpentes, também
presentes em contextos de morte, tortura ou sendo pisoteadas por
guerreiros, simbolizam tanto a morte certa quanto a alegoria do domínio
humano sobre as condições da natureza.

\SIG{Ricardo Wagner Menezes de Oliveira}

Ver também Arte; Noruega da Era Viking; Religião; Sociedade.

\begin{itemize}
\item \versal{BOURNS}, Timothy. \emph{The Language of Birds in Old Norse Tradition}.
Dissertação de Mestrado, Universidade da Islândia, 2012.

\item \versal{EINARSDÓTTIR}, Katrín Sif. \emph{The Role of Horses in the Old Norse
Sources}: Transcending worlds, mortality, and reality. Dissertação de
Mestrado, Universidade da Islândia, 2013.

\item \versal{GRÄSLUND}, Anne-Sofie. The material culture of Old Norse Religion. In:
 \versal{BRINK}, Stefan; \versal{PRICE}, Neil (eds.). \emph{The Viking World}. London:
Routledge, 2008b, pp. 249-256.

\item \versal{GRÄSLUND}, Anne-Sofie. Wolves, serpents, and birds: their symbolism
meaning in Old Norse beliefs. In: \versal{ANDRÉN}, Anders; \versal{JENNBERT}, Kristina;
 \versal{RAUDVERE}, Catharina (eds.). \emph{Old Norse Religion in long-term
perspectives}: origins, changes, and interactions. Lund: Nordic Academic
Press, 2004, pp. 124-129.

\item \versal{LANGER}, Johnni (org.). \emph{Dicionário de Mitologia Nórdica}. São
Paulo: Editora Hedra, 2015.

\item \versal{OLIVEIRA}, Ricardo Wagner Menezes de. \emph{Feras petrificadas}: O
simbolismo religioso dos animais na era viking, 127 f. Dissertação de
Mestrado em Ciências das Religiões. João Pessoa: Universidade Federal da
Paraíba, 2016.

\item \versal{PLUSKOWSKI}, Aleksander. Harnessing the hunger: Religious appropriations
of animal predation in early medieval Scandinavia. In: \versal{ANDRÉN}, Anders;
 \versal{JENNBERT}, Kristina; \versal{RAUDVERE}, Catharina (eds.). \emph{Old Norse Religion
in long-term perspectives}: origins, changes, and interactions. Lund:
Nordic Academic Press, 2006, pp. 119-123.
\end{itemize}
\section{\versal{SOCIEDADE}}

\emph{Características gerais:} Não sobreviveram fontes contemporâneas
sobre a sociedade nórdica na Era viking. Os pesquisadores utilizam
informações fragmentadas sobre o tema em inscrições rúnicas e em fontes
posteriores, como documentos jurídicos, sagas islandesas e documentos
estrangeiros, além de reconstituições realizadas pela Arqueologia. De
forma geral, a sociedade nórdica na Era viking era dividida em duas
categorias principais: homens livres e não livres. Os homens livres
tinham o direito de portar armas e falar nas assembleias locais, além de
serem protegidos pela lei. Mas a sociedade não era igualitária, havendo
profundas diferenças sociais e econômicas. Essa desigualdade era
refletida na escala de compensação paga para compensar um assassinato:
os mais ricos influenciavam as vítimas, sendo a maior quantidade paga
para a família da parte culpada. Ou ainda pode ser exemplificada no
tratamento após a morte: enquanto poucos recebiam um funeral e
enterramento grandioso do ponto de vista material e social, grande parte
possuía um enterro modesto e alguns medíocres, como os escravos.

O poema éddico \emph{Righstula} apresenta um quadro social do mundo
nórdico dividido em três categorias sociais: \emph{jarl} (nobre), \emph{karl}
(fazendeiro) e \emph{thraell} (escravo). Na prática, a situação era muito mais
complexa, pois a sociedade da Era viking era hierarquizada, mas não
necessariamente estática, sendo que o poder ou a ousadia podia modificar
a situação de um indivíduo: riquezas e status como consequência da
pirataria e comércio ou ainda, serviços a um rei.

\emph{Os homens livres:} Para o historiador John Haywood, somente na
Islândia os homens livres eram tratados com uniformidade no assunto das
compensações por vendeta. A categoria social mais numerosa entre os
homens livres na Era viking foi a dos fazendeiros (ver verbete
\emph{bóndi}). Também existiam homens livres respeitados que não
pertenciam à aristocracia, classificados como \emph{drengr} (rapazes) ou
\emph{thegn} (guerreiro, ver verbete). Segundo Katherine Holman,
originalmente o termo \emph{thegn} significava simplesmente homem livre,
proprietário de terras ou um guerreiro. Posteriormente, passou a ter
influências da área britânica e passou a ter um sentido associado aos
serviços militares para um rei.

Alguns homens livres que não possuíam propriedades nem terras arrendadas
e eram empregados como trabalhadores nas fazendas. Outros ganhavam a
vida trabalhando diariamente como artesãos nas habitações, incluindo
construção de navios e ferraria, mas o seu número era muito pequeno em
relação à demais categorias sociais. Algumas atividades ocupavam meio
período ou tinham funções parciais, como poetas, médicos, sacerdotes,
escultores de pedras, guerreiros e mercadores.

Também existiam pessoas livres que eram muito pobres e mesmo ocorriam
vagabundos. Ambos eram equiparados ao mais baixo escalão da sociedade,
os escravos, não tendo residência e nem direitos jurídicos. Também não
era permitido a pobres e vagabundos a concessão de casamentos e alguns
eram castrados em determinadas penalidades. Vagabundos eram proibidos de
pedir comida durante as assembleias e a lei permitia que fossem
retirados, desde que não sofressem nenhuma injuria permanente.

\emph{A escravidão:} Os escravos eram o mais baixo patamar da sociedade.
Eles eram bens considerados móveis, com direitos mínimos. Suas únicas
relações com o resto da sociedade eram definidas por seu proprietário.
Não podiam possuir nenhum tipo de herança e nem legar bens, bem como
participar de transações comerciais. Os escravos podiam ser condenados à
morte se já estivessem muito velhos, doentes ou incapacitados para o
trabalho. Nas sagas islandesas, os escravos eram descritos como
covardes, estúpidos, tolos e duvidosos.

Escravos que ganhavam a liberdade eram ostensivamente livres, mas o seu
\emph{status} continuava extremamente baixo. As crianças de escravos
libertos eram completamente livres na Islândia, ao contrário de outras
regiões da Escandinávia, onde podiam levar até quatro gerações para se
libertar da escravidão. Alguns dos poucos direitos que os escravos
podiam ter era acumular algum tipo de propriedade e com o tempo, comprar
a sua liberdade. Podiam casar e era permitida a vingança quando se
tratava de sua esposa. Nunca existiu uma economia predominantemente
agrícola no mundo nórdico. Os escravos eram geralmente requeridos em
trabalhos nas fazendas. O trabalho escravo desapareceu da Escandinávia
após o século \versal{XI}.

\emph{O mundo aristocrático:} A aristocracia hereditária exercia
considerável influência em cada região e nas assembleias locais. Esse
poder era originado tanto da propriedade das terras quanto do
oferecimento de proteção para as pessoas sem influência e como retorno
do seu suporte político e militar. Líderes locais, como os \emph{hersar}
na Noruega, formavam uma das camadas mais altas da aristocracia.
Conhecidos posteriormente como \emph{lendrmadr} (latifundiários), eles
exerciam autoridade nos favorecidos do rei, tendo alto prestígio no
\emph{hirð} e atuavam como comandantes nas expedições militares
marítimas. Altas somas de compensação eram destinadas as suas famílias
no caso de alguém ter sido assassinado ou ferido.

Na Escandinávia da Era viking foi desenvolvida uma pequena classe de
grandes governantes que tiveram o título de \emph{Jarl} (nobres), que
significava provavelmente apenas homem de prestígio. Na Noruega a
compensação para o assassinato de um \emph{jarl} era duas vezes o preço de um
homem comum e a metade de um rei. Os grandes jarlar, como Hladir na
Noruega e Orkney, eram considerados governantes que exerciam poder real
sobre o seu território. A resistência dos jarlar à centralização do
poder da realeza provocou um sério obstáculo para a criação de uma
Noruega unificada.

Segundo a arqueóloga Else Roesdahl, os acadêmicos conhecem muito pouco
sobre a categoria intermediária que existia nesta sociedade, sua
mobilidade e intercalamento com os outros grupos, sejam eles mais ricos
ou mais pobres. Existiam muitas pessoas pobres (que não eram
necessariamente escravas). A maioria dos escravos eram obtidos no
exterior e alguns podiam alcançar a sua liberdade através dos seus
proprietários.

Na Islândia da Era viking as diferenças sociais parecem ter sido menores
do que em outras regiões da Escandinávia. Pesquisas em sepultamentos
indicam que não ocorreram o uso de grandes montículos funerários ou
enterros em embarcações ou mesmo a demarcação do alto status por meio de
extravagância material. A maior categoria dos homens livres na Islândia
foi a dos \emph{goðar}. No início, eles eram apenas fazendeiros livres,
mas sendo os primeiros entre seus pares. A figura do \emph{goði} (ver verbete)
era o ponto de contato entre seus seguidores e os poderes regionais e
nacionais de governo e era sempre o primeiro homem para resolver
diversas questões. Ele era o primeiro homem e o líder de um distrito. O
\emph{goði} também servia como sacerdote da antiga religião nórdica e possuía
uma relação especial com os deuses.

\SIG{Johnni langer}

Ver também Bóndi; Cotidiano; Goði; Mulheres; Realeza; Sexo e
sexualidade.

\begin{itemize}
\item \versal{BOYER}, Régis. Les structures de la societé viking. \emph{Les vikings}.
Paris: Perrin, pp. 255-279.

\item \versal{CHIESA}, Gianna. Cultura e societá. \emph{Storia e cultura dela
Scandinavia}: uomini e mondi del Nord. Milano: Bompiani, 2015, pp.
203-232.

\item \versal{HAYWOOD}, John. Social classes. \emph{Encyclopaedia of the Viking Age}.
London: Thames and Hudson, 2000, pp. 180-181.

\item \versal{ROESDAHL}, Else. Society. In: \emph{The vikings}. London: Penguin Books,
1998, pp. 52-64.

\item \versal{SHORT}, William R. Social structure and gender. In: \emph{Icelanders in
the Viking Age}. London: McFarland, 2010, pp. 32-40.
\end{itemize}
\section{\versal{SONATORREK}}

\emph{Sonatorrek} é um poema escáldico de lamentação composto por Egill
Skallagrímsson (910-990 d.C.), que é o principal poeta escáldico de
origem islandesa. De acordo com Simek \& Pálsson (1987, p. 64), já na
Idade Média, a reputação desse poeta era tanta que o próprio historiador
e político Snorri Sturluson (1179-1241 d.C.) compôs uma saga sobre a
vida dele, a \emph{Egils saga Skállagrímssonar.} Em alguns dos
manuscritos dessa saga se encontra o presente poema.

Embora a primeira estrofe esteja registrada nos principais manuscritos
da \emph{Saga sobre Egill Skallagrímsson} (que é uma das únicas obras
que contém esse poema) como, por exemplo, o manuscrito \versal{AM} 132 fol.
(conhecido como \emph{Möðruvallabók}, Reykjavik, c. 1330-1370), apenas
nos manuscritos \versal{AM} 453 (conhecido como \emph{Ketilsbók}, por conta do
copiador Ketill Jörundsson) e \versal{AM}462 4to. todo o poema está composto.
Para os exemplos dados neste artigo utilizamos as obras de Jónsson
(1967; 1973), que apresentam os poemas na forma diplomática e
interpretativa. Apresentaremos as duas variações neste artigo. Jónsson
utiliza o manuscrito \versal{AM}453 em sua obra (1967, p. 40)

O nome do poema significa ``a perda árdua dos filhos'', cujo primeiro
elemento \emph{sona} e o segundo elemento \emph{torrek} significam
``filhos'' e ``perda'', respectivamente. No tocante ao segundo elemento
\emph{torrek}, Magnússon (2008) afirma que é uma composição entre o
prefixo \emph{tor}- ``difícil'' (compare com o adjetivo do islandês
moderno \emph{torleystur} ``difícil de desprender, solucionar'') e
\emph{rek}, que seria provavelmente uma derivação do verbo \emph{reka}
(germânico antigo *\emph{wrekan}), no sentido de ``empurrar,
afugentar''.

O poema trata sobre a difícil perda de dois filhos do poeta: Gunnar, que
morreu de febre e Böðvarr, que morreu em um naufrágio. Embora os nomes
dos filhos não sejam citados no poema, Snorri Sturluson os cita na saga:
\emph{Egill hafði þá átt son er Gunnar hét ok hafði sá ok andazk litlu
áðr} (p. 146) ``Egill tinha tido um outro filho que se chamava Gunnar e
que morreu um pouco antes'' e \emph{lauk þar svá at skipit kafði undir
þeim ok týndusk þeir allir. En eptir um daginn skaut upp líkunum}
``ocorreu que o navio afundou e todos morreram. No dia seguinte os
corpos apareceram`` (\versal{EINARSSON}, 2003, p. 145-146; trad. nossa).

De acordo com Turville-Petre (1976), o poema pode ser dividido em sete
partes:

Entre as estrofes 1 e 4, o eu-lírico se empenha para achar palavras que
correspondam à sua tristeza. Na primeira estrofe: \emph{era nü vænlegt
um vidris þife} {[}\emph{esa nú vænligt of Viðurs þýfi}{]} ``agora há
pouca esperança a respeito do roubo de Viður''. O ``roubo de Viður'', ou
seja, o ``roubo de Odin'', é um epiteto que faz referência à poesia.
Esse epiteto ocorre na obra \emph{Edda em prosa}, capítulo
\emph{Skáldskaparmál}, em que Odin rouba o hidromel da poesia do gigante
Suttungr e, em seguida, cede aos deuses e aos homens dotados de poesia:
\emph{Þa braz hann iarnar ham og fláug sem akafazt} {[}...{]} \emph{En
Svttvnga-mioð gaf Oþin asvnvm ok þeim monnvm, er yrkia kvnv}
(transcrições do manuscrito \emph{Codex Regius} por Finnur Johnsson,
1931, p. 85). Uma versão padronizada foi realizada por Anthony Faulkes
(1998): \emph{þá brásk hann í arnarham og flaug sem ákafast} ``e então
se transformou na forma de uma águia e voou'' (trad. nossa) e \emph{en
Suttung mjǫð gaf Óðinn Ásunum ok þeim mǫnnum, er yrkja kunnu} ``mas Odin
cedeu o hidromel de Suttungr aos \emph{æsir} e aos homens que podiam
fazer poesia'' (trad. nossa). Como o poeta estava triste naquele
momento, havia pouca esperança para compor um ``roubo de Odin''.

Na segunda parte da divisão de Turville-Petre, entre as estrofes 5 e 12,
Egill se lamenta por conta da morte de Böðvarr: \emph{Grimt var um hlid,
þat er hraun um braut faudr mïns ä frændgarde; veit eg ofullt ok opid
standa sonar skard, es mier siär um vann} [\emph{Grimt vǫrum hlið, þat
`s hrǫnn of braut fǫður míns á frændgarði; veitk ófult ok opit standa
sonar skarð, es mér sær of vann}] ``Horrível foi a pancada da onda,
que abriu um buraco na cerca dos descendentes do meu pai; vejo o não
preenchido e a abertura que repousa; a fissura (falta) do filho que o
mar me causou''. No mesmo trecho, na estrofe 7, o eu-lírico afirma que
Rán, deusa que governa o mar, o devastou: \emph{miauk hefur rän riskt um
mig} [\emph{mjǫk hefr Rǫ́n of rysktan mik}]; e também, se ele se
vingasse com sua espada, a vida dos forjadores da cerveja se acabaria:
veiztü um \emph{þä sauk, sverde of ræag var aulsmid allra tïma}
[\emph{veizt ef sǫk, sverði of rækak, vas ǫlsmið allra tíma}]. Aqui
o poeta utiliza a palavra que corresponde ao tipo \emph{ale}. O problema
é que ele está velho e não tem seguidores para apoiá-lo em tal
propósito: \emph{þvïat alþiöd firi algum verdr gamals þegns gengeleise}
[\emph{þvít alþjóð fyr augum verðr gamals þegns gengileysi}].

Na terceira parte, entre as estrofes 13 e 19, a morte do irmão mais
velho de Egill é lembrada: \emph{opt kiemur mier mana biarnar ï birvind
brædraleise} {[}\emph{oft kømr mér mána brúðar í byrvind bræðraleysi}{]}
``sempre vem a mim na brisa da navegação da noiva da lua, a falta de meu
irmão''. Neste trecho percebe-se que Jónsson corrige \emph{biarnar} ``do
urso'' por \emph{brúðar} ``da noiva''\emph{.} De acordo com Egilsson
(1931, p. 67, 73 e 396), \emph{a noiva da lua} é um \emph{kenning} para
``giganta'' e a \emph{brisa da navegação da giganta} é um \emph{kenning}
para ``mente'', portanto, ``sempre vem a mim, na mente, a falta de meu
irmão''. Na quarta parte, entre as estrofes 20 e 21, o poeta lembra de
Gunnar, seu primeiro filho, que morreu de febre: \emph{Sizt son minn
sottar brïme heiptuglegr ür heime nam} {[}\emph{Síz son minn sóttar
brími heiptugligr ór heimi nam}{]} ``desde que o fogo vingativo da
doença tomou o meu filho desse mundo''.

Na penúltima parte, entre as estrofes 22 e 24, o poeta ataca Odin, com
quem ele tinha boa relação até o deus quebrar os termos de amizade:
\emph{Ätta eg gott vid geira drotinn} {[}\emph{Áttak gótt við geirs
dróttin}{]} ``tinha boas relações com o senhor da lança''; \emph{adr
umat vagna runne sigur haufunde um sleit vid mig} {[}\emph{áðr vinan
vagna rúni, sigrhǫfundr, of sleit við mik}{]} ``até o confidente das
carruagens, o senhor da vitória, quebrar a amizade''. Os
\emph{kenningar} ``senhor da lança'', ``confidente das carruagens e
``senhor da vitória'' fazem referência a Odin (\versal{EGILSSON}, 1931, p. 178,
473, 494, respectivamente) e mostra que Egill tem uma devoção por Odin,
o deus da poesia (\versal{NORTH}, 1990, p. 289). No entanto, Odin compensou Egill
com duas habilidades: \emph{ïþrot .... vamme firda} {[}\emph{íþrótt ...
vammi firða}{]} ``arte sem erros'' e \emph{er ge giórda mier} \emph{vïsa
fiandr ad velaundum} {[}\emph{es gerðak mér vísa fjandr af vélǫndum}{]}
``me permitiu desmascarar trapaceiros e transformá-los em inimigos
públicos``, quer dizer, a ``arte de fazer poesia'' e a ``capacidade de
desmascarar inimigos''. Na estrofe 25, a última, Egill aceita em paz sua
perda e espera a morte.

De acordo com North (1990, p. 289), Egill adaptou sua tragédia a um
gênero e não o gênero à sua tragédia; além do mais, o poema mostra que a
fé do eu-lírico é solidamente pagã. Um elemento pagão neste poema é a
devoção de Egill a Ódin, deus da poesia, na estrofe 22. Também é
plausível que, assim como outros mercenários de seu tempo, Egill olhava
para Odin como um reflexo de sua vida poética e bélica. Nordal (1961)
sugere que Egill cresceu no culto islandês a Thor, deus da agricultura,
antes de se mudar para o exterior e começar a idolatrar Odin; no
entanto, no final da vida de Egill, Odin permitiu (ou causou) a morte de
Böðvarr e, portanto, ocorreu uma traição. O autor afirma que, a partir
de então, se iniciou um antagonismo entre Thor e Odin, mas apenas Odin
parece ter sido culpado pela morte de Böðvarr, sendo o mar visto como um
capanga das ordens de Odin; também é possível que Odin deliberadamente
falhou em retirá-los do curso e, então, evitar o afogamento (\emph{apud}
\versal{NORTH}, 1990, p. 289-290). Além disso, o eu-lírico percebe que nem
poderia vingar a morte de seu filho, pois, além de ele não ter mais
seguidores, seria impossível enfrentar Rán e Ægir, já que são entidades
do mar. North (1990) questiona se essa tragédia testa a fé do eu-lírico,
Egill. De fato, como é demonstrado no poema, Egill sofreu uma crise em
sua fé, uma vez que acusa Odin (na estrofe 22) de quebrar a amizade
entre eles; porém, apesar da amizade ter acabado, as obrigações ainda
continuaram e o eu-lírico até mesmo admite que Odin, em troca pela perda
do filho, o compensou com habilidades (p. 291-292). Portanto, o autor
afirma que Egill não tem nenhuma crise religiosa para enfrentar, pois
suas crenças pagãs são para ajudá-lo e não são questionáveis; elas
também fazem com que ele experimente a catarse da elegia, um gênero
feminino, como uma alternativa para a vingança (p. 299).

No que diz respeito à métrica do poema, ele esta em \emph{kviðuháttr}
(\versal{ROSS}, 2005, p. 23; \versal{POOLE}, 2005, p. 267), cuja métrica tem nos versos
ímpares três sílabas e nos versos pares, quatro sílabas; e, além do
mais, não há rima interna (consulte a entrada \textbf{Poesia
Escáldica}). Exemplificaremos com a última estrofe do poema, a número
25: \emph{Nú erum \textbf{t}orvelt // \textbf{T}veggja bága;
\textbf{n}jǫrva \textbf{n}ipt // á \textbf{n}ési stendr; skalk þó
\textbf{g}laðr // \textbf{g}óðum vilja; ok ó-\textbf{h}ryggr //
\textbf{h}eljar bíða}; com a seguinte tradução apenas do conteúdo
proposta: ``Agora está difícil para mim. A irmã do inimigo do Tveggi
está lá no promontório. Feliz, com boa vontade e sem preocupação espero
pela morte''. Há \emph{kenningar} no poema e, particularmente nesta
estrofe: \emph{Tveggja bági} ``inimigo do Tveggi'' = [\versal{FENRIR}].
\emph{Tveggi} é um \emph{heiti} para Odin (compare o poema escáldico
\emph{Óðins nöfn}, estrofe 8; e
~\href{https://en.wikipedia.org/wiki/Völuspá}{\emph{\emph{Völuspá}}},
estrofe 63). Portanto, \emph{nipt tveggja bága} ``irmã do [\versal{FENRIR}]``
= [\versal{HEL}], a deusa do reino dos mortos. Estas interpretações tiveram
como base Egilsson (1931). Para saber mais sobre os \emph{kenningar},
consulte a entrada Kenning.

\SIG{Yuri Fabri Venancio}

\SIG{Ver também Egils saga; Heiti; Kenning; Literatura; Poesia escáldica.}

\begin{itemize}
\item \versal{EGILSSON}, Sveinbjörn. \emph{Lexicon Poeticum Antiquæ Linguæ
Septentrionalis. Ordbog over det norske-islandske Skjaldesprog. Forøget
og udgivet for det kongelige nordiske Oldskriftselskab}. 2 Udgave ved
Finnur Jónsson. København: S. L. Møllers Bogtrykkeri, 1931

\item \versal{EINARSSON}, Bjarni. \emph{Egils Saga}. London: Viking Society for
Northern Research, 2003.

\item \versal{FAULKES}, Anthony. \emph{Edda. Skáldskaparmál. 1. Introduction, Text and
Notes}. London: Viking Society for Northern Research, 1998.

\item  \versal{JÓNSSON}, Finnur. \emph{Edda Snorra Sturlusonar af komissionaren for det
Arnamagnæanske legat}. København: Gyldendalske Boghandel -- Nordisk
Forlag, 1931.

\item  \versal{JÓNSSON}, Finnur. \emph{Den Norsk-Islandske Skjaldedigtning. A. Tekster
efter Håndskrifterne}. Første Bind. København: Rosenkilde og Bagger,
1967.

\item  \versal{JÓNSSON}, Finnur\emph{. Den Norsk-Islandske Skjaldedigtning. B. Rettet
Tekst}. Første Bind. København: Rosenkilde og Bagger, 1973.

\item \versal{MAGNÚSSON}, Ásgeir Blöndal. \emph{Íslensk orðsifjabók. Reykjavik}: Orðbók
Háskólans, 2008.

\item \versal{NORTH}, Richard. The Pagan Inheritance of Egill's~Sonatorrek\emph{.}
In:\emph{~Poetry in the Scandinavian Middle Ages (7th International Saga
Conference).} Spoleto: Presso la sede del Centro studi, \versal{LCCN} 90178700,
pp.~147-167.

\item \versal{POOLE}, Russell. Metre and Metrics. In: \versal{McTURK}, Rory (ed.). \emph{A
Companion to Old Norse-Icelandic Literature}. Malden/Oxford/Victoria:
Blackwell Publishing Ltd, 2005, pp. 265-284.

\item \versal{ROSS}, Margaret Clunies. \emph{A History of Old Norse Poetry and
Poetics}. Cambridge: D. S. Brewer, 2005.

\item \versal{SIMEK}, Rudolf; \versal{PÁLSSON}, Hermann. \emph{Lexikon der altnodischen
Literatur}. Stuttgart: Alfred Kröner,
1987.\\[2\baselineskip]\versal{TURVILLE-PETRE}, Gabriel. \emph{Scaldic Poetry}.
Oxford: Clarendon Press, 1976.
\end{itemize}
\section{\versal{STARAJA LADOGA}}

Staraia Ladoga (Velha Ladoga, em russo) é o nome dado a um dos mais
antigos assentamentos escandinavos na Rússia europeia e hoje um
importante sítio arqueológico. Até o século \versal{XVIII} a região era conhecida
somente como Ládoga, sem o adjetivo. Ficava localizada ao longo rio
Volkhov e próxima aos lagos Ládoga e Ílmen, na zona florestal do
centro-norte da Rússia. Dependendo do sistema de transliteração, a área
é igualmente escrita como "Staraja Ladoga" ou "Staraya Ladoga". Em
fontes escandinavas o local era conhecido como \emph{Aldeigja} ou
\emph{Aldeigjuborg}. Atualmente encontra-se no Oblasto de Leningrado no
noroeste russo. Juntamente com Kiev e Nóvgorod, Stáraia Ládoga foi uma
das primeiras cidades fundadas pelos varegues na Rússia, e seria
conforme especialistas a primeira capital dos nórdicos que se assentaram
na região, sendo estabelecida em meados do século \versal{VIII} e o provável
local de retorno dos Rus presentes nos \emph{Annales Bertiniani}.

Objetos escandinavos datados do século \versal{VII} foram encontrados na região,
mas evidências de um assentamento efetivo viking em Ladoga só são
documentadas a partir de 750. Antes dos nórdicos, a área era ocupada por
povos fino-úgricos e baltos, mas eventualmente eslavos do norte ocuparam
a região. Segundo Wladislaw Duczko, a área foi escolhida pela
possibilidade de acessar diversas rotas fluviais a partir do rio Volkhov
como o Dniepre e o Volga que são cortados por rios menores provenientes
do lago Ilmen. Os varegues chegaram em Ládoga, de acordo com Jonathan
Shepard, em busca da prata utilizada pelo Califado Abássida em seus
\emph{dirhams,} e lá permaneceram por causa das vantagens geográficas e
pelo domínio da rota do rio Volkhov\emph{.}

Ladoga funcionava primariamente como um posto comercial, e tinha uma
forte atividade mercante ainda no século \versal{VIII}, com várias moedas do
Califado Abássida, datando de aproximadamente 786, tendo sido encontradas
na região, mostrando um laço comercial forte com os árabes. Além da
prata, foram descobertos diversos pentes, pingentes e contas de vidro,
indicando que um dos setores predominantes da economia de Ladoga seria o
comércio de produtos manuais e artesanato. Havia também uma grande
quantidade de âmbar e animais cujas peles eram utilizadas no comércio de
luxo na região, provavelmente explorados e sendo utilizado com fins
comerciais. Shepard afirma que algumas mercadorias de Ladoga chegaram a
diversas partes do Ocidente como a Frísia e a Germânia, e objetos
provenientes da Irlanda e Bretanha foram encontrados em Ladoga. É
possível que os postos comerciais de Birka (na atual Suécia) e Hedeby
(na atual Alemanha) e o posto de Ladoga tivessem conexões.

Ladoga não foi muito mencionada por fontes de Rus. Edições posteriores
da \emph{Crônica dos Anos Passados} fazem alusão ao território como
sendo a primeira cidade em que o varegue Riurik se fixou em Rus. De
acordo com as sagas, o jarl norueguês Eirík Hákonarsson conquistou e
destruiu Ladoga em 997, quando Vladimir~\versal{I} Sviatoslavich de Kiev
(980-1015) era príncipe de Kiev. Escandinavos suecos também governaram
Ladoga a partir de Iaroslav Vladimirovich, o Sábio (1016-1018,
1019-1054), quando este deu a sua esposa Ingigerth o controle da cidade.
A princesa sueca por sua vez delegou a cidade aos seu jarl Rognvald, e
provavelmente uma microdinastia sueca assumiu o controle logo após sua
morte. O rei norueguês Haroldo Hardrada (1046-1066) e sua esposa
Elizaveta Iaroslavna, filha de Iarosláv e Ingigerth e conhecida na
\emph{Saga de Haraldr Sigurtharson} como Ellisif, passaram por Ladoga no
caminho para a Suécia no século~\versal{XI}, indicando que a rota ainda existia.
Ladoga eventualmente foi absorvida por Velikii Novgorod por volta do
século~\versal{XII}.

\SIG{Leandro César Santana Neves}

Ver também: Crônica dos Anos Passados; Kiev; Novgorod; Rus; Rússia da
Era Viking; Varegues.

\begin{itemize}
\item \versal{DUCZKO}, Wladsyslaw. \emph{Viking Rus: studies on the presence of
Scandinavians in Eastern Europe.} Leiden: Koninklijke Brill \versal{NV}, 2004.

\item \versal{FRANKLIN}, Simon; \versal{SHEPARD}, Jonathan. \emph{The Emergence of Rus
750-1200.} Essex: Longman, 1996.

\item \versal{MUCENIECKS}, André Szczawlinska. \emph{Austrvegr e Garđaríki -
(re)significações do leste na Escandinávia tardo-medieval.} Tese de
Doutorado em História Social. São Paulo: Faculdade de Filosofia, Letras
e Ciências Humanas, \versal{USP}, 2014.

\item \versal{SHEPARD}, Jonathan. The Viking Rus and Byzantium. In: \versal{BRINK}, Stefan;
\versal{PRICE}, Neil (eds.). \emph{The Viking World.} London: Routledge, 2008,
pp. 476-516.
\end{itemize}
\section{\versal{SUÉCIA DA ERA VIKING}}

Muito se escreveu e discutiu sobre o conceito de \emph{viking} e seu
enquadramento temporal. No entanto, devido ao espaço e intuito deste
verbete, tais questões não serão levantadas. Em uma tentativa de melhor
classificar os diversos achados em territórios nórdicos, diversos
autores apontam cronologias aproximadas que flutuam entre os anos
750-1050 d.C. Para nosso trabalho, considerá-lo-emos, ainda, o período
da conversão ao Cristianismo que tem relação direta com o
estabelecimento dos três reinos -- de certa forma autônomos -- da
Escandinávia.

É durante esse período que a Escandinávia vivencia o crescimento do
comércio e contato com a Europa e Oriente próximo que implicou em um
pequeno grau de urbanização daqueles territórios. Entenderemos por
cidade uma comunidade composta por pessoas cujas ocupações primárias não
são as do campo, ou seja, as pessoas que habitavam as primeiras cidades
da Escandinávia, estabelecidas até o século \versal{IX} d.C., empregavam-se de
trabalhos como a troca (comércio) e artesanatos.

Da primeira onda de desenvolvimento urbano escandinavo (c. 700-800
d.C.), quatro cidades podem ser destacadas, sendo Birka a mais antiga e
única da Suécia. Estabelecida a 30 quilômetros a oeste de Estocolmo,
próxima ao lago Mälaren, na região da ``terra preta'' -- uma das mais
férteis de toda Suécia -- a cidade foi de vital importância para o
controle e expansão das trocas. Vale lembrar que sua localização é
estratégica, uma vez que está voltada para o Báltico. Assim, mercadores
frísios, anglo-saxões, eslavos, árabes e bizantinos circulavam pela
região oferecendo cerâmicas, sal, mel, pedra-sabão e outros produtos. Já
Birka, tinha a oferecer peles provenientes da \emph{Norrland} (norte da
atual Suécia) negociadas pela aristocracia local com os caçadores
(provavelmente sami) da região que serviam, também, como presentes nas
para os chefes locais.

A região compõe um dos sítios arqueológicos mais ricos de toda a Era
viking e as escavações são datadas de 1870. Em 1993, todo o complexo de
Birka e a mansão real de Alsnöhus foram adicionados à lista de
patrimônio da \versal{UNESCO}. A área total equivale a seis hectares e a
principal edificação, segundo as escavações, era o porto. As ruas do
assentamento foram construídas em paralelo ou em ângulo reto em relação
ao litoral. Dividida em aproximadamente 100 lotes, a cidade contava
ainda com uma paliçada e linhas de defesa.

Uma característica interessante do sítio de Birka são seus enterramentos
e conteúdos que refletem as relações sociais da população local. Há
diversos casos de homens e mulheres enterrados completamente vestidos e
munidos de joias, armas e ferramentas, além de diversos objetos
importados. Assim, Birka é basilar para compreendermos os contatos dos
nórdicos com a Europa e Oriente próximo, além de nos ajudar na
construção de uma imagem mais realista, complexa e precisa dos
habitantes daquela região.

O declínio e abandono de Birka, por volta de 975 d.C. ainda é alvo de
muitos debates. Dagfinn Skre, por exemplo, coloca que a explicação do
abandono por ``causas geográficas'', a elevação da terra em relação ao
mar, não parece ser suficiente enquanto explicação. É preciso considerar
o processo de disputa por poder político que chamamos de ``unificação'',
pois é durante a Era viking e início da Idade Média que formas mais
complexas de poder surgem na Escandinávia. Sigtuna é fundada com o
objetivo de substituir, do ponto de vista político-administrativo,
Birka. Já a ponte econômica promovida pelo Báltico é capitaneada pelos
assentamentos de Gotland.

O desenvolvimento de meios de pagamentos (moedas emitidas \emph{in
loco}) nas cidades e últimas décadas da Era viking e a introdução e
conversão ao Cristianismo são concomitantes ao processo de formação dos
reinos escandinavos, uma vez que as dinâmicas sociais da região passam a
depender mais da lei e instituições (mais ou menos) concretas do que da
personalidade dos chefes locais.

Dos três reinos surgidos no período medieval, o da Suécia é o último a
ser efetivado e, também, o mais complexo de ser estudado, visto que a
documentação escrita latinizada é escassa. Sendo assim, as principais
evidências que temos para estudar essas formações históricas são
narrativas (principalmente estrangeiras). Durante a Idade do Ferro
germânica e da Era viking, havia duas regiões separadas que no século
\versal{XII} transforma-se na Suécia: a \emph{Svealand}, terra dos suíones, que
fica ao Norte, na região do lago Mälaren e próximas às atuais Estocolmo
e Uppsala. Götaland, terra dos godos, fica próxima ao lago Vättern e se
encontra ao Sul, em regiões menos remotas da Europa se compararmos com a
Svealand.

A unificação, explicada por uma perspectiva mais tradicional, é
entendida como completa quando cada reino possuía um regente reconhecido
como a cabeça de cada território ``nacional''. Embora tenha sido
comandado por forças individuais ou militares, esse processo é mais
complexo e só existem referências a administrações centralizadas e
instituições políticas nas fontes no final do século \versal{XIII}.

O processo de formação do reino da Suécia deve ser entendido como um
processo de desenvolvimento gradual da (1) sobreposição de soberanias,
(2) do surgimento de organizações militares formalizadas e (3) do
estabelecimento do Cristianismo na região. O caso do rei Olof
Skötkonnung (r. 995-1022), ou Olavo, o Tesoureiro, nos ajuda a entender
esse processo. A ele credita-se a fundação de um reino cristão na
Suécia, tendo sido batizado em 1008 em Husaby. Há evidencias
numismáticas que o colocam como ``rei dos godos e príncipe dos
suíones'', portanto, seria o primeiro associado aos dois povos e,
consequentemente, como rei da Suécia. No entanto, Skötkonnung reconhecia
o rei dinamarquês, Svein Baba-bifurcada como seu senhor em uma relação
de sobreposição de soberania. Seu próprio nome, ``o rei do imposto'' ou
``Tesoureiro'', pode indicar que os suíones pagavam tributos a outros
reis. Ele, ainda, teria lutado contra Olavo Tryggvason em Svöld e casado
suas filhas, Astrid e Ingigerd, com os reis Olavo Haraldsson da Noruega
e Jaroslav, o Sábio da Rússia.

Um reino unido, nesse período, no entanto, não significava um poder
centralizado. O rei era eleito entre famílias particulares e influentes
da região. Logo, disputas pelo poder real eram travadas por pretendentes
detentores de apoio regional, o que nos ajuda a entender a instabilidade
política do reino que, de tempos em tempos, fora reorganizado. A figura
do conde (do sueco \emph{jarl} e inglês \emph{earl}), autoridade
particular, é muito importante pare entendermos esse processo, uma vez
que, provavelmente, as relações de poder estabelecias pelo conde são
mais importantes do que as do rei. O cargo de conde não era hereditário
e sua função era exercer o poder real onde o rei não estava sem possuir
vínculo territorial com a região. Alguns, ainda, eram associados a
atividades marciais e não havia, necessariamente, apenas um deles.

O terceiro vetor importante para o processo de unificação é o
Cristianismo, trazido pelas elites locais, e seu estabelecimento na
Suécia. A Igreja teria sido, portanto, uma forma de aumentar o poder
real, seu prestígio e o controle sobre as pessoas e territórios. Após a
cristianização, as formas de demonstração pública de poder também
passaram por mudanças que afetaram, também, o panorama urbano local. As
igrejas, monastérios e castelos construídos se tornam símbolos visíveis
de que a cidade, agora, possui propósitos maiores.

A sobredita Sigtuna foi fundada no século~\versal{X} (c. 975 d.C.) ainda no
período pagão por Érico, o Vitorioso. No capítulo cinco da
\emph{Ynglinga saga}, há uma referência a sua fundação:

\begin{quote}
Odin estabeleceu residência no lago Mälaren [...] [e] se
apropriou de todo o distrito, o chamou de Sigtuna e lá ergueu um grande
templo onde, de acordo com os costumes do povo Asgard, ocorriam
sacrifícios. Além disso, ele concedeu aos sacerdotes do templo domínios.
\end{quote}

Pelas escavações arqueológicas, sabemos que Sigtuna funcionava como um
centro de comércio doméstico e era uma arena de encontro das elites e
realeza. A cidade foi também a sede da primeira cunhagem sueca e onde as
primeiras moedas de Olavo Skötkonnung foram emitidas. A mansão real de
\emph{Forsnsigtuna}, mencionada no trecho anterior como \emph{Velha
Sigtuna}, próxima à cidade de Sigtuna foi sede da realeza peripatética
no período. No século \versal{XII}, fora entregue a um bispo católico e, até o
século \versal{XVII}, foi mantida como propriedade do Estado. Diversas igrejas
como as de Santa Gertrudes, São Nicolau, São Olavo e São Pedro eram
sediadas por Sigtuna, embora não se tenha chegado à conclusão de qual
delas era a sé episcopal. Além disso, havia um monastério dominicano e
um hospital dedicado a São Jorge.

A cidade de Gamla Uppsala, por exemplo, se torna um importante centro religioso e
régio. Antes da conversão ao Cristianismo, a cidade já
possuía funções próximas, visto que, segundo Adão de Bremen, rituais aos
deuses Odin, Thor e Frey que incluíam sacrifícios humanos, aconteciam no
assentamento a cada nove anos em um templo pagão. No entanto, não há
evidências arqueológicas de tal edifício. O que se sabe é que Gamla
Uppsala, após substituir Sigtuna, era a sede episcopal da Igreja
Católica a partir dos anos 30 do século \versal{XII} e que a cidade contava
também com uma \emph{Thing}.

As \emph{things}, assembleias legais e políticas, eram fundamentais
nessa dinâmica, visto que eram nelas que se promovia o encontro entre o
rei e seus representantes com as elites locais e a população comum. Nos
três casos de unificação as \emph{things} aparecem como um ponto
central, mas, diferente da Dinamarca ou Noruega, no caso sueco, as
elites locais parecem ter exercido maior influência por mais tempo em
graus de poder mais alto. A (frágil) monarquia sueca. Em constante
construção, pode ter sido mais confrontada por oposições provinciais
mais fortes do que as dos territórios vizinhos.

Em suma, o processo de cristianização e de construção de uma monarquia
cristã foram as maiores transformações da Suécia no final da Era viking
até o fim do período Medieval. O cristianismo e seu vínculo político
possibilitou a ``europeização'' da região, provocando transformações nas
dinâmicas existentes a partir da introdução de perspectivas diferentes e
as cidades podem ser vistas como arenas de embates econômicos, sociais e
políticos. No entanto, a Suécia não pode ser entendida como uma unidade
política coerente em que a posição de rei não estivesse fora de
possíveis disputas e durante o final do século~\versal{XI} e início do \versal{XII}, uma
série de conflitos civis assolaram os reinos escandinavos.

Entre os séculos~\versal{XIII} e \versal{XIV}, a eleição de reis torna-se uma cerimônia
formal e, a partir daí, a influência da elite política, dos bispos e dos
oficiais da lei (\emph{langmän}) emerge e a autoridade do rei se restringe.
É do mesmo período a criação de impostos permanentes, a ascensão do
conselho real como um órgão permanente, o aumento do controle sobre
pessoas e terras e, também, a separação do clero e aristocracia como
grupos privilegiados naquela dinâmica social.

Instalados na periferia da Europa, esses reinos cristãos --
principalmente Dinamarca e Suécia -- embarcam em uma ``era de Cruzadas''
contra os habitantes pagãos da região Báltica. A Igreja foi importante
para o processo de expansão do reino da Suécia tanto a leste quanto a
norte. No início do século \versal{XIII}, a Finlândia passa a englobar de forma
gradual a esfera política e eclesiástica da Suécia, assim como a
\emph{Norrland}, que pouco povoada, teve sua incorporação efetiva
encabeçada pela Igreja: quase um século depois, em 1345, cerimônias
batismais ocorrem em Tornio, quando o arcebispo de Uppsala visita à
Lapônia.

\SIG{Vítor Bianconi Menini}

Ver também Birka; Gamla Uppsala; Gotland (Gotlândia).

\begin{itemize}
\item \versal{AMBROSIANI}, Björn. Birka. In: \versal{BRINK}, Stefan; \versal{PRICE}, Neil (eds.).
\emph{The Viking World}. New York: Routledge, 2008, pp. 94-100.

\item \versal{DOUGLAS}, Price Theron.~\emph{Ancient Scandinavia}\textbf{:~}An
Archaeological History from the first humans to the Vikings. New York:
Oxford University Press, 2015.

\item \versal{HOLMAN}, Katherine.~\emph{Historical dictionary of the
Vikings}\textbf{.~}Lanham, Maryland: The Scarecrow Press Inc., 2003.

\item \versal{LINDKVIST}, Thomas. Introductory survey: Early political organisation.
In: \versal{HELLE}, Knut (org.). \emph{The Cambridge History of Scandinavia.}
Cambridge: Cambridge University Press, 2003, pp. 160-167.

\item \versal{LINDKVIST}, Thomas. Kings and provinces in Sweden. In: \versal{HELLE}, Knut
(org.). \emph{The Cambridge History of Scandinavia.} Cambridge:
Cambridge University Press, 2003, pp. 221-234.

\item \versal{LINDKVIST}, Thomas. The emergence of Sweden. In: \versal{BRINK}, Stefan; \versal{PRICE},
Neil (eds.). \emph{The Viking World}. New York: Routledge, 2008, pp.
668-674.

\item \versal{RICHARDS}, Julian D.~\emph{The Vikings}:\textbf{~}A very short
introduction. New York: Oxford University Press, 2005.

\item \versal{ROS}, Jonas. Sigtuna. In: \versal{BRINK}, Stefan; \versal{PRICE}, Neil (eds.). \emph{The
Viking World}. New York: Routledge, 2008, pp. 140-144.

\item \versal{SKRE}, Dagfinn. Introduction survey: development of urbanism in
Scandinavia. In: \versal{BRINK}, Stefan; \versal{PRICE}, Neil (eds.). \emph{The Viking
World}. New York: Routledge, 2008, pp. 83-92.
\end{itemize}
\section{\versal{SUICÍDIO}}

O suicídio entre os nórdicos da Era Viking está associado estreitamente
a questões religiosas, sociais e militares. Em 925 d.C. uma tropa de
nórdicos cometeu suicídio para não serem mortos pelos franceses, segundo
os \emph{Annales} de Flodoardo de Reims. Alguns inclusive morreram
relutantes com o ato. Em outras situações, ocorreu registro de suicídio
com caráter mais individual, como o rei Sigerferth em 964. Também
existem menções a suicídios provocados por ordens de terceiros ou pela
situação da morte de outras pessoas: o escravo Karka se matou após o
jarl Hákon ter solicitado (\emph{Ágrip}, c. 1180); na Islândia, escravos
de origem irlandesa cometeram suicídio devido à morte de seu mestre,
atirando-se de um penhasco (\emph{Landnamabók} 8).

Um tipo específico de suicídio que surge nas fontes literárias é o
acompanhar da esposa de algum líder morto durante o seu funeral, como
Brynhild atirando-se na pira funerária de Sigurd (\emph{Völsunga saga}
33) ou Signy com a pira de Balder, de modo semelhante ao sati das
mulheres hindús. Do mesmo modo, as sagas islandesas também possuem
referências a essas práticas, como a rainha Audr junto ao rei Eric
(\emph{Óláfs saga Tryggvasonar} 1). Para Hilda Davidson, a prática do
sati desapareceu da Escandinávia antes da chegada do cristianismo, mas
outros acadêmicos (como Eric Christiansen) acreditam que ela na
realidade foi uma construção literária e anacrônica ou produto do
folclore tardio. De qualquer modo existem algumas crônicas que registram
historicamente a prática, como o relato do cronista árabe Ahmad Ibn
Rustah do século~\versal{X} d.C., mencionando uma elaborada câmara sepulcral de
um líder nórdico da Rússia, com depósitos de comida, bebidas, vasilhames
e moedas. Segundo este cronista, a esposa do chefe foi colocada viva
dentro da sepultura. Também no mesmo século temos o relato de outro
viajante muçulmano, Ibn Fadlan, descreve o sacrifício voluntário de uma
escrava para acompanhar o seu senhor durante o sepultamento.

Outro caso de suicídio na Era viking diz respeito às auto-imolações
realizadas em períodos de fome ou no momento de alguma doença ou
ferimento. A \emph{Gautreks saga} 1 menciona a existência de um penhasco
sueco chamado Gillingshamar, onde em crises de fome as pessoas mais
velhas se atiravam no precipício e acreditavam que entrariam no
Valhalla. Isso recorda uma prática de outra região do mundo, Aokigahara,
uma floresta situada no monte Fuji, Japão, que desde o século~\versal{XVIII}~ foi
um local tradicional de pessoas idosas cometerem suicídio, salvando as
gerações mais novas das crises de fome na região. Também entre os
inuítes e muitas outras culturas ocorria a prática do suicídio
benevolente, onde os envolvidos se sacrificam para salvar ou preservar
outras pessoas de uma comunidade, por diversos motivos. Ainda sobre o
tema deste tipo de suicídio na Escandinávia Medieval, o escritor William
Temple registrou em 1679 em Oxenstierna (Suécia) a existência de uma
rocha chamada Salão de Odin, onde as pessoas iam se atirar quando
chegavam na velhice, quando tinham enfermidades ou ferimentos mortais.
Por sua vez Edmund Burke em 1770 registrou a tradição islandesa de
rochas utilizadas como locais típicos de suicídio na Islândia durante os
tempos pré-cristãos.

\SIG{Johnni Langer}

Ver também Cotidiano; Sociedade; Religião.

\begin{itemize}
\item \versal{CHRISTIANSEN}, Eric. \emph{The norsemen in the Viking Age}. London:
Blacwell, 2006.

\item \versal{DAVIDSON}, Hilda. \emph{The road to hel}: A Study of the Conception of
the Dead in Old Norse Literature. London: Praeger, 1968.

\item \versal{MURRAY}, Alexander. \emph{Suicide in the Middle Ages}. Oxford: Oxford
University Press, 2009.

\item \versal{TEMPLE}, William. \emph{Miscellanea}. London: J.R., 1690.
\end{itemize}
\chapter{T \textarn{t} \textart{t}}
\section{\versal{TAPEÇARIA DE BAYEUX}}

Localizada na Catedral de Bayeux, na França, a primeira menção desta
tapeçaria consta em um inventário da catedral feito ainda no ano de
1476. Contudo, a datação exata de sua confecção permanece incerta, bem
como seu local de origem. Atualmente, são duas as hipóteses que abrangem
essa questão; uma alega que ela teria sido encomendada em Canterbury,
outra defende que ela teria sido confeccionada por normandos que
residiam em Bayeux.

Essa tapeçaria feita em linho possui cerca de 70 metros e 34 centímetros
de extensão, medindo 50 centímetros de altura. Edifícios e árvores foram
bordados em locais estratégicos de forma que dividissem seu conteúdo em
72 cenas, explicitando tanto os eventos que precederam a Batalha de
Hastings, quanto a batalha propriamente dita.

Foram ilustrados, ao todo, 1.512 objetos na tapeçaria, dos quais 623 são
pessoas, 202 são cavalos ou mulas; 55 cães; 505 outros diversos tipos de
animais; 37 construções; 41 barcos/navios e, por fim, 49 árvores. Também
ocorrem, em diversos pedaços da obra, inscrições em latim que visam
elucidar com um pouco mais de precisão algumas cenas específicas.
Aparentemente foi envolvido muito esforço para tornar a narrativa da
tapeçaria algo grandioso e exagerado, visto que, segundo ressalta Frank
Fowke, de todos seus 70 metros, a tapeçaria utiliza de apenas 33
centímetros para expressar a parte realmente histórica de seu conteúdo.

A autoria da tapeçaria é intensamente debatida até os dias de hoje e
ainda não se sabe quem, ao certo, foi seu criador. Como o tema
definitivamente mais marcante na tapeçaria é a conquista de território
inglês por parte do rei normando Guilherme, acreditava-se, então, que sua
esposa, a Rainha Matilda, havia sido responsável pelo trabalho. Conforme
lembra Helen Candee, pensava-se que a rainha havia confeccionado a peça
como evidência de sua devoção ao marido, que estava fora em sua missão
de conquista. Segundo essa lenda, certamente carregada de certa
romantização, a rainha costurava, na tapeçaria, seus sentimentos
secretos de amor e admiração envolvendo o marido ausente, enquanto
aguardava por seu retorno. Assim, esperando meses em casa pelo marido a
Rainha teria, juntamente de suas criadas, o tempo e as condições
necessárias para bordar a tão extensa e detalhada tapeçaria.

Contudo, conforme essa primeira hipótese perdia força, começou-se a
voltar mais atenção para que se descobrisse quem havia financiado e
patrocinado a confecção da obra. A ideia mais aceita até o momento é a
de que o patrono da tapeçaria teria sido Odo, bispo da igreja de Bayeux.
Considerando-se que era Conde de Kent, além de uma figura de muita
influência em Canterbury -- principal cidade do condado naquele momento
e provável local de produção da tapeçaria --, acredita-se que Odo teria
os recursos políticos e financeiros para patrocinar todo o processo de
confecção desta obra.

Esta hipótese de que Odo seria o patrocinador da Tapeçaria de Bayeux tem
perdurado por muito tempo. Afinal, seria uma maneira relativamente
simples e direta de se explicar por que sua figura e tantas outras a ela
relacionadas possuem presença tão marcante ao longo da narrativa de
Bayeux, a ponto de parecer inconcebível que outro patrocinador, que não
o próprio Odo, desejasse glorificá-lo a tal ponto.

Contudo, Elizabeth Pastan e Stephen White combatem essa hipótese,
alegando que ela foi baseada em uma concepção ultrapassada de
apadrinhador. Tal modelo de apadrinhamento, segundo os autores, concebe
a ideia de alguém que, financiando a obra, determina e restringe seu
conteúdo, exigindo, com veemência, sua presença. Portanto, a
consequência deste modelo é encarar a arte -- o resultado final -- como
personificação das vontades e ideais de seu financiador, retratando sua
própria agenda política e pessoal. Apesar dessa concepção analítica do
apadrinhador ser verdadeira em certos casos, principalmente no contexto
renascentista, é preciso cuidado ao tentar aplicá-la em conteúdos
medievais.

A problematização dessa ideia de apadrinhador se encontra no fato de que
ela se embasa quase que exclusivamente em argumentos de cunho pessoal e
psicológico, perpetuados por vários historiadores que construíram uma
imagem pejorativa de Odo enquanto sendo alguém excessivamente narcisista
e arrogante. Mas de que maneira se chegou a essa conclusão é ainda algo
obscuro, partindo deste ponto a crítica de Pastan e White. Ademais,
aceitar essa premissa torna-se difícil, visto que, do ano de 1070 em
diante, o bispo apoiou toda a comunidade de monges em Saint Augustine de
maneira ativa, oferecendo-lhes propriedades e riquezas. Inclusive, antes
de ser preso em 1082, Odo era considerado como o guardião e protetor da
abadia em questão. Parece inverossímil que o bispo precisasse chegar ao
ponto de exigir, dos artistas de Canterbury, que fosse retratado de
maneira positiva na tapeçaria.

Há uma notável dualidade presente na narrativa de Bayeux. Sua primeira
parte oferece um relato um tanto quanto simpático a Haroldo~\versal{II}, o rei
anglo-saxão, enquanto que a segunda parte é evidentemente pró-normanda.
Richard Koch trabalha com o conceito de dupla narrativa no que concerne
à tapeçaria: uma delas religiosa, dispondo igrejas, clérigos e o
sagrado; a outra secular, denotando a caça, a aristocracia e a guerra. A
própria figura de Odo representa tal dualidade: bispo e clérigo, mas
também guerreiro.

Segundo essa linha de pensamento, a Tapeçaria de Bayeux possuiria um
propósito acima de tudo moral. Ela explicita as consequências
catastróficas da deslealdade e da quebra de um juramento -- da parte de
Haroldo -- levando à morte, à condenação e à derrota. Isso explicaria a
diferença com que o mesmo é representado nas cenas que sucedem ao
juramento. Haroldo foi representado de maneira simpática nas primeiras
cenas da tapeçaria, mas, após a cena do juramento, nota-se culpa e
preocupação cercando sua figura. Quando chega para conversar com o Rei
Eduardo e contar os desdobramentos de sua aventura, Haroldo não é mais
representado como o nobre e orgulhoso homem do início da narrativa. Pelo
contrário, ele encontra-se quase que dobrado, recolhido, consumido pela
vergonha e pela culpa enquanto o rei parece lhe perguntar o que ele
havia feito. Na cena seguinte, o Rei Eduardo morre em sua cama.

A relação existente entre os normandos, a conquista da Inglaterra, a
Escandinávia viking e a Tapeçaria de Bayeux tem sido analisada até hoje
pelas óticas da arqueologia e da história da arte. É relevante pensar,
nesse contexto interpretativo, nas identidades nacionais presentes na
peça. A Batalha de Hastings e a subsequente conquista normanda da
Inglaterra são comumente descritas -- e reduzidas -- como um sangrento
encontro entre os normandos, juntos de seus aliados, e os anglo-saxões e
seus coligados, sendo o prêmio final a coroa da Inglaterra anglo-saxã.

É importante lembrar que Guilherme da Normandia subjugou não um reino
puramente anglo-saxão, mas também anglo-dinamarquês. Afinal, as
incursões dos vikings dinamarqueses na costa inglesa, no fim do século
\versal{VIII}, culminaram num
assentamento dinamarquês no território em questão e, por fim, ao
estabelecimento da Danelaw em 879, confirmando o reconhecimento, naquele
momento, dos dinamarqueses enquanto integrantes da Grã-Bretanha.
Portanto, a atividade escandinava na região inglesa demandava uma
atenção especial do Rei Guilherme e seu reino. Consequentemente, deve-se
considerar que entre os espectadores da Tapeçaria de Bayeux estariam
ingleses, normandos, anglo-normandos e também os anglo-dinamarqueses. É
provável que um dos papéis da tapeçaria fosse o de atuar como aviso de
caráter intimidador, explicitando o fim catastrófico que aguardava
qualquer um que aspirasse à conquista do trono inglês -- incluindo os
dinamarqueses.

Seria praticamente impossível que os artistas envolvidos na produção da
tapeçaria ignorassem os elementos escandinavos presentes na cultura
material e visual da época. Ainda assim, o fator nórdico presente na
obra não foi reconhecido até a redescoberta da tapeçaria, por volta de
1730. Contudo, conforme lembra Shirley Brown, foi apenas no século \versal{XIX}~
que o interesse no passado escandinavo ressurgiu de maneira definitiva.
Hector Estrup, historiador dinamarquês, surpreendeu-se com o número de
conteúdos na tapeçaria que, segundo ele, evocavam a herança nórdica: as
vestimentas de Guilherme da Normandia, idênticas às usadas por Rollo --
viking que tornou-se o primeiro Duque da Normandia -- em seu sarcófago;
homens bebendo em cornos tipicamente nórdicos; os escudos dos
cavaleiros, ornamentados com figuras pintadas, especialmente cruzes e
dragões; barcos carregando escudos nas amuradas, além de serem
representados em um formato longo que seria típico das embarcações
vikings; a filha de Guilherme que veste um manto similar aos utilizados
pelas mulheres nórdicas.

Uma comparação entre o material da literatura nórdica e as imagens da
tapeçaria também esclarecem outros aspectos. Por exemplo, um aspecto que
dialoga a favor da tapeçaria como sendo também uma referência às raízes
nórdicas dos normandos é seu tema central, a cena do juramento. A
sacralidade de um juramento estava no cerne do código de honra nórdico,
conforme apontam várias fontes literárias. Além disso, banquetes que
precedem uma grande batalha também são comuns na literatura nórdica,
lembrando que há, no tapete, tal cena. Phillis Ackerman analisa o
estandarte de Haroldo, onde há um dragão bordado. Segundo sua hipótese,
a intenção seria que o estandarte inflasse com o vento, dando a
impressão de que havia um verdadeiro dragão guardando Haroldo e seus
guerreiros. Esse tema seria um resgate da \emph{Saga dos Volsungos},
quando, em situação análoga, a mãe de Sigurd lhe confecciona um
estandarte com um corvo que, impulsionado pelo vento, pareceria estar
abrindo suas asas e alçando voo. Há também uma analogia trazida pelos
animais representados, já que tanto o dragão quanto o corvo são
relacionados a Odin.

Marit Monsen Wang interpretou outro símbolo na tapeçaria como sendo uma
referência aos nórdicos. Trata-se do portal onde se encontra Aelfgyva.
Para Wang, esse portal seria referência a um ponto de conexão entre a
vida mortal e a vida no outro mundo, conforme descrito nas \emph{Eddas}.

Enfim, é provável que, até por volta da metade do século \versal{XI}, os
normandos fossem ainda considerados como vikings -- ou ao menos enquanto
homens do norte -- unidos pela memória de seu ancestral em comum, Rollo.
Dessa forma, a Tapeçaria de Bayeux não pode ser lida e interpretada sem
que se enxergue, nela, o reflexo dessa herança nórdica nos normandos. Ao
invés de interpretá-la como sendo meramente um documento anglo-normando,
deve-se vê-la como retrato de uma sociedade também sobre influência
anglo-dinamarquesa.

\SIG{Victor Hugo Sampaio Alves}

Ver também Inglaterra da Era Viking; Normandia; Viking.

\begin{itemize}
\item \versal{ACKERMAN}, Phyllis. The Norsemen and their Descendants. \emph{Tapestry:
The Mirror of Civilization}. Oxford: Oxford University Press, 1970.

\item \versal{BROWN}, Shirley. The Bayeux Tapestry and the Vikings.
\emph{Peregrinations: Journal of Medieval Art \& Architecture}, vol. 2,
n. 4, 2009, pp. 10-50.

\item \versal{CANDEE}, Helen Churchill. \emph{The Tapestry Book}. New York: Frederick
A. Stokes Company, 1912.

\item \versal{FOWKE}, Franke. \emph{The Bayeux Tapestry: a history and description}.
London: G. Bell and Sons, 1913.

\item \versal{KOCH}, Richard. Sacred Threads: The Bayeux Tapestry as a Religious
Object. \emph{Peregrinations: Journal of Medieval Art \& Architecture},
vol. 2, n. 3, 2009, pp. 134-165.

\item \versal{PASTAN}, Elizabeth. \& \versal{WHITE}, Stephen. Problematizing Patronage: Odo of
Bayeux and the Bayeux Tapestry. \emph{The Bayeux Tapestry: New
Interpretations}. Woodbridge: Boydell Press, 2009, pp. 01-24.

\item \versal{WANG}, Marit. Portalsymbolikk. \emph{Viking}, vol. 34, 1970, pp. 73-96.
\end{itemize}
\section{\versal{TAPEÇARIA DE OSEBERG}}

Trata-se, originalmente, de vários fragmentos de tapeçarias ilustradas
que foram encontradas no túmulo de Oseberg, na Noruega. Muitos desses
fragmentos estavam em péssimas condições, o que complicou sua datação e
reconstrução a princípio. Por esse mesmo motivo, as tapeçarias
precisaram ser restauradas para que somente depois pudessem ser
analisadas, quando chegaram a constituir, finalmente, um todo.

O túmulo de Oseberg, escavado em 1904, pertencia a uma mulher que teria
sido enterrada com uma acompanhante por volta do ano de 834 d.C. A
primeira suspeita de que não se tratava do túmulo de um homem foi a
falta de armas e outros utensílios de guerra, enquanto, por outro lado,
notou-se a presença de acessórios para cozinhar, uma das atividades
comumente relegadas às mulheres na Escandinávia medieval. O fato de
terem sido enterradas duas pessoas juntas também aguçou a curiosidade
dos pesquisadores, que qualificaram o achado de Oseberg como uma
descoberta única. Posteriormente, estudos genéticos e arqueológicos
confirmaram que as duas figuras no túmulo eram, de fato, do sexo
feminino: uma delas teria entre 40 e 50 anos e a outra, mais jovem,
entre 25 e 40.

A câmara mortuária continha um grande e elegante barco enterrado, que
provavelmente representaria um transporte para os falecidos. Na mesma
câmara encontravam-se trenós, cavalos, provisões, camas, pequenas
carroças e até mesmo ferramentas agrícolas. Segundo Graham-Campbell,
essa natureza grandiosa e exagerada dos bens encontrados em Oseberg
revelaria a posição dessa mulher cuja família foi capaz de dispor tantas
riquezas materiais em sua honra. Portanto, inicialmente levantou-se a
hipótese de que, muito provavelmente, essa mulher deveria se tratar de
alguém de alta posição social, como uma rainha. A outra figura feminina,
acreditava-se, era uma escrava que havia sido sacrificada junto de sua
dona -- como era costume ser feito; mas tal suposição, ressalta Johnni
Langer, foi derrubada por recentes análises de \versal{DNA}, que comprovaram
tratarem-se de mãe e filha.

Dentre os achados do túmulo encontravam-se estreitos fragmentos de
tapeçaria, tecidos com lãs de diversas cores. São dois os pedaços de
tapeçaria resgatados que se tornaram mais conhecidos. Um deles mostra
uma procissão de figuras armadas, representando tanto homens quanto
mulheres, alguns a pé, outros a cavalo. Alguns cavalos puxavam carroças
-- ilustradas de forma muito similar à própria carroça encontrada no
túmulo de Oseberg. O outro fragmento mostra dois pássaros negros
sobrevoando e rondando um cavalo que também puxava uma carroça. A
tentativa de reconstrução da tapeçaria, uma vez feita, fez parecer com
que seu conteúdo tratasse de uma grande procissão de guerreiros e
montarias que seguem da direita para a esquerda. Todos os cavalos
aparecem puxando carroças, e os guerreiros que estão a pé foram
representados portando escudos e armas -- quase que exclusivamente
lanças.

Segundo Kirsten Ruffoni, esta poderia se tratar de uma representação da
batalha de Bråvalla, acontecimento lendário em que os Daneses
enfrentaram os Svear -- de território sueco. Já outras duas partes da
tapeçaria, que foram encontradas dobradas juntas no chão da câmara
mortuária, denotam fortemente uma procissão ou cerimônia de caráter
religioso. Estas partes da tapeçaria foram mais preservadas, de modo que
as análises a seu respeito puderam estender-se. No primeiro fragmento é
possível notar um homem grande montando um cavalo branco, enquanto que
quatro pássaros -- falcões ou gaviões -- os acompanham, voando em cima,
à frente, e atrás do cavalo. Para Ruffoni, falcões e gaviões costumam
ser símbolos da realeza, o que a leva a crer que esta primeira figura
fosse algum rei.

Abaixo do suposto rei encontra-se um cavalo puxando uma carroça onde
duas figuras, provavelmente mulheres, estão sentadas. Um pássaro negro
semelhante a um corvo as sobrevoa. É no mínimo curioso notar essas duas
figuras femininas representadas juntas na tapeçaria, levando em conta
que ela foi encontrada no túmulo de Oseberg, onde haviam sido enterradas
justamente duas mulheres.

Próximas a essas figuras principais encontram-se muitas representações
da suástica. Na região da Escandinávia, afirma Johnni Langer, este
símbolo surge enquanto uma variação da espiral. Antes representado com
inúmeros braços, a partir do período de imigração a espiral se
populariza retratada com quatro braços, mas permanecendo, em essência,
uma figuração solar. Em certas ocasiões, como em potes cerâmicos para
uso funerário na área germânica setentrional e continental, as suásticas
podem significar a passagem ou o transitar das estações do ano, ou seja,
operam como símbolo de sazonalidade que retoma o tema de transição da
própria vida humana. Langer também lembra que, nas insígnias reais de
reis pagãos anglo-saxões, as suásticas eram representadas nas bainhas de
espadas como símbolos de vitória e proteção marcial aos seus portadores.
Estes dois contextos de uso da suástica -- embora não sejam os únicos --
parecem os mais apropriados à narrativa da Tapeçaria de Oseberg,
invocando os temas da realeza e da marcialidade, ou da morte e passagem
para o outro mundo.

No fim do primeiro fragmento há outra figura desproporcionalmente
grande, um homem portando uma espada. Outras pessoas, bem menores, o
cercam, mas estas não possuem montarias; algumas delas carregam lanças.
A parte da frente de um cavalo conecta a tapeçaria ao segundo fragmento
da peça, que, por sua vez, começa com a parte traseira do animal. Este
cavalo puxa uma carroça, e abaixo dele outro equino foi retratado
realizando a mesma ação. Dois pássaros negros aparecem novamente,
sobrevoando este último cavalo. Também próximos a ele se encontram
outras figuras humanas portando lanças, além de outras representações de
suásticas. No entanto, nenhum desses dois últimos cavalos puxam alguém
na carroça, como o primeiro retratado na tapeçaria o faz. Ao invés de
pessoas encontram-se, nessas carroças, apenas bordados coloridos e
decorativos.

Próximo a cada cavalo há uma figura humana segurando um cajado. Em um
dos casos, essa figura com o cajado está simultaneamente segurando as
rédeas do animal, o que sugere ser ele o seu guia. Como o guia está
caminhando entre os cavalos, parece que ninguém comandava nenhuma das
carroças diretamente -- com exceção da primeira, onde havia duas
mulheres. Como o mal estado da tapeçaria não permite que enxerguemos com
exatidão o que continham essas carroças, pode ser que elas
representassem divindades enquanto condutoras das mesmas.

Esta linha interpretativa é seguida por Anne Stine Ingstad que, ao
juntar os vestígios do túmulo de Oseberg à narrativa da tapeçaria lá
encontrada, sugere que ao menos uma das duas mulheres enterradas era
tida como uma poderosa sacerdotisa ou semideusa. Segundo seu pensamento,
a mulher retratada na primeira carruagem da tapeçaria -- e enterrada em
Oseberg -- estaria no comando de uma importante procissão realizada à
deusa Freyja, sendo, portanto, sua representante no mundo dos homens.

Outra cena intrigante, encontrada em pedaços menores da tapeçaria que
restaram, retratam uma cena de provável sacrifício. É possível notar
corpos de homens pendendo de árvores com os galhos retorcidos. Próxima
aos corpos, uma mulher segura uma espada pela lâmina, enquanto que uma
outra, do seu lado, ergue as mãos em posição de reza. Para Kirsten
Ruffoni, esta cena, assim como a tapeçaria como um todo, possui certa
inclinação em exaltar o feminino, o que tornaria possível que se
fizessem ligações entre a tapeçaria, o túmulo de Oseberg e o culto a
alguma divindade feminina que fosse organizado e celebrado por mulheres.
Neste ponto, há uma afinidade com a ideia defendida por Ingstad.

Terry Gunnel corrobora com tais ideias, afirmando que os conteúdos
expressos na tapeçaria, embora enigmáticos, inclinam o olhar analítico
para interpretá-la como se fosse uma procissão religiosa. Ele retoma os
argumentos de Ingstad, dizendo que, na primeira carruagem, ao menos uma
das duas figuras humanas é uma mulher sacerdotisa, e que as outras
carruagens presentes na tapeçaria provavelmente retratavam,
originalmente, figuras sagradas de alguma espécie. Além das carruagens
que denotariam esse tom de procissão, há uma linha de figuras femininas
que aparentam estar dançando de maneira ritualística, além de outra
fileira de pessoas cuja postura é muito similar às ilustrações
encontradas em alguns petróglifos da Idade do Bronze. Somados esses
fatos à última cena da tapeçaria, que representa um sacrifício, seria
difícil tomar o simbolismo dessas ilustrações como sendo desprovido de
significados mágicos e ritualísticos.

No entanto, se é difícil elaborar dados conclusivos a respeito desse
tipo de material iconográfico que são as tapeçarias, o caso de Oseberg
apresenta ainda algumas peculiaridades a mais. Isso se deve não só ao
problema do estado precário em que o material foi encontrado, mas das
consequências advindas desse fato, que deixam espaços em branco difíceis
de serem preenchidos. Signe Fuglesang tece vários comentários a respeito
dessa problemática. Quando o túmulo de Oseberg foi escavado, ainda em
1904, os vários pedaços de tapeçaria estavam rasgados e manchados e sua
narrativa, enfim, fracionada. Além disso, o grande número de fragmentos
encontrados nas mais variadas partes do túmulo pode indicar que haveria
mais de uma tapeçaria a ser reconstruída: poderíamos estar diante de
duas ou três delas, por exemplo.

Apesar de um excelente trabalho de restauração, há que se fazer a
ressalva de que a ilustração mais frequentemente reproduzida da
tapeçaria e utilizada como objeto para análise data do ano de 1950. As
questões relativas à iconografia, bem como a identificação de qualquer
material narrativo hipertextual provavelmente continuarão difíceis de
serem decifrados. Basicamente, os principais conteúdos e temas mais
detectáveis na Tapeçaria de Oseberg seriam a marcha religiosa; o
sacrifício por enforcamento e a batalha de Bråvalla, conforme citado
anteriormente.

\SIG{Victor Hugo Sampaio Alves}

Ver também Oseberg; Suécia da Era Viking.

\begin{itemize}
\item \versal{FUGLESANG}, Signe Horn. Ekphrasis and Surviving Imagery in Viking
Scandinavia. \emph{Viking and Medieval Scandinavia} 3, 2007, pp.
193-224.

\item \versal{GRAHAM-CAMPBELL}, James. \emph{Os Viquingues: Origens da cultura
escandinava}. Rio de Janeiro: Del Prado, 1997.

\item \versal{GUNNEL}, Terry. \emph{The origins of drama in Scandinavia}. Cambridge:
D.S Brewell, 1995.

\item \versal{INGSTAD}, Anne Stine. The interpretation of the Oseberg find. \emph{The
ship as a symbol in Prehistoric and Medieval Scandinavia.}
Nationalmuseet: Copenhagen, 1995, pp. 139-149.

\item \versal{LANGER}, Johnni. Erfi: As Práticas Funerárias na Escandinávia Viking e
suas Representações. \emph{Brathair} 5 (11), 2005, pp. 114-127.

\item \versal{LANGER}, Johnni. Símbolos Religiosos dos Vikings: guia iconográfico.
\emph{História, imagem e narrativas} 11, 2010, pp. 01-28.

\item \versal{RUFFONI}, Kirsten. \emph{Viking Age Queens: The example of Oseberg}.
London: \versal{LAP} Lambert Academic Publishing, 2013.
\end{itemize}
\section{\versal{TAPEÇARIAS DE ÖVERHOGDAL}}

Trata-se de um grupo de tecidos descobertos na Suécia, em Överhogdal,
que se mantiveram extremamente bem preservados até os dias de hoje. Eles
foram encontrados por Jonas Holm dentro da sacristia da Igreja de
Överhogdal, no ano de 1909. Em seguida, o artista Paul Jonze os levou
para a cidade de Östersund, onde Ellen Widén, esposa do governador,
assumiu os cuidados das tapeçarias.

A princípio, acreditava-se que haviam sido confeccionadas durante a
Idade Média, mas testes de radiocarbono conduzidos posteriormente
apontaram que as tapeçarias foram produzidas durante a Era viking, entre
os anos 800 e 1000. Basicamente, trabalha-se com a ideia de que as
imagens nelas reproduzidas apontam tanto para o imaginário nórdico/pagão
quanto para alguns elementos cristãos. Contudo, observa-se notoriamente
a predominância da temática pagã, como, por exemplo, a provável
representação de Sleipnir, o corcel do deus Odin, e a Yggdrasil, o
freixo do mundo, no centro.

As quatro partes da tapeçaria que sobreviveram totalizam cerca de 323
figuras humanas representadas, além de 146 animais e bestas, todos
movendo-se no sentido da direita para a esquerda. Erik Schjeide
argumenta que, na iconografia escandinava da Era Viking de maneira
geral, era costume ``ler'' as narrativas das obras realmente da direita
para a esquerda, conforme afirmado anteriormente também por
Graham-Campbell. Isso faria sentido especialmente se aplicado às
tapeçarias de Överhogdal, já que praticamente todas as criaturas
ilustradas encontram-se viradas para a esquerda, oferecendo certo fluxo
e continuidade voltados a essa direção.

Os animais maiores, bem como várias das figuras humanas, aparentam estar
correndo em direção a uma imagem que mais parece uma grande árvore.
Provavelmente, trata-se de Yggdrasil. Alguns poucos estudiosos sugeriram
que o conteúdo das tapeçarias mostra, na verdade, a cristianização da
região de Härjedalen, mas tal argumento carece de fontes e de maiores
embasamentos. Após a datação mais precisa das tapeçarias provida pelo
teste de radiocarbono, levantou-se a hipótese de que as ilustrações
poderiam estar relacionadas ao Ragnarök, a série de eventos que precede
o fim do universo segundo os escandinavos.

No entanto, ainda não se chegou a nada conclusivo. Ao contrário, por
exemplo, da Tapeçaria de Bayeux, os estudos sobre as tapeçarias de
Överhogdal são ainda escassos. Grande parte deles provém de
pesquisadores escandinavos e, uma vez que são publicados em seu idioma
nativo, o acesso de outros pesquisadores a esse material é difícil.

Em seu artigo analisando a Pedra Rúnica de Ockelbo, Johnni Langer
oferece alguns apontamentos também a respeito das tapeçarias de
Överhogdal. A grande figura no centro da tapeçaria seria, de fato, o
desenho de uma árvore que remete à Yggdrasil; vê-se, nela, um pássaro no
topo e outro em sua base. Esta é a grande árvore que seria o centro do
mundo e o ponto de ligação de todo o cosmos, antevista, já pelos
germanos antigos, como Irminsul, que significa ``coluna gigantesca''.
Essa árvore mítica era habitada por diversos animais, sendo um deles
justamente uma águia que ficava no topo, tal qual ilustrado na tapeçaria
-- e que costuma ser relacionada ao deus Odin --, conforme dito no poema
\emph{Grímnismál}.

Das poucas imagens da Era Viking que fazem alusão a esse mito, uma
delas, ressalta Johnni Langer, é justamente uma das cenas da Tapeçaria
de Överhogdal. Afinal, ela retrata não só uma grande árvore em seu
centro, como também um pássaro em seu cimo. O pesquisador também lembra
que uma das criaturas representadas é um cavalo de oito patas, uma
alusão direta ao corcel de Odin, Sleipnir. Estes dois últimos detalhes
reforçam o ideal de uma representação predominantemente pagã presente
nesta tapeçaria.

Neil Price retoma a questão da representação dos seres equinos presentes
no material em questão. Primeiramente, descreve a tapeçaria não como um
arranjo de sequências narrativas coesas -- como a de Bayeux --, mas como
uma massa de figuras combinadas de modo a criar uma única imagem e
contexto, apesar de multifacetada. Notam-se, presentes nessa grande
imagem narrativa, nada menos que quatro cavalos de oito patas,
representados juntamente de outros três cavalos com seis patas e outro
com sete. Além disso, há, também, um alce de seis patas e seis renas de
seis patas. Na maioria das ilustrações não há ninguém montando ou
cavalgando esses animais, salvo um caso em que se nota um cavaleiro.
Ademais, curiosamente, um dos cavalos de oito patas e uma das renas são
representados com o que parece ser um falo proeminente.

A segunda tapeçaria, contudo, mostra apenas um cavalo de oito patas, que
está montado por duas pessoas. Price argumenta que seria difícil tomar
todos os cavalos da tapeçaria enquanto representações unicamente de
Sleipnir, e aponta que a noção de várias criaturas desse mesmo tipo,
representadas juntas, denotaria um contexto tipicamente xamanístico.
Esses animais equinos de várias patas foram continuamente registrados
como as montarias dos xamãs na região da Sibéria, entre os Buriates,
sendo recorrente até em lugares como Índia e o Japão. Ainda segundo
Price, outros elementos da Tapeçaria de Överhogdal possuem paralelos
Sibérios, sendo possível que todos esses animais de numerosas patas
representem diferentes componentes do xamanismo Escandinavo e Sámi
durante a Era viking. Contudo, esse material iconográfico ainda carece
de muitos outros estudos antes que se possa alegar ter atingido
quaisquer conclusões.

Atualmente, essas tapeçarias encontram-se exibidas no museu de Jamtli --
o museu regional de Jämtland -- e em Härjedalen, em Östersund, na
Suécia.

\SIG{Victor Hugo Sampaio Alves}

Ver também Folclore; Religião; Suécia da Era Viking.

\begin{itemize}
\item \versal{GRAHAM-CAMPBELL}, James. \emph{Viking Art}. London: Tames \& Hudson,
2013.

\item \versal{LANGER}, Johnni. O Céu dos Vikings: Uma Interpretação Etnoastronômica da
Pedra Rúnica de Ockelbo (\versal{GS} 19). \emph{Domínios da Imagem}, vol. 7, n.
12, 2013, pp. 97-112.

\item \versal{PRICE}, Neil. The Archaeology of Seiðr: Circumpolar Traditions in Viking
Pre-Christian Religion. \emph{Brathair}, vol. 4, n. 2, 2004, pp.
109-126.

\item \versal{SCHJEIDE}, Erik. \emph{Crafting Words and Wood: Myth, Carving and
Húsdrápa}. Dissertação em Filosofia apresentada ao programa de Línguas e
Literaturas Escandinavas da Universidade da Califórnia, Berkeley. 2015.
\end{itemize}
\section{\versal{TAPEÇARIA DE SKOG}}

Esta tapeçaria feita em lã pertence originalmente à Igreja de Skog,
localizada na cidade de Hälsingland, na Suécia. Desde 1914, contudo, ela
encontra-se no Museu de Antiguidades Nacionais, em Estocolmo. Datada do
século \versal{XII}, logo após a Suécia ter se convertido ao cristianismo, essa
obra tem sido analisada como narrativa de uma cultura em transição,
perambulando entre o passado pagão e a nova religião monoteísta que
havia acabado de ser adotada nacionalmente.

No centro da tapeçaria nota-se uma espécie de congregação reunida numa
estrutura muito similar a uma igreja, constituindo a principal temática.
Além disso, em cada ponta de seu telhado encontra-se uma cabeça de
dragão, virada para fora como se protegesse a construção. Há, dentro da
igreja, uma torre com um sino e outra estrutura similar do lado de fora.
Considerando essa ilustração como o centro da tapeçaria, encontramos, à
sua direita, numerosos cavaleiros em suas montarias, enquanto que se
aproximam, pela esquerda da igreja, animais semelhantes a leões -- que,
vale lembrar, não são naturais do território escandinavo. Comumente,
atribui-se a essa parte da tapeçaria a conotação de que a igreja
encontra-se sofrendo um ataque.

Dimand ressalta uma curiosa figura que poderia representar uma deidade
pagã. Trata-se de um cavaleiro de três cabeças. Esta figura aparece,
ainda, conduzindo as bestas análogas a leões em direção à Igreja, como
se liderasse um ataque visando destruí-la. Algumas pessoas próximas à
congregação estão fora da Igreja tocando os enormes sinos, o que pode
representar um sinal de alerta e perigo, confirmando a ideia de que os
cristãos estavam sob ataque.

Detalhes interessantes estão também nas bordas. Ela é dividida em
pequenas seções retangulares, cada uma contendo temáticas e motivos
específicos. Dimand afirma que este modo de ornamentar as bordas é
encontrado também em pranchas de pedra encontradas em Gotland e também
na Escócia dos séculos \versal{IX} e \versal{X}. Além disso, as bordas da tapeçaria
apresentam também losangos, linhas paralelas em zigue-zague, padrões
irregulares de ganchos e cruzes. Estas últimas estão presentes
principalmente na borda esquerda, fora as cruzes em diagonal e as
chamadas cruzes ``negativas'' -- em formato parecido com um ``\versal{X}'' --,
muitas vezes bordadas em linha escura.

Os dois tipos de cruz que foram bordados na tapeçaria em questão revelam
conexões tanto com a arte oriental quanto com a ocidental. As cruzes
positivas, em que um dos cabos é longo, provavelmente são de origem
oriental e vieram da região da Rússia, com quem os vikings da região da
Suécia tiveram intenso e frequente contato. Já as cruzes negativas, com
ambos os braços curtos, são encontradas, por exemplo, numa pedra
ornamentada em Nigg, na Escócia.

Localizadas na extremidade esquerda da tapeçaria encontram-se três
figuras que são tanto emblemáticas quanto polêmicas: três homens, muito
próximos um do outro, com ornamentos em suas cabeças que lembram coroas,
cada um com traços ou objetos específicos que os diferenciam dos demais.
O único fato que pode ser afirmado com certeza é que, no momento de
confecção da tapeçaria, essas três figuras foram ilustradas de forma que
se destacassem e se diferenciassem das outras presentes.

Costuma-se atribuir a essas figuras as identidades dos reis/santos
escandinavos Olavo, Canuto e Érico. No entanto, Terje Leiren questiona essa
afirmação, embasando-se na descrição de um templo sueco feita por Adão
de Bremen -- cronista alemão da Alta Idade Média --, quando da tentativa
de cristianizar os escandinavos. Afinal, conforme afirma Rodrigo
Marttie, por mais que a \emph{Gesta Hammaburgensis Ecclesiae Pontificum}
-- História dos Arcebispos de Hamburg-Bremen -- de Bremen fosse de cunho
propagandístico, objetivando recuperar a memória de sua arquidiocese que
havia sido parcialmente incendiada, suas descrições da religião
escandinava antiga podem auxiliar em certos pontos.

Durante sua visita ao famoso templo de Uppsala, na Suécia, Adão de
Bremen descreve as estátuas de três deuses que supostamente eram
cultuados naquele lugar: o mais poderoso deles, Thor, ocupava a posição
central, sentado em um trono, enquanto que Odin e Freyr encontravam-se
cada um em um de seus lados. Conforme elabora Terje Leiren, apesar de
ser o deus criador e pai de tudo, Odin não ocupava o lugar central
nesses cultos por ser mais temido do que amado, e, portanto, mantinha-se
certa distância respeitosa dele. Já Thor, popular deus dos camponeses,
assumia a posição de deidade central no momento dos cultos. Portanto,
aplicando esse princípio à narrativa da Tapeçaria de Skog, é possível
levantar a hipótese de que os deuses nórdicos poderiam ser facilmente
representados de modo disfarçado, passando-se justamente por santos da
Igreja Católica.

Como a Suécia havia acabado de se converter ao cristianismo, era muito
provável que um artista contratado pela Igreja desejasse representar ali
os seus deuses antigos, embora encobertos pela temática cristã de modo
que fossem tidos por santos ou outras entidades queridas ao
cristianismo. Segundo essa linha de raciocínio, contudo, ainda assim é
possível identificar, por análise iconográfica, referências explícitas a
tais deuses. Santo Olavo, por exemplo, tornou-se frequentemente associado
a Thor devido a suas representações portando um machado. No entanto, no
caso específico da Tapeçaria de Skog, apesar de portar o supracitado
machado, Santo Olavo foi também representado sem um de seus olhos. Esta
poderia se tratar de uma referência clara a Odin, que sacrificou um de
seus olhos para que pudesse beber da fonte de Mimir e assim obter
conhecimento. Além disso, na tapeçaria esta figura também é a que se
encontra representada mais próxima de uma árvore que supostamente seria
Yggdrasil, onde, segundo os mitos escandinavos, o deus se pendurou de
cabeça para baixo, transpassado por uma lança, para que obtivesse o
segredo das runas.

Já o rei Canuto, figura central, segura um objeto duvidoso que poderia, a
princípio, estar representando um grande crucifixo, mas que também
talvez fosse um disfarce para ilustrar o deus Thor segurando seu
martelo, Mjöllnir. Por sua vez, localizado à direita, o rei Érico é
ilustrado segurando uma espiga de milho, o que nos remete imediatamente
à ideia de fertilidade simbolizada pelo deus Freyr, divindade do grupo
de deuses vanes associado justamente à fertilidade, abundância, à paz e
à riqueza.

Por mais que não se possa chegar ainda a uma conclusão definitiva sobre
essas três figuras, vale ressaltar que, se fossem vistas dessa forma,
elas estariam representadas na Tapeçaria de Skog da mesma maneira e na
mesma ordem em que se encontravam suas estátuas na descrição que Adão de
Bremen fez do templo de Uppsala. Thor estaria, nesse arranjo, ocupando a
posição central dos três deuses, enquanto que Odin e Freyr guardariam
cada um de seus lados.

\SIG{Victor Hugo Sampaio Alves}

Ver também Folclore; Religião; Suécia da Era Viking.

\begin{itemize}
\item \versal{ALVES}, Victor Hugo Sampaio. Breves considerações sobre os mitos nórdicos
na Tapeçaria de Skog. \emph{Notícias Asgardianas,} n. 11, 2016, pp.
12-21.

\item \versal{BREMEN}, Adão de. \emph{History of the Archbishops of Hamburg-Bremen}.
Tradução e notas de Francis J. Tschan. New York: Columbia University
Press, 2002.

\item \versal{DIMAND}, Maurice Sven. Mediaeval Textiles in Sweden. \emph{The Art
Bulletin}; 6 (1), 2015, pp. 11-16.

\item \versal{LEIREN}, Terje. \emph{From Pagan to Christian: The Story in the
12\textsuperscript{th}-Century Tapestry of the Skog Church, Hälsingland,
Sweden}. Universidade de Washington. 1999. Disponível em:
\href{http://faculty.washington.edu/leiren/skog.html}{\emph{http://faculty.washington.edu/leiren/skog.html}}.
Acesso em 07/02/2017.

\item \versal{MARTTIE}, Rodrigo Mourão. Adão de Bremen. In: \versal{LANGER}, Johnni (org.).
\emph{Dicionário de Mitologia Nórdica}: símbolos, mitos e símbolos. São
Paulo: Hedra, 2015, pp. 15-17.
\end{itemize}
\section{\versal{TAXAÇÕES E TRIBUTOS}}

Os relatos históricos acerca da tributação e taxações vikings são
escassos e em muitas vezes se desconhece os termos aplicados para os
impostos ou sobre o que eles incendiam. O conhecimento que se possui em
geral se refere a algumas fontes posteriores, específicas de
determinados reinos escandinavos ou da arqueologia ao se estudar as
mercadorias, moedas e outros vestígios materiais. Pelo fato de os
nórdicos do Período Viking não terem feito uso da escrita, não dispomos
de documentos comerciais, econômicos, alfandegários etc.

Antes do século~\versal{X}, o uso de moedas pelos vikings não foi uma prática
regular, logo, uma das medidas adotadas por várias cidades mercantis era
o uso de lingotes de prata como moeda de troca. Os lingotes de prata que
poderiam ser também moldados em formato de anéis, pesavam entre 48g a
50g, correspondendo ao peso habitual de lingotes encontrados em Gotland,
Frísia e na costa do Báltico.

A prata por ser um metal mais facilmente encontrado na Europa e na Ásia
islâmica, tornou-se o principal metal a ser usado nas transações de
vários povos. No caso dos nórdicos muitos lingotes usados no comércio,
poderiam ser objetos de prata derretida ou fundida, ou até mesmo pedaços
de outros objetos feitos de prata. Como o comércio e os impostos
alfandegários eram pautados no peso da prata, o uso de balanças foi
amplamente generalizado. Pesos de cobre e ferro foram os mais usados.

No entanto, Dagfinn Skyre salienta que em alguns casos os impostos
também eram pagos com ouro. Apesar de ser um metal mais raro de ser
encontrado na Europa, ainda assim, encontrou-se lingotes e anéis de ouro
similares aos feitos de prata, o que indica que o ouro possa ter sido
usado como moeda para o comércio, pagamento de tributos, taxas, etc.

Os reis e senhores (\emph{jarl}) cobravam tributos e taxas pelo direito
dos súditos de poderem comercializar livremente num mercado, ou por
viver sob sua proteção. Esse tributo poderia ser pago em prata,
mercadorias, grãos ou animais. Entretanto se desconhece a extensão dessa
prática nos territórios escandinavos. Thomas Streissguth comenta que a
tributação pelo uso da terra e outras propriedades comunais, não parece
ter sido uma prática comum entre os nórdicos.

Mary Valante observou que no caso irlandês tal prática foi bem comum e
difundida por toda a ilha na qual vivia sob jurisdição escandinava. É
provável que os nórdicos possam ter se baseado na tributação saxã,
germânica e franca da época, nas quais os senhores já cobravam o
pagamento de tributos \emph{in natura} ou em dinheiro, para usarem suas
pontes, portos, estradas, feiras etc.

Todavia, um dos tributos mais conhecido dos nórdicos era o
\emph{danegeld}, cobrado a partir de outros povos como forma de
extorsão. No caso um chefe viking instituía o \emph{danegeld} como um
acordo de trégua para o povo atacado. Estes deveriam anualmente pagar o
tributo geralmente em prata, ou em outras mercadorias e produtos para
evitar que seu território fosse saqueado.

\SIG{Leandro Vilar Oliveira}

Ver também Comércio; Moedas e cunhagem; Realeza; Sociedade.

\begin{itemize}
\item \versal{GULLBEKK}, Svein H. Coinage and monetary economies. In: \versal{BRINK}, Stefan;
 \versal{PRICE}, Neil (eds.). \emph{The Viking World}. London/New York: Routledge,
2008, pp. 159-169.

\item \versal{SKYRE}, Dagfinn (ed.). \emph{Means and Exchange}: dealing with Silver and
the Viking Age (Kaupang Excavation Project Publications Series, vol. 2).
Oslo: Aarhus University Press, 2007.

\item \versal{STREISSGUTH}, Thomas. \emph{Life among the Vikings}. San Diego: Lucent
Books, 1999.

\item \versal{VALANTE}, Mary A. Taxation, Tolls and Tribute: The Language of Economics
and Trade in Viking-Age Ireland. \emph{Proceedings of the Harvard Celtic
Colloquium}, vol. 18/19, 1998/1999, pp. 242-258.
\end{itemize}
\section{\versal{TECELAGEM E TECNOLOGIA TÊXTIL}}

A primeira evidência de produção têxtil na Escandinávia data do Período
Neolítico, há aproximadamente, 4.200 a C. Até essa época a tecelagem de
grandes tecidos não era tão desenvolvida, pois as técnicas empregadas
para a cardação, fiação e tecelagem ainda eram rudimentares e, portanto,
só permitia que pequenas peças fossem produzidas gastando para isso
muito tempo. Com o avanço das técnicas foi possível que essa produção
têxtil ficasse mais dinâmica, e as mulheres, encarregadas de todas as
etapas do processo, desde a tosquia das ovelhas e carneiros até os
arremates finais do tecido, pudessem tecer mais e com melhor qualidade.

Até o início da Idade do Bronze, os teares grandes ainda não estavam
amplamente difundidos na Escandinávia. A primeira prova direta de
tecnologia de tecelagem de tear nesta área vem de Jutlândia do
Norte, na Dinamarca, durante o início da Idade do Bronze (cerca de 1.800
a C). Ao longo do tempo, outros tecidos antigos foram encontrados nos
pântanos da Dinamarca, datando da Idade do Bronze em aproximadamente
1.500 a C. Um desses achados foi o enterro de Egtved, que continha o
corpo de uma jovem vestida com várias roupas destinadas para fins
rituais. A saia usada pela jovem mulher mostrou técnicas de tecelagem
mais avançadas, com uma urdidura mais uniforme e uma trama melhor acabada, de modo que se conclui que o tear que foi utilizado já possuía uma tecnologia mais avançada. A camisa encontrada no enterro foi tecida em um tear mais
simples, como se vê em sua trama e urdidura. Essas roupas proporcionam uma
visão dos materiais tecidos e da tecnologia têxtil utilizada durante a
Idade do Bronze na Escandinávia, assim como que as práticas funerárias incluíam um
vestuário ritual específico.

A partir de um número limitado de achados têxteis da Idade do Ferro
pré-romana na Escandinávia, concluiu-se que o tecido havia sido
produzido com uma tecnologia simples, em um tear de denominado
"warp-lock". Este tipo de tear permitia que a tecelã criasse bordas
finalizadas ou ourelas, que nada mais são do que o acabamento do tecido que
o arremata nas laterais tanto no sentido do comprimento como da largura,
impedindo assim que os fios se desfaçam em todos os quatro lados do
tecido. Esses arremates na urdidura do tecido foram detectados nos
restos das roupas de uma mulher, encontrada na Jutlândia em 1879, vestindo roupas de lã e um manto de pele de ovelha.
Na mesma área, foi encontrada uma peça de vestuário de lã, que foi
tecida com a parte superior dobrada sobre a própria trama, a fim de
criar uma saliência no tecido. Essas roupas estavam excepcionalmente bem
preservadas e esse nível de preservação se dá graças às águas
subterrâneas ácidas e falta de oxigênio que ajuda a preservar a lã, o
cabelo e a pele, mas destrói a matéria óssea e vegetal. Além das condições ambientais propícias para a sua
conservação, deve-se levar em conta que essas peças foram tecidas com um
tear de urdidura que permitia que as peças tecidas fossem elaboradas cada vez com mais qualidade e que se produzissem estruturas e padrões de tecelagem mais complexos.

Durante o Período de Migração (300/400-550/600) a Escandinávia, assistiu
a um aumento na produção de têxteis, mas também um aumento do comércio
de tecidos inclusive com tecidos provenientes da Europa continental e do
Mediterrâneo. As inovações da Idade do Ferro, como, por exemplo, os
teares de urdidura e os fusos espirais permitiram que os tecidos fossem
modelados com brocados encontradas nas sepulturas do período de migração
norueguês. A evidência dessa produção têxtil pode ser encontrada no
sítio do Período de Migração de Vallhagar, Gotland, onde inúmeros pesos
de tear e espirais de fuso foram escavados. Durante o Período Vendel
(550/600-750/800), ocorre uma alteração nos tecidos produzidos na
Escandinávia. Esse período viu um declínio na produção têxtil com a
possível importação de tecidos pela aristocracia escandinava. Esta aderiu
ao vestuário dos Francos, que incluía a importação de tecidos de
qualidade idêntica aos utilizados pela nobreza franca.

Durante a Era viking houve uma expansão comercial e o estabelecimento
de redes comerciais, proporcionando maior acesso à importação e
exportação de tecidos. Com a expansão do comércio, houve uma maior
demanda pelos mais diversos tipos de tecidos e em grandes quantidades,
incluindo aqueles tecidos e fibras utilizados nas atividades marítimas; como a população crescia, e consigo a demanda por mais roupas e velas para os navios, a produção em larga escala de tecidos tornou-se
fundamental. A variedade de fibras e estruturas têxteis encontradas nos
tecidos que sobreviveram, é uma espécie de testemunha das habilidades e
produtividade das tecelãs. Foram encontradas evidências de uma variedade
de tipos de lã vindas das Ilhas Britânicas e do Leste da Europa e de
Bizâncio que eram consideradas itens de luxo devido a sua textura e
também às técnicas utilizadas na sua elaboração. Os tecidos utilizados
para o vestuário, para o uso cotidiano em tarefas domésticas e os fios
empregados para a tecelagem das velas de navios, foram criados a partir
de uma variedade de fios de excelente qualidade -- as mais finas lãs,
linhos e até seda. As roupas confeccionadas com os tecidos mais finos e
caros tornaram-se ao longo do tempo produtos mais comuns para aqueles
que possuíam um status mais elevado, mas os tecidos mais rústicos, feitos
a partir de sarças e outras fibras vegetais, continuaram a ser usados
pelos mais pobres e no cotidiano.

Na produção dos tecidos, eram usadas algumas ferramentas. A maioria dos
tecidos não sobreviveram com o passar do tempo, restando apenas
fragmentos, mas algumas ferramentas para tecelagem foram encontradas
praticamente intactas, o que possibilita um estudo pormenorizado do seu
uso na produção de tecidos na Era viking. Essas ferramentas podem ser
consideradas a prova mais importante e abundante da produção e da
tecnologia de têxtil durante os séculos~\versal{VII}, \versal{IX} e \versal{X}. Algumas dessas
ferramentas incluem fusos, lançadeiras, pesos de tear e agulhas. Uma das
ferramentas têxteis mais comuns encontradas em sítios arqueológicos é o
fuso circular que podia ser feito com um pedaço de cerâmica ou
pedra. Também foram encontrados alguns feitos em âmbar, osso, marfim
e bronze. O peso na ponta do fuso o deixava parado permitindo que assim se girasse a fibra do fio.

Durante o Período Viking existem muitos exemplos de todas essas
ferramentas encontradas principalmente em túmulos. Nas sepulturas de
Gotland, foram encontrados fusos, lançadeiras e agulhas. Embora as
agulhas fossem consideradas uma ``tecnologia antiga'', era necessária
habilidade e conhecimento para usá-las corretamente.

\SIG{Luciana de Campos}

Ver também Cotidiano; Cultura material; Mulheres; Sociedade.

\begin{itemize}
\item \versal{KLESSIG}, Barbara K. \emph{Textile production tools from Viking Age
graves in Gotland, Sweden}. Dissertação de Mestrado em Artes pela
Faculdade Estadual de Humboldt, 2015.

\item \versal{SMITH}, Michèle Hayer. Weaving wealth: cloth and trade in Viking Age and
Medieval Iceland. In: \emph{Textiles and the Medieval Economy}:
Production, Trade, and Consumption of Textiles, 8th--16th Centuries.
Edited by Angela Ling Huang, Carsten Jahnke; Ancient Textile Series, vol
16, Oxbow Books, pp. 23-40.
\end{itemize}
\section{\versal{TECNOLOGIA}}

Os povos escandinavos do Período Viking, voltados à expansão marítima,
necessitavam de mão de obra capaz de desenvolver embarcações. Com efeito, pode-se
classificar a sociedade Escandinava de tal período histórico como
uma sociedade baseada em avanços de tecnologia náutica. As colonizações,
saques e contatos comerciais desenvolvidos por tais povos estavam, assim,
nas mãos de carpinteiros e tecelões que fossem capazes de desenvolver os
meios de expansão que permitiria a esses povos o alcance de regiões como
a Islândia, Groenlândia e o norte da América.

A navegação do Período Viking foi desenvolvida pela observação do sol,
das estrelas, de marcos mnemônicos terrestres e da observação de
animais. Parte dessas observações desenvolveriam equipamentos como a
bússola solar e a pedra do sol, aquela sendo uma bússola que funcionava
pela sombra gerada no momento de incidência solar em uma haste de metal
presa a uma placa de madeira marcada por coordenadas geográficas e esta
sendo uma pedra que permitia a captação das luzes solares mesmo em
momentos de céu fechado.

A grande tecnologia que impulsionou a expansão escandinava foi a vela, e
como fonte de pesquisa sobre o momento inicial de seu desenvolvimento, podemos nos utilizar
das imagens presentes nas estelas de Gotland. Ao comparar as imagens
anteriores e posteriores ao século~\versal{VII}, as representações nas estelas
nos demonstravam apenas embarcações movimentadas por remos, passando, após o
momento supramencionado, a representar a existência de velas. A
primeira embarcação movimentada por vela descoberta e datada pela
arqueologia, foi a embarcação de Oseberg -- que, por métodos
dendrocronológicos, é situada nos anos 825. O mastro da
embarcação de Oseberg foi montando na sobrequilha, colocado logo à
frente do meio do navio e era mantido por um apoio que se encontrava no
nível do convés. Tal apoio foi denominado como garfo, devido a lembrança que sua estrutura traz do utensílo. (\versal{BONDE}, \versal{CHRISTENSEN}, 1993, p.
575-583).

As embarcações eram de tipos diferentes, a começar pelas suas
variedades de funções: podemos salientar suas utilizações para o
transporte de guerreiros, o comércio e o transporte de cargas pelo mar
aberto. Tais funções faziam com que as embarcações do Período Viking
variassem também em sua estrutura, sendo chamadas de navios
aquelas que, normalmente, possuíam mais de doze remos, e de barcos
aquelas com menos de doze remos. Embora, em primeiro momento, as
embarcações de guerra e de comércio possam ser apontadas como sendo a
mesma embarcação. Isso se deve ao fato de que embarcações mercantes foram por vezes
utilizadas para guerra. Todavia, quando tratamos de uma utilização
predominante, esse quadro tende a mudar.

As embarcações de guerra eram desenvolvidas tanto para a utilização de remos como para a utilização de vela, sendo normalmente largas, longas e esguias. Tais embarcações eram divididas lateralmente, definindo o espaço
para a localização dos pares de remadores, chamados quartos, sendo que
algumas embarcações possuíam mais de trinta quartos e mais de sessenta
remos, de modo a serem denominadas "Longships".

As embarcações mercantis, por sua vez, eram mais compridas do que as de
guerra, possuíam um formato arredondado, um grande bordo livre e um
calado profundo. Como, na maior parte da vezes, as embarcações mercantis eram utilizadas para a navegação, seus mastros geralmente eram fixos. Em suas proas e popas era possível permanecer de pé ou ainda
permanecer sentado nos apoiadores de remo, quando esses existiam, e a
carga, normalmente carregada por essas embarcações, ocupava a parte
central da mesma.

Tais embarcações podiam ser divididas em duas, as \emph{Byrdingur} e as \emph{Knorr},
sendo a primeira direcionada para o comércio costeiro com uma
tripulação que variava de doze a vinte homens -- a embarcação Skuldelev 3,
que possui 13,8 m. de comprimento e bancos de 3,3 m., é apontada como
um exemplo de Byrdingur. A segunda, por sua vez, é tipologizada como a
maior das embarcações comerciais, e como exemplo dessa temos a Skuldelev
1, com 16,3 m. de comprimento e com bancos de 4,5 m.. Normalmente utilizada para navegação em alto mar (\versal{BONDE}, \versal{STYLEGAR}, 2011, p.
247-261).

As técnicas de construção naval foram identificadas por métodos de
arqueologia experimental desenvolvidas em Roskilde, no momento de
recuperação dos Skuldelevs, a partir do corte de madeira -- em sua maioria
carvalho -- de maneira radial, técnica que requer troncos com ao menos um
metro de diâmetro e com poucos nós. As pranchas produzidas dessa forma
são muito fortes devido ao seguimento do grão da madeira, e após secas
dificilmente encolhem ou deformam significativamente. Tal variação
diminuta é de suma importância, uma vez que a madeira é, por muitas
vezes, trabalhada logo após ser derrubada, algo que se deve a um melhor manuseio em relação à madeira seca. Contudo, mais ao norte, onde a árvore
majoritária é o pinho, o tronco é cortado em dois e cada metade é
talhada para formar as tábuas utilizadas nas embarcações. A curvatura
das pranchas de madeira era, na medida do possível, estabelecida pela
curvatura natural da madeira. Tal técnica permitia que a dimensão e o
peso da embarcação fossem reduzidos ao mínimo, permitindo o alcance de
um dos objetivos essenciais: tornar os navios flexíveis e fortes
(\versal{CROOME}, 1999, p. 382-393).

Seguindo as linhas naturais, a construção do navio já podia ser prevista
desde seu início, uma vez que após o posicionamento da base, feita de uma
haste de madeira oca que contava com linhas incisivas de cada lado, podia-se ver a partir das próprias linhas correspondências com os encaixes das demais tábuas do návio. Essas tábuas se juntariam à haste da base, continuando a linha de cada um dos lados no ângulo correto. As tábuas da popa e da proa eram,
provavelmente, armazenadas em água para que não secassem, deformassem e
rachassem antes de suas utilizações. A análise do Skuldelev 3
demonstrou-nos que o desenho da haste básica era baseado em segmentos de
círculos com diferentes diâmetros determinados pelo comprimento desejado
da quilha. O corte da madeira deveria ter, assim, uma regra básica e um
método simples determinado pelas demarcações feitas com corda e giz, uma
vez que o corte por si só determinava o formato e o encaixe da
embarcação (\versal{CRUMLIN-PEDERSEN}, 1986, p. 209-228).

\SIG{Munir Lutfe Ayoub}

Ver também Bússola solar; Embarcações; Navegação.

\begin{itemize}
\item \versal{BONDE}, Niels; \versal{CHRISTENSEN}, Arne Emil. Dendrochronological dating of the
Viking Age ship burials at Oseberg, Gokstad and Tune,
Norway.~\emph{Antiquity}, vol. 67, n. 256, 1993, pp. 575-583.

\item \versal{BONDE}, Niels; \versal{STYLEGAR}, Frans-Arne. Roskilde 6--et langskib fra
Norge--Proveniens og alder.~\emph{Kuml}, vol. 60, n. 60, 2011, pp.
247-261.

\item \versal{CROOME}, Angela. The Viking Ship Museum at Roskilde: expansion uncovers
nine more early ships; and advances experimental ocean-sailing
plans.~\emph{The International Journal of Nautical Archaeology}, vol.
28, n. 4, 1999, pp. 382-393.

\item \versal{CRUMLIN-PEDERSEN}, Ole. Aspects of Viking-Age Shipbuilding: In the Light
of the Construction and Trials of the Skuldelev Ship-Replicas Saga
Siglar and Roar Ege.~\emph{Journal of Danish Archaeology}, vol. 5, n. 1,
1986, pp. 209-228.
\end{itemize}
\section{\versal{THING}}

\emph{Thing} era a assembleia na qual a lei e justiça eram discutidas.
Eram realizadas com intervalos regulares e existiam em níveis locais,
regionais e nacionais. A \emph{Thing} mais conhecida é a \emph{Althing}
realizada na Islândia. Entretanto também se tem conhecimento de
assembleias realizadas na Noruega e Dinamarca, e eram também
estabelecidas em outras colônias escandinavas fora de sua terra natal,
como Escócia, Inglaterra e Irlanda. Nas \emph{Things,} o rei e a nobreza
local, assim como as elites regionais e o povo, reuniam-se para debates,
mas a influência relativa das partes variava de reino para reino. Sobre
essa e outras perspectivas, a elite local da Suíça aparenta manter sua
influência em um grau e tempo maior em comparação com as da Noruega e da
Dinamarca.

Na Noruega, as \emph{lögthings,} assembleias da lei, eram obviamente
instrumentais, servindo para pavimentar o caminho para as reformas eclesiásticas e
reais. Contudo, o oposto ocorria com a monarquia suíça que era
provavelmente confrontada por uma forte oposição dentro do sistema
provincial das \emph{Things.} Na \emph{Gesta Danorum}, famosa obra de Saxo Grammaticus, que data aproximadamente de 1200, é descrito que o rei dinamarquês deve ser aclamado nas quatro \emph{Things}
regionais, expondo a importância das instâncias regionais na legitimação
do poder real.

Na Islândia, a \emph{Thing} era uma unidade política no período de
autarquia, pois os islandeses aceitaram uma legislação comum, além de possuírem um sistema hierárquico das \emph{Things}, onde a \emph{Althing} se situava no topo. Era do âmbito das funções do \emph{godar} exercer o
seu papel judicial e administrativo. O poder foi concentrando-se em um
número decrescente de líderes e famílias, convertendo-se em
um sistema de senhoria territorial. Apesar da centralização de poder,
nenhum senhor local tinha os recursos para estender seu controle sobre
toda a Islândia, o que acabou levando a um conflito político que conduziu
a anexação da ilha ao reino norueguês.

\SIG{André Araújo de Oliveira}

Ver também Althing; Godi; Islândia na Era Viking.

\begin{itemize}
\item \versal{HOLMAN}, Katherine. \emph{Histocial Dictionaries of the Vikings}. Oxford:
The Scarecrow Press Inc., 2003.

\item \versal{SIGURÐSSON}, Jón Viðar. Iceland. In: \versal{BRINK}, Stefan; \versal{PRICE}, Neil (eds.).
\emph{The Viking World}. New York. Routledge, 2008, pp. 571-578.

\item  \versal{VÉISTEINSSON}, Orri. \emph{The Christianization of Iceland}: Priest,
Power and social change 1000-1300. Oxford: Oxford University Press,
2000.
\end{itemize}
\section{\versal{TRELLEBORG}}

Trelleborg é um forte circular situado em Zealand, na Dinamarca -- campo militar
que teve papel central na formação do reino de Haroldo Dente Azul durante
o século~\versal{X}, o forte exerceu função chave no controle e na administração
das províncias que emergiam no reino dinamarquês. Sua característica
inovadora na ilha de Zealand demonstra a influência estrangeira no reino
que estava surgindo. Todavia, esse forte não seria o único desse
período a apresentar esse padrão; podemos apontar outros fortes
circulares como os de Aggersborg ao norte da Jutlândia, Fyrkat próximo a
Horbo, Nonnebakken em Odense e Borgeby próximo da atual região de Lund,
na atual Suécia. Todos fortes que demonstram uma aparente uniformidade e
que seguiram um plano arquitetônico de formato e construção similares,
fato que levam os arqueólogos a afirmarem a construção dessas
localidades como contemporâneas e apontá-las todas para o século~\versal{X}
(\versal{BRUSGAARD}, 2012; \versal{DOBAT}, 2008, p. 27-67).

A influência estrangeira sobre os fortes dinamarqueses pode ser também
demonstrada pela amostra de artefatos de Trelleborg, tais indicam um
contato abrangente com as mais distantes regiões. Dentre esses artefatos
temos a cerâmica que possui paralelo com as cerâmicas eslavas da costa
sudeste do litoral báltico. Influência eslava que pode ser estudada
também por outros tantos artefatos da região, como, por exemplo, os
carretéis de costura e os pentes fabricados com cornos de cervos.

Situados no lado externo da saída leste do forte, haviam depósitos
funerários espalhados de ambos os lados da principal via, sentido
leste-oeste, que se ligavam ao complexo. A área escolhida para tais parece,
dessa forma, conscientemente integrada no padrão geral do complexo,
ressaltando a dialética da utilização de ambas as partes, forte e
sepulturas, para a formação de um conjunto de significados. Adentrar ao
forte pela sua entrada oeste era, assim, ter de passar pelos que ali
estavam depositados, em um constante contato com os antigos guerreiros
que haviam habitado a região.

Parte dos depósitos funerários de Trelleborg foi escavada entre os anos
de 1938 e 1940, um total de 133 sepulturas foram estudadas, tais
contavam com o depósito de 157 indivíduos, os fragmentos de ossos
espalhados indicam que o cemitério originalmente continha depósitos
adicionais. Grande parte das sepulturas continham depósitos individuais,
mas três das 133 sepulturas estudadas podem ser interpretadas como
contendo depósitos de muitos indivíduos, tais sepulturas são a de número
23 que continha 11 indivíduos e as de número 47 e 87, que contavam com 5 indivíduos cada, além de podermos destacar também depósitos duplos
como os de número 97 e 98. Todos os depósitos eram inumações em túmulos
de pouca profundidade, originalmente marcados sobre o chão,
provavelmente por pequenos montículos. As sepulturas tinham orientação
leste-oeste, com boa parte dos mortos com suas faces voltadas em sentido
leste.

Os achados de artefatos nas sepulturas da região não foram de grande
volume, um total de 27 sepulturas continham pequenas facas, em 9 casos
achadas junto com pedras de afiar. Outros artefatos presentes nos
depósitos funerários incluem a presença de contas de vidro em nove dos
depósitos, além de alguns outros achados de acessórios de vestimenta.
Apenas três depósitos contavam com armas, todos possuindo exclusivamente
machados (\versal{RAFFIELD}, 2013, p. 1-29).

O estudo arqueológico de maior repercussão sobre o forte de Trelleborg
não ocorreu, no entanto, durante o século~\versal{XX}, mas viria apenas a ocorrer
em 2011 e envolveria a análise de isótopo de estrôncio dos ossos achados
nos depósitos funerários da região. O principal objetivo de tal estudo
era conseguir determinar os locais de origem dos homens que haviam sido
depositados em Trelleborg, uma vez que a arquitetura do forte e os
achados da região apontavam para influências advindas do mundo eslavo
(\versal{PRICE}, \versal{FREI}, \versal{DOBAT}, \versal{LYNNERUP}, \versal{BENNIKE}, 2011, p. 476-489).

Os estudos de isótopo de estrôncio são realizados pela comparação dos
índices do isótopo presente no osso humano com os índices de isótopo
presentes em ossos de animais ou mesmo na flora das diferentes regiões.
Esse método utiliza-se da variação de isótopo de estrôncio que ocorre de
forma distinta nas mais diversas formações geológicas, dependendo da
idade de tais formações e do conteúdo original de rubídio nas rochas e
sedimentos de cada região. Os isótopos de estrôncio movem-se das rochas
para as formações ósseas humanas por meio da cadeia alimentar, estrôncio
que no corpo humano age como um substituto do cálcio na formação dos
ossos. O osso é continuamente remodelado durante a vida humana, dessa
forma a composição química dos ossos reflete os últimos anos de vida de
um individuo, no entanto, o esmalte dentário forma-se durante a infância
e sofre pouca mudança em fases posteriores. O isótopo de estrôncio
presente no esmalte dos dentes humanos é utilizado, assim, para
delimitar o local de origem de cada individuo, enquanto o mesmo isótopo
em outros ossos indica o local em que esses viveram os últimos anos de
suas vidas.

Ao fim, o estudo do isótopo de estrôncio no esmalte dos dentes dos
indivíduos de Trelleborg demonstrou uma grande variedade em seus
valores, fato que levou os arqueólogos a afirmarem a origem desses
homens não apenas para os atuais países nórdicos, como Noruega e Suécia,
mas de todo o mar báltico. Os estudos do forte concluíram, assim, que a
armada que o ocupava era composta por um misto de mercenários das mais
diversas regiões do norte da Europa.

\SIG{Munir Lutfe Ayoub}

Ver também Dinamarca da Era Viking; Fortificações; Tecnologia; Viking.

\begin{itemize}
\item \versal{BRUSGAARD}, Nathalie Østerled.~\emph{Places of Cult and Spaces of Power}.
Master Thesis, Leiden University, 2012.

\item \versal{DOBAT}, Andres Siegfried. Danevirke Revisited: An investigation into
military and socio-political organisation in South Scandinavia (c \versal{AD} 700
to 1100).~\emph{Medieval Archaeology}, vol. 52, n. 1, 2008, pp. 27-67.

\item \versal{PRICE}, T, Doulgas; \versal{FREI}, Karen Margarita; \versal{DOBAT}, Andres Siegfried;
 \versal{LYNNERUP}, Niels; \versal{BENNIKE}, Pia. Who was in Harold Bluetooth's army?
Strontium isotope investigation of the cemetery at the Viking Age
fortress at Trelleborg, Denmark.~\emph{Antiquity}, vol. 85, n. 328,
2011, pp. 476-489.

\item \versal{RAFFIELD}, Ben. Antiquarians, Archaeologists, and Viking
Fortifications.~\emph{Journal of the North Atlantic}, vol. 20, n. 1-29,
2013, pp. 01-29.
\end{itemize}
\chapter{U \textarn{u}}
\section{\versal{URBANIZAÇÃO}}

Comparada a outras regiões da Europa, nas quais nota-se um processo
urbano que retoma a Roma imperial, a urbanização da Escandinávia começou
durante a Alta Idade Média, tendo um começo lento, o qual foi alavancado
nos séculos \versal{IX} e \versal{X} com a política e a economia, isso em termos da Era
viking (séculos \versal{VIII-XI}), pois os países nórdicos somente começaram a se
tornar mais urbanizados no final do medievo. Dagfinn Skre comenta que
durante a Era Viking apenas 1\% ou 2\% da população escandinava habitava
cidades, e esse percentual no começo do século \versal{IX} estava restrito
basicamente a quatro cidades: Birka na Suécia, Kaupang na Noruega, Ribe
e Hedeby na Dinamarca. Juntas essas cidades deveriam, segundo Skyre,
contar com uma população de 3 a 4 mil habitantes.

Por mais que se fale em cidades vikings, é preciso ter em mente que se
tratava de pequenos núcleos urbanos, em geral mal passavam dos mil
habitantes. Muitas das cidades escandinavas, devido à prosperidade que
alcançaram nos séculos \versal{IX} e \versal{X}, foram muradas. Assim, uma das
características básicas de uma cidade escandinava da Era viking era ser
um aglomerado de casas, de forma desordenada na maioria das vezes,
embora em alguns casos houvessem ruas largas e pavimentadas, cercadas
por uma muralha de madeira e terra. Cidades como Hedeby, Birka e Dublin
possuíam essa característica específica.

Pelo fato da maioria das edificações terem sido construídas com madeira,
elas não resistiram ao passar do tempo. Logo, pouco se conhecem da
extensão, organização e aparência dessas cidades. Muito do que se
conhece da disposição de sua geografia advém de relatos de viajantes.
Por outro lado, algumas cidades foram abandonadas e outras surgiram em
cima das cidades velhas, o que dificulta o trabalho arqueológico de
investigação.

Hans Andersson comenta que os motivos que levaram ao surgimento das
cidades entre os vikings não são homogêneos. Determinados fatores locais
podem ter sido essenciais para o surgimento de cidades. Em alguns casos
nota-se que muitas cidades costeiras como Kaupang, Hedeby, Ribe e Birka
teriam se formado a partir de fatores econômicos e de sobrevivência.
Pelo fato de tais cidades possuírem rios navegáveis ou saídas para o
mar, isso favorecia o transporte, a pesca, o contato e o comércio. À
medida que alguns camponeses passaram a enxergar a possibilidade de
vender seu excedente agrícola ou fabricar objetos, ferramentas e armas
para fora, feiras começaram a se desenvolver em tais locais.

Apesar de fatores diferentes terem influenciado o início dos centros
urbanos, o comércio foi a principal motivação para o desenvolvimento
urbano como observa Hans Andersson, Helen Clarke e Dagfinn Skre. Os
habitantes das cidades deixaram de se preocupar com os afazeres
agrícolas e pecuários e focaram na produção manufatureira e no comércio.
Em geral tal característica é comum entre vários povos e ainda hoje se
sucede. As cidades podem concentrar as indústrias, lojas e mercados, mas
a matéria-prima ainda vem de fora. No caso viking isso não foi
diferente.

Para Skre, na Escandinávia houve duas ondas de urbanização. A primeira
começou em meados do século \versal{VIII}, quando surgiu a cidade de Birka,
situada numa pequena ilha no Lago Mälaren, na Suécia. Por ser uma rota
de passagem de quem vinha do interior para a costa, Birka começou a
atrair os viajantes. Com isso as fazendas locais começaram a oferecer
alimentos, bebidas e acomodações para os viajantes. Com o tempo,
estalajadeiros, mercadores, artesãos etc., se estabeleceram ali e
formaram uma feira, atraindo os produtores locais para negociar seus
produtos e comprar mercadorias importadas. Skre sublinha que tais
características parecem ter ocorrido no caso de Ribe, Kaupang e
Vestfold.

Por sua vez, a segunda onda de urbanização somente ocorreu no final da
Era Viking, começando por volta do ano 1000. Nesse período algumas
cidades como Birka, haviam sido abandonadas, enquanto cidades como
Kaupang, Jelling, Vestfold, Ribe e Hedeby ainda se mantinham. Data dessa
segunda expansão o surgimento das cidades de Sigtuna na Suécia, Århus,
Lund e Roskilde na Dinamarca, Oslo e Trondheim na Noruega. Oslo é a
atual capital da Noruega, e Trondheim era na época a cidade mais ao norte
do reino norueguês.

O processo de urbanização escandinavo da Era Viking esteve
principalmente relacionado a fatores econômicos, pois apesar de
algumas cidades serem grandes e ricas para os padrões da época, não
significava que os reis ali habitassem. Por exemplo, na Dinamarca, as
cidades de Ribe e Hedeby eram as mais economicamente desenvolvidas, mas
os reis Gorm, o Velho, Haroldo Dente Azul e Sueno Barba Bifurcada viviam
em Jelling, situada no meio da península da Jutlândia. Ribe e Hedeby
possuíam saídas para o mar, o que favorecia o comércio, mas também a
chegada de frotas inimigas. No caso de Hedeby sua proximidade com o
Sacro Império Romano-Germânico a tornou alvo dos germanos e eslavos.

Por tais motivos, algumas dessas cidades mercantis não se tornaram a
sede dos reinos devido ao fator de que apesar de serem locais prósperos
e que possuíam defesas para caso de invasão, os monarcas preferiam
outras localidades. No entanto isso não retirava a importância política
e estratégica dessas cidades mercantis, que eram fortificadas para
inibir ataques e impedir que invasores se apossassem de suas riquezas.

Outro fator que o processo de urbanização contribuiu além do crescimento
comercial, da conexão com o comércio internacional, interesses políticos
e militares de controle dessas rotas e produções, as cidades favoreceram
para a difusão de uma fé estrangeira, o Cristianismo. Hans Andersson
aponta que uma das características do cristianismo antigo era ser
difundido primeiro nas cidades para depois alcançar os moradores do
campo. Na Escandinávia o mesmo ocorreu.

As primeiras igrejas e catedrais que se possui na Dinamarca, Noruega e
Suécia foram erguidas nas cidades mercantis, pois além de serem locais
que atraiam mercadores, atraia também toda a variedade de viajantes, por
distintos fatores e interesses. Assim ainda no século \versal{IX}, encontram-se
missionários como São Oscar viajando pela Dinamarca e Suécia e iniciando
as obras de igrejas.

\SIG{Leandro Vilar Oliveria}

Ver também Comércio; Era Viking; Escandinávia; Habitação.

\begin{itemize}
\item \versal{ANDERSSON}, Hans. Urbanisation. In: \versal{HELLE}, Knut (ed.). \emph{The
Cambridge History of Scandinavia}, vol. 1: Prehistory to 1520. New York:
Cambridge University Press, 2003, pp. 312-342.

\item \versal{CLARKE}, Helen. Cidades, comércio e ofícios. In: \versal{GRAHAM-CAMPBELL}, James
(org.). \emph{Os vikings}. Barcelona: Editora Folio \versal{S.A.} 2006, pp.
78-88.

\item \versal{SAWYER}, Peter H. \emph{Kings and Vikings}: Scandinavia and Europe \versal{AD}
700-1100. London/New York: Routledge, 1982.

\item \versal{SKRE}, Dagfinn. The development of urbanism in Scandinavia. In: \versal{BRINK},
Stefan; \versal{PRICE}, Neil (eds.). \emph{The Viking World}. London/New York:
Routledge, 2008, pp. 83-93.
\end{itemize}
\chapter{V}
\section{\versal{VALSGARDE}}

Valsgarde é uma fazenda às margens do rio Fyris, na localidade de Vendel
no distrito de Uppland, cerca de três quilômetros ao norte de Gamla
Uppsala, centro da antiga religião nórdica e dos antigos reis Svears. A
fama dessa localidade deriva dos depósitos funerários praticados entre
os séculos \versal{VI} ao \versal{XI}. A região seria escavada por arqueólogos pela
primeira vez na década de 1920 e ficaria famosa pelos achados de 1933
que contemplaram o depósito funerário número sete (\versal{ALKEMADE}, 1991, p.
267-291; \versal{ALMGREN}, 1983, p. 11-16; \versal{ANDERSON}, 1983, p. 31-38; \versal{ARRHENIUS},
1983, p. 39-70).

Tal depósito funerário apresentaria equipamentos de guerra, como um elmo
típico do período Vendel, três escudos, duas espadas, dois saxes, uma
ponta de lança, cinquenta e três pontas de flecha e dois cintos com
bainhas para espada; peças de jogos; achados têxteis, como uma cama de
penas e alguns travesseiros; pentes feitos de ossos de animais; animais
sacrificados, encontrando-se ossos de gado, porco, ovelha, coruja, galo,
pato e ganso; ferraduras de cavalo; uma sela; quatro freios; coleiras
para cães e um corno de bebida. O depósito foi realizado por inumação em
uma embarcação de 8,5 m. de comprimento construída em carvalho.

Por achados como o do deposito funerário de número sete, Valsgarde seria
classificada como um local pertencente à aristocracia e reforçaria a
importância da localidade de Vendel, contribuindo para o nome dado ao
período da Idade do ferro, dentre os séculos \versal{VI} e \versal{VIII}, que não por
acaso, quando tratamos de dividir a Idade do ferro sueca, acaba por
carregar o nome da já supramencionada localidade, sendo denominado
período Vendel.

\SIG{Munir Lutfe Ayoub}

Ver também Arqueologia da Era Viking; Suécia da Era Viking; Viking.

\begin{itemize}
\item \versal{ALKEMADE}, Monica. A history of Vendel Period archaeology: Observations
on the relationship between written sources and archaeological
interpretation. In: \versal{THEUWS}, Franciscus Cornelius Wilibald Josephus;
\item \versal{ROYMAS}, Nico (eds.). \emph{Images of the past: Studies on Ancient
Societies in Northwestern Europe}. Amesterdam: Universiteit van
Amesterdam Giffen-Institut, 1991, pp. 267-291.

\item \versal{ALMGREN}, Bertil. Helmets, Crowns and Warrior's Dress: From the Roman
Emperors to the Chieftains of Uppland.~\emph{Vendel Period Studies},
vol. 2, 1983, pp. 11-16.

\item \versal{ANDERSON}, Phyllis. Aspects of site topography and boat morphology of the
inhumation boat graves of Vendel period Sweden. \emph{Vendel Period
Studies}, vol. 2, 1983, pp. 31-38.

\item \versal{ARRHENIUS}, Birgit. The chronology of the Vendel graves.~\emph{Vendel
Period Studies}, vol. 2, 1983, pp. 39-70.
\end{itemize}
\section{\versal{VAREGUES}}

Varegue, também conhecidos na literatura lusófona como varegues ou
varângios, foi a denominação dada inicialmente pelos bizantinos e mais
tarde pelos Rus aos vikings que se aventuraram nas terras ao leste da
Escandinávia, denominada pelos nórdicos de Austrvegr em algumas de suas
fontes. Estes vikings eram inicialmente provenientes da Suécia, mas
eventualmente o termo se tornou generalizante para escandinavos em Rus e
para mercenários vindos do norte em Bizâncio. A nomenclatura, tanto
\emph{varangoi} no idioma grego quanto \emph{variag''} no eslavo
eclesiástico antigo, tem sua raiz na palavra \emph{væringi}, do nórdico
antigo. Sigfús Blöndal e Benedikt Benedikz argumentam que o termo
significaria "companheiro fiel" devido ao significado do radical
\emph{vár} (fé em, voto de fidelidade). O termo em si viria do germânico
\emph{wādrenga}, tendo o sentido de um estrangeiro que serve seu senhor
a partir de um contrato de lealdade. Mas os autores chamam atenção para
o fato de que a nomenclatura apareceu em Rus antes que em Bizâncio, e o
sentido se aplicaria mais para a relação dos escandinavos com os gregos.

A partir do século \versal{VIII}, os varegues começaram a se aventurar pelas
terras do leste em busca de prata e produtos como peles e âmbar. Ingmar
Jansson afirma que a principal atividade econômica dos varegues seria o
comércio e tributação, o que deu origem a um processo de colonização. Um
destes grupos de varegues que se assentou na Rússia Europeia foram
denominados de "Rus" por francos, bizantinos e árabes. A tradição da
\emph{Crônica dos Anos Passados} afirma que os varegues chegaram
inicialmente em Rus em meados do século \versal{IX}, com os irmãos Riúrik, Sineus
e Truvor sendo requisitados pelos eslavos locais para que os liderassem.
Mas evidências arqueológicas mostram que os nórdicos estiveram presentes
na Rússia Europeia desde o século~\versal{VIII}, com seu posto comercial em
Staraia Ladoga no atual noroeste da Rússia. É aceito pela historiografia
que a existência de uma unidade política antes do estabelecimento de
Riúrik e seus irmãos no território é possível, com tal unidade sendo
conhecida como "Khaganato de Rus", embora não se saiba onde se
localizava sua capital, com Stáraia Ladoga, Gnezdovo e Riurikovo
Gorodische sendo as principais candidatas.

A partir do final do século~\versal{IX} e motivados por um comércio direto com
Bizâncio, os Rus que antes se encontravam no norte foram ao sul, em
direção ao rio Dniepre e a cidade de Kiev. O caminho fazia parte de uma
das rotas mais importantes do Leste: a chamada "Rota dos Varegues aos
Gregos" que ligava o Mar Negro ao Mar Báltico. Mas em 860 os varegues
atacaram e pilharam os arredores da capital imperial e, mesmo sem
conseguir atacar Constantinopla devido a sua fortificação, eles se
mostraram, conforme fontes bizantinas, como forças devastadoras. Os
varegues de Rus atacaram os bizantinos várias vezes entre os séculos~\versal{IX}
e \versal{X}, aparentemente entrando em trégua após a morte de Sviatoslav
Igorevich (964-972). Alguns tratados de paz entre os varegues de Rus e
os bizantinos presentes na \emph{Crônica} dão uma noção do poder
político dos escandinavos dentro da Rússia Europeia: no tratado de 912
aparecem quinze varegues com nomes claramente escandinavos a serviço de
Oleg, o Profeta (882-912), sucessor de Riurik, e dos outros príncipes e
aristocratas de Rus que estavam sob o comando de Oleg, demonstrando que
o príncipe de Kiev tinha a primazia entre os demais. Os varegues fizeram
parte da elite de Rus durante toda a era viking até a eventual
"eslavização da elite", em especial formando o séquito militar que
servia como um exército particular dos príncipes conhecido como drujína.

Os varegues Rus também chegaram em terras islâmicas. A maior parte da
prata que os varegues buscavam vinha das regiões subjugadas ao Califado
Abássida e de outros territórios islamizados, sob a forma de moedas de
prata chamadas \emph{dirhams}. A grande quantidade destas moedas
presentes ao longo do leste geralmente são utilizadas por arqueólogos
para verificar locais de assentamentos varegues e a datação de suas
estadias ou viagens. Parte da Rússia europeia, a Bulgária do Volga foi
um dos territórios islamizados: localizada na rota que ligava o rio
Volga ao mar Cáspio, rota esta que foi inicialmente mais importante que
aquela cujo destino era Constantinopla. As trocas não aconteciam somente
nos locais árabes, ocorrendo também em Staraia Ladoga, Novgorod e Itil,
capital do Khaganato da Khazária, por meio de intermediários, geralmente
kházaros, entre os árabes e os escandinavos. Além da prata, os varegues
conseguiam com os árabes contas e seda, e em troca ofereciam escravos,
mel, cera e peles.

Alguns varegues preferiram ficar em Mikligardr (nome com o qual os escandinavos
chamavam Constantinopla) e atuaram como uma guarda mercenária a serviço
do exército bizantino e do Imperador, em uma elite militar conhecida
como "Guarda Varegue". É provável que esta elite tenha existido desde o
século~\versal{IX}, com uma menção de Rus vindos da Suécia e enviados pelo
Imperador bizantino Teófilo (829-842) presente na fonte carolíngia
\emph{Annales Bertiniani,} mas ela só passou a ganhar força a partir da
última década do século~\versal{X} no governo de Basílio~\versal{II} (960-1025), e do envio
de tropas varegues por Vladimir~\versal{I} Sviatoslávitch (978-1015) para
auxiliar o Imperador contra uma revolta interna. A Guarda
Varegue era inicialmente composta somente por varegues de Rus, porém
noruegueses, dinamarqueses e até bretões se juntaram à elite
guerreira com o passar dos anos, e é provável que até mesmo alguns
eslavos também fizeram parte. Mesmo com a maioria sendo escandinava, o
líder não era necessariamente um varegue, podendo ser bizantino desde
que compreendesse o idioma dos varegues.

Segundo Raffaele D'Amato, o número de guerreiros chegou a seis mil no
século~\versal{X}, mas esta quantidade pode ser exagero das fontes. Nestas, o
número oscila entre as descrições de batalhas, com o número máximo
podendo chegar realmente a três mil varegues. A Guarda Varegue foi
utilizada esporadicamente e somente em últimas instâncias, e nos grandes
cercos eles tinham o direito do primeiro saque e pilhagem. Não se sabe
ao certo a estratégia de batalha dos varegues, mas eles utilizavam
primariamente machados e lanças. Conforme o relato de Anna Commena, é
possível que parte da Guarda possuíse uma cavalaria ou alguns varegues
pudessem batalhar a cavalo. De acordo com Blöndal e Benedikz, a Guarda
Varegue sobreviveu até o século~\versal{XIII} aceitando mercenários do norte que
não necessariamente eram escandinavos, mas alguns resquícios
permaneceram na organização militar bizantina até a dominação otomana em
1453. Um dos membros mais ilustres, o rei norueguês Haroldo Hardrada
(1046-1066) fez parte da Guarda Varegue entre 1034 e 1041, quando um
incidente com os imperadores bizantinos fez com que ele fosse preso e
fugisse de Constantinopla.

Além de Rus e do Império Bizantino, os varegues se aventuraram pela
região do Báltico. Conforme diz André Muceniecks, as expedições nórdicas ao
leste são diretamente relacionadas com as primeiras incursões no
Báltico. A arqueologia mostra que havia atividade de escandinavos na
Estônia e na Letônia desde o século~\versal{VI}, enquanto na Lituânia não havia
um contato expressivo. O rio Daugava possuía uma rota que permitia o
acesso à "Rota do Varegues aos Gregos", e há indícios de postos
comerciais na área fundados no século~\versal{X}. Mas a quantidade de
\emph{dirhams} encontrada ao longo da região é pequena, indicando que o
Báltico não era fundamental na comercial varegue. Heiki Valk ainda
afirma que não houve um impacto escandinavo considerável na cultura dos
povos bálticos, mesmo com a relação entre ambos sendo baseada em
aspectos comerciais e militares.

\SIG{Leandro César Santana Neves}

Ver também: Crônica dos Anos Passados; Kiev; Mikligardr; Novgorod; Olga
de Kiev; Rus; Rússia da Era Viking; Staraia Ladoga; Vladimir~\versal{I} de Kiev.

\begin{itemize}
\item \versal{BLÖNDAL}, Sigfús. \emph{The Varangians of Byzantium.} Traduzido por
Benedikt S. Benedikz. Cambridge: Cambridge University Press, 1978.

\item  \versal{D'AMATO}, Raffaele. \emph{The Varangian Guard 988-1453} ("Men-at-Arms"
series 459). Oxford: Osprey, 2010.

\item \versal{DUCZKO}, Wladsyslaw. \emph{Viking Rus: studies on the presence of
Scandinavians in Eastern Europe.} Leiden: Koninklijke Brill \versal{NV}, 2004.

\item \versal{JANSSON}, Ingmar. Warfare, Trade or Colonisation? Some General Remarks on
the Eastern Expansion of the Scandinavians in the Viking Period. In:
 \versal{HANSSON}, Pär. \emph{The Rural Viking in Russia and Sweden}. Örebro:
Örebro kommuns bildningsförvaltning, 1997, pp. 09-55.

\item \versal{MUCENIECKS}, André Szczawlinska. \emph{Austrvegr e Garđaríki -
(re)significações do leste na Escandinávia tardo-medieval.} Tese de
Doutorado em História Social. São Paulo: Faculdade de Filosofia, Letras
e Ciências Humanas, \versal{USP}, 2014.

\item \versal{SHEPARD}, Jonathan. The Viking Rus and Byzantium. In: \versal{BRINK}, Stefan;
 \versal{PRICE}, Neil (eds.). \emph{The Viking World.} London: Routledge, 2008,
pp. 476-516.

\item \versal{VALK}, Heiki. The Vikings and the Eastern Baltic. In: \versal{BRINK}, Stefan;
\versal{PRICE}, Neil (eds.). \emph{The Viking World.} London: Routledge, 2008,
pp. 485-495.
\end{itemize}
\section{\versal{VESTUÁRIO}}

As descobertas arqueológicas de roupas do Período Viking podem ser
consideradas raras, pois todas as peças do vestuário eram feitas com
fibras naturais e, portanto, degradáveis. Esses achados na sua maioria
foram provenientes de descobertas realizadas em enterramentos,
geralmente são pequenos pedaços de material preservados por acaso. Nosso
conhecimento sobre o vestuário viking é complementado por fontes
escritas, como as sagas, e por roupas retratadas em figuras pequenas,
como pingente e algumas tapeçarias.

Os homens e as mulheres se vestiram de acordo com sexo, idade e status
econômico e social. Os homens usavam as calças e as túnicas, e as
mulheres usavam faixas que envolviam as pernas e o baixo ventre, como
roupas íntimas. As roupas mais comuns, usadas no cotidiano e
principalmente para o trabalho nas fazendas, eram feitas de materiais que
cultivados pelas próprias famílias que os consumiam, como lã e
linho, que eram cardados, fiados e tecidos pelas mulheres. Alguns
fragmentos de tecido encontrados em túmulos de indivíduos mais ricos
mostram que algumas roupas eram importadas. Os mais abastados exibiam
sua riqueza utilizando adornos nas roupas feitos com fios de seda e
ouro, importados principalmente de Bizâncio. Alguns acessórios, como se
denomina hoje, complementavam o vestuário, tais como joias e peles de
diferentes animais, os quais eram usadas para complementar as capas e túnicas
e também como estolas que tinham uma finalidade estética ao mesmo tempo aqueciam o pescoço e as orelhas.

As mulheres normalmente usavam uma veste que recebia o nome de
\emph{smokkr}. Essa veste era composta por uma túnica (que é uma \emph{chemise},
utilizada pelas mulheres durante toda a Idade Média) larga feita de
linho ou lã -- que servia tanto para uso diurno quanto como roupa de dormir, facilitando assim o ato de vestir-se pela manhã -- e um avental, usado sobre a túnica, feito de um tecido mais
grosso, geralmente lã, que era ajustado ao corpo e costurado. Além disso,
esses aventais possuíam pregas que eram costuradas no vestido para
modelá-lo, deixando-o mais resistente. Esse avental era ajustado
sobre o peito e preso por uma alça em cada ombro. A alça era fixada na
frente com um par de broches, geralmente de forma oval e abaulados, feitos
com diferentes materiais. As mulheres mais ricas usavam broches de metais
preciosos; as mais pobres usavam os mesmos ornamentos feitos geralmente
com ossos e até com madeira. Entre os dois broches, era comum se usar um
colar feito com contas coloridas de vidro, âmbar, madeira e conchas. O tamanho do colar também dependia do quão abastada era a dona, de modo que as mulheres mais ricas possuíam os maiores deles. As mulheres também usavam uma capa
sobre os ombros, que ficava presa com um pequeno broche redondo, e, além das peles, o manto e o vestido podiam ser decorados com bordas tecidas e faixas decorativas.

As roupas usadas pelas crianças eram muito semelhantes às roupas usadas
por seus pais. As meninas jovens vestiam túnicas e aventais,
enquanto meninos jovens vestiam túnicas e calças.

Os homens geralmente usavam uma túnica, uma calça e uma capa. A túnica
era uma camisa de braços compridos sem botões e poderia descer até os
joelhos, muito semelhante a chemise feminina. Sobre os ombros, o homem
usava um manto, que ficava preso com um broche. O manto ficava recolhido
sobre o braço com o qual ele puxava a espada ou o machado. Desta
maneira, era possível reconhecer se um homem era destro ou canhoto. As
calças tinham um corte simples e bem largas para não prender os
movimentos, com duas partes iguais, costuradas nas laterais internas e
externas, enquanto na cintura eram presas por cordões de tecido ou couro. Em
torno das pernas, para mantê-las aquecidas, enrolavam faixas largas de
lã. Como calçado, tanto os homens como as mulheres usavam sapatos ou
botas de couro, muitas vezes forrados com pele para ficarem mais
quentes. Suas roupas não possuíam bolsos e homens e mulheres podiam
usar cintos ou então cordões presos ao redor da cintura para segurar
suas roupas. Em seu cinto, podiam levar uma bolsa ou uma faca. A bolsa
poderia conter vários objetos, como um pente, um limpador de unhas,
peças de jogo, moedas de prata e agulhas. Alguns homens também usavam
uma espécie de gorro, forrados e ornamentados com pele. Algumas roupas
recebiam um tratamento especial para ficarem impermeáveis. Os fios de lã
eram tratados com cera de abelha para torná-los macios e óleo de peixe
para que fossem impermeáveis.

Durante a Era Viking, os mais ricos tiveram acesso a produtos vindos de
várias partes do mundo e isso se refletiu em suas roupas. O estilo de
vestuário da corte bizantina, em particular, inspirou algumas das roupas
usadas pela alta aristocracia dinamarquesa. Prova disso é que os
enterros dinamarqueses que datam do final da década de 900 apresentam
fragmentos de roupas e ornamentos que faziam parte dos círculos
judiciais europeus cristãos, que importavam tecidos de Bizâncio. Nesses
círculos, a seda estava entre os materiais mais procurados e estava
associada ao prestígio. Além disso, as diferentes cores de seda
simbolizavam riqueza e poder. As cores azul e vermelha eram
especialmente procuradas justamente por serem as mais chamativas e
estarem diretamente associadas ao poder social e econômico. Estes
tecidos e cores foram usadas pelo príncipe Mammen de Bjerringhøj, na
Jutlândia, Dinamarca. Suas roupas vermelhas e azuis eram as de um homem
muito rico e poderoso.

As roupas foram tecidas em muitas cores diferentes. O fio colorido
poderia ser produzido ao ser fervido com várias plantas que produziam
corantes. As cores que os arqueólogos identificaram como sendo as mais
usadas na Era Viking foram o amarelo, vermelho, roxo e azul. O azul
só foi encontrado nos enterros de indivíduos ricos, pois aparentemente
era uma cor de difícil acesso e custava caro. A cor azul podia ser
encontrada em plantas nativas ou então importadas como é o caso do
índigo, que era adquirido no exterior. Cerca de 40\% dos achados de
tecido da Era Viking foram identificados como linho. O linho deve,
portanto, ter sido uma planta importante para a produção de roupas para
toda a comunidade. As pesquisas apontam que eram necessários mais de
vinte quilos de fibra de linho para produzir material suficiente para
fazer uma túnica, por exemplo. Além disso, a tarefa -- desde a semeadura do linho
até a túnica totalmente pronta, provavelmente exigia quase
quatrocentas horas de trabalho. Vários locais da Dinamarca, foram
encontrados vestígios da produção, quase industrial de linho. O linho
deve, portanto, ter sido um produto importante no comércio da Era
Viking.

\SIG{Luciana de Campos}

Ver também Cotidiano; Cultura material; Mulheres; Tecelagem

\begin{itemize}
\item \versal{GUTAOP}, Else Marie. \emph{Medieval Manner or Dress}. Documents, images
and surviving examples of Old Northen Europe, Emphasizing Gotland in the
Baltic Sea. The Country Museum of Gotland, 2001.

\item \versal{JESCH}, Judith. \emph{Women in the Viking Age}. London: Boydell \& Brewer
Ltd, 1999.

\item \versal{SMITH}, Michèle Hayeur. Textiles, Wool and Hair. In: \versal{SVEINBJARNARDÓTTIR},
Guðrún (ed.). \emph{Reykholt}: the church excavations. Reykjavík: The
National Museum of Iceland, 2016, pp. 139-150.
\end{itemize}


\section{\versal{VIKING}}

Termo de origem e significado polêmico, discutido pela historiografia
contemporânea e empregado genericamente com dois sentidos:
\emph{étnico}, enquanto sinônimo para habitante da Escandinávia durante
a Era Viking; \emph{ocupacional}, se referindo a ações náuticas
efetuadas por alguns nórdicos. O debate sobre o uso do termo envolve
diversas perspectivas da Escandinavística, além do seu uso popular pela
mídia e arte.

\emph{Etimologia}: Existem diversas explicações para a origem do termo
viking, concentradas em torno de cinco hipóteses principais:

1. Pessoas da região de \emph{Viken} (\emph{Vík} em nórdico
antigo)\emph{,} no sudoeste da Noruega, ou então, significaria
simplesmente ``homens de Viken''. Segundo Eldar Heide, as fontes não
indicam objetivamente uma ligação com essa região em específico.

2. Pessoas que saíram da baía (\emph{vik}). Foi derivada do termo
feminino \emph{vík}, baía, enseada, referindo-se às pessoas que
embarcavam em baías.

3. Alguém que está afastado de sua casa seria relacionada com a palavra
\emph{víkja} (mover, caminhar, trilhar), que assume um significado de
viking como alguém que está afastado de sua casa. Nesta hipótese,
principalmente o masculino Víkingr é levado em conta.

4. Do inglês antigo \emph{wicing} (f.), pessoa que visitou o \emph{wic}.
Contração da palavra báltica \emph{wic} -- segundo Stefan Brink, uma
germanização da palavra latina \emph{vicus} (porto, local de comércio),
que é encontrado em locais como Ipswich, Norwich, Hamwich. Para Eldar
Heide, o termo foi originado do período merovíngio, enquanto para Otto
Gronvik teria sido tomado de empréstimo da palavra anglo-frísia
\emph{wítsing} (guerreiro acampado). Essa ideia vem de encontro ao fato
de muitas rotas eram conhecidas pelos guerreiros. Viking seria então
alguém que teria visitado esses \emph{vicii} ou \emph{wics} e
posteriormente foram denominados de \emph{wicingas}, \emph{víkingar}.
Vik como locais de comércio na Europa Setentrional visitados pelos
piratas, pois muitos eram conhecidos como \emph{wic} (centros de
comércio, como Hamwic na Inglaterra). Como houve um grande
desenvolvimento comercial a partir do século \versal{VIII}, inicialmente estes
locais eram visitados por comerciantes, que passaram depois a serem
piratas e depredarem estas mesmas regiões. A expressão sair à viking
(\emph{fara í Víking}) pode ter derivado dos locais onde os piratas
estavam localizados ou então onde estavam protegidos antes de atacar os
seus alvos. Isso é ainda mais relevante se percebermos que nos
manuscritos anglo-saxões e glossários anglo-latinos do século~\versal{X} (como no
poema \emph{Widsith}: \emph{siþþan hy forwræcon wicinga cynn}, desde que
repeliram seus parentes vikings), o termo \emph{wicing} está atrelado a
um sentido de pirataria ou saque (os próprios saxões eram conhecidos
como \emph{archipirata}) e foi utilizada até o século~\versal{XIII}. Mas
Christine Fell observa que essa atividade náutica não era exclusivamente
relacionada aos escandinavos, sendo a semântica da palavra associada de
modo mais genérico e deste modo, questiona a associação direta entre as
palavras \emph{viking} (inglês moderno), \emph{víkingr} (nórdico antigo)
e \emph{wicing} (inglês antigo). No final da Era Viking, temos os usos
de víkingr -- no sentido de saqueador ou pirata; e guerreiro do mar ou
assediador (víking).

5. Derivado da palavra \emph{vika} (feminino, sueco antigo, unidade de
distância náutica). Essa hipótese foi reforçada em 1944 por Fritz
Askeberg e posteriormente por Clas Brunius em 1982. A ideia básica é que
o sentido da palavra seria bem anterior à Era Viking, apesar dele ser
ainda utilizado neste período pelos escandinavos, num sentido de pessoas
que foram para outras regiões. Em 1983 o pesquisador Av Daggfeldt propôs
outra interpretação, mas ainda seguindo esta hipótese da relação com o
termo em nórdico antigo \emph{vikja} (turno). Para ele, a expressão
viking significava ``os remadores que trocam de turno'', também baseado
em runas encontradas em remos da Groenlândia, na qual mencionavam a
constante substituição devido ao cansaço da atividade. O sentido da
expressão, deste modo, seria bem mais antigo que a Era Viking e
associado estritamente com atividades náuticas durante o período das
migrações e não seria exclusivamente nórdico, mas germânico em geral. Em
2005 Eldar Heide aprofunda essa mesma hipótese, justificando que a
origem etimológica (século \versal{IV}) condiz com a nova tecnologia das velas e
remos entre os germanos antigos. Assim, associação da palavra viking
(homem do remo) com os povos escandinavos seria secundária e posterior.
Ela já seria usada, por exemplo, entre os antigos frísios (como em
\emph{witzing}, século \versal{V}).

\emph{Terminologias não-escandinavas para os nórdicos}: Fora da
Escandinávia, outros nomes para viking foram comumente utilizados, como
pagãos, nórdicos, danes, rus, estrangeiros. Em anglo-saxão, a palavra
\emph{wicing} é aplicada em algumas tribos germânicas antes da Era
Viking, mas durante os séculos \versal{IX} e \versal{X} ela passa a ser aplicada para os
viajantes escandinavos. No poema da batalha de Maldon (séc. \versal{XI}), a
palavra é utilizada significando marinheiro nórdico. Não é consenso que
o termo \emph{wicing} tenha relação direta com o termo viking. O termo
Viking não era comumente utilizado na Era Viking. Na França os
escandinavos eram conhecidos por \emph{Nordmanni} ou \emph{Dani} e na
mesma época, na Inglaterra eram chamados de pagãos ou Danes. Na Irlanda
eles eram denominados de pagãos e uma distinção era feita entre
noruegueses (conhecidos por \emph{Finngall}, estrangeiros brancos) e
Danes (\emph{Dubgall,} esntrangeiros negros). No Leste, os suecos eram
conhecidos por \emph{rus´} (sueco antigo: \emph{*roþs}, remadores) ou
\emph{varjag} (do nórdico antigo \emph{væringi}). Outro termo comum para
escandinavo era pagão (\emph{paganus, gentilies}, em latim;
\emph{majus}, em árabe). Foi na Inglaterra do século \versal{IX} (fora da
Escandinávia) que o termo viking foi mais comumente aplicado para os
nórdicos.

\emph{Viking nas inscrições rúnicas}: Segundo Stefan Brink, no nórdico
antigo, \emph{víkingr} é uma palavra masculina, normalmente traduzida
por guerreiro do mar (uma pessoa), enquanto \emph{víking} é feminina,
significando expedições militares no mar (a atividade). O termo
masculino, \emph{Víkingr}, é utilizado em nomes pessoais masculinos em
inscrições rúnicas da Escandinávia, ou mesmo associado a antropônimos
como \emph{Toki Vikingr}. A palavra é relacionada nestas inscrições a
homens que participaram de expedições com outras pessoas, uma jornada
coletiva. Certamente a maioria destes indivíduos faziam parte de
expedições militares, conduzidas por um grupo de guerreiros (\emph{lið})
sob o commando de um líder, chefe ou rei.

Isso é exemplicado na inscrição (Harlingstorp, Suécia, séc. \versal{XI}), que
narra como Toli atravessou o Oeste com vikings (\emph{varþ dauþr a
vestrvegum i vikingu}). A inscrição de Hablingbo (\versal{G} 370, Gotland), narra
que Helgi tinha partido para Oeste com vikings (\emph{með vikingum}).
Outro exemplo é fornecido pela inscrição de Bro (\versal{U} 617), onde uma mulher
patrocinou o monumento para o seu falecido marido: \emph{saR x uaR x
uikika x uaurþr x miþ x kaeti} (que foi um guarda de Gettir contra os
vikings). Para Judit Jesch, as pessoas da Era Viking sabiam da conexão
entre o substantivo víkingr e o nome pessoal Víkingr, este ultimo tendo
conotações positivas.

O termo feminino, víking, indica a atual expedição, a jornada, e somente
ocorre três vezes no \emph{corpus} epigráfico rúnico, sendo duas
dinamarquesas (\versal{D} 330 e \versal{D} 334) e uma sueca (Vg 61). Todas comemoram
homens que morreram partindo \emph{à viking} (\emph{i víkingu}). A
inscrição de Gardstanga (Suécia, \versal{D} 330) informa que um grupo fez parte
da expedição (\emph{váru víða óneisir í víkingu}, foram muito longe em
atividades vikings). Para Stefan Brink, o mais coerente seria considerar
que o viking (masculino), é quem estava fora em viking (feminino), não
tendo deixado a Escandinávia para uma jornada pacífica. O componente
semântico do sentido guerreiro seria o verdadeiro originador da palavra.

\emph{Viking na literatura nórdica medieval}: O termo viking começou a
ser usado de forma geral a partir do século~\versal{VII} na Inglaterra
anglo-saxônica até meados do ano 1300, quando desapareceu de forma geral
das línguas escandinavas e anglo-saxãs, com exceção do islandês -- que a
utilizou estritamente no sentido de pirata. Neste período, não tinham
conotações étnicas ou geográficas, sendo utilizadas nas fontes
anglo-saxãs e nórdicas como qualquer pessoa que realizasse atividades
como pirata, marinheiro ou depredador. Assim, nestas fontes, um viking
não era sinônimo de escandinavo em geral. O cronista Adão de Bremen
registrou em \emph{Gesta Hammaburgensis ecclesiae pontificum} \versal{IV}. 6
(1070 d.C.) a seguinte afirmação: ``\emph{Ipsi enim pyratae, quos illi
Wichingos appellant}'' (Eles mesmo se intitulam piratas, apelidados de
wichingos).

Na poesia escáldica no século~\versal{X} temos alguns casos da utilização da
palavra, empregados num sentido de incursores, mas usualmente ``eles, os
inimigos'': ingleses, bretões ou nórdicos. Não existem casos de usos
onomásticos para o termo Víkingr na poesia escáldica. O contexto
pejorativo em que a expressão é utilizada aparece em vários poemas,
geralmente para inimigos ou adversários de um rei. O poema
\emph{Liðsmannaflokkr}, ao descrever um ataque à Inglaterra, caracteriza
os vikings como inimigos (\emph{hríð víkingar kníðu}), assim como a
\emph{Eiríksdrápa} de Markús Skeggjason: \emph{Víking hepti konungr
fíkjum} (O rei parou os vikings com muita força).

Mas a partir do século~\versal{XI} a poesia escáldica começa a dar um sentido
positivo, de modo muito semelhante, como por exemplo na poesia de Egil
Skallagrimson. A \emph{Egils saga} narra que após o então menino Egil
ter matado outo garoto com um machado, sua mãe Bera afirmou que ele se
comportava como um verdadeiro viking (\emph{kvað Egil vera víkingsefni,
Lausavísur 3, Egils saga} 40) e quando ele fosse mais velho, seria
conveniente lhe dar um barco de Guerra, ou seja, uma ocupação altamente
valorizada pelo contexto poético. Foi preservado outro poema de Egil, no
qual ele afirma que em um navio partiria com vikings (\emph{fara á brott
með víkingum, Egils saga} 40). Na mesma saga, novamente se utiliza o
termo, desta vez na narrative prosaica para descrever as expedições
piratas de junto a Thorolf na Suécia, na qual se descreve a bravura dos
guerreiros também na poesia de Egil (\emph{gangr vas harðr af víkingum},
\emph{Egils saga} 48).

Outro exemplo de poesia escáldica é o poema \emph{Víkingarvísur},
escrito por Sigvatr Þórðarson. Apesar do manuscrito original não conter
este nome (que foi conferido no século \versal{XIX}), ele contém diversas
referências das atividades do jovem Olavo Haraldsson (\emph{leið vikinga
sceiða}) enquanto pirateava e batalhava nas costas da Inglaterra.

Na poesia éddica também encontramos referências. Na narrativa de
Brunhilde, quando ela viaja para Hel, é interrogada por uma giganta, que
afirma que ela manchou sua mão com sangue de homens (em suas atividades
enquanto valquíria), no que Bruhilde responde que teria participado de
expedições vikings (\emph{þótt ek værakí víkingu}, \emph{Helreið
Brynhildar} 3). No mesmo poema, o herói Sigurd é denominado de viking
danês (\emph{víkingr Dana}, \emph{Helreið Brynhildar} 12). Neste
sentido, a literatura nórdica medieval tanto utiliza o termo enquanto
sinônimo para pirata, como também acaba destacando aspectos marciais e
heroicos das atividades náuticas, empregadas conjuntamente com a mesma
palavra.

Em algumas sagas islandesas o termo também ocorre. Na \emph{Olafs saga
Tryggvasonar en mesta} (século \versal{XIII}), Snorri Sturluson utiliza três
variações do termo. Em primeiro lugar, ao descrever as ações de
pirataria de jovens reis, como Haraldo e Olavo (\emph{vikingum}, cap. 15;
\emph{vikingar}, cap. 16; \emph{i víking}, cap. 52). Em segundo, para
caracterizar ações náuticas para o Oeste (\emph{vestrviking}, cap. 2). E
por último, em referência aos famosos mercenários da região de Jórmsborg
(\emph{Iomsvikingum}, cap. 86), também citados na \emph{Jómsvíkinga
saga}.

\emph{Viking e etnicidade na Alta Idade Média}: A historiadora Clare
Downham percebe que a Escandinávia da Era Viking foi constituída por um
mosaico de etnias e de tensões locais, nem sempre unidos pelos reinos da
Dinamarca, Suécia ou Noruega. A arqueologia reforça a ideia de laços
locais muito fortes, concentrados em múltiplas camadas de comunidades,
famílias e regiões. Filiações nacionais certamente foram muito fracas.
Existem poucas fontes sobre como os nórdicos viam as atividades náuticas
fora da Escandinávia e, portanto, quais as suas percepções da terra
natal. As fontes geralmente são escritas por estrangeiros. Outra questão
é o intenso hibridismo cultural derivado do contato dos nórdicos com os
povos de outras regiões, obtido pelas migrações sucessivas e originando
novas identidades. Os vikings da Rússia e Ucrânia fundiram-se aos
elementos da sociedade eslava, báltica e oriental, enquanto que na
Normandia eles foram rapidamente inseridos na cultura dos francos. Em
Dublin a fusão das culturas gaélicas e escandinava foi testemunhada pela
onomástica, arte e religiosidade. A hibridização cultural é uma das
marcas do século~\versal{X}. Em outras áreas a identidade viking persistiu por
mais tempo, como a Islândia e as ilhas Faroé. A Era Viking teve um
impacto significante em muitas identidades locais da Europa medieval.
Ao mesmo tempo, porém, essa identidade viking foi ampla e multifacetada,
com muitas variações locais e algumas estruturas em comum, como a
linguagem. As pessoas que viviam na Escandinávia não tinham consciência
de nossas periodizações modernas e nem teriam considerado os vikings
como fatores cruciais em suas vidas, mas ao mesmo tempo a Era Viking
testemunhou a diáspora nórdica, bem como grandes transformações
culturais, econômicas e políticas por toda a Europa. O sucesso dos
vikings como fenômeno estava relacionado com suas habilidades para se
adaptarem e se modificarem de acordo com as circunstâncias locais.

Em recente tese de doutorado, a historiadora Katherine Cross demonstrou
que a identidade viking foi utilizada politicamente na Europa
Setentrional durante a Era Viking. Durante os séculos~\versal{IX},~\versal{X}~e~\versal{XI} as
ações dos incursionistas nórdicos alterou profundamente o panorama
social, político e cultural do mundo medieval. Essencialmente apoiada em
material genealógico, literário e histórico da Inglaterra e Normandia, a
historiadora demonstra como essas duas regiões diferiram suas percepções
sobre o patrimônio Viking e seu impacto nas relações étnicas do período.
A tese demonstra o desenvolvimento de uma única identidade viking na
Normandia (definida em contraste com os francos), ao contrário da área
inglesa, onde identidades vikings e escandinavas foram desenvolvidas em
diversos sentidos e implementadas em ocasiões diferenciadas. Mas em
qualquer situação, reivindicações desta identidade não eram uma
expressão de contato com a Escandinávia. Os textos normandos definem a
identidade viking dentro de um contexto franco. Limites genealógicos,
históricos e geográficos foram construídos entre a Normandia e a área
dos francos -- mas estas narrativas não visavam o resto do mundo
nórdico. Na Inglaterra, por sua vez, a herança viking e escandinava foi
utilizada na negociação de diversas formas de relacionamento. Às vezes
era utilizada para definir limites dentro da Inglaterra, mas também foi
usada para delimitar os habitantes do reino e as forças externas. O
contato contínuo entre a Inglaterra e a Dinamarca fez com que o
significado de viking e dinamarquês permanecesse ambíguo e dependente de
um contexto. A falta de um contato entre a França e a Escandinávia
permitiu que os normandos e seus vizinhos impusessem um significado
consistente para uma identidade viking.

\emph{Viking na arte e nacionalismo moderno}: Algumas das primeiras
traduções modernas da literatura nórdica medieval, como a
\emph{Heimskringla} em 1633 (na tradução ao dinamarquês de Peder
Claussøn Friis), ainda utilizavam a tradução de viking como pirata e não
como substituto para nórdico. Foi a partir do século \versal{XVIII} que os
acadêmicos resgataram o termo das fontes medievais numa tentativa de
criar um passado mais glorioso para seus países em processo de
nacionalização. Especialmente as sagas islandesas tornaram-se uma fonte
predileta para essa associação e como efeito direto disso, a palavra
viking é mencionada no \emph{Oxford English Dictionary} de 1807. Mas
foram com os poetas suecos Erik Gustav Geijer, com o poema
\emph{Vikingen}, e Esaias Tegnér, com sua nova versão de \emph{Friðþjófs
saga}, que o termo viking se tornou uma palavra utilizada quase que
exclusivamente como referência ao povo escandinavo.

Erik Geijer (1783-1847) foi um professor, historiador, músico e poeta
dedicado ao liberalismo e ao nacionalismo sueco. Foi o líder da
sociedade gótica, um grupo dedicado a promover performances do mundo
nórdico antigo, como a leitura das \emph{Eddas} e o consumo de hidromel,
além de editar a revista \emph{Iduna}, dedicado a estudos antiquários e
poesia antiga. Neste periódico, publicou em 1811 o poema
\emph{Vikingen}, talvez a mais influente obra artística para a
propagação do imaginário moderno sobre o tema dos nórdicos aventureiros.
Na narrativa, Geijer reconstitui a trajetória de um personagem com a
idade de quinze anos que inicialmente se sente deslocado de seu ambiente
doméstico e bucólico. O mar torna-se um atrativo e um dia ele abandona
sua mãe e embarca em uma expedição marítima. Com a espada herdada de seu
falecido pai, ele promote conquistar o seu país. Após lutar contra
fortalezas e palácios e beber muito hidromel, rapta uma donzela em
Valland. Tempos depois, após uma intensa vida de aventuras, ele acaba
morrendo no mar com vinte anos. Geijer acaba valorizando o comportamento
sangrento do personagem, inclusive ao raptar mulheres, mas o ponto
central do poema é a liberdade e a glória proporcionada pela aventura
marítima. Viking aqui torna-se tanto uma palavra associada a um guerreiro
que sai em expedições (\emph{kämpe}), quanto o rei do mar
(\emph{sjökonung}). Liberdade, independência e auto-governo na obra de
Geijer confundem-se com a própria ideia de nação que a Suécia vinha
construindo ao início do século \versal{XIX}. Esse referencial foi tema da
pintura \emph{Viking} (1845), do norueguês Frederik Nikolai Jensen, onde
um nórdico sequestra uma jovem mulher, esta com os ombros desnudos e com
olhar apavorado. Envolvendo ela com seu braço (que porta um machado), o
saqueador possui a autoridade da conquista e do saque, num misto de
autoridade masculina quanto de poder militar, algo antevisto na poesia
de Geijer.

Também influenciado por Erik Geijer, em 1825 o professor e bispo sueco
Esaias Tegnér publica a sua versão da \emph{Frithiofs saga}, que se
tornaria o grande épico nacional da Suécia do Oitocentos, traduzidos
para diversas línguas. Esta obra é uma narrativa de amor entre o herói,
Frithiof, e sua irmã adotiva, Ingeborg, cuja paixão é rejeitada pelos
irmãos. Especialmente em um trecho desta obra, \emph{Vikingabalk} (O
código Viking), as empreitadas náuticas recebem uma valorização heroica,
transformadas em uma série de normas, cujo âmago central são: coragem,
masculinidade, honra. Tegnér romantizou diversas passagens da
\emph{Edda} e radicalizou ainda mais a visão de um herói invencível, que
não se detem por nenhum obstáculo ou temor. Neste sentido, a covardia é
vista como um elemento passível de morte. Mas ao contrário de Geijer, o
viking de Tegnér é moldado pelo cristianismo, ou seja, é um personagem
``civilizado'', sem comportamentos desregrados ou depredadores. Mas
também os antigos nórdicos são associados positivamente ao contexto do
paganismo, ainda que repleto de estereótipos, como no poema \emph{A
Vikings hall} (1830), do norueguês Henrik Wergeland. Nele, os guerreitos
estão inseridos num contexto de grandes beberagens, junto aos seus
armamentos e sob a proteção de Odin e Thor.

A partir da década de 1830 o termo viking passa a ser constante em
inúmeros livros, obras artísticas, seminários e cartas, originando uma
moda romântica -- os nórdicos vistos como os antepassados ou pela figura
do ``outro'', revelando aspectos das noções de identidade por parte dos
intelectuais e artistas do século \versal{XIX}. Especialmente o período vitoriano
vai revelar muitas das suas noções de raça, nação, gênero e classe
aplicadas ao contexto medieval, muito mais do que a referenciais
históricos precisos. O termo também vai fazer sucesso do outro lado do
Atlântico. Em 1841, foi publicada a primeira menção norte-americana ao
termo: ``I was a Viking old'', no poema \emph{The skeleton in Armor},
por Henry Wadsworth Longfellow, professor de línguas estrangeiras no
Harvard College, em Massachussetts. Em síntese, o poema trata das
aventuras heroicas e românticas de um escandinavo, desde as terras
nórdicas até sua vinda para os Estados Unidos, onde constrói a torre de
Newport para sua amada esposa. O protagonista é valente guerreiro que
não tem medo de nada, que explora as escuras florestas em busca de
grandes animais selvagens, como o urso, uiva como o lobo e navega pelo
mar desconhecido. Nas novas terras, o viajante volta a explorar as
exuberantes matas. Mas essa imagem de heroísmo também tem uma
contrapartida. O viking imaginário, além de sua coragem, é um grande
beberrão, gosta de grandes festas com muita cerveja. O poema termina com
o característico brinde escandinavo: ``Skoal! To the Northland!
Skoal!''. Esta idealização do Viking como um intrépido aventureiro iria
marcar profundamente a sociedade norte-americana, especialmente nas
posteriores comemorações do dia de Leif Erickson.

No contexto da época, a palavra viking tornou-se tão usual que o
linguista norueguês Ivar Aasen a inseriu em seu dicionário \emph{Aasen~\versal{I} 
(1850)}, no sentido de habitante masculino da região de Vig, um
contexto totalmente distante de como o termo aparecia no medievo. Ela
também se torna frequentemente um referencial inserido no contexto
político do momento, como em 1861 quando o professor londrino George
Dasent proclamou: ``Os vikings eram como a Inglaterra no século \versal{XIX}.
Criaram fábricas e ferrovias antes de todo mundo, foram os melhores na
corrida da civilização e progresso. Não admira que eles sempre
ganhavam''.

Em 1899 a dupla de escritoras Edith Somerville e Violet Florence Martin
criaram o termo vikingismo, aplicado num contexto de comportamento
masculino extremado, sendo a partir desta época usual em inglês
coloquial. Mais recentemente, o acadêmico Andrew Wawn reutiliza o
conceito num sentido das percepções imaginárias do período vitoriano
sobre o passado escandinavo e elenca algumas das acepções que o viking
possuía para a arte oitocentista: bárbaros, aventureiros, mercenários,
piratas, soldados, colonos pioneiros, fazendeiros, democratas
primitivos, berserkir psicopatas, amantes ardorosos, entre outros.

Neste sentido, talvez nenhuma obra artística do século \versal{XIX} tenha sido
mais canônica do que a pintura \emph{A viking funeral}, de Frank Bernard
Dicksee, realizada em 1893. Nela, um grupo de guerreiros nórdicos
arremessa uma embarcação sobre o mar, com o corpo de um homem e seus
equipamentos, enquanto outro grupo observa a cena. Em primeiro plano, um
dos guerreiros porta a tocha que acendeu a pira funerária e com o outro
braço saúda o morto. A cena é tipicamente pré-rafaelita, com as cores
vermelhas e brilhantes sendo destacadas na pira e na tocha, enquanto o
resto do quadro é muito mais escuro. O quadro foi influenciado
diretamente pela descrição do funeral do deus Balder (\emph{Eddas}) e
passou a ser um referencial canônico para qualquer funeral de reis e
líderes da Era Viking, tanto no cinema quanto quadrinhos e literatura --
mesmo não existindo referências históricas que confirmem essa prática.
Outros elementos do quadro de Dicksee também são estereotipados, como os
escudos de metal, capacetes com chifres e guerreiros com o busto nu,
inspirados essencialmente pela ópera wagneriana. Para Joseph Kestner,
esta pintura foi um produto da ideologia do império britânico, onde os
vikings são vistos como antecipadores do domínio marítimo do mundo pelos
ingleses, do mesmo modo que as citadas ideias de George Dasent. Neste
caso, o poder barbárico (representando pela força masculina e bélica) é
visto positivamente num sentido de ter antecedido os feitos históricos
da nação britânica.

A segunda metade do Oitocentos também foi envolvida pelos usos políticos
do conceito. Na Finlândia, o periódico \emph{Vikingen} foi criado em
1870 por um grupo de intelectuais envolvidos na conservação do uso da
língua sueca neste país (falado por uma minoria da população total).
Neste caso, o nome da revista ressignificava o passado com as pretensões
nacionalistas do período. Um dos ativistas mais influenciados por este
grupo fino-sueco foi o escritor Arvid Mörne, que mesclou o fervor
nacionalista da língua com ideias socialistas. Em seus escritos sobre as
sagas dos reis, ele expressava ideias patrióticas com o idealismo
liberal e socialista.

O conceito Era Viking é adotado a partir dos anos 1870 na Suécia e no
final do Oitocentos começou a ser disseminado em várias línguas e
utilizado também por arqueólogos e historiadores. No século \versal{XX} o termo
viking passou a ser relacionado genericamente com a Era Viking -- um
período de intensa atividade nórdica fora da Escandinávia. Os
estereótipos, as imagens canônicas e todo o imaginário criado em torno
dos aventureiros nórdicos penetra a cultura popular contemporânea,
especialmente na Europa e Estados Unidos, especialmente associados à
referenciais da cultura material (como as embarcações e o elmo com
chifres).

\emph{O termo viking na academia contemporânea:} Alguns pesquisadores
atuais (como John Lind e Frederik Svanberg) defendem que o termo viking
deveria ser substituído por outros mais neutros, como escandinavos,
nórdicos ou pela a região específica que se pretende pesquisar, mas ao
mesmo tempo, percebem que na prática isso é inconcebível, devido ao
enraizamento do termo tradicional na cultura popular, na academia e na
sociedade em geral. A pesquisadora Gudrun Whitehead analisou o impacto
da imagem dos vikings e de seus estereótipos na formação de uma
identidade nacional islandesa e britânica, por meio dos museus que
possuem conteúdo relacionado e também concorda com um uso
contextualizado do mesmo.

Influenciado por Stefan Brink e a historiografia britânica, o arqueólogo
T. Douglas Price (\emph{Ancient Scandinavia}, 2015) continua a utilizar
o termo em equivalência com nórdico, especialmente em questões
relacionadas a expansão, colonização e comércio no início da Era Viking.
Em 2008, na obra \emph{The Cambridge History of Scandinavia}, a
dinamarquesa Elsa Roesdahl e o norueguês Preben Sørensen utilizam o
termo \emph{cultura viking} no sentido das tradições anteriores ao
cristianismo (p. 11) e como o conjunto das tradições culturais presentes
na Era Viking (p. 121). Diversos pesquisadores europeus (como Judith
Jesch, Katrina Burge, Shane McLeod, Bernard Mees, Clare Downham, entre
outros) vêm empregando o conceito de \emph{identidade viking}, num
sentido de diversas tradições políticas, culturais e linguísticas
desenvolvidas por comunidades nórdicas dentro e fora da Escandinávia
durante a Era Viking -- especialmente em confronto com outras etnias e
conectados a conceitos mais amplos (como diáspora e hibridismo
transcultural).

No Brasil, uma das utilizações mais antigas do termo foi com o livro
\emph{Vikings, os senhores do mar} (1970), escrito por Roberto Pereira
de Andrade. Neste manual de popularização científica, o autor questiona
o sentido étnico tradicional e já apontava algumas indicações do uso
ocupacional nas fontes medievais, mas sem maiores aprofundamentos. Foi
somente em 2001 com o artigo \emph{Fúria odínicia: a criação da imagem
oitocentista sobre os Vikings,} publicado no periódico acadêmico
\emph{Varia História} (\versal{UFMG}), que o Brasil tomou contato pela primeira
vez com uma historiografia sobre esse tema. No artigo, os autores
realizam um balanço das pesquisas arqueológicas (p. 215-216), a
etimologia e o conceito de viking nas crônicas e fontes literárias
medievais (p. 217-218), as representações artísticas do medievo (p.
218), as representações literárias, históricas e artísticas do século
\versal{XIX} (p. 219-227) e os modernos estereótipos na cultura midiática e
popular (p. 226-229). Em 2010, o tema voltou a ser debatido no artigo
\emph{Notas sobre o termo viking} (\emph{Alethéia} n. 2), realizando
algumas discussões sobre os usos ocupacionais e étnicos do conceito.

\SIG{Johnni Langer}

Ver também Era Viking; Escandinávia; Vikings e Alemanha moderna; Vikings
na literatura; Vikings na música; Vikings nas artes plásticas; Vikings
no cinema; Vikings nos quadrinhos; Vikings na televisão; Vikings no
Brasil.

\begin{itemize}
\item \versal{BRINK}, Stefan. Who were the Vikings? In: \versal{BRINK}, Stefan; \versal{PRICE}, Neil
(eds.). \emph{The Viking world}. London: Routledge, 2008, pp. 04-07.

\item \versal{CROSS}, Katherine Clare. \emph{Enemy and Ancestor}: Viking Identities and
Ethnic Boundaries in England and Normandy, c.950-c.1015. Tese de
Doutorado, University College London, 2014.

\item \versal{DAGGFELDT}, Av Bertil. Vikingen: roddaren. \emph{Fornvännen} 78, 1983,
pp. 92-94.

\item \versal{DOWNHAM}, Clare. Viking ethnicities: a historiographic overview.
\emph{History compass} 10 (1), 2002, pp. 01-12.

\item \versal{FELL}, Christine. Old english wicing: a question of semantics.
\emph{Proceedings of the British Academy}, 1986, pp. 295-316.

\item \versal{HEIDE}, Eldar. Viking -- rower shifting? An etymological contribution.
\emph{Arkiv för nordisk filologi} 120, 2005, pp 41-54.

\item \versal{HEIDE}, Eldar. Rus `eastern Viking' and the viking `rowershifting'
etymology. \emph{Arkiv för nordisk filolologi} 121, 2006, pp. 75-77.

\item \versal{JESCH}, Judith. \emph{Ships and men in the Late Viking Age}. London:
Boydel, 2001.

\item \versal{LANGER}, Johnni; \versal{SANTOS}, Sérgio. Fúria odínica: a criação da imagem
oitocentista sobre os Vikings. \emph{Varia Historia} n. 25, 2001, pp.
214-230.

\item \versal{LANGER}, Johnni. The origins of the imaginary Viking. \emph{Viking
Heritage} 4, 2002, pp. 06-09.

\item \versal{LIND}, John. ``Vikinger'', vikingetid og vikingeromantik. \emph{Kulm}
(Årbog for Jysk Arkæologisk Selskab) 61, 2012, pp. 151-168.
\end{itemize}
\section{\versal{VIKINGS E ALEMANHA MODERNA}}

O termo ``vikings'' possui conotações históricas passíveis de discussão
quanto à uniformidade de sua conceituação, a depender da visão
historiográfica, antropológica, sociológica e mesmo étnica a que se
filie o pesquisador que se debruce sobre o tema. Deslocando-se a partir
das regiões que hoje englobam a Dinamarca, Noruega e Suécia, como bem
assevera Johnni Langer (2015), o vocábulo pode, de forma resumida, ser
associado a todos aqueles que partiram da Escandinávia Medieval em
aventuras marítimas, e para os quais os objetivos contumazes consistiam em pilhagens,
saques e expedições beligerantes, cunhando-se, de certa maneira, um período da Alta Idade Média europeia denominada Era Viking. Período este que chegou ao auge de suas atividades entre os anos 800 e 1050, tendo como seu
\emph{terminus a quo} o ataque ao mosteiro de Lindisfarne, localizado a
nordeste da atual Inglaterra, em 08 de junho de 793, e como seu
\emph{terminus ad quem} o século \versal{XI} -- com seu ápice na derrota e morte de Haroldo Hardrada diante de Haroldo Godwinson na batalha de Stamford
Bridge, em 25 de setembro de 1066.

Historiadores do porte de Boyer (1986; 2001), Brøndsted (2004),
Graham-Campbell (2006) e Johnni Langer (2015), apenas para citar alguns,
cuja bibliografia é de mais fácil acesso aos pesquisadores e
interessados lusófonos, enfocam aspectos da cultura viking
através dos mais diversos prismas, como práticas de religiosidade,
cosmologia, estruturação social, organização política, aspectos
culturais, dentre outros. Todavia, a pluralidade de abordagens sobre os
vikings ressalta sempre o traço de belicosidade e ferocidade dos homens
do norte, como atestado pela frase, ``\emph{Libera nos, domine, a furore
normannorum}''. Debruçando-se apenas em suas características guerreiras
efetivas, Birro (2011) destaca o armamento, as táticas empregadas em
combate, a coesão entre as fileiras a partir de testemunhos literários e
documentação da cultura material a partir de análises historiográficas.
A eficiência dos \emph{raids}, já a partir dos deslocamentos por mar em
\emph{drakkars}, e completada com ataques rápidos e ferozes, criou e
propagou uma aura quase que de invencibilidade dos homens do norte. Nove
séculos depois haveria a retomada dos valores marciais vikings como
tradições inerentes ao mundo germânico continental, que tinha na
Alemanha nacional-socialista seu maior centro difusor.

Já no século \versal{XIX} estabeleceu-se um viés fortemente nacionalista na
historiografia europeia, que buscava legitimar as nações através de uma
remissão a um passado remoto. No caso do Império Alemão, formado em 1871
após a vitória conjunta da Prússia e seus aliados sobre a França, a
legitimação histórica deu-se também pelo ufanismo reinante naquele
momento. Na música, na literatura e nas artes em geral,
voltou-se o olhar para a Antiguidade em uma busca das raízes de uma
``germanicidade'' comum, uma \emph{Deutschtum.} Richard Wagner, ao final
da ópera \emph{Tristan und Isolde}, faz surgir em cena um grupo de
guerreiros com elmos córneos, isto é, com galhadas, o que
não condiz efetivamente com o que se possui de comprovações arqueológicas a respeito dos vikings. É certo, contudo, que a representação que mais os
associou ao imaginário nacionalista e pangermanista dos Oitocentos
aludia a aspectos guerreiros e ao seu físico. Langer (2001) aponta para
a imagem de bárbaros rudes, guerreiros intrépidos e ferozes, além de
possuírem estatura avantajada -- 1,70m a 1,80m e tom de pele geralmente
aloirado ou ruivo. Tais concepções gerais da época foram fundamentais
para o surgimento e consolidação de um estereótipo largamente manipulado
política e ideologicamente nos anos 30 do século seguinte.

Os anos que se seguiram à derrota do Império Alemão na Primeira Guerra
Mundial e à instauração da República de Weimar em 1919 foram
politicamente turbulentos na Alemanha, e já nos anos 20 o Partido
Nacional-socialista dos Trabalhadores Alemães (\versal{NSDAP}) afigurava-se como
força política emergente, a qual, porém, sofreu um golpe repentino com o
fracassado \emph{Putsch} de 1923. A ascensão do \versal{NSDAP} ao poder foi então
concebida através do viés democrático, e praticamente um ano após as
eleições presidenciais de 1932, Adolf Hitler foi convidado pelo então
presidente, Paul von Hindenburg, a assumir o cargo de chanceler do
\emph{Reich}. Com a chegada do \versal{NSDAP} ao poder, a política re-armamentista
e nacionalista do Partido estendeu-se ideologicamente sobre todos os
campos da cultura e vida alemãs da década de 30. Nesse sentido, a
retomada e consolidação de valores considerados como basilares para o
regime incluía um retorno ao legado guerreiro vitorioso do passado, que
se assentava nas vitoriosas migrações e conquistas dos antigos povos
bárbaros de ascendência germânica. Paralelamente a isso, a teoria racial
da pretensa superioridade ariana, desenvolvida especialmente por Alfred
Rosemberg (1930), difundiu-se no sistema escolar alemão, com o que a
supremacia nórdica também se fundamentava pelo tom de pele, cor dos
olhos e altura elevada. Tais elementos, juntos, encontravam-se na
principal força paramilitar de sustentação do regime, criadas em 1933,
os Esquadrões de Proteção ou, em sua sigla em alemão, \versal{SS}.

Com o início da Segunda Guerra Mundial (1939-1945) e a crescente
necessidade de mais soldados para os fronts, as \versal{SS}, já denominadas como
\emph{Waffen \versal{SS}} (\versal{SS} em Armas), começaram a intensificar uma propaganda
de cooptação e recrutamento de jovens oriundos de países ocupados pela
Alemanha, cuja aproximação ``racial'' e histórica serviria como elo de
união do conquistado e do conquistador. Com esse intuito, ecos positivos
reverberaram em várias regiões da Europa ocupada e principalmente da
Dinamarca, Flandres, Holanda e Noruega, com o alistamento de jovens
dispostos a lutar contra o bolchevismo. Criaram-se, no decurso do
conflito, as seguintes divisões das \emph{Waffen \versal{SS}} com preponderante
ascendência germânica: 6ª Divisão de Montanha \versal{SS} -- \emph{Nord}; 11ª
Divisão de Infantaria Motorizada de Voluntários das \versal{SS} --
\emph{Nordland}; 23ª Divisão Blindada de Voluntários das \versal{SS} --
\emph{Nederland}; 27ª Divisão de Infantaria de Voluntários das \versal{SS} --
\emph{Langemarck}; 28ª Divisão de Infantaria de Voluntários das \versal{SS} --
\emph{Wallonien}; 34ª Divisão de Infantaria de Voluntários das \versal{SS} --
\emph{Landstorm Nederland}. Contudo, a mais importante dentre essas
divisões de voluntários e a que mais representava o ideal
nacional-socialista de deslocamento do passado viking para o guerreiro
alemão do século \versal{XX} era a 5ª Divisão Blindada \versal{SS} -- \emph{Wiking},
criada em 1940.

Bragança Júnior (2014) discute o fato de que a absorção de voluntários
com o perfil que se coadunava com as teorias de uma ``raça ariana'' foi
facilitada pela propaganda visual em cartazes de recrutamento, que
demonstravam uma solidez e vigor físicos, tão inquebrantáveis quanto os
valores que defendiam. A associação entre virilidade
viking e o soldado das \emph{Waffen \versal{SS}} é a ponte que une a
ancestralidade nórdica à contemporaneidade, levando ao supersoldado da ideologia
nazista. O homem do Norte é chamado para participar de uma missão
histórica em defesa de sua pátria, tendo ao seu lado a imagem do ``seu''
antepassado viking.

Outro ponto de vinculação nazista entre o elemento nórdico
contemporâneo, a saber, o norueguês, e o viking histórico completa-se
com o motivo das embarcações desses, os \emph{drakkars}, que levavam os
guerreiros a realizarem seus temidos \emph{raids} na Europa a partir do
século \versal{VIII}, observado no cartaz de recrutamento para a \emph{Legion
Norske}, da Noruega.

Ressalte-se, em suma, a desapropriação indevida e falaciosa de um
passado histórico, em que a personagem histórica e do imaginário --
viking -- serviu aos propósitos totalitários do nacional-socialismo
durante os treze anos do \emph{Reich} de mil anos.

Álvaro Alfredo Bragança Júnior

Ver também Era Viking; Escandinávia; Vikings na literatura; Vikings na
música; Vikings nas artes plásticas; Vikings no cinema; Vikings nos
quadrinhos; Vikings na televisão; Vikings no Brasil.

\begin{itemize}
\item \versal{BIRRO}, Renan Marques. \emph{Uma história da Guerra viking}. Vitória:
 \versal{DLL}, \versal{UFES}, 2011.

\item \versal{BOYER}, Régis. \emph{Die Piraten des Nordens -- Leben und Sterben als
Wikinger}. 2. Auflage. Stuttgart: Klett-Cota, 2001.

\item \versal{BOYER}, Régis. \emph{Le mythe viking dans les letrres françaises}. Paris:
Editions Du Porte-Glaive, 1986.

\item \versal{BRAGANÇA JÚNIOR}, Álvaro Alfredo. O germano e os \emph{Ritter} a serviço
do nacionalsocialismo: propaganda e reapropriação política da imagem dos
germanos e dos cavaleiros medievais na Alemanha dos anos 40.
\emph{Brathair} 2 (14), 2014, pp. 79-96.

\item \versal{BRØNDSTED}, Johannes. \emph{Os Vikings}: História de uma fascinante
civilização\emph{.} São Paulo: Hemus, 2004.

\item \versal{GRAHAM-CAMPBELL}, James. \emph{Os vikings}. Barcelona: Folio, 2006.

\item \versal{KEEGAN}, John. \emph{Waffen-\versal{SS}}. Soldados da morte. Rio de Janeiro:
Renes, 1973.

\item \versal{LANGER}, Johnni. Vikings. In: \versal{LANGER}, Johnni (org.). \emph{Dicionário de
mitologia nórdica}. São Paulo: Hedra, 2015, pp. 546-549.

\item \versal{LANGER}, Johnni; \versal{SANTOS}, Sérgio Ferreira dos Fúria odínica: a criação da
imagem oitocentista sobre os vikings. \emph{Vária História} 25, 2001,
pp. 214-230.

\item \versal{ROSENBERG}, Alfred. \emph{Der Mythus des 20. Jahrhunderts.} Eine Wertung
der seelisch-geistigen Gestaltenkämpfe unserer Zeit. München:
Hoheneichen, 1930.
\end{itemize}
\section{\versal{VIKINGS NA ÁFRICA E MEDITERRÂNEO}}

A expansão viking, principalmente em seu auge no século \versal{IX},
levou incursões a chegarem à Península Ibérica e dali adentrarem o
Mediterrâneo e até mesmo a fazer contatos com cidades na África. Não se
sabe ao certo como os vikings tomaram conhecimento acerca do sul da
Europa, possivelmente em contato com os saxões e francos, eles passaram
a ter noção da extensão da Europa e da existência de um mar interior e
de outro continente ao sul da Europa.

A primeira incursão viking conhecida na história à Península Ibérica foi
datada de 844, mas o arqueólogo Neil Price assinala que algumas fontes
árabes apontam para possíveis incursões vikings desde o começo do século
\versal{IX}, a região do País Basco, hoje na fronteira entre Espanha e França.
Mas como esses relatos são ainda inconclusivos, toma-se a data de 844
para o primeiro ataque a península.

De acordo com a \emph{Chronicon Rotensis} de 883, no ano de 844, as
cidades de Gijón no Reino da Galiza e La Coruña no Reino das Astúrias
foram atacadas pelos vikings. Não obtendo o êxito esperado nesses dois
ataques, a frota contornou a península, indo atacar Lisboa, na época
território controlado pelo Emirado de Córdoba (756-929). Desde o começo
do século \versal{VIII} os árabes haviam chegado na Península Ibérica e
rapidamente, em menos de um século, já ocupavam a porção sul da
península. Lisboa foi saqueada por volta de 20 de agosto.

De lá a expedição continuou contornando a costa e adentrou o
Mediterrâneo pelo Estreito de Gibraltar. A partir de setembro, as cidades
de Cádiz, Medina, Sidonia e Algeciras foram saqueadas e incendiadas
pelos ataques vikings. Possivelmente, a cidade de Asilah no atual
Marrocos também foi atacada nessa época, consistindo no primeiro ataque
viking conhecido no continente africano. No mês de outubro, navegando
pelo rio Guadalquivir, a cidade de Sevilha foi saqueada e ocupada por
cerca de um mês. A partir de Sevilha, várias cidades e outras
localidades vizinhas foram alvos da pilhagem viking.

Após os ataques nos domínios árabes do Emirado de Córdoba não se sabe se
aquela expedição viking teria prosseguido adiante na exploração do
Mediterrâneo. Apesar de terem saqueado várias cidades nessa passagem do
ano de 844, os árabes se mostraram bastantes difíceis de serem
derrotados e extorquidos. Uma nova expedição viking à península Ibérica
somente ocorreria vários anos depois.

Em 859 teve início a segunda grande expedição à Espanha e ao
Mediterrâneo, tendo sido liderada por Hasten e Björn Flanco de Ferro,
contando com cerca de 62 navios. Nessa segunda expedição, os dois líderes
evitaram atacar as importantes cidades árabes, optando em assaltar
portos menores e mais vulneráveis na região de Algarves, no sul do atual
Portugal, e portos na costa marroquina. como Mazimma. De acordo com o
historiador al-Bakri, os \emph{majus}, termo árabe para se referir aos
vikings, teriam ocupado Mazimma por oito dias, tendo sequestrado duas
nobres e exigido pagamento de resgate. De lá a frota seguiu para
assaltar localizações nas Ilhas Baleares. Segundo as crônicas árabes da
época, os vikings tinham como objetivo chegar até Roma.

De fato, a expedição viking de Björn e Hasten foi avistada ao sul da
França. As cidades de Narbonne, Nmes, Aries e Valencia foram alvos de
ataque. Posteriormente a frota chegou à península Itálica. Em 860 a
cidade de Luca foi saqueada pelos vikings. De acordo com relatos
posteriores, eles acreditavam que Luca fosse Roma. Além dessa cidade,
Pisa e Fiesole também foram atacadas. Desse ponto em diante o destino da
frota viking é desconhecido. Aponta-se que continuaram a navegar pela
costa italiana e teriam se dirigido para o oriente.

Em 861 fontes árabes voltam a mencionar a passagem da frota viking, a
qual atacou seus territórios novamente. E em 862, a cidade de Pamplona
foi invadida pela frota de Björn e Hasten. Na ocasião como apontam os
cronistas árabes, o rei García de Pamplona foi feito refém, e
cobraram-se 70 mil peças de ouro para liberá-lo, assim como sua cidade.
Depois de tal acontecimento a frota viking que possuía por volta de 20
navios, retornou para a França, após três anos de viagem.

Outra grande expedição como essa não voltaria a ser vista. No século~\versal{X}
novos ataques ao Reino da Galiza ocorreram em 951, 965-966. Nesse ano
também se relatou uma batalha entre vikings e lusitanos ao norte de
Lisboa. Em 968 o chefe Gunnraudr assassinou Sisnando, bispo de Santiago
de Compostela. Na época a cidade já era um local de peregrinação cristã.
Novos ataques à península, principalmente a Galiza devido a sua
localização no norte, ocorreriam até o século~\versal{XI}, mas de forma
esporádica.

No entanto se desconhece até onde outros navios se aventuraram pelo
Mediterrâneo, pois os árabes intensificaram a vigilância do Estreito de
Gibraltar após a expedição de 859-862. Encontram-se menções de que
navios vikings foram avistados na Grécia e no Egito, mas não há provas
que corroborem tais avistamentos.

\SIG{Leandro Vilar Oliveira}

Ver também Era Viking; Viking; Vikings na Península Ibérica.

\begin{itemize}
\item \versal{HAYWOOD}, John. \emph{Historical Atlas of Vikings}. London: The Penguin
Books, 1995.

\item \versal{LOGAN}, F. Donald. \emph{The Vikings in History}. London/New York:
Routledge, 1991.

\item \versal{MIKKELSEN}, Egil. The Vikings and Islam. In: \versal{BRINK}, Stefan; \versal{PRICE}, Neil
(eds.). \emph{The Viking World}. London/New York: Routledge, 2008, pp.
543-549.

\item \versal{PRICE}, Neil. The vikings in Spain, North Africa and the Mediterranean.
In: \versal{BRINK}, Stefan; \versal{PRICE}, Neil (eds.). \emph{The Viking World}.
London/New York: Routledge, 2008, pp. 462-469.
\end{itemize}
\section{\versal{VIKINGS NA FRANÇA}}

A primeira incursão escandinava ao Reino Franco aconteceu em 799, na
região de Vendée. Até a década de 830, contudo, os reides vikings foram
esporádicos, concentrados especialmente na Frísia, Flandres e no
estuário do Sena. As principais regiões do Império Carolíngio atingidas
pelas invasões foram a bacia do Sena, a Aquitânia, a Bretanha, a
Nêustria e a área do Meuse, baixo Reno. Na perspectiva de Simon
Coupland, podemos dividir a expansão viking em três fases, tanto na
França quanto na Inglaterra: 1) 790s-840: reides raros e de pequeno
porte às regiões costeiras; 2) 841-875: aumento no número, escopo e
escala das incursões; 3) 876-911: estabelecimento no território ocupado
(colonização).

Neil Price, por sua vez, num estudo de caso sobre a Bretanha, separou as
atividades vikings em cinco etapas: 1) 799-856, primeiros reides; 2)
856-892, agressão à França; 3) 892-907, paz de Alan, o Grande; 4)
907-939, conquista e ocupação da Bretanha; 5) 939-1076, últimos vikings.
Para o autor, o ano de 856 é um marco fundamental, pois assinala o
início de ofensivas mais intensas à Frância ocidental. Com efeito, a
chamada Grande Invasão (856-862) e o cerco de Paris (885-886)
distinguiam-se das investidas anteriores em virtude de sua meticulosa
organização. Tais balizas cronológicas variam entre os historiadores,
visto que, evidentemente, são meras convenções didáticas -- na
realidade, nunca existiu um plano viking (consciente e coletivo) que
coordenasse as etapas, natureza e alcance da expansão na Europa. As
invasões e colonizações de partes da Frância eram feitas por grupos
independentes, que, muitas vezes, guerreavam entre si.

As explicações para as frequentes vitórias escandinavas sobre os
carolíngios já foram (e ainda são) muito debatidas pela historiografia.
Segundo Albert d'Haenens, as causas do sucesso viking foram a mobilidade
de suas tropas (tanto na terra quanto nos mares/rios) e as estratégias
militares, como o ataque surpresa. Janet L. Nelson, por sua vez, aponta
certos motivos, como a escolha do momento propício para a ofensiva (à
noite, p. ex.), a destreza naval e, talvez o mais importante, a
capacidade para construir boas fortificações. Numa visão recente,
Coupland afirma que as razões capitais foram as divisões políticas entre
os francos, bem como a tática dos vikings de erguer acampamentos em
locais de difícil acesso (como em ilhas) e evitar uma batalha aberta e
demorada para reagrupar, reorganizar e, depois, voltar a lutar -- sempre
em ataques rápidos.

Seja como for, esses triunfos vikings construíram ao longo dos séculos
uma imagem de ``catástrofe'' do mundo franco, em decorrência de
profundas e duradouras crises socioeconômicas e políticas que teriam
ocorrido. Devemos, no entanto, salientar de antemão que a ideia de um
``catastrofismo'' deriva sobretudo do exagero das fontes textuais
daquela época, escritas quase sempre por clérigos. De fato, elas
apresentam muitas vezes uma dicotomia religiosa entre ``pagãos''
(vikings) e ``cristãos'' (francos), num discurso que via os nórdicos
como ``ameaças apocalípticas'', o ``flagelo'' enviado por Deus para
punir os pecados dos carolíngios. As testemunhas oculares também
registravam exageros numéricos, muitos deles relacionados à quantidade
de inimigos -- em 885, por exemplo, o monge Abbon Cernuus (c. 850-923)
afirma que Paris foi atacada por ``mais de mil vezes quarenta homens'',
cifra que não faz sentido quando confrontada à demografia (franca e
viking) e às possibilidades limitadas de transporte e manutenção da
tropa em território hostil.

Em verdade, a partir da década de 1960, os historiadores e arqueólogos
iniciaram uma revisão historiográfica que matizou a visão
``catastrófica'' encontrada nas fontes cristãs e aquelas que eram
oriundas de uma interpretação errônea (literal, p. ex.) dos documentos.
É claro que os textos medievais não foram abandonados, mas eles passaram
a ser interpretados principalmente à luz das descobertas arqueológicas,
cada vez mais frequentes a partir da segunda metade do século \versal{XX}. Com
relação ao impacto dessas invasões no mundo carolíngio, observamos a
mesma reconsideração nos historiadores contemporâneos.

Pierre Bauduin, por exemplo, lançou uma tese que minimiza os resultados
das invasões vikings na civilização franca. Ele sustenta a ideia de uma
``acomodação'' dos invasores, que, obviamente, não estavam numa contínua
guerra com os carolíngios. Na realidade, em várias regiões a
consequência da chegada dos nórdicos foi muito menor do que se imagina.
As destruições e combates não foram tão arrasadores e frequentes; muitas
vezes, esses recém-chegados eram absorvidos e seus assentamentos
assimilados por concessões e negociações. Houve uma ``aproximação''
entre vikings e carolíngios, na qual variadas estratégias e compromissos
eram usados para atenuar os problemas que surgiam durante a integração e
coexistência. Para o historiador, o caso paradigmático de ``acomodação''
foi o estabelecimento dos escandinavos na Nêustria (séculos~\versal{IX-X}), cujo resultado seria a formação da Normandia. Já o caso clássico da
``integração'' concedida pelos carolíngios aos vikings seria o batismo
dos chefes nórdicos em solo franco.

Existe uma antiga teoria de que os vikings eram motivados por um
``paganismo militante'', que os fizeram conduzir uma guerra religiosa
contra as populações cristãs da Frância. Essa proposição foi retomada
por John Michael Wallace-Hadrill (1975), que a defendeu com seis
argumentos principais: 1) a alta frequência do uso do termo ``pagão'' em
referência aos nórdicos; 2) a destruição de igrejas e mosteiros; 3) o
ataque a altares, sacristias e relicários; 4) a tortura de monges e a
morte deles sem uma razão clara; 5) a prática de sacrifício ritual; 6) a
aparente conversão de certos francos ao paganismo. Já Lucien Musset
havia afirmado que o ``paganismo agressivo não tinha inspirado muito os
vikings'', e coube a Coupland rebater cada um desses pontos.

Para Coupland, 1) o termo ``pagão'' nem sempre é o mais citado nas
fontes -- na verdade, ele aparece como sinônimo de ``bárbaro'',
inclusive em referência aos muçulmanos e eslavos; 2) os edifícios
religiosos eram atacados por guardarem riquezas e serem pouco
protegidos; 3) a destruição de relíquias e outros itens era causada,
quase sempre, para a obtenção de ouro e prata que eles continham; 4) não
está claro que as torturas e mortes de cristãos foram causadas por uma
``motivação pagã'', pois os vikings preferiam fazer prisioneiros, que
poderiam escravizar ou vender (resgate); 5) não podemos generalizar, já
que provavelmente existe apenas uma evidência de sacrifício, que teria
ocorrido em 845 na região do Sena; 6) nos casos de conversão ao
paganismo, não há qualquer sinal de adoração aos deuses ou mesmo uma
obrigação para isso.

A presença dos vikings no mundo carolíngio chamou a atenção da Igreja,
que, juntamente com a Monarquia, passou a atuar na conversão desses
pagãos. Algumas vezes, o batismo era precedido pela troca de reféns; na
maioria dos casos, o ``padrinho'' (monarca) franco entregava presentes
ao chefe viking batizado. O dinamarquês Haroldo Klak foi o primeiro
soberano escandinavo a ser convertido ao cristianismo, o que aconteceu
em Mainz (826) por iniciativa do imperador Luís, o Piedoso. As
conversões em território franco continuaram nas décadas seguintes, como
o batismo de Weland (862) por Carlos, o Calvo, além daqueles de Godfrid
(882) e Hundeus (897), ambos por Carlos, o Simples. É claro que nem
todas as conversões tiveram êxito, como foi o caso do chefe viking
Rodulf, que, mesmo já sendo batizado, terminou ``sua vida de cão com uma
morte apropriada'' em 873, pelo menos é o que afirma uma fonte
carolíngia. De acordo com Stéphane Coviaux, a partir da segunda metade
do século~\versal{IX}, os governantes francos praticaram esses batismos, em
primeiro lugar, para conter as invasões vikings e proteger o reino -- o
sentido ``missionário'' era secundário.

Com a ``fundação'' da Normandia e sua progressiva cristianização, os
ataques vikings diminuíram. Houve, contudo, uma tentativa de conquista
da Bretanha, onde os nórdicos conseguiram estabelecer um principado em
Nantes (921). Esse domínio, porém, não durou muito tempo: em 939, o
chefe bretão Alan~\versal{II} conseguiu expulsá-los da região. O medo dos
escandinavos ainda assombraria o território francês por muitas décadas,
como quando a cidade bretã de Dol foi queimada pelos vikings em 1014.
Também no início do século~\versal{XI}, Emma de Ségur, esposa do visconde de
Limojes, foi raptada pelos nórdicos e libertada apenas mediante a
intervenção do duque normando.

\SIG{Guilherme Queiroz de Souza}

Ver também França na Era Viking; Normandia; Rollo.

\begin{itemize}
\item \versal{BAUDUIN}, Pierre. \emph{Le monde franc et les Vikings}: \versal{VIII}e-Xe siècle.
Paris: Albin Michel, 2009.

\item \versal{COUPLAND}, Simon. The Rod of God's Wrath or the People of God' Wrath? The
Carolingian Theology of the Viking Invasions. \emph{The Journal of
Ecclesiastical History}, vol. 42, 1991, pp. 535-554.

\item \versal{COUPLAND}, Simon. The Vikings in Francia and Anglo-Saxon England to 911.
In: \versal{MCKITTERICK}, Rosamond (ed.). \emph{The New Cambridge Medieval
History (c. 700-c. 900)}. Cambridge: Cambridge University Press, 2008,
vol. 2, pp. 190-201.

\item \versal{COUPLAND}, Simon. The Carolingian Army and the Struggle against the
Vikings. \emph{Viator}, vol. 35, 2004, pp. 49-70

\item \versal{COVIAUX}, Stéphane. Baptême et conversion des chefs scandinaves du \versal{IX}e au
\versal{XI}e siècle. In: \versal{BAUDUIN}, Pierre (dir.). \emph{Les fondations scandinaves
en Occident et les débuts du duché de Normandie}. Caen: Publications du
\versal{CRAHM}, 2005, pp. 67-80.

\versal{D'HAENENS}, Albert. \emph{As Invasões Normandas: Uma Catástrofe?}. São
Paulo: Perspectiva, 1997.

\item \versal{MUSSET}, Lucien. \emph{Las invasiones: el segundo asalto contra la
Europa cristiana, siglos \versal{VII-XI}}. Barcelona: Labor, 1982.

\item \versal{NELSON}, Janet L. The Frankish Empire. In: \versal{SAWYER}, Peter (ed.). \emph{The
Oxford Illustrated History of the Vikings}. Oxford-New York: Oxford
University Press, 1997, pp. 19-47.

\item \versal{NISSEN JAUBERT}, Anne. Some aspects of Viking research in France.
\emph{Acta Archaeologica}, vol. 71, 2000, pp. 159-169.

\item \versal{PRICE}, Neil. \emph{The Vikings in Brittany}. London: Viking Society for
Northern Research, University College London, 1989.
\end{itemize}
\section{\versal{VIKINGS NA ITÁLIA}}

A presença dos Vikings na Itália é um bastante difícil de ser encontrado
nas documentações. Encontramos poucos relatos referidos a alguns
momentos históricos que descrevem as formas como os Vikings estiveram na
Península Itálica. Isso aconteceu, seja enfrentando viagens navegando
pelo Oceano Atlântico e pelo Mar Mediterrâneo, ou provindo pelas vias
terrestres.

Assim, nos deparamos com narrativas que descrevem fatos supostamente
ocorridos no período entre 859 e 862 d.C., nos quais os vikings,
durante uma expedição marinha, depois de saquear várias cidades do
Portugal e da Espanha, chegaram na região sul da França onde se
estabeleceram por um tempo. Em seguida, tornando a navegar, saíram em
direção da Península Itálica, beirando as costas da Ligúria e da
Toscana, chegaram ao estuário do Rio Arno, entrando e navegando pelo
rio, alcançaram as cidades de Pisa e Fiesole, cidades que, naquele
período, eram abastadas e desenvolvidas. Por esta razão, conta-se que
foram depredadas pelos vikings em procura de fortuna.

Em outra narrativa encontramos relatos sobre uma outra cidade,
localizada na costa da Toscana, que foi saqueada pelos vikings. Trata-se
de Luni, uma antiga localidade etrusca que, mais tarde, tornou-se uma
famosa colônia romana. Referido a isso, Dudo de São Quentin, em um texto
do século~\versal{XI}, conta que, depois de terem saqueados as cidades de Pisa e Fiesole, voltando pelo mar e continuando nas suas navegações, os
vikings, comandados por Hasting, acreditaram ter alcançado a cidade de
Roma, que, devido a sua história, era meta almejada pelos guerreiros nórdicos.
Assim, com a pretensão de entrar sem grandes esforços na cidade,
aprontaram o seguinte estratagema: mensageiros foram enviados no local
para comunicar que o chefe viking estaria muito doente, e que, antes de
morrer, teria o desejo de ser batizado. Deste modo, aceitando a vontade
de Hasting, as autoridades locais prepararam uma solenidade em honra do
chefe estrangeiro. Porém, durante a cerimônia, o chefe simula de morrer,
e foi assim que em seguida a armadilha se realizou. Durante o velório
organizado pelos governantes, o finto cadáver sai do caixão e, com um
golpe de espada, mata o bispo que estava celebrando o ritual. Foi dessa
maneira que, no meio da confusão, os guerreiros vikings, que já estavam
preparados, invadiram e saquearam a cidade. Só posteriormente
perceberam que o lugar saqueado não era Roma, mas que se tratava da
cidade de Luni. Ainda assim, continuaram na navegação, desistindo de
procurar a antiga cidade italiana.

Outro relato de povos escandinavos que estiveram na Península Itálica, é
aquele que se refere aos varegues. Trata-se de navegadores nórdicos
conhecidos como comerciantes, piratas e mercenários, os quais, nas suas
navegações, utilizando-se do sistema de rios dos territórios da Rússia,
chegaram até Constantinopla, então capital do Império Bizantino. Assim,
pela reputação de serem também grandes guerreiros, no final do século~\versal{X},
foram engajados para fazer parte da guarda pessoal do imperador do
Império Bizantino. Por isso, ocorreu que no início do século~\versal{XI}, quando
parte da Itália (a atual Apúlia) ainda era dominada pelos bizantinos,
diversos soldados varegues foram agregados às tropas que suportavam o
catepano Basil Boioannesn, regente da cidade de Bari. Arturo Mariano
Iannace relata que, quando estava em curso uma luta entre bizantinos e
longobardos, Basil enviou um destacamento de soldados varegues em apoio
às tropas que estavam combatendo em Canas em 1018, conseguindo assim uma
valiosa vitória. Nos anos seguintes participaram de outras lutas que
ocorriam com frequência na área. Outras narrativas relatam que, no andar
do tempo, alguns deles se estabeleceram no lugar de forma definitiva,
deixando de serem soldados, formando núcleos familiares na Ápulia. Como
nota interessante, acrescentamos que Haroldo Hardrada, o futuro Haroldo~\versal{III}, 
rei da Noruega, foi uma figura muito importante da guarda varegue,
na qual permaneceu por quase dez anos antes de ser coroado. Histórias
contam da participação de Haroldo em batalhas contra os normandos no sul
da Itália aproximadamente em 1038.

Além desses primeiros relatos sobre a presença dos vikings na Itália,
encontramos novas informações que descrevem como pessoas que provinham
dos países nórdicos foram para esse país. Vários escritos escandinavos,
em particular de origens islandeses e noruegueses, reportam ``viagens
ao sul'' (\emph{sudrferdir, sudrgöngur}) que tinham um escopo comercial
ou diplomático ou, como no caso dos clérigos, referiam-se às visitas
\emph{ad limina Petri}, ou seja, aos encontros que aconteciam cada cinco
anos, nos quais os bispos se encontravam em Roma com o Papa. Além disso,
também os primeiros peregrinos escandinavos convertidos ao cristianismo,
começavam a viajar para encontrar o Papa, ou para visitar lugares
sagrados, às vezes transitando pela Península Itálica em direção de
Jerusalém.

Fabrizio D. Raschellà argumenta que um dos primeiros casos de
peregrinação escandinava em Roma foi aquele do escaldo islandês Sighvatr
Þórðarson, o qual, além de ser o poeta preferido pelo rei Óláfr
Haraldsson (Olavo \versal{II} da Noruega), era também um conselheiro de confiança
dele. Sighvatr, aproximadamente em 1027, foi para Roma em visita ao Papa
e, provavelmente em função da sua proximidade com Óláfr, a sua não foi
somente uma viagem de cunho religioso, mas também uma viagem de natureza
política. Mas, naquele período, Sighvatr não era o único representante
escandinavo presente em Roma: Canuto~\versal{II}, ou como era
mais conhecido, Canuto, o Grande, rei da Dinamarca, se encontrava na
cidade para o coroamento de imperador Conrado~\versal{II} do Sacro Império
Romano-Germânico. Nesse caso, a permanência do rei escandinavo em Roma
deu-lhe a oportunidade de desempenhar diferentes ofícios, além de
cumprir a formalidade de prestigiar o cerimonial da coroação de Conrado~\versal{II}, 
ou servir como reverência ao vigário de Pedro. Supomos que que foi
uma boa ocasião para entrelaçar relações com as outras autoridades que
se encontravam na cidade, e, mormente, para ser reconhecido e respeitado
como figura reinante frente aos outros soberanos. A confirmar isso, o
historiador medieval inglês do século~\versal{XII}, Guilherme de Malmesbury,
reproduziu uma carta em que o rei da Dinamarca escreveu aos notáveis da
Ânglia, na qual Canuto explicaria todas as particularidades da sua
estadia em Roma, e os resultados obtidos devido a sua argúcia política.
Destarte, lendo esse relato, percebemos quanto, para Canuto, foi
importante permanecer por um tempo em Roma, delonga que permitiu ao
rei manifestar e adquirir uma posição e uma força ainda maiores
do que aquelas que já tinha.

Concluindo, podemos deduzir que a presença dos escandinavos na Itália na
Era Viking se deu de múltiplas formas, cada uma com seu valor no
contexto dos acontecimentos.

\SIG{Lorenzo Sterza}

Ver também Era Viking; Viking; Vikings na África e Mediterrâneo.

\begin{itemize}
\item \versal{BERGAMO}, Nicola. \emph{L'esercito di Bisanzio in Italia (535-1071):
dalla riconquista giustinianea alla caduta di Bari}. Roma: Soldiershop, 
2017.

\item \versal{CIANCI}, Eleonora. Vichinghi, Variaghi e la ``Grande città''. In:
 \versal{FAZZINI}, Elisabetta (org.). \emph{Culture del Mediterraneo: Radici,
contatti, dinamiche}. Milão: Edizioni Universitarie di Lettere Economia
Diritto, pp. 45-61, 2014.

\item \versal{DI MAURO}, Nicola. \emph{Normanni: I predoni venuti dal Nord}.
Florença-Milão: Giunti, 2003.

\item \versal{IANNACE}, Arturo Mariano. \emph{Per una biografia del catepano Basilio
Boioannes: il contributo della storiografia e della trattatistica
militare}. 2016/2017. Tese (Curso de Graduação em Ciências Históricas.
Medievo, Idade moderna, Idade Contemporânea -- Faculdade de Letras e
Filosofia) -- Sapienza Università di Roma, 2017.

\item \versal{MALMESBURY}, Guglielmo di. Org. \versal{PIN}, Italo. \emph{Gesta Regum}. 
Pordenone: Studio Tesi, 1992.

\item \versal{MARTURANO}, Aldo. \emph{I Variaghi: Un'organizzazione di tipo mafioso
apparsa in Terra Russa nel primo Medioevo}. Disponível em:
\textless{}https://www.centrostudilaruna.it/variaghi.html\textgreater{}.
Acesso em: 12 jun. 2017.

\item \versal{RASCHELLÀ}, Fabrizio D. I pellegrinaggi degli scandinavi nel medioevo.
In: \versal{STOPANI}, Renato (org.). \emph{990-1990 millenario del viaggio di
Sigeric, Arcivescovo di Canterbury}. Centro Studi Romei, pp.
31-41, 1990.

\item \versal{SCAMPOLI}, Emiliano. \emph{Firenze, archeologia di una città}: (secoli \versal{I}
a.C.-\versal{XIII} d.C.). Florença: Firenze University Press, p. 169, 2010.

\item  \versal{TEATRO} Universale. De Pellegrinaggi. \emph{Teatro Universale: Raccolta
enciclopedica e scenografica}, vol. 5, p. 139, 1838.
\end{itemize}
\section{\versal{VIKINGS NA LITERATURA}}

As histórias, lendas e mitologias dos nórdicos têm sido uma fonte de
inspiração para um enorme número de obras literárias, tendo uma grande
influência nos escritores e poetas, especialmente da Escandinávia,
Alemanha e Inglaterra. Inúmeras obras literárias foram escritas ao longo
dos últimos dez séculos com temáticas relacionadas aos vikings.

Os antigos nórdicos foram grandes narradores e poetas. As sagas, mitos e
poesia dos vikings se equiparam aos grandes tesouros literários do
mundo. Nelas homens fortes e mulheres inteligentes se envolvem em uma
luta heroica contra uma natureza dura e inflexível e enfrentam os
eternos problemas humanos: amor e ódio, crime e castigo, viagens e
aventura. Essa literatura foi produzida principalmente na Islândia, ilhada
em meio ao Atlântico Norte e povoada principalmente por exilados
noruegueses, que encorajados pela tradição oral que mantiveram, produziram
assim uma vasta literatura, em verso e em prosa. Com diferença do que se
passou na Inglaterra ou Alemanha, a nova fé cristã não antagonizou
completamente os islandeses com a antiga religião pagã. A partir do
século \versal{XII}, certos eruditos e poetas islandeses, como Ari Thorgilsson e
Snorri Sturluson, dedicaram-se a recompilar e redigir os poemas,
histórias, mitos e lendas que desde épocas pagãs se transmitiam de
maneira oral. A principal obra de Snorri, a \emph{Edda Prosaica}, tem
sido uma das principais fontes sobre a mitologia nórdica ao longo dos
séculos. De igual maneira, foi principalmente durante o século \versal{XIII} que
a maioria das sagas, relatos geralmente épicos escritos em prosa, foram
redigidos.

Todo esse \emph{corpus} literário islandês teve uma imensa repercussão
não somente na literatura posterior, mas também nas artes plásticas e
nos estudos históricos. Essas fontes apresentavam uma sociedade heroica,
criando uma impressão atrativa e colorida dos vikings, fascinando
as gerações posteriores de escritores, artistas e historiadores, que
empregaram essas sagas, poemas e lendas como fontes de inspiração.

Devido a um crescente interesse pelo passado escandinavo, a partir dos
séculos~\versal{XVI} e~\versal{XVII} muitas destas obras medievais começaram a ser
empregadas como fontes primárias. Três livros tiveram um importante
papel nesse redescobrimento dos vikings: a \emph{Gesta Danorum}
(\emph{História danesa}) de Saxo Grammaticus (impressa pela primeira vez em
1514), a \emph{Historia de omnibus Gothorum Sueonunque regibus}
(\emph{História de todos os reis dos godos e suecos}) de Johannes Magnus
(1554), e a \emph{Historia de Gentibus Septentrionalibus} (\emph{Descrição dos
povos do Norte}) de Olaus Magnus (1555). Para o século~\versal{XVII} um grande
número dos manuscritos de onde as sagas e \emph{Eddas} estavam escritas
foram transportados para a Dinamarca e Suécia, o que facilitou seu
estudo e tradução. Entre os eruditos que empregaram essas fontes,
destaca-se o dinamarquês Ole Worm, que publicou um número considerável
de livros sobre runas e monumentos antigos da Dinamarca e viajou
extensivamente pela Europa, sendo uma figura importante na divulgação do
conhecimento sobre os vikings fora da Escandinávia. Também alguns destes
textos islandeses começaram a ser traduzidos para o latim. Na segunda metade
do século \versal{XVII} o dinamarquês Peder Hansen Resen traduziu pela primeira
vez a \emph{Edda Prosaica} de Snorri Sturluson ao latim, junto a vários
fragmentos da \emph{Edda Poética}, como o \emph{Hávamál} e a
\emph{Völuspá}. Nesta mesma época o antiquário dinamarquês Thomas
Bartholin, o Jovem, dedicou-se a colecionar e recompilar antigos
manuscritos islandeses, alguns dos quais traduziu ao latim e publicou.

Estas novas obras foram bastante conhecidas na Europa Ocidental e
ocasionaram um certo interesse nos escritores e poetas, principalmente
nos países com uma relação histórica e cultural com os vikings, a saber, os países nórdicos, mas também Inglaterra e Alemanha. Com o
surgimento do romantismo e o nacionalismo, muitas nações começaram a
voltar a atenção ao período pagão ou medieval tentando identificar um
passado nacional. Na Escandinávia, graças a todas as novas fontes e
obras sobre os antigos nórdicos, ocorre o descobrimento próprio de seu
passado heroico como nações independentes de antiguidade e
individualidade; enquanto isso, a Inglaterra intensificou o interesse na
literatura nórdica pois se acreditava que os textos islandeses
representavam as tradições pré-cristãs dos anglo-saxões, que não
foram registradas nas ilhas britânicas devido a sua rápida
cristianização.

Pela segunda metade do século \versal{XVIII}, começaram a produzir bastantes
textos literários com temática nórdica, os quais se pode dividir em
dois grupos. Um deles pretendia traduzir poemas dos manuscritos
escandinavos, por meio de traduções prévias do nórdico antigo ao latim.
Essas obras se descrevem melhor como ingênuas ``imitações'' ou
``adaptações'' livres, e não como intentos de tradução filologicamente
precisas. O segundo grupo consiste em obras de temática viking
majoritariamente originais.

O suíço Paul-Henri Mallet traduziu numerosos fragmentos da \emph{Edda}
de Snorri e outros textos islandeses ao francês e os publicou com o
título de \emph{Monumens de la mythologie et de la poesie des Celtes, et
particulièrement des anciens Scandinaves} em 1756. Apesar de mesclar os
então populares celtas com os igualmente misteriosos escandinavos, a
obra de Mallet teve um êxito notável e imediatamente foi traduzida a
vários outros idiomas e propagou as ideias do Norte pagão através dos
círculos letrados da Europa.

Como parte do romanticismo nacionalista na Suécia e Dinamarca, no teatro
começou-se a empregar temáticas do passado nórdico. Em 1773, Johannes
Ewald escreveu, na Dinamarca, sua extremamente popular obra de teatro
\emph{A morte de Balder}, baseada no trágico destino do filho de Odin. A peça obteve um grande êxito e foi representada em múltiplas ocasiões, causando forte influência em toda a Escandinávia e Europa do Norte.

Na Inglaterra vários escritores e poetas românticos escreveram obras com
temática viking. O poeta inglês Thomas Gray escreveu dois poemas com
temática viking: ``The Fatal Sisters'' e ``The Descent of Odin'', ambos
publicados em 1768. Varias décadas mais tarde, em 1817, o escritor
romântico Walter Scott, que tinha um grande interesse pela literatura
nórdica, publicou o poema ``Harold the Dauntless'', sobre a relação entre um violento nobre viking do norte da Inglaterra com
o cristianismo e sua
posterior conversão. Alguns outros escritores e poetas ingleses de
finais do século \versal{XVIII} e princípios do \versal{XIX} que escreveram textos com
temática viking são William Blake, William Wordsworth e Ann Radcliffe.

Durante o século \versal{XIX}, período apicilar do nacionalismo romântico, a
temática viking era muito popular na literatura da Escandinávia e do norte da Europa. O heroísmo pagão, a mitologia e o folclore, assim como o valor
e o desdém pela morte dos antigos nórdicos, resultaram em temáticas
incrivelmente atrativas para os artistas da época. Em 1825 o escritor
sueco Esaias Tegnér publicou o poema épico \emph{Frithiofs saga},
baseado diretamente nas sagas e outras fontes nórdicas. Foi um dos
poemas do nacionalismo romântico mais populares de princípios do século
\versal{XIX}, não somente na Escandinávia mas também no resto da Europa, já que
foi traduzido para vários idiomas. Sua importância foi tal que por
muitos anos foi considerada como uma referência para a interpretação da
Era Viking.

Na Rússia czarista do século \versal{XIX} a história e o folclore também
começaram a ter um especial atrativo. O papel que os vikings do Oriente,
os rus ou varegues, tiveram nas origens da Rússia começaram a ser mais
estudados e desde então estiveram rodeados de debates e controvérsias.
Isso ocasionou o surgimento de um interesse, principalmente literário,
pelos primeiros líderes escandinavos do antigo estado russo registrados
na \emph{Crônica de Nestor}, obra medieval onde se narra a história
antiga russa. O poeta Alexander Pushkin foi um dos influenciados pelas
histórias desses personagens semilendários, a quem celebra em sua obra
\emph{A canção do sábio Oleg}, de 1825.

Uma das obras oitocentistas com temática nórdica e germânica mais
importantes é sem dúvida o ciclo de quatro óperas do alemão Richard
Wagner, \emph{O anel dos Nibelungos}, elaborado entre 1848 a 1874. As
obras estão baseadas em personagens e histórias das sagas nórdicas assim
como da \emph{Gesta dos Nibelungos}. A trama se desenrola ao redor de um
anel mágico que concede o poder para dominar o mundo, sendo cobiçado
por deuses, gigantes e heróis. Na história, figuram deidades nórdicas
como Odin (sob o nome de Wotan), Freya e as três Nornas, assim como
heróis como Sigfrido. Estreando em 1876, o ciclo de óperas teve
bastante êxito e durante os anos seguintes foi apresentado em diversos
países, como Inglaterra e Itália.

Durante a segunda metade do século \versal{XIX} até as primeiras décadas do \versal{XX},
os vikings seguiram sendo um tema popular, emanando certa fascinação
no Reino Unido, França, Alemanha e Estados Unidos, para onde um
grande número de escandinavos havia migrado durante esse mesmo período
de tempo. Nesses anos foi publicado um grande número de livros populares
sobre histórias, lendas e mitologia dos nórdicos, a maioria deles
ilustrados e dirigidos a um público infantil. Nesse cenário destacam-se
autores como o estadunidense Hamilton Wright Mabie, que em 1882 publicou
o livro \emph{Norse Stories, Retold from the Eddas}.

Já no século~\versal{XX} o interesse pelo passado viking continuou na literatura
e nas artes. A novela histórica \emph{Röde Orm}, do sueco Frans G.
Bengtsson, foi publicada em dois momentos: em 1941 e 1945, e narra as
aventuras do viking Röde Orm (Serpente Roxa) e suas viagens pela Europa
do século~\versal{X}. A novela teve um incrível êxito, tem sido traduzida para
mais de vinte idiomas, passando a ser um dos livros mais lidos na Suécia
até a atualidade.

A história e a mitologia nórdicas foram uma importante fonte de
inspiração para a literatura fantástica, gênero que a partir do século~\versal{XX}
começou a crescer notoriamente em popularidade. Um autor de menção
indispensável nesse sentido é \versal{J.R.R.} Tolkien, acadêmico
de Oxford que em 1937 publicou sua famosa novela \emph{O Hobbit}, com o
qual apresentaria o fantástico mundo da Terra Média, no qual se
desenvolve a maioria de seus outros escritos, como a trilogia de \emph{O
Senhor dos Anéis} (1954-1955). Tolkien foi um grande amante e estudioso
das mitologias germânicas e nórdicas -- as quais tiveram uma enorme
influência em sua obra.

Nos últimos anos têm ocorrido publicações de um grande número de
novelas históricas e de fantasia com temática viking. Para mencionar
duas delas, \emph{American Gods} do britânico Neil Gaiman, publicada em
2001, onde se narra que em nosso mundo os deuses antigos vivem entre os
humanos, sendo os deuses nórdicos Odin e Loki personagens primordiais na
novela. Outro exemplo é a série de novelas históricas \emph{The Saxon
Chronicles} de Bernard Cornwell (2004-), que ocorre na Inglaterra dos
séculos \versal{IX} e \versal{X}, quando as Ilhas Britânicas foram invadidas pelos
daneses.

\SIG{Daniel Salinas Córdova}

Ver também Era Viking; Escandinávia; Vikings e Alemanha moderna; Vikings
na música; Vikings nas artes plásticas; Vikings no cinema; Vikings nos
quadrinhos; Vikings na televisão; Vikings no Brasil.

\begin{itemize}
\item \versal{BORGES}, Jorge Luis. \emph{Literaturas germánicas medievales}. Buenos
Aires: Alianza Editorial-Emece, 1979.

\item \versal{RIX}, Robert W. (ed.). Norse Romanticism: Themes in British
Literature, 1760-1830. \emph{Romantic Circles Electronic Edition}, 2012.
Disponível em: https://www.rc.umd.edu/editions/norse/index.html. Acesso em: 23 nov. 2017.

\item \versal{WAWN}, Andrew. \emph{The Vikings and the Victorians. Inventing the Old
North in 19th Century Britain}. Cambridge: D. S. Brewer, 2000.

\item \versal{WILSON}, David M. \emph{Vikings and Gods in European Art}. Aarhus:
Moesgård Museum, 1997.
\end{itemize}
\section{\versal{VIKINGS NA MÚSICA}}

O cenário sobre o mundo nórdico tem sido ampliado cada vez mais nos mais
diversos meios culturais da sociedade, e devido a uma massificação desse
universo por parte de uma indústria cultural, que passamos a ver com um
olhar mais atento a presença desta temática dentro da vasta gama que
compõe a música. Sem mergulhar em um ``ídolo das origens'', mas
compreendendo que essa presença viking na música não surge em um
universo contemporâneo, mas acompanha toda uma trajetória conjuntural
histórica, que varia com bases em recortes temporais e espaciais, este
verbete acaba por riscar uma superfície de um mar profundo, que é os
\versal{``Vikings na Música''}.

A primeira obra que devemos apontar é a tetralogia de Richard Wagner em seu
famoso ciclo de ópera chamado \emph{O Anel dos Nibelungos} (c. \emph{Der
Ring des Nibelungen}), composta com base em fontes
literárias do mundo nórdico, especialmente a \emph{Saga dos Volsungos} (c.
\emph{Völsungasaga}). Sem dúvida, a composição foi uma
virada nas formas de ser ver os nórdicos e suas fontes pois, através de uma visão ``operalizada'' (se nos permitem o neologismo), Wagner cria uma ótica particular que se torna representativa dos
``vikings'', como comenta Árni Björnsson. A tetralogia se divide nas
seguintes partes: \emph{Das Rheingold} (``O Ouro do Reno''), \emph{Die
Walküre} (``A Valquíria''), \emph{Siegfried} e \emph{Götterdämmerung} (``O
Crepúsculo dos Deuses''). Essa produção, feita entre 1848 e 1874, impactou de modo tão incisivo o imaginário sobre nórdicos que elementos próprios dela, como os elmos de chifres e com asas, apetrechos de
vestimentas, tranças e cortes de cabelo, entre tantos outros, tornaram-se marcas estereotípicas dos vikings em todo mundo.

A grande popularização dessa obra se deve não apenas ao seu triunfo como realização artística, mas sobretudo pelas condições históricas em que tomou forma. No século \versal{XIX}, ocorre um forte
momento de formação das identidades nacionais e seus nacionalismos, como
comenta Eric Hobsbawm, e isso acaba por propagar os ideais e referências
dessa fonte, popularizando ainda mais a sua representação específica
sobre os nórdicos. Além disso, Wagner foi amplamente recomendado como
um autor ufanista e de referência dentro do regime nazista, o que o
propaga ainda mais por toda a Europa no século \versal{XX}, juntamente com uma
associação de grandes discursos que acompanhavam as visões de
``realidade'' produzidas em seu ciclo de óperas.

Mas, como menciona Sam Dunn, umas das habilidades de Richard Wagner é a
forma musical da composição de sua ópera. Apesar de largas pesquisas e
referências, algo que muitas obras salientam, é sua ousadia musical e
visual que lhe colocam em um patamar diferenciado, possibilitando a
popularização de sua obra no imaginário. A sonoridade dele resgatava
elementos que ficaram distantes da música devido à conjuntura medieval,
que durante vários séculos condenou o uso do trítono -- que se
acreditava ter uma ligação com a invocação do Diabo, e por isso se
tornou conhecida como o ``Som do Diabo'', ``Nota do Diabo'',
\emph{Diabolo in Musica}, um intervalo entre alturas de
notas musicais, com a quarta aumentada ou a quinta diminuta em que se
insere o fá sustenido ou sol bemol. O uso desse intervalo é facilmente
ouvido nas composições do Black Sabbath, que vai usar desse
imaginário de ``nota'' demoníaca para compor mais traços para sua
imagem.

O uso desse intervalo, que se inicia com Ludwig van Beethoven
(mas claramente em sua famosa 5° sinfonia), traz sensação ímpar na
música, apresentando uma tensão composta de uma celeridade que transmite
sensação de movimento. Wagner amplia esse recurso em sua obra:
além de transmitir essa sensação, sua composição tem peso e um tratamento de grave inédito, devido ao uso de tubas, contrabaixos e octobaixos
(seguindo bases do \emph{Traité del'harmonie réduite à ses principes
naturels} de J. P. Rameau de 1722). Para se entender tal sonoridade,
recomenda-se ouvir \emph{Götterdämmerung}, famoso pelo prelúdio
``Cavalgada das Valquírias'' (\emph{Die Walküre}),
como comenta Richard Taruskin. Logo, sua ``rebeldia'' musical e suas
pesquisas na mitologia e literatura nórdica, somadas a uma realidade
conjuntural específica, trouxeram para essa obra um destaque enorme no
que tange à temática nórdica na música, tornando Richard Wagner um dos maiores compositores
desse imaginário.

A partir do final da década de 1960, recebendo grande
influência do \emph{blues} e do peso de composições como as de Wagner,
há um crescimento e popularização do \emph{rock} e suas mais variadas
vertentes, tornando-se um vetor de propagação da temática nórdica na música, como
comenta Iam Christe.

Partindo do \emph{rock} tradicional, que resgata lendas, mitos e figuras
dracônicas, como Dio, até referências mais diretas como do \emph{rock}
progressivo de Asgærd, em 1972, tem-se a formação de uma
produção cultural que busca trazer uma profusão histórica em suas
letras, assim como criar uma identidade visual que por vezes remete
a elementos medievais. Mas vale o destaque para a banda Led Zepellin,
no álbum \emph{Led Zepellin \versal{III}}, lançado em outubro de 1970. Nele, encontra-se uma canção intitulada ``Immigrant Song'', inspirada
em uma turnê na Islândia e composta pelo vocalista Robert Plank. A canção
retrata o descobrimento da América do Norte por Leifr Eiríksson,
assim como vários elementos da religião pagã. A música se tornou popular
entre os fãs, sendo chamada de \emph{Hammer of the Gods} (Martelo dos
Deuses), a ponto de virar chave para o título de uma biografia da banda
lançada em 1985: \emph{Hammer of the God: The Led Zeppelin Saga}, de
Stephen Davis. A banda Iron Maiden, um dos maiores ícones do
\emph{heavy metal}, também merece nota, pois seu primeiro EP (\emph{Extended Play}), \emph{The Soundhouse Tapes}, lançado em nove de novembro de 1979, tem como primeira música
``Invasion'', que retrata a belicosidade e
violência dos vikings, ainde que de uma forma estereotipada, em razão das
produções que a antecederam.

Mas, de fato, a visão de mundo que surge nesse período da década de 1970 e
1980, juntamente com uma rebeldia musical, é um traço de um movimento
composto de jovens de contracultura que se forma em várias partes do
mundo, como reflexo desse movimento do \emph{rock}. Já nos fins da década de
1980, quando o \emph{heavy metal}, estilo que se origina do \emph{rock}, se
consolida, tem-se ainda mais um crescimento dessa temática na
música, com destaque para um movimento de contracultura que se forma na
Noruega, usando o \emph{black metal} como ferramenta de expressão. Esse
estilo surge em meados de 1980, marcando-se por batidas
agressivas, vocais rasgados (guturais e viscerais), músicas aceleradas
(através do uso do \emph{tremolo picking} e do \emph{blast beat} de uma aplicação
de \emph{lo-fi} na sonoridade) e com um alto impacto sonoro.

Tal gênero teve fortes influências de estilos que lhe precediam, como o
\emph{heavy metal} e o \emph{punk}. Com um estilo cru e agressivo, que articula em suas letras
temas como satanismo, paganismo, anticristianismo, misantropia, revolta
ao sistema, liberdade social e cultural, o \emph{black metal} acaba por influenciar
fortemente toda uma juventude da década de 1980. Entrementes, a ``origem'' do
nome surge de um álbum da banda Venom de 1982, o álbum
\emph{Black Metal}, que hoje seria pertencente a um outro subgênero do
metal, o \emph{thrash metal}, embora seja inegável o ar de agressividade e
de anticristianismo -- mesmo sem uma força filosófica na sua concepção --
que a banda cria, dando assim base para a fundação de um novo estilo, o \emph{black metal}, que se modificará a partir de sua influência.

Deve-se compreender que a música como forma de expressão artística
circula dentro da sociedade e é interpretada pelos sujeitos sociais,
possuindo uma energia própria, a chamada energia social. Esta circula e
implica uma série de processos cognitivos e psicológicos, com base na
conjuntura dos receptores, formando assim novas interpretações e estilos
dentro da mesma dinâmica artística, podendo gerar fases de produção no
gênero.

O processo supracitado articulará elementos que serão o dínamo para uma segunda fase do estilo
do \emph{black metal}, quando, a partir de 1991, na Noruega um movimento forte de
contracultura e de um anticristianismo, exaltado por uma busca de ideais
pagãos e misantropos, ressignificará estética e artisticamente o estilo. Essa segunda fase é conhecida como a fase do \emph{black metal} norueguês, marcando-se por controvérsias e problemáticas, sobretudo com relação
aos conflitos e ataques diretos a instituições cristãs,
ocasionando concomitantemente publicidade e divulgação midiática do estilo e suas
concepções artísticas e estéticas. Tal momento ficará marcado pela
presença de bandas como Mayhem, Darkthrone, Burzum, Gorgoroth, Immortal, Emperor e muitas outras.
O cenário em torno do gênero crescia em bandas e fãs pelo mundo,
principalmente na Europa, mas é na Noruega, entre os anos de 1991-1994, que um
grupo/movimento chamado \emph{the black circle} ou \emph{black metal inner circle} se destacará.

Esse grupo, devido aos seus atos, acabará por ser visto como satânico, mas o que se tem
de fato é que a força inicial do movimento reside em um resgate da cultura
pagã e dos vikings, o que por vezes implicava em um ataque ao
cristianismo. Tem-se que os ataques a túmulos e igrejas tornaram ainda mais
midiáticas suas expressões e ideias, demonizando-os diversas vezes. Mas
as músicas produzidas aqui foram de grande força para uma
representatividade da temática nórdica na música. Ademais, esse elemento de resgate
ao pagão acabará por moldar novos estilos que saem dessa concepção
de rebeldia e contracultura, tais como o \emph{pagan metal}, o \emph{viking
metal} e o \emph{folk metal}, segundo comenta Aslhey Walsh.

Estes estilos vão ser grandes massificadores da temática nórdica e de
representações sobre os vikings, a ponto de um estilo musical inteiro se
formar em torno dessa concepção, originado a partir da banda sueca
Bathory (c. 1988) -- \emph{o viking metal} --, e chegando hoje a
uma popularidade mundial, com nomes famosos, como o de Amon Amarth.
Ainda nesse mesmo período, outro subgênero do \emph{heavy metal},
através de um resgate de mitologias, lendas, narrativas literárias,
fantasias medievais e outros elementos, acabou por dar ainda mais
destaque para os vikings, o \emph{power metal}.

Portanto, o rock e o metal serão os grandes propagadores da temática viking na
música a partir da segunda metade do século \versal{XX}, constituindo-se, quantitativamente,
como um dos cenários mais vastos de produção com essa temática. Não à toa, é uma tarefa hercúlea e beirante do impossível mapear quanto
desses gêneros foi produzido em torno de representações nórdicas, ainda
mais pelo constante crescimento em torno disto.

Os dois pontos aqui escolhidos representam, segundo nos consta, os principais
elementos, na música, que destacaram a temática viking, principalmente pela
quantidade de suas representações, embora também pela forma que esses
elementos se difundiram no imaginário popular e na indústria
cultural. Mas o que se tem de fato é que a temática viking está presente em múltiplos estilos
e em múltiplas temporalidades, seja no \emph{pop}, no \emph{folk}, nas
músicas eletrônicas (como o \emph{techno viking}, usado amplamente nas
mídias sociais) e em estilos que buscam resgatar uma sonoridade do
medievo em tempos contemporâneos. Riscou-se, assim, apenas a superfície dos pontos que trouxeram popularizações e remodelações da temática viking (em múltiplas concepções desse termo) na
música.

\SIG{José Lucas Cordeiro Fernandes}

Ver também Era Viking; Escandinávia; Vikings e Alemanha moderna; Vikings
na literatura; Vikings nas artes plásticas; Vikings no cinema; Vikings
nos quadrinhos; Vikings na televisão; Vikings no Brasil.

Björnsson, Árni. \emph{Wagner and the Volsungs: Icelandic Sources of
Der Ring des Nibelungen}. University College London: Viking
Society for Northern Research, 2003.

\begin{itemize}
\item \versal{CHRISTE}, Iam. \emph{Sound of the Beast: The Complete Headbanging
History of Heavy Metal}. Goodreads Author, 2004.

\item \versal{DUNN}, Sam; \versal{McFAYDEN}, Scot; \versal{FELDMAN}, Sam. \emph{Metal: A Headbanger's
Journey}. Documentário. \versal{EUA}: Warner Home Video, 2005.

\item \versal{FERNANDES}, José Lucas Cordeiro. A Sabedoria Perdida: uma análise da
imagética de \emph{Dauði Baldrs} do Burzum (1994-1997).
\emph{Notícias Asgardianas}, n. 11, 2016, pp. 95-105.

\item \versal{FRIDH}, Sanna. \emph{In Pursuit of the Vikings: an anthropological and
critical discourse analysis of imagined communities in Heavy Metal}. 
Master's Thesis in Global Studies. Gothenburg: University of Gothenburg
-- School of Global Studies, 2012.

\item  \versal{KAHN-HARRIS}, Keith. \emph{Extreme Metal: Music and Culture on the Edge}.
Oxford: Berg, 2007.

\item \versal{MARSHALL}, David (org.). \emph{Mass Market Medieval: essays on the
Middle Ages in Popular Culture}. North Carolina: McFarland, 2007.

\item  \versal{MOYNIHAN}, Michael; Søderlind, Didrik. \emph{Lords of Chaos: The Bloody
Rise of the Satanic Metal Underground}. Los Angeles: Feral House, 2003.

\item \versal{TARUSKIN}, Richard. \emph{Music in the Nineteenth century}. Oxford, New
York: Oxford University Press, 2005.

\item  \versal{VON HELDEN}, Imke. Scandinavian Metal Attack: The Power of Northern
Europe in Extreme Metal. In: \versal{HILL}, Rosemary; \versal{SPRACKLEN}, Karl (eds.).
\emph{Heavy Fundametalisms: Music, Metal and Politics}. E-Book, Oxford,
2009, pp. 33-42.

\item \versal{WALSH}, Aslhey. \emph{A great heathen fist from the North: Vikings,
Norse Mythology, and Medievalism in Nordic Extreme Metal Music}. 
Master's Thesis for Nordic Viking and Medieval Culture- \versal{ILN}. Trykk:
Reprosentralen, Universitetet i Oslo, 2013.

\item \versal{WESTON}, Donna; \versal{BENNET}, Andy (orgs.). \emph{Pop Pagans: Paganism and
Popular Music}. Durham: Acumen, 2013.
\end{itemize}
\section{\versal{VIKINGS NA PENÍNSULA IBÉRICA}}

Localizada na periferia do que era o mundo conhecido pelos escandinavos, a península Ibérica foi um
cenário secundário da Era Viking, conhecendo-se dezenas de investidas,
mas, regra geral, sem a intensidade e as consequências sentidas na
França ou nas Ilhas Britânicas. Pelo menos assim o indicam fontes
existentes, embora se deva notar que elas são apenas fragmentos de
informação que sobreviveram até aos nossos dias e que, na sua maioria,
consistem em narrativas ou alusões breves. A nossa visão dos
acontecimentos é por isso limitada e podemos dizer com segurança que
terão ocorrido episódios dos quais não se fez ou não sobreviveu qualquer
notícia.

Isso mesmo é sugerido pelas fontes existentes, dado que algumas ignoram
ataques que outras referem, dão um relato parcial que surge com maior
extensão noutros textos ou oferecem uma visão geral que esconde
acontecimentos por relatar. Por exemplo, as crônicas nada dizem sobre as
investidas que atingiram a cidade de Tui, mas dois documentos de
cartulários medievais mencionam incursões contra a povoação nos séculos
\versal{X} e \versal{XI}, inclusive uma que terá levado à morte ou captura do bispo e a
pilhagem da povoação. Outro caso são as três crônicas que mencionam a
grande incursão de 968-9, em que apenas a de Sampiro refere o ano de
pilhagem da Galiza após a batalha de Fornelos. E se atendermos às datas
oferecidas pelo \emph{Muqtabis} de Ibn Hayyan, notamos que os vikings
que estiveram treze dias na região de Lisboa, em agosto de 844,
demoraram cerca de um mês para chegar a Sevilha, numa viagem marítima que
podia ter sido feita em menos tempo, mas que por algum motivo foi
demorada. Se isso se deveu a mau tempo, batalhas navais ou
ataques costeiros é algo que não sabemos, precisamente porque as fontes
não dizem tudo e são apenas fragmentos de informação sobrevivente.

Esse limite de nosso conhecimento da Era Viking na península Ibérica
deve obrigar-nos a moderar a nossa análise e conclusões sobre o
sucedido. Em concreto, não podemos assumir sem mais que episódios
ocorridos em locais ou alturas próximas estão ligados ou foram
protagonizados pelo mesmo grupo de nórdicos, porque podem ter sido
levados a cabo por bandos distintos, mas sem que essa distinção seja
óbvia por força da nossa visão limitada do sucedido. Por exemplo, já se
propôs que os nove meses de saque que atingiram a região entre os rios
Douro e Ave em 1015-16 foram da responsabilidade dos mesmos vikings que,
em setembro de 1016, chegaram ao castelo de Vermoim, a poucos
quilômetros a norte do rio Ave. Na base desse raciocínio está a
proximidade geográfica e cronológica dos acontecimentos, mas, dado que
temos apenas fragmentos de informação e não um registro completo das
incursões nórdicas, o que sabemos não tem que estar necessariamente
ligado, mesmo que seja próximo. Neste caso, o episódio de 1015-16 pode
ter sido levado a cabo por um bando de vikings e o de 1016 por outro,
tal como, segundo os \emph{Anais de São Bertino}, em 859 havia um grupo
no rio Sena e outro um pouco mais a norte, no Somme.

À complexidade fragmentária das fontes junta-se a parcialidade
ideológica própria da época e ainda a diversidade de origens e formas de
preservação dos textos, os quais provêm do norte cristão da Ibéria
medieval, do sul muçulmano do território, da Europa latina além Pirenéus
e, é claro, da Escandinávia, seja em manuscritos próprios ou cópias, em
versões tardias ou próximas dos acontecimentos, com ou sem transmissão
oral pelo meio e com ou sem enfabulamento. A suposta expedição ibérica
de Olavo Haraldsson por alturas de 1013 é um caso exemplar dos problemas
que podem advir dessa complexidade, pois o relato que o coloca junto do
estreito de Gibraltar é produto da fusão de um poema do século \versal{XI} e
duzentos anos de transmissão (escrita ou oral?) num período conturbado
da História da Noruega. E desse modo não podemos assumir sem mais a
veracidade do que é dito no \emph{Heimskringla} ou encará-lo como um
relato coeso feito de uma só vez por um único autor, como se as sagas de
reis fossem uma espécie de diário de bordo.

Outro tema que merece um olhar crítico é o das motivações dos vikings
que chegaram à península Ibérica. Ainda hoje circula a ideia de que
foram atraídos pelo ouro ou esplendor de Santiago de Compostela, o que é
anacrônico, dado que o santuário jacobeu era um local de culto menor no
século \versal{IX}, quando se deram os primeiros ataques nórdicos em território
ibérico. Basta pensar que as grandes crônicas asturianas da época -- a
\emph{Albeldense} e a de \emph{Alfonso \versal{III}} -- datadas de 883 em diante,
nada dizem sobre Santiago de Compostela ou a suposta descoberta do
túmulo do apóstolo Tiago. E se assim é, se os textos nativos ignoram o
santuário, custa acreditar que os vikings que atingiram a Galiza em 844,
858 e 859 estivessem mais bem informados. Foi só a partir de meados do
século \versal{X} que o santuário jacobeu começou a adquirir fama internacional,
pelo que, antes disso, o mais provável é que os nórdicos tenham atacado
o território ibérico devido a um desejo genérico de fama e fortuna
rápidas, talvez sem noção do que iam encontrar ao navegarem para
ocidente dos Pireneus.

O próprio alcance geográfico de algumas das incursões contradiz a ideia
de um enfoque especial em Santiago de Compostela. Por exemplo, em 844,
após passarem ao largo das Astúrias e atacarem a região da Corunha,
seguiram até Lisboa e de lá para Sevilha. Também em 858 navegaram até a
foz do Guadalquivir e um ano depois, em 859, não só atingiram o extremo
sul da península Ibérica, como atravessaram o estreito de Gibraltar,
atacaram o que é hoje a região de Múrcia e chegaram ao sul da França,
onde passaram o inverno numa ilha no rio Ródano. A partir dessa base,
levaram a cabo ataques a povoações francas e italianas antes de pelo
menos parte do grupo regressar ao sul ibérico. Nada disto seria de
esperar se, conforme dizem alguns, os vikings tivessem como alvo o
putativo túmulo do apóstolo Tiago em Compostela.

Também no século~\versal{X} houve dispersão geográfica, embora menor, uma vez
que, de acordo com as fontes sobreviventes, os nórdicos não terão
passado do Algarve ou do sul de Espanha. Por exemplo, em 966 terá havido
uma batalha naval no rio Arade, que passa junto à povoação de Silves, ou
então perto da foz do mesmo, talvez ao largo do que é hoje a cidade de
Portimão. E na década de 970, um conjunto de pequenas notas palacianas
árabes dão notícia do avistamento de vikings ou pelo menos do alarme
causado pela sua aproximação na região da foz do Tejo ou mais a sul. A
mesma fonte contém ainda uma possível alusão a um ataque à cidade de
Santander, talvez em junho de 971, e uma prova de cooperação entre o
califado de Córdova e um aristocrata dos reinos cristãos do norte, por
ventura Gonçalo Moniz, conde de Coimbra. A ser essa a identidade do
nobre não é no entanto claro se a colaboração norte-sul se ficou a dever
apenas a um medo comum dos vikings ou também às especificidades da
política coimbrã, já que, por ser um território de fronteira, o condado
era propenso a laços com o sul muçulmano. Basta pensar que quando
Almançor entrou em Coimbra, em 987, os filhos de Gonçalo Moniz
aliaram-se ao caudilho islâmico.

O século~\versal{X} assistiu também a uma das maiores investidas vikings em
território ibérico, quando, em 968, uma frota descrita como tendo cem
navios desembarcou na costa oeste galega, pilhou os arredores de Iria
Flávia -- hoje Padrón e à época sede episcopal -- e derrotou em batalha
o bispo de Iria-Compostela, que morreu no combate. O que seguiu, segundo
a \emph{Crônica de Sampiro}, foi um ano de pilhagem da Galécia,
não se sabendo com precisão por onde andaram os vikings. Mas
uma série de documentos do século~\versal{X}, preservados em diversos cartulários
medievais, aludem a ataques e ao medo, embora não digam ao certo em que
anos, sendo possível que pelo menos alguns deles estejam a referir-se
aos acontecimentos de 968-9. E não deixa de ser significativo que essa
investida seja também a única cujo líder é conhecido por nome:
Gunderedo, talvez uma latinização de Gunnrauðr ou Gunrød. O que é
indicativo do que se terá passado nessa altura, já que um ano de
presença viking na Galiza terá produzido contatos diários com os
nórdicos, a ponto de o nome do seu líder ter sido retido e depois
vertido para o texto das grandes crônicas medievais. A incursão de
Gunderedo terminou em 969 com a morte do mesmo no seguimento de uma
derrota nórdica em local desconhecido, mas que pelo menos numa fase
final terá tido lugar numa zona costeira ou perto de um rio navegável,
dado que a \emph{Crônica de Sampiro} refere a destruição da frota
nórdica.

Para o século \versal{XI}, a imagem transmitida pelas fontes sobreviventes é uma
de maior concentração no nordeste ibérico, sem que se saiba se isso se
deve a um enfoque de fato ou apenas à quantidade e origem limitadas das
fontes de informação que sobreviveram até hoje. Mas são desse período
dois textos preciosos que nos dão um vislumbre mais cotidiano da
atividade viking no ocidente ibérico, já que se referem à captura e
posterior resgate de habitantes locais. Um desses documentos, datado de
1018, oferece inclusive um parâmetro cronológico para um grande ataque
que começou três anos antes, em julho de 1015, e só terminou na
primavera de 1016, período durante o qual os nórdicos pilharam a região
entre os rios Douro e Ave. O que quer dizer que houve certamente uma
base de inverno viking no que é hoje o norte de Portugal, embora a sua
exata localização não seja conhecida, dado que não é referida no texto e
não se conhecem, pelo menos até hoje, quaisquer vestígios arqueológicos.
Igualmente fascinante é um documento de 1026 que refere a libertação de
duas mulheres na região Santa Maria da Feira, episódio cuja data exata
não é conhecida, mas que tem a particularidade de ser um resgate não em
ouro ou prata, mas gêneros. Em concreto, uma pele de lobo, uma camisa,
três lenços, uma espada, uma vaca e uma porção de sal. Como é que se
chegou a essa lista de bens é algo que o texto não diz, mas pode ter
havido alguma forma de regateio, caso em que é legítimo perguntar se
terá sido verbal e, nesse caso, se terão havido intérpretes, por ventura
um ou mais nórdicos que soubessem latim ou uma variante da língua
latina.

Após a Era Viking, a presença violenta de marinheiros nórdicos na
península Ibérica prolongou-se por mais algumas décadas, embora sob a
forma de cruzados que, em viagem rumo à Palestina, navegavam ao longo da
costa ibérica e por vezes investiam contra povoações costeiras. É o caso
em particular do rei norueguês Sigurðr \emph{Jórsalafari}, que atacou
Sintra, Lisboa e provavelmente Alcácer do Sal por volta de 1109. E em
1147, descendentes dos nórdicos que colonizaram a Normandia no século~\versal{X}
participaram na conquista da cidade de Lisboa. Embora se chamassem
normandos e fossem mais precisamente anglo-normandos, pois eram já um
produto da tomada do trono inglês por Guilherme o Conquistador em 1066,
o termo não deve induzir-nos ao erro e levar-nos a pensar neles como
vikings. Podiam ser seus descendentes, mas os normandos da Normandia do
século \versal{XI} em diante eram indivíduos assimilados pela cultura francesa da
época e integrados no universo latino-cristão, retendo apenas alguns
vestígios das suas origens nórdicas. É o caso da toponímia, do
vocabulário e da construção naval, esta última patente na Tapeçaria de
Bayeux e a primeira presente ainda hoje na paisagem regional.

Não se dispõe do mesmo grau de vestígios da Era Viking na península
Ibérica, apesar da tradição popular ou análises menos cuidadosas que
elevam embarcações tradicionais galego-portuguesas ao estatuto de sinais
da presença nórdica no território. É o caso, por exemplo, do rabelo do
rio Douro, cujo casco trincado, calado baixo e forma esguia tornam-no
semelhante aos barcos dos vikings. Mas as parecenças não têm que ser
explicadas pela presença dos nórdicos no que é hoje o norte de Portugal
entre os séculos~\versal{IX} e \versal{XI}, até porque a Idade Viking ocorreu há cerca de
mil anos, não há dez ou cem, tendo havido múltiplas ocasiões de contato
e troca de influências com o norte da Europa sem que se tenha que
apontar logo para os vikings. Por exemplo, sabe-se que alguns dos
cruzados que participaram na conquista de Lisboa em 1147 ficaram em
Portugal, podendo ter introduzido na cultura portuguesa elementos de
origem nórdica que subsistiam na Normandia, entre eles a construção
naval. Ou então podemos apontar para as relações comerciais e
diplomáticas entre Portugal e os reinos de França e Inglaterra, dois
territórios colonizados por grandes números de escandinavos e onde
ficaram vestígios culturais da Era Viking, os quais podem depois ter
sido transmitidos aos portugueses por franceses e ingleses do século \versal{XII}
em diante.

Já para a toponímia, há pelo menos duas possibilidades de locais cujo
nome talvez esteja relacionado com os nórdicos. São eles a povoação
leonesa de Lordemanos e a aldeia portuguesa de Lordemão, na atual 
Coimbra. Ambas podem estar relacionadas com a palavra \emph{lordemani},
que é usada para identificar os vikings nas fontes ibéricas, mas se há
uma relação direta entre o vocábulo e os topônimos, o motivo é uma
incógnita. Pode ser uma memória de presença nórdica temporária nos
moldes de um acampamento ou base, embora não se conheça qualquer noticia
de um ataque a Coimbra ou no rio Mondego -- o que por sua vez pode ser
apenas o produto de fontes fragmentárias -- e para o exemplo leonês haja
apenas uma referência enigmática à chegada dos vikings aos ``campos'',
talvez os Campos Góticos em Leão. Mas também pode ser uma reminiscência
toponímica de uma colonização desses locais por vikings, voluntária ou
forçada, como por exemplo cativos nórdicos levados a trabalhar a terra
em zonas de fronteira durante a chamada Reconquista Cristã, quando a
insegurança obrigava muitas vezes a constituir ou fixar populações
recorrendo a criminosos, despojados ou servos.

De resto, o impacto de longo prazo da Era Viking na península Ibérica
parece ter sido pouco ou mesmo nenhum, seja porque a presença dos
marinheiros nórdicos foi de fato limitada à pirataria, sem processos de
conquista ou colonização em larga escala como os que ocorreram nas ilhas
britânicas, seja porque o conflito multissecular da 
Reconquista Cristã dissipou ou apagou quaisquer impactos do período
viking no território ibérico. Afinal, a chegada dos nórdicos à península
não introduziu a guerra numa região que estava em paz, não quebrou uma
unanimidade religiosa e também não levou à criação de novos reinos ou
entidades regionais à semelhança da Escócia ou da Normandia. Afetou a
vida regional, sem dúvida, e nesse sentido conhecem-se algumas
fortificações que foram erguidas para proteger as populações ou
comunidades religiosas. Mas o impacto não foi duradouro e mesmo a
possibilidade de os ataques vikings terem gerado um incremento naval no
al-Andalus deve ser encarada com limites, dado que esse efeito pode ter
sido apagado pelo tempo logo no século \versal{XIII}, quando a presença islâmica
na península Ibérica ficou reduzida ao pequeno reino de Granada.

\SIG{Hélio Pires}

Ver também Era Viking; Escandinávia; Viking.

\begin{itemize}
\item \versal{AZEVEDO}, Rui Pinto de. A expedição de Almançor a Santiago de Compostela
em 997 e dos piratas normandos à Galiza em 1015-16. \emph{Revista
Portuguesa de História}, 14, 1973, pp. 73-93.

\item \versal{PIRES}, Hélio. Nem Tui, nem Gibraltar: Oláfr Haraldsson e a Península
Ibérica. \emph{En la España Medieval}, 38, 2015, pp. 313-328.

\item \versal{PIRES}, Hélio. \emph{Os Vikings em Portugal e na Galiza}. Lisboa: Zéfiro,
2017.

\item \versal{PRICE}, Neil. The Vikings in Spain, North Africa and the Mediterranean.
In: \versal{BRINK}, Stefan; \versal{PRICE}, Neil (eds.). \emph{The Viking World}.
London/New York: Routledge, pp. 462-469.
\end{itemize}
\section{\versal{VIKINGS NA TELEVISÃO}}

A primeira aparição dos vikings em produções para \versal{TV} remete a 1959 em
\emph{Tales of the Vikings}, série de 39 episódios, sobre as façanhas de
um chefe viking e seus dois filhos. A série foi realizada pela Brynaprod
\versal{S.A.}, produtora de propriedade do ator Kirk Douglas, numa tentativa de
aproveitar-se do sucesso do filme \emph{The Vikings} (\emph{Vikings, os
Conquistadores}), lançado no ano anterior, protagonizada e também
produzida por Douglas. A série, no entanto, contava com um orçamento
consideravelmente mais baixo que o longa; enquanto o filme dispunha de
elenco estelar e gravações externas na própria Escandinávia, a série
tinha no elenco atores desconhecidos do público e era completamente
gravada em estúdio, além de ainda ser em preto e branco, pois a
transmissão em cores para \versal{TV} só se popularizaria na década seguinte. Por
outro lado, a representação dos vikings na série segue o mesmo padrão
que o filme trouxe: bravos guerreiros sedentos por batalhas, com a
exceção das cenas mais violentas e com teor sexual presentes no longa,
já que a censura para a televisão era muito mais rígida do que para o
cinema.

No mesmo ano, enquanto \emph{Tales of the Vikings} era produzido para a
\versal{TV} americana, na Inglaterra o canal \versal{BBC} produzia uma série animada
infantil intitulada \emph{Noggin the Nog}, sobre o jovem príncipe
Nogging, filho de Canuto, rei dos Nogs. A animação era bastante simples,
utilizando a técnica de \emph{stopmotion} conhecida como \emph{cut-out
animation}, que faz uso de recortes para criar o movimento dos objetos,
cenários e personagens, que curiosamente tiveram seu visual inspirado no
famoso Xadrez de Lewis, uma coleção de peças de xadrez que se acredita
serem originários da Escandinávia Medieval. \emph{Noggin the Nog} trazia
um enredo bastante fantasioso onde o povoado fictício de Nog era atacado
por dragões e outras feras mitológicas, tudo narrado como se fosse uma
antiga saga sobre os ``homens do norte''. A série teve 21 episódios até
1965 e na década de 1980 um \emph{remake} foi produzido, com a
colorização dos antigos episódios e com o acréscimo de seis episódios
inéditos, além de ter dado origem a uma longa série de livros infantis
vendida até hoje. Também nesta década, em 1966, o super-herói da Marvel
Comics que teve sua criação inspirada na mitologia nórdica ganha sua
primeira aparição na \versal{TV} com a série animada \emph{Mighty Thor}, contando com
apenas 13 episódios onde o deus do trovão loiro e com elmo alado
enfrenta ameaças de seu próprio panteão, como seu meio-irmão Loki, e de outras mitologias, a exemplo do semideus grego Hércules.

Na década de 1970 uma outra série animada para \versal{TV} é realizada, desta vez
uma coprodução entre Alemanha, Áustria e Japão chamada \emph{Vicky the
Viking}, baseada em um livro infantil de 1963 do autor sueco Runer
Jonsson, sobre um garoto viking que usa sua inteligência para ajudar seu
pai, o chefe da vila de Flake, e seus amigos em várias aventuras. A
série teve 78 episódios transmitidos entre 1974 e 1975 e apresenta
vários estereótipos como os famosos elmos com chifres e as embarcações
com vários escudos pendurados à mostra, muito provavelmente com uma
forte influência dos quadrinhos do viking \emph{Hägar, o Horrível}
criado pouco tempo antes pelo cartunista americano Dik Browne. Em 2009 e
2011 foram produzidos dois longas \emph{live action} adaptando a série e
mais recentemente, em 2013, foi feito um \emph{remake} em animação 3D da série
produzido por um estúdio francês de animação.

Em 1980, a série documental \emph{Vikings!} da \versal{BBC} de Londres estrearia
uma longa tradição de programas em formato de documentário para \versal{TV},
sejam episódios individuais ou séries completas, sobre a temática dos
vikings. A série era uma adaptação do livro do jornalista
islandês Magnus Magnusson, criador, apresentador e narrador do programa,
dividido em 10 episódios que apresentavam desde aspectos da cultura
material quanto eventos históricos importantes da Era Viking. A grande
particularidade das produções documentais sobre a temática é a forma
simplificada, e muitas vezes descuidada, com que o assunto é
tratado, frequentemente reforçando estereótipos, exageros e até mesmo
fornecendo informações equivocadas. Essa, infelizmente, não é uma
especificidade da temática viking, mas uma prática cada vez mais
difundida por canais ditos especializados, qual seja, a ``espetacularização'' do
conhecimento histórico e da História como um todo. No entanto, há raras
exceções com pelo menos um mínimo de coerência nas informações, como por
exemplo as recentes produções do canal \versal{BBC} \emph{The Viking Sagas} de
2011, a minissérie em três episódios \emph{Vikings} de 2012 e o especial
para \versal{TV} de 2014 focado na arte da Era Viking \emph{The Culture Show:
Viking Art}.

Em 1991, o personagem Príncipe Valente, criado para os quadrinhos por Hal
Foster em 1937, ganha uma série animada para \versal{TV} com duas temporadas,
somando 65 episódios, intitulada \emph{The Legend of Prince Valiant}. Nela, o protagonista, acompanhado por seus amigos Rei Arthur e os
cavaleiros da Távola Redonda, protegem Camelot dos malvados guerreiros
vikings. Aqui há forte teor de fantasia influenciado pela explosão de
filmes do subgênero conhecido como espada e feitiçaria, a
exemplo de \emph{Conan the Barbarian} (\emph{Conan, o Bárbaro, 1982}),
personagem que também ganharia uma série em animação em 1992 e outra em
\emph{live action} em 1997, e que assim como os filmes, quadrinhos e
livros dos quais se originam, apresentavam forte influência da cultura
viking e de seus estereótipos. Ainda na década de noventa, em 1997, uma
série animada chamada \emph{Loggerheads} é produzida na Alemanha em
parceria com a Irlanda. Com um humor bastante influenciado pelas séries
animadas da Warner Bros. famosas na época, como \emph{Animaniacs} e
\emph{Pink e Cérebro}, a série contou com apenas uma temporada de 16
episódios.

Em 2001, estreia no Reino Unido a minissérie de oito episódios
\emph{There's a Viking in my Bed} sobre um garoto do mundo atual que, de
repente, encontra debaixo de sua cama um guerreiro da Era Viking.
Apresentando um humor infantil extraído principalmente das situações em
que o protagonista enfrenta diante do problema de esconder um
brutamontes medieval das pessoas ao seu redor, a produção tem inspiração
na série de livros de mesmo nome escritos por Jeremy Strong. Seguindo a
mesma ideia de crianças modernas que, de alguma forma, encontram-se com
o universo da Era Viking, em 2005 outra série infantil é realizada:
\emph{Jul il Valhal} é uma produção para a televisão dinamarquesa sobre
jovens amigos que encontram um portal para Valhalla, a morada
dos deuses nórdicos. A série contou com apenas 24 episódios, mas teve
relativa boa recepção, ganhando em 2007 uma continuação em forma de filme
para \versal{TV}. Outra curiosa produção escandinava sobre a temática, desta vez
da Suécia, retratava a vida de um pequeno vilarejo da Era Viking através
do famoso formato de séries cômicas americanas conhecido como
\emph{sitcom}. \emph{Hem till Midgård} contou com duas temporadas de 12
episódios cada, transmitidos na \versal{TV} sueca entre 2003 e 2004, e trazia
piadas relacionadas ao mesmo tempo à Suécia contemporânea e à medieval,
com personagens vestidos de formas extravagantes e usando armas
exageradamente esdrúxulas.

Após o sucesso do longa de animação \emph{How to Train Your Dragon}
(\emph{Como Treinar Seu Dragão}) em 2010, a produtora Dreamworks
tratou de expandir o universo do jovem viking Soluço e seus amigos
domadores de dragões para a \versal{TV}, primeiro, ainda em 2010, com uma
minissérie em três episódios chamada \emph{How to Train Your Dragon
Legends}, depois em 2012 com \emph{Riders of Berk}, de 20 episódios, e
em 2014, juntamente com a continuação do filme, \emph{Defenders of
Berk}, também de 20 episódios. Em 2015 estreia a nova série,
\emph{Dragons: Race to the Edge} (\emph{Dragões: Corrida Até o Limite})
atualmente com a quinta temporada em produção. Todas as séries, assim
como os filmes que as originaram, mostram uma versão extremamente
fantasiosa da Era Viking, explorando intensamente e levando para a
realidade o mito dos dragões tão presente nas narrativas medievais
escandinavas. Além disso, é constante a presença de elmos com enormes
chifres e armas descomunais.

Em 2013, é lançada a mais pretensiosa produção para \versal{TV} envolvendo a
temática viking até hoje, criada por Michael Hirst para o canal
History, a série \emph{Vikings} contou com orçamento milionário e
filmagens na Irlanda, aproveitando-se do belo cenário do país,
propiciando maior verossimilhança visual. A série tem seu
protagonista inspirado no semilendário herói Ragnar Lothbrok e apresenta
sua ascensão de um simples camponês em um pequeno povoado até tornar-se
rei, mostrando suas incursões até a região hoje conhecida como
Grã-Bretanha. A série, apesar de subverter vários eventos da Era Viking
em prol de assegurar uma narrativa estimulante para manter o interesse
da ótima audiência do programa, traz uma representação razoavelmente
acertada em alguns pontos, como as vestimentas, arquitetura, armamentos
e cultura material no geral. Peca, por outro lado, em outros pontos como em
algumas expressões da religiosidade e da organização social. No entanto,
tendo em vista o histórico de produções superestereotipadas sobre o
tema tanto para \versal{TV} como para qualquer outro meio como cinema,
literatura, quadrinhos, games, entre outros, a série apresenta um
excelente avanço nas representações culturais e artísticas sobre os
vikings, considerando ser uma produção de tão amplo alcance. \emph{Vikings} tem
atualmente sua quinta temporada em produção, dando poucos sinais de
desgaste de sua audiência.

Por influência do sucesso da série \emph{Vikings}, a \versal{BBC} América produz
em 2015 uma produção adaptando a aclamada série literária de Bernard
Cornwell \emph{Crônicas Saxônicas}. A série \emph{The Last Kingdom} nos
apresenta a estória do saxão Uhtred, desde sua infância, quando é
capturado pelos vikings que atacaram o castelo de seu pai onde vivia,
até sua maturidade como guerreiro, quando precisa se dividir entre seu
passado e fé ligados aos vikings e a proteção das terras onde nasceu. A
narrativa faz uma combinação entre fatos, eventos e personagens
históricos reais, como o rei Alfredo, o Grande, e criações fictícias,
como o próprio protagonista. A série tem atualmente sua terceira
temporada em produção e uma boa recepção de público e crítica. Mais
recentemente, em 2016, uma série norueguesa de comédia intitulada
\emph{Vikingane} brinca com uma pequena vila na Era Viking ironizando o
modo de vida das pessoas daquela época. No mesmo ano uma minissérie de
12 episódios inspirada no poema épico \emph{Beowulf} é produzida no
Reino Unido, intitulada \emph{Beowulf: Return to the Shieldlands}, cuja
produção demonstra pouca ou nenhuma preocupação em evitar equívocos e
quebrar estereótipos em suas representações.

Desde a década de 1960, os vikings fazem também pequenas aparições, quase
sempre como vilões e ostentando vistosos capacetes chifrudos, em
episódios de séries de \versal{TV}, principalmente de animações, como em 1968 no
episódio \emph{Freeze's Frozen Vikings} do desenho animado \emph{The
Batman/Superman Hour}, em 1977 no episódio \emph{The Curse of the Viking
Lake} do \emph{Show do Scooby-Doo}, em um hilário número musical dos
\emph{Muppets} em 1980, num encontro entre duas mitologias na série
animada da Disney \emph{Hercules} em 1998 no episódio \emph{Hercules and
the Twilight of the Gods} e mais recentemente, em 2004, no episódio
\emph{A Kick in the Asgard} de \emph{As Terríveis Aventuras de Billy e
Mandy} ou em 2008 no episódio \emph{Dear Vikings} de \emph{Bob Esponja},
entre muitas outras não mencionados.

\SIG{Elvio Franklin Menezes Teles Filho}

Ver também Era Viking; Viking; Vikings na literatura; Vikings nas artes
plásticas; Vikings no cinema.

\begin{itemize}
\item \versal{BURKE}, Peter. \emph{Testemunha Ocular: História e Imagem}. Bauru (\versal{SP}):
Edusc, 2004.

\item \versal{HARTY}, Kevin J. (org.) \emph{The Viking on Film: essays on depictions
of the Nordic Middle Ages}. North Carolina: McFarland \& Company, 2011.

\item \versal{LANGER}, Johnni. Fúria odínica: a criação da imagem oitocentista sobre
os Vikings. \emph{Varia História}, Belo Horizonte, \versal{MG}, vol. 25, n. 25,
2001, pp. 214-230.

\item \versal{LANGER}, Johnni. Fé, exotismo e macabro: algumas considerações
sobre a religião nórdica antiga no cinema. \emph{Ciências da Religião},
Mackenzie Online, vol. 13, 2015, pp. 155-180.

\item \versal{NOTÍCIAS ASGARDIANAS} n. 10, 2015, dossiê Série Vikings.
\end{itemize}
\section{\versal{VIKINGS NAS ARTES PLÁSTICAS}}

Desde o final do século \versal{XVIII} e da primeira metade do \versal{XIX}, os vikings têm
sido temática recorrente nas artes plásticas, especificamente na
Escandinávia. Com o surgimento do romanticismo nacionalista, muitos
artistas europeus encontraram no passado antigo ou medieval uma
importante fonte de inspiração para suas obras, formando parte do
desenvolvimento das identidades nacionais na qual se buscava a origem
das nações em seus passados remotos. Vários artistas do norte da Europa
utilizaram o tema dos vikings como um modelo do passado glorioso nórdico
e germânico.

Após o fim da Era Viking, fora da Escandinávia praticamente todo o
conhecimento sobre os vikings se perdeu. Durante a Idade Média nos
países nórdicos, Islândia principalmente, muitas histórias, sagas e
mitos pagãos foram registrados por escrito por autores cristãos. Essas
histórias foram redescobertas nos séculos \versal{XVI} e \versal{XVII}, sendo muitas delas
publicadas, como a \emph{História danesa} de Saxo Grammaticus. Também
certos sábios publicaram estudos sobre o passado nórdico, como Johannes
Magnus e Olaus Magnus. Algumas dessas edições estavam ilustradas com
gravuras, cuja iconografia foi frequentemente reproduzida durante os
duzentos anos seguintes. Essas fontes mostraram uma sociedade heroica na
qual a guerra, a inteligência e valor se combinaram com uma grande série
de deuses e heróis projetando uma visão atrativa e misteriosa sobre os
vikings. Essa imagem fascinou as futuras gerações de escritores,
artistas e historiadores tanto da Escandinávia como de outros países
europeus.

A partir do século \versal{XVII} o conhecimento sobre os vikings e o passado
nórdico começou a se propagar fora da Escandinávia, principalmente
graças ao erudito dinamarquês Ole Worm, quem viajou extensivamente pela
Europa e publicou um número considerável de livros sobre runas e
monumentos antigos da Dinamarca. Na Europa Ocidental, principalmente na
Inglaterra, pouco a pouco surgiu um interesse pelos nórdicos, seu
idioma, escrita e literatura. Diversas sagas, \emph{Eddas} e outros
textos islandeses começaram a ser traduzidos ao latim e outros idiomas
por autores e acadêmicos escandinavos. Durante esse período as temáticas
nórdicas estiveram praticamente ausentes na arte de Europa Ocidental,
mas não tanto na Escandinávia. O rei dinamarquês Christian~\versal{IV}, que
reinou de 1588 a 1648, comissionou uma série de pinturas ilustrativas
sobre a história da Dinamarca a um número de artistas holandeses para
decorar o grande salão do castelo real de Kronborg. Algumas dessas
pinturas representavam pela primeira vez certas paisagens dos escritos
de historiadores dos séculos~\versal{XVI}~e~\versal{XVII} e das fontes medievais que esses
utilizaram.

Depois da série de Kronborg, artistas e patrocinadores, tanto da
Escandinávia como do resto da Europa, prestaram pouca atenção às
temáticas vikings por mais de um século. Isso mudou com a chegada do
romantismo, na segunda metade do século \versal{XVIII}. Aquele foi um período no
qual a interpretação da história nórdica floresceu. Foi a época do
movimento do \emph{Sturm und Drang} (Tormenta e ímpeto) na literatura e
na arte alemã, que, inspirado na primitiva tradição heroica
germânica, exaltava a natureza, os sentimentos e o individualismo
humano, contrapondo-se ao culto ilustrado do racionalismo. Foi um
período da arte europeia no qual houve uma confluência entre o desejo de validar a
mitologia e o pensamento do norte, porém também sem poder deixar
completamente para trás os moldes clássicos. Essa situação se
exemplifica muito claramente com a pintura mais famosa do suíço Johann
Heinrich Füssli \emph{Thor lutando contra a serpente de Midgard} de
1790, na qual a figura desnuda de Thor claramente foi inspirada por
Michelangelo e Giulio Romano, cujas obras Füssli havia estudado na
Itália, porém o tema segue claramente a mitologia nórdica que começava a
ser cada vez mais conhecida na Europa.

Na Escandinávia, com o auge do nacionalismo, se deu um florescente
interesse pelo passado como um tópos romântico. Na Dinamarca o escultor
e desenhista neoclássico Johannes Wiedewelt promoveu a postura de
representar os heróis e deuses nórdicos seguindo os moldes de figuras
clássicas: Odin como um Júpiter escandinavo, Thor como Vulcano ou Marte,
Freya como Vênus. O pintor escandinavo mais importante desse período, quanto ao interesse pela idade heroica da Escandinávia, foi o
dinamarquês Nicolai Abildgaard, que pintou várias cenas históricas e
mitológicas, como por exemplo \emph{Ymir amamentado pela vaca Audhumbla}
de 1777.

No início do século~\versal{XIX}, impulsionada pelas sociedades patrióticas na
Escandinávia, a mitologia nórdica se tornou ainda mais uma fonte para a
arte escandinava: na década de 1810, tanto em Estocolmo como em
Copenhague, foram até mesmo oferecidos cursos sobre o uso correto desse material mitológico nas
belas artes. Porém, os artistas ainda não lograram abandonar completamente
o classicismo e encontrar uma estética puramente nórdica. Por exemplo, o
escultor sueco Bengt Erland Fogelberg, entre 1828 e 1844, realizou três
grandes esculturas em mármore de Odin, Thor e Balder encomendadas pela
comissão do rei Karl \versal{XIV} Johan, as quais continuavam tendo grandes
similitudes com os modelos de Marte e Hércules que Fogelberg estudou em
Roma. Na Dinamarca sucedeu algo similar com o \emph{Friso Ragnarök} do
escultor H.E. Freund, no qual se representam cenas da batalha final que
terminará com o mundo seguindo um modelo clássico, no qual as armas,
vestimentas e muitas das poses dos personagens são claramente
greco-romanas.

Nessa época também teve início a sistematização dos restos
arqueológicos na Escandinávia. Os museus nacionais dos países
escandinavos foram reorganizados e se publicaram vários catálogos de
peças arqueológicas que resultaram em muitas influências nos pintores
históricos. Com estas novas fontes à disposição, os artistas podiam
basear suas obras em objetos e ornamentos reais e próprios da
Escandinávia em vez dos elementos da tradição clássica greco-romana.

Por outra parte, principalmente na Noruega, começou a se desenvolver uma
tradição pictórica diferente, que evocava o nacionalismo norueguês
através das paisagens românticas, com ênfase às árvores,
montanhas e águas norueguesas. Um bom exemplo disso é a pintura a óleo de Johannes
Flintoe \emph{Duelo em Skiringssalr} de 1835, que ilustra uma cena da
\emph{Saga de Egil Skallagrímson}. A obra de Flintoe abordou um aspecto selvagem na
iconografia dos vikings que até então era inédito na pintura.

O interesse pelo passado nórdico começou a crescer fora da Escandinávia,
embora em muitos países estava mesclado com o ciclo ossiânico da
mitologia irlandesa e a grande épica medieval alemã da
\emph{Nibelungenlied}. Influenciadas pelo teatro, as representações
associadas a essas histórias estavam repletas de elmos alados e harpas,
os quais em meados do século \versal{XIX} também começaram a aparecer em certas
obras de arte como um dos principais atributos dos heróis e guerreiros
vikings. Uma das primeiras pinturas de importância sobre o passado
viking da França é a obra romântica \emph{O conde Eudes defende Paris
contra os normandos}, de Jean-Victor Schnetz, terminada em 1837. Essa pintura representa a heroica defesa da capital francesa durante o sítio viking
de 885-886. Tal fato histórico, assim como o assentamento viking na
Normandia, inspirou um grande número de ilustrações românticas de
livros ao longo do século. Na Alemanha, o movimento romântico foi, como
em muitos outros países, uma expressão da busca da identidade nacional.
A atenção se centrou na \emph{Canção dos Nibelungos}, que compartilha
muitas personagens e heróis com a mitologia nórdica. Vários artistas
pintaram cenas ou ilustraram edições da \emph{Nibelungenlied}, sendo um
dos mais destacados Julius Schnorr von Carolsfeld, quem no período de
quarenta anos pintou os muros e tetos da Residência de Munique com cenas
do \emph{Cantar}.

Na Inglaterra, também houve certas pinturas que representavam o
passado da ilha antes da invasão normanda, entre elas algumas com
temáticas relacionadas aos ataques daneses contra os reinos saxões.
Um exemplo é a colorida obra \emph{Alfredo o rei saxão na tenda de
Guthrum o danes} de 1852, onde se representa uma das lendas mais
populares sobre Alfredo, o Grande. Na Rússia de finais do século \versal{XIX} e
princípios do \versal{XX} alguns artistas também mostraram certo interesse pelo
papel que os nórdicos orientais, os rus ou varegues, tiveram na formação
do estado russo. O mais destacado é sem dúvida Nicolai Roerich, cuja
obra \emph{Visitantes de ultramar}, de 1901, é a mais famosa nesse
aspecto. Nela se mostram dois barcos cheios de vikings navegando pela
costa russa.

No final do século \versal{XIX} houve um renascimento, na pintura, das temáticas
mitológicas e lendárias pela Europa. Isso se deve em parte à produção de
livros infantis acessíveis e amplamente ilustrados, através dos quais pela
primeira vez se divulgou informações sobre os ditos temas a um público muito
mais amplo e menos especializado. As ilustrações sobre o passado viking
da Dinamarca e a mitologia nórdica de Lorenz Frølich foram extremamente
populares, graças ao êxito da história ilustrada da Dinamarca de Adam
Fabricius, originalmente publicada em 1852 e reeditada constantemente
durante os seguintes cinquenta anos.

Durante a onda do crescente nacionalismo que culminou com a separação da
Noruega da Suécia em 1905, os artistas daneses voltaram uma vez mais a
ver o heroico passado viking como fonte de inspiração. Como parte desse
movimento, Peter Nicolai Arbo pintou uma de suas mais emblemáticas obras:
\emph{A caçada selvagem de Odin} em 1868 e \emph{A valquíria} em 1869.

As novas descobertas arqueológicas sobre o passado nórdico durante a
segunda metade do século \versal{XIX} influenciaram a maneira que os artistas
retratavam esse passado. O descobrimento do barco de Gokstad em 1880
mudou drasticamente a maneira que se representavam os navios vikings.
Agora os pintores não tinham que depender somente de sua imaginação e
dos exemplos da tradição clássica, podendo desenhar seguindo o
modelo de um barco viking real. O barco de Gokstad não somente estimulou
um interesse geral nos vikings como grandes navegantes, também
impulsionou o desenvolvimento de um estilo romântico nacionalista na
Noruega, o qual se expressou principalmente na arquitetura e nas artes
aplicadas. O barco e seus objetos, assim como as tradicionais igrejas de
madeira (\emph{stavkirke}), serviram como inspiração para desenhistas
daneses e, em menor medida, suecos. A joalheria, os móveis, telas e
porcelana, e a decoração de interiores foram alguns dos itens onde se empregou de modo amplo
motivos inspirados na arte viking, decorando-os
com complexos nós e representações de cabeças de bestas e dragões.

A obra com temática sobre o heroico passado escandinavo mais notável do
século \versal{XX} é sem dúvida a controversa \emph{Midvinterblot} (Sacrifício de
inverno) de Carl Larsson, uma pintura de grande formato comissionada
pelo Museu Nacional da Suécia. A cena, feita para decorar o
último muro livre da escadaria central do edifício, representa o
sacrifício pagão do rei Domaldi no templo de Uppsala para evitar um período de fome. Larlsson utilizou os textos de Adão de Bremen e Snorri Sturlusson
como inspiração. Desde que se expuseram os primeiros estudos a obra
recebeu muitas críticas, principalmente pela sua temática, sobre a qual se
dizia que tinha pouco a ver com a Suécia moderna, e também pela presença
de numerosos anacronismos nos objetos e armas presentes na obra. Apesar
das recriminações e de vários estudos e versões, em 1915 Larlsson
terminou sua monumental pintura (de 6.4 x 13.6 metros) e a considerou sua obra-prima. O debate continuou e fortes opiniões foram
expressas a favor e contra a obra por artistas, historiadores da arte,
políticos, arqueólogos e o público em geral. Finalmente, a pintura foi
rejeitada pelo conselho do Museu Nacional e iniciou uma grande, e por vezes desafortunada, peregrinação, que terminou em 1997, quando
enfim o Museu Nacional da Suécia comprou a obra e a instalou
permanentemente na área onde Carl Larsson havia planejado que fosse exposta.

A imagem que se tinha sobre os vikings durante as décadas de 1930 e 1940
esteve marcada pela visão nazista do passado nórdico. Os limites entre
a Escandinávia e a Alemanha, que desde finais do século \versal{XVIII} começaram a
ser mais evidentes, se articularam em uma suposta fraternidade fomentada pela
propaganda na década de trinta e imposta aos países ocupados nos anos
quarenta do século \versal{XX}. Os pôsteres e a propaganda emitidas pelos
partidos nacional-socialistas daneses e noruegueses exploraram
constantemente a imagem dos guerreiros nórdicos loiros, utilizando
elementos como runas, barcos e armas vikings. Um pôster promovendo a
Legião Norueguesa ao longo de 1942 mostra um soldado da Waffen-\versal{SS} e um
civil norueguês em um barco viking com uma cabeça de dragão ao fundo,
acompanhado da legenda ``Com a Waffen-\versal{SS} e a Legião da Noruega contra o
inimigo comum, contra o bolchevismo''. Artistas como o alemão Wilhelm
Petersen focaram na figura humana em um tipo de neoclassicismo com
qualidades realistísticas seguindo o ideal nórdico do nazismo. Petersen
estudou amplamente os restos arqueológicos da Escandinávia e suas
representação dos vikings eram as mais fiéis produzidas até então.

Pese a carga ideológica que o nazismo impôs aos vikings, depois da
Segunda Guerra Mundial, estes continuam sendo muito populares e sua
imagem tem sido renovada graças às histórias em quadrinhos (sendo o mais
famoso o \emph{Thor} da Marvel, desenhado por Jack Kirby), filmes
comerciais, séries de televisão e videogames. Nas últimas décadas,
pintores e escultores nórdicos como Asger Jorn, Arne Vinje Gunnerud,
Bror Marklund, Sigurd Vasegaard, Anker Eli Petersen e Bjørn Nørgaard têm
trabalhado temas relacionados com o passado viking. Sem dúvida alguma,
as histórias e mitologias dos antigos escandinavos seguem cativando até
nossos dias.

\SIG{Daniel Salinas Córdova}

Ver também Era Viking; Escandinávia; Vikings e Alemanha moderna; Vikings
na literatura; Vikings na música; Vikings no cinema; Vikings nos
quadrinhos; Vikings na televisão; Vikings no Brasil.

\begin{itemize}
\item \versal{CUELLAR ROMERO}, Sara. La mitología nórdica en el arte. \emph{Los
vikingos en la historia. \versal{II} Jornadas de Cultura Vikinga}. Granada:
Libros \versal{EPCCM}, 2015, pp. 387-450.

\item \versal{SALINAS CÓRDOVA}, Daniel. Nicolai Roerich y las representaciones de los
rus en su pintura. \emph{Notícias Asgardianas,} n. 11, 2016, pp. 22-33.
=======

\item \versal{WAWN}, Andrew. \emph{The Vikings and the Victorians. Inventing the Old
North in 19th Century Britain}. Cambridge: D. S. Brewer, 2000.

\item \versal{WILSON}, David M. \emph{Vikings and Gods in European Art}. Aarhus:
Moesgård Museum, 1997.
\end{itemize}
\section{\versal{VIKINGS NO BRASIL}}

Um dos grandes paradigmas da arqueologia do século \versal{XIX} foi identificar
traços do Velho Mundo nas Américas, especialmente os vinculados aos
fenícios e vikings. Durante a consolidação do segundo império
brasileiro, a partir dos anos 1830, a ideia de que nórdicos haviam
aportado no Brasil antes de Cabral torna-se uma das mais profícuas entre
os historiadores e arqueólogos do Rio de Janeiro.

Inicialmente essa tese era defendida por dinamarqueses sócios do \versal{IHGB},
como Carl Rafn, Peter Claussen e Peter Lund. Estes dois últimos
realizaram estudos de campo em Minas Gerais de 1833 a 1845. E também uma
equipe do \versal{IHGB} copiou as inscrições da Pedra da Gávea e confiou a sua
decifração ao tcheco Rochus Schüch, que trabalhava no gabinete de D.
Leopoldina (futuro Museu Nacional), que imediatamente as relacionou com
as runas. O mesmo erudito ainda acreditava que as línguas dos índios da
Amazônia eram semelhantes às nórdicas.

Logo outras pessoas buscaram mais supostos indícios dessa intrépida aventura
náutica em nosso passado. Em Santa Catarina, na ilha do Arvoredo,
aludia-se a inscrições semelhantes à Gávea. Imediatamente a tese viking
tomou ares internacionais. Em 1841 Pierre Lerebous publicou um livro no
qual a cidade perdida da Bahia, descrita no famoso manuscrito 512 da
Biblioteca Nacional, teria origem nórdica: a célebre estátua da praça
central seria inclusive uma alegoria do deus Thor.

Mas um fato pouco conhecido na atualidade é a teoria de que eles
estiveram no litoral catarinense. Tudo teve início com narrativas sobre
supostas inscrições que teriam sido encontradas na Ilha do Arvoredo, \versal{SC}
(ao noroeste de Florianópolis), durante o início do Oitocentos. Nessa
época circulavam estórias sobre supostos ``Letreiros'', como eram
conhecidas as manifestações visuais dos antigos indígenas (conhecidas em
nossos dias como petróglifos ou gravuras rupestres). Embebidos em ideias
eurocêntricas, tanto os moradores locais quanto os intelectuais da
região não acreditavam que essas esculturas geométricas teriam sido
realizadas pelos antigos habitantes da região, mas seriam vestígios de
povos "mais avançados" perdidos na bruma dos tempos -- no caso,
navegantes europeus antes de Colombo e Cabral.

Mergulhado nesse referencial, o viajante e artista Jean-Baptiste Debret
percorreu essa região e realizou um registro dos petróglifos indígenas
da Ilha do Arvoredo, posteriormente inserido em sua obra \emph{Viagem
pitoresca e histórica ao Brasil} (1834). Nela, percebemos claramente que
ele concede um referencial civilizatório aos vestígios, tomados como
``inscrições''. No início do Oitocentos, diversos estudos deram fama ao
referencial da epigrafia arqueológica -- os hieróglifos egípcios foram
traduzidos em 1822 por Champollion, lançando um modismo intelectual da
busca por antigas e misteriosas escritas perdidas pelo mundo todo. E
além disso, o caráter ``monumental'' era algo recorrentemente buscado,
tendo o painel da Ilha do Arvoredo todos esses elementos: era inóspito,
localizado no mar, afastado das grandes cidades da época.

Em 1839 os historiadores do \emph{Instituto Histórico e Geográfico
Brasileiro} (\versal{IHGB}) iniciaram seus estudos na famosa Pedra da Gávea, \versal{RJ},
que também supostamente conteria uma inscrição misteriosa. O
bibliotecário e mineralogista do gabinete imperial, Rochus Schuch,
enviou uma cópia das inscrições da Gávea, alegando que as mesas eram
``runas'', portanto, teriam sido esculpidas pelos navegantes nórdicos
durante a Idade Média. Schuch foi influenciado pelas publicações do
escandinavista Carl Rafn, que em seu livro \emph{Antiquitates
Americanae} (1837) afirmava que os vikings estiveram na América do Norte
(especialmente na região da Nova Inglaterra), tendo como base uma série
de inscrições em rochedos. Os arqueólogos modernos confirmam que também
se tratavam de gravuras esculpidas pelos indígenas locais, assim como os
da Ilha do Arvoredo, mas para os referenciais da época eram provas
concretas da passagem de navegadores europeus.

Os acadêmicos do \versal{IHGB} tomaram muito entusiasmo pelos escritos de Carl
Rafn, tanto que acabaram traduzindo alguns de seus artigos na
\emph{Revista do Instituto}. Também o paleontólogo e correspondente do
\versal{IHGB}, Peter Lund, de origem dinamarquesa e que estava pesquisando em
Minas Gerais durante essa época, acreditava que os nórdicos haviam
visitado o litoral brasileiro durante o medievo.

No final de 1839, o \versal{IHGB} recebe uma carta de Florianópolis, aludindo às
ditas inscrições da Ilha do Arvoredo, que poderiam ser de origem
escandinava, confirmando as hipóteses dos pesquisadores cariocas.
Imediatamente um sócio corresponde do Instituto, Falcão da Frota, é
enviado para pesquisar o dito ``letreiro'', o que acaba não acontecendo.

Com a vinda dos anos 1840, as pesquisas arqueológicas do \versal{IHGB}
concentram-se na busca da cidade perdida da Bahia (hoje sabemos que foi
uma localidade imaginária). E após a década de 1850, a hipótese viking
acaba sendo transferida para o espaço amazônico, uma região ainda mais
misteriosa e inacessível que nosso litoral. As ditas inscrições da Gávea
acabaram caindo no ostracismo intelectual após o final do império (a
geologia moderna confirma que são produtos de erosão) e os petróglifos
da Ilha do Arvoredo são hoje buscados pelos turistas e arqueólogos.
Quanto a sua ligação com os vikings, foi curta, mas instigante,
demonstrando que por diversas vezes a academia procurou criar uma origem
gloriosa para a nação brasileira, afastando-se do seu verdadeiro passado
material.

Com a aproximação dos anos 1850, a teoria viking deixa de atrair a
atenção dos intelectuais cariocas. Mas ela persistiu esparsamente, como
nos escritos do naturalista João Barboza Rodrigues. Ele procurava
encontrar os vestígios do que denomina de ``Os filhos de Odin'', a
herança da cultura nórdica pelos habitantes da Amazônia em sua cultura
material, língua, religião e costumes. Posteriormente a tese nórdica
acaba sendo questionada pela academia e encontra eco somente na arte,
seja nos quadrinhos, cinema ou então, na literatura de Arqueologia
fantástica e nos neodifusionistas, que insistem na veracidade dessa tese
pela internet até nossos dias.

\SIG{Johnni Langer}

Ver também Era Viking; Escandinávia; Vikings e Alemanha moderna; Vikings
na literatura; Vikings na música; Vikings nas artes plásticas; Vikings
no cinema; Vikings nos quadrinhos; Vikings na televisão.

\begin{itemize}
\item \versal{LANGER}, Johnni. Os vikings no Brasil. \emph{Habitus}, vol. 1, 2003, pp.
75-102.

\item \versal{LANGER}, Johnni. Os vikings no Brasil. \emph{Nossa História}, vol. 3,
2004, pp. 21-25.

\item \versal{LANGER}, Johnni. Vikings na selva. \emph{Revista de História da
Biblioteca Nacional}, vol. 80, 2012, pp. 80-84.

\item \versal{LANGER}, Johnni. Os vikings no Brasil: a história de um mito
arqueológico. In: \emph{Deuses, monstros, heróis}, Editora \versal{U}n\versal{B}, 2009,
pp. 149-168.
\end{itemize}
\section{\versal{VIKINGS NO CINEMA}}

As primeiras representações cinematográficas de vikings datam dos
primeiros anos dessa forma de arte, muito antes dos filmes falados e em
cores, e há pouquíssimas informações sobre elas. O primeiro filme de
que temos conhecimento sobre a temática chama-se \emph{The Viking's
Bride}, produzido no Reino Unido em 1907 e com poucos minutos de
duração, cuja narrativa exibe a estória de um viking que tem sua noiva raptada por um
grupo inimigo no dia de seu casamento e precisa resgatá-la. Pela
dificuldade de acesso, é difícil dizer como, visualmente, o filme
representa aspectos materiais da cultura e sociedade viking, mas o
enredo sugere alguns estereótipos, como o sequestro de mulheres. A esse
curta seguiram-se algumas produções norte americanas, também de curtas
metragens, todas dirigidas por J. Stuart Blackton: \emph{The Viking's
Daughter: The Story of the Ancient Norsemen} de 1908, também sobre um
sequestro, e no mesmo ano \emph{The Elf King: a Norwegian Fairy Tale},
que pelo título supomos tratar de temas míticos, e em 1910 \emph{The
Last of the Saxons}, baseado em um livro de 1847. Em 1914 são produzidos
\emph{The Viking Queen} e \emph{The Oath of a Viking}, dirigidos
respectivamente por Walter Edwin e J. S. Dawley.

Em 1924, o renomado diretor alemão Fritz Lang produz sua obra épica
\emph{Die Nibelungen} (\emph{Os Nibelungos)}, dividido em dois longas
intitulados \emph{A Morte de Siegried} e \emph{A Vingança de Kriemhild},
tendo como inspiração o ciclo de óperas \emph{O Anel dos Nibelungos} de
Richard Wagner, do final do século \versal{XIX} e no poema épico \emph{A Canção
dos Nibelungos}, do século \versal{XIII}. Lang dedica o filme, logo na
introdução, ao povo alemão, demonstrando o intuito em criar um efeito de
identidade tão caro à população alemã no período após a Primeira Grande
Guerra. O filme herda da ópera também alguns estereótipos, como elmos
contendo asas e chifres e uma divisão em cânticos. Também influenciado
pela ópera wagneriana e pela própria obra de Fritz Lang, estreia, em
1928 o primeiro grande filme americano envolvendo a temática do mito da
colonização nórdica nos Estados Unidos, \emph{The Viking} (\emph{Os
Deuses Vencidos}). Baseado no romance de 1902 \emph{The Thrall of Leif
the Lucky}, o filme apresenta ainda mais estereótipos e reforça o tom
fantasioso acerca da brutalidade e coragem dos vikings, além de
indumentária e armamentos espalhafatosos.

Apenas muitos anos depois, na década de 1950, os vikings voltam a
aparecer em produções cinematográficas. Em 1954 há a primeira adaptação
dos quadrinhos de Hal Foster \emph{Prince Valiant (O Príncipe Valente)}.
O filme copia a representação estereotipada que já existia nos
quadrinhos de Foster, com a presença de elmos cornudos e armas
exageradas, além de apresentar os vikings como vilões na narrativa. Em
1957 é lançado um filme bastante audacioso para a época, dirigido pelo
famoso Roger Corman \emph{The Saga of the Viking Women and Their Voyage
to the Waters of the Great Sea Serpent} (\emph{A Mulher Viking e a
Serpente Marinha}), fala sobre um grupo de mulheres que decidem sair ao
resgate dos homens do vilarejo após estes serem raptados por inimigos.
Aqui ainda há presença dos estereótipos, sendo o mais claro a exploração
da sensualidade através dos trajes das personagens, mas é interessante
notar a inversão dos papéis de gênero em relação à temática do rapto nas
primeiras produções.

No ano seguinte, 1958, a maior e mais ambiciosa produção cinematográfica
sobre o tema até o momento é produzida, \emph{The Vikings}
(\emph{Vikings, os Conquistadores}), dirigido por Richard Fleischer e
elenco com grandes astros da época como Tony Curtis, Janet Leigh e Kirk
Douglas. Baseado no livro \emph{The Viking} de 1951, escrito por Edison
Marshall, conta a estória de um grande guerreiro e um escravo que se
apaixonam por uma princesa raptada sem saberem que são meio irmãos. Ao
contrário de \emph{Prince Valiant}, o filme representa os vikings como
heróis e guerreiros valorosos e, por ter recebido consultoria de
arqueólogos e historiadores durante a produção, traz bastante fidelidade
à cultura material e equipamentos, além de ter sido o primeiro do gênero
a utilizar a verdadeira paisagem escandinava como cenário, tendo suas
cenas externas filmadas nos belos fiordes noruegueses, o que contribuiu
imensamente para o deslumbramento do público e para a sensação de
veracidade histórica, algo desejado pela produção. No entanto, durante a
produção e mesmo antes, o filme foi alvo da rígida censura americana da
época, que não aprovou o excesso de violência e sexualidade evidenciadas
no roteiro. O pretexto utilizado pelos produtores do filme era de que
tais cenas e sua brutalidade trariam maior verossimilhança ao filme, já
que, segundo eles, assim era a sociedade dos vikings. Após a liberação
de algumas dessas cenas, a utilização da justificativa passou a ser
recorrente em filmes posteriores, reforçando bastante certos estereótipos
e difundindo vários clichês do gênero.

Na década seguinte, por influência do enorme sucesso de \emph{The
Vikings}, o tema da Escandinávia Medieval e seus intrépidos habitantes
passa a ser explorado amplamente pela cultura pop no geral. Entre as
produções mais relevantes, podemqos destacar o filme \emph{The Long
Ships} (\emph{Os Legendários Vikings}) de 1964, um dos poucos a retratar
a aproximação dos vikings com populações islâmicas. No entanto, há forte
presença de estereótipos, como na cena em que um grupo de guerreiros
vikings entra em um harém cheio de mulheres e as atacam violenta e
sexualmente. É importante perceber que o público desse gênero
cinematográfico, até então e durante muito tempo, era majoritariamente
masculino, o que explica a grande quantidade de cenas de teor sexual
presentes nesses filmes, muitas vezes sem nenhuma utilidade para o
desenrolar do enredo. Nessa década de 1960 o cinema italiano sofria
forte influência das grandes produções hollywoodianas, ocasionando a
realização de vários filmes que tentavam ``imitar'' gêneros americanos, como, por
exemplo, o chamado \emph{Western Spaghetti}, em alusão ao faroeste
estadunidense, e os filmes de aventura e épicos. Assim, uma grande
quantidade de filmes de aventura com temática viking acaba
sendo produzida na Itália, geralmente com orçamentos diminutos. Alguns
exemplos são \emph{Gli Invasori} (\emph{A Vingança dos Vikings}),
\emph{I Tartari (Os Bravos Tártaros}) e \emph{L'ultimo dei Vikinghi}
(\emph{O Último dos Vikings}), todos de 1961, \emph{I Normanni} (\emph{A
Batalha que Salvou um Império}) de 1962, \emph{Erik, il Vichingo}
(\emph{Érico, o Viking}) e \emph{Il Tesoro della Foresta Pietrificata
(Audácia dos Vikings)} ambos de 1965, sendo este último inspirado na
ópera wagneriana, e \emph{I Coltelli del Vendicatore} (\emph{Os Punhos
do Vingador}) de 1966, entre outros. Esses filmes, na maioria das vezes,
seguiam o exemplo das produções americanas, repetindo e reforçando
estereótipos e por vezes até exagerando-os como forma de chamar a
atenção do público.

Nessa época começam a aparecer, ainda que escassamente, as primeiras
produções de filmes escandinavos sobre o tema. O famoso diretor sueco
Ingmar Bergman realiza, em 1960, \emph{Jungfrukällan} (\emph{A Fonte da
Donzela}). O filme apresenta questões sobre a permanência de costumes da
religiosidade pagã em uma Suécia já cristianizada, além de temas
recorrentes na filmografia do diretor como a moralidade e problemas
filosóficos. Em 1967 é lançada uma coprodução entre Suécia, Islândia e
Dinamarca,{\emph{Den
røde kappe}}, baseado na famosa lenda do folclore escandinavo
\emph{Hagbard~and~Signe}. Em 1976 um curioso filme é lançado na
Dinamarca, realizado por uma dupla de artistas plásticos dinamarqueses,
Poul Gernes e Per Kirkeby. \emph{Normannerne} trata-se de uma espécie de
visita guiada por sagas e lendas da mitologia escandinava com alguns
momentos de reconstituição, funcionando quase como um filme didático
sobre o assunto. Em 1981 é produzido na Islândia o filme \emph{Útlaginn}
inspirado na \emph{Saga de Gísli}, uma das muitas sagas de famílias
islandesas. Também na Islândia, o proeminente cineasta Hrafn
Gunnlaugsson dirige e escreve uma série de três filmes, que ficou
conhecida como \emph{Raven Trilogy} (\emph{A trilogia do Corvo}),
explorando belamente a temática dos vikings, são eles \emph{Hrafninn
Flýgur} de 1984, sobre um irlandês que busca vingança após um grupo de
guerreiros vikings matarem seus pais, \emph{Í Skugga Hrafnsins} de 1988,
inspirado na lenda de Tristão e Isolda e \emph{Hvíti Yíkingurinn} de
1991, todos os três fazendo uma excelente representação de costumes,
cultura, sociedade e especialmente da religiosidade da Escandinávia
Medieval. É visível nas produções escandinavas uma maior preocupação com
a não disseminação dos estereótipos em relação às produções americanas. Nestas últimas, há,
em sua maioria, um enfoque narrativo na belicosidade e na
violência, enquanto nas primeiras há uma maior e melhor
representatividade da vida cotidiana e na religiosidade.

Em 1982 é lançado \emph{Conan the Barbarian} (\emph{Conan, o Bárbaro}),
adaptação para os cinemas do personagem criado por Robert E. Howard.
Apesar de se passar em um universo criado pelo autor, são claras as
influências da cultura e da religiosidade viking nas aventuras do
guerreiro, além da permanência de todos os estereótipos vistos
anteriormente. Após o sucesso de \emph{Conan the Barbarian}, rapidamente
a produção de filmes do subgênero da fantasia conhecido como
\emph{espada e feitiçaria} cresce exponencialmente, geralmente
caracterizados por guerreiros supermusculosos, armas exageradas e
personagens femininas hipersexualizadas, seguido novamente por uma
leva de filmes italianos que tentavam ``pegar carona'' nessas produções.
Esses filmes, juntamente com as histórias em quadrinhos, levaram a uma
forte associação da cultura viking com a fantasia, tornando-os quase
sinônimos. Indo para um outro caminho, em 1989 o grupo de comédia
britânico Monty Python produz o filme \emph{Erik the Viking} (\emph{As
Aventuras de Érico, o Viking}) dirigido por Terry Jones, satirizando
tanto o modo de vida viking, fazendo uso de clichês e estereótipos de
forma irônica e inteligente, quanto a própria sociedade da época em que
o filme foi produzido.

No fim dos anos 1990 dois filmes sobre o tema tiveram maior destaque: a
segunda adaptação para o cinema de \emph{Prince Valiant} de 1997, ainda
na onda dos filmes ``espada e feitiçaria'' da década anterior, e um
filme que já se tornou um clássico sobre a temática viking, \emph{The
13th Warrior} (\emph{13º Guerreiro}) de 1999. O filme é inspirado no
livro \emph{Devoradores de Mortos} de Michael Chrichton, que, por sua
vez, toma como base os manuscritos do árabe Ahmad ibn Fadlan
sobre suas viagens por terras escandinavas, e também o famoso poema
épico anglo-saxão \emph{Beowulf}. E apesar de conservar a presença de
alguns estereótipos, faz uma boa representação do cotidiano e da
estrutura social das aldeias, bem como da indumentária e armamentos. No
mesmo ano, uma outra reinterpretação de Beowulf chega aos cinemas,
intitulado \emph{Beowulf} (\emph{Beowulf: O Guerreiro das Sombras}),
estrelado por Christopher Lambert, nos apresenta uma versão
pós-apocalíptica do mito, unindo ficção-científica à fantasia.

Na década seguinte houve pelo menos mais quatro adaptações
cinematográficas do poema, a primeira em 2005, \emph{Beowulf and
Grendel} (\emph{A Lenda de Grendel}), com ótimas representações da
cultura material, principalmente dos equipamentos e indumentárias
militares e com cenas externas gravadas na Islândia. Em 2007 foi
produzido para \versal{TV}, \emph{Grendel}, um filme com estereótipos
extremamente exagerados, como elmos com chifres enormes e guerreiros
batalhando sem nenhuma proteção peitoral, e efeitos especiais ainda mais
esdrúxulos com um Grendel que mais se assemelha a um lobisomem do que
qualquer outra coisa. E no mesmo ano estreia aquela que é possivelmente a mais
conhecida adaptação para os cinemas do poema até hoje, \emph{Beowulf}
(\emph{A Lenda de Beowulf}), dirigido por Robert Zemeckis e com roteiro
de Neil Gaiman. O filme utiliza uma técnica de captura de movimento
transformando os atores em animação, de forma que também
cenário, objetos de cena e até os trajes dos personagens fossem criados
digitalmente. Apesar do tom fantasioso dado ao enredo, diferente da
adaptação de 2005, as representações são bastante coerentes. E por
último, outra adaptação que, como a de 1999, entrelaça ficção científica
e fantasia, \emph{Outlander} (\emph{Outlander -- Guerreiro vs.
Predador}) de 2008, o filme conta sobre uma nave alienígena que cai na
Noruega do século \versal{VIII} ao tentar fugir de uma fera extraterrestre. O
único tripulante da nave precisa, então, da ajuda dos habitantes locais
para combater a terrível criatura. Aqui, ainda que com algumas
ressalvas, como a alusão a mulheres guerreiras, e alguns equívocos sobre
eventos históricos, há uma boa representação da cultura material e da
sociedade escandinava da época em que o filme é ambientado, com destaque
às cenas de combate, sem muitos excessos.

Ainda em 2007 o filme \emph{Pathfinder} (\emph{Desbravadores}) chama
atenção pelos exageros com que os guerreiros vikings são representados,
quase como monstros gigantes, com armaduras grotescas e um dialeto
gutural e apresentando uma fotografia estranhamente escura, o que se
tornaria uma referência de muitos dos filmes posteriores sobre o tema. O
filme é uma adaptação de uma produção norueguesa de 1987
dirigida por Nils Gaup, \emph{Ofelas} (\emph{Fugindo da Morte}), que,
diferente de seu sucessor norte americano, nos entrega um belo enredo
minimalista com ótimas representações da religiosidade pagã e da cultura
dos habitantes do extremo norte da Escandinávia por volta do ano 1000
d.C.

A partir do final da primeira década dos anos 2000, com o advento de
várias facilidades técnicas e de produção, vários filmes independentes e
de baixo orçamento são produzidos, alguns deles para serem exibidos na
\versal{TV} e outros para \emph{home video} ou internet; alguns com uma melhor
representação dos vikings como \emph{A Viking Saga -- Son of Thor}, de
2008, \emph{1066} (\emph{1066 - a Batalha Pela Terra Média}), de 2009,
\emph{A Viking Saga -- The Darkest Day}, de 2013, outros com um maior
grau de fantasia e estereótipos, muito inspirados pelo sucesso da
adaptação milionária do herói da Marvel, Thor, para os cinemas em 2011,
como \emph{Almighty Thor} (\emph{O Poderoso Thor}), de 2011, e
\emph{Vikingdom} (\emph{Vikingdom -- O Reino Viking}), de 2013. Outro
forte catalisador para produções sobre a temática foi a estreia da série
\emph{Vikings} em 2013, gerando pequenas produções como \emph{Hammer of
the Gods} (\emph{Martelo dos Deuses}), de 2013, \emph{Northmen -- A Saga
Viking} (\emph{Northmen -- A Saga Viking}), de 2014, e \emph{Sword of
Vengeance} (\emph{Espada da Vingança}), de 2015. Mais recentemente, em
2016, Nils Gaup, diretor de \emph{Ofelas} (\emph{Fugindo da Morte}),
inspira-se na pintura \emph{Birkebeinerne~på Ski over Fjeldet med
Kongsbarnet}, do norueguês Knud Bergslien, para produzir o filme
\emph{Birkebeinerne} (\emph{O Último Rei}). Esse filme trata da fuga e a proteção de
um bebê, filho bastardo do rei, durante a guerra civil norueguesa, que ocorreu entre
os séculos \versal{XII} e \versal{XIII}. E ainda que questões sociais e políticas não sejam
aprofundadas no filme, ele traz uma narrativa instigante e bastante
crível do ponto de vista histórico.

No campo da animação, as primeiras produções de que se tem notícia são
dois curtas do Pernalonga de 1957 e 1961, respectivamente \emph{What's
Opera, Doc?}, parodiando a ópera wagneriana, e \emph{Prince Violent},
claramente uma brincadeira com o personagem de Hal Foster. Ambos os
curtas, como é de se esperar, reproduzem uma série de estereótipos e
exageros na intenção de acentuar a comicidade das produções. Em 1986
uma surpreendente animação dinamarquesa chamada \emph{Valhalla}, de
Peter Madsen e Jeffrey James Varab, trata questões mitológicas de forma
leve, com forte inspiração nas animações da Disney. O sucesso do longa
gerou ainda oito curtas, tendo um de seus personagens como protagonista.
Em 1986, o viking \emph{Hägar, o Horrível}, criado em 1973 pelo
quadrinista Dik Browne, ganha um curta metragem em animação,
apresentando um pequeno compilado dos estereótipos vistos em seu
material original. Apenas muitos anos depois, em 2006, os vikings são
referenciados em uma animação de destaque: \emph{Astérix et les Vikings
(Asterix e os Vikings)} traz o gaulês criado por Albert Uderzo e René
Goscinny em sua oitava aventura animada, baseada no álbum \emph{Asterix
e os Normandos} de 1966. Nela, são apresentados vikings com capacetes
chifrudos e bebendo cerveja em canecas feitas de caveira humana, além de
terem todos nomes terminados em ``af'', em contraponto aos gauleses, que
têm nomes terminados em ``ix''. Em 2009, uma coprodução entre França,
Bélgica e Irlanda nos mostra a vida em um mosteiro prestes a ser atacado
por vikings, os quais, por serem mostrados do ponto de vista cristão, são
representados como criaturas horrendas, lembrando muito os vistos em
\emph{Pathfinder} (\emph{Desbravadores}). Em 2010, é lançada \emph{How to Train Your Dragon} (\emph{Como Treinar Seu Dragão}), a maior produção cinematográfica de animação tendo os vikings como protagonistas
até então. Baseando-se na série de livros infantis de Cressida Cowell, o
filme apresenta uma visão extremamente fantástica, de forma que mal
podemos dizer que se trata de uma comunidade viking se eles mesmos assim
não se denominassem e não portassem os famosos capacetes com chifres. O
enorme sucesso do filme originou ainda duas continuações e várias
séries televisas.

\SIG{Elvio Franklin Menezes Teles Filho}

Ver também Era Viking; Escandinávia; Vikings e Alemanha moderna; Vikings
na literatura; Vikings na música; Vikings nas artes plásticas; Vikings
nos quadrinhos; Vikings na televisão; Vikings no Brasil.

\begin{itemize}
\item \versal{BURKE}, Peter. \emph{Testemunha Ocular: História e Imagem}. Bauru (\versal{SP}):
Edusc, 2004

\item \versal{HARTY}, Kevin J. (org.). \emph{The Viking on Film: essays on depictions
of the Nordic Middle Ages}. North Carolina: McFarland \& Company, 2011.

\item \versal{LANGER}, Johnni.~Fúria odínica: a criação da imagem oitocentista sobre os
Vikings. \emph{Varia História}, vol. 25, n. 25, 2001, pp. 214-230.

\item \versal{LANGER}, Johnni. A volta dos bárbaros: Asterix e os Vikings no cinema e
na \versal{HQ}. \emph{História, Imagem e Narrativas}, vol. 3, 2006, pp. 267-274.

\item \versal{LANGER}, Johnni. Vikings, cultura e região: o mito arqueológico dos
Estados Unidos. \emph{Olho da História}, (\versal{UFBA}), n. 18, 2012.

\item \versal{LANGER}, Johnni. ~Fé, exotismo e macabro: algumas considerações sobre a
religião nórdica antiga no cinema. \emph{Ciências da Religião}
(Mackenzie. Online), vol. 13, 2015, pp. 155-180.
\end{itemize}
\section{\versal{VIKINGS NOS QUADRINHOS}}

A exemplo do cinema, as histórias em quadrinhos desde os seus primórdios
tiveram os guerreiros nórdicos como um de seus grandes temas. Em parte
porque grande soma dos leitores era de adolescentes ávidos por aventuras
em locais exóticos e distantes de sua sociedade. Por outro lado, devido
ao fascínio que a Era Viking manteve no Ocidente desde o romantismo,
envolvida em mistérios e estereótipos diversos.

Desta maneira, um dos grandes clássicos da nona arte, \emph{Príncipe
Valente}, teve como principal protagonista um descendente dos nórdicos.
Com desenhos belíssimos, sequências formidáveis e uma narrativa
envolvente, criada pelo genial Hal Foster em 1937, a série fundia
História Medieval com fantasia -- fazendo com que personagens históricos
circulassem entre seres fantásticos --, paisagens fidedignas e se mesclando
a seres como dragões e outras criaturas monstruosas. Ao mesmo tempo em
que são os heróis (na figura do protagonista), os vikings também foram
vilões -- o quadrinho de Foster foi um dos principais propulsionadores
do estereótipo do guerreiro nórdico: dotado de chifres e equipamentos
imaginários, beberrão, irreverente e intrépido. A fusão de personagens e
datas em um mesmo e anacrônico período também acabou sendo um modelo
para o cinema até os anos 1960: no início da Alta Idade Média, os
vikings encontram-se com o rei Artur em uma sociedade totalmente
feudalizada, em meio a castelos, torneios e armaduras completas.
\emph{Príncipe Valente} recebeu várias versões ao cinema, sendo mais
famosa a de 1954. O quadrinho foi tema de uma dissertação de mestrado em
História Comparada pela \versal{UFRJ}: \emph{Entre luzes e trevas: o Príncipe
Valente e as representações políticas e civilizacionais nos quadrinhos},
de autoria de Carlos Manoel de Hollanda Cavalcanti.

Um dos quadrinhos franceses mais famosos, o gaulês Asterix recebeu a
visita dos nórdicos em 1967, no volume \emph{Asterix et les Normans}, com
várias traduções brasileiras. Como em grande parte da coleção, o humor,
a ironia e o cômico histórico fazem parte dessa narrativa, mas também
não faltam os estereótipos, presentes especialmente na figura do viking
portador de grande quantidade de crânios (para beber, para uso em
amuletos, em rituais etc). Recebeu uma adaptação cinematográfica em
2006, de grande sucesso. A principal novidade do filme em relação ao
quadrinho original foi a inclusão de uma personagem feminina, Abba.

Em 1981, o álbum europeu \emph{Os vikings} foi integrante de uma
famosa coleção francesa de 1981 (A descoberta do mundo, com nomes de
peso como Hugo Pratt, Guido Crepax, Sergio Toppi, Enrique Sió, Sergio
Toppi, entre outros). O álbum e a coleção como um todo, diferenciam-se
pelo seu forte caráter histórico, caracterizando-se por ser uma espécie de História Universal aos
moldes da historiografia dos Annales em forma de quadrinho. O álbum
sobre vikings possui duas narrativas: a primeira, \emph{Drakkars a
leste}, reconstituí a trajetória dos nórdicos no mundo eslavo, com o
maravilhoso traço de Eduardo Coelho e argumento de Jean Ollivier. A
segunda, \emph{Os reis do mar}, de José Bielsa e Jacques Bastian,
descreve as expedições nórdicas no Atlântico Norte, baseadas
especialmente nas sagas islandesas. O ponto alto dessa segunda estória
fica para a narrativa de Freydís Eiríksdóttir em Vínland, em uma de suas
melhores reconstituições visuais até nossos dias.

Uma sensacional série franco-belga foi criada em 1977 pela dupla
Rosinski e Van Hamme, \emph{Thorgal}, fundindo História Medieval com
fantasia aos moldes do universo de Howard Carter e algumas pitadas de
ficção científica. O traço é muito colorido e as narrativas envolventes,
com personagens admiráveis e belas sequências. No Brasil foram
publicados os quatro primeiros álbuns (A feiticeira traída; Os três
anciões do país d´Aran; A galera negra e; A ilha dos mares gelados, todos
pela \versal{VHD}). A série atualmente conta com 35 álbuns. Alguns elementos das
sagas islandesas foram fundidos com elementos históricos e crônicas
medievas, concedendo um ritmo extremamente dinâmico para as narrativas,
mas sempre dentro do referencial dos nórdicos como heróis aventureiros.

Uma curiosa produção dos quadrinhos italianos (fumetti), publicada
originalmente em 1980, com o famoso personagem Tex, foi \emph{A ilha
misteriosa}. A premissa remete ao filme \emph{A ilha do topo do mundo}
(produção Disney de 1974), onde um grupo de caubóis encontra uma
comunidade nórdica da Era Viking isolada e vivendo incólume, tal como teria sido no
medievo, em pleno século \versal{XIX}. O resultado é um tanto grotesco, onde os
vikings são representados como bárbaros primitivos e supersticiosos.
Essa ideia de uma ``cápsula do tempo'' fez muito sucesso na ficção
televisiva e quadrinistica, resultando em outros encontros espetaculares
entre nórdicos com culturas de temporalidades diversificadas:
\emph{Tarzan e os vikings} (animação para a \versal{TV}, 1976); a terceira versão
de \emph{Jonny Quest} (episódio: Alligators and Okeechobee Vikings, de
1996). Mais uma vez, motivos utópicos se convertem em elementos para
contrapor os valores do leitor com os elementos históricos e imaginários
do passado, valorizando o mundo medieval como fonte para a aventura --
em especial, os vikings enquanto mantenedores de um passado exótico
e quase sempre muito misterioso.

Mas nem sempre a Escandinávia foi tomada dentro de um referencial
positivo. \emph{Desbravadores}, um quadrinho de 2006 criado por Laeta
Kalogridis e Christopher Shy, com uma bela arte sequencial, cores
escuras e fortes e boas sequências de ação, é um bom exemplo disso. A
narrativa gira em torno do encontro entre os indígenas norte-americanos
e os nórdicos, estes últimos vistos de forma negativa e muito
estereotipada. O filme homônimo de 2007 conseguiu piorar visualmente
ainda mais os vikings no imaginário coletivo, sendo estes muito mais
seres surgidos do mais profundo inferno cristão do que colonos e
exploradores do Novo Mundo -- com equipamentos e vestimentas sempre em
tons escuros, com dezenas de formatos de chifres e embarcações com
imaginários esporões laterais. O comportamento também é macabro: cruéis
exploradores que penetram na América para capturar escravos para o
mercado europeu, destruindo a pureza da cultura nativa norte-americana.

Também o Oriente teve interesse na história nórdica. Vinland, série de
mangá japonês criado por Makoto Yukimura em 2005, manteve suas
narrativas criadas em torno da colonização nórdica no Atlântico Norte,
mas com resultados pouco precisos. Equipamentos, cotidiano, contexto
histórico e outros detalhes são superficiais ou fantasiosos. Alguns
escandinavos utilizam \emph{shurikens} (estrelas com pontas afiadas para
arremesso) e balestras, equipamentos desconhecidos na Era Viking. De um
ponto de vista artístico, a obra também é muito inferior a outros
quadrinhos japoneses de teor histórico, como a série \emph{Lobo
Solitário}, de Koike e Gojima.

Um dos quadrinhos cômicos mais famosos de todos os tempos foi
\emph{Hägar}, criado por Dik Browne em 1973. Mais do que reconstituir os
nórdicos da Era Viking, a série ironiza o estilo de vida e a sociedade
norte-americana, com resultados formidáveis e fazendo muito sucesso até
nossos dias. A série recebeu uma dissertação de mestrado em História
pela \versal{PUC-SP}: \emph{O humor e a crítica em Hägar}, de Fabio Antonio
Costa. Nesse estudo foi analisado o desenvolvimento de uma concepção de mundo
por meio da desconstrução de ideias e discursos através de seus múltiplos
recursos e forma de linguagem, valorizando outras manifestações
humanas e grupos sociais pouco evidentes na conturbada década de 1970
nos Estados Unidos.

Nos quadrinhos de Hägar percebemos imagens corretas e também anacrônicas
sobre a Idade Média. Para citar alguns exemplos, o formato da casa
escandinava da Era Viking está adequado dentro dos parâmetros da
cultura material: janelas sem vidro, cobertura de palha ou turfa, fronte
do telhado terminando na intersecção de dois dragões ou pontas
estilizadas. Em algumas histórias surgem armaduras completas de placas
metálicas, que realmente estão corretas, mas seu uso somente
popularizou-se depois do século \versal{XIII} e nunca foi conhecida na
Escandinávia da Era Viking. O escudo de metal para os nórdicos é errôneo
(utilizavam madeira) e o capacete com chifres é um estereótipo que foi
criado durante o século \versal{XIX}. Apesar da mulher escandinava ter um grande
poder dentro da esfera doméstica, a relação entre Helga e seu marido
Hägar não corresponde às fontes medievais: é antes um reflexo da
sociedade norte-americana pós anos 1950 e a crescente visibilidade da
mulher nos novos papéis sociais. Ao ler as séries quadrinísticas de
Hägar, o leitor deve estar atento para perceber como os vikings (e a
própria Idade Média) servem de contraponto aos valores modernos, e a
comicidade representa uma ferramenta poderosa para criticar, refletir, repensar ou
imaginar o passado e o presente.

Seguindo diversas tendências artísticas, mas ao mesmo tempo inovando em
muitos pontos, o escritor Brian Wood criou uma nova proposta
quadrinística em 2007. Leitor das sagas islandesas e de obras acadêmicas
sobre a Era Viking, lançou-se em uma empreitada para quadrinizar de
forma realista o mundo da Europa Setentrional e Oriental na transição da
Alta Idade Média para Central. O resultado geral foi razoável, com
alguns resultados mais fracos e outros excepcionais. As primeiras
histórias foram publicadas no Brasil mensalmente com o selo
\emph{Vertigo} até 2011 (n. 20, inicialmente como \emph{Vikings}, depois
como \emph{Nórdicos}). Desta fase, destacam-se o ciclo de Svein, com o
desenhista Davide Gianfelice, e os contos \emph{Irmãs de escudo}
(analisado em Langer, 2012) e \emph{Lindisfarne} -- este, sem sombra de
dúvida, o melhor momento do início da série, relatando uma atípica
situação de conversão ao paganismo por um cristão no mundo da Inglaterra
anglo-saxônica invadida pelos dinamarqueses. O quarto volume (coletânea
dos fascículos mensais), \emph{The Plague Widow}, publicado no Brasil
recentemente como \emph{Vikings: a viúva do inverno}, é inteiramente
dedicado a uma narrativa transcorrida no Volga do século \versal{XI}, numa vila
ameaçada por uma peste. Em meio a uma imensa hostilidade ambiental, a
personagem Hilda e sua filha Karin sobrevivem também a conflitos
violentos entre as lideranças da comunidade. Além da intensa
dramaticidade, a narrativa possui o melhor artista de toda a coleção,
Leandro Fernandez, que fez uma detalhada pesquisa gráfica sobre o
cotidiano material, militar, arquitetônico e ambiental do mundo nórdico
na área eslava. Com certeza, o ponto culminante de toda a obra de Wood.

O quinto volume, (\emph{Metal}), tem momentos altos e baixos. A primeira
narrativa, \emph{The Sea Road}, apresenta um interessante conto de uma
viagem de islandeses para o Ártico, com desenhos de Fiona Staples. A
segunda parte, o ciclo \emph{Metal}, possui um artista inferior,
Riccardo Burchielli, resultando em um trabalho que oscila entre o
histórico, o estereótipo e a fantasia pura. Novamente o confronto entre
paganismo e cristianismo é um dos pontos retratados por Wood, mas o uso
de armamento exagerados (como a espada do personagem Érico), a
intervenção da deusa Hulda, o ressuscitar do vilão, a existência de
vilas cristãs na Noruega do ano 700 d.C., entre outros deslizes, não
conseguem convencer o leitor mais exigente. O conto \emph{The Girl in
the Ice}, bem ao contrário, é estupendo. Trata de uma narrativa
ambientada na Islândia durante a Era dos Sturlungar, século \versal{XIII}, onde
um camponês encontra o corpo de uma moça preservado dentro do gelo,
tendo de arcar com as consequências sócio-religiosas dessa descoberta
singular.

O sexto volume, \emph{Thor´s daughter}, apresenta várias histórias
independentes. A primeira, \emph{The siege of Paris}, remonta à pilhagem
da cidade francesa em 885, com uma boa caracterização das técnicas de
combate e armamentos, mas o desenhista Simon Gane possui um traço que
tende ao semicaricatural, atrapalhando muito o andamento épico da
narrativa. O próximo conto, \emph{The Hunt}, a respeito de um caçador
sueco do ano mil, apresenta um excelente confronto entre a sobrevivência
humana e animal, mas ao mesmo tempo esse mesmo argumento de ficção
quadrinística já havia sido criado por Berardi e Milazzo para um álbum
de \emph{Ken Parker} dos anos 1990, perdendo o tom de originalidade. O
álbum se encerra com \emph{Thor´s daughter}, uma bela história
ambientada nas ilhas Hébridas em 990 d.C., mas que se aproxima mais das
sagas lendárias do que das de família. Nela, a jovem Birna, filha de um
líder da comunidade, assume o papel da chefia militar e política após a
morte do mesmo. O traço a lápis de Marian Churchland reforça uma
poderosa identidade feminina para o conto, que não corresponde ao papel
sócio-histórico da mulher na sociedade escandinava -- ao contrário da
representação altamente realista de Hilda em \emph{Vikings: a viúva do
inverno} -- mas com resultados bem mais interessantes que as guerreiras
do conto \emph{Irmãs de escudo} (\emph{The shield maidens}).

A série teve o fim oficialmente anunciado com o volume \emph{The
icelandic trilogy}, publicado em 2013, com desenhos de Paul Azaceta,
Declan Shalvey e Danijel Zezelj. O trabalho de Brian Wood poderia ser
muito melhor caso a parte gráfica fosse designada para somente um
artista, como o competente Leandro Fernandez -- ou alguém com o estilo
do autor das maravilhosas capas, Massimo Carnevale. \emph{Northlanders}
é inferior aos melhores trabalhos da escola franco-belga que retrataram
temas nórdicos medievais, como \emph{Nordman}, de Stalner e Bardet, e
principalmente as maravilhosas coleções \emph{Moi Sven}, \emph{L´Epte}
e \emph{Italia Normannorum}, todas do genial roteirista Riamel. Em todo
caso, a série é indispensável para todos aqueles que se interessam pela
Escandinávia Medieval e são fãs de quadrinhos de temática histórica.

Em 2009 outra série foi criada, \emph{Viking}, pela dupla Ivan Brandon e
Nic Klein. Com desenhos mais grosseiros e uma estética muito mais
agressiva do que a série \emph{Northlanders}, o quadrinho \emph{Viking}
também apresenta narrativas bem estereotipadas e repletas de clichês. As
cenas de aventuras, lutas, batalhas e conflitos são a tônica da obra,
não dando espaço para outros aspectos da Escandinávia da Era Viking,
como colonização, vida cotidiana no meio rural, artesanato, caça e
comércio etc. Os detalhes dos equipamentos e vestuário também são muito
superficiais e grosseiros, muito inferiores aos diversos quadrinhos
anteriores.

Nos últimos anos, a quantidade de títulos envolvendo a Era Viking ou
temas mitológicos nórdicos aumentou consideravelmente. Em 2014 foi
criado um quadrinho baseado na série televisiva \emph{Vikings}, com
roteiro de Michael Hirst e desenhos de Dennis Calero. O resultado foi
extremamente fraco, com uma narrativa pobre e ilustrações muito simples.
Um dos mais recentes e empolgantes lançamentos foi o álbum \emph{Sagas
of the Northmen} (2015), produzido por diversos escritores e
ilustradores, entre os quais Sean Fahey e Marcelo Basile. A coleção não
se baseia em protagonistas ficcionais, mas em personagens históricas e
as narrativas envolvem áreas de colonização ou influência nórdica, como
Islândia, América, Bizâncio e ilhas britânicas. As sete estórias foram
desenhadas em preto e branco, possuindo um viés clássico e bem
detalhista, com bons resultados visuais. Algumas das narrativas são bastante
tradicionais, como \emph{Satan}´s \emph{hordes}, reconstituindo o ataque
a Lindisfarne, enquanto que \emph{No king but the law} é uma
impressionante dramatização envolvendo o sistema jurídico da Islândia em
939 d.C. A estória mais impressionante é \emph{Heart of iron}, escrita
por Susan Wallis e ilustrada por Todor Hristov, a respeito da epopeia de
Freydís Eiríksdóttir em Vínland. Além do belo traço, habilmente
contrastando técnicas de claro/escuro, a ambientação une-se à uma densa
narrativa de psicologia de sobrevivência dos assentamentos nórdicos do
Novo Mundo, que foram registrados pelas sagas do Atlântico Norte.

\SIG{Johnni Langer}

Ver também Era Viking; Escandinávia; Vikings e Alemanha moderna; Vikings
na literatura; Vikings na música; Vikings nas artes plásticas; Vikings
no cinema; Vikings na televisão; Vikings no Brasil.

\begin{itemize}
\item \versal{CAVALCANTI}, Carlos Manoel de Hollanda. \emph{Entre luz e trevas: o
Príncipe Valente e as representações políticas e civilizacionais nos
quadrinhos (1936-1946)}. Rio de Janeiro: Dissertação de Mestrado em
História Comparada, \versal{UFRJ}, 2007.

\item \versal{COSTA}, Fabio Antonio. \emph{O humor e a crítica em Hagar, o Horrível, de
Dik Browne, no Jornal Folha de São Paulo (1973-1974)}. Dissertação de
Mestrado em História. São Paulo: \versal{PUC-SP}, 2013.

\item \versal{LANGENBRUCH}, Beate\emph{La Fabrique de la Normandie médiévaledans
quelques bandes dessinées historicisantes}. Rouen: Colloque La Fabrique
de la Normandie, 2011.

\item \versal{LANGER}, Johnni. Guerreiras na Era Viking? Uma análise do quadrinho Irmãs
de Escudo (Série Northlanders). \emph{Roda da Fortuna}, 1(1), 2012, pp.
267-293.

\item \versal{LANGER}, Johnni. O ensino de História Medieval pelos quadrinhos.
\emph{História, imagem e narrativas}, vol. 8, 2009, pp. 01-24.

\item \versal{LANGER}, Johnni. Os vikings e o estereótipo do bárbaro no ensino de
História. \emph{História \& Ensino}, vol. 8, 2002, pp. 85-98.

\item \versal{VADILLO}, Mônica Ann Walker. Comic Books Featuring the Middle Ages.
\emph{Itinéraires,} 2010-2013, pp. 153-163.
\end{itemize}
\section{\versal{VÍNLAND}}

Terra das Vinhas ou Terra das Parreiras são duas possíveis traduções
que podemos fazer para \emph{Vínland}, termo que os
nórdicos, durante o processo de descobrimento da América do Norte, deram
para esta nova terra que haviam descoberto. A narrativa sobre a
descoberta dessa nova terra, descrevendo sua jornada e a razão que a sustentou,
pode ser encontrada nos vestígios arqueológicos, principalmente em
L'Anse aux Meadows e também através do registro literário das Sagas
do Atlântico Norte.

A descoberta da Terra das Parreiras ocorre de forma diferente dentro
das narrativas que compõem o conjunto de sagas supracitado. Na Saga
dos Groenlandeses, temos o relato da viagem de Bjarni Herjúlfsson. Este,
após aportar em Eyrar,na Islândia, resolveu partir com a sua tripulação
para a Groenlândia, mesmo sem conhecer o trajeto: ``Deve parecer estúpida
a nossa viagem, já que nenhum de nós nunca foi ao mar da Groenlândia''
(\versal{ANÔNIMO}, 2007a, p. 61). De fato, esse desconhecimento acaba culminando
em perda de direção por parte de Bjarni, que passa vários dias sem saber
qual a direção que estavam velejando. Após se localizarem, ainda
viajaram por vários dias até que avistaram uma terra: ``Minha decisão é
que velejemos para perto da costa'', fala Bjarni (\versal{ANÔNIMO}, 2007a, p.
61).

Viram uma terra plana, repleta de florestas e pequenas elevações, sempre
mantendo a terra a bombordo. Após dois dias de afastamento dessa terra,
eles avistam uma outra terra, ainda sem saber se era a Groenlândia, algo
que consideram improvável pela ausência de geleiras. Esta terra era
plana e verde, e mesmo com a necessidade de água e madeira, Bjarni opta
por não desembarcar. Deram as costas para a terra e por mais
três dias velejaram com vento sudoeste, até que avistarem uma terceira
terra, com traços montanhosos e coberta por geleiras, e pela ``{[}...{]}
terra me parece não ter muito a oferecer'', o líder Bjarni, novamente
segue o ``curso'' sem desembarcar. Costearam a região, compreendendo que
se tratava de uma ilha, deram novamente as costas para a terra e
seguiram viagem por mais quatro dias, quando avistaram uma quarta terra,
que finalmente era a Groenlândia.

Este relato de uma terra observada é fundamental para a descoberta
efetiva da nova terra dentro da narrativa. Bjarni Herjúlfsson vai ao
encontro de Érico Hákonarson, na Noruega, e conta de sua viagem até a
Groenlândia, tornando-se membro da guarda pessoal desse líder. Ao
retornar para a Groenlândia após um ano, Leifr Eiríksson, que era filho de
Érico , o Vermelho, vai ao seu encontro, com interesse em seu relato e na
compra de seu navio e contratação da tripulação de trinta e cinco
homens que passou pela viagem confusa para a Groenlândia. Sendo assim,
fica elencado o motivo e a forma com quais as novas terras foram vistas e que
estimularam Leifr Eiríksson a viajar rumo à futura Vínland. Na
\emph{Saga de Eirík Vermelho}, é dito que Leifr Eiríksson, após ir à corte
do rei Óláfr Tryggvason, acaba sendo jogado pelo mar e perde seu rumo,
encontrando novas terras. Portanto, tem-se versões diferentes: a
primeira é bem mais detalhada acerca do processo, mas a segunda também apresenta traços e elementos
importantes.

Leifr Eiríksson é considerado o grande descobridor da América do Norte,
já que seu pai fora o colonizador e ``descobridor'' da Groenlândia, que
hoje faz parte do território da América. Segundo o que é contado
na \emph{Saga dos Groenlandeses}, Leifr parte seguindo os passos dos relatos de
Bjarni, e usando o conhecimento de sua tripulação para chegar nas terras
vistas. Primeiramente, ele chega na terceira terra vista por Bjarni,
chamando-a de \emph{Helluland}, a Terra da Placa de Rocha, em que
profere: ``Conosco não aconteceu, quanto a esta terra, como aconteceu
com Bjarni, de não termos pisado em terra. Agora darei um nome à terra e
hei de chamá-la Helluland''. Portanto, apesar de
Bjarni ter visto à terra primeiro, foi Leifr o que primeiro pisou nesta
terra, sendo efetivamente o seu descobridor.

Voltando para o navio, eles encontram algum tempo depois a segunda
terra vista por Bjarni, indo até a costa e novamente desembarcando. Nessa costa, Leifr fala: ``Pelas suas riquezas hei de dar um nome a
esta terra e chamá-la de Markland [\emph{Terra coberta por
floresta}]''. Depois de dias
chegaram em uma ilha ao norte da terra, que por um banco de areia se
separava de um cabo que dava a ver terra. Foram em direção a
esta terra, e ``Não faltavam lá salmões, nem no rio nem no lago, os
maiores salmões que eles já haviam visto. A terra lá era tão rica,
conforme se lhes mostrou, que nenhum animal doméstico precisaria de
cuidado durante o inverno [...]''.

Neste local de grande riqueza, quase uma contraposição ao cenário gélido
e ríspido da Groenlândia, Leifr resolve construir casas e dividir sua
tripulação, para que pudesse de fato efetivar uma exploração de terra,
garantindo o que podemos denominar de primeiro assentamento na nova
terra. Após um tempo de exploração, Tyrkir, pai de criação de Leifr, um
alemão -- ``o homem do sul'' -- muito próximo dele, havia sumido por um
tempo, preocupando os outros. Ao retornar para o grupo de Leifr, ele
revela: ``Eu não havia caminhado muito mais longe do que vós. Acho que
tenho uma novidade para contar; eu achei parreiras e uvas'', e assim passaram a colher essas uvas para uma viagem de
retorno à Groenlândia: ``E no início da primavera eles se aprontaram e
velejaram embora, e Leifr deu nome à terra pelas suas riquezas, chamou-a
de Vínland [\emph{Terra das Vinhas ou Terra das Parreiras}]''.

É dessa forma que os relatos de fontes escritas e literárias nos revelam
como se deu a nomenclatura de Vínland (elementos que aparecem
tanto em obras de Adão de Bremen como em Ari, o sábio), algo que foi
fundamental para se encontrar na década de 1960 os vestígios
arqueológicos em L'Anse aux Meadows (região de Terra Nova e Labrador no
Canadá), que se fizeram provas da presença nórdica na América do Norte,
sendo possivelmente postulado como o assentamento de Leifr, apesar da
existência de outras teorias e de uma impossibilidade precisa de afirmar
que o assentamento encontrado foi de fato o de Leifr Eiríksson.

Após sua viagem, ainda temos a viagem de Thorvaldr e de Thorfínnr
Karlsefni, em que ambos acabam por encontrar com novos elementos desta
terra, os esquimós -- \emph{skrælingjar.} Aqui, temos tanto trocas
amistosas quanto combates mortais. Os conflitos marcados por esses combates acabam sendo o grande vetor que
impossibilita uma efetiva colonização da América do Norte por parte dos
nórdicos, apesar de seu intento, principalmente na expedição de
Karlsefni. Esses relatos vão nos revelar ainda traços, mesmo que dúbios,
desses esquimós e aspectos de sua estruturação social,
apresentando-nos uma visão desses povos autóctones da América, mesmo sendo
uma representação de um escrito cristão nórdico do início do século
\versal{XIII}, que baseia sua narrativa em uma forte dinâmica de tradição oral.

Portanto, podemos definir Vínland como uma terra nova, com
muitas riquezas e facilidades em relação às dificuldades climáticas e ambientais da Escandinávia, principalmente da Groenlândia e da Islândia,
que sofriam com crises climáticas fortes, gerando longos períodos de
fome. É essa
condição natural que modela o sujeito da Escandinávia, que lhe injeta
uma dimensão de explorador muito antes de Colombo pensar sobre as artes
da navegação.

\SIG{José Lucas Cordeiro Fernandes}

Ver também Brattahlid; Groenlândia nórdica; Leif Eriksson; Sagas do
Atlântico Norte.

\begin{itemize}
\item \versal{ANÔNIMO}. A Saga do Groenlandeses. In: \emph{As três sagas Islandesas}.
Tradução de Théo Moosburger. Curitiba: Editora \versal{UFPR}, 2007a.

\item \versal{ANÔNIMO}. A Saga de Eiríkr Vermelho. In: \emph{As três sagas Islandesas}.
Tradução de Théo Moosburger. Curitiba: Editora \versal{UFPR}, 2007b.

\item \versal{HAYWOOD}, John. \emph{The Penguin Historical Atlas of the Vikings}.
London: Penguin Group, 1995.

\item \versal{INGSTAD}, Helge; Ingstad, Anne Stine. \emph{The Discovery of a Norse
Settlement in America: Excavations of Norse Settlement in L'Anse aux
Meadows, Newfoundland}. New York: Checkmark Books, 2001.

\item \versal{JONES}, Gwyn. \emph{The Norse Atlantic Saga: Being the Norse Voyages of
Discovery and Settlement to Iceland, Greenland, and North America}.
Oxford and New York: Oxford University Press, 1986.

\item \versal{O'DONOGHUE}, Heather. \emph{Old norse-Icelandic Literature: a short
introduction}. Hoboken: Blackwell Publisher, 2005.

\item \versal{RAFNSSON}, Sveinbjörn. The Atlantic Islands. In: \versal{SAWYER}, Peter (ed.).
\emph{The Oxford Illustrated History of the Vikings}. Oxford: Oxford
University Press, 2001, pp. 110-133.

\item \versal{ROSS}, Margaret Clunies (ed.). \emph{Old Icelandic Literature and
Society}. Cambridge: Cambridge University Press, 2000.

\item \versal{SHAFER}, John Douglas. \emph{Saga accounts of norse far-travellers}.
Durham: Durham University, 2010.

\item \versal{UMBRICH}, Andrew. \emph{Early Religious Practice in Norse Greenland:
From the Period of Settlement to the 12 th Century}. Reykjavík:
Universidade da Islândia, 2012.
\end{itemize}
\section{\versal{VLADIMIR I DE KIEV}}

Vladimir Sviatoslavich de Kiev, também conhecido como Volodimer,
Vladimir~\versal{I}, o Grande ou Vladimir, o Sol Vermelho, foi o último dos
príncipes de Kiev a professar o paganismo, assim como foi o primeiro a
adotar o Cristianismo Ortodoxo Grego como religião oficial da Rus de
Kiev. Fontes do século~\versal{XVI} afirmam que ele nasceu em 958, e a
\emph{Crônica dos Anos Passados} data sua morte no ano de 1015. Filho de
Sviatoslav Igorevich (964-972) e da servente Malusha, Vladimir foi
príncipe de Novgorod até subir ao trono kievano em 980 após tomar a
cidade de seu irmão Iaropolk Sviatoslavich (973-980) e assassiná-lo. De
acordo com Janet Martin, o governo de Vladimir até o final da década de
980 foi marcado uma forte tentativa de instauração e oficialização do
politeísmo em Kiev como um meio de centralização do poder por meio do
culto a deuses eslavos. O principal desse panteão, Perun, era uma
divindade associada ao trovão e à guerra. É provável que sua promoção
tenha sido uma estratégia de Vladimir para conseguir apoio da elite
militar descendente dos varegues que se assentaram na Rus, 
em sua maioria seguidores de Thor. Vladimir também conseguiu anexar
diversos territórios vizinhos povoados por tribos eslavas por meio de
alianças com as próprias tribos e também com os escandinavos.

Vladimir converteu-se ao cristianismo entre 987 e 988, permanecendo
cristão até sua morte em 1015. De acordo com a tradição imortalizada na
\emph{Crônica dos Anos Passados}, representantes de diversos territórios
vieram até Rus oferecendo ao príncipe suas religiões, dentre elas o
cristianismo ortodoxo grego, ao qual Vladimir eventualmente se converteu
devido à beleza e opulência da religião. O batismo ocorreu em um
contexto em que o imperador bizantino Basílio \versal{II} (960-1025) precisou de
ajuda militar da Rus para controlar uma série de revoltas militares
dentro do império. Pelo auxílio, Vladimir recebeu a mão de Anna, irmã do
imperador, como esposa. Após sua conversão, ele supostamente destruiu
todos os altares e estátuas dedicados às divindades pagãs e, se o que a
\emph{Crônica} diz for verdade, a estátua de Perun foi amarrada a um
cavalo e arrastada ao redor de Kiev, e em seguida espancada por doze
homens. Vladimir foi eventualmente canonizado no século \versal{XIII} e até hoje
é celebrado como santo pela Igreja Ortodoxa: seu dia festivo é
15 de julho.

Um fato interessante da vida de Vladimir que foi omitido nas fontes de
Rus seria sua relação com o rei norueguês Olavo Tryggvason (995-1000).
Conforme a \emph{Saga do rei Olavo Tryggvason,} o futuro monarca passou a
infância e a juventude no Principado de Novgorod, cujo príncipe na época
era Vladimir (ou Valdamar, como presente na fonte). Enquanto lá
permanecia, Olavo foi levado pelo seu tio Sigurth, um guerreiro
importante a serviço de Vladimir, à rainha Allogia de Nóvgorod após
cometer um crime. Allogia convenceu Vladimir a cuidar de Olavo como se
fosse seu filho. Alguns autores acreditam que Allogia seria na verdade
sua avó Olga de Kiev (c. 945-964), todavia é mais provável que ela seja
apenas mais uma das numerosas esposas do príncipe. Eventualmente, 
Olavo se tornou um dos melhores guerreiros de Vladimir e provavelmente
tinha a sua própria drujína (séquito militar), mas os laços de amizade
dos dois foram cortados devido à rumores e intrigas provocadas por
inveja, ocasionando a partida de Olavo para o Báltico.

Além do tratamento dado a Olavo, Vladimir manteve boas relações com os
escandinavos durante seu governo, principalmente no âmbito comercial
onde os Rus eram capazes de obter produtos bizantinos e árabes pelas
mãos dos nórdicos. Muitos varegues fizeram parte de suas numerosas
campanhas militares contra Iaropolk e contra as tribos das estepes.
Vladimir também ajudou Basílio \versal{II} cedendo guerreiros varegues. Mas o
príncipe não foi amigável com os nórdicos em todas as ocasiões. Ao
tentar obter mais aliados em sua luta contra Iaropólk, Vladimir recorreu
ao príncipe Rogvolod de Polotsk, o mesmo também sendo um varegue, e
pediu por guerreiros e pela mão de sua filha Rogneda. Esta não aceitou
pois não queria se tornar esposa de um escravo, e Vladimir então matou
Rogvolód e tomou forçosamente Rognéda como esposa. A \emph{Crônica}
também fala sobre alguns varegues que ajudaram Vladimir na campanha
contra seu irmão e que pediram por tributos. Vladimir recusou-lhes
qualquer direito a tributação e expulsou-os para Constantinopla, e ainda
enviando mensageiros dizendo para não confiar nos varegues.

\SIG{Leandro César Santana Neves}

Ver também: Crônica dos Anos Passados; Kiev; Novgorod; Olga de Kiev;
Rus; Rússia da Era Viking; Varegues.

\begin{itemize}
\item \versal{BUTLER}, Francis. \emph{Enlightener of Rus'. The Image of Vladimir
Sviatoslavich across the Centuries.} Bloomington: Slavica, 2002.

\item \versal{CROSS}, Samuel H. La tradition islandaise de saint Vladimir. \emph{Revue
des études slaves,} tomo 11, n. 3-4, 1931, pp. 133-148.

\item \versal{FRANKLIN}, Simon; \versal{SHEPARD}, Jonathan. \emph{The Emergence of Rus
750-1200.} Essex: Longman, 1996.

\item \versal{MARTIN}, Janet. \emph{Medieval Russia 980-1584.} Cambridge: Cambridge
University Press, 2007.

\item \versal{VERNADSKY}, George. \emph{Kievan Russia.} New Haven: Yale University
Press, 1972.
\end{itemize}

\chapter{W \textarn{w} \textarc{w} \textart{w}}
\section{\versal{WOLIN}}

Wolin é uma cidade situada na ilha homônima, entre os estuários dos rios
Oder e Dziwna, atualmente no noroeste da Polônia. Na Alta Idade Média a
cidade de Wolin foi um importante porto comercial do mar Báltico, além
de possuir estradas que a ligavam a outras importantes cidades
germânicas e eslavas na época. O comércio em Wolin prosperou de tal
forma que ali eram comercializadas mercadorias francas e bizantinas, 
além de moedas de prata de origem árabe.

A ilha de Wolin tornou-se um local propício
para o desenvolvimento de um núcleo urbano, além de um polo
mercantil e manufatureiro, graças, entre outros fatores, ao fato de a região ser rica em pesca e possuir muitas
fazendas. Nesse sentido, Broich e Duczko assinalam que a disponibilidade
de alimentos em Wolin possa ter sido um chamariz para que comerciantes
usassem a cidade como entreposto de rotas comerciais, ao mesmo tempo em
que feiras agrícolas possam ter se desenvolvido a ponto de atrair a
população da região para seu mercado.

Segundo John Broich, Wolin teria alcançado uma população estimada em 3
mil habitantes durante a Era Viking (séculos \versal{VIII-XI}), número alto
para a época. A cidade fazia parte da rota de importantes entrepostos
comerciais no norte da Europa, bem como Dorestad, na
Alemanha, Birka, na Suécia e Hedeby, na Dinamarca. Sua grande população
seria reflexo de sua importância como centro econômico regional, um dos
principais portos no sul do Báltico.

De acordo com Adão de Bremen, essa cidade era conhecida dos vikings pelo
nome de Jumne. Os dinamarqueses e suecos mantiveram um contato com essa
cidade por três séculos pelo menos. Adão a descreve como sendo uma
cidade próspera, grande, possuindo contatos comerciais com os saxões, os
nórdicos e os gregos (bizantinos). No entanto sua população era
predominantemente pagã ainda naqueles tempos.

Leswik Gardela comenta que se desconhece propriamente o período em que os
dinamarqueses e suecos passaram a comercializar com os habitantes de
Wolin, mas escavações arqueológicas apontam que a cidade teria sido
abandonada por volta do século \versal{VII}, voltando a ser restabelecida no
século seguinte. Por essa época o comércio escandinavo do Período Vendel
(séculos \versal{V-VIII}) estava prosperando, e os vestígios arqueológicos
atestam mercadorias de procedência eslava em território escandinavo, o
que sugere que Wolin, no século \versal{VIII}, já provavelmente mantivesse atividades comerciais
com mercadores escandinavos, mesmo que em pequena profusão.

Todavia, Wolin tornou-se um centro mercantil na Era Viking entre os
séculos \versal{IX} e \versal{X}, época em que cidades escandinavas como Birka, Hedeby, Ribe
e Kaupang estavam prosperando e a expansão nórdica já havia alcançado
todo o Arquipélago Britânico, o Mediterrâneo, Constantinopla e a Ásia.
Assim, longas rotas comerciais haviam sido desenvolvidas e
mercadorias vindas de muito longe começavam a chegar a Escandinávia, e
Wolin, devido a sua proximidade com a Dinamarca, tornou-se um dos postos
de parada importante dos mercadores que vinham do Leste Europeu.

Ao longo do século \versal{IX}, como sugere John Broich, a cidade vivenciou uma
massiva expansão urbana. Seu porto foi ampliado e novas residências foram
erguidas seguindo o que parece ter sido um projeto urbanístico, pois as
casas foram organizadas em blocos de quatro, nos quais cada construção
media entre 5 a 6 metros de extensão. Esse novo bairro foi erguido sobre
a Colina de Prata (Silver Hill). Não obstante, algumas ruas desses
blocos residenciais eram pavimentadas com troncos de carvalho, o que
sugere a riqueza da região, possivelmente se tratando de um bairro de
condições econômicas elevadas.

O crescimento urbano levou ao surgimento de um subúrbio situado na zona
sul da cidade. Mas além da construção de casas, foram construídos também
lojas e armazéns. Data do século~\versal{X}, de acordo com Katherine Holman, o
desenvolvimento de uma produção artesanal bastante prolífica na cidade.
Vestígios arqueológicos encontrados nas escavações apontam trabalhos
manufatureiros no campo da metalurgia, ourivesaria, carpintaria etc.,
tendo-se encontrado objetos de madeira, osso, âmbar, cerâmica, ouro,
prata, ferro, entre outros.

Devido a essa prosperidade, a cidade tornou-se alvo de piratas, de modo que 
os muros foram constantemente reforçados e ampliados. Além disso,
uma cadeia de fortes foi erguida ao longo do rio
Dziwna, para evitar que navios tentassem sitiar a cidade por via
fluvial.

No século~\versal{X}, de acordo com Adão de Bremen, o rei Haroldo Dente Azul da
Dinamarca teria imposto seu governo a Wolin, o que incluiu mais incentivo
comercial e a construção de uma fortaleza. Embora não se tenha certeza
se o rei Haroldo realmente governou de alguma forma a política de
Wolin, sabe-se que, por volta de 985-986, quando foi deposto por seu filho
Sueno Barba-bifurcada, o ex-monarca exilou-se em Jumne (Wolin), onde
teria morrido por volta de 987.

Wolin também é conhecida na literatura nórdica antiga por supostamente
ser a sede de um grupo de mercenários vikings chamados Jomsvikings, os
quais teriam atuado no século~\versal{X}. Segundo a \emph{Jómsvíkinga saga} (c.
1200), esses mercenários viveriam no forte Jómsborg, possivelmente
situado na ilha de Wolin, tendo lutado durante os reinados de Haroldo
Dente Azul e seu filho Sueno Barba-bifurcada. Possuiriam um estrito
código de lealdade e seriam bravos guerreiros.

No século~\versal{XI} a cidade começou a entrar em crise. John Broich aponta que
o desgaste do meio ambiente teria sido um dos fatores que comprometeram
o crescimento da cidade, pois naquele tempo a população de Wolin era
estimada em 8 mil habitantes e o número de árvores disponíveis na ilha
havia caído drasticamente, a ponto de gerar escassez de lenha e
matéria-prima. Não obstante, apesar das fortificações erguidas ao longo
dos séculos \versal{IX} e \versal{X}, no ano de 1046, Magno, o Bom, à época rei da Noruega, 
ordenou a invasão e saque de Wolin. Desde o ano de 1043, Magno
vinha realizando campanhas de pirataria no Báltico, e a captura e saque de
Wolin foi o grande feito dessas campanhas. Vestígios arqueológicos
apontam que parte da cidade e seus templos foram incendiados.

Atualmente em Wolin é celebrado anualmente, há mais de vinte anos, o
\emph{Festiwalu Slowian i Wikingow} (Festival Eslavos e Vikings), que 
consiste no evento de reconstitucionismo histórico (\emph{living history})
de temática viking e eslava mais importante da Polônia e um dos mais
reconhecidos da Europa.

\SIG{Leandro Vilar Oliveira}

Ver também Comércio; Rus; Rússia da Era Viking.

\begin{itemize}
\item \versal{BREMEN}, Adão de. \emph{History of the archibishops of Hamburg-Bremen}.
New York: Columbia University Press, 1959.

\item \versal{BROICH}, John. The Wasting of Wolin: Environmental Factors in the
Downfall of a Medieval Baltic Town. \emph{Environmental and History},
vol. 7, n. 2, special issue, 2001, pp. 187-199.

\item \versal{DUCZKO}, Wladyslaw. Viking-Age Wolin (Wollin) in the Norse context of the
Southern Coast of the Baltic Sea. \emph{Scripta Islandica}, n. 65, 2014,
pp. 143-151.

\item \versal{GARDELA}, Leszek. Vikings in Poland. A critical overwiew. In: \versal{ERIKSEN},
Marianne Hem \emph{et al}. (eds.). \emph{Viking Worlds: things,
spaces and movement}. Oxford: Oxbow Books, 2014, pp. 213-234.

\item \versal{HOLMAN}, Katherine. \emph{Historical dictionary of the vikings}. Lanham:
Scarecrow Press Inc, 2003.
\end{itemize}
