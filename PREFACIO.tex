\chapter[Prefácio, \emph{por Neil Price}]{Prefácio}\label{prefuxe1cio}

\epigraph{\emph{%
Deyr fé, deyia frændr, deyr siálfr it sama, en orztírr deyr aldregi
hveim er sér góðan getr.}\linebreak
(O gado morre, os parentes morrem, eu vou morrer do mesmo jeito, mas a
fama não morre para os que têm um bom nome.)}{\versal{Hálvamál}, 76}

Essas linhas, referentes à estrofe 76 do poema em nórdico antigo
\emph{Hávamál} ou ``As palavras do altíssimo'' (aqui numa tradução de
Johnni Langer), estão entre as mais famosas que sobreviveram até nós
provindas da Era Viking. Reproduzida em inúmeros livros, exposições e
até mesmo ressignificada como aconselhamento empresarial, elas falam de
uma preocupação permanente que também percebemos claramente nos
epitáfios das pedras rúnicas contemporâneas e nas grandes sagas
islandesas medievais e épicos heroicos: o desejo de nunca ser esquecido
e também de ser sempre lembrado.

Os Vikings nunca poderiam ter imaginado que iriam fazer tanto sucesso,
nem em seus sonhos. Mil anos mais tarde eles são conhecidos em
continentes do qual não nem sabiam que existia, e literalmente milhões
de pessoas avidamente seguem dramas televisivos recontando suas
aventuras. Eles aparecem em histórias em quadrinhos e filmes, foram
reapropriados em inúmeras marcas pelo mundo. Não é exagero dizer que
onde as línguas germânicas são faladas, os Vikings diariamente seguem
firmes na imaginação, visto que seus deuses emprestaram os nomes para os
dias da semana. Para as pessoas de qualquer parte, os Vikings são
associados com a ideia do tornar-se berserk e com o selvagem, mas também
todos nós entendemos o que significa um ``funeral viking''. Equipes de
futebol e navios de guerra levam seus nomes, assim como as sondas que
saíram pelo Sistema Solar: os Vikings são arquétipos de guerreiros,
exploradores e viajantes. Em países a milhares de quilômetros da
Escandinávia, os entusiastas recriam as vestimentas e a cultura material
dos Vikings com extraordinária facilidade, e ainda as usam com orgulho,
mesmo que às vezes tenham que viver em réplicas de habitações da Idade
do Ferro.

A memória de nenhuma outra cultura antiga tem sobrevivido neste nível de
engajamento público, com uma série de respostas emocionais.

Mas se ``todos'' estão familiarizados com os Vikings, pelo menos até
certo ponto, podemos razoavelmente perguntar apenas o que
\emph{exatamente} eles conhecem. Além desta popularidade, os detalhes --
o que há de melhor -- da Era Viking e suas pessoas ainda escapam dos
estudiosos, quanto mais de um público maior. Nem sabemos como devemos
denominá-los. Ao contrário do que é comumente utilizado na maioria das
línguas modernas fora dos países nórdicos, ``Viking'' não é e nunca foi
um termo genérico e não pode ser aplicado para toda a população que
vivia ``ali'', ``naqueles dias''. Um \emph{víkingr} ``original'' era um
tipo especial de pessoa, frequentemente (mas nem sempre) um homem, com
vínculo temporário ou permanente a um tipo de vida marítima e violenta;
em certo sentido, um pirata. Alguém pode tornar-se um Viking, depois
parar e fazer outra coisa, talvez volte a essa vida novamente ou não.
Poderia ser Viking ao mesmo tempo em que faz muitas outras coisas, cada
um com a sua própria escala de identidade e comunidade. A maioria das
pessoas que viveram na Escandinávia nunca foram Vikings e provavelmente
não queriam nada com eles. Então qual nome devemos usar? A geografia não
ajuda. Durante os séculos sétimo e décimo de nossa Era (um conceito que
eles não reconheceriam) não havia noção de Escandinávia e na maioria das
vezes nenhuma Noruega, Suécia ou Dinamarca -- ou pelo menos os
territórios que conhecemos com esse nome. ``Nórdico'' é tanto
androcêntrico quanto intrinsecamente ligado ao Ocidente. As
terminologias mudam com a linguagem e também na tradução -- do inglês em
que escrevo para o português em que este prefácio aparecerá. Precisamos
de nomes para abranger tudo e na falta de algo melhor (sobrecarregado
com o peso da tradição), ``Viking'' é o que temos\ldots{}

Para percorrer o mundo destes Vikings, de todos os tipos, temos que
entrar em locais tanto familiares como muito diferentes. Realmente
encontramos todas as coisas (exceto o capacete de chifres) que compõem
seu estereótipo -- as invasões, os túmulos de navios ardentes, as
viagens marítimas, as novas descobertas e explorações, os homens
valentes e as mulheres das lendas -- mas também os descobrimos
subvertidos e matizados a cada passo\ldots{} E acima de tudo conhecemos
pessoas que estavam incrivelmente curiosas sobre o seu mundo, que
mudaram e foram alterados por ele, com legados ainda percebidos hoje em
dia. Mas principalmente, vê-los filtrados através de uma visão de mundo
não cristã, com um sentido alterado da realidade, completamente
diferente de qualquer coisa da Terra atualmente.

Mas para fazer tudo isso é preciso um guia.

Há muitos livros populares sobre os Vikings, mostrando sua gloriosa arte
e cultura material, bem como muitas sínteses e catálogos de exposições
concedendo uma visão geral da cronologia. Mas um dicionário é algo
diferente -- uma fonte e um recurso, um lugar para procurar o que você
quer saber ou seguir um rastro de informações para ver onde ele conduz.
Se você é um acadêmico que quer uma referência acadêmica ou um leigo
interessado que deseja descobrir quem Ragnar Lodbrok realmente foi, esse
livro foi feito para você. Os verbetes reunidos aqui levam o leitor a
uma excursão mais abrangente do mundo Viking do que esbocei acima, mas
também para além do seu impacto contínuo em nossas vidas hoje em dia --
variando em todo o mundo e em diferentes mídias, da literatura romântica
aos filmes.

Este volume é também um marco de tipo diferente, em que afirma as ativas
tradições acadêmicas dos estudos Vikings na América Latina. Talvez
inevitável, a maioria das pesquisas sobre os Vikings tem um pesado sabor
euro-escandinavo, mas não há periferias no estudo do passado. Este
dicionário é uma
homenagem ao
trabalho de seu editor, mas também para todos os seus colegas que
moldaram um ambiente de pesquisa tão gratificante e estimulante para os
estudos escandinavos no Brasil e cercanias. \emph{Miðgarðr} foi maior do
que os Vikings tinham percebido, mas em livros como este, eles
finalmente viajaram para todo lugar.\medskip

\EP[2]
\hfill Professor Dr.\,Neil Price

\hfill Dep.\,Arqueologia e História Antiga 

\hfill (Universidade de Uppsala, Suécia)
