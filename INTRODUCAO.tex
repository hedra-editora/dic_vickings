\chapter[Introdução, \emph{por Johnni Langer}]{Introdução}\label{introd}

Os vikings ocupam um lugar especial no imaginário do Ocidente. Desde o
século XIX eles fazem parte das artes plásticas, da música, das
identidades nacionais. Em nossa época, eles são constante presença em
filmes e televisão e fazem muito sucesso na mídia e cultura popular. A
história de suas viagens e conquistas ainda é pouco divulgada em termos
acadêmicos pelos países neolatinos. Assim, a falta de maiores
bibliografias especializadas sobre o tema da Escandinávia da Era Viking
em língua portuguesa justifica a publicação do presente livro. Ele foi
escrito com auxílio de diversos pesquisadores e especialistas,
brasileiros e estrangeiros, mas essencialmente por componentes do
\emph{Núcleo de Estudos Vikings e Escandinavos}. Criado em 2010, o NEVE
vem dedicando-se ao estudo acadêmico e a popularização científica da
História, cultura, sociedade e religiosidade nórdica durante o período
medieval. Em 2015 foi publicado pela editora Hedra a primeira grande
sistematização sobre o mundo escandinavo em linguagem portuguesa, o
\emph{Dicionário de Mitologia Nórdica}, a partir da qual retomamos
diversos conceitos e experiências para compor o presente livro.

A principal meta do presente \emph{Dicionário de História e Cultura da
Era Viking} é proporcionar referenciais de conteúdo para todos os
interessados no tema, sejam estudantes, pesquisadores ou apenas
entusiastas. Cada verbete foi escrito visando suas particularidades
dentro do mundo nórdico medieval e também suas conexões, via remissões.
Ao final de cada verbete, são indicadas referências bibliográficas para
que o leitor possa se aprofundar nos assuntos indicados.

A estrutura geral do Dicionário foi baseada especialmente nos livros
\emph{Encyclopaedia of the Viking Age} (John Haywood); \emph{Historical
Dictionary of the Vikings} (Katherine Holman) e \emph{Medieval
Scandinavia: An Encyclopedia} (Phillip Pulsiano), mas formulado para
atender as necessidades de informações de estudantes e pesquisadores do
Brasil. O eixo básico do livro foi estruturado na denominada Era Viking
(tradicionalmente localizada entre os séculos VIII ao XI d.C.), mas
contendo vários verbetes com conteúdo anterior e posterior a este
período. Algumas entradas têm relação indireta com o recorte, como temas
envolvendo leis, literatura e localidades, sendo vinculados mais
objetivamente com a Escandinávia Medieval, dentro dos critérios da
Arqueologia escandinava (que separa o período Viking (considerado Idade
do Ferro Tardia), da Idade Média, que se inicia após o século XI).
Assim, nosso viés básico de periodização é o da historiografia francesa,
que entende o período abrangido pela Era Viking como sendo Alta Idade
Média. Maiores detalhes historiográficos são definidos no verbete
\emph{Era Viking}.

Para facilitar a leitura, optamos por simplificar e transliterar muitas
das grafias do nórdico antigo para o português moderno. A letra
\emph{þorn} (\textbf{Þ}) foi substituída pelo th, como em Thor. A letra
\textbf{ð} foi transliterada para d, como em Odin. Os símbolos
\textbf{œ} e \textbf{æ} foram omitidos em grande parte dos casos, bem
como o acento agudo em vogais e o \textbf{r} final de nominativos, como
Auðr, Ragnhildr, Leifr. Em alguns casos específicos, como citação de
fontes primárias, conservamos a grafia original, tanto em nórdico antigo
como em latim e outras linguagens medievais. Alguns critérios mais
detalhados são esclarecidos nos verbetes \emph{Linguagem} e
\emph{Norreno}.

Para o conceito de viking o presente livro adota tanto sua relação com o
referencial ocupacional (pirata, navegador, comerciante), quanto de
identidade étnica, dependendo do contexto, seja para com fontes
primárias ou com ressignificações imaginárias no mundo moderno. Os
principais norteadores são as considerações teóricas e historiográficas
definidas no verbete \emph{Viking}, mas também tratadas em outros
momentos, como \emph{Vikings na literatura}, \emph{Vikings no Brasil},
\emph{Vikings na música}, etc.

A península da Escandinávia é concebida tanto dentro de critérios
geográfico quanto de referenciais históricos e culturais, bem como o
conceito de Norte, povos nórdicos e escandinavos -- que possuem relação
direta com o imaginário sobre os vikings, desenvolvido a partir do
romantismo oitocentista e popularizado no século XX em diante. Isso é
tratado em detalhes no verbete \emph{Escandinávia}.

A metodologia básica utilizada pela maioria dos colaboradores proveio da
História Cultural, mas também foram utilizadas as metodologias e
referenciais teóricos da teoria literária, História do imaginário social
e História das Ideias. A bibliografia concentrou-se tanto na leitura de
clássicos da historiografia escandinava, como Johannes Brøndsted, Gwyn
Jones, James Graham-Campbell, como em autores da nova geração, a exemplo
de Neil Price, Stefan Brink, Leszek Gardela, entre outros. A consulta
direta a pesquisadores internacionais também foi essencial em diversos
momentos da pesquisa, a qual referenciamos nos agradecimentos. A equipe
de modo geral empregou também dissertações e teses, artigos, banco de
dados, fontes primárias disponíveis eletronicamente e outros recursos em
diversas linguagens.

Visando atender especialmente aos estudantes de História e
historiadores, disponibilizamos diversas entradas para fontes primárias,
produzidas tanto em nórdico antigo quanto latim, árabe e outras
linguagens, com o intuito de proporcionar um primeiro contato do
pesquisador com documentos fundamentais para a reconstituição do passado
nórdico. A maioria deles se encontra referenciada no verbete remissivo
\emph{Fontes primárias}.

A tônica principal do livro é referente a conteúdos de História, as
principais personalidades históricas, acontecimentos relevantes,
batalhas, armamentos, localidades e cidades, regiões, povos e etnias,
aspectos sociais, cultura material, linguagem e literatura e
historiografia. O presente livro deixou de lado quase todos os aspectos
relacionados com mitos e religiosidades (deidades, cultos, narrativas,
símbolos), visto que foram detalhados no \emph{Dicionário de Mitologia
Nórdica} (Hedra, 2015). Apena alguns verbetes foram conservados
enfocando o tema, como \emph{Religião} e \emph{Simbolismo animal},
atualizando alguns dos aspectos mais recentes das pesquisas de
Arqueologia da Religião Nórdica Antiga. A presente obra também aborda
diversas entradas relativas a sagas islandesas, mas deixamos de lado
algumas que já tiveram conteúdo publicado no \emph{Dicionário de
Mitologia Nórdica}, como Saga de Frithiof, Saga de Hjalmthér e Saga dos
Volsungos.

Não podemos deixar de agradecer ao trabalho do professor Dr. Guilherme
Queiroz de Souza, que além de elaborar alguns verbetes, também foi
responsável pela revisão do Dicionário. Aos membros do NEVE pela
dedicação e empenho na divulgação da Escandinavística brasileira e na
elaboração básica do livro. À equipe da editora Hedra pelo empenho
editorial e por abrir espaço ao tema em nosso país.

Um agradecimento especial a todos os acadêmicos estrangeiros que
auxiliaram em informações para os verbetes: Neil Price (Universidade de
Uppsala/Suécia, e também pela gentileza em escrever o prefácio do
Dicionário), Leszek Gardela (Universidade de Rzeszów/Polônia), Lars Boje
Mortensen (Universidade do sul da Dinamarca), Michèle Hayeur Smith
(Universidade Brown/Estados Unidos), Marianne Tóvinnukona (Universidade
da Islândia), Gísli Sigurðsson (Universidade da Islândia), Thomas A.
Dubois (Universidade da Pensilvânia/Estados Unidos), Charlotte
Hedenstierna-Jonson (Universidade de Uppsala/Suécia), Aleksander
Pluskowski (Universidade de Reading/Inglaterra), Eldar Heide (Bergen
University College, Noruega), Kevin Smith (Universidade Brown/Estados
Unidos), Regina Jucknies (Universidade de Colônia/Alemanha), Jenn Culler
(Estados Unidos), Gail Kellogg Hope (Estados Unidos), Som När Det Begav
Sig (Suécia), Margo Farnsworth (Islândia), Auður Hildur Hákonardóttir
(Islândia), Maria Tóvinnukona (Islândia) e Daniel Serra (Suécia).

No momento da edição final da presente obra, ocorreu o falecimento do
historiador Régis Boyer, o maior nome da escandinavística francesa. Fica
aqui a nossa homenagem a esse importante acadêmico, a maior influência
bibliográfica nos primeiros estudos da área efetuados em nosso país, ao
final da década de 1990 a meados dos anos 2000 e também citado em
diversos verbetes ao longo do presente livro. Dedicamos, desta maneira,
o Dicionário a essa figura excepcional nos estudos e na divulgação da
história e cultura da Era Viking.\medskip

\hfill João Pessoa, 31 de junho de 2017.

\hfill Prof. Dr. Johnni Langer, Universidade Federal da Paraíba

\hfill \emph{Núcleo de Estudos Vikings e Escandinavos}
