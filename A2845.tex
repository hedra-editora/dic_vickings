
\renewcommand*{\chapnumfont}{\normalfont\HUGE\bfseries\sffamily}
\renewcommand*{\chaptitlefont}{\normalfont\fontsize{50}{60}\selectfont\bfseries\sffamily}

\chapterspecial{A}{}{}

\openany

\section{\versal{ADÃO} \versal{DE} \versal{BREMEN}}

Adão de Bremen (em alemão, Adam Von Bremen; em latim, Adamus Bremensis;
\emph{ca}. 1045--1081--85) foi um dos mais famosos e importantes
cronistas alemães da Alta Idade Média. É~conhecido pelo livro
\emph{Gesta Hammaburgensis Ecclesiae Pontificum} (lat.: ``História dos
Arcebispos de Hamburg"-Bremen''), dedicado ao arcebispo Liemar
(1072--1101). É~possível que Adão tenha composto sua obra em 1075, e é
bastante provável que a mesma tenha sido revista e atualizada até
\emph{ca}. 1081.

Pouca informação sobre o autor pode ser extraída a partir do texto da
\emph{Gesta}. De si mesmo, Adão apenas nos diz ser ``\emph{minimus
sanctae Bremensis ecclesiae canonicus}'' (lat.: ``o menor de todos os
cônegos da santa Igreja de Bremen'') e um ``prosélito e estrangeiro''
(``\emph{proselitus et advena}''). Existe uma conjectura de que Adão era
original da Saxônia, com base em traços dialetais contidos no seu texto
latino. Ele foi convidado pelo arcebispo Adalberto para atuar na igreja
de Bremen em \emph{ca}. 1067--68, e, numa passagem do epílogo da
\emph{Gesta} e por uma carta de junho de 1069, sabe"-se que foi feito
\emph{magister scholarum} ainda jovem. No Livro I-14 da \emph{Chronica
Slavorum} de Helmold de Bosau (\emph{ca}. 1120--1177), existe uma
referência ao \emph{magister} Adam, que era o mais eloquente escritor da
Igreja de Hamburgo e Bremen, além de inúmeras outras citações do texto
da \emph{Gesta}. Pode"-se, portanto, inferir que Adão vivia e participava
da comunidade eclesiástica de Bremen como membro do cabido da catedral e
que o título de \emph{magister} atribuído a ele pela \emph{Chronica
Slavorum} indica"-nos mais que uma deferência do autor, mas sim, de fato,
o exercício do magistério junto à comunidade capitular da catedral.

A arquidiocese de Bremen viveu seus momentos de apogeu na Alta Idade
Média sob os arcebispos Unwan (1031--1029), Adalbrand (1035--1043) e,
especialmente, Adalberto (1043--1072), o influente conselheiro do
imperador Henrique \versal{III}. Sob o primeiro, houve uma considerável expansão
territorial da sede; sob Adalbrand, sabe"-se que a cidade de Bremen, a
catedral e boa parte da biblioteca capitular foram destruídas por um
incêndio, em 1041, e que o arcebispo iniciou longos trabalhos de
reconstrução da igreja; durante o arquiepiscopado do terceiro, a
jurisdição expandiu"-se ainda mais, tornando parte da Igreja de Bremen as
dioceses de Mecklenburgo, Oldenburgo e Ratzeburg, além das já
sufragâneas dioceses escandinavas. Adalberto tinha ambiciosos planos
para tornar a Igreja de Bremen um patriarcado para as dioceses do norte
da Europa. Contudo, esses planos foram frustrados pela morte do papa
Clemente \versal{II} -- que ele ajudou a escolher --, em 1054, e pela morte do
imperador -- sobre o qual tinha tão grande influência --, em 1056; de
sorte que comunidades cristãs dos territórios setentrionais, em 1103,
passaram à recém"-criada arquidiocese de Lund, quando o bispo local
tornou"-se metropolita de toda a Escandinávia.

É possível que a obra de Adão tenha sido, em certa medida, uma tentativa
de recuperar parte da memória perdida da arquidiocese, que teve boa
parte de sua biblioteca consumida pelo fogo, além de servir como
explícita propaganda em prol das ambições da Igreja de Bremen. A
\emph{Gesta} é, sobretudo, um trabalho de história missionária e
propagandística. Para Adão, a missão da Igreja de Hamburgo"-Bremen junto
aos povos não cristãos (\emph{legatio gentium}), especialmente do norte
da Europa, é um direito natural; o \emph{primum officium} da
arquidiocese é alargar o reino de Cristo, conforme ele também menciona
no prólogo (``\emph{quos per totam septentrionis latitudinem suae
legationi cotidie videt accrescere''}).

Adão escreve a partir de Bremen, no melhor gênero \emph{gesta
episcoporum}, sobre as regiões eslavas, sobre a Saxônia e sobre a
Escandinávia. O~livro é uma história das origens da arquidiocese de
Hamburgo e Bremen até o tempo do autor, além de fornecer uma visão ampla
da expansão do cristianismo nas regiões centro"-leste e norte da Europa
do século \versal{IX} ao \versal{XI}. O~texto inicia"-se com a conquista da Saxônia por
Carlos Magno, passando pela relação dos saxões com os dinamarqueses e
eslavos, até a derrocada política do arcebispo Adalberto, em 1066, junto
ao imperador Henrique \versal{IV} (1056--1106), e sua morte, em 1072. A~narrativa
termina com a consagração do novo arcebispo, Liemar (1073--1101), e
contém quatro livros. O~prólogo é escrito de acordo com as convenções
clássicas, e Adão faz referência a fontes escritas e orais. O~primeiro
livro trata da história das guerras de conquista dos saxões (772--804),
da fundação da sé de Bremen (787) e Hamburgo (831), bem como das
primeiras missões rumo às terras do norte da Europa e dos ataques
vikings. No segundo livro, o autor aborda as missões junto aos
dinamarqueses, suecos, noruegueses e eslavos, além de apontamentos
políticos da história germânica de \emph{ca}. 945 até \emph{ca}. 1045. O~terceiro 
livro é dedicado inteiramente ao período do arcebispo
Adalberto. O~quarto livro proporciona um relato etnográfico e geográfico
sobre ``as ilhas do norte'', \emph{i.e.} a Escandinávia. Ele mesmo nunca
visitara as regiões escandinavas a que faz menção, mas usa o encontro,
em \emph{ca}. 1070, com o rei dinamarquês Sven \versal{II} (1047--1076) e as
informações que dele recebeu em seu livro. Adão diz que as informações
sobre Dinamarca, Islândia, Vinland, Suécia e regiões ao redor do mar
Báltico são baseadas no relato do rei Sven.
\SIG{Rodrigo Mourão Marttie}

Ver também Cristianização da Escandinávia; Saxo Grammaticus; Snorri
Sturluson; Templo de Uppsala.



\begin{itemize}\footnotesize{}
\item
  \versal{ASMUSSEN}, Jacob. \emph{De fontibus Adami Bremensis commentationem
  \redondo{[…]}}. Kiel: Kiliae, Mohr. 1834.
\item
  \versal{BRUGNOLI}, Giorgio. ``Modelli Classici in Adam di Bremen'', \emph{in}  \versal{SANTINI}, Carlo (ed.). \emph{Tra testo e
  contesto. Studi di Scandinavistica medievale} -- \versal{I} Convegni di
  Classiconorroena 2. Roma: Calamo, 1994. pp.~5--12.
\item
  \versal{COIT}, Daniel (ed.) et alii. \emph{The New International
  Encyclopaedia}. Nova York: Dodd, 1905.
\item
  \versal{DIETSCH}, Walter. \emph{Cathedral of St. Peter, Bremen}. 
  Bremen: Carl Schünemann, 1960.
\item
  \versal{GOETZ}, Hans"-Werner. ``Constructing the Past. Religious Dimensions and
  Historical Consciousness in Adam of Bremen's Gesta Hammaburgensis
  Ecclesiae Pontificum'', \emph{in}  \versal{MORTENSEN}, Lars Boje (ed.). \emph{The Making
  of Christian Myths in the Periphery of Latin Christendom (c.
  1000--1300)}. Copenhague: Museum Tusculanum Press, 2006. pp.~17--52.
\item
  \versal{SCHMEIDLER}, Bernhard. \emph{Hamburg"-Bremen und nordost"-Europe vom 9.
  bis 11. jahrhundert}. Leipzig: Dieterich, 1918.
\end{itemize}

\section{\versal{AEGIR}}

O deus do mar adquire o seu nome do substantivo \emph{aegir} -- em
nórdico antigo, oceano ou mar, usado na poesia escáldica, isto é,
estamos ante a personificação do mar. Na primeira parte da
\emph{Orkneyinga saga} (Saga dos habitantes de Orkney), \emph{Fundinn
Noregr} (Fundação da Noruega), está escrito que um rei do norte da
Noruega chamado Fornjót teve três filhos: Hlér, Logi e Kári. Como
\emph{aegir}, \emph{hlér} é um substantivo que significa mar,
\emph{logi} significa fogo e \emph{kári} está incluído nos \emph{thulur}
ou nomes poéticos para vento. Na seção \emph{Skáldskaparmál} da
\emph{Edda} de Snorri Sturluson, Aegir também é identificado como um
\emph{jötunn}, especificamente como o gigante do mar Hlér, que habitou
na ilha que agora se chama Hlésey, e que estava profundamente versado na
magia negra, o que é confirmado por diferentes metáforas ou
\emph{kenningar}. No entanto, outros acreditam que Aegir era um dos
deuses primordiais, antecipando a existência dos \emph{Aesir} (Ases), os
\emph{Vanir} (Vanes), \emph{jötnar} (gigantes), \emph{álfar} (elfos) e
\emph{dvergar} (anões).

Na realidade, a nossa principal fonte sobre Aegir é justamente
\emph{Skáldskaparmál}, que consiste em um diálogo entre Aegir e Bragi, o
deus da poesia, acerca de um grande banquete onde, segundo o relato de
Snorri, os \emph{aesir} se reuniram. Quando Bragi chega ao ponto de
descrever as metáforas para o mar, podemos inferir o nome da esposa de
Aegir através de ``marido de Rán''. Snorri também oferece os nomes das
nove filhas que Aegir tem: Himinglaeva, Dúfa, Blódughadda, Hefring, Udr,
Hrönn, Bylgja, Bára, Kolga. Aqui encontramos alguns \emph{kenningar}: as
``filhas de Aegir'' são as ondas do mar, de fato a maioria dos nomes
delas realmente significa onda; ``fogo de Aegir'' é o ouro; ``cavalo de
Aegir'' é um barco etc. Posteriormente, Snorri, na voz de Bragi, escreve
que Aegir e o gigante Gymir são ambos o mesmo. Sabemos que essa
identificação não corresponde com outro gigante chamado Gymir, descrito
nos poemas éddicos (\emph{Skírnismál} e \emph{Hyndluljód}) como marido
de Auboda, e cuja filha, Gerd, se casou com o deus Frey.

Nos poemas éddicos, Aegir é habitualmente anfitrião para os deuses. As
qualidades de Aegir como anfitrião são mencionadas por Odin diante do
rei Geirröd no poema éddico \emph{Grímnismál}. Em \emph{Hymiskvida} os
deuses vão visitar Aegir, e eles precisam de um enorme caldeirão para
preparar a cerveja que será consumida. O~poema conta como Thor adquire o
caldeirão do gigante Hymir. Em \emph{Lokasenna}, Aegir oferece uma festa
para os deuses em seu salão, onde ele fornece cerveja fabricada em um
caldeirão enorme. Durante a festa, um dos servos de Aegir, Fimafeng
(Eldir é o outro), é morto por Loki, que é expulso da festa. Mas Loki
retorna impetuoso, e é neste ponto que Thor põe fim ao frenesi de
insultos de Loki, ameaçando"-o com seu martelo Mjollnir. Dado que o
relato gira em torno da festa, o poema chama"-se também
\emph{Aegisdrekka}, festa de bebida de Aegir, nos manuscritos. Em
relação ao culto, evidentemente o mar desempenhou, e ainda desempenha,
um papel muito importante na sociedade escandinava, mas o mar foi um
elemento temido e respeitado ao mesmo tempo. Os deuses do mar
(\emph{Saekonungar}), Aegir e sua esposa, Rán, foram protetores e
patronos dos marinheiros e exploradores.
\SIG{Carlos Osvaldo Rocha}

Ver também Mitologia Escandinava; Njord.



\begin{itemize}\footnotesize
\item
  \versal{LINDOW}, John. \emph{Norse Mythology: a guide to the Gods, heroes,
  rituals, and beliefs}. Oxford/Nova York: Oxford University Press, 2002.
\item
  \versal{SIMEK}, Rudolf. \emph{Dictionary of Northern Mythology}. Trad. Angela
  Hall. Cambridge: Brewer, 1993.
\end{itemize}

\section{\versal{AEGISHJÁLMUR}}

Ver Símbolos rúnicos.

\section{\versal{ÁGUIA}}

A águia é um animal muito presente na literatura e nas mitologias do
medievo europeu, geralmente simbolizada como mensageira dos deuses e do
fogo celeste, mas também acompanhando grandes heróis. Enquanto
substituto do Sol em várias religiosidades euro"-asiáticas, foi um animal
amplamente utilizado na heráldica e nas representações de realeza e
nobreza.

Na mitologia nórdica a águia foi representada no topo da árvore
Yggdrasill (\emph{Grímnismál} 31), inimiga e oposta a uma
serpente"-dragão em sua base. A~imagem de uma árvore cósmica cujo topo é
habitado por um pássaro e em sua base/raiz por uma serpente ou dragão é
comum a diversos povos espalhados pelo mundo, da Eurásia à América
pré"-colombiana, e, como nas fontes nórdicas, ambos os animais que a
habitam são inimigos, sendo a serpente assimilada à terra, e a ave, ao
céu. Como no caso escandinavo, a forma mais usual do pássaro inimigo da
serpente nos diversos mitos euro"-asiáticos é a de uma águia, cuja
representação pode ser observada na pedra rúnica de
\emph{Ramsundsberget} (\emph{Sö} 101). Isso talvez tenha sido influenciado
também pela observação na área nórdica da constelação do Cisne
(geralmente em posição elevada no céu) em contraposição à constelação de
Escorpião (visível na linha do horizonte), reforçando a dicotomia
pássaro"-serpente no imaginário escandinavo.

\imagemgrande{Pedra de Lärbro, Gotland, séc.~\versal{VIIIi}}{./img/18.png}                               % {ÁGUIA}}

Outra referência da águia na mitologia escandinava refere"-se ao deus
Odin, especialmente no mito do roubo do hidromel (\emph{Skáldskaparmál}
1; \emph{Hávamál} 104--110). Após fugir com o precioso líquido da
montanha Hnibjorg, tanto Odin quanto seu perseguidor (o gigante
Suttungr) transformam"-se em águias. Segundo Jens Peter Schjødt, esse
episódio também deve ser percebido em termos de dicotomia: a serpente
(forma animal que o deus adotou para adentrar Hnibjorg) representa o
ctônico, a terra, o baixo, o submundo, enquanto a águia é a esfera
celeste, o alto, está relacionada a Asgard. Esse simbolismo de oposição
estaria relacionado aos rituais de iniciação, onde a visita ao submundo
para obter algum tipo de conhecimento ou bem precioso faz parte da
cosmovisão e da religiosidade pré"-cristãs.

Segundo Hilda Davidson, a jornada de Odin transmutado em águia também é
mencionada em poemas arcaicos nórdicos e relaciona"-se a suas
características xamânicas em busca de conhecimento. A~pesquisadora ainda
lembra que tanto Odin quanto o deus irlandês Lug estão relacionados a
águias e pássaros em geral, sendo um símbolo celeste e de soberania
devido a sua associação com os imperadores romanos.

O simbolismo da águia também era refletido na religiosidade nórdica.
Segundo Catharina Raudvere, a \emph{fylgja} de pessoas ou famílias
muitas vezes era representada por águias, enquanto a alma (\emph{hugr}
ou \emph{hamr}) adotava temporariamente essa forma animal revelando a
origem nobre (status moral) da pessoa.

As representações imagéticas de águias durante o período de migração até
a Era Viking são muito variadas, sendo compostas por imagens em
bracteados, esculturas, pingentes e pedras rúnicas. As duas pedras
pintadas de Gotland mais famosas envolvendo águias são Hammar \versal{I} e Stora
Hammar \versal{III}, ambas relacionadas aos simbolismos e mitos odínicos. 
A~primeira contém mais referências religiosas, sendo o animal relacionado
a práticas de sacrifícios humanos, enquanto a segunda está conectada ao
mito do roubo do hidromel por Odin.
\SIG{Johnni Langer}

Ver também Águia de sangue; Alma e espiritualidade; Animais totêmicos;
Hammar \versal{I}; Hugin e Munin; Odin.



\begin{itemize}\footnotesize
\item
  \versal{GRÄSLUND}, Anne"-Sophie. ``Wolves, serpents, and birds: their symbolism
  meaning in Old Norse beliefs'', \emph{in}  \versal{ANDRÉN}, Anders et alii
  (orgs). \emph{Old Norse religion in long"-term perspectives}. Lund:
  Nordic Academic Press, 2004, pp.~124--29.
\item
  \versal{DAVIDSON}, Hilda. \emph{Myths and Symbols in Pagan Europe: Early
  Scandinavian and Celtic Religions}. Manchester: Manchester University
  Press, 1988, pp.~91, 129, 175.
\item
  \versal{JESCH}, Judith. ``Eagles, raven and wolves: beasts of battle, symbols of
  victory and death'', \emph{in}  \versal{JESCH}, Judith (Ed.). \emph{The Scandinaves:
  from the Vendel Period to the Tenth Century, an ethnographic
  perspective}. Nova York: Boydell Press, 2002, pp.~251--71.
\item
  \versal{LANGER}, Johnni. ``O~céu dos vikings: uma interpretação etnoastronômica
  da pedra rúnica de Ockelbo (Gs 19)''. \emph{Domínios da Imagem} 6(12),
  2013, pp.~97--112.
\item
  \versal{SCHJØDT}, Jens Peter. \emph{Initiation between two worlds: structure
  and symbolism in pre"-Christian scandinavian religion}. Odense: The
  University Press of Southern Denmark, 2008, pp.~163--67.
\end{itemize}

\section{\versal{ÁGUIA} \versal{DE} \versal{SANGUE}}

O \emph{blódörn} (Águia de sangue) é um ritual que consiste em abrir a
costela das vítimas, extraindo os pulmões e abrindo"-os na forma de asas.
Em algumas fontes, a prática é percebida também como um método de
tortura ou execução. Ela é mencionada em várias fontes literárias, como
\emph{Reginsmál} 26; \emph{Orkneyinga saga} 8; \emph{Gesta Danorum} 13,
315; \emph{Norna"-Gests þáttr} 6; \emph{Knútsdrápa} de Sighvatr
Thórðarson.

O \emph{blódörn} é um tema polêmico nos estudos escandinavos. Para os
autores que defendem a sua existência histórica, como Alfred Smyth,
Ronald Hutton e Régis Boyer, ele podia ter relação com os sacrifícios
humanos realizados para o deus Odin. Segundo Boyer, a prática pode ter
perdido seu caráter religioso e mesmo ter ficado em desuso, na época da
cristianização, mas auxiliou a reforçar a imagem de barbárie dos
nórdicos frente aos povos invadidos.

Em um detalhado e crítico estudo, a historiadora Roberta Frank sugere
que as narrativas envolvendo o tema nas fontes foram construções
literárias e invenções criadas para reforçar o horror dos povos
invasores, negando qualquer origem ritualística para a prática. Segundo
outros pesquisadores, os próprios poetas da Era Viking não souberam
interpretar corretamente as informações históricas, perpetuando
fantasias sobre este ritual, enquanto outros entendem que este ritual
possui relação direta com as divindades da guerra e mesmo algumas
evocações em gravuras da Idade do Bronze escandinava. A~utilização das
imagens da pedra pintada de Hammar \versal{I}, na ilha de Gotland, como evidência
para o \emph{blódörn,} é questionável. A~sequência ao lado de um
enforcado, próximo a um símbolo de valknut e uma águia, na qual um homem
de pé segura uma lança sobre o corpo de outro humano deitado (que pode
ser uma criança ou outro homem, numa escala menor), demonstra a
existência de sacrifícios humanos na Era Viking, mas não existe
detalhamento na imagem para verificarmos se o pulmão está sendo extraído
do corpo da vítima.
\SIG{Johnni Langer}

Ver também Águia; Paganismo nórdico; Odin; Xamanismo nórdico.



\begin{itemize}\footnotesize
\item
  \versal{BOYER}, Régis. ``Aigle de sang''. \emph{Héros et dieux du Nord}. Paris:
  Flammarion, 1997, p.~12.
\item
  \versal{FRANK}, Roberta. ``Viking atrocity and skaldic verse: the rite of the
  Blood"-Eagle''. \emph{English Historical Review} 99 (391), 1984, pp.~323--43.
\item
  \versal{HAYWOOD}, John. ``Blood eagle''. \emph{Encyclopaedia of the Viking Age}.
  Londres: Thames and Hudson, 2000, pp.~34--35.
\item
  \versal{LANGER}, Johnni. ``Religião e magia entre os Vikings''. \emph{Brathair}
  5(2), 2005, pp.~55--82.
\end{itemize}

\section{\versal{ÁLFABLÓT}}

\emph{Álfablót} (Sacrifício aos elfos) é um ritual pagão descrito nas
fontes literárias medievais.

Para Rudolf Simek, existiram três momentos do registro deste ritual nas
fontes. O~primeiro está relacionado ao escaldo Sighvatr Thórdarson, que
em sua obra \emph{Austrfararvísur} menciona sua viagem no outono de 1018
para a Suécia, onde foi hostilizado pelos pagãos suecos. Em parte, a
recusa de hospitalidade nas fazendas suecas teria conexão com o ritual
álfablót (que estava sendo realizado no momento da chegada de Sighvatr),
e, em especial, sua entrada foi negada por uma anciã que temia a ira de
Odin.

Um segundo momento da descrição do ritual é na \emph{Kormáks saga} 22,
onde um tipo diferente de ritual é executado: ao herói Þórvarðr é
recomendado despejar o sangue de um boi nas montanhas habitadas pelos
elfos e preparar uma refeição com a carne do animal. De acordo com a
cronologia interna da fonte, o incidente teria ocorrido no século \versal{X}, mas
como esta saga foi escrita somente após o século \versal{XIII}, Simek acredita
que a crença nos poderes dos elfos ainda continuava na Escandinávia após
a cristianização. O~terceiro momento em que o ritual foi citado é na
\emph{Ynglinga saga} 44, 48, 49, relacionada ao rei Ólafr Guðrøðarson.
Após um período de grande sucesso do reinado, Ólafr morre e é sepultado
em Geirstad. Seus súditos o denominam Geirstaðaálfr, e sacrifícios a ele
são realizados. Seu bisavô é chamado de brynjálfr em uma estrofe.
Segundo Peter Schjødt, este ritual foi executado para garantir
fertilidade e anos de paz na comunidade. Neste sentido, os elfos
estariam relacionados aos espíritos da terra, sendo ambos ctônicos e
associados com os simbolismos de morte, fertilidade e proteção da
localidade.

Na concepção de John Lindow, o ritual aos elfos também possuía conexão
explícita com os deuses. Para Catharina Raudvere, tanto os espíritos da
terra (\emph{landvættir}) quanto os elfos estão estreitamente conectados
à fazenda, mas assumem formas diferentes. Enquanto na \emph{Kormáks
saga} ele está envolvido em rituais de cura, na \emph{Ynglinga saga} ele
é uma celebração aos ancestrais.
\SIG{Johnni Langer}

Ver também Dísir; Elfos; Landvættir; Paganismo nórdico.



\begin{itemize}\footnotesize
\item
  \versal{LINDOW}, John. ``Álfablót''. \emph{Norse Mythology: a guide to the gods,
  heroes, rituals, and beliefs}. Oxford: Oxford University Press, 2001,
  pp.~53--54.
\item
  \versal{RAUDVERE}, Catherina. ``Popular Religion in The Viking Age'', \emph{in}  \versal{BRINK},
  Stefan; \versal{PRICE}, Neil (eds). \emph{The Viking World}. Nova York:
  Routledge, 2008, pp.~235--43.
\item
  \versal{SIMEK}, Rudolf. ``Álfablót''. \emph{Dictionary of Northern Mythology}.
  Londres: D.S.~Brewer, 2007, pp.~7--8.
\item
  \versal{SCHJØDT}, Jens Peter. \emph{Initiation between two worlds: structure
  and symbolism in pre"-Christian Scandinavian religion}. Odense: The
  University Press of Southern Denmark, 2008, pp.~159, 381, 384--85.
\end{itemize}

\section{\versal{ÁLFAR}}

Ver Elfos.

\section{\versal{ALFHEIMR}}

Em nórdico antigo, Alfheimr significa mundo dos elfos. De acordo com
Snorri no \emph{Gylfaginning} 16, é a residência dos elfos claros, que
segundo Simek ele teria imaginado que se situaria nos céus, enquanto os
elfos escuros seriam localizados no submundo. No poema éddico
\emph{Grímsnimál} 5, Alfheimr é dominado de residência de Freyr, uma das
muitas residências dos deuses listadas nesta fonte. Para John Lindow,
não existem fontes que conectem diretamente Freyr com os elfos. Para a
historiografia medieval, Álfheimar era uma região situada entre os rios
Gota e Glom, separando a Noruega da Suécia.
\SIG{Johnni Langer}

Ver também Elfos; Nove mundos.



\begin{itemize}\footnotesize
\item
  \versal{LINDOW}, John. ``Álfheim (elf"-land)''. \emph{Norse Mythology: a guide to
  the gods, heroes, rituals, and beliefs.} Oxford: Oxford University
  Press, 2001, p.~54.
\item
  \versal{SIMEK}, Rudolf. ``Alfheimr''. \emph{Dictionary of Northern Mythology}.
  Londres: D.S.~Brewer, 2007, p.~8.
\end{itemize}

\section{\versal{ALIMENTAÇÃO} E \versal{MITOS}}

Ver Banquetes rituais na Era Viking; Bebidas sagradas nórdicas;
Hidromel da poesia; Mitos alimentares nórdicos.

\section{\versal{ALMA} E \versal{ESPIRITUALIDADE}}

Os germanos possuíam uma concepção de alma interna, \emph{hamr} (forma)
e \emph{fylgja} (acompanhante), o duplo fiel que todo humano possui. O~hamr 
é suscetível de sair do corpo, desafiando as leis de espaço e
tempo. É~possível que esta noção tenha sido influenciada pelo xamanismo
euro"-asiático. A~palavra hamr designa a forma interna que cada um
possuiria. Como dito, é suscetível de evadir"-se do suporte corpóreo, que
entra em catalepsia ou levitação. O~hamr é capaz de retornar para outros
locais ou outras épocas, com a finalidade de acompanhar as missões com a
forma de seu possuidor. Ele assume uma forma animal, em geral simbólica
de seu suporte. Uma vez que a empreitada está cumprida, ele regressa ao
corpo de seu possuidor. A~origem destas imagens pode remontar aos
lapões, que ocupavam a Escandinávia antes dos germanos. Existem relações
entre o hamr e as representações de lobisomens, de homens
transformando"-se em lobos durante a noite (\emph{hamrammr},
\emph{rammaukin}, \emph{eigi einhamr}).

A \emph{fylgja} é uma entidade sobrenatural (espírito tutelar),
geralmente feminina, que está ligada a um indivíduo e que o acompanha
pela vida toda, sendo visível quando a morte se aproxima, sendo
espíritos tutelares com funções semelhantes às de valquírias, dises e
hamingja. É~o vocábulo etimologicamente relacionado à alma mais antigo e
também designa no nórdico antigo as membranas placentárias que envolvem
a criança no momento do nascimento. O~verbo fylgja significa ``seguir'',
no sentido de acompanhar. Este duplo possui a mesma imagem que seu
suporte material, mas também uma figura simbólica animal. A~fylgja da
família é conhecida como \emph{Aettarfylgja}. Na \emph{Hellgaquivða
hjörarðzsomar}, a fylgja de Helgi aparece sob a forma de uma mulher
andando com lobo e cobras.

Assim, hamr e fylgja são os constituintes internos da espiritualidade do
homem, enquanto o \emph{hugr} (equivalente ao \emph{mana} polinésico, a
alma do mundo) é o externo, mas todos possuem relação direta com o
destino e os mortos. O~hugr seria a alma do mundo, que se manifesta ao
homem no momento de situações reflexivas (espirros, bocejos, coceiras)
ou, mais geralmente, graças a palavras mágicas, com fins cognitivos, ou
ainda em sonhos e aparições. Este hugr podia realizar atos benéficos ou
maléficos: morder (\emph{bíta}), cavalgar (\emph{riða}) e se manifestar
por meio de pesadelo (\emph{mara}).

Por mais individualistas que os nórdicos tenham sido, suas
representações são fortemente alargadas com a ideia de família, de clã.
Assim, temos o conceito de \emph{hamingja}, a figura tutelar de um clã,
relacionada especialmente com a personalidade deste mesmo clã, como a
descrita na \emph{Saga de Viga"-Glúmr}, onde uma gigantesca mulher surge
ao herói, exatamente no momento em que ele morre, encarnando valores de
proteção, ou seja, é a forma com que o destino se aplica a uma família.
Também existia a noção de \emph{aettarfylgja}, a fylgja atrelada a toda
uma família e encarregada de velar por sua prosperidade. A
\emph{hamingja} podia ser alterada, como consequência do duelo entre
clãs familiares (\emph{hamingjaskipti}).

A noção do ``nada'' não existia entre os escandinavos antigos, era
totalmente estrangeira. A~morte não era jamais um termo absoluto nem
mesmo uma ruptura radical, era considerada uma simples mudança de
estado. Morrer era simplesmente passar ao estado dos ancestrais, com o
saber e poder tutelar. Podia"-se retornar sob outra forma pela
reencarnação ou metempsicose, que era limitada ao clã. Perpetuar um nome
era necessariamente ressuscitar um ancestral, relacionado ao
\emph{óðal}, o patrimônio indivisível que se transmite de geração a
geração.

Não ocorria uma demarcação clara entre vivos e mortos. A~circulação de
um domínio e outro não era jamais interrompida -- os mortos
frequentemente vinham informar aos vivos em aparições ou revelações. 
A~mentalidade germânica não possuía uma consciência clara de outro mundo:
foi o cristianismo que o introduziu. Se analisarmos as fontes
literárias, não teremos somente um, e sim vários mundos intercalados. Os
mortos são os guardiães dos clãs e se comunicam com os vivos através de
sonhos, aparições, signos e símbolos.

O destino não era jamais individual, mas sim inscrito dentro da
perspectiva de uma família, extremamente dotada de uma qualidade própria
de fatalidade. Quando Gauka-Þórir fala de ``nossa força'' (\emph{afl
okkat}) e de ``nossa capacidade de vitória'', ele tenta considerar essas
palavras muito além de seus companheiros de escolta: a longa corrente,
na verdade, dos ancestrais que fazem sua identidade.
\SIG{Johnni Langer}

Ver também Paganismo nórdico; Vida após a morte; Xamanismo nórdico.



\begin{itemize}\footnotesize
\item
  \versal{BOYER}, Régis. \emph{Le monde du double: la magie chez les anciens
  Scandinaves}. Paris: Berg, 1986.
\item
  \versal{BRYAN}, Eric Shane. ``Icelandic fylgjur tales and possible Old Norse
  context''. \emph{The heroic Age} 13, 2000.
\item
  \versal{DAVIDSON}, Hilda. ``The conception of the soul''. \emph{The Road to Hel: a
  study of the conception of the dead in Old Norse literature}. Nova
  York: Greenwood Press, 1968.
\item
  \versal{LANGER}, Johnni. ``Religião e magia entre os Vikings''. \emph{Brathair}
  5(2), 2005, pp.~55--82.
\item
  \versal{STRÖMBÄCK}, D.~``The concept of soul in nordic tradition''. \emph{Arv} 31,
  1975, pp.~5--22.
\end{itemize}

\section{\versal{ALVÍSSMÁL}}

Poema éddico encontrado somente no Codex Regius da \emph{Edda Poética}.
É~considerado o último poema da seção mitológica devido ao fato de ser
um anão o tema principal. Consiste em 35 estrofes em forma de diálogo
utilizando a métrica ljóðaháttr. Lee Hollander o caracteriza como sendo
um poema didático utilizado pelos escaldos para memorizar mais
facilmente o vocabulário mitológico. Segundo Rudolf Simek o Alvíssmál
teria sido escrito no século \versal{XII} e não seria derivado de uma narrativa
mítica, mas de material poético como a lista de nomes do Þulur
(\emph{Edda} de Snorri) e apresentada pelo poeta como um trabalho
mitológico. Também John Lindow concorda que a temática de duelos verbais
de anões e do próprio Thor são incomuns, demonstrando a origem tardia do
poema, mas a ação de proteção das mulheres pelo deus é apropriada ao
contexto. Mogk e Henry Bellows também acreditavam que o poema datava do
século \versal{XII}, no período denominado de renascimento da poesia escáldica.

Na narrativa, o anão Alvíss tenta conseguir em casamento a filha de
Thor, mas tem que passar por uma série de questões gnômicas feitas pelo
próprio deus. Nas primeiras estrofes, Thor indaga os nomes para a Terra,
o céu, a Lua, o Sol, vento, fogo, mar, madeira, noite etc. No desfecho,
o anão é enganado e transformado em pedra pela luz do Sol nascente.
\SIG{Johnni Langer}

Ver também Codex; Edda Poética; Mitologia Escandinava.



\begin{itemize}\footnotesize
\item
  \versal{ACKER}, Paul. ``Dwarf"-lore in Alvíssmál'', \emph{in}  \versal{ACKER}, Paul \& \versal{LARRINGTON},
  Carolyne (eds). \emph{The Poetic Edda: Essays on Old Norse
  Mythology}. Nova York e Londres: Routledge, 2002, pp.~213--27.
\item
  \versal{LINDOW}, John. ``Alvíssmál''. \emph{Norse Mythology: a guide to the gods,
  heroes, rituals, and beliefs.} Oxford: Oxford University Press, 2001,
  pp.~56--58.
\item
  \versal{SIMEK}, Rudolf. ``Alvíssmál''. \emph{Dictionary of Northern Mythology}.
  Londres: D.~S.~Brewer, 2007, pp.~12--13.
\end{itemize}

\section{\versal{AM} 748 \versal{I} 4to.}

Fragmento de manuscrito islandês que contém diversos poemas éddicos,
datado do início do século \versal{XIV} e inserido na coleção Arnamagnæan da
Biblioteca da Universidade de Copenhague, motivo de também ser
denominado \emph{Codex Arnamagnæanus.} Contém integralmente as versões
dos poemas \emph{Grimnismál}, \emph{Hymskvida} e \emph{Baldrs draumar} e
fragmentos dos poemas \emph{Skínismál}, \emph{Hárbardsljód},
\emph{Vafdrúdnismál} e \emph{Volundarkvida}. É~o único manuscrito
medieval que preservou o poema \emph{Baldrs draumar} (Os sonhos de
Balder), e todos os outros possuem outras versões no Codex Regius da
\emph{Edda Poética}, considerado superior em termos de preservação e
originado da mesma fonte do qual o \versal{AM} 748 \versal{I} 4to foi baseado.
\SIG{Johnni Langer}

Ver também Codex; Edda Poética; Mitologia Escandinava.



\begin{itemize}\footnotesize
\item
  \versal{HOLLANDER}, Lee M.~``General introduction''. \emph{The Poetic Edda}.
  Austin: University of Texas, 2008, pp.~ix"-xxix.
\end{itemize}

\section{\versal{AMULETOS} \versal{MÁGICOS}}

Objetos mágicos utilizados para proteger o portador de infortúnios ou
para conceder algum poder especial. O~uso de amuletos entre os germanos
antigos e os escandinavos é atestado pela Arqueologia e por diversos
pesquisadores. Esses amuletos são de origem animal, vegetal ou mineral
(pedaços de ossos, conchas, mandíbulas animais, raízes, fragmentos de
âmbar etc.), mas o mais comum é serem fabricados com metal, como os
bracteados. Amuletos com inscrições rúnicas ou símbolos mágicos também
são significativos, mesmo após a cristianização, alguns inclusive
utilizando cápsulas de prata ou bolsinhas com ervas.

O pesquisador Signe Fuglesang em 1989 realizou uma densa sistematização
sobre o uso de amuletos na Escandinávia da Era Viking, mas questionou
muitas interpretações. Para ele, a associação entre divindades e
amuletos é duvidosa, com exceção do martelo de Thor, sendo que o uso das
fontes literárias para estabelecer o contexto ritual dos objetos seria
nulo. Apesar disso, a maioria dos pesquisadores vem relacionando a
existência de numerosos amuletos mágicos entre os nórdicos pré"-cristãos.

Segundo Rudolf Simek, alguns amuletos não se relacionam diretamente com
proteção mágica, mas estabelecem uma conexão entre uma deidade em
particular, como pequenas estatuetas de deuses, como Freyr e Thor.
Pequenos martelos de Thor foram utilizados como pingentes"-amuletos (em
contraposição aos crucifixos), simbolizando a fé pagã durante o processo
de conversão da Escandinávia. Outras armas miniaturizadas, como
pingentes de pequenas cabeças de lança, podem significar uma conexão com
a devoção a Odin, como documentado pelos bracteados.

Escavações arqueológicas na Estônia da Era Viking revelaram uma
quantidade imensa dos mais variados tipos de amuletos pagãos,
sistematizados por Andres Tvauri. Alguns destes são bem exóticos, como
pingentes de ossos em formato de pequenos pentes, vistos como objetos de
proteção mágica ou que trazem força vital para os cabelos. A~maioria
absoluta destes amuletos possui proporções e origens diferentes, sendo
confeccionados com caninos e garras dos mais variados animais: ursos,
lobos, cachorros, raposas, porcos selvagens e domésticos, cavalos,
castores, falcões. Tvauri considera que o uso masculino destes objetos
tinha uma significação de proteção marcial, enquanto o feminino seria
utilizado para fins mágicos de fertilidade. As garras de águia eram
associadas com o deus do trovão e os relâmpagos no céu -- seu uso era um
privilégio da elite, pois estes animais só podiam ser caçados pela
aristocracia. Outros tipos de pingente, como caninos de castores, podem
estar relacionados com uma espécie de culto ao castor (com função
mágica), mas também são considerados símbolo de alto status social.

Um dos tipos de amuletos nórdicos mais estudados atualmente são os
bracteados, objetos circulares com decorações em somente um dos lados e
utilizados como pingentes, datados dos séculos \versal{V} a \versal{VII} 
d.C.~Originalmente, eram imitações dos medalhões clássicos dos imperadores
romanos. Até o presente momento foram recuperados na Escandinávia e em
outras regiões europeias mais de 650 exemplares. Além de runas e
símbolos (como círculos concêntricos, suásticas, triskelions e espirais)
os bracteados apresentam imagens de animais (porcos, aves, cavalos,
serpentes e criaturas fantásticas) e entidades antropomórficas. Estes
objetos foram encontrados em sepulturas masculinas e femininas, com
predominância destas últimas, e foram feitos em ouro ou prata. Para
Hilda Davidson, as runas presentes nos bracteados serviram para aumentar
o poder mágico do amuleto e foram símbolos de poder associados ao
destino da família. Alguns também podem ter sido utilizados como
proteção contra danos ao portador. Algumas cenas dos bracteados foram
identificadas com a morte de Balder, a mutilação de Týr por Fenrir e
outras a Odin e seus corvos.

Outros tipos de objetos considerados como amuletos são pequenas lâminas
de ouro, encontradas nas fundações de certas construções (algumas
conectadas a salões reais e centros sagrados). Geralmente as imagens
consistem em um homem abraçando uma mulher, comumente interpretadas como
sendo Freyr e Gerd, conectando o objeto com os deuses da fertilidade, o
abençoar da terra, as famílias com a comunidade. As pesquisas recentes
de Gro Steinsland relacionam estes objetos com as dinastias reais e as
ideologias aristocráticas para manutenção do poder por meio da releitura
social e política dos mitos.

Mas sem dúvida os tipos de amuletos mais importantes para entender a
religiosidade nórdica pré"-cristã são os que portam inscrições rúnicas,
devido ao fato de podermos contrastá"-los diretamente com as fontes
literárias. Segundo o minucioso estudo de Mindy MacLeod e Bernard Mees,
os tipos de inscrições mais comuns são os que solicitam ajuda aos
deuses. Algumas invocam deidades (Logathore, Wodan e Thonar) para um
amuleto amoroso, como a inscrição do broche de Nordendorf (Alemanha,
séc. \versal{VI}). Outra, como Pforzen (Alemanha, séc. \versal{VI}), é um encanto para
favorecer a caça, invocando Aigil e Airun (seres semidivinos citados no
\emph{Volundarkvida}).

Algumas vezes, palavras de encanto em amuletos rúnicos funcionam como
símbolos não alfabéticos, como o uso de suásticas, flechas e árvores
encontradas em broches, tornando o amuleto mais poderoso. A~invocação de
deidades para a cura também ocorre, como o texto de Ribe (Dinamarca, 725
d.C.): ``\emph{Ulfr auk Óðinn auk Hó. Hjalp es viðr/þæima værki. Auk
dverg unninn. Bóurr}'' (``Ulfr e Odin e o grande Tyr/Ajudam Bur contra o
mal/ E o anão é derrotado/Bóurr''). O~deus Thor também aparece
relacionado à cura, como na inscrição de Kvinneby (Suécia, séc.\,\versal{XI}):
``\emph{Hær rïsti ek þær Berg, Böfi/Mær fullty! Ïhüð es þær vïss./Em brä
haldi illu frän Böfa./Þörr gæti Hans með þëm hami sem ur hafi kam./Fly
främ ilvëtt! Fær ekki af Böfa./Guð eru undir hänum auk yfir hänum}.''
(``Aqui eu gravei para seu socorro, Bofi/Socorra"-me! O conhecimento é
certo para você/e pode o relâmpago carregar todo o mal sobre Bofi/Thor
poderoso protege com seu martelo e sai do oceano/Evite o mal! Ele não
conseguiu nada de Bofi/ Os deuses estão acima dele e abaixo dele'').
Como esta inscrição também possui o desenho de um peixe, MacLeod e Mees
acreditam que também tenha conexão com o episódio da pesca da serpente
do mundo por Thor.

Os amuletos com funções puramente curativas, obviamente, em se tratando
de uma sociedade medieval, são abundantes. Muitas inscrições do período
de transição mesclam conhecimentos clássicos, com a tradição pagã e o
folclore cristão. O~mal (as doenças, a dor, as crises e violências)
muitas vezes é percebido simbolicamente na figura do lobo e dos
gigantes, como no amuleto de Sigtuna (Suécia, séc. \versal{XI}): ``\emph{Þurs sarriðu,
Þursa dröttin; Fly þú nu, fundinn es! af þér þrjár þrár, ülfr!}''
(``Gigante da gangrena, senhor dos gigantes, foge, você foi descoberto!
Tenha para si três tormentos, lobo!''). A~runa em questão (Þurs) também
pode significar o mal causado pelos anões e elfos. No dialeto sueco
moderno, \emph{tuss} tanto significa lobo quanto gigante, ogro e
pesadelo. Essa mesma runa possui conotações negativas para as mulheres.

Ao estudar especificamente amuletos rúnicos na Dinamarca dos séculos \versal{XI}
ao \versal{XV}, o epigrafista Rike Olesen percebe os mesmos como objetos
essencialmente funcionais, sem o caráter estético presente nas fontes
literárias. Eles foram confeccionados por pessoas buscando algum tipo de
proteção, alguns já conectados diretamente com a tradição religiosa
cristã, mas ainda preservando referenciais de eficácia mágica dos tempos
pagãos. Assim, são considerados por Olesen como materiais híbridos,
conservando elementos tradicionais e adicionando temas novos.
\SIG{Johnni Langer}

Ver também Anéis; Espiral; Magia rúnica; Paganismo nórdico; Plantas
mágicas; Valknut; Símbolos rúnicos; Runas; Suástica.



\begin{itemize}\footnotesize
\item
  \versal{DAVIDSON}, Hilda. ``Early amulets''. \emph{The lost beliefs of Northern
  Europe}. Nova York: Routledge, 2001, pp.~37--45.
\item
  \versal{FUGLESANG}, Signe Horn. ``Viking and medieval amulets in Scandinavia''.
  \emph{For Vännen: Journal of Antiquarian Research} 84, 1989, pp.~ 15--27.
\item
  \versal{LANGER}, Johnni. ``Símbolos religiosos dos vikings''. \emph{História,
  Imagem e Narrativas} 11, 2010, pp.~1--28.
\item
  \versal{MACLEOD}, Mindy \& \versal{MEES}, Bernard. \emph{Runic amulets and magic
  objects}. Londres: Boydell Press, 2006.
\item
  \versal{MAREZ}, Alain. ``Magie, culte et religion/Rites et malédictions''.
  \emph{Anthologie runique}. Paris: Les Belles Lettres, 2007, pp.~ 158--96.
\item
  \versal{OLESEN}, Rikke Steenholt. ``Runic amulets from Medieval Denmark''.
  \emph{Futhark: International Journal of runic studies} 1, 2010, pp.~ 161--76.
\end{itemize}

\section{\versal{ANDVARI}}

Ver Anel; Fafnir; Nibelungos; Sigurd.

\section{\versal{ANÉIS}}

O anel é um artefato que simboliza um elo, geralmente associado a uma
promessa, uma aliança ou um vínculo social. Na tradição de estudos
medievais, é comum atribuirmos a possibilidade da entrega de um anel
durante um ritual feudo"-vassálico. Ainda que longe dessa realidade, é
possível encontrar nas sagas a entrega de um anel como forma de
pagamento por um feito, o acordo entre um rei e seus súditos, ou ainda
como identificação de elevado gênero social. Na mitologia escandinava,
os anéis também representam uma associação com as funções de certas
figuras divinas.

\imagempequena{Pedra de Tängelgärda \versal{I}, Gotland, séc.~\versal{VIII}}{./img/17.png}               % ANÉIS

Na \emph{Hrólfs Saga Kraka ok Kappa Hans} (Saga de Hrólfr Kraki e Seus
Campeões), Björn é amaldiçoado pela sua madrasta e toma a forma de um
urso. Bera, sua amante, recolhe de seu corpo um anel que possuía no
braço para que o identificassem não como o urso amaldiçoado, mas como o
príncipe Björn, filho do rei Hringr. Na mesma saga, encontramos o
\emph{Svíagriss} (Pequeno Porco dos Suecos) que é entregue pela rainha
Yrsa ao seu filho Hrólfr enquanto ele visitava o rei Aðils (que tentou
perfidamente assassinar Hrólfr e seus campeões). Junto a esse anel, a
rainha lhe deu um chifre de prata, além de vários outros tesouros
inestimáveis. Ameaçado em uma emboscada, Hrólfr agita o chifre,
esparramando todo o tesouro no chão, levando seus inimigos a parar a
perseguição para a coleta dessas riquezas (o tema da perseguição pode
ainda ser encontrado na \emph{Ynglinga saga} e na \emph{Gesta Danorum}).
O~rei Aðils, não sendo iludido por essa tática, só é parado quando
Hrólfr arremessa o \emph{Svíagriss} no chão, sendo levado a recuperar o
anel com sua lança. Ao tentar recuperar o anel, Hrólfr compara Aðils a
um suíno: ``Eu agora o fiz andar como um porco, aquele que dos suecos
era o mais poderoso''.

No \emph{Skaldskaparmál} é descrito o mesmo episódio envolvendo o
\emph{Svíagriss} de maneira similar. O~rei Hrólfr junto aos seus
campeões vão ao auxílio de Aðils, que se recusa a pagar o preço pela
ajuda que recebeu: o elmo \emph{Hildigöltr} (Javali de Batalha), a
armadura \emph{Finnsleif} (Herança dos Finns) e o próprio
\emph{Svíagriss}. As representações desses animais em equipamentos
guerreiros podem estar associados a práticas mágicas da transformação do
homem em animal, ou a inspiração sobre suas forças, ainda que esse ponto
esteja aberto à discussão na cultura material e nas narrativas
escandinavas.

O \emph{Svíagríss} (bem como todo o equipamento com desenhos de suínos)
pode estar atrelado ao culto dos deuses Freyr e Freyja, que, apesar de
estarem ligados em torno da esfera da fertilidade, também são notáveis
no aspecto guerreiro. A~feitiçaria Seiðr era praticada por Freyja, que
teria ensinado a prática aos outros deuses, especialmente Óðinn, que se
utiliza desse conhecimento no campo divinatório, de controle da
inteligência alheia e das doenças. O \emph{Svíagríss}, dessa maneira, é
um artefato precioso que invoca o domínio mágico"-guerreiro dos deuses
Freyr e Freyja (além da figura suína, que traz em si sua própria
belicosidade), mas que também pode estar ligado ao poder mágico que a
deusa representa (ainda que nenhum ritual ligado à prática de Seiðr
apareça na narrativa).

Um segundo anel presente na mitologia nórdica e que deve ser mencionado
é o \emph{Draupnir} (Gotejante), encontrado na \emph{Edda}
\emph{Poética} e na \emph{Edda em Prosa}. A~criação desse artefato está
ligada ao episódio do roubo dos cabelos dourados da deusa Sif por Loki,
narrado no \emph{Skáldskaparmál}. O~anão Brokkr confecciona tesouros
maravilhosos para os deuses, como uma aposta pela cabeça de Loki: um
javali com arreios de ouro e o barco Skíðblaðnir para o deus Freyr, o
martelo Mjöllnir para o deus Þórr (junto com novos cabelos para sua
esposa, Sif), e para o deus Óðinn a lança Gungnir junto com o anel de
ouro \emph{Draupnir}. Esse anel possui a seguinte propriedade: a cada
nove noites ele deverá verter oito anéis com o mesmo peso que ele.
O~número nove é vinculado à visão cósmica na mitologia escandinava, pois
são nove os mundos. Nove anos são necessários para que as
donzelas"-cisnes retornem a sua condição primordial de Valquíria na
\emph{Völlundarkviða} (Canção de Völundr); certos festivais religiosos
em Lejre, na Dinamarca, e Uppsala, na Suécia, aconteciam a cada nove
anos (em Gamla Uppsala era realizado um grande sacrifício de animais,
com nove tipos de machos sendo enforcados ao deus Óðinn). Por fim, o
número nove representa o sacrifício feito na árvore Yggdrasill no
intuito de receber conhecimento enforcando"-se por nove dias.

Outras referências às propriedades do \emph{Draupnir} aparecem em Ditos
de Skírnir (\emph{Skírnirsmál}), onde o deus Freyr se apaixona por uma
giganta e seu emissário, e Skírnir, é enviado para trazê"-la. Em certo
momento, Skírnir lhe oferece o anel \emph{Draupnir}: ``Então você tomará
esse anel,/aquele que foi queimado/com o jovem filho de Óðinn,/oito são
de pesos iguais/que dele vertem/a cada nona noite''. Assim responde
Gerðr, a giganta: ``O anel não irei aceitar,/mesmo sendo queimado,/com o
jovem filho de Óðinn;/o ouro não me falta/nas cortes de Gyrmir/divido as
riquezas de meu pai''. Durante o funeral de Baldr, no
\emph{Gylfaginning} (Visão de Gylfi), Óðinn deposita seu anel na pira
funerária de seu filho, reforçando seu caráter de tesouro precioso:
``Óðinn colocou seu anel de ouro na pira, aquele chamado
\emph{Draupnir}. Ele prosseguia dessa maneira: a cada nove noites vertia
ele oito anéis de mesmo peso'' (\emph{Gylfaginning} 49).

A propriedade de Óðinn sobre esse artefato está amplamente ligada aos
exercícios da guerra, principalmente ao domínio sagrado da liderança
guerreira. Os implementos são mais que simples posses: eles compartilham
integralmente as funções sagradas de cada deus. Dessa maneira, o
\emph{Draupnir} pode representar todas as atividades provenientes do
acúmulo de tesouros (uma das metáforas poéticas para ouro no
\emph{Skáldskaparmál} é justamente ``gotas do \emph{Draupnir}'' ou
``chuva do \emph{Draupnir}''), entre elas o patrocínio da inspiração
poética (que é outro domínio odínico), o financiamento das expedições
guerreiras, presentes para outras chefias guerreiras (com a finalidade
de adquirir apoio) etc.

A \emph{Völsunga saga} (Saga dos Volsungos) contém uma narrativa mais
concentrada em torno da maldição do \emph{Andvaranaut}. Andvari
amaldiçoa o anel (bem como todo o seu tesouro), o ouro enche a sacola
feita com a pele de Ótr, e Fáfnir assassina seu pai, tomando o tesouro e
se transformando no dragão. Impossibilitado de transpor a muralha de
chamas, transposição necessária para alcançar a Valquíria Brynhild,
Gunnar pede que Sigurðr vá buscá"-la. Sigurðr derrotara previamente o
dragão Fáfnir e tomara seu tesouro, inclusive o anel; foi quando saiu de
sua batalha e encontrou a Valquíria, e fizeram promessas de amor que
acabaram esquecendo por motivos relacionados à narrativa.

Quando Sirgurðr encontra Brynhild, ele retira o \emph{Andvaranaut} e lhe
dá outro anel do tesouro de Fáfnir, entregando o anel amaldiçoado a sua
esposa, Gudrun. O \emph{Andvaranaut} é a prova necessária para a
vingança de Brynhild. Em certa discussão a rainha Gudrun argumenta que
não surgiu homem mais pródigo que Sigurðr, que ele matou o dragão Fáfnir
e resgatou Brynhild das chamas, ao invés de Gunnar. O~anel Andvaranaut é
ao mesmo tempo a identificação dos feitos heroicos e o símbolo que
anuncia a tragédia que iria dar prosseguimento a essa saga: Brynhild
leva o rei Gunnar a planejar o assassinato de Sigurðr, que acaba sendo
morto por Guttormr, irmão de Gunnar (pois esse não rompe nenhum laço de
lealdade com o ato). Com o assassinato feito, Brynhild se mata,
profetizando antes a morte dos Giukungos, a linhagem de Gunnar e Gudrun.
\SIG{Pablo Gomes de Miranda}

Ver também Amuletos mágicos; Saga dos Volsungos; Sigurd.



\begin{itemize}\footnotesize
\item
  \versal{CHEVALIER}, Jean; \versal{GHEERBRANT}, Alain. \emph{Dicionário de Símbolos}. Rio
  de Janeiro: José Olympio Editor, 2002.
\item
  \versal{DUBOIS}, Thomas A.~\emph{Nordic Religions in the Viking Age}.
  Pensilvânia: University of Pennsylvania Press, 1999.
\item
  \versal{LANGER}, Johnni. ``Seiðr e magia na escandinávia medieval: reflexões
  sobre o episódio de Þorbjörg na \emph{Eiríks saga Rauða}''.
  \emph{Signum} 11(1), 2010, pp.~177--202.
\item
  \versal{MIRANDA}, Pablo Gomes de. ``Sobre os anéis de poder''. \emph{História,
  imagem e narrativas} 15, 2012, pp.~1--32.
\end{itemize}

\section{\versal{ANIMAIS} \versal{TOTÊMICOS}}

Praticamente todos os animais citados nas fontes literárias e que foram
representados imageticamente na Era Viking, são diretamente associados
ao deus Odin.

O lobo e o cão geralmente são companheiros das jornadas da alma para o
outro mundo em rituais votivos. Cachorros e lobos estão conectados com a
ideologia guerreira, especialmente para o grupo dos berserkir -- homens
jovens, não casados, especializados na arte da guerra. Várias gerações
de guerreiros combinavam o nome de termos de batalha com elementos
relacionados ao lobo -- também se referindo à iniciação de jovens no
mundo marcial.

Os pássaros -- aves de rapina, como gaviões e falcões, são
tradicionalmente signos da aristocracia, enquanto a águia é emblema de
poder. Em alguns pingentes, dois corvos metamorfoseiam"-se nas pontas dos
chifres de uma figura barbada, demonstrando a continuidade de antigos
cultos pré"-vikings na área nórdica.

Um tipo de amuleto muito difundido na Era Viking, tanto na área
finlandesa quanto nas ilhas britânicas, era o uso de dentes de ursos --
tanto imitações em bronze quanto peças originais. Supunha"-se que
continham propriedades mágicas, relacionadas à captura do espírito dos
ursos (\emph{karhunpeijaiset}). Na \emph{Hrólfs saga kraka}, o
personagem Bodvarr tem a alma transformada em urso, uma referência aos
antigos rituais pré"-cristãos ainda preservados na literatura
centro"-medieval.

Em recente estudo, o historiador Thomas DuBois analisou a relação do
simbolismo animal atrelado à dieta e ao culto dos deuses, como o gado,
cavalos, bodes, ovelhas, renas, porcos, peixes e ursos, demonstrando a
estreita relação entre cotidiano e religiosidade na Escandinávia
pré"-cristã.
\SIG{Johnni Langer}

Ver também Fenrir; Hugin e Munin; Odin; Paganismo nórdico; Xamanismo
nórdico.



\begin{itemize}\footnotesize
\item
  \versal{DUBOIS}, Thomas. ``Diet and deities: contrastive livelihoods and animal
  symbolism in Nordic Pre"-Christian Religious'', \emph{in}  \versal{RAUDVERE}, Catharina
  \& \versal{SCHJØDT}, Jens Peter (eds). \emph{More Than Mythology: Narratives,
  Ritual Practices and Regional Distribution in Pre"-Christian
  Scandinavian Religions}. Lund: Nordic Academic Press, 2012, pp.~65--96.
\item
  \versal{GRÄSLUND}, Anne"-Sophie. ``Wolves, serpents, and birds: their symbolism
  meaning in Old Norse beliefs'' \emph{in}  \versal{ANDRÉN}, Anders et alii
  (orgs.). \emph{Old Norse religion in long"-term perspectives}. Lund:
  Nordic Academic Press, 2004, pp.~124--29.
\item
  \versal{LANGER}, Johnni. ``Símbolos religiosos dos vikings''. \emph{História,
  Imagem e Narrativas} 11, 2010, pp.~1--28.
\end{itemize}

\section{\versal{AN}Õ\versal{ES} (\versal{DVERGAR})}

Os \emph{dvergar} (singular \emph{dvergar}) são um dos grupos de seres
inferiores na mitologia escandinava. Ao contrário da maioria desses
grupos, como os \emph{álfar} (``elfo'') ou as \emph{dísir} (um tipo de
espíritos femininos), muitos \emph{dvergar} têm nomes individuais e, em
alguns casos, desempenham um papel importante nos mitos. A~tradução de
\emph{dvergar} para ``anão'' responde a razões mais etimológicas que
históricas; os ``anões'' da mitologia não são, necessariamente, de
pequena estatura. O~termo é encontrado em todas as línguas germânicas,
como o \emph{dwarf} moderno inglês (antigo \emph{dweorg}) ou alemão
\emph{Zwerg} (antigo alto alemão \emph{twerg}). A~raiz original
germânica \emph{dwergaz} é de etimologia incerta, talvez relacionada a
temas indo"-europeus, ``torto, fraco'', denotando um ser deformado,
aleijado e, possivelmente, monstruoso. A~mesma palavra \emph{dvergar}
também indica um pilar de apoio no hall. De forma semelhante, um papel
desempenhado pelos anões na mitologia é como pilares de sustentação do
céu em cada ponto cardeal.

\imagempequena{Anões e Freyja, M. Foster, 1901}{./img/78.png} 

Além disso, os \emph{dvergar} aparecem com frequência na poesia éddica,
embora seu papel seja geralmente secundário. Apenas um poema, o tardio
\emph{Alvíssmál}, focaliza num deles. Eles também são ocasionalmente
associados com o mundo dos mortos, como é indicado por nomes como
\emph{Náinn}, \emph{Nar} (``cadáver''), Dáinn (``morto''). Outro aspecto
associado com os \emph{dvergar} é o conhecimento (incluindo magia) e
sabedoria: este aspecto é destacado no poema éddico \emph{Alvíssmál}
(``o discurso do que sabe tudo'') e também em nomes como Ráðspakr
(``sábio conselho'') ou Vitri (``sábio''). Embora não se deva exagerar
na utilidade da etimologia, é lógico pensar que tinha algum significado
para os homens da época, como demonstra o lar de várias listas de nomes
dos \emph{dvergar}, incluindo a que está presente (talvez interpolada)
no poema \emph{Vǫluspá}. Não temos nenhum vestígio de que fosse
concebida a existência de \emph{dvergar} femininos.

No entanto, o papel principal dos \emph{dvergar} na mitologia é o de
ferreiros e moradores de rochas. Ao contrário dos \emph{álfar}, os
\emph{dvergar} não parecem ter sido adorados ou incluídos nos nomes
humanos. Neste sentido eles se assemelham aos \emph{jötnar},
antagonistas dos deuses. Contudo, o \emph{dvergar} é geralmente
indeterminado, ausente, com ele é possível estabelecer laços comerciais
não hostis, mas não sociáveis. Eles normalmente são gananciosos, mas não
inerentemente agressivos. Podem ocupar uma posição de neutralidade na
cosmologia, e permanecem fora da batalha entre os deuses e seus inimigos
no \emph{Ragnarǫk}.

É possível que a ausência de generosidade e o desejo de acumular riqueza
seja a característica que melhor os distingue dos \emph{álfar} com que
regularmente são confundidos. Porém, os \emph{dvergar} são equivalentes
aos \emph{svartálfar} e \emph{dökkálfar} na Edda prosaica. Isso pode
explicar por que eles não receberam adoração dentro de um sistema
religioso baseado na ideia de uma relação de troca entre os adoradores e
as divindades. A~avareza dos \emph{dvergar} assemelha"-os aos
\emph{ormar} (``dragões, vermes, serpentes''), e é possível que o dragão
Fafnir fosse originalmente um \emph{dvergar} como seu irmão, o ferreiro
Reginn, pai adotivo do herói Sigurðr.

Ausente nas sagas dos islandeses, sua transformação em anões (no sentido
físico) pode ser vista nas sagas lendárias e de cavalaria. Ainda vivem
em rochas, mas agora os \emph{dvergar} têm vida familiar. Eles vão de
uma forma clara trabalhar como ajudantes ou inimigos, deixando de lado a
neutralidade. Trabalham também alinhados com um dos temas dominantes
destas sagas, ajudando na realização de quadros amorosos. Para isso são
dotados de alguns de seus poderes tradicionais, como a grande habilidade
na ferraria, mas também têm poder em novas áreas, como a capacidade de
curar. Portanto, os \emph{dvergar} das sagas se assemelham a figuras da
literatura continental, mas mantêm as características de temas
mitológicos.
\SIG{Santiago Barreiro}

Ver também Alvíssmál; Elfos.



\begin{itemize}\footnotesize
\item
  \versal{ACKER}, Paul. ``Dwarf"-lore in Alvíssmál'', \emph{in}  \versal{ACKER}, Paul \& \versal{LARRINGTON},
  Carolyne (eds.). \emph{The Poetic Edda: Essays on Old Norse
  Mythology}. Nova York e Londres: Routledge, 2002, pp.~213--27.
\item
  \versal{JAKOBSSON}, Ármann. Enabling Love: ``Dwarfs in Old Norse"-Icelandic
  Romances'', \emph{in}  \versal{WOLF}, Kirsten (ed.), \emph{Romance and Love in Late
  Medieval and Early Modern Iceland: Essays in Honor of Marianne
  Kalinke}. Ithaca: Cornell University Library, 2008, pp.~183--206.
\item
  \versal{BATTLES}, Paul. ``Dwarfs in Germanic Literature: Deutsche Mythologie or
  Grimm's Myths'', \emph{in}  \versal{SHIPPEY}, Tom (ed.). \emph{The Shadow"-Walkers: Jacob
  Grimm's Mythology of the Monstrous}. Tempe: Arizona Center for
  Medieval and Renaissance Studies, 2005, pp.~29--82.
\item
  \versal{MOTZ}, Lotte. \emph{The Wise One of the Mountain: Form, Function and
  Significance of the Subterranean Smith: A Study in Folklore}.
  Göppingen: Kümmerle, 1983.
\end{itemize}

\section{\versal{ANTROPOGONIA} \versal{NÓRDICA}}

A criação do primeiro homem vincula"-se ao personagem Búri, que de acordo
com Snorri nasceu do gelo lambido pela vaca Audhumla, e de cujo filho
Borr nasceram os primeiros deuses. Mas a raça humana descenderia
diretamente de um casal, Ask e Embla, cuja narrativa foi preservada
tanto na \emph{Edda Poética} como na \emph{Edda Menor}. Para Régis
Boyer, o termo Askr significa freixo e Embla tronco de videira. Segundo
John Lindow, a tradução de Embla é incerta, mas ele opta pela tradução
de olmo, uma ideia inicialmente defendida por Sophus Bugge. Para Rudolf
Simek, o termo tem conexão com o casal Assi e Ambri, citado na
\emph{História dos Lomgobardos} de Paulo Diácono, atestando a
antiguidade da narrativa mítica. Em 1910, o mitólogo H.~Sperber apontou
a semelhança de Embla com o grego \emph{ámpelos}, que significa vinho,
sugerindo uma conexão indo"-germânica com os rituais relacionados ao fogo
e ao sexo. Para alguns pesquisadores, como Henning Kure, os termos para
o casal não têm relação com plantas, mas com os órgãos sexuais.

\imagempequena{Ask e Embla, Brian Wildsmith, 1960}{./img/05.png}                                				 % ANTROPOGONIA NÓRDICA 

Segundo as informações da \emph{Völuspá} 17--18, Ask e Embla teriam sido
criados por três deuses, que lhes repassaram algumas virtudes: Odin
concedeu"-lhes o espírito; Hónir, o sentimento; Lódur, o ardor e a boa
coloração. Já no \emph{Gylfaginning} 9, a tríade criadora foi composta
de Odin, Vili e Ve, que ao encontrarem dois troncos de árvore em uma
praia lhes concederam o espírito e a vida, inteligência e movimento,
aparência humana, fala, audição e visão. Também forneceram roupas e
nomes, tendo os descendentes de Ask e Embla habitado Midgard. Para
Rudolf Simek, a antropogonia nórdica se repete no poema
\emph{Vafþrúðnismál} 45, com a narrativa do casal Lif e Lifþrasir, que
sobrevive ao colapso do mundo e é progenitor de uma nova raça humana.

Em 1879, A.C.~Bang comparou Ask e Embla com Adão e Eva, seguindo a linha
de interpretação de Sophus Bugge segundo a qual as narrativas nórdicas
foram amplamente influenciadas pelo cristianismo, uma ideia seguida
atualmente por vários pesquisadores -- inclusive aparecendo no
documentário \emph{Thor} (Série \emph{Confronto dos Deuses}, 2009).
Apelando ainda para uma matriz pagã da narrativa de Ask e Embla, Régis
Boyer pensa que existe um parentesco muito próximo entre Ask (freixo)
com a árvore Yggdrasill. Isso seria um reflexo de uma imaginação
naturalista, muito característica dos germanos em geral: a árvore como
símbolo da vida. Recentemente, o pesquisador Hans Hultgård realizou um
amplo e detalhado estudo comparativo desta narrativa com as mais
diversas tradições antropogônicas do Velho Mundo, passando pela área
clássica, iraniana e finlandesa, concluindo que ela pertence a uma
tradição indo"-europeia mais antiga, a de mitos em que os homens são
gerados em conexão com árvores. Em especial, Hultgård comparou a
\emph{Völuspá} 4, onde o Sol surge associado a plantas -- em um momento
cosmogônico, com a estrofe 17 narrando o surgimento de Ask e Embla. E~utilizando 
um mito frígio, que relata justamente o momento em que o Sol
ilumina os primeiros humanos, transfigurados em árvores, apela para a
ideia de que o poema éddico preservou uma narrativa muito antiga, sem
vínculo com o cristianismo.

Os primeiros humanos na cosmovisão nórdica tiveram muito destaque na
arte ocidental. Na ilustração \emph{Odin, Lodur, Hoenir skabe Ask og
Embla}, de Lorenz Frølich, 1895, o casal ganha vida pelos três deuses.
Enquanto Lodur e Hoenir seguram os braços de Ask e Embla, Odin está
sentado e abraçando os mesmos. Ao contrário de suas outras ilustrações,
Frølich representou Odin como um ancião barbudo, lembrando muito as
divindades clássicas, especialmente Netuno. O~Sol surge por trás de todo
o conjunto, concedendo um sentido óbvio de início da vida. Em outra
ilustração, datada de 1919 e realizada por Robert Engels, o casal emerge
de uma árvore, ladeada pelas três divindades. O~tom geral da composição
é muito mais grosseiro, diferenciando"-se da imagem delicada de Frølich.
Em 1948 o sueco Stig Blomberg realizou a escultura \emph{Ask och Embla},
para uma praça na cidade de Sölvesborg. O~casal ganha uma estética que
se aproxima muito das representações de Adão e Eva dentro do imaginário
cristão. Em 2003, foi criada a pintura \emph{Ask og Embla}, integrante
de uma coleção de selos das ilhas Faroe com temas nórdicos, de autoria
do artista Anker Eli Petersen. A~imagem inovou pelo uso de cores fortes
e contraste de tons claros e escuros, concedendo um sentido de submissão
humana aos deuses pagãos. Uma das poucas representações sobre Ask e
Embla que mostram o momento em que Odin, Vili e Ve encontram os troncos,
antes de dar forma humana a estes objetos, foi realizada em 1995 pelo
ilustrador James Alexander.
\SIG{Johnni Langer}

Ver também Cosmogonia nórdica; Odin; Teogonia nórdica.



\begin{itemize}\footnotesize
\item
  \versal{HULTGÅRD}, Anders. ``Ask and Embla myth in a comparative perspective'', \emph{in} 
  \versal{ANDRÉN}, Anders et alii (orgs.). \emph{Old Norse religion in
  long"-term perspectives}. Lund: Nordic Academic Press, 2004, pp.~ 58--62.
\item
  \versal{KURE}, H.~``Embla ask''. \emph{Arkiv for nordisk Filologi} 117, 2002, pp.~ 161--70.
\item
  \versal{SPERBER}, Hans. ``Embla''. \emph{Beitrage zur Geschichte der deutschen
  Sprache und Literature} 36, 1910, pp.~219--22.
\item
  \versal{STEINSLAND}, Gro. ``Antropogonimyten i Völuspá''. \emph{Arkiv for nordisk
  Filologi} 98, 1983, pp.~80--107.
\end{itemize}

\section{\versal{ÁRABES} E \versal{RELIGIOSIDADE} \versal{NÓRDICA}}

Ver Ritos Rus.

\section{\versal{ARDRE} \versal{VIII}}

Pedra pintada encontrada na ilha de Gotland, báltico sueco, datada do
século \versal{IX} d.C.~Uma das mais complexas e importantes fontes iconográficas
da mitologia nórdica. A~Estela possui três conjuntos imagéticos
principais: a base, formada por diversas narrativas mitológicas; a cena
central do navio; e o topo, separado por um detalhe ornamental linear. O~topo é muito 
semelhante artisticamente à outra Estela, a de Alskog
Tjängvide \versal{I}, o que levou diversos pesquisadores a considerá"-las como
tendo sido feitas por um mesmo escultor: ambas possuem a representação
idêntica do cavalo Sleipnir, com o deus Óðinn montado e com a mão
esquerda levantada. Porém, ao contrário de Alskog Tjängvide \versal{I}, a de
Ardre \versal{VIII} não contém nenhuma valquíria recebendo o mesmo, nem a figura
de um cão acompanhando. Ao fundo, em ambas as estelas ocorre a figuração
do palácio do Valhöll, uma estrutura circular e abobadada, semelhante às
casas longas dos escandinavos. O~guerreiro morto ocupando a parte mais
elevada (flutuando) também ocorre nas duas estelas. A~maior diferença,
entretanto, fica pelas representações do segundo plano de Ardre \versal{VIII}. 
O~conjunto imagético mais importante refere"-se ao mito do ferreiro
Völlundr, extremamente importante para os povos germânicos: o momento em
que o ferreiro utiliza um par de asas que fabricou para fugir da prisão,
ladeado pela figura de uma Valquíria, com a mesma estilística de outras
estelas e pingentes. No centro, as ferramentas de forja e ferraria; no
lado direito, o corpo descabeçado dos filhos do rei Nídud. A~base da
Estela possui imagens não identificadas: dois personagens pescando
(talvez outra representação de Thor pescando a serpente do mundo), um
homem entrando em um aposento cercado, no qual se encontra um cão e
outros dois homens, e a figura externa de um cão, de formas muito
semelhantes à dos encontrados em outras estelas, porém estes no topo,
possivelmente a representação de Garmr. Várias figuras humanas
encontram"-se cercadas por quadrados, mas a desfiguração da Estela não
permite maiores identificações. Ao fundo do navio central, ocorre a
imagem da pesca da serpente do mundo pelo deus Þórr e o gigante Hymir. 
A~representação mais enigmática é a de um gigante segurando uma figura com
várias cabeças, ao lado da valquíria.
\SIG{Johnni Langer}

Ver também Mitologia Escandinava; Paganismo nórdico; Pedras pintadas de
Gotland.



\begin{itemize}\footnotesize
\item
  \versal{LANGER}, Johnni. ``As estelas de Gotland''. \emph{Brathair} 6(1), 2006, pp.~ 10--41.
\item
  \versal{NYLÉN}, Erik \& \versal{LAMM}, Peder. \emph{Les pierres gravées de Gotland}.
  Paris: Michel de Maule, 2007.
\item
  \versal{STAECKER}, Jörn. ``Heroes, kings, and gods: discovering sagas on
  Gotlandic picture"-stones'', \emph{in}  \versal{ANDRÉN}, Anders et alii (orgs.). \emph{Old
  Norse religion in long"-term perspectives}. Lund: Nordic Academic
  Press, 2004, pp.~363--68.
\end{itemize}

\section{\versal{ARMAS} \versal{MÍTICAS}}

Ver Espadas míticas; Gungnir; Martelo de Thor.

\section{\versal{ARQUEOLOGIA} E \versal{MITOS} \versal{NÓRDICOS}}

Ver Amuletos mágicos; Ardre \versal{VIII}; Funerais e enterros; Hammar \versal{I};
Hogbacks; Ídolos e imagens; Klinte Hunninge \versal{I}; Pedras Pintadas de
Gotland; Pinturas rupestres nórdicas; Ritos nórdicos; Ritos Rus; Runas;
Sacrifício escandinavo; Templo de Uppsala; Toponímia e mitos nórdicos.

\section{\versal{ARQUÉTIPOS} \versal{ESCANDINAVOS}}

A teoria dos arquétipos constituiu uma das maiores aplicações no estudo
simbólico do mito, em especial na área nórdica. Utilizado originalmente
por Santo Agostinho, num sentido de modelo, foi popularizada com as
teorias de Carl Jung durante a primeira metade do século \versal{XX}, num sentido
de protótipo de narrativas míticas coletivas presentes em cada
indivíduo, algo posteriormente seguido por Mircea Eliade, Joseph
Campbell e Karl Kerény.

As aplicações da ideia de Mircea Eliade do modelo arquetípico sacro ao
mundo escandinavo foram: o povoamento da Islândia -- no momento em que
os colonos chegaram à ilha, eles repetiram o drama cósmico inicial da
transformação do caos na ordem da criação; a batalha do deus Þórr com o
gigante \emph{Hrungnir}, que influenciou o treinamento de jovens
guerreiros -- todo conflito teria sempre uma causa ritual; o
\emph{Ragnarok} (destruição do mundo seguido de uma nova criação) seria
a repetição do ciclo cósmico presente em todas as mitologias; a morte
primordial e cósmico"-criadora do gigante \emph{Ymir} -- reatualizada nos
sacrifícios sangrentos da cultura germânica, mas principalmente várias
citações sobre a árvore cósmica, a \emph{Yggdrasill} -- interpretada
como o eixo do mundo e símbolo do sagrado por excelência, expressão dos
valores religiosos relacionados com a vegetação.

Dentro do conceito simbólico"-psicológico de Joseph Campbell, temos
alguns exemplos do mundo nórdico, como o deus Wodan enforcado na árvore
cósmica -- simbolizando o centro do mundo; a filha deste mesmo deus,
Brunhilda, presa no círculo de fogo (símbolo da proteção paternal da
virgindade) e depois liberta por Siegfried -- uma das encarnações do
divino feminino no processo de iniciação heroica; o autossacrifício de
Wodan-Óðinn para obter conhecimento -- um dos referenciais míticos da
vitória interior no processo de transformação do herói; a trajetória de
Siegfried, especialmente sua infância e façanhas adultas.

Vários historiadores já criticaram os autores simbolistas que difundiram
o modelo arquetípico aplicado ao mundo escandinavo (Mircea Eliade, Carl
Gustav Jung, Joseph Campbell, entre outros), principalmente por motivos
de contextualização sócio"-histórica, opondo"-se especialmente ao
``fixismo'' em que o mito foi caracterizado, dando um valor maior às
estruturas diacrônicas que deram origem aos significados das imagens
míticas. Os principais problemas levantados são: 1. Não existem provas
de qualquer herança genética ou biológica de padrões arquetípicos; 2. Os
modelos comparativos de mitos entre culturas diferentes baseados em
diacronias amplas levam a hipóteses inconsistentes; 3. A~função do mito
não seria como no modelo junguiano e elidiano, estritamente relacionada
à religiosidade e ao sagrado; 4. O~suposto valor universal e arquetípico
da mitologia abstém"-se das referências ao contexto cultural, sociológico
e histórico: os simbolistas se interessam pelo mito em sua forma
particular de narrativa, mas sem esclarecê"-lo pela cultura; crítica
semelhante realizada por Carlo Ginzburg: isolar símbolos específicos
mais ou menos difusos confundindo"-os com ``universais culturais'', e
também o escasso levantamento documental dos dados míticos e de conteúdo
além do contexto histórico das obras; 5. Pouco rigor na aplicação das
teorias junguianas aos fenômenos históricos; 6. O~caráter indireto da
explicação e da comprovação das teorias de base psico"-históricas; 7.
Impossibilidade de o pesquisador ter acesso direto à psicologia profunda
de um período.
\SIG{Johnni Langer}

Ver também Tripartição do mundo nórdico.



\begin{itemize}\footnotesize
\item
  \versal{CAMPBELL}, Joseph. \emph{O~herói de mil faces}. São Paulo: Pensamento,
  1996.
\item
  \versal{ELIADE}, Mircea. \emph{O~mito do eterno retorno}. Lisboa: Edições 70,
  1985.
\item
  \versal{GINZBURG}, Carlo. \emph{História noturna: decifrando o sabá}. São
  Paulo: Companhia das Letras, 1991.
\item
  \versal{JUNG}, Carl Gustav (Ed.). \emph{O~homem e seus símbolos}. São Paulo:
  Nova Fronteira, 1996.
\item
  \versal{LANGER}, Johnni. ``\emph{Mythica Scandia}: repensando as fontes literárias da
  mitologia viking''. \emph{Brathair} 6(2), 2006, pp.~48--78.
\end{itemize}

\section{\versal{ARTE} \versal{RELIGIOSA} \versal{NÓRDICA}}

Ver Amuletos mágicos; Ardre \versal{VIII}; Hammar \versal{I}; Hogbacks; Ídolos e imagens;
Klinte Hunninge \versal{I}; Pedras pintadas de Gotland; Pinturas rupestres
nórdicas.

\section{\versal{ASES} E \versal{VANES}}

Ases são uma família de deuses, a mais importante da mitologia
escandinava. O~termo em nórdico antigo \emph{áss} (plural: \emph{æsir},
feminino: \emph{ásynja}) significa deus, e segundo Régis Boyer também
teria um sentido de força e vida, como no sânscrito \emph{asura}. John
Lindow também opina de forma semelhante, considerando que o termo deriva
de uma raiz indo"-europeia significando vida e alento. Segundo Rudolf
Simek, o termo foi anteriormente registrado pelos godos como
\emph{Ansis} (Getica \versal{XIII}, 78) e no anglo"-saxão \emph{ēsa}. No
proto"-germânico ela existia aplicada à palavra \emph{Vih"-asa} (deusa da
batalha) e foi registrada na inscrição rúnica de Vimose na Dinamarca
(``\emph{a(n)sal wïja}'', eu dedico isso aos ases), datada de 200 d.C.~
Os mais importantes deuses ases são Odin e seus filhos Thor e Balder.
Enquanto os ases são divindades proeminentes da guerra e governantes, os
vanes são os deuses da fertilidade. Na \emph{Edda} de Snorri, tanto Odin
quanto Thor são frequentemente denominados simplesmente ``o ás''. No
poema éddico \emph{Skírnismál}, não é claro qual dos dois é denominado
``o melhor dos ases''. Entretanto, o nome da runa A (ös, ansuz) é
associado a Odin. Uma das mais interessantes aplicações do termo ases
foi empregada por Snorri Sturluson no prefácio da \emph{Edda Menor} e na
\emph{Ynglinga saga}, utilizando a similaridade da palavra com a região
da Ásia para criar um referencial evemerista dos deuses.

\imagempequena{Thor pescando, Suécia, séc.~\versal{VIII}}{./img/63.png}                         				 % ASES E VANES         

Na perspectiva dumeziliana, os deuses ases regem a jurisprudência, a
soberania e a magia. Por sua vez, os vanes regem a fertilidade e a
fecundidade. O~termo vanes possui etimologia incerta, segundo Régis
Boyer, mas a palavra \emph{uen} (desejo) poderia estar associada a
Vênus. Para o pesquisador John Lindow, a palavra vanir é aparentada com
os termos para amigo e desejo, nas linguagens escandinavas, mas para
Rudolf Simek não existe explicação convincente para ela.

A família dos deuses vanes abrange Njord, seus filhos Freyr e Freyja,
Heimdall (segundo Lindow) Skadi (segundo Boyer), Ullr (segundo Simek).
Os vanes são deidades particularmente relacionadas a boas colheitas, ao
florescimento do Sol, da chuva, de bons ventos e tempo bom, tanto para
os camponeses quanto para os pescadores e marinheiros. Também são
relacionados a certas práticas mágicas, como o seidr de Freyja. Outro
elemento que as fontes relacionam em oposição aos ases é quanto à
prática de incesto, o que para Simek poderia indicar elementos
matriarcais no culto aos vanes.

Segundo Jens Peter Schjødt a diferença entre a religiosidade dos vanes e
a dos ases seria muito grande, sendo a primeira uma religiosidade
autóctone antiga, baseada numa cultura agrícola, enquanto a segunda
seria mais nova, guerreira e mais espiritual. Os vanes seriam ligados
essencialmente à fórmula arcaica: \emph{Ár ok friðr} (abundância e paz),
gerando os simbolismos de fertilidade, sexualidade, natureza e riqueza.
E~sendo um grupo ctônico, vinculado diretamente as elfos, o protótipo
incestuoso dos vanes não teria sido usado como modelo para a sociedade
humana.

Outro tema muito importante nas fontes é a guerra primordial entre ases
e vanes, relatada na \emph{Ynglinga saga} 4, \emph{Gylfaginning} 22,
\emph{Skáldskaparmál} 1, \emph{Völuspá} 21--26, \emph{Gesta Danorum} \versal{I},
7. Na \emph{Völuspá}, a causa do conflito teria sido a feiticeira
Gullveig, uma personagem não mencionada por Snorri. A~paz é alcançada no
momento em que ambas as partes decidem por uma troca mútua de deuses.

Em 1903, o acadêmico Bernhard Salin propôs a teoria de que a guerra
entre ases e vanes teria um fundo histórico: representaria um culto mais
novo, o dos indo"-europeus (de índole guerreira, a família dos ases), que
teria penetrado na região escandinava, onde prolifera o culto nativo
representado pelos vanes, de cunho mais agrário (cultura megalítica).
Posteriormente, houve a fusão entre os cultos (representada pelo fim dos
conflitos nas fontes mitológicas). Essa guerra de religião também foi
defendida por H.~Schuch e E.~Mogk, enquanto H.~Guntert e A.~Philippson
inclinaram"-se a pensar numa guerra puramente política e étnica, que
teria ocorrido no segundo milênio antes de Cristo. Georges Dumézil
criticou essa teoria, afirmando que a guerra entre ases e vanes seria o
resultado de um conflito social entre os camponeses e os seguidores do
rei/aristocracia. O~resultado da guerra teria sido a formação da
sociedade tripartida, uma ideia também seguida por J.~de Vries. Mais
recentemente, a arqueóloga Lote Hedeager propôs que a guerra entre ases
e vanes representaria mitologicamente o conflito que teria existido
entre a migração dos povos hunos em relação aos povos ostrogodos.
\SIG{Johnni Langer}

Ver também Asgard; Balder; Freyja; Freyr, Frigg; Guerra entre Ases e
Vanes; Odin; Valhala; Thor.



\begin{itemize}\footnotesize
\item
  \versal{BOYER}, Régis. ``The Aesir and the Vanir'', \emph{in}  \versal{BONNEFOY}, Yves (ed.).
  \emph{American, African and Old European Mythologies}. Chicago: The
  University of Chicago Press, 1993, pp.~237--38.
\item
  \versal{BOYER}, Régis. ``La bataille des Ases et des VANES''. \emph{Yggdrasill: a
  religion des anciens scandinaves}. Paris: Payot, 1981, pp.~198--200.
\item
  \versal{DUMÉZIL}, Georges. ``Dioses ases y dioses vanes''. \emph{Los dioses de los
  germanos}. México: Siglo Veintiuno, 1990, pp.~5--41.
\item
  \versal{ÖSTVOLD}, Tobjörg. ``The war of the Æsir and the Vanir: a myth of the
  fall in Nordic Religions''. \emph{Temenos} 5, 1969, pp.~169--202.
\item
  \versal{SIMEK}, Rudolf. ``Æsir/Vanir''. \emph{Dictionary of Northern Mythology}.
  Londres: D.S.~Brewer, 2007, pp.~3--4, 351--53.
\item
  \versal{SCHJØDT}, Jens Peter. ``Æsir and Vanir''. \emph{Initiation between two
  worlds: structure and symbolism in pre"-Christian Scandinavian
  religion}. Odense: The University Press of Southern Denmark, 2008, pp.~ 382--96.
\end{itemize}

\section{\versal{ASGARD}}

A palavra em nórdico antigo, \emph{Asgarðr}, significa recinto (para
Boyer e Lerate), casa (segundo Simek) ou terreno (para Lindow) dos
deuses. É~a fortaleza das divindades segundo as fontes da mitologia
escandinava. Na poesia éddica ela surge em dois momentos, na
\emph{Hymiskviða} 7 e na \emph{Þrymskviða} 18, e na poesia escáldica é
citada somente num poema de Þórbjorn dísarskáld, datado do século \versal{X}.~O~termo 
aparece mais frequentemente nos escritos de Snorri. Os palácios
dos deuses (como Valhala) e locais como Glaðsheimr, são situados em
Asgard. Para o pesquisador Per Vikstrand, o termo Asgard não seria tão
antigo como Midgard e teria sido construído por Snorri baseado em um
referencial da Islândia medieval acerca das antigas fortificações da Era
Viking, e em analogia com o conceito de \emph{garðr}. Por sua vez, a
localidade de Útgarðr seria ainda mais artificial, inspirada tanto em
Asgard quanto em Midgard.

Segundo Rudolf Simek, originalmente Asgard situava"-se num plano próximo
a Midgard, em oposição a Utgard. Teria sido Snorri quem utilizou o
referencial de uma cosmologia cristã e situou Asgard num plano mais
elevado, próximo do céu. Esse pensamento vem sendo reiterado por
diversos pesquisadores recentemente, mas, de um ponto de vista da
mitologia comparada, ele não procede: a noção da morada dos deuses como
sendo localizada num local acima dos homens, num plano mais celestial, é
típica de várias tradições míticas em todo o mundo euro"-asiático, sendo
o cristianismo apenas mais um a corroborar esta ideia.

A mais importante referência sobre Asgard, a de sua construção, somente
foi preservada no \emph{Gylfagginning} 42. Um mestre de obras, que na
realidade é um gigante disfarçado, procura os deuses e lhes propõe a
construção de uma grande fortaleza no prazo de três estações, que
protegeria a todos contra as invasões dos gigantes das montanhas. O~pagamento 
exigido seria a deusa Freyja, o Sol e a Lua. Os deuses
concordam com a proposta, mas exigem que o prazo seja somente uma
estação (seis meses). O~gigante utiliza um cavalo mágico para a
empreitada, Svadilfari, e praticamente conclui a operação em três dias,
antes do início do verão. Desesperados com a ideia de perder Freyja, o
Sol e a Lua e, assim, o firmamento entrar em caos, os deuses apelam a
Loki, que se transforma em uma égua no cio e afasta o cavalo mágico.
Desta união nasceu o cavalo Sleipnir. Ao descobrir a identidade do
gigante, Thor esmaga seu crânio e o envia para Niflheim.

Acreditamos que esta narrativa seja o registro de um mito celeste: a
ameaça de que o Sol, a Lua e a deusa Freyja (possivelmente vista na Era
Viking como o planeta Vênus) sejam perdidos para os gigantes, ameaçando
a ordem dos céus (no original em nórdico: ``at gifta Freyju í Jötunheima
eða spilla loftinu ok himninum svá, at taka þaðan sól ok tungl ok gefa
jötnum''). Isso é reforçado pelo fato de o crânio do gigante ser
esmagado e enviado para Niflheim -- o reino situado num eixo oposto a
Asgard, em nível inferior. Aqui temos novamente a participação dos
gigantes na elaboração do cosmos: enquanto o céu e Midgard foram
construídos a partir do crânio de Ymir, um gigante constrói Asgard e ao
mesmo tempo ameaça a estabilidade do firmamento celeste. Mais um
elemento comprovando, para a cosmovisão nórdica pré"-cristã, que a casa
dos deuses seria situada acima do mundo dos homens, em um plano
superior, próximo aos céus.
\SIG{Johnni Langer}

Ver também Ases e Vanes; Planetas e mitos nórdicos; Thor; Valhala.



\begin{itemize}\footnotesize
\item
  \versal{BERNÁRDEZ}, Enrique. ``La geografía mitológica''. \emph{Los mitos
  germánicos}. Madri: Alianza, 2010, pp.~283--88.
\item
  \versal{BOYER}, Régis. ``Ásgadr''. \emph{Héros et dieux du Nord}. Paris:
  Flammarion, 1997, p.~20.
\item
  \versal{HALVORSEN}, E.~F.~``Åsgard''. \emph{Kulturhistorisk Leksikon for Nordisk
  Medeltid} 20, 1976.
\item
  \versal{SIMEK}, Rudolf. ``Asgard''. \emph{Dictionary of Northern Mythology}.
  Londres: D.S.~Brewer, 2007, p.~20.
\item
  \versal{VIKSTRAND}, Per. ``\emph{Ásgarðr}, \emph{Miðgarðr}, and \emph{Útgarðr}: a linguistic approach
  to a classical problem'', \emph{in}  \versal{ANDRÉN}, Anders, \versal{JENNBERT}, Kristina \&
  \versal{RAUDVERE}, Catharina. (eds.). \emph{Old Norse religion in long"-term
  perspectives: origins, changes and interactions}. Lund: Nordic
  Academic Press, 2006, pp.~354--57.
\end{itemize}

\section{\versal{ASK} E \versal{EMBLA}}

Ver Antropogonia nórdica.

\section{\versal{ASTRONOMIA} \versal{NÓRDICA}}

Ver Cometas e mitos nórdicos; Constelações e mitos nórdicos; Cosmogonia
nórdica; Cosmologia nórdica; Estrelas e mitos nórdicos; Fenrir; Lua e
Sol; Planetas e mitos nórdicos; Yggdrasill como Via Láctea.

\section{\versal{AUDHUMLA}}

A vaca que teria surgido durante o início dos tempos segundo a
cosmogonia nórdica, também denominada Auðumbla, Auðumla e Auðhumla. Para
Régis Boyer, a palavra Audhumla significaria uma vaca sem chifres e
produtora de leite. Segundo o \emph{Gylfaginning} 6, o animal teria
surgido do orvalho derretido do gelo de Niflheim, cujas úberes verteram
quatro rios de leite, da qual se alimentou o protogigante Ymir. Por sua
vez, ao se alimentar lambendo o gelo, a vaca deu origem a outro ser
primordial, Búri, cujo filho dará origem aos deuses ases. O~papel mítico
de Audhumla somente foi registrado por Snorri Sturluson.

Tácito registrou o fato de que algumas tribos germânicas adoravam gado
mocho (\emph{Germania} 5), e também cita que o carro da deusa Nerthur
era puxado por um par de bois. Para Rudolf Simek, a imagem da vaca
sagrada é uma figuração estreitamente relacionada com os simbolismos da
mãe"-terra em numerosas religiões orientais, como a deusa Hathor. Muitas
pinturas e esculturas egípcias representam esta deusa como uma
gigantesca vaca, onde uma figura masculina alimenta"-se em sua úbere.
Segundo Enrique Bernárdez, a vaca Audhumla simbolizaria a maternidade
proveniente da Terra. Na concepção de Jean Chevalier, Audhumla possui um
significado estreitamente relacionado ao de outros povos indo"-europeus,
desempenhando um papel cósmico e divino, de modo muito semelhante aos do
bode e do carneiro.

Com relação aos quatro rios de leite, Hilda Davidson acredita que
refletiria a ideia da árvore do mundo como fonte de alimento, mas já o
mitólogo alemão Rudolf Simek atribui esta imagem à educação cristã de
Snorri Sturluson, sendo influenciado pela cultura clerical dos quatro
rios paradisíacos.

A primeira imagem artística de Audhumla foi realizada no manuscrito da
\emph{Edda em Prosa} de 1760, pelo islandês Ólafur Brynjúlfson. Nela
surge a representação de uma vaca vertendo quatro rios de leite e
lambendo o gelo, que forma a cabeça de Búri. O~protogigante Ymir está
ausente da composição, mas a pintura serviu de inspiração direta para
que em 1790 o pintor dinamarquês Nicolai Abildgaard realizasse a mais
famosa representação do mítico animal na arte ocidental.
\SIG{Johnni Langer}

Ver também Cosmogonia nórdica; Ginnungagap; Ymir.



\begin{itemize}\footnotesize
\item
  \versal{BOYER}, Régis. ``Audhumla''. \emph{Héros et dieux du Nord}. Paris:
  Flammarion, 1997, p.~22.
\item
  \versal{BRANSTON}, Brian. \emph{Mitología germánica ilustrada}. Barcelona:
  Vergara, 1960.
\end{itemize}

\section{\versal{AURVANDILL}}

Ver Estrelas e mitos nórdicos.
